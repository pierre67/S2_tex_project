\startcomponent c_10_safety_s2_130-he
\product prd_ba_s2_130-he

\marking[chapter]{סימני בטיחות}


\chapter{סימני בטיחות}

\setups[pagestyle:marginless]

\section{סימון אירופי חדש לחומרים מסוכנים}

{\em בצורת סולמית עם רקע לבן ושוליים אדומים.}\par\blank[1*medium]
{\em החל ב-2008 תקפה באיחוד האירופי
תקנה CLP\index{תקנה CLP} הכוללת סימן אזהרה חדש לחומרים ומוצרים מסוכנים.}\par\null

\startSymList \GHSgeneric
\SymList
	\textDescrHead{סיכון בריאותי}
	מזהיר\index{מסיכון בריאותי} סיכון בריאותי שאינו עלול לגרום למוות או לנזקים קשים לבריאות. לכך משתייך גם גירוי בעור או תגובת אלרגיה. הסמל משמש גם כאזהרה מסכנות אחרות, כיצירת דלקות.\par מחליף את:\crlf \HAZOcross\ או \HAZOpoison\ או \PHgeneric
\stopSymList

\startSymList \GHSbody
\SymList
	\textDescrHead{סיכון בריאותי חמור; עלול לגרום למוות במיוחד אצל ילדים} המוצרים עלולים לגרום לסיכונים בריאותיים חמורים. סמל זה מזהיר מסיכונים\index{סכנה+הריון} להריון, מ\index{סכנה+חומרים מסרטנים} השפעות מסרטנות וסיכונים בריאותיים דומים. השתמש בזהירות במוצרים.\par
	מחליף את:\crlf \HAZOcross\ או \HAZOpoison\
\stopSymList

\startSymList \GHSbomb
\SymList
	\textDescrHead{חומרי נפץ}
	חומרים לא יציבים\index{סכנה+פיצוץ} מתפוצצים, תערובות ותזקיקים המכילים חומרי נפץ\index{חומרי נפץ} יוצרים כאשר הם מגיבים התפשטות רבה, היכולה לגרום להרס רב; סכנת חיים בשימוש המנוגד להוראות.\par
	מחליף את:\crlf \HAZObomb\
\stopSymList


\startSymList \GHSpoison
\SymList
	\textDescrHead{הרעלה}
	המוצרים\index{סכנה+הרעלה} עלולים לגרום גם במגע עם העור או בשאיפת כמות קטנה\index{רעלים} או ע"י בליעה להרעלות קשות או קטלניות. אין לאפשר מגע ישיר.\par
	מחליף את:\crlf \HAZOpoison\
\stopSymList

\startSymList \GHSfire
\SymList
	\textDescrHead{דליק או דליק מאוד}
	המוצרים\index{סכנה+שרפה} נדלקים במהירות בקרבת חום או להבות. אין להתיז תרסיסים המסומנים בסימון זה בשום אופן על משטחים חמים או בקרבת להבות פתוחות.\par
	מחליף את:\crlf \HAZOfire\ או \HAZOfirebis\
\stopSymList

\startSymList \GHSenvironment
\SymList
	\textDescrHead{סכנה לבעלי חיים ולסביבה}
	המוצרים\index{הגנת הסביבה} עלולים לגרום לנזקים קצרי טווח או ארוכי טווח לסביבה\index{רעלים}. הם עלולים להרוג את האורגניזמים החיים במים (\eG\ דגים) או לפעול פעולה הרסנית לטווח ארוך על הסביבה. אין לשפוך לביוב או לאשפה הביתית!\par
	מחליף את:\crlf \HAZOenvironment\
\stopSymList

\startSymList \GHScorrosive
\SymList
	\textDescrHead{סכנה לעור או לעיניים}
	המוצרים\index{סכנה+פציעה לעור}\index{סכנה+פציעה בעיניים} עלולים לגרום גם לאחר מגע קצר לנזק לעור וליצירת צלקות או לגרום לנזק קבוע לעיניים. הגן על העור והעיניים בזמן השימוש!\par
	מחליף את:\crlf \HAZOcross\ או \HAZOcorrosive
\stopSymList

\page [yes]


\section{סימן אזהרה}

{\em אותיות שחורות על רקע צהוב}\par\null

\startSymList \PHgeneric
\SymList
	\textDescrHead{סימן אזהרה כללי}
	מודיע\index{סכנה+כללית}\index{סימן אזהרה} aעל סכנה בטווח המידי העלולה לגרום לפציעה שלך או של אנשים אחרים.
	\crlf\null
\stopSymList

\startSymList \PHpoison
\SymList
	\textDescrHead{אזהרה מחומרים רעילים}
	חומרים רעילים\index{סכנה+הרעלה} עלולים לגרום אם באו במגע עם העור או נשאפו לנזקים אקוטיים או כרוניים משמעותיים לבריאות או אפילו למוות.
\stopSymList

\startSymList \PHfire
\SymList
	\textDescrHead{אזהרה מחומרים דליקים}
	מנע להבות פתוחות או גיצים\index{סכנה+שרפה}. החומר דליק מאוד או יכול לגרום להאצת הבעירה. אסור לעשן!
\stopSymList

\startSymList \PHexplosive
\SymList
	\textDescrHead{אזהרה מחומרים מתפוצצים}
	חומרים מוצקים, נוזליים או בצורת ג'ל או תזקיקים העלולים להתפוצץ כתוצאה ממכה, שפשוף, שרפה, חום\,וכדומה.\index{סכנה+פיצוץ} אסור לעשן!
\stopSymList

\startSymList \PHcrushing
\SymList
	\textDescrHead{אזהרה מסכנת מעיכה}
	מודיע על תחום\index{סכנה+מעיכה} בו קיימת סכנת מעיכה עקב חלקים מכאניים נעים. התרחק מהאזור הזה, כל עוד המתקן פועל.
\stopSymList

\startSymList \PHhand
\SymList
	\textDescrHead{אזהרה מפציעות לידיים}
	קיימת סכנה\index{סכנה+מעיכה} לידיים או לחלקי גוף\index{סכנה+פציעות ליד} עלולים להימעך, \eG\ בזמן היפוך תא הנהג או גשר ההעמסה.
\stopSymList

\startSymList \PHentangle
\SymList
	\textDescrHead{אזהרה מגלילים הנעים בכיוון הנגדי / מסכנת הילכדות}
	קיימת סכנה שהאיברים\index{סכנה+הילכדות} ייתפסו מחלקים מסתובבים ויילכדו בהם. התרחק כל עוד המתקן פועל.
\stopSymList

\startSymList \PHcorrosive
\SymList
	\textDescrHead{אזהרה מחומרים חריפים}
	זהירות\index{סכנה+חומרים חריפים} לבש ציוד הגנה מתאים לבני אדם (כפפות, משקפי מגן, ביגוד מגן)..
\stopSymList

\startSymList \PHhot
\SymList
	\textDescrHead{אזהרה ממשטחים חמים}
	אל תתקרב לחלק או למתקן\index{סכנה+כוויות} ללא מיומנות מתאימה; לבש כפפות.
\stopSymList

\startSymList \PHvoltage
\SymList
	\textDescrHead{אזהרה ממתח חשמלי מסוכן}
	אין לגעת בחפצים מתכתיים\index{סכנה+מתח חשמלי}.
	סכנת פציעה או כוויות מקצר!
\stopSymList

\startSymList \PHfalling
\SymList
	\textDescrHead{אזהרה מהתרסקות}
	הזהר באזור זה במיוחד\index{סכנה+נפילה} נעל נעלים מתאימות (עם סוליות המגנות מהחלקה, חומר פחמימני).
\stopSymList

\startSymList \PHbattery
\SymList
	\textDescrHead{אזהרה מסכנה כתוצאה מסוללות} מודיע על סכנות כתוצאה מטעינת הסוללות (סוללות עופרת) \index{סכנה+סוללה} במיוחד עקב פליטת גז מימן וחומצת הגפרית הנמצאת בסוללות.
\stopSymList

\startSymList \PHremote
\SymList
	\textDescrHead{אזהרה מתנועה אוטומטית}
	מזהיר\index{סכנה+תנועה אוטומטית} מתנועה אוטומטית או בשליטה מרחוק של המתקן.
\stopSymList

% \startSymList \PHquetschgefahr
% \SymList
% \textDescrHead{Risque d’écrasement}
% Risque d’écrasement\index{risque d’écrasement}.
% \stopSymList
% % NOTE: Doppelt! (auch Bilddatei)
%
% % NOTE: Evtl. Folgendes als Ersatz für oben?

% \startSymList\PHhandcrushed
% \SymList
	% \textDescrHead{Gefahr von Handquetschungen}
	% Es besteht\index{Gefahr+Quetschung} die Gefahr, dass Hände oder Finger
	% gequetscht werden. Nähern Sie die Hände nicht an, ohne die Gefahr
	% identifiziert und beseitigt zu haben.
% \stopSymList

\startSymList \PHhandfoot
\SymList
\textDescrHead{אזהרה מחלקים נעים}
מזהיר ממכונות וחלקי רכב הנמצאים בתנועה.
\index{סכנה+חלקים נעים}.
\stopSymList

\startSymList \PHnarrowed
\SymList
	\textDescrHead{אזהרה ממסלול נסיעה הנהיה צר}
	מסלול\index{סכנה+רוחב כלי רכב} נסיעה ההולך ונהיה צר.
	% Denken Sie an die Breite des Fahrzeugs.
\stopSymList

\page [yes]


\section{סימני איסור}

{\em עיגול על רקע לבן, שוליים אדומים ועמודות מאונכות}
\par\null


\startSymList \PPfire
\SymList
	\textDescrHead{אש, אור פתוח ועישון אסורים} להבה פתוחה \index{אסור+עישון, שרפה} וגחלים אסורים (\eG\ סיגריה בוערת, גפרורים, נרות; וגם גיצים בכל צורה).
\stopSymList

\startSymList \PPentry
\SymList
	\textDescrHead{הכניסה אסורה לאלה שאינם מורשים בכך}
	לאנשים\index{איסור+כניסה} שאינם מורשים אסורה הכניסה או ההתקרבות לאזור זה.
\stopSymList

\startSymList \PPphone
\SymList
	\textDescrHead{אסור השימוש בטלפון נייד}
	יש לכבות כל טלפון נייד\index{איסור+טלפון נייד} וכל מכשיר אחר הפולט קרינה אלקטרומגנטית. הקרינה האלקטרומגנטית עלולה לגרום לתקלות באלקטרוניקה של המכשיר.
\stopSymList

\startSymList \PPspray
\SymList
	\textDescrHead{אסור להתיז מים}
	אין לכוון זרם מים או קיטור\index{איסור+זרם מים, קיטור} על חלקים פגיעים ועל מכשירים (\eG\ חיישנים, בקרות, מערכות התזה וכו'.).
\stopSymList

\startSymList \PPchildren
\SymList
	\textDescrHead{הרחק ילדים}
	מידע\index{איסור+ילדים} על סכנה מיוחדת לילדים. באופן כללי: אסור לילדים להתקרב למכונה פועלת, גם לצורך עבודות תחזוקה.
\stopSymList

\startSymList \PPwater
\SymList
	\textDescrHead{לא מי שתייה}
	אין לשתות את המים\index{איסור+לא מי שתייה} מהמיכל. קיימת סכנת הרעלה.
\stopSymList

% \page [yes]


\section{סימן לשמירה על הסביבה}

\startSymList \PSrecycle
\SymList
	\textDescrHead{מחזור}
	תקנות מיוחדות להשלכה בהתאם להוראות של אשפה מסויימת.
\stopSymList

\startSymList \PSwelt
\SymList
	\textDescrHead{שמירה על איכות הסביבה}
	מידע על התקנות התקפות של השמירה על הסביבה.
\stopSymList

\startSymList \PStrash[width=\PictoHeight,height=,]
\SymList
	\textDescrHead{השלך את האשפה בהתאם להוראות}
	עבור אשפה מסוג מיוחד, \eG\ סוללות עופרת, תקפות הוראות מיוחדות להשלכה לאשפה.
\stopSymList


\testpage[12]


\section{סימני ציווי}


{\em עגול על רקע כחול}\par\null

\startSymList \PMgeneric
\SymList
	\textDescrHead{סימן ציווי כללי}
	סימן זה ישמש רק יחד עם סימן נוסף, המסמן את הציווי.
\stopSymList


\startSymList \PMrtfm
\SymList
	\textDescrHead{ציית להוראות ההפעלה}
	יש לקרוא לפני ההכנסה לפעולה\index{קרא את הוראות השימוש} את ההוראות בקשר לנושא זה, הקשורות למכשיר או למוצר מסוימים. יש לשמור על הוראות ההפעלה נגישות בתא הנהג.
\stopSymList

\startSymList \PMproteyes
\SymList
	\textDescrHead{השתמש במשקפי מגן}
	יש להשתמש במשקפי מגן בעבודות בהן קיימת סכנת פציעה לעיניים\index{משקפי מגן}.
\stopSymList

\startSymList \PMprothands
\SymList
	\textDescrHead{השתמש בכפפות מגן}
	יש ללבוש כפפות מגן בעבודות בהן עלולות להיגרם פציעות לידיים, \index{השתמש בכפפות מגן}.
\stopSymList

\startSymList \PMprotears
\SymList
	\textDescrHead{השתמש באטמי אזניים}
	יש להשתמש באטמי אזניים\index{סכנה+שמיעה}  (\eG בקרבת מאוורר מסתובב או טורבינה מסתובבת).
\stopSymList

\startSymList \PMsafetybelt
\SymList
	\textDescrHead{השתמש בחגורת בטיחות} חגור\index{חגורת בטיחות} חגורת בטיחות לבטיחותך.
\stopSymList

\section{סימנים נוספים}

% \adaptlayout[height=+5mm]                                                 {{{

% \startSymList \SETshoe
% \SymList
% \textDescrHead{Port de chaussures de sécurité obligatoire}
% Le port de chaussures de sécurité est obligatoire\index{chaussures de sécurité}.
% \stopSymList
%
% \startSymList \SETglasses
% \SymList
% \textDescrHead{Port de lunettes des protection obligatoire}
% Le port de lunettes est obligatoire\index{lunette de protection}.
% \stopSymList
%
% \startSymList \SEToreillettes
% \SymList
% \textDescrHead{Port de casque obligatoire}
% Le port d’un casque de protection est \index{casque} obligatoire.
% \stopSymList
%
% \startSymList \SETgloves
% \SymList
% \textDescrHead{Port de gants de protection obligatoire}
% Le port de gants de protection est obligatoire\index{gants}.
% \stopSymList
%
% \startSymList \SETmainecrase
% \SymList
% \textDescrHead{Risque d’écrasement}
% Danger pour les mains\index{écrasement} et les pieds.
% \stopSymList
%
% \startSymList \SETgetriebe
% \SymList
% \textDescrHead{Risque de happement}
% Risque de happement par\index{happement} des pièces en rotation.
% \stopSymList
%
% \startSymList \SETradkeil
% \SymList
% \textDescrHead{Cale de roue}
% Sécuriser le véhicule contre toute mise\index{Cale de roue} en marche involontaire.
% \stopSymList
%}}}

\startSymList \SETfirstaid
\SymList
\textDescrHead{עזרה ראשונה}
	מציג את מקום האחסון של ציוד העזרה הראשונה. חלק חשוב מהעזרה הראשונה הוא אזעקת שירותי ההצלה.\index{ראשונה
	עזרה}\index{שיחת חירום} רשום כאן את מספרי הטלפון לשעת חירום:
	\fillinrules[n=1]{\bf
	\framed[align=right,frame=off,offset=none,width=30mm]{שירותי הצלה}}
\fillinrules[n=1]{\bf
\framed[align=right,frame=off,offset=none,width=30mm]{משטרה}}
\fillinrules[n=1]{\bf
\framed[align=right,frame=off,offset=none,width=30mm]{מכבי אש}}
\stopSymList

\startSymList \SETbrandschutzzeichen
\SymList
\textDescrHead{מטף כיבוי}
	חומרים מסוימים מצוידים במטף כיבוי אחד או בכמה מטפי כיבוי\index{מטף כיבוי}. יש לתחזק אותם בדרך כלל באופן מיוחד;מידע נוסף בקשר לכך ניתן למצוא על המכשיר או בהוראות ההפעלה של המכשיר.
\stopSymList


\page[yes]

\section{שלוש הפעולות למתן עזרה}
% NOTE [tf]: Shouldn't be in this book, IMO

\starttextbackground [CB]
\textDescrHead{אבטח את מקום התאונה ואת הנפגעים}
\startitemize
\item  בדוק את בטיחות מקום התאונה וודא שלא קיימות סכנות נוספות.
\stopitemize
\textDescrHead{קבע את מצב הפצועים}
\startitemize
\item  בדוק האם הפצועים בהכרה והאם הם נושמים כרגיל.
שחרר בעת הצורך את דרכי הנשימה.
\stopitemize
\textDescrHead{הזעק את כוחות ההצלה}
\startitemize מסור בשיחת החירום את הפרטים הבאים:\par
	\item מספר הטלפון, בו אתה זמין.
	\item סוג המקרה (מחלה, תאונה).
	\item סכנות קיימות (שרפה, פיצוץ, סכנת נפילהוכו').
	\item המיקום המדויק.
	\item מספר הפצועים ומצבם.
	\item פעולות לעזרה, שכבר התבצעו.
	\item השב על השאלות הנוספות שיוצגו לך.
\stopitemize
\stoptextbackground


\page [yes]

\setups[pagestyle:normal]

\section{סימון אירופי חדש לחומרים מסוכנים}


\placefig[margin][p4_vue_01]{מספר הפצועים ומצבם}
{\externalfigure[overview:vhc:01]}

\placefig[margin][p4_vue_01]{מספר הפצועים ומצבם}
{\externalfigure[overview:vhc:02]}

\stopcomponent

