\startcomponent c_40_control_s2_130-he
\product prd_ba_s2_130-he


\startchapter [title={Bedienelemente des Fahrzeugs},
							reference={chap:ctrl}]

\setups[pagestyle:marginless]

\placefig[here][fig:ctrl:cab:front]{Bedienelemente}
{\externalfigure[ctrl:cab:front]}

\startcolumns [n=3]
\startLongleg
	\item Lenksäule (\in{§}[sec:steeringColumn])
	\item Einstellung Lenksäule
	\item Fahr- und Bremspedal
	\item Bordcomputer \Vpad~SN (\inP[sec:vpad])
	\item Deckenkonsole (\inP[sec:ctrl:aux])
	\item Radio|/|MP3
\stopLongleg


\subsubsubject{Optionale Ausstattung}

\startLongleg [continue]
	\item Rückfahrmonitor
\stopLongleg
\stopcolumns

\startsection [title={Die Lenksäule},
							reference={sec:steeringColumn}]

\subsection{Einstellen der Lenksäule}

\textDescrHead{Neigung des Lenkrads} Drücken Sie das Pedal~\Ltwo und stellen Sie gleichzeitig die Neigung der Lenksäule ein. Lassen Sie das Pedal los, um den Mechanismus der Lenksäule wieder zu verriegeln.

\page[yes]
\setups [pagestyle:normal]


\subsection{Beleuchtungs- und Signaleinrichtungen}

\placefig [margin] [fig:column:left] {Multifunktionshebel und Drehschalter}
{\externalfigure[ctrl:column:left]}

\placefig [margin] [fig:column:right] {Fahrstufenwahlhebel}
{\externalfigure[ctrl:column:right]}


\subsubsubject{Drehschalter}

\startitemize[width=1.7em]
\sym{\textSymb{com_lowlight}} Abblendlicht (drehen~\TorqueR).
\startitemize
\sym{1} Standlicht
\sym{2} Abblendlicht
\stopitemize
\stopitemize


\subsubsubject{Multifunktionshebel}

\startitemize[width=1.7em]
\sym{\textSymb{com_lowlight}} {[}Nicht belegt{]}
\sym{\textSymb{com_light}} Lichthupe (Hebel kurz nach oben drücken)
\sym{\textSymb{com_blink}} Fahrtrichtungsanzeiger (Hebel nach vorne|/|hinten)
\sym{\textSymb{com_claxonArrow}} Hupe (Knopf außen am Hebel drücken)
\sym{\textSymb{com_wipper}} Scheibenwischer
\startitemize
\sym{J} Intervallschaltung
\sym{O} Aus
\sym{I} 1.\,Geschwindigkeitsstufe
\sym{II} 2.\,Geschwindigkeitsstufe
\stopitemize
\sym{\textSymb{com_washerArrow}} Scheibenwaschanlage (auf den Kranz am Ende des Hebels drücken).
\stopitemize


\subsubsubject{Fahrstufenwahlhebel}

Die Funktionen des Fahrstufenwahlhebels sind detailliert beschrieben im Kapitel~\about[chap:using], ab~\atpage[sec:using:start].

\stopsection

\page [yes]


\startsection [title={Weitere Funktionen},
							reference={sec:ctrl:add}]


\subsection[sec:ctrl:aux]{Deckenkonsole}

{\sl Die\index{Deckenkonsole} Deckenkonsole befindet sich vorne an der Decke des Führerhauses auf der Fahrerseite.}
\placefig [margin] [fig:console:aux] {Deckenkonsole}
{\externalfigure[ctrl:console:aux]}


\placefig [margin] [fig:console:climat] {Heizung und Klimaanlage}
{\externalfigure[ctrl:console:climat]}


\startitemize [unpacked][width=1.7em]
\sym{\textBigSymb{temoin_retrochauffant}} Außenspiegelheizung
\sym{\textBigSymb{temoin_hazard}} Warnblinkanlage
\sym{\textBigSymb{temoin_eclairage_L}} Arbeitsscheinwerfer
\stopitemize


\subsubsubject{Optionale Ausstattung}

\startLeg [unpacked][width=1.7em]
\sym{\textBigSymb{temoin_buse}} Hochdruckwasserpumpe für Wasserpistole \crlf {\sl siehe \atpage[sec:using:water:spray]}
\sym{\textBigSymb{temoin_aspiration_manuelle}} Turbine für Handsaugschlauch \crlf {\sl siehe \atpage[sec:using:suction:hose]}
\stopLeg


\subsection[sec:ctrl:climat]{Heizung und Klimaanlage}

{\sl Diese Konsole\index{Heizungskonsole} befindet sich an der Rückwand des Führerhauses, zwischen den Sitzen.}

\startitemize [unpacked][width=23mm]
\sym{\bf 0\quad I\quad II\quad III} Gebläse||Drehschalter
\sym{\externalfigure[tirette_chauffage][height=1em]} Temperatur||Schieberegler
\stopitemize


\subsubsubject{Optionale Ausstattung}

\startitemize [unpacked][width=1.7em]
\sym{\textBigSymb{temoin_climat_bk}} Klimaanlage
\stopitemize

\page [yes]

\setups [pagestyle:bigmargin]


\subsection[sec:ctrl:central]{Mittelkonsole}

{\sl Die\index{Mittelkonsole} Mittelkonsole befindet sich zwischen den Sitzen.}

\placefig [margin] [fig:console:central] {Mittelkonsole}
{\externalfigure[ctrl:console:central]}


\subsubsubject{Befeuchtung der Besen}

\startLeg [unpacked][width=1.7em]
\sym{\textBigSymb{temoin_busebalais}} Niederdruckwasserpumpe\index{Wasserpumpe} für das Befeuchtungssystem\index{Wasserpumpe+Befeuchtung} der Besen. (Position~1: \aW{Automatisch}; Position~2: \aW{Permanent})
\stopLeg


\subsubsubject{Kippen des Schmutzbehälters}

\setupinmargin[right][style=normal]
\inright{%
\startitemize
\sym{\textSymb{mand_readtheoperatingmanual}} Beachten Sie bitte die Anweisungen zum Gebrauch der Feststellbremse auf \atpage[sec:using:stop].
\stopitemize}

\startLeg [unpacked][width=1.7em]
\sym{\textBigSymb{temoin_kipp2}} Kippen des Schmutzbehälters.
Um\index{Schmutzbehälter+Kippen} den Schmutzbehälter kippen zu können, muss die Feststellbremse aktiviert sein
und der Fahrstufenwahlhebel auf Neutral stehen.
\stopLeg


\subsubsubject{Not||Aus}

\starttextbackground [FC]
\startPictPar
\externalfigure[Emergency_Stop][Pict]
\PictPar
In einem Notfall\index{Not||Aus} können Sie alle Saug- und Kehrgeräte, sowie den Fahrantrieb, durch einen Druck auf den Not||Aus||Schalter abschalten.
\stopPictPar
\stoptextbackground


\subsection[sec:foot:switch]{Fußschalter}

\placefig [margin] [fig:foot:switch] {Fußschalter}
{\vskip 60pt
\externalfigure[work:foot:switch]}

Mithilfe\index{Fußschalter} dieses Schalters am Fuß der Lenksäule (\inF[fig:foot:switch]) können Sie schnell und einfach die Besen absenken, wenn es erforderlich ist (\eG\ am Gipfel einer Steigung, Auffahrt auf Bürgersteig).

\stopsection
\page[yes]
\setups [pagestyle:marginless]


\startsection[title={Multifunktionskonsole},
							reference={ctrl:console:middle}]

\startlocalfootnotes

\startfigtext[left]{Multifunktionskonsole}
{\externalfigure[overview:joy:large]}


\subsubsubject{Joysticks}

\textDescrHead{Ohne Frontbesen (oder Frontbesen deaktiviert):}
Die Joysticks steuern unabhängig voneinender jeweils einen Besen: Heben|/|senken~(\textSymb{joystick_aa}) oder links|/|rechts~(\textSymb{joystick_gd}). Der linke Joystick steuert den linken Besen, der rechte Joystick den rechten Besen.\footnote{Um bei einem Fahrzeug, welches mit Frontbesen ausgestattet ist (Option), die Position der Seitenbesen ändern zu können, muss der Frontbesen deaktiviert werden (Taste~\textSymb{joy_key_frontbrush_act}).}

\textDescrHead{Mit Frontbesen:}
Mit dem linken Joystick können Sie den Frontbesen heben|/|senken (\textSymb{joystick_aa}) und nach links|/|rechts bewegen (\textSymb{joystick_gd}). Mit dem rechten Joystick neigen Sie den Besen auf seiner Längs-~(\textSymb{joystick_aa}) und Querachse~(\textSymb{joystick_gd}).

\placelocalfootnotes %[height=\textheight]
\stopfigtext
\stoplocalfootnotes
\vfill


\subsubsubject{Seitliche Tasten}

\startcolumns

\startPictList
\VPcltr
\PictList
Tempomat: Erhöhen der eingestellten Geschwindigkeit
\stopPictList\vskip -3pt

\startPictList
\VPclbr
\PictList
Tempomat: Verringern der eingestellten Geschwindigkeit
\stopPictList\vskip -3pt

\startPictList
\VPcrtr
\PictList
Saugmund heben
\stopPictList

\column


\startPictList
\VPcrbr
\PictList
Saugmund senken
\stopPictList\vskip -3pt

\startPictList
\VPcrtf
\PictList
Grobschmutzklappe öffnen (vorne am Saugmund)
\stopPictList\vskip -3pt

\startPictList
\VPcrbf
\PictList
Grobschmutzklappe schließen
\stopPictList

\stopcolumns


\subsubsubject{Symboltasten}

\startcolumns

\startSymVpad
\externalfigure[joy:stop]
\SymVpad
\textDescrHead{Stopp} Aktiviertes Gerät anhalten:

1\:× drücken: 3.\,Besen deaktivieren\crlf
2\:× drücken: Alles deaktivieren
\stopSymVpad

\startSymVpad
\externalfigure[joy:tempomat]
\SymVpad
\textDescrHead{Tempomat} Tempomaten auf die momentane Geschwindigkeit einstellen und aktivieren. Zum Deaktivieren Taste~\textSymb{joy:tempomat} erneut betätigen, oder bremsen. Beschleunigen|/|verlangsamen Sie mit den seitlichen Tasten.
\stopSymVpad

\startSymVpad
\externalfigure[joy:ftbrs:minus]
\SymVpad
\textDescrHead{Besengeschwindigkeit} Rotationsgeschwindigkeit des Seitenbesen oder des Frontbesens verringern.
\stopSymVpad

\startSymVpad
\externalfigure[joy:ftbrs:plus]
\SymVpad
\textDescrHead{Besengeschwindigkeit} Rotationsgeschwindigkeit des Seitenbesen oder des Frontbesens erhöhen.
\stopSymVpad

\startSymVpad
\externalfigure[joy:eng:minus]
\SymVpad
\textDescrHead{Motordrehzahl} Drehzahl des Dieselmotors verringern.
\stopSymVpad

\startSymVpad
\externalfigure[joy:eng:plus]
\SymVpad
\textDescrHead{Motordrehzahl} Drehzahl des Dieselmotors erhöhen.
\stopSymVpad
\columnbreak

\startSymVpad
\externalfigure[joy:suc]
\SymVpad
\textDescrHead{Absaugung} Saugsystem aktivieren: Saugmund wird gesenkt,
Turbine und Recycling||Wasserpumpe werden eingeschaltet.\note [recyclingwaterpump] \crlf
Stopptaste~\textSymb{joy:stop} drücken, um das System zu deaktivieren.
\stopSymVpad

\startSymVpad
\externalfigure[joy:sucbrs]
\SymVpad
\textDescrHead{Kehren|/|Saugen}  Saug-|/|Kehrsystem aktivieren: Saugmund wird gesenkt, Seitenbesen werden gesenkt und positioniert, Turbine, Besen und Recycling||Wasserpumpe werden eingeschaltet.\note [recyclingwaterpump] \crlf
Stopptaste~\textSymb{joy:stop} drücken, um das System zu deaktivieren.
\stopSymVpad

\footnotetext[recyclingwaterpump]{Frischwasserpumpe wird ebenfalls eingeschaltet, wenn der Schalter~\textBigSymb{temoin_busebalais} auf \aW{Automatisch} steht (siehe \in [sec:ctrl:central] auf \atpage [sec:ctrl:central]).}
\startSymVpad
\externalfigure[joy:ftbrs:act]
\SymVpad
\textDescrHead{Frontbesen aktiviert} Frontbesen aktivieren|/|deaktivieren.
%% NOTE @Andrew: Singular
\stopSymVpad

\startSymVpad
\externalfigure[joy:ftbrs:right]
\SymVpad
\textDescrHead{Frontbesen links} Drehrichtung für Arbeiten mit dem Frontbesen auf der linken Seite
(Drehrichtung: Uhrzeigersinn).
\stopSymVpad

\startSymVpad
\externalfigure[joy:ftbrs:left]
\SymVpad
\textDescrHead{Frontbesen rechts} Drehrichtung für Arbeiten mit dem Frontbesen auf der rechten Seite
(Drehrichtung: Gegenuhrzeigersinn).
\stopSymVpad

\stopcolumns

\stopsection

\stopchapter

\stopcomponent











