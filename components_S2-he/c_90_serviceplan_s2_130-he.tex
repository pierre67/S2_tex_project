\startcomponent c_90_serviceplan_s2_130-he
\product prd_ba_s2_130-he


\chapter[sec:schedule]{Wartungsplan}

\startregister[index][wartplan]{Wartungsplan+Fahrzeug}


\section{Allgemeines}

\placeNote[][service_picto]{}{%
\noteF
\starttextrule{\Pmtcheck \enskip Qualifikation {\definedfont[SansBoldItalic] Check} \enskip}
Kontrolle, die vom\index{Qualifikation+Wartungspersonal} Betriebspersonal
des Fahrzeuges ohne fremde Hilfe ausgeführt werden kann. Grundkenntnisse der
Motorfahrzeugtechnik sind allerdings Voraussetzung.
\stoptextrule \blank[line]

\starttextrule{\Pmtpro \enskip Qualifikation {\definedfont[SansBoldItalic] Servicearbeit} \enskip}
Reguläre Wartungsarbeit, die im Rahmen der Instandhaltung durchzuführen ist. Die
Wartung muss in einer zugelassenen Werkstatt von einer Fachkraft mit dem
erforderlichen Werkzeug durchgeführt werden.
\stoptextrule \blank[line]

\starttextrule{\Pmtspecial \enskip Qualifikation {\definedfont[SansBoldItalic] Spezialarbeit} \enskip}
Spezielle\index{Spezialarbeit} Wartungsarbeit, die nur von einer Person
ausgeführt werden darf, die eine spezifische, vom \BoschungNote||Kundendienst
anerkannte Schulung absolviert hat.
\stoptextrule \blank[line]

\starttextrule{\Pmtvisual \enskip Typ {\definedfont[SansBoldItalic] Sichtkontrolle} \enskip}
Mittels\index{Sichtkontrolle} Sichtprüfung durchzuführender Kontrollpunkt.
Festgestellte Störungen oder Schäden müssen der für die Wartung zuständigen
Person mitgeteilt werden.
\stoptextrule
}

Bei allen Wartungsarbeiten sind drei Qualifikationsstufen (Anforderung an
Personal und Ausrüstung) und vier Wartungstypen (Art der Arbeit) zu
unterscheiden.

\startTwoPar
{\em Qualifikationsstufen:}	\par \blank [medium]
\start \setupwhitespace	[none]
\symDescr{\textSymb{maint:check}} Check \par
\symDescr{\textSymb{maint:pro}} Servicearbeit \par
\symDescr{\textSymb{maint:special}} Spezialarbeit	\par
\stop
\TwoPar
{\em Wartungstypen:}	\par \blank [medium]
\start \setupwhitespace	[none]
\symDescr{\textSymb{maint:visual}} Sichtkontrolle	\par
\symDescr{\textSymb{maint:function}} Funktionskontrolle	\par
\symDescr{\textSymb{maint:level}} Füllstandskontrolle \par
\symDescr{\textSymb{maint:exchange}} Betriebsstoffaustausch \par
\symDescr{\textSymb{maint:generic}} Sonstige Tätigkeiten \par
\stop
\stopTwoPar

\handItem{Siehe hierzu die Erläuterungen in den Randspalten.}


\placeNote[][service_picto]{}{%
 \noteF
\starttextrule{\Pmtfunction \enskip Typ {\definedfont[SansBoldItalic] Funktionskontrolle} \enskip}
Kontrollen, die\index{Kontrolle+Funktion} über eine Sichtprüfung
hinausgehen, \eG\ Funktionskontrolle der Bremse, Zustandskontrolle von
Bauteilen, manuelle Kontrolle auf festen Sitz.
\stoptextrule \blank[big]
\starttextrule{\Pmtlevel \enskip Typ {\definedfont[SansBoldItalic] Füllstandskontrolle} \enskip}
Die\index{Kontrolle+Füllstand} Füllstandskontrolle von Betriebsstoffen umfasst
die Kontrolle~– per Sichtprüfung oder mit dem Messstab~– sowie ggf. das
Auffüllen mit dem entsprechenden Betriebsstoff. {\em Hierzu zählt auch die
Zentralschmierung.} Lesen Sie dazu den Abschnitt \about[sec:liqquantities],
wo Sie Informationen zu Qualität und Menge der Betriebsstoffe finden.
\stoptextrule \blank[big]
\starttextrule{\Pmtexchange \enskip Typ {\definedfont[SansBoldItalic] Betriebsstoffaustausch} \enskip}
Dazu\index{Betriebsstoffe+Austausch} gehört der Wechsel des Betriebsstoffs
sowie die anschließende Füllstandskontrolle. Informationen zu Qualität und
Menge der Betriebsstoffe finden Sie \atpage[sec:liqquantities].
\stoptextrule \blank[big]
\starttextrule{\Pmtgeneric \enskip Typ {\definedfont[SansBoldItalic] Sonstige Tätigkeiten} \enskip}
Verschiedene, entsprechend der Anweisungen fachgerecht auszuführende Wartungsarbeiten.
\stoptextrule
}%

\blank[big]
%% TODO; en
\starttextbackground[FC]
\setupparagraphs [PictPar][1][width=6em,inner=\hfill]
\startPictPar\PMgeneric~\Penvironment\PictPar
Bitte beachten Sie bei der Ausführung der Kontroll- und Wartungspunkte die
Sicherheits- und Umweltschutzbestimmungen. \BosFull{boschung}
weist jegliche Haftung für Personen- oder Materialschäden zurück, die in
Folge der Nichtbeachtung dieser Bestimmungen entstehen.
\stopPictPar
\stoptextbackground
\page


%%%%%%%%%%%%%%%%%%%%%%%%%%%%%%%%%%%%%%%%%%%%%%%%%%%%%%%%%%%%%%%%%%%%%%%%%%%%%%
\section{Wartungsplan des Fahrzeugs}

Die\index{Kontrolle, periodische} Fahrzeugwartung besteht aus regelmäßig
auszuführenden Wartungarbeiten, aus periodischen Kontrollen (Checks) sowie einem
einmalig auszuführenden Service\footnote{Alle Stundenangaben verstehen sich als
Betriebsstunden}:

\starttextbackground[FC]
\starttabulate [|w(37mm)B|w(19mm)B|p({\dimexpr\textwidth-(56mm+2.5em)\relax})|]
\NC \Pmtpro\enskip Wartungsarbeiten \NC 50\,h \NC Erster Service nach 50\,h \NC \NR
\NC \NC 600\,h \NC Reguläre Wartung alle 600\,h / 12 Monate \NC \NR
\NC \NC 1200\,h \NC Reguläre Wartung alle 1200\,h / 2 Jahre \NC \NR
\NC \NC 2400\,h \NC Reguläre Wartung alle 2400\,h / 4 Jahre \NC \NR
\NC \NC 4800\,h \NC Reguläre Wartung alle 4800\,h / 8 Jahre \NC \NR
\NC \Pmtcheck\enskip Checkliste \NC Täglich \NC Während der ganzen Arbeitssaison \NC \NR
\NC \NC Wöchentlich \NC Während der ganzen Arbeitssaison \NC \NR
\stoptabulate
\stoptextbackground
\blank [big]

Der folgende Wartungsplan bezieht sich auf das Grundfahrzeug. Beachten Sie
auch den Wartungsplan zu den Aggregaten (\eG\ Kehrmaschine, Schmutzbehälter) ab
\at{Seite}[sec:schedaggr].

Bei doppelten Angaben zum Wartungsintervall (\eG\ \quotation{Alle
600\,h~/ 12~Monate}) besitzt der jeweils zuerst eintretende Zeitpunkt Priorität.

Die Arbeiten der einzelnen Wartungspläne~– außer der einmaligen Wartung nach
50~Stunden~– sind kumulativ auszuführen: Alle 1200~Betriebsstunden ist die
600-Stunden||Wartung {\em und} die 1200-Stunden||Wartung durchzuführen; alle
2400~Betriebsstunden ist die 600-Stunden-, die 1200-Stunden {\em und} die
2400-Stunden||Wartung durchzuführen; etc.

\page [yes]

\setup[pagestyle:marginless]

\start
\setup[tbl:schedule]

\subsection[table:scheduledaily]{Tägliche Kontrolle}

\bTABLE
\bTABLEhead
\bTR \bTD Art \eTD \bTD Tägliche Kontrolle \eTD \bTD \Tcheck \eTD \bTD Ref. \eTD \eTR
\eTABLEhead
\bTABLEbody
\bTR \bTD \Tgen \eTD \bTD Fahrzeug reinigen \eTD \bTD \Tcheck \eTD \bTD \inP[sec:cleaning] \eTD \eTR
\bTR \bTD \Tvis \eTD \bTD Fahrzeug auf eventuelle Beschädigungen überprüfen \eTD \bTD \Tcheck \eTD \bTD \emptY \eTD \eTR
\bTR \bTD \Tvis \eTD \bTD Kontrolle auf Leckagen \eTD \bTD \Tcheck \eTD \bTD \emptY \eTD \eTR
\bTR \bTD \Tlev \eTD \bTD Motorölstand des Dieselmotors überprüfen (mit Messstab!) \eTD \bTD \Tcheck \eTD \bTD \inP[ssSec:vw:oilLevel] \eTD \eTR
\bTR \bTD \Tlev \eTD \bTD Kühlflüssigkeitsstand des Dieselmotors überprüfen \eTD \bTD \Tcheck \eTD \bTD \inP[sSec:vw:cooling] \eTD \eTR
\bTR \bTD \Tlev \eTD \bTD Hydraulikflüssigkeitsstand prüfen (Schauglas am Tank) \eTD \bTD \Tcheck \eTD \bTD \inP[sec:hydraulic] \eTD \eTR
\bTR \bTD \Tlev \eTD \bTD Kraftstoffstand überprüfen \eTD \bTD \Tcheck \eTD \bTD \emptY \eTD \eTR
\bTR \bTD \Tlev \eTD \bTD Scheibenwaschflüssigkeitsstand prüfen \eTD \bTD \Tcheck \eTD \bTD \inP[sec:liqquantities] \eTD \eTR
\bTR \bTD \Tfun \eTD \bTD Funktionskontrolle der Kontrollleuchten und der Beleuchtung von Instrumententafel und Kontrollpanel \eTD \bTD \Tcheck \eTD \bTD \emptY \eTD \eTR
\bTR \bTD \Tfun \eTD \bTD Funktionskontrolle der Feststellbremse \eTD \bTD \Tcheck \eTD \bTD \emptY \eTD \eTR
\bTR \bTD \Tfun \eTD \bTD Funktionskontrolle der Beleuchtungs- und Signaleinrichtungen \eTD \bTD \Tcheck \eTD \bTD \inP[sec:lighting] \eTD \eTR
\eTABLEbody
\eTABLE

\testpage [8]
\subsection[table:scheduleweekly]{Wöchentliche Kontrolle}

\bTABLE
\bTABLEhead
\bTR \bTD Art \eTD \bTD Wöchentliche Kontrolle \eTD \bTD \Tcheck \eTD \bTD Ref. \eTD \eTR
\eTABLEhead
\bTABLEbody
\bTR \bTD \Tfun \eTD \bTD Reifenfülldruck prüfen (Druck siehe Rad||Typenschild im Führerhaus) \eTD \bTD \Tcheck \eTD \bTD \inP[sec:pneumatiques] \eTD \eTR
\bTR \bTD \Tgen \eTD \bTD Luftfilterpatrone überprüfen und ggf. reinigen \eTD \bTD \Tcheck \eTD \bTD \inP[sSec:vw:airFilter] \eTD \eTR
\bTR \bTD \Tgen \eTD \bTD Alle Schmierpunkte schmieren (Chassis, Knicklenkung) \eTD \bTD \Tcheck \eTD \bTD \inP[sec:grasing:plan] \eTD \eTR
\eTABLEbody
\eTABLE
%%%%%%%%%%%%%%%%%%%%%%%%%%%%%%%%%%%%%%%%%%%%%%%%%%%%%%%%%%%%%%%%%%%%%%%%%%%%%%


\subsection [sec:50h]{Wartung nach 50\,h~– einmalig}

\bTABLE
\bTABLEhead
\bTR \bTD Art \eTD \bTD Wartung nach 50\,h~– einmalig \eTD \bTD Q. \eTD \bTD Ref. \eTD \eTR
\eTABLEhead
\bTABLEbody
\bTR \bTD \Tlev \eTD \bTD Kühlflüssigkeitsstand des Dieselmotors überprüfen \eTD \bTD \Tcheck \eTD \bTD \inP[sSec:vw:cooling] \eTD \eTR
\bTR \bTD \Tgen \eTD \bTD Luftfilterkartusche reinigen; ggf. ersetzen \eTD \bTD \Tcheck \eTD \bTD \inP[sSec:vw:airFilter] \eTD \eTR
\bTR \bTD \Tgen \eTD \bTD Hydrauliköl||Rücklauf- und Ansaugfilter ersetzen \eTD \bTD \Tcheck \eTD \bTD \inP[sec:hydraulic] \eTD \eTR
\bTR \bTD \Tlev \eTD \bTD Hydraulikflüssigkeitsstand prüfen (Schauglas am Tank) \eTD \bTD \Tcheck \eTD \bTD \inP[sec:hydraulic] \eTD \eTR
% \bTR \bTD \Tlev \eTD \bTD Zentralschmieranlage (Option): Schmiermittelvorrat und -konsistenz überprüfen \eTD \bTD \Tpro \eTD \bTD \inP[main:graissageCentral] \eTD \eTR
\bTR \bTD \Tgen \eTD \bTD Alle Schmierpunkte schmieren (Fahrgestell, Knicklenkung) \eTD \bTD \Tcheck \eTD \bTD \inP[sec:grasing:plan] \eTD \eTR
\bTR \bTD \Tlev \eTD \bTD Füllstand der Scheibenwaschflüssigkeit prüfen \eTD \bTD \Tcheck \eTD \bTD \inP[sec:liqquantities] \eTD \eTR
\bTR \bTD \Tfun \eTD \bTD Befestigungen des Führerhauses überprüfen \eTD \bTD \Tcheck \eTD \bTD \emptY \eTD \eTR
\bTR \bTD \Tfun \eTD \bTD Befestigungen des Motors am Chassis überprüfen \eTD \bTD \Tcheck \eTD \bTD \emptY \eTD \eTR
\bTR \bTD \Tfun \eTD \bTD Befestigungen der Pumpen am Motor überprüfen \eTD \bTD \Tcheck \eTD \bTD \emptY \eTD \eTR
\bTR \bTD \Tfun \eTD \bTD Befestigungen des Kombikühlers überprüfen \eTD \bTD \Tcheck \eTD \bTD \emptY \eTD \eTR
\bTR \bTD \Tfun \eTD \bTD Befestigungen der Achsen überprüfen \eTD \bTD \Tcheck \eTD \bTD \emptY \eTD \eTR
\bTR \bTD \Tfun \eTD \bTD Räder\index{Anzugsdrehmoment+Räder} auf festen Sitz prüfen (Anzugsdrehmoment: 180\,Nm) \eTD \bTD \Tcheck \eTD \bTD \emptY \eTD \eTR
\bTR \bTD \Tfun \eTD \bTD Reifenfülldruck prüfen (Druck siehe Rad||Typenschild im Führerhaus) \eTD \bTD \Tcheck \eTD \bTD \inP[sec:pneumatiques] \eTD \eTR
\bTR \bTD \Tgen \eTD \bTD Weg des Feststellbremshebels überprüfen|/|nachstellen \eTD \bTD \Tpro \eTD \bTD \emptY \eTD \eTR
\bTR \bTD \Tgen \eTD \bTD Zustand der Batterie prüfen; Pole|/|Klemmen reinigen \eTD \bTD \Tcheck \eTD \bTD \inP[sec:battcheck] \eTD \eTR
\bTR \bTD \Tgen \eTD \bTD Einstellung der Scheinwerfer gemäß Straßenverkehrsordnung prüfen, ggf. einstellen \eTD \bTD \Tcheck \eTD \bTD \inP[sec:lighting] \eTD \eTR
\bTR \bTD \Tgen \eTD \bTD Spulenkerne der Magnetventile mit Kupferfett schmieren \eTD \bTD \Tpro \eTD \bTD \inP[sec:solenoid] \eTD \eTR
\bTR \bTD \Tgen \eTD \bTD Ereignisspeicher (Vpad und Motorsteuergerät) auslesen; ggf. Ursachen der Fehler beheben \eTD \bTD \Tcheck \eTD \bTD \inP[sSec:vw:faultMemory], \inP[vpad:error] \eTD \eTR
\eTABLEbody
\eTABLE

%%%%%%%%%%%%%%%%%%%%%%%%%%%%%%%%%%%%%%%%%%%%%%%%%%%%%%%%%%%%%%%%%%%%%%%%%%%%%%


\subsection {Wartung alle 600\,h / 12 Monate}

\bTABLE
\bTABLEhead
\bTR \bTD Art \eTD \bTD Wartung alle 600\,h / 12 Monate \eTD \bTD Q. \eTD \bTD Ref. \eTD \eTR
\eTABLEhead
\bTABLEbody
\bTR \bTD \Tchg \eTD \bTD Ölwechsel Dieselmotor \eTD \bTD \Tpro \eTD \bTD \inP[ssSec:vw:oilDraining] \eTD \eTR
\bTR \bTD \Tgen \eTD \bTD Motorölfilter ersetzen \eTD \bTD \Tpro \eTD \bTD \inP[ssSec:vw:oilFilter] \eTD \eTR
\bTR \bTD \Tgen \eTD \bTD Kraftstofffilter ersetzen \eTD \bTD \Tpro \eTD \bTD \inP[ssSec:vw:fuelFilter] \eTD \eTR
\bTR \bTD \Tgen \eTD \bTD Dichtheit des Motors und der Komponenten im Motorraum prüfen \eTD \bTD \Tpro \eTD \bTD \emptY \eTD \eTR
\bTR \bTD \Tgen \eTD \bTD Dichtheit und Befestigung der Auspuffanlage prüfen \eTD \bTD \Tpro \eTD \bTD \emptY \eTD \eTR
\bTR \bTD \Tgen \eTD \bTD Zustand des Keilrippenriemens des Dieselmotors überprüfen, ggf. ersetzen \eTD \bTD \Tpro \eTD \bTD \inP[sSec:vw:belt] \eTD \eTR
\bTR \bTD \Tlev \eTD \bTD Kühlflüssigkeitsstand des Dieselmotors überprüfen \eTD \bTD \Tcheck \eTD \bTD \inP[sSec:vw:cooling] \eTD \eTR
\bTR \bTD \Tgen \eTD \bTD Luftfilterkartusche ersetzen \eTD \bTD \Tcheck \eTD \bTD \emptY \eTD \eTR
\bTR \bTD \Tgen \eTD \bTD Hydrauliköl||Rücklauf- und Ansaugfilter ersetzen \eTD \bTD \Tcheck \eTD \bTD \inP[sec:hydraulic] \eTD \eTR
\bTR \bTD \Tlev \eTD \bTD Füllstand des Hydraulikflüssigkeitstanks prüfen \eTD \bTD \Tcheck \eTD \bTD \inP[sec:hydraulic] \eTD \eTR
% \bTR \bTD \Tlev \eTD \bTD Zentralschmieranlage (Option): Schmiermittelvorrat und -konsistenz überprüfen \eTD \bTD \Tpro \eTD \bTD \inP[main:graissageCentral] \eTD \eTR
\bTR \bTD \Tgen \eTD \bTD Alle Schmierpunkte schmieren (Fahrgestell, Knicklenkung) \eTD \bTD \Tcheck \eTD \bTD \inP[sec:grasing:plan] \eTD \eTR
\bTR \bTD \Tlev \eTD \bTD Füllstand der Scheibenwaschflüssigkeit prüfen \eTD \bTD \Tcheck \eTD \bTD \inP[sec:liquids] \eTD \eTR
\bTR \bTD \Tfun \eTD \bTD Befestigungen des Führerhauses überprüfen \eTD \bTD \Tcheck \eTD \bTD \emptY \eTD \eTR
\bTR \bTD \Tfun \eTD \bTD Befestigungen des Motors am Chassis überprüfen \eTD \bTD \Tcheck \eTD \bTD \emptY \eTD \eTR
\bTR \bTD \Tfun \eTD \bTD Befestigungen der Pumpen am Motor überprüfen \eTD \bTD \Tcheck \eTD \bTD \emptY \eTD \eTR
\bTR \bTD \Tfun \eTD \bTD Befestigungen des Kombikühlers überprüfen \eTD \bTD \Tcheck \eTD \bTD \emptY \eTD \eTR
\bTR \bTD \Tfun \eTD \bTD Befestigungen der Achsen überprüfen \eTD \bTD \Tcheck \eTD \bTD \emptY \eTD \eTR
\bTR \bTD \Tgen \eTD \bTD Weg des Feststellbremshebels überprüfen|/|nachstellen \eTD \bTD \Tpro \eTD \bTD \emptY \eTD \eTR
\bTR \bTD \Tgen \eTD \bTD Bremstrommeln und -backen prüfen|/|reinigen; Bremsmechanismus reinigen \eTD \bTD \Tpro \eTD \bTD \inP[sec:brake] \eTD \eTR
\bTR \bTD \Tfun \eTD \bTD Räder\index{Anzugsdrehmoment+Räder} auf festen Sitz prüfen (Anzugsdrehmoment: 180\,Nm) \eTD \bTD \Tcheck \eTD \bTD \emptY \eTD \eTR
\bTR \bTD \Tfun \eTD \bTD Reifenfülldruck prüfen (Druck siehe Rad||Typenschild im Führerhaus) \eTD \bTD \Tcheck \eTD \bTD \inP[sec:pneumatiques] \eTD \eTR
\bTR \bTD \Tgen \eTD \bTD Zustand der Batterie prüfen; Pole|/|Klemmen reinigen \eTD \bTD \Tcheck \eTD \bTD \inP[sec:battcheck] \eTD \eTR
\bTR \bTD \Tgen \eTD \bTD Einstellung der Scheinwerfer gemäß Straßenverkehrsordnung prüfen, ggf. einstellen \eTD \bTD \Tcheck \eTD \bTD \inP[sec:lighting] \eTD \eTR
\bTR \bTD \Tgen \eTD \bTD Spulenkerne der Magnetventile mit Kupferfett schmieren \eTD \bTD \Tpro \eTD \bTD \inP[sec:solenoid] \eTD \eTR
%%Note: Chapter has been removed
%\bTR \bTD \Tgen \eTD \bTD Rostschutz prüfen, ggf. nachbessern|/|erneuern \eTD \bTD \Tspecial \eTD \bTD \inP[sec:anticorrosion] \eTD \eTR
\bTR \bTD \Tgen \eTD \bTD Ereignisspeicher (Vpad und Motorsteuergerät) auslesen; ggf. Ursachen der Fehler beheben \eTD \bTD \Tcheck \eTD \bTD \inP[sSec:vw:faultMemory], \inP[vpad:error] \eTD \eTR
\eTABLEbody
\eTABLE

%%%%%%%%%%%%%%%%%%%%%%%%%%%%%%%%%%%%%%%%%%%%%%%%%%%%%%%%%%%%%%%%%%%%%%%%%%%%%%
\subsection {Wartung alle 1200\,h / 2 Jahre}
\bTABLE
\bTABLEhead
\bTR \bTD Art \eTD \bTD Wartung alle 1200\,h / 2 Jahre \eTD \bTD Q. \eTD \bTD Ref. \eTD \eTR
\eTABLEhead
\bTABLEbody
\bTR \bTD \Tchg \eTD \bTD Hydrauliköl wechseln (Tank) \eTD \bTD \Tpro \eTD \bTD \inP[sec:hydraulic] \eTD \eTR
\bTR \bTD \Tgen \eTD \bTD Kältemittel (R134a) der Klimaanlage erneuern \eTD \bTD \Tspecial \eTD \bTD \emptY \eTD \eTR
\eTABLEbody
\eTABLE

%%%%%%%%%%%%%%%%%%%%%%%%%%%%%%%%%%%%%%%%%%%%%%%%%%%%%%%%%%%%%%%%%%%%%%%%%%%%%%
\subsection {Wartung alle 2400\,h / 4 Jahre}
\bTABLE

\bTABLEhead
\bTR \bTD Art \eTD \bTD Wartung alle 2400\,h / 4 Jahre \eTD \bTD Q. \eTD \bTD Ref. \eTD \eTR
\eTABLEhead
\bTABLEbody
	\bTR \bTD \Tgen \eTD \bTD Zahnriemen des Dieselmotors ersetzen \eTD \bTD \Tspecial \eTD \bTD \emptY \eTD \eTR
\eTABLEbody
\eTABLE

%%%%%%%%%%%%%%%%%%%%%%%%%%%%%%%%%%%%%%%%%%%%%%%%%%%%%%%%%%%%%%%%%%%%%%%%%%%%%%

\subsection {Wartung alle 4800\,h / 8 Jahre}
\bTABLE

\bTABLEhead
\bTR \bTD Art \eTD \bTD Wartung alle 4800\,h / 8 Jahre \eTD \bTD Q. \eTD \bTD Ref. \eTD \eTR
\eTABLEhead
\bTABLEbody
\bTR \bTD \Tgen \eTD \bTD Hydraulik||Schlauchleitungen überprüfen, ggf. ersetzen \eTD \bTD \Tpro \eTD \bTD \inP[sec:hydraulic] \eTD \eTR
\bTR \bTD \Tgen \eTD \bTD Wasserpumpe ersetzen (gleichzeitig mit dem Zahnriemen) \eTD \bTD \Tspecial \eTD \bTD \emptY \eTD \eTR
\eTABLEbody
\eTABLE

\stopregister[index][wartplan]

\page [yes]


\section[sec:schedaggr]{Wartung der Aggregate}

Die Wartung\startregister[index][wartplanAgg]{Wartungsplan+Aggregate} der
Aggregate umfasst regelmäßig auszuführende Wartungsarbeiten\index{Periodische
Kontrolle} sowie tägliche und wöchentliche Kontrollen:

\starttextbackground[FC]
\starttabulate [|w(37mm)B|w(19mm)B|p({\dimexpr\textwidth-(56mm+2.5em)\relax})|]
\NC \Pmtpro\enskip Wartungsarbeiten \NC 50\,h \NC Erster Service nach 50\,h \NC \NR
\NC \NC 600\,h \NC Reguläre Wartung alle 600\,h / 12 Monate \NC \NR
\NC \Pmtcheck\enskip Checkliste \NC Täglich \NC Während der ganzen Arbeitssaison \NC \NR
\NC \NC Wöchentlich \NC Während der ganzen Arbeitssaison \NC \NR
\stoptabulate
\stoptextbackground
\blank [big]

Der folgende Wartungsplan bezieht sich auf Standardaggregate, mit denen die
\sdeux\ üblicherweise ausgerüstet ist. Für spezielle Aggregate, die nicht in
dieser Bedienungsanleitung beschrieben sind, ist der Wartungsplan des
jeweiligen Aggregats zu beachten.


\subsection[table:scheduledaily]{Tägliche Kontrolle}

\bTABLE
\bTABLEhead
\bTR \bTD Art \eTD \bTD Tägliche Kontrolle \eTD \bTD Q. \eTD \bTD Ref. \eTD \eTR
\eTABLEhead
\bTABLEbody
\bTR \bTD \Tgen \eTD \bTD Schmutzbehälter und Recyclingvorrichtung reinigen \eTD \bTD \Tcheck \eTD \bTD \inP[sec:cleaning] \eTD \eTR
\bTR \bTD \Tgen \eTD \bTD Saugmund und Saugkanal reinigen \eTD \bTD \Tcheck \eTD \bTD \inP[sec:cleaning] \eTD \eTR
\eTABLEbody
\eTABLE

\subsection[table:scheduledaily]{Wöchentliche Kontrolle}

\bTABLE
\bTABLEhead
\bTR \bTD Type \eTD \bTD Wöchentliche Kontrolle \eTD \bTD Q. \eTD \bTD Ref. \eTD \eTR
\eTABLEhead
\bTABLEbody
	\bTR \bTD \Tgen \eTD \bTD Verschleiß von Seitenbesen und Frontbesen (Option) überprüfen \eTD \bTD \Tcheck \eTD \bTD \emptY \eTD \eTR
	\bTR \bTD \Tgen \eTD \bTD Saugmundklappe und Saugmundgummis überprüfen \eTD \bTD \Tcheck \eTD \bTD \emptY \eTD \eTR
	\bTR \bTD \Tgen \eTD \bTD Alle Schmierpunkte schmieren (Schmutzbehälter, Besen, Saugmund) \eTD \bTD \Tcheck \eTD \bTD \inP[sec:grasing:plan] \eTD \eTR
\eTABLEbody
\eTABLE


\subsection {Wartung nach 50\,h~– einmalig}

\bTABLE
\bTABLEhead
\bTR \bTD
Art \eTD \bTD Wartung nach 50\,h~– einmalig \eTD \bTD Q. \eTD \bTD Ref. \eTD \eTR
\eTABLEhead
\bTABLEbody
	\bTR \bTD \Tfun \eTD \bTD Befestigungen der Besen überprüfen \eTD \bTD \Tcheck \eTD \bTD \emptY \eTD \eTR
	\bTR \bTD \Tgen \eTD \bTD Saugmundklappe und Saugmundgummis einstellen \eTD \bTD \Tpro \eTD \bTD \emptY \eTD \eTR
\eTABLEbody
\eTABLE


\subsection {Wartung alle 600\,h|/|12 Monate}


\bTABLE
\bTABLEhead
\bTR \bTD Art \eTD \bTD Wartung alle 600\,h|/|12 Monate \eTD \bTD Q. \eTD \bTD Ref. \eTD \eTR
\eTABLEhead
\bTABLEbody
	\bTR \bTD \Tfun \eTD \bTD Befestigungen der Besen überprüfen \eTD \bTD \Tcheck \eTD \bTD \emptY \eTD \eTR
	\bTR \bTD \Tgen \eTD \bTD Saugmundklappe und Saugmundgummis einstellen \eTD \bTD \Tpro \eTD \bTD \inP[sec:main:suctionMouth] \eTD \eTR
\bTR \bTD \Tchg \eTD \bTD Ölwechsel Hochdruckwasserpumpe für Wasserpistole (Option) \eTD \bTD \Tpro \eTD \bTD \emptY \eTD \eTR
\eTABLEbody
\eTABLE

\stopregister[index][wartplanAgg]
\stop


\stopcomponent
% vim: fdm=indent
