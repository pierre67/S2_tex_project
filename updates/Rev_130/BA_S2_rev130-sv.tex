\defineparagraphs[SymVpad][n=2,distance=4mm,rule=off,before={\page[preference]},after={\nobreak\hrule\blank [2*medium]}]
\setupparagraphs [SymVpad][1][width=4em,inner=\hfill]

% \startcolumns

\subsection{Kontrollindikeringar på Vpad-bildskärmen} % nouveau

%% ajout

\startSymVpad
\externalfigure[vpadWarningService][height=1.7\lH]
\SymVpad
\textDescrHead{Kontakta en verkstad}(gul) Dags för normalt fordonsunderhåll
(se \about [sec:schedule] \atpage [sec:schedule])
eller ett motorfel har registrerats (kontakta en fackverkstad).

+\:Fehlermeldungen \# 650 till \# 653, eller \# 703
\stopSymVpad


\startSymVpad
\externalfigure[vpadTDPF][height=1.7\lH]
\SymVpad
\textDescrHead{Partikelfilter}(gul) Regenerering av partikelfiltret startar när driftvillkoren är lämpliga.

{\md Observera:} {\lt Stäng om möjligt {\em inte} av motorn när denna indikering lyser!}
\stopSymVpad



%%%%%%%%%%%%%%%%%%%%%%%%%%%%


\bTR\bTD \externalfigure [v:symbole:power] \eTD\bTD Slå från bildskärmen \eTD\bTD Håll intryckt i ca 5 sek. för att slå från bildskärmen på Vpad. \eTD\eTR
\bTR\bTD \framed[frame=off]{\externalfigure [v:symbole:frontBrush]\externalfigure [v:symbole:frontBrush:black]}
\eTD\bTD Tredje kvast\index{Tredje kvast} (tillval) \eTD\bTD Aktivera den tredje kvasten.
Den tredje kvasten kan nu aktiveras på det sätt som beskrivs på sidan \at[sec:using:frontBrush]. \eTD\eTR


%%%%%%%%%%%%%%%%%%%%%%%%%%%% corriger

\startsection [title={Menyerna på Vpad},
reference={vpad:menu}]



\subsection{Fler symboler på Vpad-bildskärmen}


\subsubsubject{Renvatten- och återvinningsvattenreservoar}


\subsubsubject{Sugsystem} % nouveau

{\em Den här symbolen visas endast när kvastarna är avaktiverade.}

\startSymVpad
\externalfigure[sym:vpad:sucker]
\SymVpad
\textDescrHead{Sugmunstycke} Sugsystem\index{Sugmunstycke} aktiverat:
Sugmunstycket är nedsänkt och turbinen är aktiverad.
\stopSymVpad


\subsubsubject{Sidokvast} % nouveau

{\em Den här symbolen visas endast när den tredje kvasten inte är aktiverad.}

\startSymVpad
\externalfigure[sym:vpad:sideBrush:83]
\SymVpad
\textDescrHead{Sidokvast} Kvast\index{Sopning}\index{Sidokvast} aktiverad. Rotationshastigheten (i \% av den högsta rotationshastigheten [V\low{max}]) visas under symbolen, den aktuella avlastningen för respektive kvast visas ovanför symbolen (\type{ } = flytande läge, 14 = högsta avlastning).

{\md Avlastning:} {\lt Ju lägre avlastning desto högre är kvastarnas tryck mot marken.}
\stopSymVpad


\startSymVpad
\externalfigure[sym:vpad:sideBrush:float:60]
\SymVpad
\textDescrHead{Flytande läge}(grön i nederkanten)
För att slå från avlastningen ska du trycka spaken framåt i ca 2 sekunder. Kvasten ligger nu mot marken med hela sin egenvikt. Kvastarnas rotationshastighet ligger på 60\hairspace\% av V\low{max} (exempel).
\stopSymVpad

\startSymVpad
\externalfigure[sym:vpad:sideBrush]
\SymVpad
\textDescrHead{Sidokvastar} Kvastarna är aktiverade. De står stilla och är i upplyft läge.
\stopSymVpad


\subsubsubject{Tredje kvast (tillval)} % nouveau

\startSymVpad
\externalfigure[sym:vpad:frontBrush]
\SymVpad
\textDescrHead{Tredje kvast} Den tredje kvasten\index{Tredje kvast} är aktiverad. Rotationshastigheten (i \% av den högsta rotationshastigheten [V\low{max}]) visas under symbolen.
\stopSymVpad


\startSymVpad
\externalfigure[sym:vpad:frontBrush:left]
\SymVpad
\textDescrHead{Flytande läge}(grön i nederkanten)
För att slå från avlastningen ska du trycka spaken framåt i ca 2 sekunder. Kvasten ligger nu mot marken med hela sin egenvikt. Kvastarnas rotationshastighet ligger på 70\hairspace\% av V\low{max} (exempel).

{\md Rotationsriktning:} {\lt I ovankanten visas rotationsriktningen (svart pil på gul bakgrund).}
\stopSymVpad

\stopsection




\setups [pagestyle:marginless]


\startsection [title={Arbeta med den tredje kvasten (tillval)},
reference={sec:using:frontBrush},
]

\startSteps
\item Ta\index{Sopa} fordonet i drift så som beskrivs i \in{avsnitt}[sec:using:start] \atpage[sec:using:start] .
\item Aktivera\index{Tredje kvast} \aW{arbetsläget}(knapp på utsidan av växelspaken).
\stopSteps

% \getbuffer [work:config]

\startSteps [continue]
\item Kontrollera att den tredje kvasten är aktiverad på Vpad-bildskärmen
(se \textSymb{vpadFrontBrush} \textSymb{vpadFrontBrushK}, \atpage[vpad:menu]).
\item Tryck på knappen~\textSymb{joy_key_frontbrush_act} för att aktivera den tredje kvastens hydraulsystem.
\item Tryck på knappen ~\textSymb{joy_key_frontbrush_left} eller~\textSymb{joy_key_frontbrush_right} för att den tredje kvasten ska rotera i önskad riktning.

\item Reglera rotationshastigheten med hjälp av knapparna~\textSymb{joy_key_frontbrush_increase} och ~\textSymb{joy_key_frontbrush_decrease} på multifunktionskonsolen.

\item Positionera kvastarna så som visas på figurerna nedan med hjälp av spakarna.

\stopSteps

{\md Observera:} {\lt För att du ska kunna positionera sidokvastarna måste du aktivera den tredje kvastens hydraulsystem med hjälp av knappen ~\textSymb{joy_key_frontbrush_act}.}
\vfill

\start
\setupcombinations [width=\textwidth]

\placefig[here][fig:brush:position]{Positionera den tredje kvasten}
{\startcombination [2*1]
{\externalfigure [work:frontBrush:move]}{Uppåt/nedåt; åt vänster/höger}
{\externalfigure [work:frontBrush:incline]}{Tvär-/längslutning}
\stopcombination}
\stop



\stopsection