\defineparagraphs[SymVpad][n=2,distance=4mm,rule=off,before={\page[preference]},after={\nobreak\hrule\blank [2*medium]}]
\setupparagraphs [SymVpad][1][width=4em,inner=\hfill]

% \startcolumns

\subsection{Indicaciones de control en la pantalla Vpad} % nouveau

%% ajout

\startSymVpad
\externalfigure[vpadWarningService][height=1.7\lH]
\SymVpad
\textDescrHead{Acudir al taller}(amarillo) Mantenimiento regular del vehículo (véase \about [sec:schedule] \atpage [sec:schedule]) o se ha registrado un error en el motor (necesario un taller técnico especializado).

+\:Mensajes de error \# 650 a \# 653, o \# 703
\stopSymVpad


\startSymVpad
\externalfigure[vpadTDPF][height=1.7\lH]
\SymVpad
\textDescrHead{Filtro de partículas}(amarillo) Se iniciará la regeneración del filtro de partículas cuando el estado de servicio lo permita.

{\md Aviso:} {\lt ¡{\emNo} apague el motor mientras sea posible y esté encendida esta indicación!}
\stopSymVpad



%%%%%%%%%%%%%%%%%%%%%%%%%%%%


\bTR\bTD \externalfigure [v:symbole:power] \eTD\bTD Desconectar pantalla \eTD\bTD Mantener pulsada unos 5 s para desconectar la pantalla del Vpad. \eTD\eTR
\bTR\bTD \framed[frame=off]{\externalfigure [v:symbole:frontBrush]\externalfigure [v:symbole:frontBrush:black]}
\eTD\bTD Tercer cepillo\index{3er cepillo} (opción) \eTD\bTD Activar el tercer cepillo.
Ahora puede activarse el tercer cepillo como se describe en la página \at[sec:using:frontBrush]. \eTD\eTR


%%%%%%%%%%%%%%%%%%%%%%%%%%%% corriger

\startsection [title={Los menús del Vpad}, reference={vpad:menu}]



\subsection{Otros símbolos en la pantalla Vpad}


\subsubsubject{Depósito de agua limpia y agua reciclada}


\subsubsubject{Sistema de aspiración} % nouveau

{\em Este símbolo se muestra solo cuando los cepillos están desactivados.}

\startSymVpad
\externalfigure[sym:vpad:sucker]
\SymVpad
\textDescrHead{Boca de aspiración} Sistema de aspiración\index{boca de aspiración} activada:
La boca de aspiración está bajada y la turbina está activada.
\stopSymVpad


\subsubsubject{Cepillo lateral} % nouveau

{\em Este símbolo se muestra solo cuando el tercer cepillo está desactivado.}

\startSymVpad
\externalfigure[sym:vpad:sideBrush:83]
\SymVpad
\textDescrHead{Cepillo lateral} Cepillo\index{barrer}\index{cepillo lateral} activado. La velocidad de rotación (en \% respecto a la velocidad de rotación máx. [V\low{max}]) se muestra debajo del símbolo. La descarga actual del cepillo correspondiente se muestra encima del símbolo (\type{ } = posición de flotación, 14 = descarga máxima).

{\md Descarga:} {\lt Cuanto menor sea la descarga, mayor será la presión de los cepillos sobre el suelo.}
\stopSymVpad


\startSymVpad
\externalfigure[sym:vpad:sideBrush:float:60]
\SymVpad
\textDescrHead{Posición de flotación}(verde en el borde inferior)
Para desconectar la descarga mantenga pulsado el joystick durante unos 2 s hacia delante; el cepillo reposa ahora con todo su peso sobre el suelo. La velocidad de rotación de los cepillos está a 60\hairspace\% del V\low{max} (ejemplo).
\stopSymVpad

\startSymVpad
\externalfigure[sym:vpad:sideBrush]
\SymVpad
\textDescrHead{Cepillos laterales} Los cepillos están activados. Están parados y elevados.
\stopSymVpad


\subsubsubject{Tercer cepillo (opción)} % nouveau

\startSymVpad
\externalfigure[sym:vpad:frontBrush]
\SymVpad
\textDescrHead{Tercer cepillo} El tercer cepillo\index{3er cepillo} está activado. La velocidad de rotación (en \% respecto a la velocidad de rotación máx. [V\low{max}]) se muestra debajo del símbolo.
\stopSymVpad


\startSymVpad
\externalfigure[sym:vpad:frontBrush:left]
\SymVpad
\textDescrHead{Posición de flotación}(verde en el borde inferior)
Para desconectar la descarga mantenga pulsado el joystick durante unos 2 s hacia delante; el cepillo reposa ahora con todo su peso sobre el suelo. La velocidad de rotación de los cepillos está a 70\hairspace\% del V\low{max} (ejemplo).

{\md Sentido de rotación:} {\lt En el borde superior se muestra el sentido de rotación (flecha negra sobre fondo amarillo).}
\stopSymVpad

\stopsection




\setups [pagestyle:marginless]


\startsection [title={Trabajar con el tercer cepillo (opción)},
reference={sec:using:frontBrush},
]

\startSteps
\item Ponga\index{barrer} el vehículo en funcionamiento como se describe en el \in{apartado}[sec:using:start] \atpage[sec:using:start].
\item Active\index{3er cepillo} el \aWmodo de {trabajo} (botón exterior en la palanca de marchas).
\stopSteps

% \getbuffer [work:config]

\startSteps [continue]
\item Asegúrese de que el tercer cepillo en la pantalla Vpad está activado (véase \textSymb{vpadFrontBrush} \textSymb{vpadFrontBrushK}, \atpage[vpad:menu]).
\item Pulse la tecla~\textSymb{joy_key_frontbrush_act} para accionar el sistema hidráulico del tercer cepillo.
\item Pulse la tecla~\textSymb{joy_key_frontbrush_left} o~\textSymb{joy_key_frontbrush_right}, para que el tercer cepillo rote en el sentido deseado.

\item Ajuste la velocidad de rotación mediante las teclas~\textSymb{joy_key_frontbrush_increase} y~\textSymb{joy_key_frontbrush_decrease} de la consola multifuncional.

\item Posicione el cepillo mediante el joystick como se muestra en las figuras.

\stopSteps

{\md Aviso:} {\lt Para poder posicionar los cepillos laterales deberá desactivarse el sistema hidráulico del tercer cepillo mediante la tecla~\textSymb{joy_key_frontbrush_act}.}
\vfill

\start
\setupcombinations [width=\textwidth]

\placefig[here][fig:brush:position]{Posicionar el tercer cepillo}
{\startcombination [2*1]
{\externalfigure [work:frontBrush:move]}{Hacia arriba/abajo, a derecha/izquierda}
{\externalfigure [work:frontBrush:incline]}{Inclinación transversal/longitudinal}
\stopcombination}
\stop



\stopsection


%%% Änderungen Benedikt Sturny

% S. 68
\item Active el freno de estacionamiento y coloque la palanca de marchas en la posición \aW{neutra}. (Necesario para activar el interruptor de volcado del depósito.)


% S. 93

\subsection[niveau_hydrau]{Nivel de llenado}

Una mirilla transparente\index{nivel de llenado+líquido hidráulico}\index{mantenimiento+equipo hidráulico} permite comprobar el nivel de aceite hidráulico.
Cuando el nivel de aceite hidráulico se reduzca, deberá determinarse la causa antes de volver a llenarlo. Cumpla los intervalos de cambio prescritos (tabla arriba) y las especificaciones para el líquido hidráulico (tabla \at{página }[sec:liqquantities]).


\subsubsection{Añadir líquido hidráulico}

Añada líquido hidráulico hasta que la mirilla central esté completamente cubierta.
Arranque el motor y añada algo de aceite en caso necesario hasta alcanzar el nivel de llenado requerido.
