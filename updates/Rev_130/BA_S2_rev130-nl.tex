\defineparagraphs[SymVpad][n=2,distance=4mm,rule=off,before={\page[preference]},after={\nobreak\hrule\blank [2*medium]}]
\setupparagraphs [SymVpad][1][width=4em,inner=\hfill]

% \startcolumns

\subsection{Controleweergaven op het Vpad-scherm} % nouveau

%% ajout

\startSymVpad
\externalfigure[vpadWarningService][height=1.7\lH]
\SymVpad
\textDescrHead{Naar garage }(geel) Normaal voertuigonderhoud vereist (zie \about [sec:schedule] \atpage [sec:schedule]) of er werd een motorfout geregistreerd (gespecialiseerde garage vereist).

+\:Foutmeldingen\#650 tot \#653, of \#703
\stopSymVpad


\startSymVpad
\externalfigure[vpadTDPF][height=1.7\lH]
\SymVpad
\textDescrHead{Partikelfilter}(geel) Regeneratie van de partikelfilter wordt gestart van zodra de bedrijfstoestand dat toelaat.

{\md Opmerking:} {\lt Schakel indien mogelijk de motor {\em niet} uit zolang deze weergave brandt!}
\stopSymVpad



%%%%%%%%%%%%%%%%%%%%%%%%%%%%


\bTR\bTD \externalfigure [v:symbole:power] \eTD\bTD Scherm uitschakelen \eTD\bTD Ca. 5 sec ingedrukt houden om het scherm van de Vpad uit te schakelen. \eTD\eTR
\bTR\bTD \framed[frame=off]{\externalfigure [v:symbole:frontBrush]\externalfigure [v:symbole:frontBrush:black]}
\eTD\bTD Derde borstel \index{3e borstel} (optie) \eTD\bTD Derde borstel vrijschakelen.
De derde borstel kan nu geactiveerd worden, zoals beschreven op pagina \at[sec:using:frontBrush]. \eTD\eTR


%%%%%%%%%%%%%%%%%%%%%%%%%%%% corriger

\startsection [title={De menu's van de Vpad}, reference={vpad:menu}]



\subsection{Meer symbolen op het scherm van de Vpad}


\subsubsubject{Voorraad vers en gerecycleerd water}


\subsubsubject{Zuigsysteem} % nouveau

{\em Dit symbool wordt enkel weergegeven als de borstels gedeactiveerd zijn.}

\startSymVpad
\externalfigure[sym:vpad:sucker]
\SymVpad
\textDescrHead{Zuigmond } zuigsysteem\index{zuigmond} geactiveerd:
Zuigmond is neergelaten en turbine is geactiveerd.
\stopSymVpad


\subsubsubject{Zijborstel} % nouveau

{\em Dit symbool wordt enkel weergegeven als de derde borstel niet geactiveerd is.}

\startSymVpad
\externalfigure[sym:vpad:sideBrush:83]
\SymVpad
\textDescrHead{Zijborstel} borstel\index{vegen}\index{zijborstel} geactiveerd. De draaisnelheid (in \% van de max. draaisnelheid [V\low{max}]) wordt onder het symbool weergegeven, de huidige ontlasting van de desbetreffende borstel wordt boven het symbool weergegeven (\type{ } = zweefstand, 14 = maximale ontlasting).

{\md Ontlasting:} {\lt Hoe lager de ontlasting is, des te hoger is de druk van de borstel op de grond.}
\stopSymVpad


\startSymVpad
\externalfigure[sym:vpad:sideBrush:float:60]
\SymVpad
\textDescrHead{Zweefstand }(groen aan de onderste rand)
Om de ontlasting uit te schakelen, houdt u de joystick ca. 2 sec naar voren gedrukt; de borstel ligt nu met zijn volledige eigen gewicht op de grond. De draaisnelheid van de borstel is 60\hairspace\% van de V\low{max} (voorbeeld).
\stopSymVpad

\startSymVpad
\externalfigure[sym:vpad:sideBrush]
\SymVpad
\textDescrHead{Zijborstel} De borstels zijn geactiveerd. Ze staan stil en zijn opgetild.
\stopSymVpad


\subsubsubject{Derde borstel (optie)} % nouveau

\startSymVpad
\externalfigure[sym:vpad:frontBrush]
\SymVpad
\textDescrHead{Derde borstel} De derde borstel \index{3e borstel} is geactiveerd. De draaisnelheid (in \% van de max. draaisnelheid [V\low{max}]) wordt onder het symbool weergegeven.
\stopSymVpad


\startSymVpad
\externalfigure[sym:vpad:frontBrush:left]
\SymVpad
\textDescrHead{Zweefstand }(groen aan de onderste rand)
Om de ontlasting uit te schakelen, houdt u de joystick ca. 2 sec naar voren gedrukt; de borstel ligt nu met zijn volledige eigen gewicht op de grond. De draaisnelheid van de borstel is 70\hairspace\% van de V\low{max} (voorbeeld).

{\md Draairichting:} {\lt Aan de bovenste rand wordt de draairichting weergegeven (zwarte pijl met gele achtergrond).}
\stopSymVpad

\stopsection




\setups [pagestyle:marginless]


\startsection [title={Met de derde borstel werken (optie)},
reference={sec:using:frontBrush},
]

\startSteps
\item Neem\index{vegen} het voertuig in gebruik zoals beschreven in  \in{hoofdstuk }[sec:using:start] \atpage[sec:using:start].
\item Activeren van de \index{3e borstel} Zie \aW{werk}modus (knop buiten aan de keuzehendel voor het rijniveau).
\stopSteps

% \getbuffer [work:config]

\startSteps [continue]
\item Zorg ervoor dat de derde borstel op het scherm van de Vpad geactiveerd is (zie \textSymb{vpadFrontBrush} \textSymb{vpadFrontBrushK}, \atpage[vpad:menu]).
\item Druk op de toets ~\textSymb{joy_key_frontbrush_act} om de hydraulica van de derde borstel te activeren.
\item Druk op de toets~\textSymb{joy_key_frontbrush_left} of~\textSymb{joy_key_frontbrush_right} om de derde borstel in de gewenste richting te laten draaien.

\item Stel de draaisnelheid in met behulp van de toetsen~\textSymb{joy_key_frontbrush_increase} en~\textSymb{joy_key_frontbrush_decrease} van de multifunctionele console.

\item Plaats de borstel met behulp van de joystick zoals getoond in onderstaande afbeeldingen.

\stopSteps

{\md Opmerking:} {\lt Om de zijborstel te kunnen plaatsen, moet met de toets ~\textSymb{joy_key_frontbrush_act} de hydraulica van de derde borstel gedeactiveerd worden.}
\vfill

\start
\setupcombinations [width=\textwidth]

\placefig[here][fig:brush:position]{Plaatsen van de derde borstel}
{\startcombination [2*1]
{\externalfigure [work:frontBrush:move]}{Naar boven/beneden; naar links/rechts}
{\externalfigure [work:frontBrush:incline]}{Hellend}
\stopcombination}
\stop



\stopsection


%%% Änderungen Benedikt Sturny

% S. 68
\item Activeer de vastzetrem en zet de keuzehendel voor het rijniveau in \aW{neutraal}. (Vereist voor de vrijgave van de schakelaar om de container te kantelen).


% S. 93

\subsection[niveau_hydrau]{Vulniveau}

Een transparant kijkglas\index{Vulstand+Hydraulische vloeistof}\index{Onderhoud+Hydraulische installatie} maakt een controle van het hydraulische oliepeil mogelijk.
Als het hydraulische oliepeil is gedaald, moet de oorzaak worden vastgesteld voordat er weer wordt bijgevuld. Houd u aan de voorgeschreven verversingsintervallen (tabel boven) en specificaties voor de hydraulische vloeistof (tabel \at{pagina}[sec:liqquantities]).


\subsubsection{Hydraulische vloeistof bijvullen}

Vul de hydraulische vloeistof bij tot het middelste kijkglas volledig vol is.
Start de motor en vul evt. wat bij, tot de vereiste vulstand is bereikt.
