\defineparagraphs[SymVpad][n=2,distance=4mm,rule=off,before={\page[preference]},after={\nobreak\hrule\blank [2*medium]}]
\setupparagraphs [SymVpad][1][width=4em,inner=\hfill]

% \startcolumns

\subsection{Témoins d'avertissement affichés à l’écran du Vpad} % nouveau

%% ajout

\startSymVpad
\externalfigure[vpadWarningService][height=1.7\lH]
\SymVpad
\textDescrHead{Aller au garage} (jaune) L’échéance du prochain service d’entretien est atteinte
(voir \about [sec:schedule] \atpage [sec:schedule])
ou un défaut du moteur nécessitant l'intervention d'un spécialiste est enregistré.


+\:message d’erreur \# 650 à \# 653, ou \# 703
\stopSymVpad


\startSymVpad
\externalfigure[vpadTDPF][height=1.7\lH]
\SymVpad
\textDescrHead{Filtre antiparticule} (jaune) La procédure de régénération du filtre antiparticule démarrera
dès que les conditions le permettront.

{\md Note:} {\lt il est préférable de ne pas arrêter le moteur lorsque le témoin est allumé.}
\stopSymVpad



%%%%%%%%%%%%%%%%%%%%%%%%%%%%


\bTR\bTD \externalfigure [v:symbole:power] \eTD\bTD Arrêt de l'écran \eTD\bTD Une pression prolongée (env. 5\,s) provoque l'arrêt de l'écran du Vpad. \eTD\eTR
\eTD\bTD Troisième balai\index{3\high{e} balai} (option) \eTD\bTD Activer|/|désactiver les fonctions du 3\high{e} balai.
Il peut alors être activé selon la procédure décrite à la page \at[sec:using:frontBrush]. \eTD\eTR


%%%%%%%%%%%%%%%%%%%%%%%%%%%%

\startsection [title={Description des menus du Vpad}]



\subsection{Autres symboles affichés à l’écran du Vpad}


\subsubsubject{Réserve d'eau claire et de recyclage}


\subsubsubject{Système d'aspiration} % nouveau

{\em Ce symbole est affiché uniquement lorsque les balais ne sont pas activés.}

\startSymVpad
\externalfigure[sym:vpad:sucker]
\SymVpad
\textDescrHead{Bouche d'aspiration} Système\index{bouche d'aspiration} d'aspiration activé:
la bouche d'aspiration est abaissée et la turbine est activée.
\stopSymVpad


\subsubsubject{Système de balais latéraux} % nouveau

{\em Ce symbole est affiché uniquement lorsque le 3\high{e} balai n'est pas activé.}

\startSymVpad
\externalfigure[sym:vpad:sideBrush:83]
\SymVpad
\textDescrHead{Balais latéraux} Les balais\index{balayage}\index{balais latéraux} sont activés. La vitesse de rotation en \% de la vitesse
maximale [V\low{max}] est affichée au-dessous du symbole, la force de délestage
est indiquée au-dessus de chaque balai (\type{~}~= flottante, 14~= délestage maximum).

{\md Délestage:} {\lt plus la force de délestage est faible, plus la pression des balais au sol est élevée.}
\stopSymVpad


\startSymVpad
\externalfigure[sym:vpad:sideBrush:float:60]
\SymVpad
\textDescrHead{Position flottante} (bande inférieure verte)
Une pression prolongée (env. 2\,s) du joystick vers l'avant supprime le délestage
du balai; le balai repose sur le sol de tout son poids. La vitesse de rotation des balai est à 60\,\% de V\low{max}
(exemple).
\stopSymVpad

\startSymVpad
\externalfigure[sym:vpad:sideBrush]
\SymVpad
\textDescrHead{Balais latéraux} Les balais sont activés. La rotation est à l'arrêt et les balais sont en position haute.
\stopSymVpad


\subsubsubject{Troisième Balai (option)} % nouveau

\startSymVpad
\externalfigure[sym:vpad:frontBrush]
\SymVpad
\textDescrHead{Troisième balai} Le 3\high{e} balai\index{3\high{e} balai} est activé. La vitesse de rotation en \% de la vitesse
maximale [V\low{max}] est affichée au-dessous du symbole.
\stopSymVpad


\startSymVpad
\externalfigure[sym:vpad:frontBrush:left]
\SymVpad
\textDescrHead{Position flottante} (bande inférieure verte)
Une pression prolongée (env. 2\,s.) du joystick vers l'avant supprime le délestage
du balai; le balai repose sur le sol de tout son poids. La vitesse de rotation du balai est à 70\,\% de V\low{max}
(exemple).

{\md Sens de rotation:} {\lt la bande supérieure indique le sens de rotation actif
(flèche noire sur rectangle jaune).}
\stopSymVpad

\stopsection




\setups [pagestyle:marginless]


\startsection [title={Travailler avec le troisième balai (option)},
							reference={sec:using:frontBrush},
							]

\startSteps
\item Procédez\index{balayage} à la mise en service du véhicule selon la procédure décrite \in{§}[sec:using:start], \atpage[sec:using:start].
\item Activez\index{3\high{e} balai} le mode \quote{travail}
(pressez sur le bouton à l’extrémité du sélecteur de marche).
\stopSteps

% \getbuffer [work:config]

\startSteps [continue]
\item Vérifiez que le 3\high{e} balai soit activé à l'écran du Vpad
(voir \textSymb{vpadFrontBrush} \textSymb{vpadFrontBrushK},\atpage[vpad:menu]).
\item Pressez la touche \textSymb{joy_key_frontbrush_act} pour actionner le système hydraulique du 3\high{e} balai.
\item Pressez la touche \textSymb{joy_key_frontbrush_left} ou \textSymb{joy_key_frontbrush_right} pour actionner la rotation du 3\high{e} balai, selon le sens de rotation souhaité.

\item Ajustez la vitesse de rotation du balai au moyen des touches \textSymb{joy_key_frontbrush_increase}
et \textSymb{joy_key_frontbrush_decrease} de la console de médiane.

\item Positionnez le balai au moyen des joysticks selon l'illustration ci||dessous.

\stopSteps

{\md Remarque:} {\lt pour déplacer les balais latéraux, pressez la touche \textSymb{joy_key_frontbrush_act} pour désactiver le système hydraulique du 3\high{e} balai.}
\vfill

\start
\setupcombinations [width=\textwidth]

\placefig[here][fig:brush:position]{Positionnement du troisième balai}
{\startcombination [2*1]
{\externalfigure [work:frontBrush:move]}{Déplacez le balai haut|/|bas; gauche|/|droite}
{\externalfigure [work:frontBrush:incline]}{Inclinez le balai sur l'axe transversal|/|longitudinal}
\stopcombination}
\stop



\stopsection

