\defineparagraphs[SymVpad][n=2,distance=4mm,rule=off,before={\page[preference]},after={\nobreak\hrule\blank [2*medium]}]
\setupparagraphs [SymVpad][1][width=4em,inner=\hfill]

% \startcolumns

\subsection{Kontrollanzeigen auf dem Vpad-Bildschirm} % nouveau

%% ajout

\startSymVpad
\externalfigure[vpadWarningService][height=1.7\lH]
\SymVpad
\textDescrHead{Werkstatt aufsuchen}(gelb) Reguläre Fahrzeugwartung fällig
(siehe \about [sec:schedule] \atpage [sec:schedule])
oder ein Motorfehler wurde registriert (Fachwerkstatt erforderlich).

+\:Fehlermeldungen \# 650 bis \# 653, oder \# 703
\stopSymVpad


\startSymVpad
\externalfigure[vpadTDPF][height=1.7\lH]
\SymVpad
\textDescrHead{Partikelfilter}(gelb) Regenerierung des Partikelfilters wird gestartet, sobald es der Betriebzustand erlaubt.

{\md Hinweis:} {\lt Stellen Sie wenn möglich den Motor {\em nicht} ab, solange diese Anzeige aufleuchtet!}
\stopSymVpad



%%%%%%%%%%%%%%%%%%%%%%%%%%%%


\bTR\bTD \externalfigure [v:symbole:power] \eTD\bTD Bildschirm ausschalten \eTD\bTD Etwa 5\,s gedrückt halten, um den Bildschirm des Vpad auszuschalten. \eTD\eTR
\bTR\bTD \framed[frame=off]{\externalfigure [v:symbole:frontBrush]\externalfigure [v:symbole:frontBrush:black]}
\eTD\bTD Dritter Besen\index{3.\,Besen} (Option) \eTD\bTD Dritten Besen freischalten.
Der dritte Besen kann nun aktiviert werden wie auf Seite~\at[sec:using:frontBrush] beschrieben. \eTD\eTR


%%%%%%%%%%%%%%%%%%%%%%%%%%%% corriger

\startsection [title={Die Menüs des Vpad},
				reference={vpad:menu}]



\subsection{Weitere Symbole auf dem Vpad-Bildschirm}


\subsubsubject{Frischwasser- und Recyclingwasservorrat}


\subsubsubject{Saugsystem} % nouveau

{\em Dieses Symbol wird nur angezeigt, wenn die Besen deaktiviert sind.}

\startSymVpad
\externalfigure[sym:vpad:sucker]
\SymVpad
\textDescrHead{Saugmund} Saugsystem\index{Saugmund} aktiviert:
Saugmund ist abgesenkt und Turbine ist aktiviert.
\stopSymVpad


\subsubsubject{Seitenbesen} % nouveau

{\em Dieses Symbol wird nur angezeigt, wenn der dritte Besen nicht aktiviert ist.}

\startSymVpad
\externalfigure[sym:vpad:sideBrush:83]
\SymVpad
\textDescrHead{Seitenbesen} Besen\index{Kehren}\index{Seitenbesen} aktiviert. Die Rotationsgeschwindigkeit (in \% der max. Rotatationsgeschwindigkeit [V\low{max}]) wird unterhalb des Symbols angezeigt, die aktuelle Entlastung des jeweiligen Besens wird oberhalb  des Symbols angezeigt (\type{~}~= Schwimmstellung, 14~= Maximale Entlastung).

{\md Entlastung:} {\lt Je niedriger die Entlastung, desto höher der Druck der Besen auf den Boden.}
\stopSymVpad


\startSymVpad
\externalfigure[sym:vpad:sideBrush:float:60]
\SymVpad
\textDescrHead{Schwimmstellung}(grün am unteren Rand)
Um die Entlastung auszuschalten, halten Sie den Joystick für etwa 2\,s nach vorn gedrückt; der Besen liegt nun mit seinem ganzen Eigengewicht auf dem Boden auf. Die Rotationsgeschwindigkeit der Besen ist bei 60\hairspace\% der V\low{max} (Beispiel).
\stopSymVpad

\startSymVpad
\externalfigure[sym:vpad:sideBrush]
\SymVpad
\textDescrHead{Seitenbesen} Die Besen sind aktiviert. Sie stehen still und sind angehoben.
\stopSymVpad


\subsubsubject{Dritter Besen (Option)} % nouveau

\startSymVpad
\externalfigure[sym:vpad:frontBrush]
\SymVpad
\textDescrHead{Dritter Besen} Der dritte Besen\index{3.\,Besen} ist aktiviert. Die Rotatationsgeschwindigkeit (in \% der max. Rotatationsgeschwindigkeit [V\low{max}]) wird unterhalb des Symbols angezeigt.
\stopSymVpad


\startSymVpad
\externalfigure[sym:vpad:frontBrush:left]
\SymVpad
\textDescrHead{Schwimmstellung}(grün am unteren Rand)
Um die Entlastung auszuschalten, halten Sie den Joystick für etwa 2\,s nach vorn gedrückt; der Besen liegt nun mit seinem ganzen Eigengewicht auf dem Boden auf. Die Rotationsgeschwindigkeit der Besen ist bei 70\hairspace\% der V\low{max} (Beispiel).

{\md Rotationsrichtung:} {\lt Am oberen Rand wird die Rotationsrichtung angezeigt (schwarzer Pfeil auf gelbem Hintergrund).}
\stopSymVpad

\stopsection




\setups [pagestyle:marginless]


\startsection [title={Mit dem dritten Besen arbeiten (Option)},
							reference={sec:using:frontBrush},
							]

\startSteps
\item Nehmen\index{Kehren} Sie das Fahrzeug in Betrieb, so wie in \in{Abschnitt}[sec:using:start] \atpage[sec:using:start] beschrieben.
\item Aktivieren\index{3.\,Besen} Sie den\aW{Arbeits}modus (Knopf außen am Fahrstufenwahlhebel).
\stopSteps

% \getbuffer [work:config]

\startSteps [continue]
\item Stellen Sie sicher, dass der dritte Besen auf dem Vpad||Bildschirm aktiviert ist
(siehe \textSymb{vpadFrontBrush} \textSymb{vpadFrontBrushK}, \atpage[vpad:menu]).
\item Drücken Sie die Taste~\textSymb{joy_key_frontbrush_act}, um die Hydraulik des dritten Besens zu betätigen.
\item Drücken Sie die Taste~\textSymb{joy_key_frontbrush_left} oder~\textSymb{joy_key_frontbrush_right}, um den dritten Besen in der gewünschten Richtung rotierne zu lassen.

\item Stellen Sie die Rotationsgeschwindigkeit mithilfe der Tasten~\textSymb{joy_key_frontbrush_increase} und~\textSymb{joy_key_frontbrush_decrease} der Multifunktionskonsole ein.

\item Positionieren Sie den Besen mithilfe der Joysticks, so wie in den Abbildungen unten gezeigt.

\stopSteps

{\md Hinweis:} {\lt Um die Seitenbesen positionieren zu können, muss mittels der Taste~\textSymb{joy_key_frontbrush_act} die Hydraulik des dritten Besens deaktiviert werden.}
\vfill

\start
\setupcombinations [width=\textwidth]

\placefig[here][fig:brush:position]{Positionieren des dritten Besens}
{\startcombination [2*1]
{\externalfigure [work:frontBrush:move]}{Nach oben|/|unten; nach links|/|rechs}
{\externalfigure [work:frontBrush:incline]}{Quer-|/|Längsneigen}
\stopcombination}
\stop



\stopsection


%%% Änderungen Benedikt Sturny

% S. 68
\item Aktivieren Sie die Feststellbremse und legen Sie den Fahrstufenwahlhebel auf \aW{Neutral}. (Erforderlich zur Freigabe des Behälter||Kippen||Schalters).


% S. 93

\subsection[niveau_hydrau]{Füllstand}

Ein transparentes
Schauglas\index{Füllstand+Hydraulikflüssigkeit}\index{Wartung+Hydraulikanlage}
ermöglicht eine Prüfung des Hydraulikölstands.
Wenn der Hydraulikölstand gesunken ist, muss die Ursache
ermittelt werden, bevor wieder aufgefüllt wird. Halten Sie sich an die
vorgeschriebenen Wechselintervalle (Tabelle oben) und Spezifikationen für die
Hydraulikflüssigkeit (Tabelle \at{Seite}[sec:liqquantities]).


\subsubsection{Hydraulikflüssigkeit nachfüllen}

Füllen Sie Hydraulikflüssigkeit nach, bis das mittlere Schauglas voll gedeckt ist.
Starten Sie den Motor und füllen Sie ggf. etwas nach, bis der erforderliche
Füllstand erreicht ist.
