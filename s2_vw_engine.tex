\startsection [title={Contrôle quotidien},
							reference={sec:main:daily},
							]


Vérifiez le niveau d’huile du moteur avec la jauge à huile (portillon de service arrière droit}.


Travaux supplémentaires:

Purgez le filtre d’alimentation en carburant.


\stopsection

\startsection [title={Service régulier toutes les 600\,h},
							reference={sec:main:600},
							]


Diagnostic embarqué: 

– Interrogez la mémoire de défauts de tous les systèmes de surveillance du moteur.
– Effectuez une inspection visuelle des fuites et dommages éventuels dans le compartiment moteur.



Système de graissage:
– Vidangez|/|remplir l’huile du moteur
– Remplacez le filtre à huile

Système de refroidissement:
– Contrôlez le niveau du liquide de refroidissement
– Contrôlez le point de congélation du liquide de refroidissement à l’aide d’un réfractomètre (-25\,°C)

Courroie poly-V:
– Contrôlez l’état et la tension de la courroie, remplacer si nécessaire

Périphérique:
– Contrôlez|/|nettoyez le radiateur de refroidissement
– Contrôlez l’état et le fonctionnement du système de ventilation (enclenchement|/|déclenchement)
– Remplacez le filtre à air
– Contrôlez|/|réparez l’étanchéité et la fixation du système d’admission d’air
– Contrôlez|/|réparez l’étanchéité et la fixation du système d’échappement
– Contrôlez|/|réparez la fixation des éléments périphériques du moteur (faisceaux de câbles, conduites, etc.)

Travaux supplémentaires

Alimentation en carburant:
– Remplacez le filtre d’alimentation en carburant

Filtre antiparticule:
– Régénérez le filtre antiparticule

%%%%%%%%%%%%


\stopsection

\startsection [title={Service régulier toutes les 2400\,h},
							reference={sec:main:2400},
							]


Travaux supplémentaires

Distribution:
– Remplacez la courroie de distribution

%%%%%%%%%%%%


\stopsection

\startsection [title={Service régulier toutes les 4000\,h},
							reference={sec:main:4000},
							]


Travaux supplémentaires

Refroidissement:
– Remplacez la pompe à eau (en même temps que le remplacement de la courroie de distribution)







%%%%%%%%%%%%%%%%%%%%%%%%%%%%%%%%%%%%%%%%%%%%%%%%
%%%%%%%%%%%%%%%%%%%%%%%%%%%%%%%%%%%%%%%%%%%%%%%%


\startsection [title={Entretien du moteur diesel},
							reference={sec:workshop:vw},
							]


\subsection [sSec:vw:diagTool]{Système de diagnostique embarqué}

L’unité de commande du moteur~– J623~– est équipé d’une mémoire de défauts.
Les défauts détectés sur un élément surveillé sont enregistrés dans la mémoire de défauts,
notifiés et classés par type.

Les défauts sont évalués, puis triés et
enregistrés dans la mémoire jusqu’à ce qu’ils soient effacés.

Les défauts sporadiques sont affichés avec le suffixe \quote{SP}.
Les défauts sporadiques peuvent être causés par un mauvais contact ou
un débranchement temporaire d’un câble électrique.
Si un défaut ne se reproduit plus durant 50 démarrages, il est supprimé
de la mémoire de défauts.

Si un défaut détecté peu influencer le fonctionnement du moteur,
la lampe témoin de diagnostic du moteur~– K29~– et|/|ou la lampe témoin du système
de dépollution des gaz d’échappement s’allume\,(nt).

Le contenu de la mémoire de défauts peut être lu avec l’outil de diagnostic, de test
et d’information système~– VAS 5051/B~–.

Après avoir réparé un ou plusieurs défauts, il faut effacer la
mémoire de défauts.

\subsubsection[sSec:vw:diagTool:connect]{Branchez le système de diagnostic}

\starttextbackground [FC]
\startPictPar
\PMgeneric
\PictPar
Vous trouverez les instructions détaillées à propos du système de
diagnostic~– VAS 5051/B~– dans le manuel fourni avec l’appareil.

Il est également possible d’utiliser un appareil alternatif, par ex. DiagRA.
\stopPictPar
\stoptextbackground

\page [yes]


\subsubsubsubject{Prérequis}

\startitemize
\item Les fusibles doivent être en ordre
\item La tension de la batterie doit être supérieure à 11,5\,V
\item Tous les consommateurs électriques doivent être désactivés
\item Le branchement à la masse doit être en ordre
\stopitemize


\subsubsubsubject{Procédure}

\startSteps
\item Branchez la prise de diagnostic de l’appareil~– VAS 5051/B~– à l’interface de diagnostic du véhicule.
\item Mettez le contact ou, selon la fonction recherchée, démarrez le moteur.
\stopSteps

\subsubsubsubject{Sélection du mode opératoire}

\startSteps [continue]
\item Pressez le bouton \quote{Vehicle self-diagnosis} à l’écran.
\stopSteps


\subsubsubsubject{Sélection du mode système}

\startSteps [continue]
\item Pressez le bouton \quote{01-Engine electronics} à l’écran.
\stopSteps

L’écran affiche le code et l’identification de l’unité de commande du moteur.

Le cas échéant, vérifiez le code de l’unité de commande du moteur.


\subsubsubsubject{Sélection de la fonction de diagnostic}

Toutes les fonctions de diagnostic valables sont affichées à l’écran.

\startSteps [continue]
\item Pressez le bouton pour la fonction désirée à l’écran.
\stopSteps



\subsection [sSec:vw:faultMemory]{Mémoire de défauts}


\subsubsection{Lecture de la mémoire de défauts}

\subsubsubject{Procédure d’interrogation de la mémoire de défauts}

\startSteps
\item Mettez le moteur en marche, régime de ralenti.
\item Branchez le système de diagnostique (voir \in{section}[sSec:vw:diagTool:connect])
et sélectionnez \quote{engine control unit}.
\item Sélectionnez la fonction de diagnostic \quote{004-Contents of fault memory}.
\item Sélectionnez la fonction de diagnostic \quote{004.01-Read fault memory}.
\stopSteps

{\sla Uniquement si le moteur ne démarre pas:}

\startitemize [2]
\item Mettez le contact (contacteur à clé).
\item Si aucun défaut n’est enregistré dans l’unité de commande du moteur,
l’écran affiche \quote{0 fault detected}.
\item Si un ou plusieurs défauts sont enregistrés dans l’unité de commande du moteur,
ils sont affichés successivement à l’écran.
\item Quittez la fonction de diagnostic.
\item Coupez le contact.
\item Réparez chaque défaut potentiel en vous aidant de la table des défauts (voir documentation d’atelier),
effacez la mémoire de défauts.
\stopitemize

\starttextbackground [FC]
\startPictPar
\PMrtfm
\PictPar
Si un ou plusieurs défauts ne peuvent pas être effacés, faites appel au service après||vente \boschung.
\stopPictPar
\stoptextbackground


\subsubsubject{Défauts statiques}

En cas de présence d’un ou plusieurs défauts statiques dans la mémoire de défauts,
nous recommandons d’effectuer les réparations nécessaires, en accord avec le client,
en vous aidant du guide de recherche de défauts.



\subsubsubject{Défauts sporadiques}

Dans le cas où seuls des défauts sporadiques sont présents dans la mémoire de défauts,
et qu’aucun dysfonctionnement en relation avec le système électronique n’a été signalé par le client,
effacez la mémoire de défauts.

\startSteps [continue]
\item Pressez à nouveau sur \quote{continue} \inframed{>} pour poursuivre le test.
\item Pour quitter le guide de recherche de défauts, pressez sur \quote{skip}, puis \quote{close}.
\stopSteps

Maintenant toutes la mémoire de défauts est à nouveau interrogée.

Un message est maintenant affiché pour confirmer que tous les défauts sporadiques ont été effacés.
Le protocole de diagnostic est automatiquement envoyé {\em online}.

Le test du système électronique du moteur est complet.


\subsubsection{Effacement de la mémoire de défauts}


\subsubsubject{Procédure d’effacement de la mémoire de défauts}

{\sla Prérequis:}

\startitemize [2]
\item Les réparations nécessaires doivent avoir été effectuées.
\stopitemize

{\sla Procédure:}

\starttextbackground [FC]
\startPictPar
\PMrtfm
\PictPar
Lorsque les défauts ont été corrigés, la mémoire de défauts doit être à nouveau interrogée
et ensuite effacée selon la procédure ci||dessous.
\stopPictPar
\stoptextbackground


\startSteps
\item Mettez le moteur en marche, régime de ralenti.
\item Branchez le système de diagnostique (voir \in{section}[sSec:vw:diagTool:connect])
et sélectionnez \quote{engine control unit}.
\item Sélectionnez la fonction de diagnostic \quote{004-Contents of fault memory}.
\item Sélectionnez la fonction de diagnostic \quote{004.10-Erase fault memory}.
\stopSteps


\starttextbackground [FC]
\startPictPar
\PMrtfm
\PictPar
Si un défaut ne peut pas être effacé, cela signifie que la faute n’est pas corrigée.
\stopPictPar
\stoptextbackground

\startSteps [continue]
\item Quittez la fonction de diagnostic.
\item Coupez le contact.
\stopSteps


\subsection [sSec:vw:lub] {Graissage du moteur diesel}

\subsubsection [ssSec:vw:oilLevel] {Vérification du niveau d’huile du moteur}

\starttextbackground [FC]
\startPictPar
\PMrtfm
\PictPar
Le niveau d’huile du moteur doit impérativement rester en||dessous de la marque {\CAP max.} de la jauge à huile.
Un niveau d’huile trop élevé risque d’endommager le convertisseur catalytique.
\stopPictPar
\stoptextbackground

\startSteps
\item Arrêtez le moteur et laissez||le au repos environ trois minutes.
\item Retirez la jauge à huile de son orifice
et essuyez||la proprement; introduisez||la à nouveau complètement.
\item Retirez||la à nouveau et observez le niveau d’huile comme suit:
\startitemize [A]
\item Niveau maximum, aucun remplissage nécessaire.
\item Niveau correct, un ajout jusqu’à atteindre la marque A est autorisé.
\item Le niveau est insuffisant, un remplissage est nécessaire jusqu’à atteindre
au moins la zone B.
\stopitemize
\item Si le niveau est au||dessous de la marque C, il est nécessaire de remplir
jusqu’à atteindre la marque A.
\stopSteps





\stopsection

