% coding: utf-8

\startcomponent c_30_overview_s2_095-nb

\startchapter [title={Oversikt over kjøretøyet}]

\setups [pagestyle:marginless]


\placefig [here] [] {Oversikt over venstre side av kjøretøyet}
{\externalfigure [overview:side:left:nb]}


\page [yes]


\placefig [here] [] {Oversikt over høyre side av kjøretøyet}
{\externalfigure [overview:side:right:nb]}

\page [yes]

\setups [pagestyle:normal]


\startsection [title={Generelt}]

\placefig[margin][p4_vue_01]{\sdeux\ ved overlevering}
{%
\startcombination [1*3]
{\externalfigure[overview:vhc:01]}{}
{\externalfigure[overview:vhc:02]}{}
{\externalfigure[overview:vhc:03]}{}
\stopcombination}

Med feiemaskinen \BosFull{sdeux} gir Boschung hele erfaringen og kompetansen videre, som de har samlet i flere tiår gjennom kontinuerlig samarbeid med deres kunder og samarbeidspartnere.
Kravene til kommunene og tjenesteyterne har med hensyn til mobilitet og mangfoldighet har økt enormt i løpet av denne perioden. Utviklerne av \sdeux\ har stilt seg denne utfordringen, ledet av behovene til kundene, og som har vært påkrevd av de fremtidsrettede forslagene for forbedring til Boschung-kundeservicen.
Av denne syntesen med kundeorientering og konsekvent anvendelse av den oppnådde praksiserfaringen har \sdeux\ blitt utviklet.


\subsection{Innovativ teknologi}

Den kompakte feiemaskinen \BosFull{sdeux} utmerker seg i sin klasse spesielt gjennom den lave vekten (2300\,kg), den store kapasiteten (smussbeholder av 2,0-m\high{3}-klassen), de kompakte målene (bredde 1,15\,m) og den særdeles ergonomiske arbeidsplassen for føreren av kjøretøyet.

Gjennom den smale konstruksjonen blir \sdeux\ til \quotation{universal}-feiemaskinen for gater og fortau i byer og landsbyer. Den kraftige dieselmotoren i kombinasjon med den kompakte hydrostatiske fremdriften (radialstempel-hydromotorer på forhjulene) sørger den til enhver tid for størst mulig mobilitet, uavhengig av forholdene på bruksstedet eller fyllenivået til smussbeholderen.

Hydraulikkpumpene drives av en dieselmotor av type \aW{VW 2.0 CDI} iht. Euro-V-standard. Den gir et dreiemoment på 285\,Nm ved 1750~omdreininger og en maksimal ytelse på 75\,kW ved 3000~omdreininger. Dette gjør at maskinen allerede ved lavt motorturtall~- og dermed mindre støy~– kan brukes effektivt. Som \sdeux\ standard har den et partikkelfilter.

\stopsection


\startsection [title={Innovasjoner som service for kundene}]

Leddstyringen til \sdeux\ sørger for en liten vendekrets og dermed for maksimal bevegelighet. Spesielle materialer som Domex® og utviklingen av kjøretøyet som er komplett basert på CAD, tillater en betydelig nyttelast på 1200\,kg.

\placefig[margin][overview:cab:frontright]{\sdeux\ driftsklar}
{\externalfigure[overview:cab:twoleft][width=\Bildwidth]}

Førerhuset som har glassvinduer på alle sider har to komfortable sitteplasser, utstyrt med tre-punkt-sikkerhetsbelter. \sdeux\ kan valgfritt utstyres med et klimaanlegg.

Med en topphastighet på 40\,km/h kan kjøretøyet uten problemer kjøre i bytrafikken. Gjennom den komfortable for- og bakakselfjæringen kan en til og med kjøre trygt og komfortabelt på svært dårlige strekninger.

Feieaggregatet~– montert på to leddarmer~– befinner seg fullstendig i synsfeltet til operatøren og sugeenheten er plassert godt synlig foran forakselen. Den dobbel svingbar frontkost er tilgjengelig som tilleggsutstyr.

\page [yes]


\subsection{Lyddempet og komfortabelt førerhus}

Førerhuset\index{Førerhus} til \sdeux\ styres fra høyre side og er konsipert for to personer. Den er lydisolert og montert på vibrasjonsdempende silentblocker.

Dører og gulv er av glass, noe som resulterer i et oversiktlig synsfelt. Frontruten strekker seg over hele forsiden til kjøretøyet og gir uhindret på arbeidet til kostene.

Førersetet har en mekanisk eller~– valgfritt~– pneumatisk fjæring. Fører- og passasjersetet er montert på justerbare glideskinner.


\subsubsubject{Ergonomi}

\startfigtext[right][overview:joy:sideview]{Kontrollpanel}
{\externalfigure[overview:joy:top]}
Multifunksjonspanelet, på venstre side av førersetet, gjør at en kan nå alle viktige funksjoner med en hånd. De to kostene kan styres uavhengig av hverandre med to joysticks med tommelen og pekefingeren. Bryteren for kostene og frontkosten (valgfritt), for motorturtallet, fartsholderen osv. befinner seg også på multifunksjonspanelet.
\stopfigtext

På den nedre kanten til synsfeltet til føreren av kjøretøyet er det en berøringsskjerm, som i sanntid viser alle viktige informasjoner for maskinens funksjoner, uten at sikten utover svekkes.

\placefig[margin][overview:vhc:left]{\sdeux\ foran historiske murer}
% \placefig[margin][overview:vhc:left]{\sdeux\ sur site historique}
{\externalfigure[overview:vhc:left]}

\page [yes]


\subsubsubject{Førerplass}

\index{Førerplass} Kjøretrinnvalgspaken(\quotation{girskifte}) befinner seg til høyre på rattsøylen. Det finnes to forover- og ett bakoverkjøretrinn til disposisjon. På utsiden av kjøretrinnvalgspaken er trykknappen til omkobling mellom de to kjøremodusene \aW{Arbeid} og \aW{Kjøring}. \sdeux\ trenger ikke stoppes for å koble om. (Mer om dette i kapittel \about[sec:using:work], \atpage[sec:using:work].)

\placefig[margin][fig:overview:steeringwheel]{Førerplass}
{\externalfigure[overview:driver:place]}

Ved rygging slås monitoren til ryggekameraet på og det lyder et akustisk varselsignal (kan deaktiveres via Vpad).

Multifunksjonsspaken på venstre side av rattsøylen inkluderer bryteren til vindusviskeren (to trinn og intervall) samt lys- og akustisk horn.

I kapittel \about[chap:using] fra \atpage[chap:using] finner du informasjoner om denne og ytterligere funksjoner \sdeux.

\page [yes]

\setups[pagestyle:marginless]


\subsection[overview:brushsystem]{Feie- og sugeenhet}

\subsubsubject{Kost}

\startfigtext[left][fig:overview:steeringwheel]{Feie-/sugeenhet}
{\externalfigure[system:brush]}
Kostene\index{Feie} er plassert på justerbare hoder, som igjen er montert på leddarmer. Støvet som virvles opp ved feiing bindes ved at det sprøytes på vann: Hver kost er utstyrt med en dyse, som henter vannet fra ferskvanns- eller resirkuleringsvanntanken.

En bryter\index{Suge} på multifunksjonspanelet aktiverer samtidig kosten og vannpumpen.\footnote{For vannpumpen se kapittel \in[chap:using] \about[chap:using], spesielt \about[sec:using:work], \atpage[sec:using:work].}
Posisjonene til kostene og deres tverr- og langsgående helling kan styres direkte med den tilsvarende joysticken på multifunksjonspanelet.
\stopfigtext

Kostene er beskyttet av et mekanisk og hydraulisk antikollisjonssystem.


\subsubsubject{Sugeenhet}

I arbeidsposisjon (senket) ligger sugeenheten på 4~ruller og dekker hele flaten mellom kostene som er kjørt fra hverandre. Gjennom den \quotation{slepte} posisjonen er den ved kollisjoner med hinder for det meste beskyttet mot mekaniske skader. Ved rygging løftes sugeenheten automatisk.

En tykk, gummileppe som kan skiftes ut sørger for tett ende mot gatens overflate. En elektro-hydraulisk styrbar luke på forsiden av sugeenheten tillater opptak av grovere smuss.


\subsubsubject{Smussbeholder}

Aluminiums-smussbeholderen kan vippes opp til 55° og på en høyde på 1,5\,m (utløpshøyde). I den ender sugekanalen som kommer nedenfra med en åpningsdiameter på 180\,mm.

Undertrykket ved suging opprettes av en høyytelses turbin, som er montert horisontalt i smussbeholderen. Den har en vedlikeholdsluke for rengjøring og visuell kontroll.

I lukkeren til smussbeholderen finnes de to gitter for luftinntak av rustfritt stål. For rengjøring kan de vippes ut uten bruk av verktøy. Lukkeren kan låses opp og åpnes for hånd.

Ved hjelp av en luke, som kan snus med hånden, kan en lett koble om luftstrømmen mellom sugekanalen og håndsugeslangen (tilleggsutstyr).


\subsection{Luftfuktingssystem}

\subsubsubject{Ferskvannssystem}

\index{Feie+fukting} Tanken av ABS-støpegods befinner seg i stående posisjon bak førerhuset. Kapasiteten\index{Ferskvanns+-tank} er på 190\,l.

En elektrisk pumpe (10\,l/min) transporterer vannet til sprøytedysene over de enkelte kostene (i tillegg til den valgfrie tredje kosten).


\subsubsubject{Resirkulering av spillvann}

Spillvannet strømmer gjennom mikro-perforeringer til de indre veggene til spillvannsbeholderen og renner deretter gjennom resirkuleringsluken ned i tanken for resirkulert vann. \index{Tank for resirkulert +vann} Tanken for resirkulert vann fatter 140\,l.

En elektrisk dykkepumpe (10\,l/min) transporterer vannet til sprøytedysene på innsiden av sugeenheten og sugekanalen.


\testpage [8]
\subsubsubject{Tank for resirkulert vann}

Tanken for resirkulert vann har en vann-hydraulikkvæske-varmeveksler med to funksjoner:

\startitemize[width=42mm,style=\md, command={\setupwhitespace[small]}]
\sym{Funksjon om sommeren} Vannet leder varmen til hydraulikkvæsken gjennom konveksjon til aluminiumsveggene til tanken, hvorfra det avgis til omgivelsesluften.

\sym{Funksjon om vinteren} Hydraulikkvæsken varmer opp vannet i tanken. Dette gjør at det er mulig å sprøyte på sugekanalen og sugeenheten selv ved temperaturer som er litt lavere enn frysepunktet.
\stopitemize


\subsubsubject{Overvåking av vann fyllenivåer}

\startitemize[width=42mm,style=\md, command={\setupwhitespace[small]}]
\sym{Ferskvann} Hvis fyllenivået ikke er tilstrekkelig vises symbolet~\textSymb{vpad_water} på Vpad-skjermen.
\sym{Resirkulert vann} Hvis fyllenivået til resirkuleringstanken befinner seg under varmeveksleren (se oppe), vises symbolet~\textSymb{vpad_rwater_orange} (gult) på Vpad-skjermen. Hvis fyllenivået ikke er tilstrekkelig vises symbolet~\textSymb{vpad_rwater} (rødt).
\stopitemize

\stopsection

\page [yes]

\setups[pagestyle:normal]


\startsection [title={Identifisering av kjøretøyet}]

\subsection{Kjøretøytypeskilt}

Kjøretøytypeskiltet\index{Identifikasjon+kjøretøy} befinner seg i førerhuset, overfor panelet, under passasjersetet (se \inF[fig:identity:location], \atpage[fig:identity:location]).


\subsection{Motorkode og -nummer}

Motorkoden befinner seg på typeskiltet til motoren (klistremerke), på den forkrummede metalledningen til kjølekretsen foran på motoren (løft smussbeholderen).

Motornummeret er stanset inn på motoren (\inF[identity:engine:number]). Den består av ni alfanumeriske tegn: De første tre er motorkoden, de seks påfølgende serienummeret til motoren.


\placefig[margin][idvhc]{Kjøretøytypeskilt}
{\externalfigure[s2:id:plaque]}

\placefig[margin][identity:engine:code]{Motortypeskilt}
{\externalfigure[engine:id:code]}

\placefig[margin][identity:engine:number]{Motornummer}
{\externalfigure[engine:id:number]}

\page [yes]


\subsection [sec:plateWheel]{Hjultypeskilt}

Typeskiltet til felger og dekk\index{Dekk+fylletrykk} befinner seg i førerhuset\index{Felger+dimensjoner}, under passasjersetet.


\subsection{Understellnummer}

Understellnummeret\index{Identifikasjon+understellnummer} (chassisnummer) er stanset inn på høyre side av kjøretøyet, under førerhuset, på understellet.


\subsection{\symbol[europe][CEsign]-samsvar og -merking}

~\symbol[europe][CEsign]-samsvarsmerkingen befinner seg i førerhuset, overfor panelet, under passasjersetet.

\sdeux\ oppfyller de grunnleggende sikkerhets- og helsekravene til maskindirektivet\index{Sertifikat+CE-samsvar}\index{maskindirektiv} 2006/42/EU\footnote{direktiv 2006/42/EU til Europaparlamentet og rådet av 17.~mai 2006}.
% \textrule

\placefig[margin][idpneus]{Dekktrykk}
{\externalfigure[identity:tires]}

\placefig[margin][fig:identity:location]{Typeskilt}
{\externalfigure[identity:location]}

\stopsection

\page [yes]

\setups [pagestyle:marginless]


\startsection[title={Tekniske data},
reference={donnees_techniques}]

\subsection [sec:measurement] {Kjøretøyets dimensjoner}

\placefig[here][fig:measurement]{\select{caption}{Bredde~– kost i hvilestilling eller kjørt ut~–, lengde og høyde til kjøretøyet}{Kjøretøyets dimensjoner}}
{\Framed{\externalfigure[s2:measurement]}}

\page [yes]

\placefig[here][fig:measurement]{\select{caption}{Høyden til kjøretøyet med smussbeholder vippet opp}{Høyde til kjøretøyet}}
{\Framed{\externalfigure[s2:measurement:02]}}

\page [yes]

\starttabulate [|lBw(45mm)|p|l|rw(35mm)|]
\FL
\NC Gruppe\index{Dimensjoner} \NC \bf Dimensjoner \NC \bf Enhet \NC \bf Verdi \NC\NR
\ML
\NC Kjøretøyets dimensjoner \NC lengde (over alt) \NC \unite{mm} \NC 4588,00 \NC\NR
\NC\NC lengde med 3.\,kost \NC \unite{mm} \NC 5020,00 \NC\NR
\NC\NC bredde til kjøretøyet \NC \unite{mm} \NC 1150,00 \NC\NR
\NC\NC bredde til kjøretøyet (over alt) \NC \unite{mm} \NC 1575,00 \NC\NR
\NC\NC høyde uten roterende varsellys \NC \unite{mm} \NC 1990,00 \NC\NR
\NC\NC hjulavstand \NC \unite{mm} \NC 1740,00 \NC\NR
\NC\NC sporbredde \NC \unite{mm} \NC 894,00 \NC\NR
\ML
\NC Feiebredde \NC standardkost \NC \unite{mm} \NC 2300,00 \NC\NR
\NC\NC med 3.\,kost \NC \unite{mm} \NC 2600,00 \NC\NR
\NC\NC diameter kost \NC \unite{mm} \NC 800,00 \NC\NR
\NC\NC bredde sugeenhet \NC \unite{mm} \NC 800,00 \NC\NR
\ML
\NC Lastfordeling \NC tomvekt\note[weight:empty] foraksel \NC \unite{kg} \NC ca. 1100,00 \NC\NR
\NC\NC tomvekt\note[weight:empty] bakaksel \NC \unite{kg} \NC ca. 1200,00 \NC\NR
\NC\NC tomvekt\note[weight:empty] \NC \unite{kg} \NC ca. 2300,00 \NC\NR
\NC\NC till. totalvekt \NC \unite{kg} \NC 3500,00 \NC\NR
\LL
\stoptabulate


\subsection{Sporradius og feieradius}

\starttabulate [|lBw(45mm)|p|l|rw(35mm)|]
\FL
\NC Dimensjon\index{Dimensjoner} \NC \bf Mål \NC \bf Enhet \NC \bf Verdi \NC\NR
\ML
\NC Sporradius\index{Sporradius}\index{Mål+Sporradius} \NC minimal venderadius med kost \NC \unite{mm} \NC 3325,00 \NC\NR
\ML
\NC Feieradius \NC utover \NC \unite{mm} \NC 3425,00 til 3850,00 \NC\NR
\NC\NC innover \NC \unite{mm} \NC 2025,00 til 1675,00 \NC\NR
\LL
\stoptabulate

%% TODO en/de/fr: Footnote on preceeding page
\footnotetext[weight:empty]{Standardkonfigurasjon, med fører (ca. 75\,kg).}

\placefig[here][pict:steerin_sweeping:radius]{Spor-/vendekrets og feieradius}
{\externalfigure[steerin_sweeping:radius]}

\page [yes]


\subsection{Hjul og dekk}

\starttabulate[|lBw(45mm)|p|rw(55mm)|]
\FL
\NC Komponent \NC \bf Utstyr \NC \bf Verdi \NC\NR
\ML
\NC Dekk \NC standarddimensjoner \NC 205/70 R 15 C \NC\NR
\ML
\NC Felger \NC standarddimensjoner \NC 6J\;×\;15 H2 ET 60 \NC\NR
\ML
\NC Dekktrykk \NC standard, foran/bak \NC 4,5/5,8\,bar \NC\NR
\LL
\stoptabulate


\subsection{Dieselmotor}

\starttabulate [|lBw(45mm)|l|rp|]
\FL
\NC \bf Gruppe\index{Dieselmotor+Identifikasjon} \NC \bf Parameter \NC \bf Verdi\NC\NR
\ML
\NC Motortype \NC \NC VW CJDA TDI 2.0 – 475 NE \NC\NR
\NC Generelt \NC takting \NC firetaktsmotor \NC\NR
\NC\NC antall sylindere \unite{n} \NC 4 \NC\NR
\NC\NC boring x slag \unite{mm} \NC 81\;×\;95,5 \NC\NR
\NC\NC totalt slagvolum \unite{cm\high{3}} \NC 1968 \NC\NR
\NC\NC ventiler per sylinder \NC 4 \NC\NR
\NC\NC rekkefølge til ventilstyringen \NC 1-3-4-2 \NC\NR
\NC\NC laveste tomgangsturtall \unite{min\high{−1}} \NC 830 +50/−25 \NC\NR
\NC Ytelse/dreiemoment \NC maks. turtall \unite{min\high{−1}} \NC 3400 \NC\NR
\NC\NC maks. ytelse \unite{kW} ved \unite{min\high{−1}} \NC 75 til 3000 \NC\NR
\NC\NC maks. dreiemoment \unite{Nm} ved \unite{min\high{−1}} \NC 285 til 1750 \NC\NR
\NC Spesifikt forbruk\index{Dieselmotor+forbruk} \NC drivstoff \unite{g/kWh} \NC 224 (ved maks. ytelse) \NC\NR
\NC\NC Olje \unite{g/kWh} \NC 0,22 \NC\NR
\NC Drivstoffsystem \NC innsprøytningssystem \NC Direkte innsprøytning \quote{Common Rail} \NC\NR
\NC\NC drivstofftilførsel \NC tannhjulspumpe \NC\NR
\NC\NC dading \NC Ja \NC\NR
\NC\NC ladeluftkjøling \NC Ja \NC\NR
\NC\NC ladetrykk \unite{mbar} \NC 1300\NC\NR
\NC Smørekrets\index{Dieselmotor+Smøring} \NC type \NC Tvungen smøring med olje-/vannveksler \NC\NR
\NC\NC linjemating \NC rotorpumpe \NC\NR
\NC\NC oljeforbruk \unite{liter/20\,h} \NC <\:0,1 \NC\NR
\NC Kjølekrets\index{Dieselmotor+kjøling} \NC total kapasitet \unite{l} \NC ca. 12 \NC\NR
\NC\NC kalibreringstrykk ekspansjonsbeholder \unite{bar} \NC 1,4 \NC\NR
\NC\NC termostat (åpning) \unite{°C} \NC 87 \NC\NR
\NC\NC termostat (fullt) \unite{°C} \NC 102 \NC\NR
\NC Avgass \NC partikkelfilter \NC Ja \NC\NR
\NC\NC avgassbehandling \NC Ja \NC\NR
\NC\NC standard \NC Euro 5 \NC\NR
\LL
\stoptabulate


\subsection{Kjøreytelser}

\starttabulate[|lBw(45mm)|p|l|rw(35mm)|]
\FL
\NC Kjøreytelse\index{Kjøreytelser} \NC \bf Konfigurasjon \NC \bf Enhet \NC \bf Verdi \NC\NR
\ML
\NC Hastighet \NC \aW{arbeids}modus \NC \unite{km/h} \NC 0 til 18 (trinnløst) \NC\NR
\NC\NC \aW{kjøre}modus \NC \unite{km/h} \NC 0 til 40 \NC\NR
\ML
\NC Hastighetsbegrensning \NC justerbar \NC \unite{km/h} \NC 0 til 25 \NC\NR
\LL
\stoptabulate


\subsection{Elektrisk anlegg}

{\starttabulate [|lw(65mm)|p|rw(30mm)|]
\FL
\NC \bf Gruppe \NC \bf Komponent \NC \bf Verdi \NC\NR
\ML
\NC Batteri \NC blybatteri \NC 12\,V 63\,Ah \NC\NR
\NC Strømforsyning \NC dynamo \NC 14,8\,V 90\,A \NC\NR
\NC Starter \NC ytelse \NC 1,8\,kW \NC\NR
\NC Audioutstyr \NC radiotilkobling\index{radiotilkobling} og høyttalere\index{Høyttaler} \NC standardutstyr \NC\NR
\NC Belysning og signalinnretning foran \NC parkeringslys \NC 12\,V 5\,W \NC\NR
\NC\NC nærlys \NC H7, 12\,V 55\,W \NC\NR
\NC\NC arbeidslyskaster \NC G886, 12\,V 55\,W \NC\NR
\NC\NC blinklys \NC 12\,V 21\,W \NC\NR
\NC Belysnings-/signalinnretninger bak \NC kombinerte bremselys \NC 12\,V 5/21\,W \NC\NR
\NC\NC blinklys \NC 12\,V 21\,W \NC\NR
\NC\NC ryggelys \NC 12\,V 21\,W \NC\NR
\NC\NC nummerplatelys \NC 12\,V 5\,W \NC\NR
\NC Tilleggsbelysning \NC roterende varsellys \NC H1, 12\,V 55\,W \NC\NR
\LL
\stoptabulate
}
\stopsection

\stopchapter

\stopcomponent

