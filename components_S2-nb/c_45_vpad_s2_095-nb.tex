% coding: utf-8

\startcomponent c_45_vpad_s2_095-nb

\startchapter[title={Kjørecomputer (Vpad)},
reference={sec:vpad}]

\setups[pagestyle:marginless]


\startsection[title={Beskrivelse av Vpad},
reference={vpad:description}]

\startfigtext [left] {Vpad SN i førerhuset}
{\externalfigure[vpad:inside:view]}
\textDescrHead{Innovativ, intelligent … } Enheten\Vpad\ er blitt utviklet for styring av aggregater i det kommunaltekniske innsatsområdet, for å imøtekomme denne stadig mer komplekse teknologien og til fulle å kunne utnytte et mangfold av funksjoner.
Med \Vpad\ har operatøren et system tilgjengelig, som ikke er begrenset til, kun å vise sanntids-informasjon~– visuelt eller akustisk~– over samtlige arbeids- og maskinprosesser.
Det som utmerker \Vpad\ spesielt og som den setter nye standarder for, er den intuitive brukerveiledningen, brukerergonomien og kommandologikken.

Takket være dens mangfold av funksjoner er \Vpad\ meget fleksibel i bruk og utgjør dermed mye mer enn kun en elektronisk styreenhet.
\stopfigtext

\textDescrHead{… universell} kompatibilitet og fleksibilitet har vært i fokus under utviklingen av \Vpad:
Som modulær styreenhet kan den tilpasses lokale forhold og utstyrsvarianter, og grunnet dens tallrike elektroniske grensesnitt og metoder for dataoverføring~– også via WLAN~– står alle muligheter åpne.
\Vpad\ arbeider med det siste innen elektronikk med 32-bit-teknologi og sanntids-driftssystem.
\vfill


\startfigtext[left]{Multifunksjonspanel}
{\externalfigure[console:topview]}
\textDescrHead{… og modulær} Grunnet dens modularitet har \Vpad\ en enorm fordel:
Slik kan den i \sdeux\ som standard anvendte versjonen~SN når som helst utvides trinnvis med ytterligere komponenter, som for eksempel et modem eller et panel (se figur).
Modulariteten er ikke begrenset til maskinvaren, for også på programvaresiden kan dette systemet i stor grad utvides og tilpasses de endrede behov.

Multifunksjonspanelet til \sdeux\ er et høyt utviklet grensesnitt mellom operatør og maskin. Det komplette feie-/sugesystemet lar seg styre via dette panelet.
\stopfigtext

\page [yes]


\subsection[vpad:home]{Hovedskjerm}

%% Note: outcommented by PB
% \placefig[left][fig:vpad:engineData]{Accueil mode transport}
% {\scale[sx=1.5,sy=1.5]
% {\setups[VpadFramedFigureHome]
% \VpadScreenConfig{
% \VpadFoot{\VpadPictures{vpadClear}{vpadBeacon}{vpadEngine}{vpadSignal}}}
% \framed{\null}}
% }


\start

\setupcombinations[width=\textwidth]

\placefig [here][fig:vpad:engineData]{Hovedskjerm}
{\startcombination [2*1]
{\setups[VpadFramedFigureHome]% \VpadFramedFigureK pour bande noire
\VpadScreenConfig{
\VpadFoot{\VpadPictures{vpadClear}{vpadBeacon}{vpadEngine}{vpadSignal}}}%
\scale[sx=1.5,sy=1.5]{\framed{\null}}}{\aW{Kjøre}modus}
{\setups[VpadFramedFigureWork]% \VpadFramedFigureK pour bande noire
\VpadScreenConfig{
\VpadFoot{\VpadPictures{vpadClear}{vpadBeacon}{vpadEngine}{vpadSignal}}}%
\scale[sx=1.5,sy=1.5]{\framed{\null}}}{\aW{Arbeids}modus}
\stopcombination}

\stop

\blank [1*big]

Hovedskjermen til \Vpad\ omfatter alle nødvendige elementer for overvåkning av samtlige funksjonene til \sdeux.

I den øvre delen finner du kontrollvisningene.

I den midtre delen vises bl.\,a. de følgende data i sanntid:
Hastigheten, turtallet og temperaturen til motoren, drivstoffnivået, fyllenivået til resirkuleringsvannet osv.

Modusen \aW{Kjøre} symboliseres med en hare~\textSymb{transport_mode}, modusen \aW{Arbeid} symboliseres av en skilpadde~\textSymb{working_mode}.

Menylinjen ved nedre kanten viser menyene som er tilgjengelige: Trykk på midten av den berøringsfølsome skjermen (touch-screen), for visning av ytterligere menyer.

\page [yes]

\start % local group for temporary redefinition of \textDescrHead [TF]
\define[1]\textDescrHead{{\bf#1\fourperemspace}}
\startcolumns

\startSymVpad
\externalfigure[vpadTEnginOilPressure][height=1.7\lH]
\SymVpad
\textDescrHead{Motoroljetrykk}(rød) For lavt motoroljetrykk. Slå umiddelbart av motoren.

+\:Feilmelding \# 604
\stopSymVpad

\startSymVpad
\externalfigure[vpadWarningBattery][height=1.7\lH]
\SymVpad
\textDescrHead{Batteriladning}(rød) For lav batteriladestrøm. Ta kontakt med verkstedet.
\stopSymVpad

\startSymVpad
\externalfigure[vpadWarningEngine1][height=1.7\lH]
\SymVpad
\textDescrHead{Motordiagnose}(gul) Feil i motorstyringen. Ta kontakt med verkstedet.
\stopSymVpad

\startSymVpad
\externalfigure[vpadWarningService][height=1.7\lH]
\SymVpad
\textDescrHead{Verksted må oppsøkes}(gul) Standard kjøretøyvedlikehold har forfalt. Se til vedlikeholdsplanen.

+\:Feilmeldinger \# 650 til \# 653, eller \# 703
\stopSymVpad

\startSymVpad
\externalfigure[vpadTBrakeError][height=1.7\lH]
\SymVpad
\textDescrHead{Bremsesystem}(rød) Feil i bremsesystemet. Ta kontakt med verkstedet.

+\:Feilmelding \#  902
\stopSymVpad


\startSymVpad
\externalfigure[vpadTBrakePark][height=1.7\lH]
\SymVpad
\textDescrHead{Parkeringsbremse}(rød) Parkeringsbremsen er aktivert.

+\:Feilmelding \#  905
\stopSymVpad

\startSymVpad
\externalfigure[vpadTEngineHeating][height=1.7\lH]
\SymVpad
\textDescrHead{Forglødeanlegg}(gul) Motoren blir forglødet.

En blinkende lampe indikerer, at det er registrert en feil i hendelsesminnet.
\stopSymVpad

\columnbreak

\startSymVpad
\externalfigure[vpadTFuelReserve][height=1.7\lH]
\SymVpad
\textDescrHead{Drivstoffnivå}(gul) Drivstoffnivå er svært lavt (reserve).
\stopSymVpad

\startSymVpad
\externalfigure[vpadTBlink][height=1.7\lH]
\SymVpad
\textDescrHead{Varselblinklys}(grønn) Varselblinklys er aktivert.
\stopSymVpad

\startSymVpad
\externalfigure[vpadTLowBeam][height=1.7\lH]
\SymVpad
\textDescrHead{Parkeringslys}(grønn) Parkeringslys er koblet inn.
\stopSymVpad

\startSymVpad
\HL\NC \externalfigure[vpadSyWaterTemp][height=1.7\lH]
\SymVpad
\textDescrHead{Temperatur}(rød) For høy temperatur på hydraulikkvæsken eller motoren. Ta kontakt med verkstedet.

+\:Feilmelding \# 700 eller \# 610
\stopSymVpad

\startSymVpad
\externalfigure[vpadWarningFilter][height=1.7\lH]
\SymVpad
\textDescrHead{Filter er tett}(rød) Det kombinerte hydraulikkfilteret eller luftfilteret er tett.

+\:Feilmelding \# 702 eller \# 851
\stopSymVpad

\startSymVpad
\externalfigure[vpadTSpray][height=1.7\lH]
\SymVpad
\textDescrHead{Vannpistol}(gul) Høytrykksvannpumpen for vannpistolen er deaktivert.

Bryter \textSymb{temoin_buse} i takpanelet.
\stopSymVpad

\startSymVpad
\externalfigure[vpadTClear][height=1.7\lH]
\SymVpad
\textDescrHead{Feilmelding}(rød) En feilmelding foreligger i minnet til \Vpad. Trykk på tasten~\textSymb{vpadClear}, for å vise alle registrerte meldinger. Ta kontakt med verkstedet.
\stopSymVpad

\stopcolumns
\stop % local group for temporary redefinition of \textDescrHead

\stopsection

\page [yes]


\section{Vpad-menyer}

\start

\setupTABLE [background=color,
frame=off,
option=stretch,textwidth=\makeupwidth]

\setupTABLE [r] [each] [style=sans, background=color, bottomframe=on, framecolor=TableWhite, rulethickness=1.5pt]
\setupTABLE [r] [first][backgroundcolor=TableDark, style=sansbold]
\setupTABLE [r] [odd][backgroundcolor=TableMiddle]
\setupTABLE [r] [even] [backgroundcolor=TableLight]
\bTABLE [split=repeat]
\bTABLEhead
\bTR\bTD Meny \eTD\bTD Betegnelse\index{Vpad+Visning} \eTD\bTD Funksjon \eTD\eTR
\eTABLEhead

\bTABLEbody
\bTR\bTD \externalfigure [v:symbole:clear] \eTD\bTD Feilmelding(er) \eTD\bTD Vise og kvittere (bekrefte) feilmeldingene som er registrert i Vpad. \eTD\eTR
\bTR\bTD \framed[frame=off]{\externalfigure [v:symbole:beacon]\externalfigure [v:symbole:beacon:black]} \eTD\bTD Roterende varsellys \eTD\bTD Roterende varsellys på/av \eTD\eTR
\bTR\bTD \externalfigure [v:symbole:engine] \eTD\bTD Sanntidsdata \eTD\bTD Vise sanntids driftsdata for motor og hydraulikk\eTD\eTR
\bTR\bTD \externalfigure [v:symbole:oneTwoThree] \eTD\bTD Teller \eTD\bTD Visning av driftstimetellere: Dagsteller, sesongteller, totalteller\eTD\eTR
\bTR\bTD \externalfigure [v:symbole:serviceInfo] \eTD\bTD Vedlikeholdsintervall \eTD\bTD Viser datoen samt gjenværende driftstimer frem til neste vedlikehold eller til neste større service \eTD\eTR
\bTR\bTD \externalfigure [v:symbole:trash] \eTD\bTD Teller \eTD\bTD Tilbakestille teller eller tilbakestille serviceintervall \eTD\eTR
\bTR\bTD \externalfigure [v:symbole:sunglasses] \eTD\bTD Skjermmodus \eTD\bTD Koble om typen skjermbelysning mellom \aW{Dag} og \aW{Natt} \eTD\eTR
\bTR\bTD \externalfigure [v:symbole:color] \eTD\bTD Lysstyrke/kontrast \eTD\bTD Innstilling for lysstyrken og kontrasten til skjermen \eTD\eTR
\bTR\bTD \externalfigure [v:symbole:select] \eTD\bTD Valg \eTD\bTD Valg av den markerte innføringen eller kvittering av en feilmelding \eTD\eTR
\bTR\bTD \externalfigure [v:symbole:return] \eTD\bTD Bekreftelse \eTD\bTD Bekreftelse av valget \eTD\eTR
\bTR\bTD \framed[frame=off]{\externalfigure [v:symbole:up]\externalfigure [v:symbole:down]} \eTD\bTD Opp/ned, \\Pfeile \eTD\bTD Forskyve markering oppover/nedover eller øke/redusere valgt verdi \eTD\eTR
\bTR\bTD \externalfigure [v:symbole:rSignal] \eTD\bTD Varseltone for rygging \eTD\bTD Aktivere/deaktivere varselsignal for rygging \eTD\eTR
\eTABLEbody
\eTABLE
\stop


\subsubsubject{Øvrige visninger på Vpad}

\start % local group for temporary redefinition of \textDescrHead [TF]
\define[1]\textDescrHead{{\bf#1\fourperemspace}}

\startcolumns

\startSymVpad
\externalfigure[sym:vpad:water]
\SymVpad
\textDescrHead{Fyllenivå ferskvann} Fyllenivået for ferskvann er ikke tilstrekkelig (maks. 190\,l; bak førerhuset).
\stopSymVpad

\startSymVpad
\externalfigure[sym:vpad:rwater:yellow]
\SymVpad
\textDescrHead{Fyllenivå resirkuleringsvann}(gul) Fyllenivået til resirkuleringsvannet er under varmeveksleren. Det skjer ingen kjøling av hydraulikkvæsken og ingen oppvarming av befuktningssystemet til sugekanalen.
\stopSymVpad

\startSymVpad
\externalfigure[sym:vpad:rwater]
\SymVpad
\textDescrHead{Fyllenivå resirkuleringsvann}(rød) Fyllenivået til resirkuleringsvannet er ikke tilstrekkelig (maks. 140\,l; under smussbeholderen).
\stopSymVpad

\stopcolumns
\stop % local group for temporary redefinition of \textDescrHead

\page [yes]

\startsection[title={Innstilling av lysstyrken til skjermen},
reference={sec:vpad:brightness}]

Skjermen til \Vpad\ kan drives i to forhåndskonfigurerte lysstyrketrinn: Modus \aW{Dag}~– \textSymb{vpadSunglasses}, normal lysstyrke~– og modus \aW{Natt}~– \textSymb{vpadMoon}, redusert lysstyrke.
Med tasten \textSymb{vpadColor} kan du få tilgang til forskjellige parametere.

For å endre de forhåndskonfigurerte lysstyrketrinnene, gå frem som følger:

\startSteps
\item Trykk på midten av den berøringsfølsome skjermen (touchscreen), for å skrolle gjennom menylinjen på nedre skjermkanten.
\item Trykk på symbolet \textSymb{vpadSunglasses} hhv.
\textSymb{vpadMoon}, for å velge modusen, som du vil endre.
\item Trykk \textSymb{vpadColor}, for å vise parametrene.
\item Benytt pilsymbolene~\textSymb{vpadUp}\textSymb{vpadDown} til å markere parameteren, som du ønsker å endre, og velg den ut med~\textSymb{vpadSelect}.
\item Endre verdien ved hjelp av symbolene\textSymb{vpadMinus}\textSymb{vpadPlus}. Forsiktig, ikke reduser lysstyrken så kraftig (\VpadOp{162} -255), at det ikke lenger er mulig å se noe på skjermen!
\stopSteps
\blank [1*big]

\start
\setupcombinations[width=\textwidth]
\startcombination [3*1]
{\setups[VpadFramedFigureHome]% \VpadFramedFigureK pour bande noire
\VpadScreenConfig{
\VpadFoot{\VpadPictures{vpadGPS}{vpadTachygraph}{vpadSunglasses}{vpadColor}}}%
\framed{\null}}{Trykk på midten av berøringsskjermen}
{\setups[VpadFramedFigure]
\VpadScreenConfig{
\VpadFoot{\VpadPictures{vpadReturn}{vpadUp}{vpadDown}{vpadSelect}}}%
\framed{\bTABLE
\bTR\bTD \VpadOp{160} \eTD\eTR
\bTR\bTD [backgroundcolor=black,color=TableWhite] \VpadOp{162}\hfill 15 \eTD\eTR
\bTR\bTD \VpadOp{163}\hfill 180 \eTD\eTR
\bTR\bTD \VpadOp{164}\hfill 55 \eTD\eTR
\bTR\bTD \VpadOp{165}\hfill 3 \eTD\eTR
\eTABLE}}{Velg med \textSymb{vpadSelect}}
{\setups[VpadFramedFigure]% \VpadFramedFigureK pour bande noire
\VpadScreenConfig{
\VpadFoot{\VpadPictures{vpadReturn}{vpadMinus}{vpadPlus}{vpadNull}}}%
\framed[backgroundscreen=.9]{\bTABLE
\bTR\bTD \VpadOp{160} \eTD\eTR
\bTR\bTD \VpadOp{162}\hfill -80 \eTD\eTR
\bTR\bTD \VpadOp{163}\hfill 180 \eTD\eTR
\bTR\bTD \VpadOp{164}\hfill 55 \eTD\eTR
\bTR\bTD \VpadOp{165}\hfill 3 \eTD\eTR
\eTABLE}}{Endre verdi med \textSymb{vpadMinus}\textSymb{vpadPlus}}
\stopcombination
\stop
\blank [1*big]

\startSteps [continue]
\item Bekreft verdien med \textSymb{vpadReturn}.
\item Trykk en gang til på symbolet \textSymb{vpadReturn}, for å komme tilbake til hovedskjermen.
\stopSteps

\stopsection

\page [yes]


\startsection[title={Driftstime- og kilometerteller},
reference={vpad:compteurs}]

Programvaren til \Vpad\ har tre forskjellige måleperioder~– \aW{Dag}, \aW{Sesong}, \aW{Total}~–, i hvilke det kan løpe forskjellige tellere, som \aW{Tilbakelagt strekning}, \aW{Driftstimer} (motor eller børste), \aW{Arbeidstid} (per fører).

For å lese av tellerne eller å tilbakestille de, gå frem som følger:

\startSteps
\item Trykk på midten av berøringsskjermen, for å skrolle gjennom menylinjen.
\item Trykk på symbolet \textSymb{vpadOneTwoThree}, for å vise dagstelleren.
\item Ved hjelp av tilbake-/frem-symbolene~\textSymb{vpadBW}\textSymb{vpadFW} kan det skiftes over til total- eller sesongtelleren.
\item Trykk på \textSymb{vpadTrash}, for å tilbakestille den viste telleren.
\item Det vises et dialogvindu som oppfordrer deg til å bekrefte tilbakestillingen.
\stopSteps
\blank [1*big]

\start
\setupcombinations[width=\textwidth]
\startcombination [3*1]
{\setups[VpadFramedFigure]% \VpadFramedFigureK pour bande noire
\VpadScreenConfig{
\VpadFoot{\VpadPictures{vpadOneTwoThree}{vpadTachygraph}{vpadSunglasses}{vpadColor}}}%
\framed{\bTABLE
\bTR\bTD \VpadOp{120} \eTD\eTR
\bTR\bTD \VpadOp{123}\hfill 87.4\,h \eTD\eTR
\bTR\bTD \VpadOp{125}\hfill 62.0\,h \eTD\eTR
\bTR\bTD \VpadOp{126}\hfill 240.2\,km \eTD\eTR
\bTR\bTD \VpadOp{124}\hfill 901.9\,km \eTD\eTR
\bTR\bTD \VpadOp{127}\hfill 2,1\,l/time \eTD\eTR
\eTABLE}}{Trykk på symbolet~\textSymb{vpadOneTwoThree}, deretter på~\textSymb{vpadBW} eller~\textSymb{vpadFW}}
{\setups[VpadFramedFigure]
\VpadScreenConfig{
\VpadFoot{\VpadPictures{vpadReturn}{vpadBW}{vpadFW}{vpadTrash}}}%
\framed{\bTABLE
\bTR\bTD \VpadOp{121} \eTD\eTR
\bTR\bTD \VpadOp{123}\hfill 522.0\,h \eTD\eTR
\bTR\bTD \VpadOp{125}\hfill 662.8\,h \eTD\eTR
\bTR\bTD \VpadOp{126}\hfill 1605.5\,km \eTD\eTR
\bTR\bTD \VpadOp{124}\hfill 2608.4\,km \eTD\eTR
\bTR\bTD \VpadOp{127}\hfill 2,0\,l/time \eTD\eTR
\eTABLE}}{Bruk \textSymb{vpadTrash} for å tilbakestille telleren}
{\setups[VpadFramedFigure]% \VpadFramedFigureK pour bande noire
\VpadScreenConfig{
\VpadFoot{\VpadPictures{vpadReturn}{vpadTrash}{vpadNull}{vpadNull}}}%
\framed{\bTABLE
\bTR\bTD \VpadOp{121} \eTD\eTR
\bTR\bTD \null \eTD\eTR
\bTR\bTD \VpadOp{136} \eTD\eTR
\bTR\bTD \null \eTD\eTR
\bTR\bTD \VpadOp{137} \eTD\eTR
\eTABLE}}{Bekreft med \textSymb{vpadTrash}}
\stopcombination
\stop
\blank [1*big]

\startSteps [continue]
\item Legg inn passordet, når påkrevd, og bekreft så tilbakestillingen ved hjelp av symbolet \textSymb{vpadTrash}.
\item Trykk på symbolet \textSymb{vpadReturn}, for å komme tilbake til hovedskjermen.
\stopSteps

\stopsection

\page [yes]

\startsection[title={Vedlikeholdsintervaller},
reference={vpad:maintenance}]

Vedlikeholdsplanen til \sdeux\ omfatter to grunntyper av vedlikehold: standard vedlikehold og stor service (fagverksted som har serviceavtale med \boschung-kundeservice).

For å lese av eller å tilbakestille tellerne, gå frem som følger:
\startSteps
\item Trykk på midten av berøringsskjermen, for å skrolle gjennom menylinjen.
\item Trykk på symbolet \textSymb{vpadServiceInfo}, for å vise vedlikeholdsintervallene.
\item Benytt pilsymbolene~\textSymb{vpadUp}\textSymb{vpadDown} for å skifte til ønsket intervall.
\item Trykk på symbolet ~\textSymb{vpadTrash}, for å tilbakestille et vedlikeholdsintervall. Tast inn passordet ved hjelp av~\textSymb{vpadPlus}\textSymb{vpadMinus} og bekreft med~\textSymb{vpadSelect}).
\item Det vises et dialogvindu som oppfordrer deg til å bekrefte tilbakestillingen.
\stopSteps
\blank [1*big]

\start
\setupcombinations[width=\textwidth]
\startcombination [3*1]
{\setups[VpadFramedFigure]% \VpadFramedFigureK pour bande noire
\VpadScreenConfig{
\VpadFoot{\VpadPictures{vpadReturn}{vpadNull}{vpadNull}{vpadTrash}}}%
\framed{\bTABLE
\bTR\bTD[nc=2] \VpadOp{190} \eTD\eTR
\bTR\bTD \VpadOp{191}\eTD\bTD \VpadOp{195}\hfill 600\,h \eTD\eTR % [backgroundcolor=black,color=TableWhite]
\bTR\bTD \VpadOp{192}\eTD\bTD \VpadOp{195}\hfill 600\,h \eTD\eTR
\bTR\bTD \VpadOp{193}\eTD\bTD \VpadOp{195}\hfill 2400\,h \eTD\eTR
\eTABLE}}{Trykk på symbolet~\textSymb{vpadTrash}, for å tilbakestille et intervall}
{\setups[VpadFramedFigure]
\VpadScreenConfig{
\VpadFoot{\VpadPictures{vpadReturn}{vpadMinus}{vpadPlus}{vpadSelect}}}%
\framed{\bTABLE
\bTR\bTD \VpadOp{190} \eTD\eTR
\bTR\bTD \hfill 2014-03-31 \eTD\eTR
\bTR\bTD \null \eTD\eTR
\bTR\bTD \null \eTD\eTR
\bTR\bTD \null \eTD\eTR
\bTR\bTD \null \eTD\eTR
\bTR\bTD \VpadOp{002}\hfill 0000 \eTD\eTR
\eTABLE}}{Tast inn passordet (tallkode)}
{\setups[VpadFramedFigure]% \VpadFramedFigureK pour bande noire
\VpadScreenConfig{
\VpadFoot{\VpadPictures{vpadReturn}{vpadUp}{vpadDown}{vpadSelect}}}%
\framed{\bTABLE
\bTR\bTD \VpadOp{190} \eTD\eTR
\bTR\bTD[backgroundcolor=black,color=TableWhite] \VpadOp{041}\eTD\eTR % [backgroundcolor=black,color=TableWhite]
\bTR\bTD \VpadOp{042} \eTD\eTR
\bTR\bTD \VpadOp{043} \eTD\eTR
\eTABLE}}{Velg og bekreft med~\textSymb{vpadSelect}}
\stopcombination
\stop
\blank [1*big]

\startSteps [continue]
\item Bekreft tilbakestillingen ved hjelp av~\textSymb{vpadSelect}.
\item Trykk på symbolet \textSymb{vpadReturn}, for å komme tilbake til hovedskjermen.
\stopSteps

\stopsection

\page [yes]


\startsection[title={Feilstyring via Vpad},
reference={vpad:error}]


\Vpad\ viser feil\index{Vpad+feilmeldinger}, som er diagnostisert av de elektroniske styresystemene og overført av CAN-Bus.
Når en mindre alvorlige feil blir registrert, lyser symbolet~\textSymb{VpadTClear} (rødt).
Når det dreier seg om en feil av høy prioritet, lyser symbolet~\textSymb{VpadTClear} og en alarmtone lyder i tillegg.
For å stoppe alarmen, må feilmeldingen kvitteres (\aW{notert} bekreftes).

For å lese feilmeldinger og kvittere de, gå frem som følger:

\startSteps
\item Trykk på symbolet~\textSymb{vpadClear} på skjermen til \Vpad.
\item Trykk på symbolet~\textSymb{vpadClear}, for å kvittere den valgte meldingen.
\item Ved siden av den kvitterte meldingen vises det nå et \aW{\#}-symbol, som marker meldingen som \aW{notert}, og markeringen hopper til neste melding (såfremt det finnes flere).
\item Etter at alle meldingene er blitt kvittert, går visningen tilbake til hovedskjermen.
\stopSteps
\blank [1*big]

\start
\setupcombinations[width=\textwidth]
\startcombination [3*1]
{\setups[VpadFramedFigure]% \VpadFramedFigureK pour bande noire
\VpadScreenConfig{
\VpadFoot{\VpadPictures{vpadReturn}{vpadUp}{vpadDown}{vpadSelect}}}%
\framed{\bTABLE
\bTR\bTD \VpadEr{000} \eTD\eTR
\bTR\bTD [backgroundcolor=black,color=TableWhite] \VpadEr{851a} \eTD\eTR
\bTR\bTD \VpadEr{902} \eTD\eTR
\eTABLE}}{Visning av meldingene}
{\setups[VpadFramedFigure]
\VpadScreenConfig{
\VpadFoot{\VpadPictures{vpadReturn}{vpadUp}{vpadDown}{vpadSelect}}}%
\framed{\bTABLE
\bTR\bTD \VpadEr{000} \eTD\eTR
\bTR\bTD [backgroundcolor=black,color=TableWhite] \VpadEr{851} \eTD\eTR
\bTR\bTD \VpadEr{902} \eTD\eTR
\eTABLE}}{Kvitter med~\textSymb{vpadClear}}
{\setups[VpadFramedFigureHome]% \VpadFramedFigureK pour bande noire
\VpadScreenConfig{
\VpadFoot{\VpadPictures{vpadClear}{vpadBeacon}{vpadBeam}{vpadEngine}}}%
\framed{\null}}{Tilbake til hovedskjermen}
\stopcombination
\stop
\blank [1*big]

\startSteps [continue]
\item For å vise meldingen på nytt, trykk på symbolet~\textSymb{vpadClear}. Feilmeldinger blir først slettet fra \Vpad\, når årsaken til problemet er fjernet.
\stopSteps


\subsection{De vanligste feilmeldingene (med feilsøk)}

\subsubsubject{\VpadEr{604}} % {\#\ 604 Pression huile moteur basse}

+ \textSymb{vpadTEnginOilPressure}~– Slå umiddelbart av motoren. Kontroller oljenivået, ta kontakt med verkstedet.


\subsubsubject{\VpadEr{609}} % {\#\ 609 Température eau refroidissement moteur haute}

+ \textSymb{vpadSyWaterTemp}~– Avbryt arbeidet. La motoren gå videre uten belastning og følg med på temperaturutviklingen:

Når temperaturen synker, kontroller fyllenivåene på kjølevæske, motorolje og hydraulikkvæske samt tilstanden til kjøleren.
Når fyllenivåene og kjøleren er i orden, kjør maskinen forsiktig til verkstedet for videre feildiagnose.

\subsubsubject{\VpadEr{610}} % {\#\ 610 Température eau refroidissement moteur trop haute}

+ \textSymb{vpadSyWaterTemp}~– Avbryt arbeidet. Kontroller fyllenivåene på kjølevæsle og motorolje, ta omgående kontakt med verkstedet.


\subsubsubject{\VpadEr{650}} % {\#\ 650 Se rendre à un garage}

+ \textSymb{vpadWarningService}~– Ta omgående kontakt med ditt verksted.
% \VpadEr{651} % {\#\ 651 Moteur en mode urgence}


\subsubsubject{\VpadEr{652}} % {\#\ 652 Inspection véhicule}
% \VpadEr{653} % {\#\ 653 Grand service moteur}

+ \textSymb{vpadWarningService}~– Det neste standard vedlikeholdet har forfalt. Konsulter vedlikeholdsplanen og avtal en termin med ditt verksted.


\subsubsubject{\VpadEr{700}} % {\#\ 700 Température d'huile hydraulique}

+ \textSymb{vpadSyWaterTemp}~– Avbryt arbeidet. La motoren gå videre uten belastning og følg med på temperaturutviklingen:

Når temperaturen synker, kontroller fyllenivåene på kjølevæske, motorolje og hydraulikkvæske samt tilstanden til kjøleren.
Når fyllenivåene og kjøleren er i orden, kjør maskinen forsiktig til verkstedet for videre feildiagnose.


\subsubsubject{\VpadEr{702}} % {\#\ 702 Filtre d'huile hydraulique}

+ \textSymb{vpadWarningFilter}~– Hydraulikk-tilbakeløpet- og/eller sugefilteret er tett. Skift omgående ut filterelementet.
% \VpadEr{703} % {\#\ 703 Vidange d'huile hydraulique}


\subsubsubject{\VpadEr{800}} % {\#\ 800 Interrupteur d'urgence actionné}

+ \textSymb{vpadTClear}~– Du har betjent Nødstopp-bryteren. Slå av tenningen og start motoren på nytt, for å slette meldingen.


\subsubsubject{\VpadEr{801}} % {\#\ 905 Frein à main actionné}

Smussbeholderen er hevet eller ikke senket fullstendig. Hastigheten til kjøretøyet er begrenset til 5\,km/h, så lenge smussbeholderen ikke er senket.

\subsubsubject{\VpadEr{851}} % {\#\ 851 Filtre à air}

+ \textSymb{vpadWarningFilter}~– Luftfilteret er tett. Skift omgående ut filterelementet.


\subsubsubject{\VpadEr{902}} % {\#\ 902 Pression de freinage}

+ \textSymb{vpadTBrakeError}~– Bremsetrykket er ikke tilstrekkelig. Avbryt arbeidet og ta omgående kontakt med verkstedet.
% \VpadEr{904} % {\#\ 904 Interrupteur de direction d'avancement}


\subsubsubject{\VpadEr{905}} % {\#\ 905 Frein à main actionné}

+ \textSymb{vpadTBrakePark}~– Parkeringsbremsen er ikke løsnet fullstendig. Hastigheten til kjøretøyet er begrenset til 5\,km/h, så lenge parkeringsbremsen ikke er løsnet.


\stopsection

\stopchapter

\stopcomponent














