% coding: utf-8

\startcomponent c_40_control_s2_095-nb


\startchapter [title={Betjeningselementer til kjøretøyet},
reference={chap:ctrl}]

\setups[pagestyle:marginless]

\placefig[here][fig:ctrl:cab:front]{Betjeningselementer}
{\externalfigure[ctrl:cab:front]}

\startcolumns [n=3]
\startLongleg
\item Rattsøyle (\in{§}[sec:steeringColumn])
\item Innstilling rattsøyle
\item Gass- og bremsepedal
\item Kjøredatamaskin \Vpad~SN (\inP[sec:vpad])
\item Takpanel (\inP[sec:ctrl:aux])
\stopLongleg


\subsubsubject{Valgfritt utstyr}

\startLongleg [continue]
\item Ryggemonitor
\item Radio/MP3
\stopLongleg
\stopcolumns

\startsection [title={Rattsøylen},
reference={sec:steeringColumn}]

\subsection{Innstilling av rattsøylen}

\textDescrHead{Vinkel til rattet} Trykk på pedalen~\Ltwo og still samtidig inn vinkelen til rattsøylen. Slipp pedalen, for at mekanismen til rattsøylen igjen blir låst.

\page[yes]
\setups [pagestyle:normal]


\subsection{Belysnings- og signalinnretninger}

\placefig [margin] [fig:column:left] {Multifunksjonsspak og dreiebryter}
{\externalfigure[ctrl:column:left]}

\placefig [margin] [fig:column:right] {Kjøretrinnvalgspak}
{\externalfigure[ctrl:column:right]}


\subsubsubject{Dreiebryter}

\startitemize[width=1.7em]
\sym{\textSymb{com_lowlight}} Nærlys (dreie~\TorqueR)
\startitemize
\sym{1} Parkeringslys
\sym{2} Nærlys
\stopitemize
\stopitemize


\subsubsubject{Multifunksjonsspak}

\startitemize[width=1.7em]
\sym{\textSymb{com_lowlight}} {[}Ikke tilordnet{]}
\sym{\textSymb{com_light}} Lyshorn (trykk spaken kort oppover)
\sym{\textSymb{com_blink}} Kjøreretningsindikator (spak forover/bakover)
\sym{\textSymb{com_claxonArrow}} Horn (trykk knapp utvendig på spaken)
\sym{\textSymb{com_wipper}} Vindusvisker
\startitemize
\sym{J} Intervallkobling
\sym{O} Av
\sym{I} 1.\,Hastighetstrinn
\sym{II} 2.\,Hastighetstrinn
\stopitemize
\sym{\textSymb{com_washerArrow}} Vindusspyleranlegg (trykk på kransen på enden av spaken).
\stopitemize


\subsubsubject{Kjøretrinnvalgspak}

Funksjonene til kjøretrinnvalgspaken er detaljert beskrevet i kapittel~\about[chap:using], f.o.m.~\atpage[sec:using:start].

\stopsection

\page [yes]


\startsection [title={Øvrige funksjoner},
reference={sec:ctrl:add}]


\subsection[sec:ctrl:aux]{Takpanel}

{\sl \index{Takpanelet } Takpanelet befinner seg foran i taket til førerhuset på førersiden.}
\placefig [margin] [fig:console:aux] {Takpanel}
{\externalfigure[ctrl:console:aux]}


\placefig [margin] [fig:console:climat] {Oppvarming og klimaanlegg}
{\externalfigure[ctrl:console:climat]}


\startitemize [unpacked][width=1.7em]
\sym{\textBigSymb{temoin_retrochauffant}} Sidespeiloppvarming
\sym{\textBigSymb{temoin_hazard}} Varselblinklys
\sym{\textBigSymb{temoin_eclairage_L}} Arbeidslyskaster
\stopitemize


\subsubsubject{Valgfritt utstyr}

\startLeg [unpacked][width=1.7em]
\sym{\textBigSymb{temoin_buse}} Høytrykksvannpumpe for vannpistol \crlf {\sl se \atpage[sec:using:water:spray]}
\sym{\textBigSymb{temoin_aspiration_manuelle}} Turbin for håndsugeslange \crlf {\sl se \atpage[sec:using:suction:hose]}
\stopLeg


\subsection[sec:ctrl:climat]{Oppvarming og klimaanlegg}

{\sl Dette panelet\index{Panel for oppvarming} befinner seg på bakveggen til førerhuset, mellom setene.}

\startitemize [unpacked][width=23mm]
\sym{\bf 0\quad I\quad II\quad III} Vifte-dreiebryter
\sym{\externalfigure[tirette_chauffage][height=1em]} Temperatur-skyveregulator
\stopitemize


\subsubsubject{Valgfritt utstyr}

\startitemize [unpacked][width=1.7em]
\sym{\textBigSymb{temoin_climat_bk}} Klimaanlegg
\stopitemize

\page [yes]

\setups [pagestyle:bigmargin]


\subsection[sec:ctrl:central]{Midtre panel}

{\sl Det\index{midtre panelet} Midtre panelet befinner seg mellom setene.}

\placefig [margin] [fig:console:central] {Midtre panel}
{\externalfigure[ctrl:console:central]}


\subsubsubject{Fukting av kostene}

\startLeg [unpacked][width=1.7em]
\sym{\textBigSymb{temoin_busebalais}} Lavtrykksvannpumpe\index{Vannpumpe} for befuktningssystemet\index{Vannpumpe+Befuktning} av kostene. (Posisjon~1: \aW{Automatisk}; posisjon~2: \aW{Permanent})
\stopLeg


\subsubsubject{Vipping av smussbeholderen}

\setupinmargin[right][style=normal]
\inright{%
\startitemize
\sym{\textSymb{mand_readtheoperatingmanual}} Følg anvisningene for bruk av parkeringsbremsen på \atpage[sec:using:stop].
\stopitemize}

\startLeg [unpacked][width=1.7em]
\sym{\textBigSymb{temoin_kipp2}} Vipping av smussbeholderen. For å vippe\index{smussbeholderen+} for å kunne vippe smussbeholderen, må parkeringsbremsen være aktivert og kjøretrinnvalgspaken stå i nøytral stilling.
\stopLeg


\subsubsubject{Nødstopp}

\starttextbackground [FC]
\startPictPar
\externalfigure[Emergency_Stop][Pict]
\PictPar
I et nødstilfelle\index{Nødstopp} kan du koble ut alle suge- og feieapparater ved å trykke på nødstopp-bryteren.
\stopPictPar
\stoptextbackground


\subsection[sec:foot:switch]{Fotbryter}

\placefig [margin] [fig:foot:switch] {Fotbryter}
{\vskip 60pt
\externalfigure[work:foot:switch]}

Ved hjelp av\index{Fotbryter} denne bryteren på sokkelen til rattsøylen (\inF[fig:foot:switch]) kan du raskt og enkelt senke kostene, når det er nødvendig (\eG\ på toppen av en stigning, kjøring opp på fortau).

\stopsection
\page[yes]
\setups [pagestyle:marginless]


\startsection[title={Multifunktionskonsole},
reference={ctrl:console:middle}]

\startlocalfootnotes

\startfigtext[left]{Multifunksjonspanel}
{\externalfigure[overview:joy:large]}


\subsubsubject{Joysticker}

\textDescrHead{Uten frontkost (eller frontkost deaktivert):}
Joystickene styrer uavhengig av hverandre hver en kost: Heve/senke~(\textSymb{joystick_aa}) eller venstre/høyre~(\textSymb{joystick_gd}). Venstre joystick styrer venstre kost, høyre joystick styrer høyre kost.\footnote{For å kunne endre posisjonen til sidekostene på et kjøretøy som er utstyrt med frontkost (ekstrautstyr), må frontkosten være deaktiveres (tast~\textSymb{joy_key_frontbrush_act}).}

\textDescrHead{Med frontkost:}
Med venstre joystick kan du heve/senke frontkosten (\textSymb{joystick_aa}) og bevege den til venstre/høyre (\textSymb{joystick_gd}). Med høyre joystick kan du vippe kosten på sin langsgående~(\textSymb{joystick_aa}) og tverrgående akse~(\textSymb{joystick_gd}).

\placelocalfootnotes %[height=\textheight]
\stopfigtext
\stoplocalfootnotes
\vfill


\subsubsubject{Sidetaster}

\startcolumns

\startPictList
\VPcltr
\PictList
Fartsholder: Øke den innstilte hastigheten
\stopPictList\vskip -3pt

\startPictList
\VPclbr
\PictList
Fartsholder: Redusere den innstilte hastigheten
\stopPictList\vskip -3pt

\startPictList
\VPcrtr
\PictList
Heve sugeenheten
\stopPictList

\startPictList
\VPcrbr
\PictList
Senke sugeenheten
\stopPictList\vskip -3pt

\startPictList
\VPcrtf
\PictList
Åpne grovsmussluken (foran på sugeenheten)
\stopPictList\vskip -3pt

\startPictList
\VPcrbf
\PictList
Stenge grovsmussluken
\stopPictList

\stopcolumns


\subsubsubject{Symboltaster}

\startcolumns

\startSymVpad
\externalfigure[joy:stop]
\SymVpad
\textDescrHead{Stopp} Stoppe aktivert apparat:

1\:× trykkes: 3.\,Deaktivere kosten\crlf
2\:× trykkes: Deaktivere alt
\stopSymVpad

\startSymVpad
\externalfigure[joy:tempomat]
\SymVpad
\textDescrHead{Fartsholder} Innstille fartsholdere på den nåværende hastigheten og aktivere. For å deaktivere, trykk tasten~\textSymb{joy:tempomat} på nytt, eller brems. Bruk sidetastene til å akselerere/saktne.
\stopSymVpad

\startSymVpad
\externalfigure[joy:ftbrs:minus]
\SymVpad
\textDescrHead{Kosthastighet} Redusere rotasjonshastigheten til sidekosten eller frontkosten.
\stopSymVpad

\startSymVpad
\externalfigure[joy:ftbrs:plus]
\SymVpad
\textDescrHead{Kosthastighet} Øke rotasjonshastigheten til sidekosten eller frontkosten.
\stopSymVpad

\startSymVpad
\externalfigure[joy:eng:minus]
\SymVpad
\textDescrHead{Motorturtall} Redusere turtallet til dieselmotoren.
\stopSymVpad

\startSymVpad
\externalfigure[joy:eng:plus]
\SymVpad
\textDescrHead{Motorturtall} Øke turtallet til dieselmotoren.
\stopSymVpad
\columnbreak

\startSymVpad
\externalfigure[joy:suc]
\SymVpad
\textDescrHead{Oppsuging} Aktivere sugesystem: Sugeenhet senkes, turbin og resirkuleringsvannpumpe kobles inn.\note [recyclingwaterpump] \crlf
Trykk stopptasten~\textSymb{joy:stop}, for å deaktivere systemet.
\stopSymVpad

\startSymVpad
\externalfigure[joy:sucbrs]
\SymVpad
\textDescrHead{Feie/Suge}Aktivere suge-/feiesystem: Sugeenhet senkes, sidekoster senkes og posisjoneres, turbin, koster og resirkuleringsvannpumpe kobles inn.\note [recyclingwaterpump] \crlf
Trykk stopptasten~\textSymb{joy:stop}, for å deaktivere systemet.
\stopSymVpad

\footnotetext[recyclingwaterpump]{Ferskvannspumpe kobles også inn, når bryteren~\textBigSymb{temoin_busebalais} står på \aW{Automatisk} (se \in [sec:ctrl:central] på \atpage [sec:ctrl:central]).}
\startSymVpad
\externalfigure[joy:ftbrs:act]
\SymVpad
\textDescrHead{Frontkost aktivert} Aktivere/deaktivere frontkost.
%% NOTE @Andrew: Singular
\stopSymVpad

\startSymVpad
\externalfigure[joy:ftbrs:right]
\SymVpad
\textDescrHead{Frontkost venstre} Dreieretning for arbeid med frontkosten på venstre side (dreieretning: med urviseren).
\stopSymVpad

\startSymVpad
\externalfigure[joy:ftbrs:left]
\SymVpad
\textDescrHead{Frontkost høyre} Dreieretning for arbeid med frontkosten på høyre side (dreieretning: mot urviseren).
\stopSymVpad

\stopcolumns

\stopsection

\stopchapter

\stopcomponent













