% coding: utf-8
\startcomponent c_60_work_s2_095-nb


\startchapter [title={S2 i hverdagen},
reference={chap:using}]

\setups [pagestyle:marginless]


% \placefig[margin][fig:ignition:key]{Clé de contact}
% {\externalfigure [work:ignition:key]}
\startregister[index][chap:using]{Igangsetting}

\startsection [title={Igangkjøring},
reference={sec:using:start}]


\startSteps
\item Sørg for at vanlig kontroll og vedlikehold er blitt utført forskriftsmessig.
\item Start motoren med tenningsnøkkelen: Slå på tenningen, fortsett å vri nøkkelen med urviseren og hold den, helt til motoren starter (kun mulig hvis kjøretrinnvalgspaken står på nøytral).
\stopSteps

\start
\setupcombinations [width=\textwidth]

\placefig[here][fig:select:drive]{Kjøretrinnvalgspak}
{\startcombination [2*1]
{\externalfigure [work:select:fDrive]}{Valgspak i stilling \aW{foroverkjøring}}
{\externalfigure [work:select:rDrive]}{Valgspak i stilling \aW{bakoverkjøring}}
\stopcombination}
\stop


\startSteps [continue]
\item Vri bryteren til kjøretrinnvalgspaken, for å aktivere et kjøretrinn i \aW{kjøre}modusen:
\startitemize [R]
\item Første trinn
\item Andre trinn (automatikkdrift; starter automatisk i første trinn)
\stopitemize

eller trykk på knappen på utsiden av spaken, for å aktivere/deaktivere \aW{arbeids}modusen.
\stopSteps

\startbuffer [work:config]
\starttextbackground [FC]
\startPictPar
\PMrtfm
\PictPar
I arbeidsmodusen er kun det første kjøretrinnet tilgjengelig og motoren turner med 1300\,min\high{\textminus 1}.

Motorturtallet styres ved hjelp av tastene~\textSymb{joy_key_engine_increase} og~\textSymb{joy_key_engine_decrease} multifunksjonspanelet.
\stopPictPar
\stoptextbackground
\stopbuffer

\getbuffer [work:config]

\startSteps [continue]
\item Trykk kjøretrinnvalgspaken oppover og forover (foroverkjøring) eller oppober og bakover (bakoverkjøring). Se bildene oppe.
\item Løsne parkeringsbremsen før du begynner å akselerere.
\stopSteps

\starttextbackground [FC]
\startPictPar
\PMrtfm
\PictPar
{\md Løsne parkeringsbremsen komplett!} Posisjonen til parkeringsbremsespaken overvåkes av en elektronisk sensor: Hvis parkeringsbremsen ikke er fullstendig løsnet, er kjørehastigheten begrenset på 5\,km/h.
\stopPictPar
\stoptextbackground

\startSteps [continue]
\item Trykk langsomt gasspedalen, for å sette kjøretøyet i bevegelse.
\stopSteps


%% NOTE: New text [2014-04-29]:
\subsection [sSec:suctionClap] {Sugekanalluke}

Sugesystemet oppretter en luftstrøm enten fra sugeenheten eller håndsugeslangen (valgfritt) til smussbeholderen.

En luke (\inF[fig:suctionClap], \atpage[fig:suctionClap]) som betjenes for hånd, tillater at luftstrømmen kan kobles om mellom sugeenheten og håndsugeslangen.

\placefig [here] [fig:suctionClap] {Sugekanalluke}
{\startcombination [2*1]
{\externalfigure [work:suctionClap:open]}{Sugekanal åpen}
{\externalfigure [work:suctionClap:closed]}{Sugekanal lukket}
\stopcombination}

I normal drift~– ved arbeider med sugeenheten~– må sugekanalen være åpen (omkoblingsspaken peker oppover).

For å kunne bruke håndsugeslangen, må sugekanalen være lukket (omkoblingsspaken peker nedover). På denne måten ledes luftstrømmen gjennom håndsugeslangen.
%% End new text

\stopsection


\startsection [title={Stansing},
reference={sec:using:stop}]

\index{Driftsstans}

\startSteps
\item Aktiver parkeringsbremsen (spaken mellom setene) og sett kjøretrinnvalgspaken i posisjonen \aW{Nøytral}.
\item Utfør de nødvendige kontrollarbeidene~– daglige og evt. ukentlige kontroller~– som beskrevet på \atpage[table:scheduledaily].
\stopSteps

\getbuffer [prescription:handbrake]

\stopsection


\startsection [title={Feie og suge},
reference={sec:using:work}]

\startSteps
\item Utfør\index{Feie} igangkjøringen som beskrevet i\in{§}[sec:using:start], \atpage[sec:using:start].
\item Aktiver\index{Suge} \aW{arbeids}modusen (knapp på utsiden av kjøretrinnvalgspaken).
\stopSteps

% \getbuffer [work:config]
%% NOTE: outcommented by PB

\startSteps [continue]
\item Trykk på tasten~\textSymb{joy_key_suction_brush}, for å slå på turbinen og kosten.

{\md Variant:} {\lt Trykk på tasten~\textSymb{joy_key_suction}, for å kun arbeide med sugeenheten.}

\item Still inn omdreiningshastigheten til kostene ved hjelp av tastene~\textSymb{joy_key_frontbrush_increase}\textSymb{joy_key_frontbrush_decrease} på multifunksjonspanelet.

\item Sett kostene ved hjelp av de enkelte joystickene slik i posisjon, at den optimale arbeidsbredden blir nådd.
\stopSteps

\vfill

\start
\setupcombinations [width=\textwidth]

\placefig[here][fig:brush:position]{Posisjonering av kostene}
{\startcombination [2*1]
{\externalfigure [work:brushes:enlarge]}{Kost utover/innover}
{\externalfigure [work:brush:left:raise]}{Kost opp/ned}
\stopcombination}
\stop

\page [yes]


\subsubsubject{Fukting av kostene og sugekanalen}

Betjen\index{Feie+fukting} bryteren~\textSymb{temoin_busebalais} mellom setene:

{\md Posisjon 1:} Vannpumpen går automatisk, så lenge kostene er aktivert.

{\md Posisjon 2:} Vannpumpen går permanent. (Nyttig \eG\ for innstillingsarbeider.)


\subsubsubject{Grovsmuss}

\startSteps [continue]
\item Dersom det er fare for at større smussobjekter (\eG\ PET-flasker) kan blokkere sugeenheten, må du åpne\index{Grovsmussluke} grovsmussluken ved hjelp av tastene på siden av multifunksjonspanelet eller~– hvis det ikke rekker~– løft\index{Sugeenhet+grovsmuss} sugeenheten temporært.
\stopSteps

\start
\setupcombinations [width=\textwidth]

\placefig[here][fig:suctionMouth:clap]{Håndtering av grovsmuss}
{\startcombination [2*1]
{\externalfigure [work:suction:open]}{Åpne grovsmussluken}
{\externalfigure [work:suction:raise]}{Løft sugeenheten temporært}
\stopcombination}
\stop

\stopsection


\startsection [title={Tømming av smussbeholderen},
reference={sec:using:container}]

\startSteps
\item Kjør\index{Smussbeholder+tømme} kjøretøyet til et sted som egner seg for tømming. Pass på at de gjeldende miljøvernbestemmelsene overholdes.
\item Aktiver parkeringsbremsen. (Nødvendig for frigivelse av vippe-beholder-bryteren).
\stopSteps

\getbuffer [prescription:container:gravity]

\startSteps [continue]
\item Lås opp og åpne lukkeren til smussbeholderen.
\item Betjen bryteren~\textSymb{temoin_kipp2} (midtre panel, mellom setene), for å vippe opp smussbeholderen.
\item Når beholderen er tom, må du vaske den innvendig med en vannstråle. For dette kan du bruke den integrerte vannpistolen (valgfritt utstyr).
\stopSteps

\start
\setupcombinations [width=\textwidth]
\placefig[here][fig:brush:adjust]{Håndtering av smussbeholderen}
{\startcombination [3*1]
{\externalfigure [container:cover:unlock]}{Låsing av lukkeren}
{\externalfigure [container:safety:unlocked]}{Sikringsstang}
{\externalfigure [container:safety:locked]}{Sikringsstang låst}
\stopcombination}
\stop

\startSteps [continue]
\item Kontroller/rengjør tetningene og kontaktfatene til tetningene til beholderen, resirkuleringssystemet og sugekanalen.
\stopSteps

\getbuffer [prescription:container:tilt]

\startSteps [continue]
\item Betjen bryteren~\textSymb{temoin_kipp2}, for å senke smussbeholderen (Fjern evt. først sikringsstengene fra hydraulikksylindrene).
\item Lås lukkeren til smussbeholderen.
\stopSteps

\stopsection


\startsection [title={Håndsugeslange},
reference={sec:using:suction:hose}]

\sdeux\ kan valgfritt\index{Håndsugeslange} utstyres med en håndsugeslange. Denne er festet på lukkeren til smussbeholderen; betjeningen er enkel.

{\sla Forutsetninger:}

Smussbeholderen er fullstendig senket; \sdeux\ befinner seg i \aW{arbeids}modusen. (Se \in{§}[sec:using:start], \atpage[sec:using:start].)

\startfigtext[left][fig:using:suction:hose]{Håndsugeslange}
{\externalfigure[work:suction:hose]}
\startSteps
\item Trykk tasten~\textSymb{temoin_aspiration_manuelle} til takpanelet, for å aktivere sugesystemet.
\item Trekk fast parkeringsbremsen, før du forlater førerhuset.
\item Lukk sugekanalen med luken til sugekanalen. (Se \in{§}[sSec:suctionClap], \atpage[sSec:suctionClap].)
\item Trekk håndsugeslangen på munnstykket ut av holderen og begynn med arbeidet.
\item Etter at arbeidet er fullført må du igjen trykke tasten~\textSymb{temoin_aspiration_manuelle}, for å slå av sugesystemet.
\stopSteps
\stopfigtext

\stopsection

\page [yes]

\setups[pagestyle:normal]


\startsection [title={Høytrykkvannpistol},
reference={sec:using:water:spray}]

\sdeux\ kan valgfritt\index{Vannpistol} utstyres med en høytrykkvannpistol. Vannpistolen er festet i den bakre vedlikeholdsluken på høyre side og er forbundet med en 10-meter-slangekrok~– på motsatt side av kjøretøyet.

Gå frem som følgende, for å bruke vannpistolen:

{\sla Forutsetninger:}

Det er tilstrekkelig vann i ferskvannstanken; \sdeux\ befinner seg i \aW{arbeids}modusen. (Se \in{§}[sec:using:start], \atpage[sec:using:start].)

\placefig[margin][fig:using:water:spray]{Høytrykkvannpistol}
{\externalfigure[work:water:spray]}

\startSteps
\item Trykk tasten~\textSymb{temoin_buse} til takpanelet, for å aktivere høytrykkvannpumpen.
\item Trekk fast parkeringsbremsen, før du forlater førerhuset.
\item Åpne den bakre vedlikeholdsluken på høyre side og ta ut vannpistolen.
\item Vikle av så mye slange du behøver og begynn med arbeidet.
\item Etter at arbeidet er fullført må du igjen trykke tasten~\textSymb{temoin_buse}, for å slå av høytrykkvannpumpen.
\item Trekk kort på slangen, for å løsne blokkeringen og vikle opp slangen.
\item Fest vannpistolen igjen i holderen og lukk vedlikeholdsluken.
\stopSteps

\stopsection
\stopregister[index][chap:using]

\stopchapter
\stopcomponent

