% coding: utf-8

\startcomponent c_20_prescriptions_s2_095-nb


\chapter [safety:risques] {Sikkerhetsforskrifter}

\setups [pagestyle:marginless]


\section{Grunnleggende instrukser}

\subsubject{Lovmessig grunnlag}

Ulykker kan føre med seg alvorlige konsekvenser, så vel for arbeidsgiveren som for arbeidstakeren. Vi ønsker enda en gang å minne om pliktene som påligger begge parter:\note[prescription:user:right].

Arbeidsgiveren er forpliktet å overholde de følgende punkter før en ansatt tildeles ansvaret for betjeningen av feiemaskinen:

\startSteps
\item Alle førere av kjøretøy må være utdannet til kjøring av det gjeldende kjøretøyet. Det må foreligge bevis for utdannelsen.
\item Alle førere av kjøretøy må disponere et formelt sertifikat. Dette skal kun utstedes, når de følgende tre betingelser er oppfylt:
\startitemize [2]
\item Den ansatte har bestått en medisinsk yrkestest gjennomført av bedriftslegen.
\item Den overordnede skal ha instruert den ansatte slik, at denne er kjent med lokalitetene til arbeidsoppgavene og er fortrolig med alle sikkerhetsforskrifter som vedrører bruksstedet til det gjeldende kjøretøyet.
\item Den ansatte har bestått en yrkestest, som attesterer kunnskapene som er nødvendig for føring av det gjeldende kjøretøyet.
\stopitemize
\stopSteps

Når topphastigheten til kjøretøyet er på mer enn 25\,km/time\note[prescription:user:right], må kjøretøyet være offisielt godkjent og føreren av kjøretøyet må være innehaver av følgende sertifikat:
\startitemize
\item Førerkort klasse B\note[prescription:lisence] for kjøretøy med tillatt totalvekt på mindre enn 3,5~tonn hhv.
\item Førerkort klasse C\note[prescription:lisence] for kjøretøy med tillatt totalvekt på mer enn 3,5~tonn.
\stopitemize

Hvis topphastigheten til kjøretøyet er på 25\,km/h, må føreren av kjøretøyet minst kjenne til trafikkreglene som gjelder på offentlige veier og plasser, også hvis det for føring av kjøretøyet ikke er nødvendig med et førerkort av klasse B\note[prescription:user:right].

\footnotetext [prescription:user:right] {Forpliktelsene til arbeidsgiveren og ansatte kan variere avhengig av landet eller regionen. Gjør deg kjent med forskriftene som gjelder i landet eller regionen.}

\footnotetext[prescription:lisence] {Direktiv 2006/126/EU til Europaparlamentet og rådet av 20.~desember 2006 om førerkortet.}


\subsubject{Brukerbetingelser}

\sdeux\ skal utelukkende brukes hvis den befinner seg feilfri driftstilstand. I tillegg må operatøren overholde sikkerhetshenvisningene og forskriftene som er beskrevet i denne bruksanvisningen. Funksjonsfeil, som reduserer sikkerheten, må omgående utbedres eller fjernes/repareres av en egnet fagbedrift.
\blank [big]

\startSymList
\externalfigure [s2_inspection] [width=4.5em]
\SymList
{\md Daglig vedlikehold:}
La kjøretøyet undergå en inspeksjon etter hver arbeidsinnsats og reparer synlige skader og defekter. I tilfelle av skader eller funksjonsfeil til kjøretøyet må du omgående informere fagverkstedet. Hvis dette ikke er mulig, må du omgående stoppe kjøretøyet og sikre sammenbruddstedet.
\stopSymList


\subsubject{Hensiktsmessig bruk}

\sdeux\ er konsipert for rengjørings- og vedlikeholdsarbeider av gater, veier og plasser. All bruk utenfor denne rammen gjelder som ikke-hensiktsmessig. Som følge av dette tilbakeviser \boschung\ ethvert ansvar for skader som oppstår grunnet av dette. Ved ikke-hensiktsmessig bruk har alene operatøren ansvar for følgene. {\em Hensiktsmessig bruk omfatter også overholdelse av sikkerhetshenvisningene og vedlikeholdsplanen, som er inkludert i denne bruksanvisningen.}


\section{Kjøring på offentlige veier}

\subsubject{Generelle forskrifter}

I tillegg til brukshenvisningene må en overholde alle generelt gjeldende regler, de gjeldende lovbestemmelsene og andre forskrifter og bestemmelser for ulykkesforebyggelse og miljøvern.


\subsubject{Passasjerplass}

Passasjerer~må kun ta plass på et sete som samsvarer og er ment til dette formålet, det såkalte {\em passasjersetet}.


\subsubject{Sikkerhetsbelte}

\startSymList
% \externalfigure [prescription:safety:belt]
\PMbelt
\SymList
Førere og passasjerer til \sdeux\ må, i henhold til de gjeldende trafikkreglene alltid ha på sikkerhetsbelte når de tar plass i kjøretøyet.
\stopSymList


\subsubject{Se og bli sett}

\startSymList
\externalfigure [travaux_deviation] [width=3.5em]
\SymList
Sørg for at du er godt synlig, spesielt på svært trafikkerte veier.

Hvis føreren av kjøretøyet ved et bestemt manøver eller en bestemt aktivitet ikke har tilstrekkelig sikt, må han ha hjelp av en assistent, som han har øyekontakt med.
\stopSymList


\subsubject{Belysning og signalmidler}

Avhengig av trafikkreglene som gjelder må en evt. også på dagen slå på lyskastere og/eller roterende varsellys til kjøretøyet.


\subsubject{Bruk av mobiltelefoner}

\startSymList
\PPphone
\SymList
Bruken av en mobiltelefon eller trådløs enhet mens du kjører på offentlige veier er forbudt med mindre kjøretøyet er utstyrt med en håndfrienhet.

Telefoni\index{Sikkerhet+mobiltelefon} på rattet~– også med håndfrienhet~– svekker konsentrasjonen på veitrafikken.
\stopSymList


\section{Vedlikeholdsforskrifter}

\subsubject{Vedlikeholdsinstrukser}

Vedlikeholdspersonalet må ha lest bruksanvisningen til \sdeux, spesielt avsnittene om sikkerhet og vedlikehold, før de begynner med arbeidet.


\subsubject{Nødvendige kvalifikasjoner}

\startSymList
\externalfigure [mecanicienne] [width=3.5em]
\SymList
Kun personer, som har de nødvendige kunnskapene fra en tilsvarende opplæring, har lov, til å utføre vedlikeholdsarbeider på \sdeux\. Dette gjelder spesielt for arbeider på motoren, bremsesystemet, rattet og elektronikk- og hydraulikkanlegget.
\stopSymList


\testpage [6]
\subsubject{Oppsyn}

\startSymList
\externalfigure [mecanicien_hyerarchie] [width=3.5em]
\SymList
Personer som befinner seg i utdannelse~– praktikum eller under opplæring~– må kun arbeide på kjøretøyet under oppsyn av en fagkyndig person. Gjennomfør stikkprøver for å sjekke at personellet overholder bruksanvisningen og sikkerhetsforskriftene.
\stopSymList


\subsubject{Sveisearbeider}

\startSymList
\externalfigure [pince_soudure2] [width=3.5em]
\SymList
Før gjennomføring av sveisearbeider på karosseriet eller chassisen må batteriet og alle elektroniske styreenheter kobles fra.
\stopSymList

\subsubject{Kjøretøyrens}

\startSymList
\externalfigure [washer_pressure] [width=3.5em]
\SymList
Før rengjøring av \sdeux\ må du lese avsnittet \about[sec:cleaning] fra og med \atpage[sec:cleaning], spesielt avsnittet om rengjøringsinstruksene.
\stopSymList


\subsubject{Tilgjengelighet av kjøretøydokumentasjonen}

\startSymList
\externalfigure [lecteur_1] [width=3.5em]%\PMrtfm
\SymList
Ved innsats må du alltid oppbevare kjøretøydokumentasjonen lett tilgjengelig inne i førerhuset til kjøretøyet.
\stopSymList


\section{Spesielle bruksbestemmelser}

\subsubject{Kjøretøyhøyde}

\startSymList
\PPmaxheight
\SymList
Ved arbeider/kjøring i lukket terreng (underjordiske parkeringshus, underganger, strømlinjer osv.) må du alltid forsikre deg om at takhøyden for \sdeux\ er tilstrekkelig (se \in{avsnitt}[sec:measurement], \atpage[sec:measurement]).
\stopSymList


\subsubject{Stabilitet til kjøretøyet}

Unngå alle manøvre, som kan redusere stabiliteten til kjøretøyet. Ved økt hastighet i svinger kan \sdeux\ velte på grunn av den smale konstruksjonen og økt tyngdepunkt ved full smussbeholder.


\subsubject{Utilsiktet kjøretøybevegelse}

Når du forlater kjøretøyet må du sikre det mot bruken av uvedkommende personer. Du må prinsipielt trekke parkeringsbremsen før du forlater kjøretøyet; sikre hjulene evt. med stoppeklosser.

\startbuffer [prescription:handbrake]
\starttextbackground [CB]
\startPictPar
\PPstop
\PictPar
{\md Trekk parkeringsbremsen godt fest!} Ellers kan kjøretøyet begynne å bevege seg utilsiktet, selv\index{Parkeringsbrems+farepotensial} på veldig små skråninger, og føre til en ulykke med fare for dødelige personskader til tredje.

{\lt Gjennom det hydrostatiske fremdriftssystemet reduseres trykket i hydraulikkretsen trinnvist ved stillstand, noe som fører til reduksjon av motorens holdekraft. Av denne grunn er det spesielt viktig, at parkeringsbremsen alltid trekkes når en forlater kjøretøyet.}
\stopPictPar
\stoptextbackground

\stopbuffer

\getbuffer [prescription:handbrake]


\testpage [6]
\subsubject{Smussbeholder}

\startbuffer [prescription:container:gravity]
\starttextbackground [CB]
\startPictPar
\PHgravite
\PictPar
{\md Ulykkefare:}
{\lt Når smussbeholderen vippes opp fordlees tyngdepunktet oppover. Dette fører til økt fare for velting av kjøretøyet. Derfor må du ved vipping av smussbeholderen passe på at kjøretøyet står på vannrett og fast undergrunn.}
\stopPictPar
\stoptextbackground

\stopbuffer

\getbuffer [prescription:container:gravity]


\startbuffer [prescription:container:tilt]
\starttextbackground [CB]
\startPictPar
\PHcrushing
\PictPar
{\md Ulykkefare:}
{\lt Aldri utfør arbeider under en smussbeholder, før sikkerhetsstengene er montert på de hydrauliske løftesylinderne til smussbeholderen.}
\stopPictPar
\stoptextbackground

\stopbuffer

\getbuffer [prescription:container:tilt]


\stopcomponent

