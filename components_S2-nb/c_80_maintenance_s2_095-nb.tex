% coding: utf-8

\startcomponent c_80_maintenance_s2_095-nb

\startchapter [title={Vedlikehold og service},
reference={chap:maintenance}]

\setups[pagestyle:marginless]


\startsection [title={Generelle henvisninger}]


\subsection{Miljøvern}

\starttextbackground [FC]
\setupparagraphs [PictPar][1][width=2.45em,inner=\hfill]

\startPictPar
\Penvironment
\PictPar
\Boschung\ omsetter miljøvern\index{Miljøvern} i praksis. Vi begynner å fokusere på årsakene og knytter alle effekter som produksjonsprosessen og produktet har på miljøet i bedriftens avgjørelser. Målene er sparsom bruk av ressurser og en skånende omgang med de naturlige levekårene, som hjelper menneske og natur når de opprettholdes. Ved å overholde bestemte regler ved drift av kjøretøyet støtter du miljøvernet. Til dette hører også egnet og forskriftsmessig omgang med stoffer og materialer innenfor rammen av kjøretøyvedlikeholdet (\eG\ avfallshåndtering av kjemikalier og spesialavfall).

Drivstofforbruk og slitasje til motoren er avhengig av driftsforholdene. Derfor ber vi deg om å ta hensyn til noen punkter:

\startitemize
\item Ikke la motoren kjøre seg varm i tomgang.
\item Slå av motoren ved driftsbetingede ventetider.
\item Kontroller drivstofforbruket i regelmessige mellomrom.
\item {\em Sørg for at vedlikeholdsarbeidene utføres iht. vedlikeholdsplanen og kun av et kompetent fagverksted.}
\stopitemize
\stopSymList
\stoptextbackground

\page [yes]


\subsection{Sikkerhetsforskrifter}

\startSymList
\PHgeneric
\SymList
For\index{Vedlikehold+Sikkerhetsforskrifter} å forhindre skader på kjøretøy og aggregater samt ulykker under vedlikeholdet, er det strengt nødvendig, at sikkerhetsforskriftene overholdes. Følg også de generelle sikkerhetsforskriftene (\about[safety:risques], \at{f.o.m. side}[safety:risques]).
\stopSymList

\starttextbackground [FC]
\startPictPar
\PMgeneric
\PictPar
\textDescrHead{Ulykkesforebyggelse}
Kontroller\index{Ulykkesforebyggelse} tilstanden til kjøretøyet etter alle vedlikeholds- eller reparasjonsarbeider. Vær spesielt oppmerksom på at alle sikkerhetsrelevante komponenter samt belysnings- og signalinnretninger fungerer feilfritt, før du kjører på offentlige trafikkveier.
\stopPictPar
\stoptextbackground
\blank [big]

\start
\setupparagraphs [SymList][1][width=6em,inner=\hfill]
\startSymList\PHcrushing\PHfalling\SymList
\textDescrHead{Stabilisering av kjøretøyet}
Før vedlikeholdsarbeider må kjøretøyet sikres mot utilsiktede kjøretøybevegelser: Still kjøretrinnvalgspaken på \aW{Nøytral}, aktiver parkeringsbremsen og sikre kjøretøyet med stoppeklosser.
\stopSymList
\stop

\starttextbackground[CB]
\startPictPar\PHpoison\PictPar
\textDescrHead{Start av motoren}
Når \index{Fare+forgiftning} motoren må startes på et sted med dårlig ventilasjon, ikke la den gå lengre enn strengt nødvendig\index{Fare+avgasser}, for å forhindre kullosforgiftninger.
\stopPictPar
\startitemize
\item Start motoren kun med forskriftsmessig tilkoblet batteri.
\item Aldri koble fra batteriet mens motoren er igang.
\item Ikke bruk starthjelp for å starte motoren.
Når\index{Batteri+ladeenhet} batteriet skal lades med en hurtiglader, må det først kobles fra kjøretøyet. Følg driftsforskriftene til hurtigladeren.
\stopitemize
\stoptextbackground

\page [yes]


\subsubsection{Beskyttelse av elektroniske komponenter}

\startitemize
\item Før\index{Elektrosveising} du begynner sveisearbeidene, koble batterikablene fra batteriet og koble sammen pluss- og jordingskabelen.
\item Koble\index{Elektronikk} de elektroniske styreenheter til og fra kun når de ikke står under spenning.
\item Feil\index{Styreenhet} polaritet i strømforsyningen (\eG\ grunnet feil tilkobling av batterier) kan ødelegge elektroniske komponenter og enheter.
\item Ved\index{Omgivelsestemperatur+ekstreme} omgivelsestemperaturer på over 80 °C (\eG\ i et tørkekammer) må elektroniske komponenter/enheter fjernes.
\stopitemize


\subsubsection{Diagnose og målinger}

\startitemize
\item Til måle- og diagnosearbeider skal det kun anvendes {\em egnede} testkabler (\eG\ originalkablene til enheten).
\item Mobiltelefoner\index{Mobiltelefon}, og liknende radioapparater, kan påvirke funksjonene til kjøretøyet, diagnoseapparatet og driftssikkerheten negativt.
\stopitemize


%%%%%%%%%%%%%%%%%%%%%%%%%%%%%%%%%%%%%%%%%%%%%%%%%%%%%%%%%%%%%%%%%%%%%%%%%%%%%%%%%%%%%%%%%

\subsubsection{Kvalifisering av personalet}

\starttextbackground[CB]
\startPictPar
\PHgeneric
\PictPar
\textDescrHead{Ulykkefare}
Ved\index{Kvalifikasjon+Vedlikeholdspersonal} feil utførelse av vedlikeholdsarbeider kan funksjonsevnen og sikkerheten til kjøretøyet påvirkes negativt. Dette fører til økt fare for ulykker og personskader.

For\index{Kvalifikasjon+Verksted} utførelse av vedlikeholds- og reparasjonsarbeider må du ta kontakt med et kvalifisert fagverksted, som har de nødvendige kunnskaper og verktøy.

I tvilstilfeller ta kontakt med \Boschung-kundeservice.
\stopPictPar
\stoptextbackground

% \page [yes]

Betjening, vedlikehold og reparasjon av \ProductId får utelukkende utføres av personale som er kvalifisert og utdannet av \Boschung-kundeservice.

Kompetansen for betjening, vedlikehold og reparasjon tildeles av \Boschung-kundeservice.

%\adaptlayout [height=+5mm]


\subsubsection{Endringer og ombygginger}

\starttextbackground[CB]
\startPictPar
\PHgeneric
\PictPar
\textDescrHead{Ulykkefare}
Alle\index{Endring på kjøretøyet} endringer, som du selv foretar på kjøretøyet, kan redusere funksjonsevnen og driftssikkerheten til \ProductId og dermed føre til fare ulykker og personskader.
\stopPictPar

\startPictPar
\PMwarranty
\PictPar
For skader, som oppstår grunnet\index{Garanti+Betingelser} egne endringer eller modifikasjoner på \ProductId eller et aggregat, overtar \Boschung\ ingen garanti eller Kulans.
\stopPictPar
\stoptextbackground

\stopsection


\startsection [title={Driftsstoffer og smøremidler}, reference={sec:liquids}]


\subsection{Riktig omgang}

\starttextbackground[CB]
\startPictPar
\PHpoison
\PictPar
\textDescrHead{Fare for personskader og forgiftning}
Gjennom\index{Driftsstoff} hudkontakt\index{smøremiddel} eller\index{Fare+forgiftning} svelging av driftsstoffer og smøremidler kan\index{Driftsstoff+sikkerhet} det føre til alvorlige personskader eller forgiftninger. Følg alltid de lovbestemte forskriftene for håndtering, lagring og avfallshåndtering av disse stoffene.
\stopPictPar
\stoptextbackground

\starttextbackground [FC]
\startPictPar
\PMproteyes\par
\PMprothands
\PictPar
Ved omgang med drifts- og smøremidler må du alltid bruke egnet verneklær og vernemaske. Unngå innånding av damp.
Unngå all kontakt med hud, øyner eller klær. Rengjør stedet til huden, som har kommet i kontakt med driftsstoffer, omgående med vann og såpe. Hvis driftsstoffer når inn i øynene, må du skylle godt med kaldt vann og evt. oppsøke øyelege. Etter svelging av driftsstoffer må en omgående oppsøke en lege!
\stopPictPar
\stoptextbackground

\pagebreak


\startSymList
\PPchildren
\SymList
Driftsmidler må oppbevare utilgjengelig for barn.
\stopSymList

\startSymList
\PPfire
\SymList
\textDescrHead{Brannfare}
På grunn av\index{Fare+Brann} driftsstoffers høye antennelighet består det økt fare for brann ved omgangen med disse. Røyking, ild\index{Røykeforbud} og direkte lys er strengt forbudt ved omgangen med driftsstoffer.
\stopSymList


%% TODO; en
\starttextbackground [FC]
\startPictPar
\PMgeneric
\PictPar
Det er kun tillatt å bruke smøremidler, som er egnet for de i \ProductId anvendte komponentene. Bruk derfor kun produkter som \Boschung\ har godkjent og frigitt. Disse finner du i driftsstofflisten \atpage[sec:liqquantities]. Additiver\index{Additiver} for smøremidler er ikke nødvendig. Dersom du tilsetter additiver, kan det føre til tap av garantikrav\index{Garanti+betingelser}.
For mer informasjon, ta kontakt med \Boschung-kundeservice.
\stopPictPar
\stoptextbackground

\starttextbackground [FC]
\startPictPar
\Penvironment
\PictPar
\textDescrHead{Miljøvern}
Ved\index{Smøremiddel+avfallshåndtering} avfallshåndtering av drifts- og smøremidler\index{Miljøvern} eller gjenstander, som er forurenset med det (\eG\ filtre, kluter), må du passe på å\index{Driftsstoffer+avfallshåndtering} overholde miljøvernbestemmelsene.
\stopPictPar
\stoptextbackground

\page [yes]

\setups [pagestyle:normal]


\subsection[sec:liqquantities]{Spesifikasjoner og fyllemengder}

Samtlige\index{Driftsstoffer+fyllemengde}\index{Smøremiddel+fyllemengde}\index{Fyllemengder+Driftsstoffer og smøremidler}\index{spesifikasjoner+Driftsstoffer og smøremidler} som er angitt i følgende tabell er referanseverdier. Etter hvert skifte av driftsstoff eller smøremiddel må det faktiske fyllenivået kontrolleres og evt. må fyllenivået økes eller reduseres.
% \blank[big]

\placetable[margin][tab:glyco]{Frostbeskyttelse (\index{frostbeskyttelse}motor)}
{\noteF\startframedcontent[FrTabulate]
%\starttabulate[|Bp(80pt)|r|r|]
\starttabulate[|Bp|r|r|]
\NC Frostbeskyttelse opp til {[}°C{]}\NC \bf \textminus 25 \NC \bf \textminus 40 \NC\NR
\NC Destill. vann [Vol.-\%] \NC 60 \NC 40 \NC\NR
\NC Frostvæske \break [Vol.-\%] \NC 40 \NC {\em maks.} 60 \NC\NR
\stoptabulate\stopframedcontent\endgraf
Obs: Ved en volumandel på mer enn 60\hairspace\percent\ Frostvæske {\em synker} graden av frostbeskyttelse og kjøleeffekten forringes!}

\placefig[margin][fig:hydrgauge]{\select{caption}{Nivåvisning hydraulikkvæske (venstre side av kjøretøyet)}{Nivåvisning hydraulikkvæske}}
{\externalfigure[main:hy:level_temp]
\noteF Fyllenivået til hydraulikkvæsken kan leses av på seglasset og skal kontrolleres {\em daglig}.}

%\placetable[here,split][tbl:liquids]{Spécifications et volumes de remplissage des consommables}
%{\readfile{tbl_jb-fr_liquids}{}{\Warn}}


\vskip -8pt
\start
\define [1] \TableSmallSymb {\externalfigure[#1][height=4ex]}
\define\UC\emptY
\pagereference[page:table:liquids]

\setupTABLE [frame=off,style={\ssx\setupinterlinespace[line=.86\lH]},background=color,
option=stretch,
split=repeat]
\setupTABLE [r] [each] [topframe=on,
framecolor=TableWhite,
% rulethickness=.8pt
]

\setupTABLE [c] [odd] [backgroundcolor=TableMiddle]
\setupTABLE [c] [even] [backgroundcolor=TableLight]
\setupTABLE [c] [1][width=30mm]
\setupTABLE [c] [2][width=20mm]
\setupTABLE [c] [4][width=25mm]
\setupTABLE [c] [last] [width=10mm]
\setupTABLE [r] [first] [topframe=off,style={\bfx\setupinterlinespace[line=.95\lH]},
% backgroundcolor=TableDark
]
\setupTABLE [r] [2][framecolor=black]

\bTABLE

\bTABLEhead
\bTR
\bTC Gruppe \eTC
\bTC Kategori \eTC
\bTC Klassifisering \eTC
\bTC Produkt\note[Produkt] \eTC
\bTC Mengde \eTC
\eTR
\eTABLEhead

\bTABLEbody
\bTR \bTD Dieselmotor \eTD
\bTD Motorolje\eTD
\bTD \liqC{SAE 5W-30}; \liqC{VW 507.00}\eTD
\bTD Total Quartz INEO Long Life \eTD
\bTD 4,3\,l\eTD
\eTR
\bTR \bTD Hydraulikkrets \eTD
\bTD ATF-Öl \eTD
\bTD \liqC{dexron iii} \eTD
\bTD Total Equiviz ZS 46 (tank ca. 40 l) \eTD
\bTD ca. 50\,l\eTD
\eTR
\bTR \bTD Hydraulikkrets (valgfritt \aW{Bio})\eTD
\bTD ATF-Öl \eTD
\bTD \liqC{dexron iii} \eTD
\bTD Total Biohydran TMP SE 46 (tank ca. 45 l) \eTD
\bTD ca. 60\,l\eTD
\eTR
\bTR \bTD Magnetventil: Spolekjerner \eTD
\bTD Smøremiddel\eTD
\bTD Kobberfett \eTD
\bTD \emptY\eTD
\bTD n. B.\note[Bedarf] \eTD
\eTR
\bTR \bTD Diverse: Låser, dørmekanikk, bremsepedal \eTD
\bTD Smøremiddel\eTD
\bTD Universal-spray\eTD
\bTD \emptY\eTD
\bTD n. B.\note[Bedarf] \eTD
\eTR
\bTR \bTD Sentralsmøreanlegg \eTD
\bTD Universal-lagerfett\eTD
\bTD \liqC{nlgi} 000 eller 00 (Li-sidefett)\eTD
\bTD Total Multis EP 00\eTD
\bTD n. B.\note[Bedarf] \eTD
\eTR
\bTR \bTD Kjølesystem \eTD
\bTD Frost-/rustbeskyttelsesmiddel\eTD
\bTD TL VW 774 F/G; maks. 60\hairspace\% vol.\eTD
\bTD G12+/G12++ (rosa/fiolett)\eTD
\bTD ca. 14\,l \eTD
\eTR
\bTR \bTD Høytrykksvannpumpe \eTD
\bTD Motorolje\eTD
\bTD \liqC{SAE 10W-40}; \liqC{api cf – acea e6}\eTD
\bTD Total Rubia TIR 8900 \eTD
\bTD ca. 0,29\,l\eTD
\eTR
\bTR \bTD Klimaanlegg \eTD
\bTD Kuldemiddel\eTD
\bTD + 20 ml POE-olje\eTD
\bTD R 134a\eTD
\bTD 700\,g\eTD
\eTR
\bTR \bTD Vindusspyleranlegg \eTD
\bTD [nc=2] Vann og vindusspylerkonsentrat, \aW{S} Sommer, \aW{W} Vinter; vær oppmerksom på blandeforholdet \eTD
\bTD Detaljhandel \eTD
\bTD n. B.\note[Bedarf] \eTD
\eTR
\eTABLEbody

\eTABLE

\stop
% \setups[pagestyle:normal]
\footnotetext[Bedarf]{{\it n. B.} etter behov, iht. gjeldende anvisning}
\footnotetext[Produkt]{Av \Boschung\ anvendte produkter. Andre produkter, som tilsvarer spesifikasjonene, kan også anvendes.}

\stopsection

\page [yes]

\setups [pagestyle:marginless]


\startsection [title={Vedlikehold av dieselmotoren},
reference={sec:workshop:vw},
]


\subsection [sSec:vw:diagTool]{On-board-diagnosesystem}

Motorstyreenheten\startregister[index][reg:main:vw]{Vedlikehold+dieselmotor} (J623) er utstyrt med et feilminne.
Når det opptrer feil i de sensorene hhv. komponentene som overvåkes, blir disse lagret i feilminnet med angivelse av feiltypen.

Motorstyreenheten\index{Dieselmotor+diagnose} vil etter analyse av informasjonen skille mellom de ulike feilklassene og lagrer disse frem til feilminneinnholdet slettes.

Feil som kun opptrer {\em sporadisk}, vises med tillegget \aW{SP}. Årsaken til sporadiske feil kan \eG\ være en løs kontakt eller en korttids ledningsavbrytelse. Når en sporadisk feil ikke lenger opptrer innenfor 50 motorstartprosedyrer, blir den slettet fra feilminnet.

Dersom det er registrert feil som påvirker motorens driftsadferd, vil det på skjermen til Vpad-en lyse opp kontrollsymbolet \aW{Motordiagnose} \textSymb{vpadWarningEngine1}.

De lagrede feilene kan leses ut med kjøretøy-diagnose, -måle og -informasjonssystemet \aW{VAS 5051/B}.

Etter at feilen hhv. feilene er blitt utbedret, må feilminnet slettes.


\subsubsection[sSec:vw:diagTool:connect]{Igangsetting av diagnosesystemet}

\starttextbackground [FC]
\startPictPar
\PMgeneric
\PictPar
Detaljert informasjon om kjøretøydiagnosesystem VAS 5051/B finner du i bruksanvisningen for systemet.

Du kan også bruke andre kompatible diagnosesystemer, \eG\ \aW{DiagRA}.
\stopPictPar
\stoptextbackground

\page [yes]


\subsubsubsubject{Forutsetninger}

\startitemize
\item Sikringene må være i orden.
\item Batterispenningen må være på mer enn 11,5 V.
\item Alle elektriske forbrukere må være utkoblet.
\item Jordforbindelsen må være i orden.
\stopitemize


\subsubsubsubject{Fremgangsmåte}

\startSteps
\item Sett i kontakten til diagnoseledningen VAS 5051B/1 på diagnosetilkoblingen.
\item Avhengig av funksjon, enten koble inn tenningen eller start motoren.
\stopSteps

\subsubsubsubject{Velg driftsmåte}

\startSteps [continue]
\item Trykk på bryterflaten \aW{Kjøretøy-selvdiagnose} på displayet.
\stopSteps


\subsubsubsubject{Velge kjøretøysystem}

\startSteps [continue]
\item Trykk på bryterflaten \aW{01-Motorelektronikk} på displayet.
\stopSteps

På displayet vises nå styreenhet-identifikasjonen og kodingen til motorstyreenheten.

Dersom kodingen ikke stemmer overens, må styreenhets-kodingen kontrolleres.


\subsubsubsubject{Velge diagnosefunksjon}

På displayet vises alle de mulige diagnosefunksjonene.

\startSteps [continue]
\item Trykk på bryterflaten for ønsket funksjon på displayet.
\stopSteps



\subsection [sSec:vw:faultMemory]{Feilminne}


\subsubsection{Lese ut feilminne}

\subsubsubject{Arbeidsforløp}

\startSteps
\item La motoren løpe i tomgang.
\item Koble til VAS 5051/B (se \in{avsnitt}[sSec:vw:diagTool:connect]) og velg motorstyreenheten.
\item Velg diagnosefunksjonen \aW{004-Feilminneinnhold}.
\item Velg diagnosefunksjonen \aW{004.01-Utspørre feilminne}.
\stopSteps

{\sla Kun når motoren ikke starter:}

\startitemize [2]
\item Slå på tenningen.
\item Hvis det ikke er lagret noen feil i motorstyreenheten, viser displayet \aW{0 feil registrert}.
\item Hvis det er lagret feil i motorstyreenheten, blir disse listet opp under hverandre i displayet.
\item Avslutt diagnosefunksjonen.
\item Slå av tenningen.
\item Utbedre evt. de viste feilene ved hjelp av feiltabellen (se servicedokumentasjon) og slett deretter feilminnet.
\stopitemize

\starttextbackground [FC]
\startPictPar
\PMrtfm
\PictPar
Når en feil ikke lar seg slette, ta kontakt med \boschung-kundeservice.
\stopPictPar
\stoptextbackground


\subsubsubject{Statiske feil}

Dersom det foreligger en eller flere statiske feil i feilminnet, ta kontakt med Boschung-kundeservice, for å utbedre disse feil ved hjelp av \aW{Veiledet feilsøking}.


\subsubsubject{Sporadiske feil}

Dersom det i feilminnet utelukkende er lagret sporadiske feil og henvisninger, og det ikke konstateres noen feilfunksjoner i kjøretøysystemet, kan feilminnet slettes:

\startSteps [continue]
\item Trykk en gang til på tasten \aW{Videre} \inframed[strut=local]{>}, for å komme til testskjemaet.
\item For å avslutte veiledet feilsøking, trykk på tasten \aW{Sprang} og så \aW{Avslutte}.
\stopSteps

Nå utspørres alle feilminner en gang til.

I et vindu bekreftes det, at alle sporadiske feil er blitt slettet. Diagnoseprotokollen sendes automatisk (online).

Dermed er kjøretøytesten avsluttet.


\subsubsection[sSec:vw:faultMemory:errase]{Sletting av feilminnet}

\subsubsubject{Arbeidsforløp}

{\sla Forutsetninger:}

\startitemize [2]
\item Alle feil må være utbedret og feilårsakene fjernet.
\stopitemize

\page [yes]


{\sla Fremgangsmåte:}

\starttextbackground [FC]
\startPictPar
\PMrtfm
\PictPar
Etter utbedringen av feil må feilminnet utspørres på nytt og deretter slettes:
\stopPictPar
\stoptextbackground

\startSteps
\item La motoren løpe i tomgang.
\item Koble til VAS 5051/B (se \in{avsnitt}[sSec:vw:diagTool:connect]) og velg motorstyreenheten.
\item Velg diagnosefunksjonen \aW{004-Utspørre feilminne}.
\item Velg diagnosefunksjonen \aW{004.10-Slette feilminne}.
\stopSteps

\starttextbackground [FC]
\startPictPar
\PMrtfm
\PictPar
Når feilminnet ikke lar seg slette, foreligger det fortsatt en feil som må utbedres hhv. fjernes.
\stopPictPar
\stoptextbackground

\startSteps [continue]
\item Avslutt diagnosefunksjonen.
\item Slå av tenningen.
\stopSteps


\subsection [sSec:vw:lub] {Smøring av dieselmotoren}

\subsubsection [ssSec:vw:oilLevel] {Kontroller motoroljenivå}

\starttextbackground [FC]
\startPictPar
\PMrtfm
\PictPar
Motoroljenivået\index{Motorolje+-nivå} skal under ingen omstendigheter overskride \aW{Maks.}-markeringen. I motsatt fall består det\index{Fyllenivå+motorolje} fare for katalysatorskader.
\stopPictPar
\stoptextbackground

\startSteps
\item Slå av motoren og vent minst 3 minutter, slik at oljen kan renne tilbake i oljepannen.
\item Trekk ut målepinnen og tørk den ren; sett igjen inn målepinnen frem til anslaget.
\item Trekk målepinnen ut igjen og les av oljenivået:

\startfigtext[right][fig:vw:gauge]{Avlesning av oljenivået}
{\externalfigure[VW_Oil_Gauge][width=50mm]}
\startitemize [A]
\item Maksimalt fyllenivå; det skal ikke etterfylles noen olje.
\item Tilstrekkelig fyllenivå; det {\em kan} etterfylles olje frem til markeringen \aW{A} nås.
\item Ikke tilstrekkelig fyllenivå; det {\em må} etterfylles olje, frem til fyllenivået befinner seg i området \aW{B}.
\stopitemize
{\em Ved et fyllenivå over markeringen \aW{A} består det fare for katalysatorskader.}
\item Ved et fyllenivå under markeringen \aW{C} fyll på motorolje opp til markeringen \aW{A}.
\stopfigtext
\stopSteps


\subsubsection [ssSec:vw:oilDraining] {Motoroljeskift}

\starttextbackground [FC]
\startPictPar
\PMrtfm
\PictPar
Motoroljefilteret til S2 er montert stående. Det betyr, at filteret må skiftes ut {\em før} oljeskiftet. Når filterelementet tas ut, åpnes det automatisk en ventil, og oljen i filterhuset renner automatisk inn i veivhuset.
\stopPictPar
\stoptextbackground

\startSteps
\item Plasser en egnet\index{Dieselmotor+oljeskift} oppsamlingsbeholder under motoren.
\item Skru ut oljetappeskruen\index{Motorolje+-skift} og la oljen tappes av.
\stopSteps

\starttextbackground [FC]
\startPictPar
\PMrtfm
\PictPar
Pass på, at hele den brukte oljemengden kan fanges opp av oppsamlingsbeholderen.
Den nødvendige oljespesifikasjonen og fyllemengden finner du i \in{avsnitt}[sec:liqquantities].

Oljetappeskruen er forsynt med en integrert tetningsring. Oljetappeskruen må derfor alltid skiftes ut
\stopPictPar
\stoptextbackground

\startSteps [continue]
\item Skru inn en ny oljetappeskrue med tetningsring (\TorqueR 30 Nm).
\item Fyll på motorolje med egnet spesifikasjon (se \in{avsnitt}[sec:liqquantities]).
\stopSteps


\subsubsection [ssSec:vw:oilFilter] {Skifte motoroljefilter}

\starttextbackground [FC]
\startPictPar
\PMrtfm
\PictPar
\startitemize [1]
\item Overhold\index{Dieselmotor+oljefilter} forskriftene for avfallshåndtering og resirkulering.
\item Skift ut\index{Oljefilter+dieselmotor} filteret {\em før} oljeskiftet (se \in{avsnitt}[ssSec:vw:oilDraining]).
\item Før montering, smør inn tetningen til det nye filteret.
\stopitemize
\stopPictPar
\stoptextbackground

\startfigtext[right][fig:vw:oilFilter]{Oljefilter}
{\externalfigure[VW_OilFilter_03][width=50mm]}
\startSteps
Skru av \item lokket \Lone\ til filterhuset med en egnet skiftenøkkel.
\item Rengjør tetningsflatene til lokket og filterhuset.
\item Skift ut filterelementet \Lthree.
\item Skift ut O-ringene \Ltwo\ og \Lfour.
\item Skru lokket på filterhuset igjen (\TorqueR 25 Nm).
\stopSteps

\stopfigtext



\subsubsection [ssSec:vw:oilreplenish] {Etterfylle motorolje}

\starttextbackground [FC]
\startPictPar
\PMrtfm
\PictPar
\startitemize [1]
\item Rengjør\index{Motorolje} påfyllingsstussen med en klut {\em før} lokket tas av.
\item Etterfyll\index{Dieselmotor+etterfylle olje} utelukkende olje som tilsvarer den foreskrevne spesifikasjonen.
\item Etterfyll trinnvist i små mengder.
\item For å unngå overfylling, vent litt etter hver etterfylling, slik at oljen kan renne ned i motoroljepannen og frem til markeringen til målepinnen (se \in{avsnitt}[ssSec:vw:oilLevel]).
\stopitemize
\stopPictPar
\stoptextbackground

\startfigtext[right][fig:vw:oilFilter]{ Etterfylle olje}
{\externalfigure[s2_bouchonRemplissage][width=50mm]}
\startSteps
\item Trekk oljemålepinnen ca. 10 cm ut, slik at luft kan trenge ut under etterfyllingen.
\item Åpne påfyllingsåpningen.
\item Etterfyll olje under hensyntagen til de ovennevnte forskriftene.
\item Lukk igjen påfyllingsåpningen korrekt.
\item Start motoren.
\item Gjennomfør en kontroll av fyllenivået. (Se \in{avsnitt}[ssSec:vw:oilLevel].)
\stopSteps

\stopfigtext


\subsection [sSec:vw:fuel] {Drivstofftilførselssystem}

\subsubsection [ssSec:vw:fuelFilter] {Skifte drivstoffilter}

\starttextbackground [FC]
\startPictPar
\PMrtfm
\PictPar
\startitemize [1]
\item Overhold\index{Dieselmotor+drivstoffilter} de lovbestemte forskriftene for avfallshåndtering og resirkulering av spesialavfall.
\item Ikke ta av drivstoffledningen fra overdelen til filteret.
\item Festepunktene til drivstoffledningen må ikke belastes; i motsatt fall kan det føre til skader på overdelen til filteret.
\stopitemize
\stopPictPar
\stoptextbackground

\startfigtext[right][fig:vw:oilFilter]{Drivstoffilter}
{\externalfigure[s2_fuelFilter_location][width=50mm]}

{\sla Forberedelse:}

Drivstoffilterhuset\index{Drivstoffilter} er festet foran motoren, på høyre siden av chassiset.
Fjern begge festeskruene ved hjelp av en 10-mm-skiftenøkkel og en 10-mm-stjernenøkkel.

\stopfigtext


\page [yes]

\setups [pagestyle:normal]

{\sla Fremgangsmåte:}

\startLongsteps
\item Fjern alle skruene til filter-overdelen. Ta av filter-overdelen.
\stopLongsteps

\starttextbackground [FC]
\startPictPar
\PMrtfm
\PictPar
Løft av overdelen. Dersom nødvendig, sett an en vinkelskrutrekker på monteringssporet (\in{\LAa, fig.}[fig:fuelfilter:detach]) og jekk overdelen av.
\stopPictPar
\stoptextbackground

\placefig [margin] [fig:fuelfilter:detach]{Uttak av drivstoffilteret}
{\externalfigure[fuelfilter:detach]}

\placefig [margin] [fig:fuelfilter:explosion]{Drivstoffilter}
{\externalfigure[fuelfilter:explosion]}

\startLongsteps [continue]
\item Trekk filterelementet ut av underdelen til filteret.
\item Ta av tetningen (\in{\Ltwo, fig.}[fig:fuelfilter:explosion]) fra overdelen til filteret.
\item Rengjør under- og overdelen til filteret nøye.
\item Sett et nytt filterelement inn i underdelen til filteret.
\item Fukt en ny tetning (\in{\Ltwo, fig.}[fig:fuelfilter:explosion]) med litt drivstoff og sett det inn overdelen.
\item Sett overdelen passende på underdelen til filteret og trykk den jevnt fast, slik at overdelen ligger jevnt an på alle sider.
\item Skru over- og underdelen sammen igjen med alle skruene {\em for hånd}. Trekk så til alle skruene over kryss med foreskrevet tiltrekkingsmoment (\TorqueR 5 Nm).
\stopLongsteps

% \subsubsubject{Données techniques}
%
% \hangDescr{Couple de serrage des vis de fixation du couvercle:} \TorqueR 5 Nm.
%% NOTE: redundant [tf]

\startLongsteps [continue]
\item Koble inn tenningen, for å lufte ut systemet; start motor og la den gå i 1 til 2 minutter med tomgangsturtall.
\item Slett feilminnet som beskrevet på \atpage[sSec:vw:faultMemory:errase].
\stopLongsteps


\subsection [sSec:vw:cooling] {Kjølesystem}

\starttextbackground [FC]
\startPictPar
\PMrtfm
\PictPar
\startitemize [1]
\item Bruk\index{Dieselmotor+kjøling} kun kjølemiddel iht. foreskrevet spesifikasjon (se tabell \atpage[sec:liqquantities]).
\item For\index{Kjølemiddel} sikre frost- og korrosjonsbeskyttelse, skal kjølemiddelet utelukkende fortynnes med destillert vann og iht. den nedenstående tabellen.
\item Kjølemiddelkretsen skal aldri fylles opp med vann, da dette vil forringe frost- og korrosjonsbeskyttelsen.
\stopitemize
\stopPictPar
\stoptextbackground


\subsubsection [sSec:vw:coolingLevel] {Kjølemiddelnivå}

\placefig [margin] [fig:coolant:level] {Kjølemiddelnivå}
{\externalfigure[coolant:level]}


\placefig [margin] [fig:refractometer] {Refraktometer VW T 10007}
{\externalfigure[coolant:refractometer]}

\placefig [margin] [fig:antifreeze] {Kontroll av frostbeskyttelsestetthet}
{\externalfigure[coolant:antifreeze]}


\startSteps
\item Løft smussbeholderen og de bringer sikkerheten støtte til.
\item Les av\index{Fyllenivå+kjølemiddel} fyllenivået til kjølemiddelet i ekspansjonsbeholderen: Det skal befinne seg ovenfor \aW{min}-markeringen.
\stopSteps

\start
\define [1] \TableSmallSymb {\externalfigure[#1][height=4ex]}
\define\UC\emptY
\pagereference[page:table:liquids]


\setupTABLE [frame=off,style={\ssx\setupinterlinespace[line=.86\lH]},background=color,
option=stretch,
split=repeat]
\setupTABLE [r] [each] [topframe=on,
framecolor=TableWhite,
% rulethickness=.8pt
]

\setupTABLE [c] [odd] [backgroundcolor=TableMiddle]
\setupTABLE [c] [even] [backgroundcolor=TableLight]
\setupTABLE [r] [first] [topframe=off,style={\bfx\setupinterlinespace[line=.95\lH]},
% backgroundcolor=TableDark
]
\setupTABLE [r] [2][framecolor=black]

\bTABLE

\bTABLEhead
\bTR
\bTC Frostbeskyttelse inntil … \eTC
\bTC Andel G12\hairspace ++\eTC
\bTC Vol. frostbeskyttelsesmiddel \eTC
\bTC Vol. destillert vann \eTC
\eTR
\eTABLEhead

\bTABLEbody
\bTR \bTD \textminus 25 °C \eTD
\bTD 40\hairspace\% \eTD
\bTD 3,8 l \eTD
\bTD 4,2 l \eTD
\eTR
\bTR \bTD \textminus 35 °C \eTD
\bTD 50\hairspace\% \eTD
\bTD 4,0 l \eTD
\bTD 4,0 l \eTD
\eTR
\bTR \bTD \textminus 40 °C \eTD
\bTD 60\hairspace\% \eTD
\bTD 4,2 l \eTD
\bTD 3,8 l \eTD
\eTR
\eTABLEbody

\eTABLE
\stop

\adaptlayout [height=+20pt]
\subsubsection [sSec:vw:coolingFreeze] {Kjølemiddelnivå}

Kontroller\index{Frostbeskyttelsestettheten} frostbeskyttelsestettheten ved hjelp av et egnet refraktometer (se \in{fig.}[fig:refractometer]: VW T 10007).
Vær oppmerksom på skala 1: G12\hairspace ++ (se \in{fig.}[fig:antifreeze]).

\page [yes]


\subsection [sSec:vw:airFilter] {Lufttilførsel}

Luftfilteret er tilgjengelig over den bakre vedlikeholdsluken på høyre side av kjøretøyet (se \in{fig.}[fig:airFilter]).

\placefig [margin] [fig:airFilter] {Luftfilteret til motoren}
{\externalfigure[vw:air:filter]
\noteF
\startLeg
\item Sikkerhetslask
\item Underdelen til huset
\item Ventilasjonsåpning
\item Trykksensor
\stopLeg}


\subsubsubject{Innsatsforhold}

Et feiekjøretøy benyttes ofte i omgivelser med sterk støvkonsentrasjon. Derfor er det nødvendig at luftfilteret kontrolleres og rengjøres ukentlig. Se også \about[table:scheduleweekly], \atpage[table:scheduleweekly]. Når det kreves, må luftfilteret skiftes ut.


\subsubsubject{Autodiagnostic}

Sugeledningen er utstyrt med en trykksensor (\in{\Lfour, fig.}[fig:airFilter]), som kan benyttes til å påvise ladetap\footnote{Forringet luftgjennomstrømning grunnet forringet luftgjennomtrengelighet på filteret.}gjennom filteret.
Når luftfilteret er tett, lyser kontrollsymbolet \textSymb{vpadWarningFilter} på Vpad-skjermen og feilmeldingen \VpadEr{851} blir registrert.


\subsubsubject{Vedlikeholde/skifte ut}

\startSteps
\item Trekk sikkerhetslasken \Lone nedover (\in{fig.}[fig:airFilter]).
\item Drei husets underdel \Ltwo mot urviserens retning og ta det av.
\item Ta ut filterelementet og kontroller det. Skift det ut, om nødvendig.
\item Rengjør innsiden av filteret og sett filteret sammen igjen i omvendt rekkefølge.
\stopSteps

\page [yes]


\subsection [sSec:vw:belt] {Kileribbereim}

Kileribbereimen\index{Dieselmotor+kileribbereim} overfører bevegelsen til veivakselsvinghjulet over på dynamoen og klimakompressoren (valgfritt utstyr).
Et\index{Kileribbereim} spennelement i det siste segmentet (mellom dynamo og veivaksel) holder reimen spent.


\subsubsection [sSec:belt:change] {Skift av kileribbereimen}

\placefig [margin] [fig:belt:tool] {Låsetapp VW T 10060 A}
{\externalfigure[vw:belt:tool]}

\placefig [margin] [fig:belt:overview] {Spennelement}
{\externalfigure[vw:belt:overview]}

\placefig [margin] [fig:belt:tens] {Ansatspunkt for låsetapp}
{\externalfigure[vw:belt:tens]}


\subsubsubject{Med klimakompressor}


{\sla Nødvendig spesialverktøy:}

Låsetapp \aW{VW T 10060 A} til feste av spennelementet.

\startSteps
\item Marker dreieretning til kileribbereimen.
\item Bruk en krummet stjernenøkkel til å svinge armen til spennelementet i urviserens retning (\in {fig.}[fig:belt:overview]).
\item Bring borehullene (se pilene, \in {fig.}[fig:belt:tens]) til dekket og lås spennelementet med låsetappen.
\item Ta av kileribbereimen.
\stopSteps

Der Montering av kileribbereimen skjer i omvendt rekkefølge.

\starttextbackground [FC]
\startPictPar
\PMrtfm
\PictPar
\startitemize [1]
\item Vær oppmerksom på dreieretningen til kileribbereimen.
\item Pass på at reimen legges an og festes korrekt på reimskivene.
\item Start motoren og kontroller reimens gang.
\stopitemize
\stopPictPar
\stoptextbackground


\subsubsubject{Uten klimakompressor}

{\sla Nødvendig materiale:}

Reparasjonssett, bestående av reparasjonsanvisning, kileribbereim og spesialverktøy.\footnote{Se reservedelskatalog, under \aW{vedlikeholdsdeler}.}

\startSteps
\item Kutt gjennom kileribbereimen.
\item Følg de videre arbeidstrinnene i reparasjonsanvisningen.
\stopSteps

\starttextbackground [FC]
\startPictPar
\PMrtfm
\PictPar
\startitemize [1]
\item Pass på at reimen legges an og festes korrekt på reimskivene.
\item Start motoren og kontroller reimens gang.
\stopitemize
\stopPictPar
\stoptextbackground


\subsubsection [sSec:belt:tens] {Utskiftning av spennelementet}

{\sla Gjelder kun utførelse med klimakompressor}

\blank [medium]

\placefig [margin] [fig:belt:tens:change] {Utskiftning av spennelementet}
{\externalfigure[vw:belt:tens:change]
\noteF
\startLeg
\item Spennelement
\item Sikringsskrue
\stopLeg

{\bf Tiltrekkingsmoment}

Sikringsskrue:

\TorqueR 20 Nm\:+ ½ omdreining (180°).}

\startSteps
\item Demonter kileribbereimen som beskrevet (se \atpage[sSec:belt:change]).
\item Demonter perifere deler (avhengig av utrustning).
\item Skru ut sikringsskruen (\in{\Ltwo, fig.}[fig:belt:tens:change]).
\stopSteps

Montering av spennelementet skjer i omvendt rekkefølge.

\starttextbackground [FC]
\startPictPar
\PMrtfm
\PictPar
\startitemize [1]
\item Etter montering er det strengt nødvendig å benytte en ny sikringsskrue.
\item Tiltrekkingsmoment: Se \in{fig.}[fig:belt:tens:change].
\stopitemize
\stopPictPar
\stoptextbackground

\stopregister[index][reg:main:vw]

\stopsection

\page[yes]


\setups[pagestyle:marginless]


\startsection[title={Hydraulikkanlegg},
reference={sec:hydraulic}]

\starttextbackground [FC]
% \startfiguretext[left,none]{}
% {\externalfigure[toni_melangeur][width=30mm]}

\startSymPar
\externalfigure[toni_melangeur][width=4em]
\SymPar
\textDescrHead{Avfallshåndtering av driftsstoffer}
Det er ikke tillatt å kaste driftsstoffer og smøremidler i naturen eller å fjerne disse ved forbrenning.

Det er ikke tillatt å kaste brukte smøremidler sammen med husholdningsavfall eller å lede de ut i avløpsnettet eller naturen.

Brukte smøremidler får ikke blandes med andre væsker, da det består fare for dannelse av giftstoffer eller stoffer som vanskelig lar seg fjerne.
\stopSymPar
\stoptextbackground
\blank [big]

% \starthangaround{\PMgeneric}
% \textDescrHead{Qualification du personnel}
% Toute intervention sur l’installation hydraulique de votre véhicule ne peut être réalisée que par une personne dument qualifiée, ou par un service reconnu par \boschung.
% \stophangaround
% \blank[big]

\startSymList
\PHgeneric
\SymList
\textDescrHead{Renhet} Hydraulikkanlegget reagerer svært følsomt på urenheter i oljen. Derfor er det viktig, at det arbeides i helt rene omgivelser.
\stopSymList

\startSymList
\PHhot
\SymList
\textDescrHead{Sprutfare}
Før arbeider på hydraulikkanlegget til \sdeux\ må resttrykket i gjeldende hydraulikkrets slippes ut. Hete oljesprut kan føre til forbrenninger.
\stopSymList

\startSymList
\PHhand
\SymList
\textDescrHead{Klemfare}
Det er absolutt nødvendig å senke ned smussbeholderen eller å sikre den mekanisk ved hjelp av sikkerhetsstøtten, før det arbeides på hydraulikkanlegget til \sdeux.
\stopSymList

\startSymList
\PImano
\SymList
\textDescrHead{Trykkmåling}
For å måle hydraulikktrykket, plasser et manometer på et av \aW{mini-måle}-tilkoblingene til kretsen. Vær oppmerksom på, at manometeret viser et egnet måleområde.
\stopSymList

\page [yes]

\setups[pagestyle:normal]

\subsection{Vedlikeholdsintervaller}

\start

\setupTABLE [frame=off,
			 style={\ssx\setupinterlinespace[line=.93\lH]},
			 background=color,
			 option=stretch,
			 split=repeat]
\setupTABLE [r] [each] [
			 topframe=on,
			 framecolor=white,
			 backgroundcolor=TableLight,
% 			 rulethickness=.8pt,
]

% \setupTABLE [c] [odd] [backgroundcolor=TableMiddle]
% \setupTABLE [c] [even] [backgroundcolor=TableLight]
\setupTABLE [c] [1][ % width=30mm,
style={\bfx\setupinterlinespace[line=.93\lH]},
]
\setupTABLE [r] [first] [topframe=off,
			 style={\bfx\setupinterlinespace[line=.93\lH]},
			 backgroundcolor=TableMiddle,
]
% \setupTABLE [r] [2][style={\ssBfx\setupinterlinespace[line=.93\lH]}]


\bTABLE

\bTABLEhead
\bTR\bTD Vedlikeholdsarbeid \eTD\bTD intervall \eTD\eTR
\eTABLEhead

\bTABLEbody
\bTR\bTD Kontrollere for lekkasjer \eTD\bTD daglig \eTD\eTR
\bTR\bTD Kontrollere hydraulikkoljenivå \eTD\bTD daglig \eTD\eTR
\bTR\bTD Kontrollere tilstanden til hydraulikkledningene/-slangene; evt. utskiftning \eTD\bTD 600 h / 12 måneder \eTD\eTR
\bTR\bTD Skifte ut hydraulikkolje-retur- og sugefilter \eTD\bTD 600 h / 12 måneder \eTD\eTR
\bTR\bTD Smøre spolekjernene til magnetventilene med kobberfett \eTD\bTD 600 h / 12 måneder \eTD\eTR
\bTR\bTD Skifte hydraulikkolje \eTD\bTD 1200 h / 24 måneder \eTD\eTR
\eTABLEbody
\eTABLE
\stop


\subsection[niveau_hydrau]{Fyllenivå}

\placefig[margin][fig:hydraulic:level]{Fyllenivå til hydraulikkvæsken}
{\externalfigure[hydraulic:level]
\noteF
\startLeg
\item Optimalt fyllenivå
\stopLeg}

Et transparent seglass\index{Fyllenivå+hydraulikkvæske}\index{Vedlikehold+hydraulikkanlegg} gjør det mulig å kontrollere hydraulikkoljenivået.
Når hydraulikkoljenivået har sunket, må årsaken til dette fastlegges, før det fylles på igjen. Overhold de foreskrevne intervallene for utskiftning (tabell ovenfor) og spesifikasjonenefor hydraulikkvæsken (tabell \at{side}[sec:liqquantities]).


\subsubsection{Etterfylle hydraulikkvæske}

Etterfyll hydraulikkvæske, frem til fyllenivået har nådd midten av seglasset.
Start motoren og evt. etterfyll litt, frem til det nødvendige fyllenivået er nådd.


\subsection{Skifte hydraulikkvæske}

Fyllemengde og de nødvendige spesifikasjonene for hydraulikkvæsken finner du i tabellen på \at{side}[sec:liqquantities].

\startSteps
\item Åpne fylleåpningen for etterfylling av hydraulikktanken.
\item Tøm tanken ved hjelp av en oljesugepumpe eller fjern tappeskruen.

Tappeskruen befinner seg nede på hydraulikktanken, foran det venstre bakhjulet (\in{fig.}[fig:hydraulic:fluidDrain]).
\item Etterfyll hydraulikkvæske, frem til fyllenivået har nådd midten av seglasset.
Start motoren og evt. etterfyll litt, frem til det nødvendige fyllenivået er nådd.
\stopSteps

\placefig[margin][fig:hydraulic:fluidDrain]{Tappeskrue}
{\externalfigure[hydraulic:fluidDrain]}


\placefig[margin][fig:hydraulic:returnFilter]{Hydraulikkfilter}
{\externalfigure[hydraulic:returnFilter]}

\subsection[filtres:nettoyage]{Retur- og sugefilter}

\startSteps
\item Løft smussbeholderen og anbring sikkerhetsstøtten.
\item Ta av lokket til filteret på hydraulikktanken (\in{fig.}[fig:hydraulic:returnFilter]).
\item Skift ut\index{Oljefilter+hydraulikk-} filterelementet med et nytt.
\item Fukt en ny O-ring-tetning med litt hydraulikkvæske og fest den.
\item Bruk begge hender til å skru lokket på igjen (\TorqueR ca. 20 Nm).
\stopSteps

\page [yes]


\subsection[sec:solenoid]{Smøring av magnetventilene}

\placefig[margin][graissage_bobine]{Smøring av magnetventilene}
{\externalfigure[graissage_bobine][M]
\noteF
\startLeg
\item Spole til magnetventilen
\item Spolekjerne
\stopLeg}

Fuktighet og saltavleiringer, som trenger inn i kjernen til de elektromagnetiske spolene,
fører til korrodering av kjernen. Spolekjernene må en gang årlig smøres med kobberfett.
Fettet må være korrosjons-, vann- og temperaturbestandig inntil 50 °C:
\startSteps
\item Demonter spolen til magnetventilen (\in{\Lone, fig.}[graissage_bobine]).
\item Smør kjernen (\in{\Ltwo, fig.}[graissage_bobine]) med det foreskrevne spesialfettet og monter spolen på plass igjen.
\stopSteps


\subsection{Utskiftning av slangene}

Beskyttelsesgummien\index{Slanger+intervaller for utskiftning} og forsterkningsvevet til slangene utsettes for naturlig elding. Derfor er det absolutt nødvendig å skifte ut slangene til hydraulikkanlegget i de foreskrevne intervallene, også når det ikke foreligger noen {\em synlige} skader.

Pass på, at slangene blir flenset korrekt på kjøretøyet, for å utelukke for tidlig slitasje grunnet friksjon. Det må holdes tilstrekkelig avstand til andre komponenter, slik at friksjons- og vibrasjonsskader forhindres.

\stopsection

\page [yes]

\setups [pagestyle:bigmargin]


\startsection[title={Bremsesystem},
reference={sec:brake}]

\placefig[margin][fig:brake:rear]{Trommelbremse}
{\startcombination [1*2]
{\externalfigure[brake:wheelHub]}{\slx Bakhjulsnav}
{\externalfigure[brake:drum]}{\slx Mekanisme og bremsesett}
\stopcombination}

Bremsetromlene \Lfour\ må demonteres ved hvert standard vedlikehold, bremsemekanismen \Lseven\ må rengjøres og bremsesettene \Lfive, \Lsix\ må kontrolleres visuelt for slitasje (\in{fig.}[fig:brake:rear]).


\subsubject {Demontering}

\startSteps
\item Kjør kjøretøyet på en egnet kjøretøyheis og hev hjulene.
\item Ta av hjulene.
\stopSteps


{\sla Demontering av forhjulsbremsene}

\startSteps [continue]
\item Demonter bremsetrommelen \Lfour.
\stopSteps

{\sla Demontering av bakhjulsbremsene}

\startSteps [continue]
\item Ta av dekslene \Lone\ fra navet.
\item Fjern skruen \Ltwo\ og ta av mellomstykket.
\item Skru av navmutteren \Lthree\ med en pipenøkkel.
\item Ta av navet sammen med bremsetrommelen.
\stopSteps


\subsubject {Gjentatt montering}

Monter bremsetrommelen på igjen i omvendt rekkefølge. Trekk til mutrene til bakhjulsnavene \Lthree\ med det foreskrevne tiltrekkingsmomentet på 190 Nm.

\stopsection

\page [yes]

\setups [pagestyle:normal]


\startsection[title={Kontroll og vedlikehold av dekkene},
reference={sec:pneumatiques}]

Dekkene\index{Dekk+vedlikehold} må alltid være i feilfri tilstand, de skal kunne utføre de to hovedfunksjonene: godt grep og feilfri bremseadferd. Utillatelig stor slitasje og feil dekktrykk, spesielt for lavt trykk, er viktige ulykkesfaktorer.


\subsection{Sikkerhetsrelevante punkter}

\subsubsection{Slitasjekontroll}

Dekkslitasjen må kontrolleres ved hjelp av slitasjeindikatorene, som befinner seg i et profilspor (\in{fig.}[pneususure]).
Ved å foreta en visuell kontroll av dekkene kan påfallende tilstander og deres årsaker konstateres:

\placefig[margin][pneususure]{Slitasjekontroll}
{\Framed{\externalfigure[pneusUsure][M]}}

\placefig[margin][pneusdomages]{Skadde dekk}
{\Framed{\externalfigure[pneusDomages][M]}}

\startitemize
\item Slitasje på sidene av løpeflaten: Dekktrykk for lavt.
\item Forsterket slitasje i midten: Dekktrykk for høyt.
\item Asymmetrisk slitasje på sidene av dekket: Foraksel (spor, akselgeometri) feil innstilt.
\item Sprekker i løpeflaten: Dekket er for gammelt; dekkets gummi blir over tid hardere og det dannes sprekker (\in{fig.}[pneusdomages]).
\stopitemize

\starttextbackground[CB]
\startPictPar
\PHgeneric
\PictPar
\textDescrHead{Risikoer grunnet nedslitte dekk}
Et nedslitt dekk oppfyller ikke lenger dets funksjoner, spesielt når det gjelder bortledning av vann og slam; bremselengden forlenges og kjøreadferden forringes. Et nedslitt dekk har lettere for å skli, særlig på fuktig underlag. Det medfører økt fare for at dekket mister veigrepet.
\stopPictPar
\stoptextbackground


\subsubsection{Dekktrykk}

Det foreskrevne dekktrykket er notert på hjultypeskiltet, foran på panelet på passasjersiden (se \atpage [sec:plateWheel]).

Selv\index{Dekk+dekktrykk} når dekkene er i god stand, vil de over tid tape trykket mer eller mindre raskt (dess mer kjøring, desto større er trykktapet). Derfor må dekktrykket kontrolleres månedlig, mens dekkene er kalde. Når du kontrollerer trykket på varme dekk, må du legge til 0,3 bar til det foreskrevne trykket.

\start
\setupcombinations[M]
\placefig[margin][pneuspression]{Dekktrykk}
{\Framed{\externalfigure[pneusPression][M]}
\noteF
\startLeg
\item Korrekt trykk
\item For høyt trykk
\item For lavt trykk
\stopLeg
Det foreskrevne dekktrykket er notert på hjultypeskiltet, i førerhuset på passasjersiden.}
\stop

\starttextbackground[CB]
\startPictPar
\PHgeneric
\PictPar
\textDescrHead{Farer grunnet for lavt dekktrykk}
Et dekk kan komme til å revne, når trykket er for lavt. Dekket vil klemmes mer sammen, når det ikke er pumpet opp tilstrekkelig eller når kjøretøyet er overbelastet. Dette fører til at gummien blir varm og deler av dekket kan komme til å løsne i en sving.
\stopPictPar
\stoptextbackground

\stopsection

\page [yes]

\setups[pagestyle:marginless]


\startsection[title={Chassis},
reference={main:chassis}]

\subsection{Sikkerhetsrelevant feste av komponenter}

Ved hvert vedlikehold må sikkerhetsrelevante festeskruer til bestemte komponenter kontrolleres for korrekt feste, innbefattet de foreskrevne tiltrekkingsmomentene. Dette gjelder spesielt for leddkoblingssystemet og akslene.

\blank [big]

\startfigtext [left] [fig:frontAxle:fixing] {Foraksel}
{\externalfigure [frontAxle:fixing]}
{\sla Feste av forakselen}
\startLeg
\item Feste av fjærbladet: \TorqueR 150 Nm
\item Feste av trekkenhetene: \TorqueR 78 Nm
\stopLeg

{\sla Feste av bakakselen}
\startLeg
\item Feste av fjærbladet: \TorqueR 150 Nm
\stopLeg

\stopfigtext

\start

\setupTABLE [frame=off,style={\ssx\setupinterlinespace[line=.93\lH]},background=color,
option=stretch,
split=repeat]

\setupTABLE [r] [each] [topframe=on,
framecolor=white,
% rulethickness=.8pt
]

\setupTABLE [c] [odd] [backgroundcolor=TableMiddle]
\setupTABLE [c] [even] [backgroundcolor=TableLight]
\setupTABLE [c] [1][style={\bfx\setupinterlinespace[line=.93\lH]}]
\setupTABLE [r] [first] [topframe=off,style={\bfx\setupinterlinespace[line=.93\lH]},
]
% \setupTABLE [r] [2][style={\bfx\setupinterlinespace[line=.93\lH]}]


\bTABLE

\bTABLEhead
\bTR [backgroundcolor=TableDark] \bTD [nc=3] Trekkmomenter \eTD\eTR
% \bTR\bTD Position \eTD\bTD Type de vis \eTD\bTD Couple \eTD\eTR
\eTABLEhead

\bTABLEbody
\bTR\bTD Fremdriftsmotor venstre/høyre \eTD\bTD M12\:×\:35 8.8 \eTD\bTD 78 Nm \eTD\eTR
%% NOTE @Andrew: das sind Hydraulikmotoren
\bTR\bTD Arbeidspumpe \eTD\bTD M16\:×\:40 100 \eTD\bTD 330 Nm \eTD\eTR
\bTR\bTD Drivpumpe \eTD\bTD M12\:×\:40 100 \eTD\bTD 130 Nm \eTD\eTR
\bTR\bTD Fjærblader foran/bak \eTD\bTD M16\:×\:90/160 8.8 \eTD\bTD 150 Nm \eTD\eTR
% \bTR\bTD Fixation du système oscillant \eTD\bTD M12\:×\:40 8.8 \eTD\bTD 78 Nm \eTD\eTR
\bTR\bTD Feste av smussbeholderen \eTD\bTD M10\:×\:30 Verbus Ripp 100 \eTD\bTD 80 Nm \eTD\eTR
\bTR\bTD Hjulmuttere \eTD\bTD M14\:×\:1,5 \eTD\bTD 180 Nm \eTD\eTR
\bTR\bTD Feste av frontkosten \eTD\bTD M16\:×\:40 100 \eTD\bTD 180 Nm \eTD\eTR
\eTABLEbody
\eTABLE
\stop


\stopsection

\page [yes]


\startsection[title={Sentralsmøreanlegg},
reference={main:graissageCentral}]


\subsection{Beskrivelse av styremodulen}

\sdeux\ kan utstyres med\index{Sentralsmøreanlegg} et sentralsmøreanlegg (valgfritt). Sentralsmøreanlegget forsyner hvert smørepunkt til kjøretøyet i regelmessige avstander med smøremiddel.

\startfigtext [left] [vogel_affichage] {Displaymodul}
{\externalfigure[vogel_base2][W50]}
\blank
\startLeg
\item 7-sifret display: Verdier og driftstilstand
\item \LED: System i pause (standby-drift)
\item \LED: Pumpe i drift
\item \LED: Styring av systemet med syklusbryter
\item \LED: Overvåking av systemet med trykkbryter
\item \LED: Feilmelding
\item Rulleknapper:
\startLeg [R]
\item Aktiver displayet
\item Vis verdier
\item Endre verdier
\stopLeg
\item Tast til å veksle til driftsmåten, bekreftelse av verdiene
\item Aktivere en dobbel smøresyklus
\stopLeg
\stopfigtext

Sentralsmøreanlegget omfatter smøremiddelpumpen, den gjennomsiktige smøremiddelbeholderen på venstre side av chassisen og styremodulen i sentralelektronikken.
% \blank
\page [yes]


\subsubsubject{Display og taster til styremodulen}

\start

\setupTABLE [frame=off,style={\ssx\setupinterlinespace[line=.93\lH]},background=color,
option=stretch,
split=repeat]

\setupTABLE [r] [each] [topframe=on,
framecolor=white,
% rulethickness=.8pt
]

\setupTABLE [c] [odd] [backgroundcolor=TableMiddle]
\setupTABLE [c] [even] [backgroundcolor=TableLight]
\setupTABLE [c] [1][width=9mm,style={\bfx\setupinterlinespace[line=.93\lH]}]
\setupTABLE [r] [first] [topframe=off,style={\bfx\setupinterlinespace[line=.93\lH]},
]
% \setupTABLE [r] [2][style={\bfx\setupinterlinespace[line=.93\lH]}]


\bTABLE
\bTABLEhead
% \bTR [backgroundcolor=TableDark]
% \bTD [nc=4] Display og taster til styremodulen \eTD\eTR
\bTR\bTD Pos. \eTD
\bTD \LED \eTD\bTD Displaymodus \eTD
\bTD Programmeringsmodus \eTD\eTR
\eTABLEhead

\bTABLEbody
\bTR\bTD 2 \eTD
\bTD Driftstilstand {\em pause}\hskip .5em\null \eTD
\bTD Anlegget er i standby\hskip .5em\null \eTD % -drift
\bTD Pausetiden kan endres \eTD\eTR
\bTR\bTD 3 \eTD
\bTD Driftstilstand {\em Contact} \eTD
\bTD Pumpen arbeider \eTD
\bTD Arbeidstiden kan endres \eTD\eTR
\bTR\bTD 4 \eTD
\bTD Systemkontroll {\em CS} \eTD
\bTD Med den eksterne syklusbryteren \eTD
\bTD Kontrollmodusen kan deaktiveres eller endres \eTD\eTR
\bTR\bTD 5 \eTD
\bTD Systemkontroll {\em PS} \eTD
\bTD Med den eksterne trykkbryteren \eTD
\bTD Kontrollmodusen kan deaktiveres eller endres \eTD\eTR
\bTR\bTD 6 \eTD
\bTD Feil {\em Fault} \eTD
\bTD [nc=2] Det foreligger en funksjonsfeil. Årsaken vises i form av en feilkode, etter at tasten \textSymb{vogel_DK} trykkes. Utførelsen av funksjonene avbrytes. \eTD\eTR
\bTR\bTD 7 \eTD
\bTD Piltaster \textSymb{vogelTop} \textSymb{vogelBottom} \eTD
\bTD [nc=2] \items[symbol=R]{Aktivering av displayet, spørring av parametere (displaymodus), innstilling av den viste (I) verdien (programmeringsmodus)}
\eTD\eTR
\bTR\bTD 8 \eTD
\bTD Tast \textSymb{vogelSet} \eTD
\bTD [nc=2] Koble om mellom display- og programmeringsmodus eller bekrefte inntastede verdier. \eTD\eTR
\bTR\bTD 9 \eTD
\bTD Tast \textSymb{vogel_DK} \eTD
\bTD [nc=2] Hvis apparatet befinner seg i tilstanden {\em Pause}, utløses mellomsmøresyklusen ved betjening av tasten. Feilmeldinger bekreftes og slettes. \eTD\eTR
\eTABLEbody
\eTABLE
\stop
\vfill

\startfigtext [left] [vogel_touches]{Displaymodul}
{\externalfigure[vogel_base][width=50mm]}
\textDescrHead{Displaymodus} Trykk kort på en av piltastene \textSymb{vogelTop} \textSymb{vogelBottom}, for å aktivere det 7-sifrede displayet \textSymb{led_huit}. Gjennom gjentatt trykk på tasten \textSymb{vogelTop} kan en vise forskjellige parametere fulgt av deres verdier. Modusen {\em Display} kan sees på de kontinuerlig lysende \LED\char"2060s (\in{2 til 6, fig.}[vogel_affichage]).
\blank [medium]
\textDescrHead{Programmeringsmodus} For å endre verdiene må du i minst 2 sekunder trykke tasten \textSymb{vogelSet}, for å skifte til modusen {\em Programmering}: \LED\char"2060s blinker. Trykk kort på tasten \textSymb{vogelSet}, for å endre\index{Sentralsmøreanlegg+programmering} visningen, deretter endres den ønskede verdien med tastene \textSymb{vogelTop} \textSymb{vogelBottom}. Bruk\index{Sentralsmøreanlegg+display} tasten \textSymb{vogelSet} til å bekrefte.
\stopfigtext

\page [yes]


\subsection{Undermenyer i modusen {\em Display}}

\vskip -9pt

\adaptlayout [height=+5mm]

\startcolumns[balance=no]\stdfontsemicn

\startSymVogel
\externalfigure[vogel_tpa][width=26mm]
\SymVogel
\textDescrHead{Pausetid [h]} Trykk på tasten \textSymb{vogelTop}, fpor å vise de programmerte verdiene.
\stopSymVogel

\startSymVogel
\externalfigure[vogel_068][width=26mm]
\SymVogel
\textDescrHead{Gjenværende pausetid [h]} Gjenværende tid til neste smøresyklus.
\stopSymVogel

\startSymVogel
\externalfigure[vogel_090][width=26mm]
\SymVogel
\textDescrHead{Total pausetid [h]} Total pausetid mellom to sykluser.
\stopSymVogel

\startSymVogel
\externalfigure[vogel_tco][width=26mm]
\SymVogel
\textDescrHead{Smøretid [min]} Trykk på \textSymb{vogelTop}, for å vise de programmerte verdiene.
\stopSymVogel

\startSymVogel
\externalfigure[vogel_tirets][width=26mm]
\SymVogel
\textDescrHead{Apparat i standby} Visning ikke mulig siden apparatet er i standby (pause).
\stopSymVogel

\startSymVogel
\externalfigure[vogel_026][width=26mm]
\SymVogel
\textDescrHead{Smøretid [min]} Varighet til en smøreprosess.
\stopSymVogel

\startSymVogel
\externalfigure[vogel_cop][width=26mm]
\SymVogel
\textDescrHead{Systemkontroll} Trykk på \textSymb{vogelTop}, for å vise de programmerte verdiene.
\stopSymVogel

\startSymVogel
\externalfigure[vogel_off][width=26mm]
\SymVogel
\textDescrHead{Kontrollmodus} \hfill PS: Trykkbryter;\crlf
CS: Syklusbryter; OFF: deaktivert.
\stopSymVogel

\startSymVogel
\externalfigure[vogel_0h][width=26mm]
\SymVogel
\textDescrHead{Driftstimer} Trykk på \textSymb{vogelTop}, for å vise verdien i to trinn.
\stopSymVogel

\startSymVogel
\externalfigure[vogel_005][width=26mm]
\SymVogel
\textDescrHead{Del 1: 005} Driftstiden vises i to deler, gå til del 2 med tasten \textSymb{vogelTop}.
\stopSymVogel

\startSymVogel
\externalfigure[vogel_338][width=26mm]
\SymVogel
\textDescrHead{Del 2: 33,8} Den 2. delen til tallet er 33,8; resulterer til sammen en driftstid på 533,8 h.
\stopSymVogel

\startSymVogel
\externalfigure[vogel_fh][width=26mm]
\SymVogel
\textDescrHead{Feiltid} Trykk på \textSymb{vogelTop}, for å vise verdien i to trinn.
\stopSymVogel

\startSymVogel
\externalfigure[vogel_000][width=26mm]
\SymVogel
\textDescrHead{Del 1: 000} Feiltiden vises i to deler;\crlf
gå til del 2 med \textSymb{vogelTop}.
\stopSymVogel

\startSymVogel
\externalfigure[vogel_338][width=26mm]
\SymVogel
\textDescrHead{Del 2: 33,8} Den 2. delen til tallet er 33,8; resulterer til sammen en feiltid på 33,8 h.
\stopSymVogel

\stopcolumns

\page [yes]


\setups [pagestyle:marginless]


\startsection[title={Smøreskjema for manuell smøring},
reference={sec:grasing:plan}]

\starttextbackground [FC]
\startPictPar
\PMgeneric
\PictPar
Smørepunktene som er angitt i smøreskjemaet (\in{fig.}[fig:greasing:plan]) må smøres regelmessig. En regelmessig smøring må utføres, for å sikre en permanent {\em reduksjon av friksjonen} og for holde unna fuktighet og andre korrosive substanser.
\stopPictPar
\stoptextbackground

\blank [big]

\start

\setupcombinations [width=\textwidth]

\placefig[here][fig:greasing:plan]{Smøreskjema til kjøretøyet}
{\startcombination [3*1]
{\externalfigure[frame:steering:greasing]}{\ssx Leddstyring og svingmekanisme}
{\externalfigure[frame:axles:greasing]}{\ssx Akser}
{\externalfigure[frame:sucMouth:greasing]}{\ssx Sugeenhet}
\stopcombination}

\stop

\vfill

\startLeg [columns,three]
\item Løftesylinder til leddstyringen\crlf {\sl 2 smørenipler per sylinder}
\item Lager til leddstyringen\crlf {\sl 2 smørenipler på venstre side}
\columnbreak
\item Lager til svingmekanismen\crlf {\sl 1 smørenippel foran tanken}
\item Bladfjær\crlf {\sl 2 smørenipler per fjærblad}
\columnbreak
\item Sugeenhet\crlf {\sl 1 smørenippel per hjul}
\item Sugeenhet\crlf {\sl 1 smørenippel på trekkarmen}
\stopLeg



\page [yes]


\setups [pagestyle:bigmargin]


\subsubject{Smøring av smussbeholderen}

Smussbeholderen har 6 smørepunkt (2\:×\:3), som må smøres ukentlig.

\blank [big]


\placefig[here][fig:greasing:container]{Løftemekanisme til beholderen}
{\externalfigure[container:mechanisme]}


\placelegende [margin,none]{}
{{\sla Legende:}

\startLeg
\item Venstre lager til beholderen
\item Høyre lager til beholderen
\item Venstre hydraulikksylinder (oppe)
\item Venstre hydraulikksylinder (nede)

{\em Som høyre sylinder (punkt \in[greasing:point;hide]).}
\item Høyre hydraulikksylinder (oppe)
\item [greasing:point;hide]Høyre hydraulikksylinder (nede)
\stopLeg}

\stopsection

\page [yes]



\startsection[title={Elektrisk anlegg},
reference={sec:main:electric}]

\subsection{Sentralelektronikk i chassis}

\startbuffer [fuses:preventive]
\starttextbackground [CB]
\startPictPar
\PHvoltage
\PictPar
\textDescrHead{Sikkerhetsforskrifter}
Følg sikkerhetsforskriftene i\index{Sikringer+chassis} denne\index{Rele+chassis} anvisningen: Skift alltid ut alle sikringer med sikringer av foreskrevet amperetall; ta av metallsmykker før du arbeider på elektriske\index{Elektrisk anlegg} anlegg (ringer, armringer osv.).
\stopPictPar
\stoptextbackground
\stopbuffer


\subsubsubject{MIDI-sikringer}

\starttabulate[|l|r|p|]
\HL
\NC\md F 1 \NC 5 A \NC Bremselys, \aW{+\:15} OBD \NC\NR
\NC\md F 2 \NC 5 A \NC \aW{+\:15} Motorstyring \NC\NR
\NC\md F 3 \NC 7,5 A \NC \aW{+\:30} Motorstyring og OBD \NC\NR
\NC\md F 4 \NC 20 A \NC Drivstoffpumpe \NC\NR
\NC\md F 5 \NC 20 A \NC \aW{D\:+} Dynamo, \aW{+\:15} relé K 1 \NC\NR
\NC\md F 6 \NC 5 A \NC Motorstyring \NC\NR
\NC\md F 7 \NC 10 A\NC Motor-avgassbehandling \NC\NR
\NC\md F 8 \NC 20 A \NC Motorelektronikk (styring) \NC\NR
\NC\md F 9 \NC 15 A \NC Motor-avgassbehandling, forsyning, forgløding \NC\NR
\NC\md F 10\NC 30 A \NC Motorstyring \NC\NR
\NC\md F 11\NC 5 A \NC Arbeidslyskaster bak \NC\NR
%% NOTE @Andrew: Singular
\HL
\stoptabulate

\placefig [margin] [fig:electric:power:rear] {Sentralelektronikk i chassis}
{\externalfigure [electric:power:rear]
\noteF
\startKleg
\sym{K 1} Elektronisk motorstyring
\sym{K 2} Drivstoffpumpe
\sym{K 3} Frigivelse av starteren
\sym{K 4} Bremselys
\sym{K 5} {[}Reserve{]}
\sym{K 6} Arbeidslyskaster bak
%% NOTE @Andrew: Singular
\sym{K 7} Forglødingsanmlegg
\stopKleg
}


\subsubsubject{MAXI-sikring}

% \startcolumns [n=2]
\starttabulate[|l|r|p|]
\HL
\NC\md F 15 \NC 50 A \NC Hovedforsyningen til sentralelektronmikken \NC\NR
\HL
\stoptabulate

\page [yes]

\setups[pagestyle:marginless]


\subsection{Sentralelektronikk i førerhuset}

\startcolumns[rule=on]

\placefig [bottom] [fig:fuse:cab] {Sikringer og relé i førerhuset}
{\externalfigure [electric:power:front]}

%\vfill

\subsubsubject{Relé}

\index{Sikringer+Førerhus}\index{Relé+Førerhus}

\starttabulate[|lB|p|]
\NC K 2\NC Klimakompressor\NC\NR
\NC K 3\NC Klimakompressor\NC\NR
\NC K 4\NC Elektrisk vannpumpe\NC\NR
\NC K 5\NC Roterende posisjonslys\NC\NR
\NC K 10 \NC Blinkfrekvensgiver\NC\NR
\NC K 11 \NC Nærlys\NC\NR
\NC K 12 \NC Fjernlys\NC\NR
\NC K 13 \NC Arbeidslyskaster\NC\NR
\NC K 14 \NC Vindusvisker-intervallkobling\NC\NR
\stoptabulate

\vskip -24pt

\placefig [bottom] [fig:fuse:access] {Tilgangsluke til sentralelektronikken}
{\externalfigure [electric:power:cabin]}

\stopcolumns

\page [yes]


\subsubsubject{MINI-sikringer}

\startcolumns[rule=on]
% \setuptabulate[frame=on]
%\placetable[here][tab:fuses:cab]{Fusibles dans la cabine}
%{\noteF
\starttabulate[|lB|r|p|]
\NC F 1 \NC 3 A \NC Parkeringslys venstre \NC\NR
\NC F 2 \NC 3 A \NC Parkeringslys høyre \NC\NR
\NC F 3 \NC 7,5 A \NC Nærlys venstre \NC\NR
\NC F 4 \NC 7,5 A \NC Nærlys høyre \NC\NR
\NC F 5 \NC 7,5 A \NC Fjernlys venstre {[}Reserve{]} \NC\NR
\NC F 6 \NC 7,5 A \NC Fjernlys høyre {[}Reserve{]} \NC\NR
\NC F 7 \NC 10 A \NC Arbeidslyskaster oppe \NC\NR
%% NOTE @Andrew: Plural
\NC F 8 \NC 10 A \NC Arbeidslyskaster nede (reserve) \NC\NR
%% NOTE @Andrew: Plural
\NC F 9 \NC — \NC {[}ledig{]} \NC\NR
\NC F 10 \NC 10 A \NC Vindusvisker \NC\NR
\NC F 11 \NC 5 A \NC Bryter belysning og varselblinklys \NC\NR
\NC F 12 \NC 5 A \NC {[}Reserve{]} \NC\NR
\NC F 13 \NC 10 A \NC Sidespeiloppvarming \NC\NR
\NC F 14 \NC 7,5 A \NC \aW{+\:15} Radio og kamera \NC\NR
\NC F 15 \NC 10 A \NC \aW{+\:30} Varselblinklys \NC\NR
\NC F 16 \NC 5 A \NC Belysning rattsøyle \NC\NR
\NC F 17 \NC 7,5 A \NC \aW{+\:30} Radio og innvendig belysning \NC\NR
\NC F 18 \NC — \NC {[}ledig{]} \NC\NR
\NC F 19 \NC 20 A \NC \aW{+\:30} RC 12 foran \NC\NR
\NC F 20 \NC 20 A \NC \aW{+\:30} RC 12 bak \NC\NR
\NC F 21 \NC 15 A \NC 12-V-stikkontakt \NC\NR
\NC F 22 \NC 5 A \NC Tenningsnøkkel, multifunksjonspanel, Vpad \NC\NR
\NC F 23 \NC 5 A \NC Nødstopp, midtre panel, RC 12 foran \NC\NR
\NC F 24 \NC 5 A \NC Nødstopp, midtre panel, RC 12 bak \NC\NR
\NC F 25 \NC 2 A \NC \aW{+\:15} RC 12 foran \NC\NR
\NC F 26 \NC 2 A \NC \aW{+\:15} RC 12 bak \NC\NR
\NC F 27 \NC 15 A \NC Oppvarmingsvifte \NC\NR
\NC F 28 \NC 10 A \NC Klimakompressor, sentralsmøreanlegg \NC\NR
\NC F 29 \NC 15 A \NC Klimakondensator \NC\NR
\NC F 30 \NC 5 A \NC Termostat klimaanlegg \NC\NR
\NC F 31 \NC 5 A \NC \aW{+\:15} Multifunksjonspanel/Vpad \NC\NR
\NC F 32 \NC 15 A \NC Elektrisk vannpumpe, roterende varsellys \NC\NR
\NC F 33 \NC — \NC {[}ledig{]} \NC\NR
\NC F 34 \NC — \NC {[}ledig{]} \NC\NR
\NC F 35 \NC — \NC {[}ledig{]} \NC\NR
\NC F 36 \NC — \NC {[}ledig{]} \NC\NR
\stoptabulate
\stopcolumns

\page [yes]

\setups [pagestyle:bigmargin]


\subsection[sec:lighting]{Belysnings- og signalinnretning}


\placefig [here] [fig:lighting] {Belysnings- og signalinnretning til kjøretøyet}
{\externalfigure [vhc:electric:lighting]}

\placelegende [margin,none]{}{%
\vskip 30pt
{\sla Legende:}
\startLongleg
\item Parkeringslys\hfill 12 V – 5 W
\item Nærlys\hfill H7 12 V – 55 W
\item Blinklys\hfill oransje 12 V – 21 W
\item {\stdfontsemicn Arbeidslyskaster}\hfill G886 12 V – 55 W
\item Kjøreretningsindikator\hfill 12 V – 21 W
\item Rygge-/bremselys\hfill 12 V – 5/21 W
\item Ryggelyskastere\hfill 12 V – 21 W
\item {[}Fri{]}
\item Nummerplatelys\hfill 12 V – 5 W
\item Roterende varsellys\hfill H1 12 V – 55 W
\stopLongleg}

\subsubsubject{Innstilling av lyskastere}

\placefig [margin] [fig:lighting:adjustment] {Lysstråle ved 5 m}
{\externalfigure [vhc:lighting:adjustment]
\startitemize
\sym{H\low{1}} Høyde til glødetråden: 100 cm
\sym{H\low{2}} Korrektur ved 2\hairspace\%: 10 cm
\stopitemize}

{\md Forutsetninger:} Fersk-/resirkuleringsvannbeholder full, fører ved rattet.

Retningsinnstillingene til lyskasterne er forhåndsjustert fra fabrikken. Høyden og vinkelen til lysstrålen kan innstilles ved å svinge plastholderne.

Når det ved en kontroll konstateres, at innstillingen må endres, løsne sikringsskruen og korriger vinkelen slik, at den tilsvarer de lovbestemte forskriftene (se \in{fig.}[fig:lighting:adjustment]). Skru fast sikringsskruen igjen.

\page [yes]
\setups [pagestyle:marginless]


\subsection[sec:battcheck]{Batteri}

\subsubsection{Sikkerhetsforskrifter}

\startSymList
\PPfire
\SymList
\textDescrHead{Eksplosjonsfare}
Ved\index{Batteri+sikkerhetshenvisninger}\index{Fare+eksplosjon} lading av batterier dannes det eksplosivt\index{Knallgass} knallgass. Batteriene må kun lades i godt ventilerte rom! Unngå gnistdannelse! I nærheten av batteriet må du ikke ha ild, direkte lys eller røyke.
\stopSymList

\startSymList
\PHvoltage
\SymList
\textDescrHead{Fare for kortslutning}
Når\index{Batteri+vedlikehold} plussklemmen til det tilkoblede batteriet kommer i kontakt med kjøretøyets deler, består det\index{Fare+brann}\index{Fare+eksplosjon} fare for kortslutning. Dette kan føre til at gassblandingen som trer ut av batteriet eksploderer, du og andre kan bli utsatt for alvorlige personskader.

\startitemize
\item Ikke legg metallgjenstander eller verktøy på batteriet.
\item Når batteriet kobles fra må alltid minusklemmen kobles fra først, deretter plussklemmen.
\item Når batteriet kobles til må alltid plussklemmen kobles til først, deretter minusklemmen.
\item Ikke løsne eller koble fra tilkoblingsklemmene til batteriet når motoren er igang.
\stopitemize
\stopSymList


\startSymList
\PHcorrosive
\SymList
\textDescrHead{Fare for etsing}
Bruk\index{Fare+Etsing} vernebriller og syrefaste vernehansker. Batterivæske består av ca. 27% svovelsyre (H\low{2}SO\low{4}) og kan derfor forårsake etsing. Sørg for å nøytralisere\index{Batteri+Fare}\index{Batteri+-væske} batterivæske, som er kommet i kontakt med hud, med en løsning som består av tvekulsur natron og skyll av med rent vann. Hvis batterivæske når inn i øynene, må du skylle godt med kaldt vann og omgående oppsøke en lege.
\stopSymList

\startSymList
\startcombination[1*2]
{\PHcorrosive}{}
{\PHfire}{}
\stopcombination
\SymList
\textDescrHead{Lagring av batterier}
Batterier\index{Batteri+lagre} må alltid lagres opprettstående. I motsatt fall kan det føre til at batterivæske trer ut og fører til etseskader eller – ved reaksjon med andre substanser – forårsaker brann. \par\null\par\null
\stopSymList

\testpage [16]

\starttextbackground [FC]
\setupparagraphs [PictPar][1][width=2.4em,inner=\hfill]

\startPictPar
\PMproteyes
\PictPar
\textDescrHead{Vernebriller}
Ved\index{Fare+øyeskade} blanding av vann og syre kan væsken sprute i øynene. Ved syresprut i øyet må de straks skylles med rent vann og omgående oppsøke en lege!
\stopPictPar
\blank [small]

\startPictPar
\PMrtfm
\PictPar
\textDescrHead{Dokumentasjon}
Ved håndtering av batterier må en overholde sikkerhetshenvisningene, sikkerhetstiltakene og fremgangsmåtene som beskrives i denne bruksanvisningen.
\stopPictPar
\blank [small]

\startPictPar
\PStrash
\PictPar
\textDescrHead{Miljøvern}
Batterier\index{Miljøvern} inneholder skadestoffer. Gamle batterier må aldri kastes i husholdningsavfall. Batterier må kasseres miljøvennlig. Sørg for at disse overleveres til et fagverksted eller et godkjent mottak for gamle batterier.

Fylte batterier må alltid transporteres og lagres opprettstående. Ved transport må batterier sikres mot å velte. Fra ventilasjonsåpningene til låsepluggen kan batterivæske lekke ut og utsette miljøet for farer.
\stopPictPar
\stoptextbackground

\page [yes]

\setups[pagestyle:normal]


\subsubsection{Praktiske råd}

For å oppnå maksimal levetid må batteriet alltid holdes mest mulig fulladet.

Lading\index{Batteri+levetid} av batteriet for å opprettholde levetiden ved lange stillstandssperioder forlenger ikke bare levetiden til batteriet men sørger også for at kjøretøyet til enhver tid er startklart.

\placefig[margin][fig:batterycompartment]{\select{caption}{Batterirom (vedlikeholdsluke)}{batterirom}}
{\externalfigure[batt:compartment]}


\subsubsection{Vedlikehold}

Batteriet til \sdeux\ er et {\em vedlikeholdsfritt} blybatteri. Bortsett fra opprettholdelsen av ladet tilstand og rengjøringen krever batteriet ingen vedlikeholdstiltak.

\startitemize
\item Sørg for, at polene til batteriet alltid holdes rene og tørre. Smør inn polene med litt syreavvisende fett.
\item Batterier, som\index{Batteri+lading} har en hvilespenning på\index{Batteri+hvilespenning} på mindre enn 12,4 V, må lades.
\stopitemize

\placefig[margin][fig:bclean]{Rengjøring av polene}
{\externalfigure[batt:clean]
\noteF
Verwenden\index{Batteriet+rengjøres}\index{Rengjøring+batterier} med varmt vann, for å fjerne det hvite pulveret som oppstår ved korrosjon. Hvis en pol er rusten, koble fra batterikabelen og rengjør polen med en stålbørste. Sørg for å påføre en tynn fettfilm på polene.}


\subsubsection[sec:battery:switch]{Bruk av batteriskillebryteren}

{\sl Det frarådes, å betjene batteriskillebryteren regelmessig (eksempelvis daglig)!}

\startSteps
\item Slå\index{Batteriskillebryter} av tenningen og vent deretter omtrent 1 minutt.
\item Åpne batterirommet (\inF[fig:batterycompartment]).
\item Trykk på den røde knappen til batteriskillebryteren, for å avbryte strømkretsen.
\item For å lukke strømkretsen igjen, drei batteriskillebryteren ¼ omdreining i urviserens retning.
\stopSteps

% \starttextbackground [FCnb]
% \startPictPar
% \PMgeneric
% \PictPar
% Der Batterietrennschalter ist dafür vorgesehen, die Batterie für bestimmte Wartungs- und Reparaturarbeiten vorübergehend vom Stromkreis zu trennen. Es ist nicht empfehlenswert, den Batterietrennschalter regelmäßig (\eG\ täglich) zu betätigen: Bestimmte elektronische Komponenten sollten ständig unter Spannung stehen, ansonsten kann es zu Fehlermeldungen im Fehlerspeicher kommen.
% \stopPictPar
% \stoptextbackground

\stopsection

\page [yes]


\setups[pagestyle:marginless]

\section[sec:cleaning]{Kjøretøyrens}

Bruk\startregister[index][vhc:lavage]{Vedlikehold+rengjøring} rikelig med vann til å spyle av grovt slam og smuss fra karosseriet før den egentlige rengjøringen startes. Ikke bare vask sideflatene, men også hjulhusene og undersiden til kjøretøyet.

Spesielt om vinteren må kjøretøyet vaskes grundig, for å fjerne de sterkt korrosive\index{Korrosjon+forebygging} strøsaltavleiringene.

\starttextbackground [FC]
\startPictPar
\PHgeneric
\PictPar
\textDescrHead{Forhindre skader forårsaket av vann}
Aldri rengjør kjøretøyet ved hjelp av {\em vannkanoner} (\eG\ fra brannvesenet) eller {\em kaldrens på hydrokarbon-basis.} Når du arbeider med høytrykks-damprenser, må du være oppmerksom på forskriftene nedenfor som gjelder dette.
\stopPictPar
\blank[small]

\startPictPar
\pTwo[monde]
\PictPar
\textDescrHead{Miljøvern}
Rengjøring av kjøretøyet kan føre til sterke miljøbelastninger.
Rengjør kjøretøyet kun på et\index{Miljøvern} sted som er utstyrt med oljeseparator. Følg gjeldende forskrifter for miljøvern.
\stopPictPar
\blank[small]

\startPictPar
\PMwarranty
\PictPar
\textDescrHead{Rengjør faglig korrekt!}
For skader, som oppstår på grunn av ignorering av rengjøringsforskriftene, kan \BosFull{boschung} ikke stilles krav for ansvar- eller garantikrav.
\stopPictPar
\stoptextbackground


\subsection{Høytrykksrengjøring}

Til høytrykks-rengjøring\index{Rengjøring+høytrykk} av kjøretøyet egner det seg å bruke en standard høytrykksrenser.

Ved høytrykksrengjøring skal følgende punkter følges:

\startitemize
\item Maksimalt arbeidstrykk 50\,bar
\item Flatstråledyse med en sprøytevinkel på 25°
\item Sprøyteavstand minst 80\,cm
\item Vanntemperatur maksimalt 40\,°C
\item Vær oppmerksom på avsnittet \about[reiMi], \atpage[reiMi].
\stopitemize

Ved ignorering av disse\index{Lakk+skader} forskriftene kan det oppstå skader på lakk og korrosjonsbeskyttelsen\index{Skader+lakk}.

Vær også oppmerksom på bruksanvisningen og sikkerhetsforskriftene til høytrykksrenseren.

\starttextbackground[FC]
\startPictPar\PPspray\PictPar
Ved høytrykksrengjøring kan vann trenge inn på steder, hvor det kan føre til skader. Derfor må vannstrålen aldri rettes mot ømfintlige deler og redskaper:
\stopPictPar

\startitemize
\item Sensorer, elektriske forbindelser og tilkoblinger
\item Starter, dynamo, innsprøytningsanlegg
\item Magnetventiler
\item Ventilasjonsåpninger
\item Hittil ikke avkjølte mekaniske komponenter
\item Henvisnings-, varsel- og fareklistremerker
\item Elektroniske styreenheter
\stopitemize

\textDescrHead{Motorvask}
Unngå at vann kommer inn i suge-, ventilasjons- og avluftingsåpningene. Ved høytrykksrensere må strålen ikke rettes direkte på elektriske komponenter og ledninger. Ikke rett strålen på innsprøytningsanlegget! Etter motorvask må motoren konserveres; derved må du beskytte reimdriften fra produktet som brukes til konserveringen.
\stoptextbackground

\starttextbackground [FC]
\setupparagraphs [PictPar][1][width=6em,inner=\hfill]
\startPictPar
\framed[frame=off,offset=none]{\PMproteyes\PMprotears}
\PictPar
\textDescrHead{Restvann}
Ved rengjøringen samler det seg vann på bestemte steder til kjøretøyet (\eG\ i kulene til motorblokken eller drevet); dette må fjernes ved hjelp av trykkluft. Ved håndtering av trykkluft må du huske å bruke tilsvarende verneutstyr, og anlegget må tilsvare de gjeldende sikkerhetsforskriftene (multidyse).
\stopPictPar
\stoptextbackground


\subsubsection[reiMi]{Egnede rengjøringsmidler}

Bruk\index{Rengjøringsmiddel} kun rengjøringsmidler, som har følgende egenskaper:

\startitemize
\item Ingen skureeffekt
\item PH-verdi på 6–7
\item Løsningsmiddelfri
\stopitemize

For fjerning av hårdnakkede flekker kan du forsiktig bruke vaskebensin på små lakkflater eller spiritus, ikke bruk andre løsningsmidler. Fjern løsningsmiddelrester fra lakken. Rengjøringen av plastdeler med bensin kan føre til sprekker eller missfarging!

Rengjør flater med\index{Rengjøring+klistremerke} varsel- eller fareklistremerker med rent vann og bruk en myk svamp.

Unngå at det kommer vann inn i de elektriske komponentene: Ikke bruk bilbørster til rengjøring av blinklys- og lyktehus, bruk en myk klut eller svamp.

\starttextbackground [CB]
\startPictPar
\GHSgeneric\par
\GHSfire
\PictPar
\textDescrHead{Fare på grunn av kjemikalier}
Rengjøringsmidler kan føre til helse- og sikkerhetsrisikoer (lett antennelige stoffer). Vær oppmerksom på sikkerhetsforskriftene som gjelder for rengjøringsmidlene som brukes, vær oppmerksom på fare- og databladene til middelet som brukes.
\stopPictPar
\stoptextbackground

\stopregister[index][vhc:lavage]

\stopchapter
\stopcomponent


