\startcomponent c_30_overview_s2_140-pt
\product prd_ba_s2_140-pt

\chapter{Vista geral sobre o veículo}

\setups [pagestyle:marginless]


\placefig [here] [] {Vista geral sobre o lado esquerdo do veículo}
{\externalfigure [overview:side:left:de]}


\page [yes]


\placefig [here] [] {Vista geral sobre o lado direito do veículo}
{\externalfigure [overview:side:right:de]}

\page [yes]

\setups [pagestyle:normal]


\section{Geral}

\placefig[margin][p4_vue_01]{\sdeux\ durante o transporte}
{%
\startcombination [1*3]
{\externalfigure[overview:vhc:01]}{}
{\externalfigure[overview:vhc:02]}{}
{\externalfigure[overview:vhc:03]}{}
\stopcombination}

A máquina varredora \BosFull{sdeux} reflete a experiência e a competência da Boschung, adquiridas ao longo de décadas de colaboração com os seus fiéis clientes e parceiros.
Durante esse período, os requisitos impostos pelos municípios e pelos prestadores de serviços relativamente à mobilidade e polivalência da máquina aumentaram significativamente. Os criadores da \sdeux\ enfrentaram este desafio, movidos pelas necessidades dos clientes, e incentivados pelas sugestões progressivas e inovadoras do serviço de assistência ao cliente da Boschung.
Desta síntese das sugestões do cliente e colocação em prática da experiência adquirida, nasceu a \sdeux.


\subsection{Tecnologia inovadora}

Dentro da sua classe, a máquina varredora compacta \BosFull{sdeux} distingue-se pelo baixo peso (2300\,kg), pela elevada capacidade (recipiente de sujidade da classe 2,0-m\high{3}), pelas dimensões compactas (1,15\,m de largura) e pelo posto de trabalho especialmente ergonómico para o operador do veículo.

Graças à construção estreita, a \sdeux\ é uma máquina varredora \quotation{universal}, que pode ser utilizada em estradas e passeios, em cidades e aldeias. O potente motor Diesel, conjugado com o compacto sistema de acionamento hidrostático (motores hidráulicos de êmbolos radiais para as rodas dianteiras), assegura constantemente a máxima mobilidade, independentemente das condições do local de aplicação ou do nível de enchimento do recipiente de sujidade.

As bombas hidráulicas são acionadas por um motor Diesel tipo \aW{VW 2.0 CDI}, consoante a norma Euro V. Este transmite um binário de 285\,Nm a uma velocidade de rotação de 1750, e apresenta uma potência máxima de 75\,kW a uma velocidade de rotação de 3000. Deste modo, é possível utilizar a máquina de forma eficiente, mesmo com uma velocidade de rotação do motor baixa~– causando, assim, menos poluição sonora. A \sdeux\ está equipada de série com um filtro de partículas.


\section{Inovações ao serviço do cliente}

A direção articulada da \sdeux\ proporciona um círculo de viragem reduzido, garantindo assim a máxima mobilidade. Graças aos materiais especiais, como Domex®, e à projeção através de CAD do veículo, é possível chegar a uma carga útil significativa de 1200\,kg.

\placefig[margin][overview:cab:frontright]{\sdeux\ operacional}
{\externalfigure[overview:cab:twoleft][width=\Bildwidth]}

A cabina do operador, com janelas de vidro de todos os lados, está equipada com dois bancos confortáveis e cintos de segurança de três pontos. Opcionalmente, a \sdeux\ pode estar equipada com um sistema de climatização.

Uma vez que consegue atingir uma velocidade máxima de 40\,km/h, o veículo é perfeitamente adequado para a circulação na via pública. Graças às suspensões dos eixos dianteiros e traseiros é possível assegurar uma deslocação segura e confortável, mesmo em trajetos em mau estado.

O agregado de varredura~– montado sobre dois braços articulados~– encontra-se totalmente no campo de visão do operador e o bocal de aspiração está posicionado à frente do eixo dianteiro, de forma totalmente visível. Como equipamento adicional está disponível uma escova dianteira dupla basculante.

\page [yes]


\subsection{Cabina do operador anecoica e confortável}

A cabina do operador\index{Cabina do operador} da \sdeux\ está equipada com direção à direita e foi concebida para duas pessoas. A cabina possui isolamento acústico e está montada sobre blocos-amortecedores antivibração.

As portas e o fundo são de vidro; deste modo, o operador tem um campo de visão abrangente. O para-brisas ocupa toda a parte dianteira do veículo; deste modo, o operador tem uma visibilidade total sobre as escovas.

O banco do operador está equipado com uma suspensão mecânica ou~– opcionalmente~– pneumática. O banco do operador e o banco do passageiro estão montados sobre corrediças ajustáveis.


\subsubsubject{Ergonomia}

\startfigtext[right][overview:joy:sideview]{Consola de comando}
{\externalfigure[overview:joy:top]}
Graças à consola multifunções, do lado esquerdo do banco do operador, é possível realizar todas as funções elementares com apenas uma mão. As duas escovas podem ser comandadas individualmente, através de dois Joysticks, com o polegar e o indicador. Os interruptores para o comando das escovas e da escova dianteira (opção), da velocidade de rotação do motor, do Cruise Control, etc., também se encontram na consola multifunções.
\stopfigtext

Na parte inferior do campo de visão do operador do veículo encontra-se um ecrã tátil, no qual são exibidas em tempo real todas as funções importantes da máquina, sem prejudicar a visibilidade para o exterior.

\placefig[margin][overview:vhc:left]{\sdeux\ a trabalhar junto de edifícios históricos}
% \placefig[margin][overview:vhc:left]{\sdeux\ sur site historique}
{\externalfigure[overview:vhc:left]}

\page [yes]


\subsubsubject{Posto do operador}

A\index{Posto do operador} alavanca de seleção da velocidade de marcha (\quotation{Mudança de velocidade}) encontra-se do lado direito da coluna de direção; estão disponíveis duas velocidades de marcha à frente e uma velocidade de marcha-atrás. Na parte exterior da alavanca de seleção da velocidade de marcha encontra-se um botão para comutar entre os dois modos \aW{Trabalho} e \aW{Marcha}. Não é necessário parar a \sdeux\ para comutar entre os modos. (Para mais informações, ver capítulo \about[sec:using:work], \atpage[sec:using:work].)

\placefig[margin][fig:overview:steeringwheel]{Posto do operador}
{\externalfigure[overview:driver:place]}

Durante a marcha-atrás, o monitor da câmara de marcha-atrás liga e soa um sinal acústico de aviso (que pode ser desativado através do Vpad).

Na alavanca multifunções, do lado esquerdo da coluna de direção, encontra-se o interruptor de comando do limpa para-brisas (duas velocidades e intervalo), assim como o alarme luminoso e acústico.

Para mais informações relativas a estas e outras funções da \sdeux, ver capítulo \about[chap:using], a partir da \atpage[chap:using].

\page [yes]

\setups[pagestyle:marginless]


\subsection[overview:brushsystem]{Máquina varredora e de aspiração}

\subsubsubject{Escova}

\startfigtext[left][fig:overview:steeringwheel]{Máquina varredora e de aspiração}
{\externalfigure[system:brush]}
As escovas\index{Varredura} estão colocadas sobre cabeças ajustáveis que, por sua vez, estão montadas sobre braços articulados. Para eliminar a poeira que se forma durante a varredura, é pulverizada água: cada escova possui dois bocais; estes recolhem a água a partir do depósito de água limpa ou do depósito de água de reciclagem.

Através de um interruptor\index{Aspiração} da consola multifunções são ativadas, em simultâneo, as escovas e a bomba de água.\footnote{Para informações relativas à bomba de água, ver capítulo \in[chap:using] \about[chap:using], especialmente \about[sec:using:work], \atpage[sec:using:work].}
Através do respetivo Joystick da consola multifunções é possível comandar a posição das escovas, assim como a inclinação transversal e longitudinal.
\stopfigtext

As escovas estão protegidas por um sistema anticolisão mecânico e hidráulico.


\subsubsubject{Bocal de aspiração}

Na posição de trabalho (descido), o bocal de aspiração está colocado sobre 4~rolos, cobrindo totalmente a superfície entre as escovas estendidas. Graças à sua posição \quotation{arrastada}, em caso de colisão com obstáculos, o bocal de aspiração está protegido contra danos mecânicos. Durante a marcha-atrás, o bocal de aspiração é automaticamente levantado.

Um lábio de borracha grosso e substituível garante uma vedação estanque relativamente à estrada. Graças a uma tampa de comando eletro-hidráulico na parte dianteira do bocal de aspiração, é possível aspirar sujidades maiores.


\subsubsubject{Recipiente de sujidade}

O recipiente de sujidade de alumínio pode ser rebatido para cima, em até 50° e até uma altura de 1,5\,m (altura de descarga). Nele desemboca um canal de aspiração, vindo de baixo, com um diâmetro de abertura de 180\,mm.

A pressão de admissão é criada por uma turbina de potência máxima, que está montada horizontalmente no recipiente de sujidade. A turbina de potência máxima possui uma tampa de manutenção para efetuar a limpeza e o controlo visual.

Na tampa de fecho do recipiente de sujidade encontram-se duas grelhas de aspiração de aço inoxidável. Estas podem ser rebatidas (abertas) sem ferramentas. A tampa de fecho pode ser desbloqueada e aberta manualmente.

Através de uma tampa, que pode ser movimentada manualmente, a corrente de ar pode ser facilmente comutada entre o canal de aspiração e a mangueira de aspiração manual (equipamento opcional).


\subsection{Dispositivo humidificador}

\subsubsubject{Sistema de água limpa}

O\index{Varredura+Humidificação} depósito de fundição PE encontra-se atrás da cabina do operador, em posição vertical. Tem uma capacidade\index{Depósito de+ Água limpa} de 190\,l.

Uma bomba elétrica (6,5\,l/min) transporta a água para os bocais de pulverização de cada escova (incluindo para a terceira escova, opcionalmente disponível).


\subsubsubject{Reciclagem de água suja}

A água suja flui através das microperfurações das paredes interiores do recipiente de água suja e depois através da tampa de reciclagem para o depósito de água de reciclagem que se encontra por baixo. O\index{Depósito de+ Água de reciclagem} depósito de água de reciclagem tem uma capacidade de 140\,l.

Uma bomba hidráulica submersível transporta a água para os bocais de pulverização no interior do bocal de aspiração e do canal de aspiração.


\testpage [8]
\subsubsubject{Depósito de água de reciclagem}

O depósito de água de reciclagem possui um permutador de calor de fluido hidráulico e água com função dupla:

\startitemize[width=35mm,style=\md, command={\setupwhitespace[small]}]
\sym{Função no verão} A água conduz o calor do fluido hidráulico através de convecção para as paredes de alumínio do depósito, a partir de onde o calor é irradiado para o ar ambiente.

\sym{Função no inverno} O fluido hidráulico aquece a água no depósito. Deste modo, é possível
pulverizar o canal de aspiração e o bocal de aspiração mesmo quando as temperaturas estão ligeiramente abaixo do ponto de congelação.
\stopitemize


\subsubsubject{Monitorização dos níveis de enchimento de água}

\startitemize[width=35mm,style=\md, command={\setupwhitespace[small]}]
\sym{Água limpa} Se o nível de enchimento não for suficiente, é exibido o símbolo~\textSymb{vpad_water} no ecrã do Vpad.
\sym{Água de reciclagem} Se o nível de enchimento do depósito de reciclagem estiver abaixo do permutador de calor (ver em cima), é exibido o símbolo~\textSymb{vpad_rwater_orange} (amarelo) no ecrã do Vpad. Se o nível de enchimento não for suficiente, é exibido o símbolo~\textSymb{vpad_rwater} (vermelho).
\stopitemize


\subsubsubject{Pneus "wide base" (opção)}

A pressão sobre o solo\index{Pneus "wide base"} corresponde à pressão de enchimento dos pneus. Com uma pressão dos pneus de 1,8\,bar é alcançada uma pressão sobre o solo de 18\,N/cm². No entanto, a capacidade de carga do pneu para a carga por eixo assegurada já não é alcançada. Com 1,8\,bar, a uma velocidade de 40\,km/h, apenas é possível assegurar uma carga por eixo de 1495\,kg. Se for selecionada uma pressão dos pneus diferente de 3,0\,bar, a responsabilidade é do proprietário do veículo.

\subsubsubject{Indicação de sobrecarga (opção)}

Se o veículo\index{Indicação de sobrecarga} for sobrecarregado, é exibida uma mensagem no Vpad. A sobrecarga é determinada através de um sensor de ângulo no eixo traseiro. A indicação de carga está ajustada, de fábrica, em 3500\,kg; no entanto, evitar um campo de tolerância deste valor. Este ajuste de 3500\,kg pode ser modificado por uma empresa especializada.

\page [yes]
\setups[pagestyle:normal]


\section{Identificação do veículo}

\subsection{Placa de características do veículo}

A placa de características do veículo\index{Identificação+Veículo} encontra-se na cabina do operador, no lado oposto da consola, por baixo do banco do passageiro (ver \inF[fig:identity:location], \atpage[fig:identity:location]).


\subsection{Código e número do motor}

O código do motor encontra-se na placa de características do motor (autocolante), na conduta metálica em forma de cotovelo do circuito de refrigeração, na parte dianteira do motor (levantar o recipiente de sujidade).

O número do motor está gravado no motor (\inF[identity:engine:number]). Este é constituído por nove carateres alfanuméricos: os primeiros três carateres são o código do motor; os seis carateres seguintes são o número de série do motor.


\placefig[margin][idvhc]{Placa de características do veículo}
{\externalfigure[s2:id:plaque]}

\placefig[margin][identity:engine:code]{Placa de características do motor}
{\externalfigure[engine:id:code]}

\placefig[margin][identity:engine:number]{Número do motor}
{\externalfigure[engine:id:number]}

\page [yes]


\subsection [sec:plateWheel]{Placa de características das rodas}

A placa de características das jantes e dos pneus\index{Pneus+Pressão de enchimento} encontra-se na cabina do operador\index{Jantes+Dimensões}, por baixo do banco do passageiro.


\subsection{Número do quadro}

O número do quadro\index{Identificação+Número do quadro} (número do chassis) está gravado do lado direito do veículo, por baixo da cabina do operador, no quadro.


\subsection{\symbol[europe][CEsign]-Conformidade e marca}

A marca de conformidade~\symbol[europe][CEsign] encontra-se na cabina do operador, no lado oposto da consola, por baixo do banco do passageiro.

A \sdeux\ cumpre os requisitos básicos de segurança e saúde da diretiva de máquinas\index{Certificado+Conformidade CE}\index{Diretiva de máquinas} 2006/42/CE\footnote{Diretiva 2006/42/CE do Parlamento Europeu e do Conselho de 17 de~maio de 2006}.
% \textrule

\placefig[margin][idpneus]{Pressão de enchimento dos pneus}
{\externalfigure[identity:tires]}

\placefig[margin][fig:identity:location]{Placas de características}
{\externalfigure[identity:location]}

\page [yes]
\setups [pagestyle:marginless]


\startsection[title={Dados técnicos},
							reference={donnees_techniques}]

\subsection [sec:measurement] {Dimensões do veículo}

\placefig[here][fig:measurement]{\select{caption}{Largura~– Escova em posição de repouso ou estendida~–, Comprimento e altura do veículo}{Dimensões do veículo}}
{\Framed{\externalfigure[s2:measurement]}}

\page [yes]

\placefig[here][fig:measurement]{\select{caption}{Altura do veículo com recipiente de sujidade rebatido para cima}{Altura do veículo}}
{\Framed{\externalfigure[s2:measurement:02]}}

\page [yes]

\starttabulate [|lBw(45mm)|p|l|rw(35mm)|]
\FL
\NC Grupo\index{Medidas} \NC \bf Medida \NC \bf Unidade \NC \bf Valor \NC\NR
\ML
\NC Dimensões do veículo \NC Comprimento (total) \NC \unite{mm} \NC 4588,00 \NC\NR
\NC\NC Comprimento com 3.ª\,escova \NC \unite{mm} \NC 5020,00 \NC\NR
\NC\NC Largura do veículo \NC \unite{mm} \NC 1150,00 \NC\NR
\NC\NC Largura do veículo (total) \NC \unite{mm} \NC 1575,00 \NC\NR
\NC\NC Altura sem farolim rotativo \NC \unite{mm} \NC 1990,00 \NC\NR
\NC\NC Distância entre eixos \NC \unite{mm} \NC 1740,00 \NC\NR
\NC\NC Largura da via \NC \unite{mm} \NC 894,00 \NC\NR
\ML
\NC Largura de varredura \NC Escova padrão \NC \unite{mm} \NC 2300,00 \NC\NR
\NC\NC Com 3.ª\,escova \NC \unite{mm} \NC 2600,00 \NC\NR
\NC\NC Diâmetro da escova \NC \unite{mm} \NC 800,00 \NC\NR
\NC\NC Largura do bocal de aspiração \NC \unite{mm} \NC 800,00 \NC\NR
\ML
\NC Distribuição da carga \NC Peso vazio\note[weight:empty] eixo dianteiro \NC \unite{kg} \NC aprox. 1100,00 \NC\NR
\NC\NC Peso vazio\note[weight:empty] eixo traseiro \NC \unite{kg} \NC aprox. 1200,00 \NC\NR
\NC\NC Peso vazio\note[weight:empty] \NC \unite{kg} \NC aprox. 2300,00 \NC\NR
\NC\NC Peso total admissível \NC \unite{kg} \NC 3500,00 \NC\NR
\LL
\stoptabulate

%% TODO en/de/fr: Footnote on preceeding page
\footnotetext[weight:empty]{Configuração padrão, com operador (aprox. 75\,kg).}

\page [yes]

\adaptlayout [height=+5mm]

\subsection{Raio de via e raio de varredura}

\starttabulate [|lBw(45mm)|p|l|rw(35mm)|]
\FL
\NC Dimensão\index{Dimensões} \NC \bf Medida \NC \bf Unidade \NC \bf Valor \NC\NR
\ML
\NC Raio de via\index{Raio de via}\index{Medida+Raio de via} \NC Raio de viragem mínimo com escova \NC \unite{mm}	\NC 3325,00 \NC\NR
\ML
\NC Raio de varredura \NC Exterior \NC \unite{mm} \NC 3425,00 a 3850,00 \NC\NR
\NC\NC Interior \NC \unite{mm} \NC 2025,00 a 1675,00 \NC\NR
\LL
\stoptabulate

\start

\setupcaption [fig]	[%
% style=\stdfontsemicn\itx\setupinterlinespace,
% headstyle=\stdfontsemicn\mdx\setupinterlinespace,
% width=max,
% inbetween={\blank[small]},
align=center,
]

\placefig[here][pict:steerin_sweeping:radius]{Raio de via|/|viragem e raio de varredura}
{\externalfigure[steerin_sweeping:radius]}

\stop
\page [yes]


\subsection{Rodas e pneus}

{\sla Dimensões padrão}

\starttabulate[|lBw(45mm)|p|rw(55mm)|]
\FL
\NC Componente \NC \bf Equipamento \NC \bf Valor \NC\NR
\ML
\NC Pneus \NC Dimensões padrão \NC 205/70 R 15 C \NC\NR
\ML
\NC Jantes \NC Dimensões padrão \NC 6J\;×\;15 H2 ET 60 \NC\NR
\ML
\NC Pressão de enchimento dos pneus\index{Pressão de enchimento dos pneus} \NC Padrão, dianteiro|/|traseiro \NC 4,5|/|5,8\,bar \NC\NR
\LL
\stoptabulate

{\sla Pneus "wide base"}

\starttabulate[|lBw(45mm)|p|rw(55mm)|]
\FL
\NC Componente \NC \bf Equipamento \NC \bf Valor \NC\NR
\ML
\NC Pneus\index{Pneus "wide base"} \NC Pneus "wide base" \NC 275/60 R 15 107H \NC\NR
\ML
\NC Jantes \NC Pneus "wide base" \NC 8LB\;×\;15 ET 30 \NC\NR
\ML
\NC Pressão de enchimento dos pneus\index{Pressão de enchimento dos pneus} \NC Padrão, dianteiro|/|traseiro \NC 3,0|/|3,0\,bar \NC\NR
\LL
\stoptabulate


\subsection{Motor Diesel}

\starttabulate [|lBw(45mm)|l|rp|]
\FL
\NC \bf Grupo\index{Motor Diesel+Identificação} \NC \bf Parâmetro \NC \bf Valor\NC\NR
\ML
\NC Tipo de motor \NC \NC VW CJDA TDI 2.0 – 475 NE \NC\NR
\NC Geral \NC 	Temporização \NC Motor de quatro tempos \NC\NR
\NC\NC Número de cilindros \unite{n} \NC 4 \NC\NR
\NC\NC Diâmetro x curso \unite{mm} \NC 81\;×\;95,5 \NC\NR
\NC\NC Cilindrada total \unite{cm\high{3}} \NC 1968 \NC\NR
\NC\NC Válvulas por cilindro \NC 4 \NC\NR
\NC\NC Sequência do comando das válvulas \NC 1-3-4-2 \NC\NR
\NC\NC Velocidade de rotação em vazio mín. \unite{min\high{−1}} \NC 830 +50/−25 \NC\NR
\NC Potência|/|binário \NC Velocidade de rotação máx. \unite{min\high{−1}} \NC 3400 \NC\NR
\NC\NC Potência máx. \unite{kW} a \unite{min\high{−1}} \NC 75 a 3000 \NC\NR
\NC\NC Binário máx. \unite{Nm} a \unite{min\high{−1}} \NC 285 a 1750 \NC\NR
\NC Consumo específico\index{Motor Diesel+Consumo} \NC Combustível \unite{g/kWh} \NC 224 (à potência máx.) \NC\NR
\NC\NC Óleo \unite{g/kWh} \NC 0,22 \NC\NR
\NC Sistema de combustível \NC Sistema de injeção \NC Injeção direta \quote{Common Rail} \NC\NR
\NC\NC Alimentação de combustível \NC Bomba de engrenagem \NC\NR
\NC\NC Sobrealimentação \NC Sim \NC\NR
\NC\NC Arrefecimento do ar de sobrealimentação \NC Sim \NC\NR
\NC\NC Pressão de sobrealimentação \unite{mbar} \NC 1300\NC\NR
\NC Circuito de lubrificação\index{Motor Diesel+Lubrificação} \NC Tipo \NC Lubrificação forçada com radiador de óleo|/|água \NC\NR
\NC\NC Alimentação da conduta \NC Bomba de rotor \NC\NR
\NC\NC Consumo de óleo \unite{Litros/20\,h} \NC <\:0,1 \NC\NR
\NC Circuito de refrigeração\index{Motor Diesel+Refrigeração} \NC Capacidade total \unite{l} \NC aprox. 12 \NC\NR
\NC\NC Pressão de calibração do reservatório de expansão \unite{bar} \NC 1,4 \NC\NR
\NC\NC Termóstato (abertura) \unite{°C} \NC 87 \NC\NR
\NC\NC Termóstato (cheio) \unite{°C} \NC 102 \NC\NR
\NC Gás de escape \NC Filtro de partículas \NC Sim \NC\NR
\NC\NC Tratamento do gás de escape \NC Sim \NC\NR
\NC\NC Norma \NC Euro 5 \NC\NR
\LL
\stoptabulate


\subsection{Desempenhos}

\starttabulate[|lBw(45mm)|p|l|rw(35mm)|]
\FL
\NC Desempenho\index{Desempenhos} \NC \bf Configuração \NC \bf Unidade \NC \bf Valor \NC\NR
\ML
\NC Velocidade \NC Modo \aW{Trabalho} \NC \unite{km/h} \NC 0 a 18 (contínuo) \NC\NR
\NC\NC \aW{Modo Marcha} \NC \unite{km/h} \NC 0 a 40 \NC\NR
\ML
\NC Limitação de velocidade \NC Ajustável \NC \unite{km/h} \NC 0 a 25 \NC\NR
\LL
\stoptabulate


\subsection{Instalação elétrica}

{\starttabulate [|lw(65mm)|p|rw(30mm)|]
\FL
\NC \bf Grupo \NC \bf Componente \NC \bf Valor \NC\NR
\ML
\NC Bateria \NC Bateria de chumbo-ácido \NC 12\,V 75\,Ah \NC\NR
\NC Alimentação elétrica \NC Dínamo \NC 14,8\,V 140\,A \NC\NR
\NC Motor de arranque \NC Potência \NC 1,8\,kW \NC\NR
\NC Equipamento áudio \NC Conexão de rádio\index{Conexão de rádio} e altifalante\index{Altifalante} \NC Equipamento de série \NC\NR
% \NC Sécurité et surveillance \NC Tachygraphe\index{tachygraphe} \NC En option \NC\NR
% \NC\NC Enregistreur de fin de parcours\index{fin de parcours} \NC En option \NC\NR
\NC Dispositivos de iluminação/sinalização dianteira \NC Luz de presença \NC 12\,V 5\,W \NC\NR
\NC\NC Luz de cruzamento (médios) \NC H7, 12\,V 55\,W \NC\NR
\NC\NC Faróis de trabalho \NC G886, 12\,V 55\,W \NC\NR
\NC\NC Luzes indicadoras de mudança de direção \NC 12\,V 21\,W \NC\NR
\NC Dispositivos de iluminação/sinalização traseira \NC Luzes de travagem combinadas \NC 12\,V 5/21\,W \NC\NR
\NC\NC Luzes indicadoras de mudança de direção \NC 12\,V 21\,W \NC\NR
\NC\NC Luzes de marcha-atrás \NC 12\,V 21\,W \NC\NR
\NC\NC Luz de matrícula \NC 12\,V 5\,W \NC\NR
\NC Iluminação adicional \NC Farolim rotativo \NC H1, 12\,V 55\,W \NC\NR
\LL
\stoptabulate
}
\stopsection

\stopcomponent
