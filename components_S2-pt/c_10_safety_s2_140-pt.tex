\startcomponent c_10_safety_s2_140-pt
\product prd_ba_s2_140-pt

\marking[chapter]{Símbolos de segurança}


\chapter{Símbolos de segurança}

\setups[pagestyle:marginless]

\section{Novos símbolos de perigo europeus}

{\em Forma losangular com fundo branco e margem vermelha.}\par\blank[1*medium]
{\em Desde 2008, vigora na UE o regulamento CLP\index{Regulamento CLP}, com novos símbolos de atenção que alertam para substâncias e produtos perigosos.}\par\null

\startSymList \GHSgeneric
\SymList
	\textDescrHead{Perigo para a saúde}
	Alerta para\index{Perigo para a saúde} perigos para a saúde que não resultam em morte ou danos corporais graves. P. ex., irritações da pele ou alergias. Este símbolo é também utilizado para alertar para outros perigos, como inflamabilidade.\par
	Substitui:\crlf \HAZOcross\ ou \HAZOpoison\ ou \PHgeneric
\stopSymList

\startSymList \GHSbody
\SymList
	\textDescrHead{Perigo grave para a saúde; nas crianças pode até resultar em morte}
	Os produtos que contêm este símbolo podem ter consequências nefastas para a saúde. Este símbolo é também utilizado para alertar para perigos\index{Perigo+Gravidez} inerentes à gravidez, substâncias cancerígenas\index{Perigo+Substâncias cancerígenas} e outros riscos graves para a saúde. Os produtos que contêm este símbolo devem ser utilizados com cuidado.\par
	Substitui:\crlf \HAZOcross\ ou \HAZOpoison\
\stopSymList

\startSymList \GHSbomb
\SymList
	\textDescrHead{Substâncias explosivas}
	Quando reagem, as substâncias explosivas\index{Perigo+Explosão} instáveis,
	misturas e produtos que contêm substâncias explosivas\index{Substâncias explosivas}, apresentam um efeito violento e expandido, que pode causar estragos significativos; em caso de utilização inadequada dos mesmos, existe perigo de vida.\par
	Substitui:\crlf \HAZObomb\
\stopSymList


\startSymList \GHSpoison
\SymList
	\textDescrHead{Intoxicação}
	Se os produtos\index{Perigo+Intoxicação} que contêm este símbolo entrarem em contacto com a pele, forem inalados\index{Substâncias tóxicas} ou ingeridos - mesmo que em quantidades pequenas -, podem provocar intoxicações graves ou até mortais. Não permitir o contacto direto com os produtos.\par
	Substitui:\crlf \HAZOpoison\
\stopSymList

\startSymList \GHSfire
\SymList
	\textDescrHead{Facilmente ou extremamente inflamável}
	Os produtos\index{Perigo+Fogo} que contêm este símbolo inflamam rapidamente na proximidade de calor ou chamas. Os sprays que contêm este símbolo não podem ser pulverizados sobre superfícies quentes ou na proximidade de chamas.\par
	Substitui:\crlf \HAZOfire\ ou \HAZOfirebis\
\stopSymList

\startSymList \GHSenvironment
\SymList
	\textDescrHead{Perigo para animais e meio ambiente}
	Os produtos\index{Proteção do meio ambiente} que contêm este símbolo podem causar danos\index{Substâncias tóxicas} temporários ou permanentes no meio ambiente. Podem causar a morte de organismos que habitam em meios aquáticos (\eG\ peixes) ou causar danos permanentes no meio ambiente. Não deitar estes produtos na água residual ou no lixo doméstico!\par
	Substitui:\crlf \HAZOenvironment\
\stopSymList

\startSymList \GHScorrosive
\SymList
	\textDescrHead{Perigo para pele ou olhos}
	Os produtos\index{Perigo+Lesões cutâneas}\index{Perigo+Lesões oculares} que contêm este símbolo podem causar lesões cutâneas e cicatrizes ou lesões oculares, mesmo em caso de contacto fugaz. Proteger sempre a pele e os olhos durante o manuseio destes produtos!\par
	Substitui:\crlf \HAZOcross\ ou \HAZOcorrosive
\stopSymList

\page [yes]


\section{Símbolos de atenção}

{\em Letras pretas sobre fundo amarelo}\par\null

\startSymList \PHgeneric
\SymList
	\textDescrHead{Símbolo de atenção geral}
	Alerta\index{Perigo+Símbolo de atenção}\index{Geral} para um perigo iminente que pode resultar em ferimentos para o utilizador ou terceiros.
	\crlf\null
\stopSymList

\startSymList \PHpoison
\SymList
	\textDescrHead{Símbolo de atenção para substâncias tóxicas}
	Se as substâncias tóxicas\index{Perigo+Intoxicação} entrarem em contacto com a pele, forem inaladas ou ingeridas, podem provocar danos agudos ou crónicos graves para a saúde, ou até morte.
\stopSymList

\startSymList \PHfire
\SymList
	\textDescrHead{Símbolo de atenção para substâncias inflamáveis}
	Evitar chamas abertas ou formação de faíscas\index{Perigo+Fogo}. As substâncias que contêm este símbolo são facilmente inflamáveis e podem agir como um acelerador de incêndio. Proibido fumar!
\stopSymList

\startSymList \PHexplosive
\SymList
	\textDescrHead{Símbolo de atenção para substâncias explosivas}
	Substâncias ou preparações sólidas, líquidas ou em gel que, em caso de embate, fricção, fogo, calor, etc.\,, podem resultar em explosão.\index{Perigo+Explosão} Proibido fumar!
\stopSymList

\startSymList \PHcrushing
\SymList
	\textDescrHead{Símbolo de atenção para perigo de esmagamento}
	Alerta para uma área\index{Perigo+Esmagamento} onde existe perigo de esmagamento, devido aos componentes mecânicos móveis. Manter-se afastado desta 	área enquanto o dispositivo estiver em funcionamento.
\stopSymList

\startSymList \PHhand
\SymList
	\textDescrHead{Símbolo de atenção para ferimentos nas mãos}
	Existe o perigo de\index{Perigo+Esmagamento} as mãos ou outras partes do corpo\index{Perigo+Ferimentos nas mãos} serem esmagadas, \eG\ durante o rebatimento da cabina do operador ou da plataforma.
\stopSymList

\startSymList \PHentangle
\SymList
	\textDescrHead{Símbolo de atenção para rolos em sentido contrário / para perigo de colhimento}
	Existe o perigo de os membros do corpo\index{Perigo+Colhimento} serem colhidos pelos componentes rotativos. Manter-se afastado enquanto o dispositivo estiver em funcionamento.
\stopSymList

\startSymList \PHcorrosive
\SymList
	\textDescrHead{Símbolo de atenção para substâncias corrosivas}
	Manusear com cuidado\index{Perigo+Substâncias corrosivas}; utilizar equipamento de proteção pessoal adequado (luvas, óculos de proteção, vestuário de proteção).
\stopSymList

\startSymList \PHhot
\SymList
	\textDescrHead{Símbolo de atenção para superfícies quentes}
	Não se aproximar do componente ou do dispositivo\index{Perigo+Queimaduras} se não se possuir os conhecimentos suficientes; utilizar luvas.
\stopSymList

\startSymList \PHvoltage
\SymList
	\textDescrHead{Símbolo de atenção para tensão elétrica perigosa}
	Não tocar com objetos metálicos\index{Perigo+Tensão elétrica}.
	Perigo de ferimentos ou queimaduras em caso de curto-circuito!
\stopSymList

\startSymList \PHfalling
\SymList
	\textDescrHead{Símbolo de atenção para perigo de queda}
	Nesta área, agir com especial\index{Perigo+Queda} prudência; utilizar calçado adequado (com sola antiderrapante, resistente a hidrocarbonetos).
\stopSymList

\startSymList \PHbattery
\SymList
	\textDescrHead{Símbolo de atenção para perigos causados por baterias} Alerta para perigos que surgem ao carregar as baterias (bateria de chumbo-ácido)\index{Perigo+Bateria}, causados sobretudo pelo vazamento de hidrogénio gasoso e pelo ácido sulfúrico contido nas baterias.
\stopSymList

\startSymList \PHremote
\SymList
	\textDescrHead{Símbolo de atenção para arranque automático}
	Alerta para\index{Perigo+Arranque automático} um possível arranque automático ou controlado à distância de um dispositivo.
\stopSymList

% \startSymList \PHquetschgefahr
% \SymList
% \textDescrHead{Risque d’écrasement}
% Risque d’écrasement\index{risque d’écrasement}.
% \stopSymList
% % NOTE: Doppelt! (auch Bilddatei)
%
% % NOTE: Evtl. Folgendes als Ersatz für oben?

% \startSymList\PHhandcrushed
% \SymList
	% \textDescrHead{Gefahr von Handquetschungen}
	% Es besteht\index{Gefahr+Quetschung} die Gefahr, dass Hände oder Finger
	% gequetscht werden. Nähern Sie die Hände nicht an, ohne die Gefahr
	% identifiziert und beseitigt zu haben.
% \stopSymList

\startSymList \PHhandfoot
\SymList
\textDescrHead{Símbolo de atenção para componentes móveis}
Alerta para componentes móveis da máquina/veículo
\index{Perigo+Componentes móveis}.
\stopSymList

\startSymList \PHnarrowed
\SymList
	\textDescrHead{Símbolo de atenção para via de deslocação estreita}
	Via de deslocação\index{Perigo+Largura do veículo} estreita.
	% Denken Sie an die Breite des Fahrzeugs.
\stopSymList

\page [yes]


\section{Símbolos de proibição}

{\em Circular, com fundo branco, margem e traço diagonal vermelho}
\par\null


\startSymList \PPfire
\SymList
	\textDescrHead{Proibido fazer fogo, chama aberta e fumar}Proibido
	fazer\index{Proibição+Fumar, fogo} chama aberta e fogo, de qualquer forma (\eG\ cigarro aceso, fósforo, vela; também formação de faíscas de qualquer tipo).
\stopSymList

\startSymList \PPentry
\SymList
	\textDescrHead{Proibido acesso a pessoas sem autorização}
	Pessoas\index{Proibição+Acesso} sem autorização não podem aceder, nem aproximar-se, desta área.
\stopSymList

\startSymList \PPphone
\SymList
	\textDescrHead{Proibido utilizar telefonia móvel}
	Telemóveis\index{Proibição+Telefonia móvel} e quaisquer aparelhos emissores de radiação eletromagnética devem estar desligados. A radiação eletromagnética pode causar falhas de funcionamento no sistema eletrónico do aparelho.
\stopSymList

\startSymList \PPspray
\SymList
	\textDescrHead{Proibido pulverizar com água}
	Nunca direcionar um jato de água ou vapor\index{Proibição+Jato de água, vapor} contra 	componentes ou aparelhos sensíveis (\eG\ sensores, aparelhos de comando, sistema de injeção, etc.).
\stopSymList

\startSymList \PPchildren
\SymList
	\textDescrHead{Manter as crianças afastadas}
	Aviso\index{Proibição+Crianças} relativo a especial perigo para crianças. Em geral, é válido: as crianças devem manter-se afastadas de uma máquina que esteja em funcionamento, mesmo durante os trabalhos de manutenção.
\stopSymList

\startSymList \PPwater
\SymList
	\textDescrHead{Água imprópria para consumo}
	Não beber a água do depósito\index{Proibição+Água imprópria para consumo}. Perigo de intoxicação.
\stopSymList

% \page [yes]


\section{Símbolo de proteção do meio ambiente}

\startSymList \PSrecycle
\SymList
	\textDescrHead{Reciclagem}
	Normas específicas para a eliminação correta de determinados resíduos.
\stopSymList

\startSymList \PSwelt
\SymList
	\textDescrHead{Proteção do meio ambiente}
	Aviso para as normas relativas à proteção do meio ambiente em vigor.
\stopSymList

\startSymList \PStrash[width=\PictoHeight,height=,]
\SymList
	\textDescrHead{Eliminar os resíduos de acordo com as normas}
	Para determinados resíduos, \eG\ baterias de chumbo-ácido, vigoram normas de eliminação especiais.
\stopSymList


\testpage[12]


\section{Símbolos de obrigação}


{\em Circular com fundo azul}\par\null

\startSymList \PMgeneric
\SymList
	\textDescrHead{Símbolo de obrigação geral}
	Este símbolo apenas pode ser utilizado em conjunto com um símbolo suplementar, que especifique a obrigação.
\stopSymList


\startSymList \PMrtfm
\SymList
	\textDescrHead{Observar as instruções de uso}
	Antes da colocação em funcionamento, observar impreterivelmente\index{Ler instruções de uso} as instruções relativas a este tema, de um determinado aparelho ou produto. As instruções de uso devem ser guardadas na cabina do operador, de modo a que estejam sempre
	acessíveis.
\stopSymList

\startSymList \PMproteyes
\SymList
	\textDescrHead{Utilizar proteção ocular}
	Durante os trabalhos em que existe perigo de lesões oculares, utilizar sempre
	proteção ocular\index{Proteção ocular}.
\stopSymList

\startSymList \PMprothands
\SymList
	\textDescrHead{Utilizar luvas de proteção}
	Durante os trabalhos em que existe perigo de ferimentos nas mãos, utilizar sempre luvas de proteção\index{Utilizar luvas de proteção}.
\stopSymList

\startSymList \PMprotears
\SymList
	\textDescrHead{Utilizar proteção auditiva}
	Utilizar sempre proteção auditiva\index{Perigo+Audição} (\eG na proximidade de um ventilador ou turbina em funcionamento).
\stopSymList

\startSymList \PMsafetybelt
\SymList
	\textDescrHead{Utilizar cinto de segurança} Para sua segurança, coloque\index{Cinto de segurança} o cinto de segurança.
\stopSymList

\section{Símbolos suplementares}

% \adaptlayout[height=+5mm]                                                 {{{

% \startSymList \SETshoe
% \SymList
% \textDescrHead{Port de chaussures de sécurité obligatoire}
% Le port de chaussures de sécurité est obligatoire\index{chaussures de sécurité}.
% \stopSymList
%
% \startSymList \SETglasses
% \SymList
% \textDescrHead{Port de lunettes des protection obligatoire}
% Le port de lunettes est obligatoire\index{lunette de protection}.
% \stopSymList
%
% \startSymList \SEToreillettes
% \SymList
% \textDescrHead{Port de casque obligatoire}
% Le port d’un casque de protection est \index{casque} obligatoire.
% \stopSymList
%
% \startSymList \SETgloves
% \SymList
% \textDescrHead{Port de gants de protection obligatoire}
% Le port de gants de protection est obligatoire\index{gants}.
% \stopSymList
%
% \startSymList \SETmainecrase
% \SymList
% \textDescrHead{Risque d’écrasement}
% Danger pour les mains\index{écrasement} et les pieds.
% \stopSymList
%
% \startSymList \SETgetriebe
% \SymList
% \textDescrHead{Risque de happement}
% Risque de happement par\index{happement} des pièces en rotation.
% \stopSymList
%
% \startSymList \SETradkeil
% \SymList
% \textDescrHead{Cale de roue}
% Sécuriser le véhicule contre toute mise\index{Cale de roue} en marche involontaire.
% \stopSymList
%}}}

\startSymList \SETfirstaid
\SymList
\textDescrHead{Primeiros socorros}
	Indica o local onde é guardado o kit de primeiros socorros. O rápido contacto dos serviços de socorro é uma parte fundamental dos primeiros socorros.\index{Primeiros socorros}\index{Chamada de emergência} Indicar aqui os contactos de emergência:
	\fillinrules[n=1]{\bf
	\framed[align=right,frame=off,offset=none,width=30mm]{Serviços de socorro}}
\fillinrules[n=1]{\bf
\framed[align=right,frame=off,offset=none,width=30mm]{Polícia}}
\fillinrules[n=1]{\bf
\framed[align=right,frame=off,offset=none,width=30mm]{Bombeiros}}
\stopSymList

\startSymList \SETbrandschutzzeichen
\SymList
\textDescrHead{Extintor de incêndios}
	Alguns aparelhos estão equipados com um ou vários extintores de incêndios\index{Extintor de incêndios}. Por norma, estes necessitam de ser submetidos a uma manutenção especial; para mais informações, ver o aparelho ou as instruções de uso do aparelho.
\stopSymList


\page[yes]

\section{Os três passos da prestação de assistência}
% NOTE [tf]: Shouldn't be in this book, IMO

\starttextbackground [CB]
\textDescrHead{Eliminar fontes de perigo do local do acidente e tomar medidas para garantir a segurança das pessoas envolvidas}
\startitemize
\item  Verificar se o local do acidente reúne as condições de segurança e assegurar que não surgem mais fontes de perigo.
\stopitemize
\textDescrHead{Averiguar o estado dos feridos}
\startitemize
\item  Verificar se os feridos estão conscientes e se respiram normalmente.
Event. desobstruir as vias respiratórias.
\stopitemize
\textDescrHead{Informar os serviços de socorro}
\startitemize Durante a chamada de emergência, deve indicar as seguintes informações:\par
	\item O número de telefone através do qual pode ser contactado.
	\item O tipo de ocorrência (doença, acidente).
	\item Riscos existentes (incêndio, explosão, desmoronamento, etc.).
	\item O local exato da ocorrência.
	\item O número de feridos e o estado em que estes se encontram.
	\item Medidas que já foram tomadas.
	\item Responda a quaisquer outras questões que lhe sejam colocadas.
\stopitemize
\stoptextbackground

\stopcomponent

