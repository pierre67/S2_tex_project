\startcomponent c_60_work_s2_140-pt
\product prd_ba_s2_140-pt


\startchapter [title={A S2 no dia a dia},
							reference={chap:using}]

\setups [pagestyle:marginless]


% \placefig[margin][fig:ignition:key]{Clé de contact}
% {\externalfigure [work:ignition:key]}
\startregister[index][chap:using]{Colocação em funcionamento}

\startsection [title={Colocação em funcionamento},
							reference={sec:using:start}]


\startSteps
\item Assegurar que os controlos e manutenções regulares foram executados de acordo com as normas.
\item Ligar o motor através da chave de ignição: ligar a ignição, continuar a rodar a chave no sentido horário e manter até o motor ligar (apenas possível se a alavanca de seleção da velocidade de marcha estiver na posição "Neutro").
\stopSteps

\start
\setupcombinations [width=\textwidth]

\placefig[here][fig:select:drive]{Alavanca de seleção da velocidade de marcha}
{\startcombination [2*1]
{\externalfigure [work:select:fDrive]}{Alavanca de seleção na posição {\em Marcha em frente}}
{\externalfigure [work:select:rDrive]}{Alavanca de seleção na posição {\em Marcha-atrás}}
\stopcombination}
\stop


\startSteps [continue]
\item Rodar o interruptor da alavanca de seleção da velocidade de marcha para engatar uma velocidade de marcha no modo {\em Marcha}:
\startitemize [R]
\item Primeira velocidade
\item Segunda velocidade (operação automática; liga automaticamente na primeira velocidade)
\stopitemize

ou pressionar o botão na parte exterior da alavanca, para ativar|/|desativar o modo {\em Trabalho}.
\stopSteps

\startbuffer [work:config]
\starttextbackground [FC]
\startPictPar
\PMrtfm
\PictPar
No modo Trabalho apenas está disponível a primeira velocidade e o motor trabalha a 1300\,min\high{\textminus 1}.

Controlar a velocidade de rotação do motor através dos botões~\textSymb{joy_key_engine_increase} e~\textSymb{joy_key_engine_decrease} da consola multifunções.
\stopPictPar
\stoptextbackground
\stopbuffer

\getbuffer [work:config]

\startSteps [continue]
\item Acionar a alavanca de seleção da velocidade de marcha para cima e para a frente (marcha em frente) ou para cima e para trás (marcha-atrás). Ver figura em cima.
\item Antes de acelerar, desengatar o travão de mão.
\stopSteps

\starttextbackground [FC]
\startPictPar
\PMrtfm
\PictPar
{\md Desengatar completamente o travão de mão!} A posição da alavanca do travão de mão é monitorizada por um sensor eletrónico: se o travão de mão não estiver completamente desengatado, a velocidade de marcha está limitada em 5\,km/h.
\stopPictPar
\stoptextbackground

\startSteps [continue]
\item Acionar lentamente o pedal do acelerador para colocar o veículo em movimento.
\stopSteps


%% NOTE: New text [2014-04-29]:
\subsection [sSec:suctionClap] {Tampa do canal de aspiração}

O sistema de aspiração gera uma corrente de ar, a partir do bocal de aspiração ou a partir da mangueira de aspiração manual (opção), para o recipiente de sujidade.

Através de uma tampa de acionamento manual (\inF[fig:suctionClap], \atpage[fig:suctionClap]) é possível comutar a corrente de ar entre o bocal de aspiração e a mangueira de aspiração manual.

\placefig [here] [fig:suctionClap] {Tampa do canal de aspiração}
{\startcombination [2*1]
{\externalfigure [work:suctionClap:open]}{Canal de aspiração aberto}
{\externalfigure [work:suctionClap:closed]}{Canal de aspiração fechado}
\stopcombination}

Na operação normal~– trabalhar com bocal de aspiração~– o canal de aspiração deve estar aberto (alavanca de comutação aponta para cima).

Para que a mangueira de aspiração manual possa ser utilizada, o canal de aspiração deve estar fechado (alavanca de comutação aponta para baixo). Deste modo, a corrente de ar é transportada através da mangueira de aspiração manual.
%% End new text

\stopsection


\startsection [title={Colocação fora de funcionamento},
							reference={sec:using:stop}]

\index{Colocação fora de funcionamento}

\startSteps
\item Acionar o travão de mão (alavanca entre os bancos) e ajustar a alavanca de seleção da velocidade de marcha na posição {\em Neutro}.
\item Executar os trabalhos de controlo necessários~– controlos diários e event. semanais~– conforme indicado na \atpage[table:scheduledaily].
\stopSteps

\getbuffer [prescription:handbrake]

\stopsection


\startsection [title={Varrer e aspirar},
							reference={sec:using:work}]

\startSteps
\item Executar a\index{Varredura} colocação em funcionamento do veículo conforme indicado na \in{§}[sec:using:start], \atpage[sec:using:start].
\item Ativar\index{Aspiração} o modo {\em Trabalho} (botão na parte exterior da alavanca de seleção da velocidade de marcha).
\stopSteps

% \getbuffer [work:config]
%% NOTE: outcommented by PB

\startSteps [continue]
\item Premir o botão~\textSymb{joy_key_suction_brush} para ligar a turbina e a escova.

{\md Variante:} {\lt premir o botão~\textSymb{joy_key_suction} para trabalhar apenas com o bocal de aspiração.}

\item Ajustar a velocidade de rotação das escovas através dos botões~\textSymb{joy_key_frontbrush_increase}\textSymb{joy_key_frontbrush_decrease} da consola multifunções.

\item Através do respetivo Joystick, colocar as escovas numa posição em que consigam alcançar a largura de trabalho ideal.
\stopSteps

\vfill

\start
\setupcombinations [width=\textwidth]

\placefig[here][fig:brush:position]{Posicionamento das escovas}
{\startcombination [2*1]
{\externalfigure [work:brushes:enlarge]}{Escovas para fora|/|para dentro}
{\externalfigure [work:brush:left:raise]}{Escovas para cima|/|para baixo}
\stopcombination}
\stop

\page [yes]


\subsubsubject{Humidificação das escovas e do canal de aspiração}

Acionar\index{Varredura+Humidificação} o interruptor~\textSymb{temoin_busebalais} entre os bancos:

{\md Posição 1:} a bomba de água trabalha automaticamente enquanto as escovas estiverem ativas.

{\md Posição 2:} a bomba de água trabalha permanentemente. (Útil \eG\ para trabalhos de ajuste.)


\subsubsubject{Sujidade grossa}

\startSteps [continue]
\item Caso exista o risco de sujidades maiores (\eG\ garrafas de plástico PET) bloquearem o bocal de aspiração, abrir\index{Tampa de sujidade grossa} a tampa de sujidade grossa através dos botões laterais da consola multifunções, ou~– se isso não for suficiente~– levantar temporariamente\index{Bocal de aspiração+Sujidade grossa} o bocal de aspiração.
\stopSteps

\start
\setupcombinations [width=\textwidth]

\placefig[here][fig:suctionMouth:clap]{Procedimento em caso de sujidade grossa}
{\startcombination [2*1]
{\externalfigure [work:suction:open]}{Abrir a tampa de sujidade grossa}
{\externalfigure [work:suction:raise]}{Levantar temporariamente o bocal de aspiração}
\stopcombination}
\stop

\stopsection


\startsection [title={Esvaziar o recipiente de sujidade},
							reference={sec:using:container}]

\startSteps
\item Deslocar\index{Recipiente de sujidade+Esvaziar} o veículo até um local adequado para o esvaziamento. Prestar atenção para que as normas relativas à proteção do meio ambiente em vigor sejam cumpridas.
\item Acionar o travão de mão e ajustar a alavanca de seleção da velocidade de marcha na posição {\em Neutro}. (Necessário para liberar o interruptor de esvaziamento do recipiente).
\stopSteps

\getbuffer [prescription:container:gravity]

\startSteps [continue]
\item Desbloquear e abrir a tampa de fecho do recipiente de sujidade.
\item Acionar o interruptor~\textSymb{temoin_kipp2} (consola central, entre os bancos) para abrir o recipiente de sujidade.
\item Quando o recipiente estiver vazio, lavar o interior com um jato de água. Para tal, pode-se utilizar a pistola de água integrada (equipamento opcional).
\stopSteps

\start
\setupcombinations [width=\textwidth]
\placefig[here][fig:brush:adjust]{Manuseio do recipiente de sujidade}
{\startcombination [3*1]
{\externalfigure [container:cover:unlock]}{Bloqueio da tampa de fecho}
{\externalfigure [container:safety:unlocked]}{Barra de segurança}
{\externalfigure [container:safety:locked]}{Barra de segurança bloqueada}
\stopcombination}
\stop

\startSteps [continue]
\item Verificar|/|limpar os vedantes e as superfícies de contacto dos vedantes do recipiente, do sistema de reciclagem e do canal de aspiração.
\stopSteps

\getbuffer [prescription:container:tilt]

\startSteps [continue]
\item Acionar o interruptor~\textSymb{temoin_kipp2} para descer o recipiente de sujidade. (Event. remover previamente as barras de segurança dos cilindros hidráulicos.)
\item Bloquear a tampa de fecho do recipiente de sujidade.
\stopSteps

\stopsection


\startsection [title={Mangueira de aspiração manual},
							reference={sec:using:suction:hose}]

A \sdeux\ pode ser opcionalmente\index{Mangueira de aspiração manual} equipada com uma mangueira de aspiração manual. Esta está fixada na tampa de fecho do recipiente de sujidade; a sua operação é simples.

{\sla Requisitos:}

O recipiente de sujidade está completamente descido; a \sdeux\ encontra-se no modo {\em Trabalho}. (Ver \in{§}[sec:using:start], \atpage[sec:using:start].)

\startfigtext[left][fig:using:suction:hose]{Mangueira de aspiração manual}
{\externalfigure[work:suction:hose]}
\startSteps
\item Premir o botão~\textSymb{temoin_aspiration_manuelle} da consola do tejadilho para ativar o sistema de aspiração.
\item Antes de abandonar a cabina do operador, acionar bem o travão de mão.
\item Fechar o canal de aspiração através da tampa do canal de aspiração. (Ver \in{§}[sSec:suctionClap], \atpage[sSec:suctionClap].)
\item Retirar a mangueira de aspiração manual do seu suporte, puxando o adaptador bucal, e iniciar os trabalhos.
\item Após a conclusão dos trabalhos, acionar novamente o botão~\textSymb{temoin_aspiration_manuelle} para desligar o sistema de aspiração.
\stopSteps
\stopfigtext

\stopsection

\page [yes]

\setups[pagestyle:normal]


\startsection [title={Pistola de água de alta pressão},
							reference={sec:using:water:spray}]

A \sdeux\ pode ser opcionalmente\index{Pistola de água} equipada com uma pistola de água de alta pressão. A pistola de água está fixada na porta de manutenção direita traseira e conectada com um enrolador de mangueira de 10 metros~– no lado oposto do veículo.

Para montar a pistola de água, proceder do seguinte modo:

{\sla Requisitos:}

O depósito de água limpa contém água suficiente; a \sdeux\ está no modo {\em Trabalho}. (Ver \in{§}[sec:using:start], \atpage[sec:using:start].)

\placefig[margin][fig:using:water:spray]{Pistola de água de alta pressão}
{\externalfigure[work:water:spray]}

\startSteps
\item Premir o botão~\textSymb{temoin_buse} da consola do tejadilho para ativar a bomba de água de alta pressão.
\item Antes de abandonar a cabina do operador, acionar bem o travão de mão.
\item Abrir a porta de manutenção direita traseira e retirar a pistola de água.
\item Desenrolar a mangueira até ao comprimento necessário e iniciar os trabalhos.
\item Após a conclusão dos trabalhos, acionar novamente o botão~\textSymb{temoin_buse} para desligar a bomba de água de alta pressão.
\item Puxar ligeiramente a mangueira para soltar o bloqueio e enrolar a mangueira.
\item Fixar a pistola de água novamente no seu suporte e fechar a porta de manutenção.
\stopSteps

\stopsection

\page [yes]


\setups [pagestyle:marginless]


\startsection [title={Trabalhar com a terceira escova (opção)},
		reference={sec:using:frontBrush},
		]

\startSteps
\item Colocar\index{Varredura} o veículo em funcionamento, conforme indicado na \in{secção}[sec:using:start] \atpage[sec:using:start].
\item Ativar\index{3.ª\,escova} o modo{\em Trabalho} (botão na parte exterior da alavanca de seleção da velocidade de marcha).
\stopSteps

% \getbuffer [work:config]

\startSteps [continue]
\item Certificar que a terceira escova está ativada no ecrã do Vpad
(ver \textSymb{vpadFrontBrush} \textSymb{vpadFrontBrushK}, \atpage[vpad:menu]).
\item Premir o botão~\textSymb{joy_key_frontbrush_act} para acionar o sistema hidráulico da terceira escova.
\item Premir o botão~\textSymb{joy_key_frontbrush_left} ou~\textSymb{joy_key_frontbrush_right} para que a terceira escova gire na direção pretendida.

\item Ajustar a velocidade de rotação através dos botões~\textSymb{joy_key_frontbrush_increase} e~\textSymb{joy_key_frontbrush_decrease} da consola multifunções.

\item Posicionar a escova através dos Joysticks, conforme indicado na figura em baixo.

\stopSteps

{\md Aviso:} {\lt para que seja possível posicionar as escovas laterais, é necessário desativar o sistema hidráulico da terceira escova, através do botão~\textSymb{joy_key_frontbrush_act}.}
\vfill

\start
\setupcombinations [width=\textwidth]

\placefig[here][fig:brush:position]{Posicionamento da terceira escova}
{\startcombination [2*1]
{\externalfigure [work:frontBrush:move]}{Para cima|/|para baixo; para a esquerda|/|direita}
{\externalfigure [work:frontBrush:incline]}{Inclinação transversal e longitudinal}
\stopcombination}
\stop



\stopsection

\stopregister[index][chap:using]

\stopchapter
\stopcomponent
