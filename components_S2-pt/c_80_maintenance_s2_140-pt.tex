\startcomponent c_80_maintenance_s2_140-pt
\product prd_ba_s2_140-pt

\startchapter [title={Manutenção e conservação},
							reference={chap:maintenance}]

\setups[pagestyle:marginless]


\startsection [title={Avisos gerais}]


\subsection{Proteção do meio ambiente}

\starttextbackground [FC]
\setupparagraphs [PictPar][1][width=2.45em,inner=\hfill]

\startPictPar
\Penvironment
\PictPar
	\Boschung\ coloca as normas de proteção do meio ambiente\index{Proteção do meio ambiente} em prática. Nós avaliamos todos os aspetos e, na hora de tomar decisões, temos em consideração todos os efeitos que o processo de produção e o produto têm sobre o meio ambiente.
	O nosso objetivo é garantir uma utilização eficiente dos recursos e uma gestão racional e cuidadosa dos recursos naturais, vitais para o homem e a preservação da natureza.
	O cumprimento de determinadas normas durante a utilização do veículo contribui para a proteção do meio ambiente. Isto inclui também a utilização adequada e
	de acordo com as normas de substâncias e materiais no âmbito da manutenção do veículo (\eG\ a eliminação de químicos e resíduos tóxicos).

	O consumo de combustível e o desgaste de um motor dependem das condições de operação. Por essa razão, pedimos que os seguintes pontos sejam tidos em consideração:

\startitemize
	\item Não deixar o motor aquecer ao ralenti.
	\item Desligar o motor durante os tempos de espera.
	\item Controlar regularmente o consumo de combustível.
	\item {\em Os trabalhos de manutenção devem ser executados de acordo com o plano de manutenção, por uma oficina especializada.}
\stopitemize
\stopSymList
\stoptextbackground

\page [yes]


\subsection{Normas de segurança}

\startSymList
\PHgeneric
\SymList
De modo\index{Manutenção+Normas de segurança} a evitar danos no veículo e nos agregados, assim como acidentes durante os trabalhos de manutenção, é fundamental que as normas de segurança sejam cumpridas. Observar igualmente as normas de segurança gerais (\about[safety:risques], \at{a partir
da página}[safety:risques]).
\stopSymList

\starttextbackground [FC]
\startPictPar
\PMgeneric
\PictPar
\textDescrHead{Prevenção de acidentes}
	Após todos os trabalhos de manutenção ou reparação, controlar\index{Prevenção de acidentes} o estado do veículo. Certificar sobretudo que todos os componentes relevantes para a segurança, tais como dispositivos de iluminação e sinalização, funcionam perfeitamente, antes de iniciar a circulação na via pública.
\stopPictPar
\stoptextbackground
\blank [big]

\start
\setupparagraphs [SymList][1][width=6em,inner=\hfill]
\startSymList\PHcrushing\PHfalling\SymList
\textDescrHead{Estabilização do veículo}
	Antes de executar quaisquer trabalhos de manutenção, proteger o veículo contra deslocações inadvertidas: colocar a alavanca de seleção da velocidade de marcha na posição \aW{Neutro}, acionar o travão de mão e colocar calços nas rodas.
\stopSymList
\stop

\starttextbackground[CB]
\startPictPar\PHpoison\PictPar
\textDescrHead{Ligar o motor}
	Se\index{Perigo+Intoxicação} for necessário ligar o motor num local com má ventilação, deixá-lo trabalhar apenas durante o tempo estritamente\index{Perigo+Gases de escape} necessário, para evitar intoxicações com monóxido de carbono.
	\stopPictPar
\startitemize
	\item Ligar o motor apenas se a bateria estiver corretamente conectada.
	\item Nunca desconectar a bateria com o motor em funcionamento.
	\item Não ligar o motor por meio de um auxiliar de arranque.
	Caso\index{Bateria+Carregador} se pretenda carregar a bateria por meio de um carregador rápido, a bateria deve ser previamente desconectada do veículo. Observar as normas de operação do carregador rápido.
\stopitemize
\stoptextbackground

\page [bigpreference]


\subsubsection{Proteção dos componentes eletrónicos}

\startitemize
	\item Antes\index{Soldadura elétrica} de iniciar os trabalhos de soldadura, desconectar os cabos da bateria e conectar os cabos positivos e os cabos de massa.
	\item Apenas\index{Sistema eletrónico} conectar e desconectar aparelhos de comando eletrónicos se os mesmos não estiverem sob tensão.
	\item Uma\index{Aparelho de comando} polaridade errada na alimentação elétrica (\eG\
	devido a baterias conectadas de forma errada) pode resultar na destruição de componentes eletrónicos e
	aparelhos.
	\item No caso de\index{Temperaturas ambiente+Extremas} temperaturas ambiente superiores a 80\,°C (\eG\ numa câmara de secagem), remover os componentes eletrónicos|/|aparelhos.
\stopitemize


\subsubsection{Diagnóstico e medições}

\startitemize
	\item Para os trabalhos de medição e diagnóstico, utilizar apenas cabos de teste {\em adequados} (\eG\ os cabos originais do aparelho).
	\item Telemóveis\index{Telemóvel} e aparelhos de rádio similares podem influenciar negativamente as funções do veículo e do aparelho de diagnóstico e, consequentemente, a segurança operacional.
\stopitemize


\subsubsection{Qualificação dos técnicos}

\starttextbackground[CB]
\startPictPar
\PHgeneric
\PictPar
\textDescrHead{Perigo de acidentes}
Se\index{Qualificação+Técnicos de manutenção} os trabalhos de manutenção forem executados de forma incorreta, isso pode influenciar negativamente o funcionamento e a segurança do veículo. A consequência é um maior risco de acidentes e ferimentos.

Os\index{Qualificação+Oficina} trabalhos de manutenção e reparação devem ser executados por uma oficina especializada que possua os conhecimentos e ferramentas necessários.

Em caso de dúvida, contactar o serviço de assistência ao cliente da \Boschung.
\stopPictPar
\stoptextbackground

A operação, a manutenção e a reparação da \ProductId apenas podem ser efetuadas por técnicos qualificados e que tenham recebido formação por parte do serviço de assistência ao cliente da \Boschung
.

Os técnicos adquirem as competências para a operação, conservação e reparação por parte do serviço de assistência ao cliente da \Boschung.


\subsubsection{Alterações e conversões}

\starttextbackground[CB]
\startPictPar
\PHgeneric
\PictPar
\textDescrHead{Perigo de acidentes}
Quaisquer\index{Alteração no veículo} alterações no veículo que sejam efetuadas por si, podem influenciar negativamente o funcionamento e a segurança da \ProductId, acarretando assim um risco de acidentes e ferimentos.
\stopPictPar

\startPictPar
\PMwarranty
\PictPar
A \Boschung\ não assume qualquer responsabilidade, nem concede direitos de garantia, pelos danos que\index{Garantia+Condições} sejam resultantes de intervenções ou modificações por conta própria na \ProductId ou num agregado.
\stopPictPar
\stoptextbackground

\stopsection


\startsection [title={Consumíveis e lubrificantes}, reference={sec:liquids}]


\subsection{Utilização correta}

\starttextbackground[CB]
\startPictPar
\PHpoison
\PictPar
\textDescrHead{Perigo de ferimentos e intoxicação}
Se\index{Combustível} os consumíveis\index{Lubrificante} ou\index{Perigo+Intoxicação} lubrificantes entrarem em contacto com a pele ou forem ingeridos, isso pode provocar\index{Combustível+Segurança} ferimentos graves ou intoxicações. Durante o manuseio, armazenamento ou eliminação destas substâncias, observar sempre as normas legais em vigor.
\stopPictPar
\stoptextbackground

\starttextbackground [FC]
\startPictPar
\PMproteyes\par
\PMprothands
\PictPar
Durante o manuseio de consumíveis e lubrificantes, utilizar sempre vestuário de proteção adequado e proteção respiratória. Evitar a inalação de vapores.
Evitar qualquer contacto com a pele, os olhos ou o vestuário. Em caso de contacto com a pele, lavar imediatamente com água e sabão.  Em caso de contacto com os olhos, passar abundantemente por água limpa e event. consultar um médico oftalmologista. Em caso de ingestão, consultar imediatamente um médico!
\stopPictPar
\stoptextbackground

\startSymList
\PPchildren
\SymList
Guardar os consumíveis num local que não esteja ao alcance das crianças.
\stopSymList

\startSymList
\PPfire
\SymList
\textDescrHead{Perigo de incêndios}
	Devido\index{Perigo+Fogo} à elevada inflamabilidade dos consumíveis, o seu manuseio implica um elevado risco de incêndio.  Durante o manuseio dos consumíveis é estritamente proibido fumar e fazer fogo\index{Proibido fumar} assim como expor os consumíveis diretamente à luz.
\stopSymList

\starttextbackground [FC]
\startPictPar
\PMgeneric
\PictPar
Apenas podem ser utilizados lubrificantes que sejam adequados para os componentes da \ProductId. Por esse motivo, utilizar apenas produtos que tenham sido testados e autorizados pela \Boschung. Estes estão indicados na lista de consumíveis, \atpage[sec:liqquantities]. Aditivos\index{Aditivos} para lubrificantes não são necessários. Se forem adicionados aditivos, isso pode resultar na anulação dos direitos de garantia\index{Garantia+Condições}.
Para mais informações, contactar o serviço de assistência ao cliente \Boschung.
\stopPictPar
\stoptextbackground

\starttextbackground [FC]
\startPictPar
\Penvironment
\PictPar
\textDescrHead{Proteção do meio ambiente}
Durante a\index{Lubrificantes+Eliminação} eliminação de consumíveis e lubrificantes\index{Proteção do meio ambiente} ou substâncias que estejam contaminadas com os mesmos (\eG\ filtros, panos), garantir\index{Consumíveis+Eliminação} que as normas de proteção do meio ambiente são cumpridas.
\stopPictPar
\stoptextbackground

\page [yes]

\setups [pagestyle:normal]


\subsection[sec:liqquantities]{Especificações e quantidades de enchimento}

Todas\index{Consumíveis+Quantidade de enchimento}\index{Lubrificantes+Quantidade de enchimento}\index{Quantidades de enchimento+Consumíveis e lubrificantes}\index{Especificações+Consumíveis e lubrificantes} as quantidades de enchimento indicadas na tabela são valores de referência. Após uma mudança de consumível|/|lubrificante, controlar o nível de enchimento e event. aumentar ou reduzir a quantidade de enchimento.
% \blank[big]

\placetable[margin][tab:glyco]{Produto anticongelante (\index{Produto anticongelante}Motor)}
{\noteF\startframedcontent[FrTabulate]
%\starttabulate[|Bp(80pt)|r|r|]
\starttabulate[|Bp|r|r|]
\NC Anticongelante até {[}°C{]}\NC \bf \textminus 25 \NC \bf \textminus 40 \NC\NR
\NC Água. destilada [Vol.||\%] \NC 60 \NC 40 \NC\NR
\NC Produto anticongelante \break [Vol.||\%] \NC 40 \NC {\em máx.} 60 \NC\NR
\stoptabulate\stopframedcontent\endgraf
Atenção: numa percentagem de volume superior a 60\hairspace\percent\
de produto anticongelante, o anticongelante {\em desce} e a capacidade de refrigeração piora!}

\placefig[margin][fig:hydrgauge]{\select{caption}{Indicação de nível
do fluido hidráulico (lado esquerdo do veículo)}{Indicação de nível 
fluido hidráulico}}
{\externalfigure[main:hy:level_temp]
\noteF O nível de enchimento do depósito de fluido hidráulico pode ser consultado no óculo de inspeção e deve ser verificado {\em diariamente}.}

\vskip -8pt
\start
\define [1] \TableSmallSymb {\externalfigure[#1][height=4ex]}
\define\UC\emptY
\pagereference[page:table:liquids]

\setupTABLE	[frame=off,style={\ssx\setupinterlinespace[line=.86\lH]},background=color,
			option=stretch,
			split=repeat]
\setupTABLE	[r]	[each]	[topframe=on,
						framecolor=TableWhite,
						% rulethickness=.8pt
						]

\setupTABLE	[c]	[odd]	[backgroundcolor=TableMiddle]
\setupTABLE	[c]	[even]	[backgroundcolor=TableLight]
\setupTABLE	[c]	[1]		[width=30mm]
\setupTABLE	[c]	[2]		[width=20mm]
\setupTABLE	[c]	[4]		[width=25mm]
\setupTABLE	[c]	[last]	[width=10mm]
\setupTABLE	[r] [first]	[topframe=off,style={\bfx\setupinterlinespace[line=.95\lH]},
						% backgroundcolor=TableDark
						]
\setupTABLE	[r]	[2]		[framecolor=black]

\bTABLE

\bTABLEhead
	\bTR
	\bTC Grupo \eTC
	\bTC Categoria \eTC
	\bTC Classificação \eTC
	\bTC Produto\note[Produkt] \eTC
	\bTC Quantidade \eTC
	\eTR
\eTABLEhead

\bTABLEbody
 \bTR \bTD	Motor Diesel \eTD
  \bTD Óleo do motor\eTD
  \bTD \liqC{SAE 5W-30}; \liqC{VW\,507.00}\eTD
  \bTD Total Quartz INEO Long Life \eTD
  \bTD	4,3\,l\eTD
  \eTR
 \bTR \bTD Circuito hidráulico \eTD
  \bTD Óleo hidráulico \eTD
  \bTD \liqC{ISO VG 46} \eTD
  \bTD  Total Equiviz ZS 46 (depósito aprox. 40\,l) \eTD
  \bTD aprox. 50\,l\eTD
  \eTR
 \bTR \bTD Circuito hidráulico (opção~\aW{Bio})\eTD
  \bTD Óleo hidráulico \eTD
  \bTD \liqC{ISO VG 46} \eTD
  \bTD  Total Biohydran TMP SE 46\eTD
  \bTD aprox. 50\,l\eTD
  \eTR
 \bTR \bTD Válvulas solenoides: núcleos da bobina \eTD
  \bTD Lubrificante\eTD
  \bTD Massa lubrificante com cobre \eTD
  \bTD \emptY\eTD
  \bTD c.\,n.\note[Bedarf] \eTD
  \eTR
 \bTR \bTD	Diversos: fechaduras, mecanismo da porta, pedal do travão \eTD
  \bTD Lubrificante\eTD
  \bTD Spray universal\eTD
  \bTD \emptY\eTD
  \bTD c.\,n.\note[Bedarf] \eTD
  \eTR
 \bTR \bTD	Sistema de lubrificação central \eTD
  \bTD Massa lubrificante para rolamentos universal\eTD
  \bTD \liqC{nlgi}~2 \eTD
  \bTD Total Multis EP~2\eTD
  \bTD c.\,n.\note[Bedarf] \eTD
  \eTR
 \bTR \bTD	Sistema de refrigeração \eTD
  \bTD Produto anticongelante|/|antiferrugem\eTD
  \bTD TL VW 774 F/G; máx. 60\hairspace\% vol.\eTD
  \bTD G12+|/|G12++ (rosa|/|violeta)\eTD
  \bTD aprox. 14\,l \eTD
  \eTR
 \bTR \bTD	Bomba de água de alta pressão \eTD
  \bTD Óleo do motor\eTD
  \bTD \liqC{SAE 10W-40}; \liqC{api cf~– acea e6}\eTD
  \bTD Total Rubia TIR 8900 \eTD
  \bTD 0,29\,l \eTD
  \eTR
 \bTR \bTD	Sistema de climatização \eTD
  \bTD Refrigerante\eTD
  \bTD + 20\,ml óleo POE\eTD
  \bTD R\,134a\eTD
  \bTD	700\,g\eTD
  \eTR
 \bTR \bTD	Sistema limpa para-brisas \eTD
	\bTD [nc=2] Água e líquido concentrado limpa para-brisas, \aW{S}~verão, \aW{W}~inverno; observar a relação de mistura \eTD
	\bTD Comércio a retalho \eTD
  \bTD c.\,n.\note[Bedarf] \eTD
  \eTR
\eTABLEbody

\eTABLE
\stop

\footnotetext[Bedarf]{{\it c.\,n.} conforme necessário, consoante as respetivas instruções}
\footnotetext[Produkt]{Produtos utilizados pela \Boschung. Também podem ser utilizados outros produtos, desde que estejam em conformidade com as especificações.}

\stopsection

\page [yes]

\setups [pagestyle:marginless]


\startsection [title={Manutenção do motor Diesel},
							reference={sec:workshop:vw}]


\subsection [sSec:vw:diagTool]{Sistema de diagnóstico a bordo (OBD)}

O\startregister[index][reg:main:vw]{Manutenção+Motor Diesel} aparelho de comando do motor (J623) está equipado com uma memória de erros.
Se surgirem falhas nos sensores ou componentes monitorizados, estas são gravadas na memória de erros, com indicação do tipo de erro.

Após avaliação da informação, o\index{Motor Diesel+Diagnóstico} aparelho de comando do motor distingue entre as diversas classes de erro e grava-as até o conteúdo da memória de erros ser apagado.

Os erros que apenas surgem {\em esporadicamente} são exibidos com a indicação \aW{SP}. Um erro esporádico pode ser causado \eG\ por um mau contacto ou uma interrupção momentânea do circuito. Se um erro esporádico não surgir durante 50 arranques do motor, é eliminado da memória de erros.

Se tiverem sido detetados erros que influenciem o funcionamento do motor, o símbolo de controlo \aW{Diagnóstico do motor}~\textSymb{vpadWarningEngine1} acende no ecrã no Vpad.

Os erros gravados podem ser consultados através do sistema de diagnóstico, medição e informação do veículo \aW{VAS\,5051/B}.

Assim que o(s) erro(s) tiver(em) sido eliminado(s), a memória de erros deverá ser apagada.


\subsubsection[sSec:vw:diagTool:connect]{Colocação em funcionamento do sistema de diagnóstico}

\starttextbackground [FC]
\startPictPar
\PMgeneric
\PictPar
Para informações detalhadas relativas ao sistema de diagnóstico do veículo VAS\,5051/B, consultar o manual de instruções do sistema.

Também podem ser utilizados outros sistemas de diagnóstico, desde que sejam compatíveis, \eG\ \aW{DiagRA}.
\stopPictPar
\stoptextbackground

\page [yes]


\subsubsubsubject{Requisitos}

\startitemize
\item Os fusíveis devem estar em bom estado.
\item A tensão da bateria deve ser superior a 11,5\,V.
\item Todos os consumidores elétricos devem estar desligados.
\item O terminal de terra deve estar em bom estado.
\stopitemize


\subsubsubsubject{Modo de procedimento}

\startSteps
\item Inserir o conector da linha de diagnóstico VAS\,5051B/1 na conexão de diagnóstico.
\item Consoante a função ligar a ignição ou ligar o motor.
\stopSteps

\subsubsubsubject{Selecionar o modo de operação}

\startSteps [continue]
\item No ecrã, premir o botão \aW{Autodiagnóstico do veículo}.
\stopSteps


\subsubsubsubject{Selecionar o sistema do veículo}

\startSteps [continue]
\item No ecrã, premir o botão \aW{01-Sistema eletrónico do motor}.
\stopSteps

No ecrã é então exibida a identificação do aparelho de comando e a codificação do aparelho de comando do motor.

Se as codificações não coincidirem, é necessário verificar a codificação do aparelho de comando.


\subsubsubsubject{Selecionar a função de diagnóstico}

No ecrã são exibidas todas as funções de diagnóstico executáveis.

\startSteps [continue]
\item No ecrã, premir o botão para a função pretendida.
\stopSteps



\subsection [sSec:vw:faultMemory]{Memória de erros}


\subsubsection{Consultar a memória de erros}

\subsubsubject{Curso dos trabalhos}

\startSteps
\item Deixar o motor trabalhar ao ralenti.
\item Conectar o VAS\,5051/B (ver \in{secção}[sSec:vw:diagTool:connect])
e selecionar o aparelho de comando do motor.
\item Selecionar a função de diagnóstico \aW{004-Conteúdo da memória de erros}.
\item Selecionar a função de diagnóstico \aW{004.01-Consultar memória de erros}.
\stopSteps

{\sla Apenas se o motor não ligar:}

\startitemize [2]
\item Ligar a ignição.
\item Se não houver quaisquer erros no aparelho de comando do motor, é exibida a mensagem \aW{0~erros detetados} no ecrã.
\item Se houver erros no aparelho de comando do motor, estes são exibidos em forma de lista no ecrã.
\item Terminar a função de diagnóstico.
\item Desligar a ignição.
\item Eliminar eventuais erros exibidos com base na tabela de erros (com base na documentação de serviço) e, de seguida, apagar a memória de erros.
\stopitemize

\starttextbackground [FC]
\startPictPar
\PMrtfm
\PictPar
Se não for possível apagar um erro, contactar o serviço de assistência ao cliente da \boschung.
\stopPictPar
\stoptextbackground


\subsubsubject{Erros estáticos}

Se houver um ou vários erros estáticos na memória de dados, contactar o serviço de assistência ao cliente da Boschung, para eliminar esses erros através da \aW{localização guiada de erros}.


\subsubsubject{Erros esporádicos}

Se na memória de erros apenas estiverem guardados erros ou avisos esporádicos, e não se verificar um funcionamento errado do sistema eletrónico do veículo, a memória de erros pode ser apagada:

\startSteps [continue]
\item Premir novamente o botão \aW{Seguinte}~\inframed[strut=local]{>} para aceder ao plano de verificação.
\item Para terminar a localização guiada de erros, premir o botão \aW{Ir para} e depois \aW{Terminar}.
\stopSteps

São então novamente consultadas todas as memórias de erros.

Numa janela é exibida a confirmação de que todos os erros esporádicos foram apagados.
% Das Diagnoseprotokoll wird automatisch (online) verschickt.

O teste do sistema do veículo está assim terminado.


\subsubsection[sSec:vw:faultMemory:errase]{Apagar a memória de erros}

\subsubsubject{Curso dos trabalhos}

{\sla Requisitos:}

\startitemize [2]
\item Todos os erros, e as respetivas causas, devem ter sido eliminados.
\stopitemize

\page [yes]


{\sla Modo de procedimento:}

\starttextbackground [FC]
\startPictPar
\PMrtfm
\PictPar
Após a eliminação dos erros, a memória de erros deve ser novamente consultada e, de seguida, apagada:
\stopPictPar
\stoptextbackground

\startSteps
\item Deixar o motor trabalhar ao ralenti.
\item Conectar o VAS\,5051/B (ver \in{secção}[sSec:vw:diagTool:connect])
e selecionar o aparelho de comando do motor.
\item Selecionar a função de diagnóstico \aW{004-Consultar memória de erros}.
\item Selecionar a função de diagnóstico \aW{004.10-Apagar memória de erros}.
\stopSteps

\starttextbackground [FC]
\startPictPar
\PMrtfm
\PictPar
Se não for possível apagar a memória de erros, isso significa que ainda existe um erro e é necessário eliminá-lo.
\stopPictPar
\stoptextbackground

\startSteps [continue]
\item Terminar a função de diagnóstico.
\item Desligar a ignição.
\stopSteps


\subsection [sSec:vw:lub] {Lubrificação do motor Diesel}

\subsubsection [ssSec:vw:oilLevel] {Verificar o nível do óleo do motor}

\starttextbackground [FC]
\startPictPar
\PMrtfm
\PictPar
O\index{Nível do óleo +do motor} nível do óleo não pode exceder a marcação \aW{Máx.}. Caso contrário, existe\index{Nível do óleo+ do motor} perigo de danos no catalisador.
\stopPictPar
\stoptextbackground

\startSteps
\item Desligar o motor e esperar, pelo menos, 3~minutos, para que o óleo regresse ao cárter de óleo.
\item Retirar a vareta de medição e limpá-la; voltar a inserir a vareta até ao batente.
\item Retirar novamente a vareta e verificar o nível do óleo:

\startfigtext[right][fig:vw:gauge]{Verificar o nível do óleo}
{\externalfigure[VW_Oil_Gauge][width=50mm]}
\startitemize [A]
\item Nível de enchimento máximo; não reencher óleo.
\item Nível de enchimento suficiente; {\em pode-se} reencher óleo até à marcação~\aW{A}.
\item Nível de enchimento insuficiente; {\em tem} de se reencher óleo até à marcação \aW{B}.
\stopitemize
{\em Se o nível de enchimento se encontrar acima da marcação~\aW{A} existe perigo de danos no catalisador.}
\stopfigtext
\stopSteps


\subsubsection [ssSec:vw:oilDraining] {Mudança do óleo do motor}

\starttextbackground [FC]
\startPictPar
\PMrtfm
\PictPar
O filtro de óleo do motor da S2 está montado em posição vertical. Isso significa que o filtro deve ser substituído {\em antes} da mudança do óleo. Ao retirar o elemento filtrante é aberta uma válvula e o óleo que se encontra no corpo do filtro flui automaticamente para o corpo da cambota.
\stopPictPar
\stoptextbackground

\startSteps
\item Colocar um\index{Motor Diesel+Mudança do óleo} recipiente coletor adequado por baixo do motor.
\item Desaparafusar o tampão de drenagem do óleo\index{Mudança do+ Óleo do motor} e deixar escoar o óleo.
\stopSteps

\starttextbackground [FC]
\startPictPar
\PMrtfm
\PictPar
Assegurar que o recipiente coletor tem capacidade para recolher todo o óleo escoado.
Para informações relativas à especificação do óleo e à quantidade de enchimento, ver \in{secção}[sec:liqquantities].

O tampão de drenagem do óleo possui um anel vedante que não pode ser removido. Por isso, é necessário substituir sempre o tampão de drenagem do óleo.
\stopPictPar
\stoptextbackground

\startSteps [continue]
\item Enroscar um tampão de drenagem do óleo com anel vedante novo (\TorqueR~30\,Nm).
\item Encher óleo do motor da especificação indicada (ver \in{secção}[sec:liqquantities]).
\stopSteps


\subsubsection [ssSec:vw:oilFilter] {Substituir o filtro de óleo do motor}

\starttextbackground [FC]
\startPictPar
\PMrtfm
\PictPar
\startitemize [1]
\item Observar\index{Motor Diesel+Filtro de óleo} as normas relativas à eliminação e reciclagem.
\item Substituir\index{Filtro de óleo+Motor Diesel} o filtro {\em antes} da mudança do óleo (ver \in{secção}[ssSec:vw:oilDraining]).
\item Antes da montagem, lubrificar levemente o vedante do filtro novo.
\stopitemize
\stopPictPar
\stoptextbackground

\startfigtext[right][fig:vw:oilFilter]{Filtro de óleo}
{\externalfigure[VW_OilFilter_03][width=50mm]}
\startSteps
\item Desenroscar a tampa~\Lone\ do corpo do filtro com uma chave de bocas adequada.
\item Limpar a superfície de vedação da tampa e do corpo do filtro.
\item Substituir o elemento filtrante \Lthree.
\item Substituir os vedantes em O \Ltwo\ e \Lfour.
\item Enroscar a tampa novamente no corpo do filtro (\TorqueR25\,Nm).
\stopSteps



%\subsubsubject{Données techniques}
%
%
%\hangDescr{Couple de serrage du couvercle:} \TorqueR~25\,Nm.
%
%\hangDescr{Huile moteur prescrite:} Selon tableau \atpage[sec:liqquantities].
%% NOTE: Redundant [tf]

\stopfigtext



\subsubsection [ssSec:vw:oilreplenish] {Reencher óleo do motor}

\starttextbackground [FC]
\startPictPar
\PMrtfm
\PictPar
\startitemize [1]
\item {\em Antes} de\index{Óleo do motor}  retirar a tampa, limpar o tubo de enchimento com um pano.
\item Reencher\index{Motor Diesel+Reencher óleo} apenas óleo da especificação indicada.
\item Reencher gradualmente em pequenas quantidades.
\item Para evitar um enchimento excessivo, após reencher, aguardar um pouco até o óleo fluir para o cárter de óleo do motor, até à marcação da vareta de medição (ver \in{secção}[ssSec:vw:oilLevel]).
\stopitemize
\stopPictPar
\stoptextbackground

\startfigtext[right][fig:vw:oilFilter]{Reencher óleo}
{\externalfigure[s2_bouchonRemplissage][width=50mm]}
\startSteps
\item Puxar a vareta de medição do óleo cerca de 10~cm para fora, para que, durante o reenchimento, o ar possa sair.
\item Abrir o orifício de enchimento.
\item Reencher óleo, observando as normas acima indicadas.
\item Fechar cuidadosamente o orifício de enchimento.
\item Ligar o motor.
\item Controlar o nível de enchimento (ver \in{secção}[ssSec:vw:oilLevel]).
\stopSteps

\stopfigtext


\subsection [sSec:vw:fuel] {Sistema de alimentação de combustível}

\subsubsection [ssSec:vw:fuelFilter] {Substituir o filtro de combustível}

\starttextbackground [FC]
\startPictPar
\PMrtfm
\PictPar
\startitemize [1]
\item Observar\index{Motor Diesel+Filtro de combustível} as normas legais relativas à eliminação e reciclagem de resíduos tóxicos.
\item Não retirar as tubagens de combustível da parte superior do filtro.
\item Não exercer força sobre os pontos de fixação das tubagens de combustível; caso contrário, podem ocorrer danos na parte superior do filtro.
\stopitemize
\stopPictPar
\stoptextbackground

\startfigtext[right][fig:vw:oilFilter]{Filtro de combustível}
{\externalfigure[s2_fuelFilter_location][width=50mm]}

{\sla Preparação:}

O\index{Filtro de combustível} corpo do filtro de combustível está fixado à frente do motor, do lado direito do chassis.
Remover os dois parafusos de fixação, utilizando uma chave de caixa de 10 mm e uma chave anular de 10 mm.

\stopfigtext


\page [yes]

\setups [pagestyle:normal]

{\sla Modo de procedimento:}

\startLongsteps
\item Remover todos os parafusos da parte superior do filtro. Remover a parte superior do filtro.
\stopLongsteps

\starttextbackground [FC]
\startPictPar
\PMrtfm
\PictPar
Levantar a parte superior. Se necessário, colocar uma chave de fenda angular na ranhura de montagem (\in{\LAa, fig.}[fig:fuelfilter:detach]) e retirar a parte superior.
\stopPictPar
\stoptextbackground

\placefig [margin] [fig:fuelfilter:detach]{Remoção do filtro de combustível}
{\externalfigure[fuelfilter:detach]}

\placefig [margin] [fig:fuelfilter:explosion]{Filtro de combustível}
{\externalfigure[fuelfilter:explosion]}

\startLongsteps [continue]
\item Retirar o elemento filtrante da parte inferior do filtro.
\item Retirar o vedante (\in{\Ltwo, fig.}[fig:fuelfilter:explosion]) da parte superior do filtro.
\item Limpar cuidadosamente a parte inferior e a parte superior do filtro.
\item Inserir um novo elemento filtrante na parte inferior do filtro.
\item Humedecer um vedante novo (\in{\Ltwo, fig.}[fig:fuelfilter:explosion]) com um pouco de combustível e inseri-lo na parte superior.
\item Colocar a parte superior corretamente sobre a parte inferior do filtro e pressionar uniformemente, para que a parte superior esteja posicionada uniformemente.
\item Voltar a aparafusar {\em manualmente} a parte superior e a parte inferior com todos os parafusos. Apertar todos os parafusos em cruz, com o binário de aperto indicado (\TorqueR5\,Nm).
\stopLongsteps

% \subsubsubject{Données techniques}
%
% \hangDescr{Couple de serrage des vis de fixation du couvercle:} \TorqueR 5\,Nm.
%% NOTE: redundant [tf]

\startLongsteps [continue]
\item Ligar a ignição para purgar o sistema; ligar o motor e deixá-lo trabalhar durante 1~a 2~minutos ao ralenti.
\item Apagar a memória de erros consoante indicado na \atpage[sSec:vw:faultMemory:errase].
\stopLongsteps


\subsection [sSec:vw:cooling] {Sistema de refrigeração}

\starttextbackground [FC]
\startPictPar
\PMrtfm
\PictPar
\startitemize [1]
\item Utilizar apenas\index{Motor Diesel+Refrigeração} fluido refrigerante da especificação indicada (ver tabela \atpage[sec:liqquantities]).
\item Para\index{Fluido refrigerante} garantir que as propriedades anticongelantes e anticorrosivas do fluido refrigerante se mantêm, este apenas pode ser diluído com água destilada, consoante a tabela em baixo.
\item Nunca encher o circuito de refrigeração com água; desta forma, o fluido refrigerante iria perder as suas propriedades anticongelantes e anticorrosivas.
\stopitemize
\stopPictPar
\stoptextbackground


\subsubsection [sSec:vw:coolingLevel] {Nível do fluido refrigerante}

\placefig [margin] [fig:coolant:level] {Nível do fluido refrigerante}
{\externalfigure[coolant:level]}


\placefig [margin] [fig:refractometer] {Refratómetro VW~T\,10007}
{\externalfigure[coolant:refractometer]}

\placefig [margin] [fig:antifreeze] {Controlar a densidade do produto anticongelante}
{\externalfigure[coolant:antifreeze]}


\startSteps
\item Levantar o recipiente de sujidade e colocar o apoio de segurança.
\item Verificar\index{Nível de enchimento+Fluido refrigerante} o nível de enchimento do fluido refrigerante no reservatório de expansão: o nível de enchimento deve encontrar-se acima da marcação \aW{Mín.}.
\stopSteps

\start
\define [1] \TableSmallSymb {\externalfigure[#1][height=4ex]}
\define\UC\emptY
\pagereference[page:table:liquids]


\setupTABLE	[frame=off,style={\ssx\setupinterlinespace[line=.86\lH]},background=color,
			option=stretch,
			split=repeat]
\setupTABLE	[r]	[each]	[topframe=on,
						framecolor=TableWhite,
						% rulethickness=.8pt
						]

\setupTABLE	[c]	[odd]	[backgroundcolor=TableMiddle]
\setupTABLE	[c]	[even]	[backgroundcolor=TableLight]
\setupTABLE	[r] [first]	[topframe=off,style={\bfx\setupinterlinespace[line=.95\lH]},
						% backgroundcolor=TableDark
						]
\setupTABLE	[r]	[2]		[framecolor=black]

\bTABLE

\bTABLEhead
 \bTR
 \bTC Produto anticongelante até … \eTC
 \bTC Percentagem G12\hairspace ++\eTC
 \bTC Vol. produto anticongelante \eTC
 \bTC Vol. água destilada \eTC
 \eTR
\eTABLEhead

\bTABLEbody
 \bTR \bTD \textminus 25\,°C \eTD
  \bTD 40\hairspace\% \eTD
  \bTD 3,8\,l \eTD
  \bTD 4,2\,l \eTD
  \eTR
 \bTR \bTD \textminus 35\,°C \eTD
  \bTD 50\hairspace\% \eTD
  \bTD 4,0\,l \eTD
  \bTD 4,0\,l \eTD
  \eTR
 \bTR \bTD \textminus 40\,°C \eTD
  \bTD 60\hairspace\%  \eTD
  \bTD 4,2\,l \eTD
  \bTD 3,8\,l \eTD
  \eTR
\eTABLEbody

\eTABLE
\stop

\adaptlayout [height=+20pt]
\subsubsection [sSec:vw:coolingFreeze] {Nível do fluido refrigerante}

Verificar\index{Densidade do produto anticongelante} a densidade do produto anticongelante através de um refratómetro adequado (ver \in{fig.}[fig:refractometer]: VW T\,10007).
Observar a escala~1: G12\hairspace ++ (ver \in{fig.}[fig:antifreeze]).

\page [yes]


\subsection [sSec:vw:airFilter] {Abastecimento de ar}

O filtro de ar pode ser acedido através da porta de manutenção traseira, do lado direito do veículo (ver \in{fig.}[fig:airFilter]).

\placefig [margin] [fig:airFilter] {Filtro de ar do motor}
{\externalfigure[vw:air:filter]
\noteF
\startLeg
\item Tala de segurança
\item Parte inferior do corpo
\item Orifício de purga
\item Sensor de pressão
\stopLeg}


\subsubsubject{Condições de utilização}

A máquina varredora é frequentemente utilizada em ambientes com elevada formação de poeira. Por esse motivo, é necessário inspecionar e limpar o filtro de ar semanalmente. Ver também \about[table:scheduleweekly], \atpage[table:scheduleweekly]. Se necessário, substituir o filtro de ar.


\subsubsubject{Autodiagnóstico}

A tubagem de aspiração possui um sensor de pressão (\Lfour, \in{fig.}[fig:airFilter]) que permite detetar quedas de carga\footnote{Fluxo de ar reduzido devido a capacidade de permeabilidade do ar do filtro reduzida.}, através do filtro.
Se o filtro de ar estiver danificado, o símbolo de controlo~\textSymb{vpadWarningFilter} no ecrã do Vpad acende e a mensagem de erro \VpadEr{851} é registada.


\subsubsubject{Conservação/substituição}

\startSteps
\item Puxar a tala de segurança~\Lone para baixo (\in{fig.}[fig:airFilter]).
\item Rodar a parte inferior do corpo~\Ltwo no sentido anti-horário e remover.
\item Remover o elemento filtrante e inspecioná-lo. Se necessário, substituir.
\item Limpar o interior do filtro e voltar a montar o filtro de ar, procedendo em ordem inversa.
\stopSteps

\page [yes]


\subsection [sSec:vw:belt] {Correia trapezoidal}

A\index{Motor Diesel+Correia trapezoidal} correia trapezoidal transmite o movimento do volante da cambota para o dínamo e o compressor do sistema de climatização (equipamento opcional).
Um\index{Correia trapezoidal} elemento tensor no último segmento (entre o dínamo e a cambota) mantém a correia sob tensão.


\subsubsection [sSec:belt:change] {Substituição da correia trapezoidal}

\placefig [margin] [fig:belt:tool] {Mandril de encaixe VW T\,10060\,A}
{\externalfigure[vw:belt:tool]}

\placefig [margin] [fig:belt:overview] {Elemento tensor}
{\externalfigure[vw:belt:overview]}

\placefig [margin] [fig:belt:tens] {Local de encaixe do mandril de encaixe}
{\externalfigure[vw:belt:tens]}


\subsubsubject{Com compressor do sistema de climatização}


{\sla Ferramenta especial necessária:}

Mandril de encaixe \aW{VW T\,10060\,A} para segurar o elemento tensor.

\startSteps
\item Definir o sentido de funcionamento da correia trapezoidal.
\item Com uma chave anular em forma de cotovelo, rodar o braço do elemento tensor no sentido horário (\in {fig.}[fig:belt:overview]).
\item Colocar os orifícios (ver setas, \in {fig.}[fig:belt:tens]) em direção à cobertura e fixar o elemento tensor com o mandril de encaixe.
\item Retirar a correia trapezoidal.
\stopSteps

A montagem da correia trapezoidal é efetuada em ordem inversa.

\starttextbackground [FC]
\startPictPar
\PMrtfm
\PictPar
\startitemize [1]
\item Observar o sentido de funcionamento da correia trapezoidal.
\item Observar a fixação correta nas polias da correia.
\item Ligar o motor e controlar o funcionamento da correia.
\stopitemize
\stopPictPar
\stoptextbackground


\subsubsubject{Sem compressor do sistema de climatização}

{\sla Material necessário:}

Kit de reparação, constituído por: instruções de reparação, correia trapezoidal e ferramenta especial.\footnote{Ver catálogo de peças sobressalentes, em \aW{Componentes de manutenção}.}

\startSteps
\item Cortar a correia trapezoidal.
\item Seguir os passos seguintes indicados nas instruções de reparação.
\stopSteps

\starttextbackground [FC]
\startPictPar
\PMrtfm
\PictPar
\startitemize [1]
\item Observar a fixação correta nas polias da correia.
\item Ligar o motor e controlar o funcionamento da correia.
\stopitemize
\stopPictPar
\stoptextbackground


\subsubsection [sSec:belt:tens] {Substituição do elemento tensor}

{\sla Apenas para versão com compressor do sistema de climatização}

\blank [medium]

\placefig [margin] [fig:belt:tens:change] {Substituição do elemento tensor}
{\externalfigure[vw:belt:tens:change]
\noteF
\startLeg
\item Elemento tensor
\item Parafuso de segurança
\stopLeg

{\bf Binário de aperto}

Parafuso de segurança:

\TorqueR~20\,Nm\:+\,½~rotação (180°).}

\startSteps
\item Desmontar a correia trapezoidal conforme indicado (ver \atpage[sSec:belt:change]).
\item Desmontar componentes periféricos (consoante o equipamento).
\item Desenroscar o parafuso de segurança (\in{\Ltwo, fig.}[fig:belt:tens:change]).
\stopSteps

A montagem do elemento tensor é efetuada em ordem inversa.

\starttextbackground [FC]
\startPictPar
\PMrtfm
\PictPar
\startitemize [1]
\item Após a montagem, utilizar um parafuso de segurança novo.
\item Binário de aperto: ver \in{fig.}[fig:belt:tens:change].
\stopitemize
\stopPictPar
\stoptextbackground

\stopregister[index][reg:main:vw]

\stopsection

\page[yes]


\setups[pagestyle:marginless]


\startsection[title={Instalação hidráulica},
							reference={sec:hydraulic}]

\starttextbackground [FC]
% \startfiguretext[left,none]{}
% {\externalfigure[toni_melangeur][width=30mm]}

\startSymPar
\externalfigure[toni_melangeur][width=4em]
\SymPar
\textDescrHead{Reciclagem de consumíveis}
Consumíveis e lubrificantes usados não podem ser eliminados nem queimados na natureza.

Lubrificantes usados não podem ser escoados na rede de esgotos nem eliminados na natureza, e também não podem ser colocados no lixo doméstico.

Lubrificantes usados não podem ser misturados com outros fluidos - perigo de se formarem substâncias tóxicas ou substâncias difíceis de eliminar.
\stopSymPar
\stoptextbackground
\blank [big]

% \starthangaround{\PMgeneric}
% \textDescrHead{Qualification du personnel}
% Toute intervention sur l’installation hydraulique de votre véhicule ne peut être réalisée que par une personne dument qualifiée, ou par un service reconnu par \boschung.
% \stophangaround
% \blank[big]

\startSymList
\PHgeneric
\SymList
\textDescrHead{Limpeza} A instalação hidráulica é muito sensível a impurezas no óleo. Por isso, é extremamente importante trabalhar num ambiente limpo.
\stopSymList

\startSymList
\PHhot
\SymList
\textDescrHead{Perigo de salpicos}
Antes de executar trabalhos na instalação hidráulica da \sdeux, é necessário descarregar a pressão residual no circuito hidráulico. Salpicos de óleo quente podem provocar queimaduras.
\stopSymList

\startSymList
\PHhand
\SymList
\textDescrHead{Perigo de esmagamento}
Antes de executar trabalhos na instalação hidráulica da \sdeux, o recipiente de sujidade deve estar obrigatoriamente descido ou fixado mecanicamente através do apoio de segurança.
\stopSymList

\startSymList
\PImano
\SymList
\textDescrHead{Medição da pressão}
Para medir a pressão hidráulica, colocar o manómetro numa das conexões \aW{Minimess} do circuito. Prestar atenção para que o manómetro aponte para uma área de medição adequada.
\stopSymList

\page [yes]

\setups[pagestyle:normal]

\subsection{Intervalos de manutenção}

\start

	\setupTABLE	[frame=off,
				style={\ssx\setupinterlinespace[line=.93\lH]},
				background=color,
				option=stretch,
				split=repeat]
	\setupTABLE	[r]	[each]	[
							topframe=on,
							framecolor=white,
							backgroundcolor=TableLight,
							% rulethickness=.8pt,
							]

	% \setupTABLE	[c]	[odd]	[backgroundcolor=TableMiddle]
	% \setupTABLE	[c]	[even]	[backgroundcolor=TableLight]
	\setupTABLE	[c]	[1]		[ % width=30mm,
							style={\bfx\setupinterlinespace[line=.93\lH]},
							]
	\setupTABLE	[r] [first]	[topframe=off,
							style={\bfx\setupinterlinespace[line=.93\lH]},
							backgroundcolor=TableMiddle,
							]
	% \setupTABLE	[r]	[2]		[style={\ssBfx\setupinterlinespace[line=.93\lH]}]


\bTABLE

\bTABLEhead
\bTR\bTD Trabalho de manutenção \eTD\bTD Intervalo \eTD\eTR
\eTABLEhead

\bTABLEbody
\bTR\bTD Controlar quanto a fugas \eTD\bTD Diariamente \eTD\eTR
\bTR\bTD Controlar o nível do óleo hidráulico \eTD\bTD Diariamente \eTD\eTR
\bTR\bTD Verificar o estado das tubagens/mangueiras hidráulicas; event. substituir \eTD\bTD 600\,h / 12~meses \eTD\eTR
\bTR\bTD Substituir o filtro de retorno e aspiração de óleo hidráulico \eTD\bTD 600\,h / 12~meses \eTD\eTR
\bTR\bTD Lubrificar os núcleos da bobina das válvulas solenoides com massa lubrificante com cobre \eTD\bTD 600\,h / 12~meses \eTD\eTR
\bTR\bTD Mudar o óleo hidráulico \eTD\bTD 1200\,h / 24~meses \eTD\eTR
\eTABLEbody
\eTABLE
\stop


\subsection[niveau_hydrau]{Nível de enchimento}

\placefig[margin][fig:hydraulic:level]{Nível de enchimento do fluido hidráulico}
{\externalfigure[hydraulic:level]
\noteF
\startLeg
\item Nível de enchimento ideal
\stopLeg}

Através do óculo de inspeção transparente\index{Nível de enchimento+Fluido hidráulico}\index{Manutenção+Instalação hidráulica} é possível verificar o nível do óleo hidráulico.
Se o nível do óleo hidráulico tiver descido, antes de proceder ao reenchimento é necessário averiguar a causa da descida. Observar e respeitar os intervalos de mudança (tabela em cima) e as especificações do fluido hidráulico (tabela \at{página}[sec:liqquantities]).


\subsubsection{Reencher fluido hidráulico}

Reencher fluido hidráulico até o óculo de inspeção do meio estar totalmente coberto.
Ligar o motor e event. reencher um pouco de fluido, até alcançar o nível de enchimento requerido.


\subsection{Mudança do fluido hidráulico}

A quantidade de enchimento e as especificações do fluido hidráulico podem ser consultadas na
tabela, na \at{página}[sec:liqquantities].

\startSteps
\item Abrir o orifício de reenchimento do depósito de fluido hidráulico.
\item Esvaziar o depósito com um aspirador de óleo ou remover o tampão de drenagem.

O tampão de drenagem encontra-se na parte inferior do depósito de fluido hidráulico, à frente da roda traseira esquerda (\in{fig.}[fig:hydraulic:fluidDrain]).
\item Reencher fluido hidráulico até o óculo de inspeção do meio estar totalmente coberto.
Ligar o motor e event. reencher um pouco de fluido, até alcançar o nível de enchimento requerido.
\stopSteps

\placefig[margin][fig:hydraulic:fluidDrain]{Tampão de drenagem}
{\externalfigure[hydraulic:fluidDrain]}


\placefig[margin][fig:hydraulic:returnFilter]{Filtro hidráulico}
{\externalfigure[hydraulic:returnFilter]}

\subsection[filtres:nettoyage]{Filtro de retorno e aspiração}

\startSteps
\item Levantar o recipiente de sujidade e colocar o apoio de segurança.
\item Retirar a tampa do filtro no depósito de fluido hidráulico (\in{fig.}[fig:hydraulic:returnFilter]).
\item Substituir\index{Filtro de óleo+Sistema hidráulico} o elemento filtrante por um novo.
\item Humedecer um vedante em O novo com um pouco de fluido hidráulico e montar o vedante em O.
\item Voltar a enroscar a tampa com as duas mãos (aprox.~20\,Nm).
\stopSteps

\page [yes]


\subsection[sec:solenoid]{Lubrificação das válvulas solenoides}

\placefig[margin][graissage_bobine]{Lubrificação das válvulas solenoides}
{\externalfigure[graissage_bobine][M]
\noteF
\startLeg
\item Bobina da válvula solenoide
\item Núcleo da bobina
\stopLeg}

Humidade e resíduos que entrem no núcleo das bobinas eletromagnéticas podem causar a corrosão dos núcleos. Os núcleos das bobinas devem ser lubrificados com massa lubrificante com cobre uma vez por ano. A massa lubrificante deve ser resistente a corrosão, água e temperaturas até 50\,°C:
\startSteps
\item Desmontar a bobina da válvula solenoide (\in{\Lone, fig.}[graissage_bobine]).
\item Lubrificar o núcleo (\in{\Ltwo, fig.}[graissage_bobine]) com a massa lubrificante especial indicada e voltar a montar a bobina.
\stopSteps


\subsection{Substituição das mangueiras}

A borracha de proteção\index{Mangueiras+Intervalos de substituição} e o tecido de reforço das mangueiras estão sujeitos a um desgaste normal. Por esse motivo, as mangueiras da instalação hidráulica devem ser obrigatoriamente substituídas nos intervalos indicados, mesmo que não haja danos {\em visíveis}.

Prestar atenção para que as mangueiras sejam corretamente enroscadas no veículo, para evitar um desgaste prematuro provocado por fricção. As mangueiras devem estar a uma distância suficiente dos restantes componentes, para evitar danos causados por fricção e vibração.

\stopsection

\page [yes]

\setups [pagestyle:bigmargin]


\startsection[title={Sistema de travagem},
							reference={sec:brake}]

\placefig[margin][fig:brake:rear]{Tambor de travão}
{\startcombination [1*2]
{\externalfigure[brake:wheelHub]}{\slx Cubo de roda traseira}
{\externalfigure[brake:drum]}{\slx Mecanismo e guarnições de travão}
\stopcombination}

Em todas as manutenções regulares, os tambores de travão~\Lfour\ devem ser desmontados, o mecanismo de travagem~\Lseven\ deve ser limpo e as guarnições de travão~\Lfive, \Lsix\  devem ser submetidas a um controlo visual (\in{fig.}[fig:brake:rear]).


\subsubject {Desmontagem}

\startSteps
\item Deslocar o veículo até uma plataforma elevatória adequada e levantar as rodas.
\item Remover as rodas.
\stopSteps


{\sla Desmontagem dos travões das rodas dianteiras}

\startSteps [continue]
\item Desmontar os tambores de travão~\Lfour.
\stopSteps

{\sla Desmontagem dos travões das rodas traseiras}

\startSteps [continue]
\item Retirar a cobertura~\Lone\ do cubo.
\item Remover o parafuso~\Ltwo\ e retirar a peça intermédia.
\item Desenroscar a porca do cubo~\Lthree\ com uma chave de caixa.
\item Retirar o cubo com o tambor de travão.
\stopSteps


\subsubject {Remontagem}

Voltar a montar o tambor de travão em ordem inversa. Apertar as porcas dos cubos de rodas traseiras~\Lthree\ com o binário de aperto indicado de 190\,Nm.

\stopsection

\page [yes]

\setups [pagestyle:normal]


\startsection[title={Controlo e manutenção dos pneus},
							reference={sec:pneumatiques}]

Os pneus\index{Pneus+Manutenção} devem estar sempre em bom estado, para que sejam capazes de cumprir as suas duas funções principais: boa aderência e excelente comportamento de travagem. Um desgaste excessivo ou uma pressão de enchimento incorreta - especialmente uma pressão demasiado baixa - são fatores de acidente significativos.


\subsection{Pontos relevantes para a segurança}

\subsubsection{Controlo de desgaste}

O desgaste dos pneus deve ser controlado com base nos indicadores de desgaste nas ranhuras (\in{fig.}[pneususure]).
Quaisquer anomalias nos pneus, bem como as suas causas, podem ser determinadas através de um controlo visual:

\placefig[margin][pneususure]{Controlo de desgaste}
{\Framed{\externalfigure[pneusUsure][M]}}

\placefig[margin][pneusdomages]{Pneus danificados}
{\Framed{\externalfigure[pneusDomages][M]}}

\startitemize
\item Desgaste nas extremidades do pneu: pressão de enchimento demasiado baixa.
\item Desgaste acentuado no centro do pneu: pressão de enchimento demasiado elevada.
\item Desgaste assimétrico nas extremidades do pneu: eixo dianteiro (via, geometria dos eixos) ajustado de forma incorreta.
\item Fissuras no pneu: pneu demasiado velho; com o tempo, a borracha do pneu vai ficando mais dura e apresenta mais fissuras (\in{fig.}[pneusdomages]).
\stopitemize

\starttextbackground[CB]
\startPictPar
\PHgeneric
\PictPar
\textDescrHead{Riscos causados por pneus desgastados}
Um pneu desgastado já não cumpre a sua função, especialmente no que concerne à circulação em pisos com água e lama; a distância de travagem aumenta e a dirigibilidade piora. Um pneu desgastado desliza muito facilmente, especialmente se o piso estiver molhado. O risco de o pneu perder a aderência ao piso aumenta.
\stopPictPar
\stoptextbackground


\subsubsection{Pressão de enchimento dos pneus}

A pressão de enchimento dos pneus recomendada está indicada na placa de características das rodas, na parte dianteira da consola, do lado do passageiro (ver \atpage [sec:plateWheel]).

Mesmo\index{Pneus+Pressão de enchimento} que os pneus estejam em bom estado, com o passar do tempo vão perdendo ar (quanto mais o veículo circular, maior é a perda de pressão). Por essa razão, a pressão dos pneus deve ser verificada mensalmente, com os pneus frios. Se a pressão dos pneus for verificada com os pneus quentes, deve-se adicionar mais 0,3\,bar do que o indicado.

\start
\setupcombinations[M]
\placefig[margin][pneuspression]{Pressão de enchimento dos pneus}
{\Framed{\externalfigure[pneusPression][M]}
\noteF
\startLeg
\item Pressão correta
\item Pressão demasiado elevada
\item Pressão demasiado baixa
\stopLeg
A pressão dos pneus recomendada está indicada na placa de características das rodas, na cabina do operador, do lado do passageiro.}
\stop

\starttextbackground[CB]
\startPictPar
\PHgeneric
\PictPar
\textDescrHead{Perigos causados por uma pressão de enchimento dos pneus demasiado baixa}
Se a pressão for demasiado baixa, o pneu pode romper. Quando não está suficientemente enchido ou o veículo está sobrecarregado, o pneu fica sujeito a uma maior compressão. Desta forma, a borracha fica quente e, durante a circulação em curvas, podem sair partes do pneu.
\stopPictPar
\stoptextbackground

\stopsection

\page [yes]

\setups[pagestyle:marginless]


\startsection[title={Chassis},
			  reference={main:chassis}]

\subsection{Fixações de componentes relevantes para a segurança}

Em todas as manutenções deve-se controlar o assento correto dos parafusos de fixação relevantes para a segurança de determinados componentes, assim como o binário de aperto indicado. Isto é especialmente válido para o sistema de direção articulada e os eixos.

\blank [big]

\startfigtext [left] [fig:frontAxle:fixing] {Eixo dianteiro}
{\externalfigure [frontAxle:fixing]}
{\sla Fixações do eixo dianteiro}
\startLeg
\item Fixação da folha de mola: \TorqueR~150\,Nm
\item Fixação das unidades de tração: \TorqueR~78\,Nm
\stopLeg

{\sla Fixações do eixo traseiro}
\startLeg
\item Fixação da folha de mola: \TorqueR~150\,Nm
\stopLeg

\stopfigtext

\start

\setupTABLE	[frame=off,style={\ssx\setupinterlinespace[line=.93\lH]},background=color,
			option=stretch,
			split=repeat]

\setupTABLE	[r]	[each]	[topframe=on,
						framecolor=white,
						% rulethickness=.8pt
						]

\setupTABLE	[c]	[odd]	[backgroundcolor=TableMiddle]
\setupTABLE	[c]	[even]	[backgroundcolor=TableLight]
\setupTABLE	[c]	[1]		[style={\bfx\setupinterlinespace[line=.93\lH]}]
\setupTABLE	[r] [first]	[topframe=off,style={\bfx\setupinterlinespace[line=.93\lH]},
						]
% \setupTABLE	[r]	[2]		[style={\bfx\setupinterlinespace[line=.93\lH]}]


\bTABLE

\bTABLEhead
\bTR [backgroundcolor=TableDark] \bTD [nc=3] Binários de aperto \eTD\eTR
% \bTR\bTD Position \eTD\bTD Type de vis \eTD\bTD Couple \eTD\eTR
\eTABLEhead

\bTABLEbody
\bTR\bTD Motores de acionamento esquerdo|/|direito \eTD\bTD M12\:×\:35~8,8 \eTD\bTD 78\,Nm \eTD\eTR
%% NOTE @Andrew: das sind Hydraulikmotoren
\bTR\bTD Bomba de trabalho \eTD\bTD M16\:×\:40~100 \eTD\bTD 330\,Nm \eTD\eTR
\bTR\bTD Bomba de acionamento \eTD\bTD M12\:×\:40~100 \eTD\bTD 130\,Nm \eTD\eTR
\bTR\bTD Folhas de mola dianteiras|/|traseiras \eTD\bTD M16\:×\:90|/|160~8,8 \eTD\bTD 150\,Nm \eTD\eTR
% \bTR\bTD Fixation du système oscillant \eTD\bTD M12\:×\:40~8.8 \eTD\bTD 78\,Nm \eTD\eTR
\bTR\bTD Fixação do recipiente de sujidade \eTD\bTD M10\:×\:30 Verbus Ripp~100 \eTD\bTD 80\,Nm \eTD\eTR
\bTR\bTD Porcas de roda \eTD\bTD M14\:×\:1,5 \eTD\bTD 180\,Nm \eTD\eTR
\bTR\bTD Fixação da escova dianteira \eTD\bTD M16\:×\:40~100 \eTD\bTD 180\,Nm \eTD\eTR
\eTABLEbody
\eTABLE
\stop


\stopsection

\page [yes]


\startmode [main:centralLubrication]

\startsection[title={Zentralschmieranlage},
							reference={main:graissageCentral}]


\subsection{Descrição do módulo de comando}

A \sdeux\ pode estar\index{Sistema de lubrificação central} equipada com um sistema de lubrificação central (opção). O sistema de lubrificação central alimenta regularmente todos os pontos de lubrificação do veículo com lubrificante.

\startfigtext [left] [vogel_affichage] {Módulo de indicação}
{\externalfigure[vogel_base2][W50]}
\blank
\startLeg
	\item Ecrã de 7 dígitos: Valores e estado de funcionamento
	\item \LED: Sistema em pausa (operação Standby)
	\item \LED: Bomba em funcionamento
	\item \LED: Comando do sistema através do interruptor de ciclo
	\item \LED: Monitorização do sistema através do pressóstato
	\item \LED: Mensagem de erro
	\item Botões de deslocamento no ecrã:
	\startLeg [R]
	\item Ativar o ecrã
	\item Exibir os valores
	\item Alterar os valores
	\stopLeg
	\item Botão para alterar o modo de operação; confirmação dos valores
	\item Acionamento de um ciclo de lubrificação intermédio
\stopLeg
\stopfigtext

O sistema de lubrificação central engloba a bomba de lubrificante, o recipiente de lubrificante transparente no lado esquerdo do chassis e o módulo de comando no sistema elétrico central.
% \blank
\page [yes]


\subsubsubject{Indicação e botões do módulo de comando}

\start

\setupTABLE	[frame=off,style={\ssx\setupinterlinespace[line=.93\lH]},background=color,
			option=stretch,
			split=repeat]

\setupTABLE	[r]	[each]	[topframe=on,
						framecolor=white,
						% rulethickness=.8pt
						]

\setupTABLE	[c]	[odd]	[backgroundcolor=TableMiddle]
\setupTABLE	[c]	[even]	[backgroundcolor=TableLight]
\setupTABLE	[c]	[1]		[width=9mm,style={\bfx\setupinterlinespace[line=.93\lH]}]
\setupTABLE	[r] [first]	[topframe=off,style={\bfx\setupinterlinespace[line=.93\lH]},
						]
% \setupTABLE	[r]	[2]		[style={\bfx\setupinterlinespace[line=.93\lH]}]


\bTABLE
\bTABLEhead
% \bTR [backgroundcolor=TableDark]
% \bTD [nc=4] Anzeige und Tasten des Steuermoduls \eTD\eTR
\bTR\bTD Pos. \eTD
\bTD \LED \eTD\bTD Modo de indicação \eTD
\bTD Modo de programação \eTD\eTR
\eTABLEhead

\bTABLEbody
	\bTR\bTD 2 \eTD
	\bTD Estado de funcionamento {\em Pausa}\hskip .5em\null \eTD
	\bTD A instalação está em Standby\hskip .5em\null \eTD % ||Betrieb
	\bTD O tempo de pausa pode ser alterado \eTD\eTR
	\bTR\bTD 3 \eTD
	\bTD Estado de funcionamento {\em Contact} \eTD
	\bTD A bomba trabalha \eTD
	\bTD O tempo de trabalho pode ser alterado \eTD\eTR
	\bTR\bTD 4 \eTD
	\bTD Controlo do sistema {\em CS} \eTD
	\bTD Com interruptor de ciclo externo \eTD
	\bTD O modo de controlo pode ser desativado ou alterado \eTD\eTR
	\bTR\bTD 5 \eTD
	\bTD Controlo do sistema {\em PS} \eTD
	\bTD Com pressóstato externo \eTD
	\bTD O modo de controlo pode ser desativado ou alterado \eTD\eTR
	\bTR\bTD 6 \eTD
	\bTD Falha {\em Fault} \eTD
	\bTD [nc=2] Existe uma falha de funcionamento. A causa da falha é
	indicada por meio de um código de erro, depois de o botão~\textSymb{vogel_DK}
	ter sido premido. A versão das funções é interrompida. \eTD\eTR
	\bTR\bTD 7 \eTD
	\bTD Teclas de seta \textSymb{vogelTop}~\textSymb{vogelBottom} \eTD
	\bTD [nc=2] \items[symbol=R]{Ativar o ecrã, consultar os parâmetros
	(modo de indicação), ajuste do valor indicado (I) (modo de programação)}
	\eTD\eTR
	\bTR\bTD 8 \eTD
	\bTD Botão \textSymb{vogelSet} \eTD
	\bTD [nc=2] Comutar entre o modo de indicação e o modo de programação ou
	confirmar os valores introduzidos. \eTD\eTR
	\bTR\bTD 9 \eTD
	\bTD Botão \textSymb{vogel_DK} \eTD
	\bTD [nc=2] Se o aparelho se encontrar no estado {\em Pausa}, ao
	premir o botão é acionado um ciclo de lubrificação intermédio. As mensagens de erro
	são confirmadas e apagadas. \eTD\eTR
\eTABLEbody
\eTABLE
\stop
\vfill

\startfigtext [left] [vogel_touches]{Módulo de indicação}
{\externalfigure[vogel_base][width=50mm]}
\textDescrHead{Modo de indicação} Premir brevemente uma das
teclas de seta~\textSymb{vogelTop}~\textSymb{vogelBottom} para ativar o ecrã de 7 dígitos~\textSymb{led_huit}. Ao premir novamente o botão~\textSymb{vogelTop}, são exibidos os diversos parâmetros e respetivos valores. O modo de {\em indicação} é indicado através dos \LED\char"2060s permanentemente acesos (\in{2~a 6, fig.}[vogel_affichage]).
\blank [medium]
\textDescrHead{Modo de programação} Para alterar os valores, premir o botão~\textSymb{vogelSet} durante, pelo menos, 2~segundos, para comutar para o modo de {\em programação}: Os \LED\char"2060s estão intermitentes. Premir brevemente o botão~\textSymb{vogelSet} para alterar a\index{Sistema de lubrificação central+Programação} indicação; seguidamente, alterar o valor pretendido através dos botões~\textSymb{vogelTop}~\textSymb{vogelBottom}. Confirmar através\index{Sistema de lubrificação central+Indicação} do botão~\textSymb{vogelSet}
.
\stopfigtext

\page [yes]


\subsection{Submenus no modo de {\em indicação}}

\vskip -9pt

\adaptlayout [height=+5mm]

\startcolumns[balance=no]\stdfontsemicn

\startSymVogel
\externalfigure[vogel_tpa][width=26mm]
\SymVogel
\textDescrHead{Tempo de pausa [h]} Premir o botão~\textSymb{vogelTop} para exibir os valores programados.
\stopSymVogel

\startSymVogel
\externalfigure[vogel_068][width=26mm]
\SymVogel
\textDescrHead{Tempo de pausa restante [h]} Tempo de pausa restante até ao ciclo de lubrificação seguinte.
\stopSymVogel

\startSymVogel
\externalfigure[vogel_090][width=26mm]
\SymVogel
\textDescrHead{Tempo de pausa total [h]} Tempo de pausa total entre dois ciclos.
\stopSymVogel

\startSymVogel
\externalfigure[vogel_tco][width=26mm]
\SymVogel
\textDescrHead{Tempo de lubrificação [min]} Premir~\textSymb{vogelTop} para exibir os valores programados.
\stopSymVogel

\startSymVogel
\externalfigure[vogel_tirets][width=26mm]
\SymVogel
\textDescrHead{Aparelho em Standby} Indicação não é possível porque o aparelho está em Standby (pausa).
\stopSymVogel

\startSymVogel
\externalfigure[vogel_026][width=26mm]
\SymVogel
\textDescrHead{Tempo de lubrificação [min]} Duração de um processo de lubrificação.
\stopSymVogel

\startSymVogel
\externalfigure[vogel_cop][width=26mm]
\SymVogel
\textDescrHead{Controlo do sistema } Premir~\textSymb{vogelTop} para exibir os valores programados.
\stopSymVogel

\startSymVogel
\externalfigure[vogel_off][width=26mm]
\SymVogel
\textDescrHead{Modo de controlo} \hfill PS:~pressóstato;\crlf
CS:~interruptor de ciclo; OFF:~desativado.
\stopSymVogel

\startSymVogel
\externalfigure[vogel_0h][width=26mm]
\SymVogel
\textDescrHead{Horas de serviço} Premir~\textSymb{vogelTop} para exibir o valor em dois passos.
\stopSymVogel

\startSymVogel
\externalfigure[vogel_005][width=26mm]
\SymVogel
\textDescrHead{Parte 1: 005} O tempo de funcionamento é exibido em duas partes; comutar para a parte~2 premindo o botão~\textSymb{vogelTop}.
\stopSymVogel

\startSymVogel
\externalfigure[vogel_338][width=26mm]
\SymVogel
\textDescrHead{Parte 2: 33,8} A 2.ª~parte do número é 33,8; no total, isso perfaz um tempo de funcionamento de 533,8\,h.
\stopSymVogel

\startSymVogel
\externalfigure[vogel_fh][width=26mm]
\SymVogel
\textDescrHead{Tempo de erro} Premir~\textSymb{vogelTop} para exibir o valor em dois passos.
\stopSymVogel

\startSymVogel
\externalfigure[vogel_000][width=26mm]
\SymVogel
\textDescrHead{Parte 1: 000} O tempo de erro é exibido em duas partes;\crlf comutar para a parte~2 premindo o botão~\textSymb{vogelTop}.
\stopSymVogel

\startSymVogel
\externalfigure[vogel_338][width=26mm]
\SymVogel
\textDescrHead{Parte 2: 33,8} A 2.ª~parte do número é 33,8; no total, isso perfaz um tempo de erro de 33,8\,h.
\stopSymVogel

\stopcolumns

\stopsection


\page [yes]

\stopmode % central lubrication

\setups [pagestyle:marginless]


\startsection[title={Plano de lubrificação para a lubrificação manual},
							reference={sec:grasing:plan}]

\starttextbackground [FC]
\startPictPar
\PMgeneric
\PictPar
Os pontos de lubrificação indicados no plano de lubrificação (\in{fig.}[fig:greasing:plan]) devem ser regularmente lubrificados. Uma lubrificação regular é fulcral para garantir uma {\em diminuição permanente da fricção} e para evitar a presença de humidade e outras substâncias corrosivas.
\stopPictPar
\stoptextbackground

\blank [big]

\start

\setupcombinations [width=\textwidth]

\placefig[here][fig:greasing:plan]{Plano de lubrificação do veículo}
{\startcombination [3*1]
{\externalfigure[frame:steering:greasing]}{\ssx Direção articulada e mecanismo pendular}
{\externalfigure[frame:axles:greasing]}{\ssx Eixos}
{\externalfigure[frame:sucMouth:greasing]}{\ssx Bocal de aspiração}
\stopcombination}

\stop

\vfill

\startLeg [columns,three]
\item Cilindro de elevação da direção articulada\crlf {\sl 2 niples de lubrificação por cilindro}
\item Rolamento da direção articulada\crlf {\sl 2 niples de lubrificação do lado esquerdo}
\columnbreak
\item Rolamento do mecanismo pendular\crlf {\sl 1 niple de lubrificação à frente do depósito}
\item Molas de lâmina\crlf {\sl 2 niples de lubrificação por folha de mola}
\columnbreak
\item Bocal de aspiração\crlf {\sl 1 niple de lubrificação por roda}
\item Bocal de aspiração\crlf {\sl 1 niple de lubrificação no braço de tração}
\stopLeg



\page [yes]


\setups [pagestyle:bigmargin]


\subsubject{Lubrificação do recipiente de sujidade}

O recipiente de sujidade possui 6~pontos de lubrificação (2\:×\:4), que devem ser lubrificados semanalmente.

\blank [big]


\placefig[here][fig:greasing:container]{Mecanismo de elevação do recipiente}
{\externalfigure[container:mechanisme]}


\placelegende [margin,none]{}
{{\sla Legenda:}

\startLeg
\item Rolamento esquerdo do recipiente (2\:×)
\item Rolamento direito do recipiente (2\:×)
\item Cilindro hidráulico esquerdo (superior)
\item Cilindro hidráulico esquerdo (inferior)

{\em Como o cilindro direito (ponto \in[greasing:point;hide]).}
\item Cilindro hidráulico direito (superior)
\item [greasing:point;hide] Cilindro hidráulico direito (inferior)
\stopLeg}

\stopsection

\page [yes]



\startsection[title={Instalação elétrica},
							reference={sec:main:electric}]

\subsection{Sistema elétrico central no chassis}

\startbuffer [fuses:preventive]
\starttextbackground [CB]
\startPictPar
\PHvoltage
\PictPar
\textDescrHead{Normas de segurança}
Observar as normas de segurança\index{Fusíveis+Chassis} destas\index{Relés+Chassis} instruções: Substituir os fusíveis apenas por fusíveis com os amperes indicados. Antes de executar trabalhos na instalação\index{Instalação elétrica} elétrica, retirar todas as jóias de metal (anéis, pulseiras, etc.).
\stopPictPar
\stoptextbackground
\stopbuffer


\subsubsubject{Fusíveis MIDI}

\starttabulate[|l|r|p|]
\HL
\NC\md F\,1 \NC 5\,A  \NC Luz de travagem, \aW{+\:15} OBD \NC\NR
\NC\md F\,2 \NC 5\,A  \NC \aW{+\:15} Comando do motor \NC\NR
\NC\md F\,3 \NC 7,5\,A \NC \aW{+\:30} Comando do motor e OBD \NC\NR
\NC\md F\,4 \NC 20\,A \NC Bomba de combustível \NC\NR
\NC\md F\,5 \NC 20\,A \NC \aW{D\:+} Dínamo, \aW{+\:15} Relés K\,1 \NC\NR
\NC\md F\,6 \NC 5\,A \NC Comando do motor \NC\NR
\NC\md F\,7 \NC 10\,A\NC Tratamento do gás de escape do motor \NC\NR
\NC\md F\,8 \NC 20\,A \NC Sistema eletrónico do motor (comando) \NC\NR
\NC\md F\,9 \NC 15\,A \NC Tratamento do gás de escape do motor, bomba de combustível, pré-incandescência \NC\NR
\NC\md F\,10\NC 30\,A \NC Comando do motor \NC\NR
\NC\md F\,11\NC 5\,A \NC Luz de marcha-atrás \NC\NR
%% NOTE @Andrew: Singular
\HL
\stoptabulate

\placefig [margin] [fig:electric:power:rear] {Sistema elétrico central no chassis}
{\externalfigure [electric:power:rear]
\noteF
\startKleg
\sym{K\,1} Aparelho eletrónico de comando do motor
\sym{K\,2} Bomba de combustível
\sym{K\,3} Liberação do motor de arranque
\sym{K\,4} Luzes de travagem
\sym{K\,5} {[}Reserva{]}
\sym{K\,6} Luz de marcha-atrás
\sym{K\,7} Sistema de pré-incandescência
\stopKleg
}


\subsubsubject{Fusíveis MAXI}

% \startcolumns [n=2]
\starttabulate[|l|r|p|]
\HL
\NC\md F\,15 \NC 50\,A \NC Alimentação principal do sistema elétrico central \NC\NR
\HL
\stoptabulate

\page [yes]

\setups[pagestyle:marginless]


\subsection{Sistema elétrico central na cabina do operador}

\startcolumns[rule=on]

\placefig [bottom] [fig:fuse:cab] {Fusíveis e relés na cabina do operador}
{\externalfigure [electric:power:front]}

\columnbreak

\subsubsubject{Relés}

\vskip -12pt

\index{Fusíveis+Cabina do operador}\index{Relés+Cabina do operador}

\starttabulate[|lB|p|]
\NC K\,2 	\NC Compressor do sistema de climatização\NC\NR
\NC K\,3 	\NC Compressor do sistema de climatização\NC\NR
\NC K\,4 	\NC Bomba de água elétrica\NC\NR
\NC K\,5 	\NC Farolim rotativo\NC\NR
\NC K\,10 \NC Gerador de frequência intermitente\NC\NR
\NC K\,11 \NC Luz de cruzamento\NC\NR
\NC K\,12 \NC Luz de estrada (máximos) {[}Reserva{]} \NC\NR
\NC K\,13 \NC Faróis de trabalho\NC\NR
\NC K\,14 \NC Limpa para-brisas - modo intermitente\NC\NR
\stoptabulate

\vskip -24pt

\placefig [bottom] [fig:fuse:access] {Tampa de acesso ao sistema elétrico central}
{\externalfigure [electric:power:cabin]}

\stopcolumns

\page [yes]


\subsubsubject{Fusíveis MINI}

\startcolumns[rule=on]
% \setuptabulate[frame=on]
%\placetable[here][tab:fuses:cab]{Fusibles dans la cabine}
%{\noteF
\starttabulate[|lB|r|p|]
\NC F\,1  \NC 3\,A \NC Luz de presença esquerda \NC\NR
\NC F\,2  \NC 3\,A \NC Luz de presença direita \NC\NR
\NC F\,3  \NC 7,5\,A \NC Luz de cruzamento esquerda \NC\NR
\NC F\,4  \NC 7,5\,A \NC Luz de cruzamento direita \NC\NR
\NC F\,5  \NC 7,5\,A \NC Luz de estrada esquerda {[}Reserva{]} \NC\NR
\NC F\,6  \NC 7,5\,A \NC Luz de estrada direita {[}Reserva{]} \NC\NR
\NC F\,7  \NC 10\,A \NC Farol de trabalho superior \NC\NR
%% NOTE @Andrew: Plural
\NC F\,8  \NC 10\,A \NC Farol de trabalho inferior (reserva) \NC\NR
%% NOTE @Andrew: Plural
\NC F\,9  \NC 10\,A \NC Escova dianteira \NC\NR
\NC F\,10 \NC 10\,A \NC Limpa para-brisas \NC\NR
\NC F\,11 \NC 5\,A \NC Interruptor - iluminação e luzes avisadoras de perigo \NC\NR
\NC F\,12 \NC 5\,A \NC {[}Reserva{]} \NC\NR
\NC F\,13 \NC 10\,A \NC Aquecimento dos espelhos retrovisores exteriores \NC\NR
\NC F\,14 \NC 7,5\,A \NC \aW{+\:15} Rádio e câmara \NC\NR
\NC F\,15 \NC 10\,A \NC \aW{+\:30} Luzes avisadoras de perigo \NC\NR
\NC F\,16 \NC 5\,A \NC Iluminação da coluna de direção \NC\NR
\NC F\,17 \NC 7,5\,A \NC \aW{+\:30} Rádio, interruptor de luz e iluminação interior \NC\NR
\NC F\,18 \NC — \NC {[}Livre{]} \NC\NR
\NC F\,19 \NC 20\,A \NC \aW{+\:30} RC\,12 à frente \NC\NR
\NC F\,20 \NC 20\,A \NC \aW{+\:30} RC\,12 atrás \NC\NR
\NC F\,21 \NC 15\,A \NC Tomada 12V \NC\NR
\NC F\,22 \NC 5\,A \NC Chave de ignição, consola multifunções, Vpad \NC\NR
\NC F\,23 \NC 5\,A \NC Paragem de Emergência, consola central, RC\,12 à frente \NC\NR
\NC F\,24 \NC 5\,A \NC Paragem de Emergência, consola central, RC\,12 atrás \NC\NR
\NC F\,25 \NC 2\,A \NC \aW{+\:15} RC\,12 à frente \NC\NR
\NC F\,26 \NC 2\,A \NC \aW{+\:15} RC\,12 atrás \NC\NR
\NC F\,27 \NC 25\,A \NC Ventilador de aquecimento \NC\NR
\NC F\,28 \NC 10\,A \NC Compressor do sistema de climatização, sistema de lubrificação central \NC\NR
\NC F\,29 \NC 25\,A \NC Condensador do sistema de climatização \NC\NR
\NC F\,30 \NC 5\,A \NC Termóstato do sistema de climatização \NC\NR
\NC F\,31 \NC 5\,A \NC \aW{+\:15} Consola multifunções|/|Vpad \NC\NR
\NC F\,32 \NC 15\,A \NC Bomba de água elétrica, farolim rotativo \NC\NR
\NC F\,33 \NC — \NC {[}Livre{]} \NC\NR
\NC F\,34 \NC — \NC {[}Livre{]} \NC\NR
\NC F\,35 \NC — \NC {[}Livre{]} \NC\NR
\NC F\,36 \NC — \NC {[}Livre{]} \NC\NR
\stoptabulate
\stopcolumns

\page [yes]

\setups [pagestyle:bigmargin]


\subsection[sec:lighting]{Dispositivo de iluminação e sinalização}


\placefig [here] [fig:lighting] {Dispositivo de iluminação e sinalização do veículo}
{\externalfigure [vhc:electric:lighting]}

\placelegende [margin,none]{}{%
\vskip 12pt
\stdfontsemicn
{\sla Legenda:}
\startLongleg
\item Luzes de presença\hfill 12\,V–5\,W
\item Luzes de cruzamento\hfill H7~12\,V–55\,W
\item Luzes indicadoras de mudança de direção\hfill Laranja 12\,V–21\,W
\item Faróis de trabalho\hfill G886~12\,V–55\,W
\item Indicador de mudança de direção\hfill 12\,V–21\,W
\item Luzes traseiras|/|de travagem\hfill 12\,V–5|/|21\,W
\item Luzes de marcha-atrás\hfill 12\,V–21\,W
\item {[}Livre{]}
\item Luz de matrícula\hfill 12\,V–5\,W
\item Farolim rotativo\hfill H1~12\,V–55\,W
\stopLongleg}

\subsubsubject{Ajuste dos faróis}

\placefig [margin] [fig:lighting:adjustment] {Feixe luminoso a 5\,m}
{\externalfigure [vhc:lighting:adjustment]
\startitemize\stdfontsemicn
\sym{H\low{1}} Altura do filamento: 100\,cm
\sym{H\low{2}} Correção a 2\hairspace\%: 10\,cm
\stopitemize}

{\md Requisitos:} recipiente de água limpa|/|água de reciclagem cheio, operador no posto de comando.

A projeção dos faróis é pré-ajustada de fábrica. A altura e a inclinação do feixe luminoso podem ser pré-ajustadas, girando o suporte de plástico.

Se, durante uma verificação, se constatar que é necessário alterar o ajuste, soltar o parafuso de segurança e ajustar a inclinação, de modo a que esta
esteja em conformidade com as normas legais (ver \in{fig.}[fig:lighting:adjustment]). Apertar novamente o parafuso de segurança.

\page [yes]

\setups [pagestyle:marginless]


\subsection[sec:battcheck]{Bateria}

\subsubsection{Normas de segurança}

\startSymList
\PPfire
\SymList
\textDescrHead{Perigo de explosão}
Durante\index{Bateria+Avisos de segurança}\index{Perigo+Explosão} o carregamento das baterias forma-se uma mistura gasosa de oxigénio e hidrogénio\index{Mistura gasosa de oxigénio e hidrogénio} explosiva. Carregar as baterias apenas em espaços bem ventilados! Evitar a formação de faíscas!
Não fazer fogo, chama aberta ou fumar na proximidade da bateria.
\stopSymList

\startSymList
\PHvoltage
\SymList
\textDescrHead{Perigo de curto-circuito}
Se\index{Bateria+Manutenção} o borne do terminal positivo da bateria conectada entrar em contacto com os componentes do veículo, existe\index{Perigo+Fogo}\index{Perigo+Explosão} perigo de curto-circuito.
Desta forma, a mistura de gás que sai da bateria pode explodir, provocando ferimentos graves ao operador e a terceiros.

\startitemize
\item Não colocar objetos metálicos ou ferramentas na bateria.
\item Ao desconectar a bateria, desconectar sempre primeiro o borne do terminal negativo e depois o borne do terminal positivo.
\item Ao conectar a bateria, conectar sempre primeiro o borne do terminal positivo e depois o borne do terminal negativo.
\item Com o motor em funcionamento, não soltar ou remover os terminais de ligação da bateria.
\stopitemize
\stopSymList


\startSymList
\PHcorrosive
\SymList
\textDescrHead{Perigo de corrosão}
Utilizar\index{Perigo+Corrosão} óculos de proteção e luvas de proteção à prova de ácidos. O líquido da bateria é 27\percent de ácido sulfúrico (H\low{2}SO\low{4}), pelo que pode causar corrosão.
Se o\index{Bateria+Perigo}\index{Líquido da+ Bateria} líquido da bateria entrar em contacto com a pele, aplicar uma solução de bicarbonato de sódio e lavar com água limpa.  Se o líquido da bateria entrar em contacto com os olhos, passar abundantemente por água fria e consultar imediatamente um médico.
\stopSymList

\startSymList
\startcombination[1*2]
 {\PHcorrosive}{}
 {\PHfire}{}
 \stopcombination
\SymList
\textDescrHead{Armazenamento das baterias}
Armazenar\index{Bateria+Armazenamento} as baterias sempre em posição vertical. Caso contrário, o líquido das baterias poderá vaziar, provocando corrosão ou~– em caso de reação com outras substâncias~– até incêndios. \par\null\par\null
\stopSymList

\testpage [16]

\starttextbackground [FC]
\setupparagraphs [PictPar][1][width=2.4em,inner=\hfill]

\startPictPar
\PMproteyes
\PictPar
\textDescrHead{Óculos de proteção}
Durante\index{Perigo+Lesões oculares} a mistura de água e ácidos, o líquido pode salpicar para os olhos. Se isso acontecer, passar imediatamente por água limpa e consultar um médico!
\stopPictPar
\blank [small]

\startPictPar
\PMrtfm
\PictPar
\textDescrHead{Documentação}
Durante o manuseio de baterias, observar impreterivelmente os avisos de segurança, medidas de segurança e modos de procedimento indicados neste manual de instruções.
\stopPictPar
\blank [small]

\startPictPar
\PStrash
\PictPar
\textDescrHead{Proteção do meio ambiente}
As baterias\index{Proteção do meio ambiente} contêm substâncias nocivas. Não eliminar as baterias usadas no lixo doméstico. Eliminar as baterias de acordo com as normas de proteção do meio ambiente. Entregar as baterias usadas numa oficina especializada ou num ponto de recolha para baterias usadas.

Transportar e armazenar as baterias cheias sempre de forma adequada. Durante o transporte, proteger as baterias contra tombamento. O líquido da bateria pode vazar pelos orifícios de purga dos tampões de fecho e penetrar no meio ambiente.
\stopPictPar
\stoptextbackground

\page [yes]

\setups[pagestyle:normal]


\subsubsection{Dicas úteis}

De modo a garantir uma longa vida útil, a bateria deverá estar sempre o mais carregada possível.

Uma\index{Bateria+Vida útil} carga de manutenção da bateria durante os tempos de paragem prolongados do veículo aumenta não apenas a vida útil da bateria, como também garante uma constante operacionalidade do veículo.

\placefig[margin][fig:batterycompartment]{\select{caption}{Compartimento da bateria (tampa de manutenção)}{Compartimento da bateria}}
{\externalfigure[batt:compartment]}


\subsubsection{Conservação}

A bateria da \sdeux\ é uma bateria de chumbo-ácido {\em livre de manutenção}. Para além de manter a bateria carregada e limpa, não são necessárias quaisquer medidas de conservação.

\startitemize
\item Prestar atenção para que os polos da bateria estejam sempre limpos e secos. Lubrificar ligeiramente os polos com um pouco de massa lubrificante à prova de ácidos.
\item Carregar as baterias que\index{Bateria+Carregar} apresentam uma
tensão de repouso\index{Bateria+Tensão de repouso} inferior a 12,4\,V.
\stopitemize

\placefig[margin][fig:bclean]{Limpeza dos polos}
{\externalfigure[batt:clean]
\noteF
Utilizar\index{Bateria+Limpeza}\index{Limpeza+Baterias} água quente para remover o pó branco que se formou devido à corrosão. Se um dos polos apresentar ferrugem, desconectar o cabo da bateria e limpar o polo com uma escova metálica. Aplicar uma camada fina de massa lubrificante nos polos.}


\subsubsection[sec:battery:switch]{Utilização do interruptor seccionador da bateria}

{\sl Não é recomendável que o interruptor seccionador da bateria seja acionado com frequência (por exemplo, diariamente)!}

\startSteps
\item Desligar\index{Interruptor seccionador da bateria} a ignição e aguardar cerca de 1~minuto.
\item Abrir o compartimento da bateria (\inF[fig:batterycompartment]).
\item Premir o botão vermelho do interruptor seccionador da bateria para interromper o circuito elétrico.
\item Para ativar novamente o circuito elétrico, rodar o interruptor seccionador da bateria em ¼~de rotação no sentido horário.
\stopSteps

\stopsection

\page [yes]


\setups[pagestyle:marginless]

\section[sec:cleaning]{Limpeza do veículo}

Antes\startregister[index][vhc:lavage]{Manutenção+Limpeza} da limpeza, remover toda a lama e sujidade da carroçaria, passando por água. Não lavar apenas as superfícies laterais, mas também a zona das rodas e o lado inferior do veículo.

Especialmente no inverno é importante lavar bem o veículo para remover os resíduos altamente corrosivos de\index{Corrosão+Prevenção} sal para estradas.

\starttextbackground [FC]
\startPictPar
\PHgeneric
\PictPar
\textDescrHead{Evitar danos causados por água}
Nunca limpar o veículo com {\em canhões de água} (\eG\ dos bombeiros) ou {\em produtos de limpeza a frio à base de hidrocarbonetos.} Se for utilizada uma lavadora a vapor de alta pressão, observar as normas aplicáveis (indicadas mais à frente).
\stopPictPar
\blank[small]

\startPictPar
\pTwo[monde]
\PictPar
\textDescrHead{Proteção do meio ambiente}
A limpeza do veículo pode ter um grande impacto sobre o meio ambiente.
Limpar o veículo apenas em sítios onde haja um \index{Proteção do meio ambiente} separador de óleo. Observar as normas de proteção do meio ambiente em vigor.
\stopPictPar
\blank[small]

\startPictPar
\PMwarranty
\PictPar
\textDescrHead{Limpar de forma adequada!}
A Marcel Boschung AG não assume qualquer responsabilidade, nem concede direitos de garantia, por danos que sejam resultantes da inobservância das normas de limpeza.
\stopPictPar
\stoptextbackground

\page [yes]


\subsection{Limpeza de alta pressão}

Para a limpeza de alta pressão\index{Limpeza+Alta pressão} do veículo pode ser utilizada uma lavadora de alta pressão convencional.

Durante a limpeza de alta pressão, observar os seguintes pontos:

\startitemize
	\item Pressão de trabalho máx. 50\,bar
	\item Bocal de jato plano com um ângulo de pulverização de 25°
	\item Distância de pulverização mín. 80\,cm
	\item Temperatura da água máx. 40\,°C
	\item Observar a secção \about[reiMi], \atpage[reiMi].
\stopitemize

A inobservância destas\index{Pintura+Danos} normas pode provocar danos na pintura e na proteção anticorrosiva\index{Danos+Pintura}.

Observar também o manual de instruções e as normas de segurança da lavadora de alta pressão.

\starttextbackground[FC]
\startPictPar\PPspray\PictPar
Durante a limpeza de alta pressão, a água pode penetrar em pontos onde pode causar danos. Por esse motivo, nunca direcionar o jato de água diretamente sobre componentes sensíveis e aparelhos:
\stopPictPar

\startitemize
	\item Sensores, ligações elétricas e conexões
	\item Motor de arranque, dínamo, sistema de injeção
	\item Válvulas solenoides
	\item Orifícios de ventilação
	\item Componentes mecânicos que ainda não tenham arrefecido
	\item Sinais de aviso, advertência e perigo
	\item Aparelhos de comando eletrónicos
\stopitemize

\textDescrHead{Limpeza do motor}
Evitar impreterivelmente a entrada de água nos orifícios de aspiração, ventilação e purga. Com lavadoras de alta pressão, não direcionar o jato de água diretamente sobre componentes e condutores elétricos. Não direcionar o jato sobre o sistema de injeção! Após a limpeza do motor, conservar o motor; ao fazê-lo, proteger a correia do produto de conservação.
\stoptextbackground

\starttextbackground [FC]
\setupparagraphs [PictPar][1][width=6em,inner=\hfill]
\startPictPar
\framed[frame=off,offset=none]{\PMproteyes\PMprotears}
\PictPar
\textDescrHead{Água residual}
	Durante a limpeza do veículo há uma acumulação de água em determinadas 	zonas do veículo (\eG\ nas covas do grupo motor ou da engrenagem); remover esta água 	com ar comprimido. Ter em consideração que durante os trabalhos com ar comprimido deve ser utilizado um equipamento de proteção adequado e que a instalação deve estar em conformidade com as 	normas de segurança em vigor (bocal múltiplo).
\stopPictPar
\stoptextbackground


\subsubsection[reiMi]{Produtos de limpeza adequados}

Utilizar\index{Produtos de limpeza} apenas produtos de limpeza com as seguintes propriedades:

\startitemize
	\item Não abrasivo
	\item Valor PH de 6–7
	\item Sem solventes
\stopitemize

Para eliminar sujidades de difícil remoção, aplicar cuidadosamente benzina de lavagem ou álcool sobre pequenas superfícies; não utilizar outros produtos que contenham solventes.
Remover os resíduos de solvente da pintura. A limpeza de componentes plásticos com benzina pode causar fissuras ou descolorações!

Limpar os\index{Limpeza+Autocolantes} sinais de advertência ou aviso com água limpa e uma esponja macia.

Evitar a entrada de água em componentes elétricos: não utilizar uma escova para limpar a carcaça das luzes; utilizar antes
um pano ou uma esponja macia.

\starttextbackground [CB]
\startPictPar
\GHSgeneric\par
\GHSfire
\PictPar
\textDescrHead{Perigo devido a químicos}
A utilização de produtos de limpeza pode acarretar perigos para a saúde e riscos de segurança (substâncias facilmente inflamáveis). Observar as normas de segurança aplicáveis dos produtos de limpeza utilizados; observar as fichas de perigos e dados dos produtos utilizados.
\stopPictPar
\stoptextbackground

\stopregister[index][vhc:lavage]


\page [yes]


\setups [pagestyle:bigmargin]

\startsection	[title={Ajuste do bocal de aspiração},
				 reference={sec:main:suctionMouth}]


A distância ideal\index{Bocal de aspiração+Ajuste} entre a estrada e a calha de plástico do bocal de aspiração é de 8\,mm.
Para controlar ou ajustar a distância, utilizar as três barras de ajuste que se encontram na caixa de ferramentas (cabina do operador, lado do operador).


\placefig [margin] [fig:suctionMouth] {Ajuste do bocal de aspiração}
{\Framed{\externalfigure [suctionMouth:adjust]}}

\placeNote[][service_picto]{}{%
\noteF
\starttextrule{\PHasphyxie\enskip Perigo de intoxicação e asfixia \enskip}
{\md Aviso:} durante os trabalhos de ajuste, o motor do veículo deve estar em funcionamento, para que o bocal de aspiração possa ser mantido em posição flutuante. Por isso, para que não haja perigo de intoxicação ou asfixia, deve ser obrigatoriamente utilizado um sistema de extração de gases de escape ou os trabalhos devem ser realizados num local bem ventilado.
\stoptextrule}

\startSteps
\item Estacionar o veículo num local bem ventilado, sobre uma superfície plana.
\item Ativar\index{Extração} o modo \aW{ Trabalho} (premir o botão na parte exterior da alavanca de seleção da velocidade de marcha).

Deixar o motor trabalhar ao ralenti. (Premir o botão~\textSymb{joy_key_engine_decrease} na consola multifunções para reduzir a velocidade de rotação do motor.)
\item Engatar o travão de mão e colocar calços nas rodas traseiras.
\item Premir o botão~\textSymb{joy_key_suction} para baixar o bocal de aspiração.
\item Posicionar as três barras de ajuste~\LAa\ por baixo da calha de plástico do bocal de aspiração, conforme ilustrado na figura.
\item [sucMouth:adjust]Soltar os parafusos de fixação~\Lone\ e ajuste~\Ltwo\ de cada roda; as quatro rodas descem, ficando sobre o chão.
\item Voltar a apertar os parafusos~\Lone\ e~\Ltwo\  e remover as três barras de ajuste.
\item Subir|/|descer o bocal de aspiração e verificar o ajuste através das barras de ajuste. Se o ajuste ainda não estiver correto, repetir o processo a partir do ponto~\in[sucMouth:adjust].

\stopSteps


\stopsection
\stopchapter
\stopcomponent

