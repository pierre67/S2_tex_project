\startcomponent c_40_control_s2_140-pt
\product prd_ba_s2_140-pt


\startchapter [title={Elementos de comando do veículo},
							reference={chap:ctrl}]

\setups[pagestyle:marginless]

\placefig[here][fig:ctrl:cab:front]{Elementos de comando}
{\externalfigure[ctrl:cab:front]}

\startcolumns [n=3]
\startLongleg
	\item Coluna de direção (\in{§}[sec:steeringColumn])
	\item Ajuste da coluna de direção
	\item Pedal do acelerador e do travão
	\item Computador de bordo \Vpad~SN (\inP[sec:vpad])
	\item Consola do tejadilho (\inP[sec:ctrl:aux])
	\item Rádio|/|MP3
\stopLongleg


\subsubsubject{Equipamento opcional}

\startLongleg [continue]
	\item Monitor de marcha-atrás
\stopLongleg
\stopcolumns

\startsection [title={A coluna de direção},
							reference={sec:steeringColumn}]

\subsection{Ajuste da coluna de direção}

\textDescrHead{Inclinação do volante} Pressionar o pedal~\Ltwo e, simultaneamente, ajustar a inclinação da coluna de direção. Soltar o pedal para voltar a bloquear o mecanismo da coluna de direção.

\page[yes]
\setups [pagestyle:normal]


\subsection{Dispositivos de iluminação e sinalização}

\placefig [margin] [fig:column:left] {Alavanca multifunções e interruptor rotativo}
{\externalfigure[ctrl:column:left]}

\placefig [margin] [fig:column:right] {Alavanca de seleção da velocidade de marcha}
{\externalfigure[ctrl:column:right]}


\subsubsubject{Interruptor rotativo}

\startitemize[width=1.7em]
\sym{\textSymb{com_lowlight}} Luz de cruzamento (rodar\TorqueR).
\startitemize
\sym{1} Luz de presença
\sym{2} Luz de cruzamento
\stopitemize
\stopitemize


\subsubsubject{Alavanca multifunções}

\startitemize[width=1.7em]
\sym{\textSymb{com_lowlight}} {[}Não ocupado{]}
\sym{\textSymb{com_light}} Alarme luminoso (pressionar a alavanca brevemente para cima)
\sym{\textSymb{com_blink}} Indicador de mudança de direção (alavanca para a frente|/|para trás)
\sym{\textSymb{com_claxonArrow}} Buzina (pressionar o botão na parte exterior da alavanca)
\sym{\textSymb{com_wipper}} Limpa para-brisas
\startitemize
\sym{J} Intermitente
\sym{O} Desligado
\sym{I} 1.ª\,velocidade
\sym{II} 2.ª\,velocidade
\stopitemize
\sym{\textSymb{com_washerArrow}} Sistema limpa para-brisas (pressionar a coroa na extremidade da alavanca).
\stopitemize


\subsubsubject{Alavanca de seleção da velocidade de marcha}

As funções da alavanca de seleção da velocidade de marcha são descritas detalhadamente no capítulo~\about[chap:using], a partir da~\atpage[sec:using:start].

\stopsection

\page [yes]


\startsection [title={Outras funções},
							reference={sec:ctrl:add}]


\subsection[sec:ctrl:aux]{Consola do tejadilho}

{\sl A\index{Consola do tejadilho} consola do tejadilho encontra-se na parte dianteira do tejadilho da cabina do operador, do lado do operador.}
\placefig [margin] [fig:console:aux] {Consola do tejadilho}
{\externalfigure[ctrl:console:aux]}


\placefig [margin] [fig:console:climat] {Aquecimento e sistema de climatização}
{\externalfigure[ctrl:console:climat]}


\startitemize [unpacked][width=1.7em]
\sym{\textBigSymb{temoin_retrochauffant}} Aquecimento do espelho retrovisor exterior
\sym{\textBigSymb{temoin_hazard}} Luzes avisadoras de perigo
\sym{\textBigSymb{temoin_eclairage_L}} Faróis de trabalho
\stopitemize


\subsubsubject{Equipamento opcional}

\startLeg [unpacked][width=1.7em]
\sym{\textBigSymb{temoin_buse}} Bomba de água de alta pressão para pistola de água \crlf {\sl ver \atpage[sec:using:water:spray]}
\sym{\textBigSymb{temoin_aspiration_manuelle}} Turbina para mangueira de aspiração manual \crlf {\sl ver \atpage[sec:using:suction:hose]}
\stopLeg


\subsection[sec:ctrl:climat]{Aquecimento e sistema de climatização}

{\sl Esta consola\index{Consola de aquecimento} encontra-se na parede traseira da cabina do operador, entre os bancos.}

\startitemize [unpacked][width=23mm]
\sym{\bf 0\quad I\quad II\quad III} Interruptor rotativo do ventilador
\sym{\externalfigure[tirette_chauffage][height=1em]} Regulador da temperatura
\stopitemize


\subsubsubject{Equipamento opcional}

\startitemize [unpacked][width=1.7em]
\sym{\textBigSymb{temoin_climat_bk}} Sistema de climatização
\stopitemize

\page [yes]

\setups [pagestyle:bigmargin]


\subsection[sec:ctrl:central]{Consola central}

{\sl A\index{Consola central} consola central encontra-se entre os bancos.}

\placefig [margin] [fig:console:central] {Consola central}
{\externalfigure[ctrl:console:central]}


\subsubsubject{Humidificação das escovas}

\startLeg [unpacked][width=1.7em]
\sym{\textBigSymb{temoin_busebalais}} Bomba de água de baixa pressão\index{Bomba de água} para o sistema de humidificação\index{Bomba de água+Humidificação} das escovas. (Posição~1: \aW{automático}; Posição~2: \aW{permanente})
\stopLeg


\subsubsubject{Rebatimento do recipiente de sujidade}

\setupinmargin[right][style=normal]
\inright{%
\startitemize
\sym{\textSymb{mand_readtheoperatingmanual}} Observar as instruções relativas à utilização do travão de mão na \atpage[sec:using:stop].
\stopitemize}

\startLeg [unpacked][width=1.7em]
\sym{\textBigSymb{temoin_kipp2}} Rebatimento do recipiente de sujidade.
Para que seja possível rebater\index{Recipiente de sujidade+Rebatimento} o recipiente de sujidade, o travão de mão deve estar acionado
e a alavanca de seleção da velocidade de marcha deve encontrar-se na posição "Neutro".
\stopLeg


\subsubsubject{Desativação de Emergência}

\starttextbackground [FC]
\startPictPar
\externalfigure[Emergency_Stop][Pict]
\PictPar
Num caso de emergência\index{Desativação de Emergência}, os aparelhos de aspiração e varredura, assim como o mecanismo de translação, podem ser desligados pressionando o comutador de Desativação de Emergência.
\stopPictPar
\stoptextbackground


\subsection[sec:foot:switch]{Interruptor de pedal}

\placefig [margin] [fig:foot:switch] {Interruptor de pedal}
{\vskip 60pt
\externalfigure[work:foot:switch]}

Através\index{Interruptor de pedal} deste interruptor na base da coluna de direção (\inF[fig:foot:switch]) é possível descer as escovas de forma rápida e simples, sempre que for necessário (\eG\ no topo de uma subida, ao subir para um passeiro).

\stopsection
\page[yes]
\setups [pagestyle:marginless]


\startsection[title={Consola multifunções},
							reference={ctrl:console:middle}]

\startlocalfootnotes

\startfigtext[left]{Consola multifunções}
{\externalfigure[overview:joy:large]}


\subsubsubject{Joysticks}

\textDescrHead{Sem escova dianteira (ou escova dianteira desativada):}
Cada Joystick comanda, independentemente do outro, uma escova: subir|/|descer~(\textSymb{joystick_aa}) ou esquerda|/|direita~(\textSymb{joystick_gd}). O Joystick esquerdo comanda a escova esquerda, o Joystick direito comanda a escova direita.\footnote{Num veículo com escova dianteira (opção), para poder alterar a posição das escovas laterais, é necessário desativar a escova dianteira (botão~\textSymb{joy_key_frontbrush_act}).}

\textDescrHead{Com escova dianteira:}
Através do Joystick esquerdo é possível subir|/|descer (\textSymb{joystick_aa}) a escova dianteira e movê-la para a esquerda|/|direita (\textSymb{joystick_gd}). Através do Joystick direito é possível inclinar a escova no seu eixo longitudinal~(\textSymb{joystick_aa}) e transversal~(\textSymb{joystick_gd}).

\placelocalfootnotes %[height=\textheight]
\stopfigtext
\stoplocalfootnotes
\vfill


\subsubsubject{Botões laterais}

\startcolumns

\startPictList
\VPcltr
\PictList
Cruise Control: aumentar a velocidade ajustada
\stopPictList\vskip -3pt

\startPictList
\VPclbr
\PictList
Cruise Control: reduzir a velocidade ajustada
\stopPictList\vskip -3pt

\startPictList
\VPcrtr
\PictList
Subir o bocal de aspiração
\stopPictList

\column


\startPictList
\VPcrbr
\PictList
Descer o bocal de aspiração
\stopPictList\vskip -3pt

\startPictList
\VPcrtf
\PictList
Abrir a tampa de sujidade grossa (na parte dianteira do bocal de aspiração)
\stopPictList\vskip -3pt

\startPictList
\VPcrbf
\PictList
Fechar a tampa de sujidade grossa
\stopPictList

\stopcolumns


\subsubsubject{Botões com símbolos}

\startcolumns

\startSymVpad
\externalfigure[joy:stop]
\SymVpad
\textDescrHead{Stop} Parar o aparelho ativado:

Pressionar 1\:×: desativar a 3.ª\,escova\crlf
Pressionar 2\:×: desativar tudo
\stopSymVpad

\startSymVpad
\externalfigure[joy:tempomat]
\SymVpad
\textDescrHead{Cruise Control} Ajustar o Cruise Control na velocidade atual e ativá-lo. Para desativar, pressionar novamente o botão~\textSymb{joy:tempomat} ou travar. Para aumentar|/|reduzir a velocidade, pressionar os botões laterais.
\stopSymVpad

\startSymVpad
\externalfigure[joy:ftbrs:minus]
\SymVpad
\textDescrHead{Velocidade das escovas}Reduzir a velocidade de rotação das escovas laterais ou da escova dianteira.
\stopSymVpad

\startSymVpad
\externalfigure[joy:ftbrs:plus]
\SymVpad
\textDescrHead{Velocidade das escovas}Aumentar a velocidade de rotação das escovas laterais ou da escova dianteira.
\stopSymVpad

\startSymVpad
\externalfigure[joy:eng:minus]
\SymVpad
\textDescrHead{Velocidade de rotação do motor} Reduzir a velocidade de rotação do motor Diesel.
\stopSymVpad

\startSymVpad
\externalfigure[joy:eng:plus]
\SymVpad
\textDescrHead{Velocidade de rotação do motor} Aumentar a velocidade de rotação do motor Diesel.
\stopSymVpad
\columnbreak

\startSymVpad
\externalfigure[joy:suc]
\SymVpad
\textDescrHead{Aspiração} Ativar o sistema de aspiração: o bocal de aspiração é descido,
a turbina e a bomba de água de reciclagem são ligadas.\note [recyclingwaterpump] \crlf
Pressionar o botão Stop~\textSymb{joy:stop} para desativar o sistema.
\stopSymVpad

\startSymVpad
\externalfigure[joy:sucbrs]
\SymVpad
\textDescrHead{Varrer|/|aspirar}  Ativar o sistema de aspiração|/|varredura: o bocal de aspiração é descido, as escovas laterais são descidas e posicionadas, a turbina, a escova e a bomba de água de reciclagem são ligadas.\note [recyclingwaterpump] \crlf
Pressionar o botão Stop~\textSymb{joy:stop} para desativar o sistema.
\stopSymVpad

\footnotetext[recyclingwaterpump]{A bomba de água limpa também é ligada quando o interruptor~\textBigSymb{temoin_busebalais} se encontra na posição \aW{Automático} (ver \in [sec:ctrl:central] na \atpage [sec:ctrl:central]).}
\startSymVpad
\externalfigure[joy:ftbrs:act]
\SymVpad
\textDescrHead{Escova dianteira ativada} Ativar|/|desativar a escova dianteira.
%% NOTE @Andrew: Singular
\stopSymVpad

\startSymVpad
\externalfigure[joy:ftbrs:right]
\SymVpad
\textDescrHead{Escova dianteira esquerda} Sentido de rotação para trabalhos com a escova dianteira do lado esquerdo
(sentido de rotação: sentido horário).
\stopSymVpad

\startSymVpad
\externalfigure[joy:ftbrs:left]
\SymVpad
\textDescrHead{Escova dianteira direita} Sentido de rotação para trabalhos com a escova dianteira do lado direito
(sentido de rotação: sentido anti-horário).
\stopSymVpad

\stopcolumns

\stopsection

\stopchapter

\stopcomponent











