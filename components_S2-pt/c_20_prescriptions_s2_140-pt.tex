\startcomponent c_20_prescriptions_s2_140-pt
\product prd_ba_s2_140-pt


\chapter [safety:risques] {Normas de segurança}

\setups [pagestyle:marginless]


\section{Instruções gerais}

\subsubject{Normas legais}

Os acidentes podem ter consequências graves, quer para a entidade empregadora, quer para o empregado. Assim alertamos novamente para os deveres de ambas as partes:\note[prescription:user:right].

Antes de incumbir um empregado da operação da máquina varredora, a entidade empregadora deve observar os seguintes pontos:

\startSteps
\item Todos os operadores do veículo devem ter recebido formação para a operação do veículo. O título de formação deve estar disponível.
\item Todos os operadores do veículo devem possuir uma carta de condução que os habilite para a condução deste tipo de veículos. Esta apenas pode ser emitida se as seguintes três condições forem cumpridas:
\startitemize [2]
\item O empregado realizou exames médicos de aptidão, com o médico do trabalho, tendo os resultados dos mesmos sido satisfatórios.
\item O empregado conhece os locais onde decorrerão os trabalhos e está familiarizado com todas as normas de segurança em vigor, que lhe foram transmitidas pelo seu superior hierárquico.
\item O empregado realizou um teste de aptidão que avalia os conhecimentos necessários para a operação do veículo, tendo o resultado do mesmo sido satisfatório.
\stopitemize
\stopSteps

Se a velocidade máxima do veículo for superior a 25\,km/h\note[prescription:user:right], o veículo necessita de uma autorização oficial para circular na via pública e o operador do veículo deve possuir uma carta de condução da seguinte categoria:
\startitemize
\item Carta de condução da categoria B\note[prescription:lisence] para veículos com um peso total inferior a 3,5~t, ou
\item Carta de condução da categoria C\note[prescription:lisence] para veículos com um peso total superior a 3,5~t.
\stopitemize

Se a velocidade máxima do veículo for de 25\,km/h, o operador do veículo deve estar, pelo menos, familiarizado com o código da estrada, mesmo que para a operação do veículo não seja necessário possuir uma carta de condução da categoria B\note[prescription:user:right].

\footnotetext [prescription:user:right] {As obrigações da entidade empregadora e do empregado podem variar consoante o país ou a região. Informe-se sobre as normas em vigor no seu país ou na sua região.}

\footnotetext[prescription:lisence] {Diretiva 2006/126/CE do Parlamento Europeu e do Conselho de 20 de~dezembro de 2006 relativa à carta de condução.}


\subsubject{Condições de uso}

A \sdeux\ apenas pode ser utilizada se se encontrar em perfeitas condições de funcionamento. Além disso, o operador deve respeitar as instruções de segurança e normas indicadas no presente manual de instruções. Falhas de funcionamento que influenciem a segurança devem ser imediatamente eliminadas|/|reparadas por uma empresa especializada.
\blank [big]

\startSymList
\externalfigure [s2_inspection] [width=4.5em]
\SymList
{\md Manutenção diária:}
após os trabalhos, inspecionar o veículo e proceder à reparação de danos e defeitos visíveis. Em caso de danos ou falhas de funcionamento, contactar imediatamente a oficina especializada. Se isso não for possível, parar imediatamente o veículo e tomar medidas para proteger a zona afetada.
\stopSymList


\subsubject{Utilização adequada}

A \sdeux\ foi concebida para a realização de trabalhos de limpeza e conservação de estradas, vias e praças. Qualquer outro tipo de utilização é considerado inadequado. Em caso de utilização inadequada, a \boschung\ não assume qualquer responsabilidade por eventuais danos ocorridos. Nesse caso, a responsabilidade é inteiramente da entidade operadora. {\em A utilização adequada pressupõe igualmente o cumprimento das instruções de segurança e do plano de manutenção do presente manual de instruções.}


\section{Circulação em vias públicas}

\subsubject{Normas gerais}

Para além das instruções deste manual, devem ser cumpridas todas as normas gerais aplicáveis, as normas legais em vigor e as prescrições para a prevenção de acidentes e proteção do meio ambiente.


\subsubject{Lugar do passageiro}

Este lugar destina-se para um passageiro~/ uma passageira, o chamado {\em banco do passageiro}.


\subsubject{Cinto de segurança}

\startSymList
% \externalfigure [prescription:safety:belt]
\PMbelt
\SymList
De acordo com o código da estrada, o operador e o passageiro da \sdeux\ devem colocar o cinto de segurança quando entram no veículo.
\stopSymList


\subsubject{Ver e ser visto}

\startSymList
\externalfigure [travaux_deviation] [width=3.5em]
\SymList
Assegurar que os outros condutores conseguem ver bem o veículo, especialmente durante trabalhos em estradas muito movimentadas.

Se durante a execução de uma determinada manobra ou atividade o operador do veículo não tiver uma boa visibilidade, deve solicitar o auxílio de uma pessoa auxiliar, com a qual ele mantém contacto visual.
\stopSymList


\subsubject{Iluminação e dispositivos de sinalização}

Consoante o código da estrada em vigor, poderá event. ser necessário
ligar os faróis e/ou o farolim rotativo do veículo durante o dia.


\subsubject{Utilização de telemóveis}

\startSymList
\PPphone
\SymList
Durante a circulação na via pública é proibido utilizar telemóveis ou aparelhos de comunicação por radiofrequência - exceto se o veículo estiver equipado com um sistema de mãos livres.

Telefonemas\index{Segurança+Telemóvel} durante a operação do veículo~– mesmo que por meio do sistema de mãos livres~– influenciam negativamente a concentração do operador.
\stopSymList


\section{Normas de manutenção}

\subsubject{Instruções de manutenção}

Antes de iniciar os trabalhos, os técnicos de manutenção devem ler o manual de instruções da \sdeux, especialmente as secções relativas à segurança e manutenção.


\subsubject{Qualificações necessárias}

\startSymList
\externalfigure [mecanicienne] [width=3.5em]
\SymList
Os trabalhos de manutenção na \sdeux\  apenas podem ser realizados por pessoas que tenham adquirido os conhecimentos necessários por meio de uma formação adequada. Isto é especialmente válido para os trabalhos no motor, no sistema de travagem, na direção e na instalação elétrica e hidráulica.
\stopSymList


\testpage [6]
\subsubject{Supervisão}

\startSymList
\externalfigure [mecanicien_hyerarchie] [width=3.5em]
\SymList
As pessoas que estejam a fazer uma formação~– estágio ou aprendizagem~– apenas podem executar trabalhos no veículo mediante supervisão de um técnico especializado. Verificar aleatoriamente se os técnicos cumprem as normas do manual de instruções e de segurança.
\stopSymList


\subsubject{Trabalhos de soldadura}

\startSymList
\externalfigure [pince_soudure2] [width=3.5em]
\SymList
Antes de proceder à execução de trabalhos de soldadura na carroçaria ou no chassis,
desconectar impreterivelmente a bateria e todos os aparelhos de comando eletrónicos.
\stopSymList


\testpage [5]


\subsubject{Limpeza do veículo}

\startSymList
\externalfigure [washer_pressure] [width=3.5em]
\SymList
Antes de proceder à limpeza da \sdeux\, ler a secção \about[sec:cleaning], a partir da \atpage[sec:cleaning], especialmente a secção relativa às normas de limpeza.
\stopSymList


\subsubject{Acessibilidade à documentação do veículo}

\startSymList
\externalfigure [lecteur_1] [width=3.5em]%\PMrtfm
\SymList
Durante a operação, guardar a documentação do veículo sempre na cabina do operador, num lugar onde possa ser rapidamente acedida.
\stopSymList


\section{Condições especiais de utilização}

\subsubject{Altura do veículo}

\startSymList
\PPmaxheight
\SymList
Durante os trabalhos|/|circulação em espaços fechados (garagens, passagens subterrâneas, condutas elétricas, etc.), certificar que a altura é suficiente para a \sdeux\  (ver \in{secção}[sec:measurement], \atpage[sec:measurement]).
\stopSymList


\subsubject{Estabilidade do veículo}

Evitar qualquer manobra que possa prejudicar a estabilidade do veículo. Se durante a circulação nas curvas for mantida uma velocidade elevada, e o recipiente de sujidade estiver cheio, a \sdeux\  poderá tombar, devido à sua construção estreita e ao elevado centro de gravidade.


\subsubject{Deslocação inadvertida do veículo}

Ao abandonar o veículo, tomar medidas adequadas para evitar que o mesmo seja utilizado por pessoas não autorizadas. Acionar o travão de mão antes de abandonar o veículo; event. colocar calços nas rodas.

\startbuffer [prescription:handbrake]
\starttextbackground [CB]
\startPictPar
\PPstop
\PictPar
{\md Acionar corretamente o travão de mão!} Caso contrário, o veículo pode deslocar-se inadvertidamente, mesmo\index{Travão de mão+Potencial de perigo} em terrenos com uma inclinação praticamente impercetível, provocando acidentes e ferimentos mortais em terceiros.

{\lt Através do sistema de acionamento hidrostático, quando o veículo está parado, a pressão no circuito hidráulico é reduzida gradualmente, o que causa a redução da força de retenção do motor. Por este motivo, é muito importante que, ao abandonar o veículo, o travão de mão seja acionado corretamente.}
\stopPictPar
\stoptextbackground

\stopbuffer

\getbuffer [prescription:handbrake]


\testpage [6]
\subsubject{Recipiente de sujidade}

\startbuffer [prescription:container:gravity]
\starttextbackground [CB]
\startPictPar
\PHgravite
\PictPar
{\md Perigo de acidentes:}
{\lt ao rebater o recipiente de sujidade para cima, o centro de gravidade desloca-se para cima. Deste modo, o perigo de tombamento do veículo aumenta. Durante o rebatimento do recipiente de sujidade, prestar atenção para que o veículo esteja pousado sobre um solo plano e resistente.}
\stopPictPar
\stoptextbackground

\stopbuffer

\getbuffer [prescription:container:gravity]


\startbuffer [prescription:container:tilt]
\starttextbackground [CB]
\startPictPar
\PHcrushing
\PictPar
{\md Perigo de acidentes:}
{\lt nunca executar trabalhos por baixo do recipiente de sujidade sem ter previamente colocado as barras de segurança nos cilindros de elevação hidráulicos do recipiente de sujidade.}
\stopPictPar
\stoptextbackground

\stopbuffer

\getbuffer [prescription:container:tilt]


\stopcomponent
