\startcomponent c_45_vpad_s2_120-ro
\product prd_ba_s2_120-ro

\startchapter[title={Computer de bord (Vpad)},
reference={sec:vpad}]

\setups[pagestyle:marginless]


\startsection[title={Descrierea dispozitivului Vpad},
reference={vpad:description}]

\startfigtext [left] {Dispozitivul Vpad SN în postul de comandă}
{\externalfigure[vpad:inside:view]}
\textDescrHead{Inovator, inteligent … } Dispozitivul \Vpad\ a fost conceput pentru comanda agregatelor din domeniul administrației locale, ale cărui tehnologii au devenit tot mai complexe și căruia îi stau la dispoziție o varietate de funcții.
Prin intermediul \Vpad\ utilizatorul are la dispoziție un sistem care nu se limitează la faptul de a oferi în timp real informații - în mod vizual sau acustic - despre toate operațiunile de lucru și acțiunile mașinii.
Însă prin ceea ce se remarcă în primul rând dispozitivul \Vpad\ și capitolul la care acest dispozitiv stabilește un nou standard, este meniul intuitiv, ergonomia și logica de operare.

Datorită multitudinii sale de funcții, \Vpad\ poate fi utilizat într-un mod extrem de flexibil, devenind astfel mai mult decât o simplă unitate electronică de comandă.
\stopfigtext

\textDescrHead{… universal} La crearea \Vpad\, compatibilitatea și flexibilitatea au stat în centrul atenției:
Ca unitate modulară de comandă, dispozitivul poate fi adaptat individual la situații și variante locale de echipare și datorită numeroaselor sale interfețe electronice și soluții de transfer de date – până la WLAN – toate posibilitățile sunt deschise.
Dispozitivul \Vpad\ lucrează cu cele mai moderne sisteme electronice, cu tehnologii 32 bit și cu sistem de operare în timp real.
\vfill


\startfigtext[left]{Consolă multifuncțională}
{\externalfigure[console:topview]}
\textDescrHead{… și modular} Datorită modularității sale, dispozitivul \Vpad\ dispune de un avantaj enorm:
În acest fel, versiunea SN utilizată în serie în \sdeux\ poate fi completată treptat, în orice moment cu alte componente, cum ar fi de exemplu un modem sau o consolă (a se vedea imaginea).
Modularitatea nu se limitează la Hardware, și din punctul de vedere al softului, sistemul poate fi extins și poate fi adaptat la alte necesități schimbătoare.

Consola multifuncțională a \sdeux\ este o interfață deosebit de dezvoltată între utilizator și mașină. Întregul sistem de măturat/aspirație poate fi controlat prin intermediul acestei console.
\stopfigtext

\page [yes]


\subsection[vpad:home]{Ecran principal}

%% Note: outcommented by PB
% \placefig[left][fig:vpad:engineData]{Accueil mode transport}
% {\scale[sx=1.5,sy=1.5]
% {\setups[VpadFramedFigureHome]
% \VpadScreenConfig{
% \VpadFoot{\VpadPictures{vpadClear}{vpadBeacon}{vpadEngine}{vpadSignal}}}
% \framed{\null}}
% }


\start

\setupcombinations[width=\textwidth]

\placefig [here][fig:vpad:engineData]{Ecran principal}
{\startcombination [2*1]
{\setups[VpadFramedFigureHome]% \VpadFramedFigureK pour bande noire
\VpadScreenConfig{
\VpadFoot{\VpadPictures{vpadClear}{vpadBeacon}{vpadEngine}{vpadSignal}}}%
\scale[sx=1.5,sy=1.5]{\framed{\null}}}{\aW{Mod} de mers}
{\setups[VpadFramedFigureWork]% \VpadFramedFigureK pour bande noire
\VpadScreenConfig{
\VpadFoot{\VpadPictures{vpadClear}{vpadBeacon}{vpadEngine}{vpadSignal}}}%
\scale[sx=1.5,sy=1.5]{\framed{\null}}}{\aW{Mod de}lucru}
\stopcombination}

\stop

\blank [1*big]

Ecranul principal al \Vpad\ cuprinde toate elemente necesare pentru monitorizarea tuturor funcțiilor dispozitivului \sdeux.

În zona superioară se află afișaje de control.

Zona de mijloc indică în timp real, printre altele,\, următoarele date:
Viteză, turație și temperatura motorului, nivel carburant, nivel de umplere apă de reciclare, etc.

Modul \aW{Mers} este simbolizat printr-un iepure~\textSymb{transport_mode}, modul \aW{Lucru} printr-o broască țepoasă~\textSymb{working_mode}.

Lista de meniuri din partea de jos indică toate meniurile disponibile: Apăsați în mijlocul ecranului (Touchscreen), pentru a afișa meniurile suplimentare.

\page [yes]

\defineparagraphs[SymVpad][n=2,distance=4mm,rule=off,before={\page[preference]},after={\nobreak\hrule\blank [2*medium]}]
\setupparagraphs [SymVpad][1][width=4em,inner=\hfill]

\start % local group for temporary redefinition of \textDescrHead [TF]
\define[1]\textDescrHead{{\bf#1\fourperemspace}}


\subsection{Afişaje de control pe ecranul Vpad}

% \startcolumns

\startSymVpad
\externalfigure[vpadTEnginOilPressure][height=1.7\lH]
\SymVpad
\textDescrHead{Presiune ulei motor}(roșu) Presiune ulei motor prea mică. Opriți imediat motorul.

+\:Fehlermeldung \# 604
\stopSymVpad

\startSymVpad
\externalfigure[vpadWarningBattery][height=1.7\lH]
\SymVpad
\textDescrHead{Încărcare acumulator}(roșu) Curent încărcare acumulator prea mic. Informați atelierul de reparații.
\stopSymVpad

\startSymVpad
\externalfigure[vpadWarningEngine1][height=1.7\lH]
\SymVpad
\textDescrHead{Diagnosticare motor}(galben) Eroare la sistemul de comandă al motorului. Informați atelierul de reparații.
\stopSymVpad

\startSymVpad
\externalfigure[vpadWarningService][height=1.7\lH]
\SymVpad
\textDescrHead{Mergeţi la cel mai apropiat service}(galben) Inspecţia regulată a autovehiculul este scadentă
(A se vedea \about [sec:schedule] \atpage [sec:schedule])
Sau a fost înregistrată o eroare la motor (este necesar un service de specialitate).

+\:Fehlermeldungen \# 650 până la \# 653, sau \# 703
\stopSymVpad

\startSymVpad
\externalfigure[vpadTDPF][height=1.7\lH]
\SymVpad
\textDescrHead{Filtru particule}(galben) Se începe regenerarea filtrului de particule, imediat ce starea de funcţionare permite acest lucru.

{\md Indicație:} {\lt Dacă este posibil, nu opriţi {\em motorul}, atâta timp cât apare acest afişaj!}
\stopSymVpad

\startSymVpad
\externalfigure[vpadTBrakeError][height=1.7\lH]
\SymVpad
\textDescrHead{Sistem de frânare}(roșu) Eroare la nivelul sistemului de frânare. Informați atelierul de reparații.

+\:Fehlermeldung \# 902
\stopSymVpad


\startSymVpad
\externalfigure[vpadTBrakePark][height=1.7\lH]
\SymVpad
\textDescrHead{Frână de staționare}(roșu) Frâna de staționare este activată.

+\:Fehlermeldung \# 905
\stopSymVpad

\startSymVpad
\externalfigure[vpadTEngineHeating][height=1.7\lH]
\SymVpad
\textDescrHead{Instalație de pre-încălzire}(galben) Motorul este pre-încălzit.

O lampă cu aprindere intermitentă indică că a fost înregistrată o eroare în memoria de rezultate.
\stopSymVpad

% \columnbreak

\startSymVpad
\externalfigure[vpadTFuelReserve][height=1.7\lH]
\SymVpad
\textDescrHead{Nivel de umplere combustibil}(galben) Nivelul de umplere cu combustibil este prea scăzut (rezervă).
\stopSymVpad

\startSymVpad
\externalfigure[vpadTBlink][height=1.7\lH]
\SymVpad
\textDescrHead{Lumini intermitente de avertizare}(verde) Luminile intermitente de avertizare sunt activate.
\stopSymVpad

\startSymVpad
\externalfigure[vpadTLowBeam][height=1.7\lH]
\SymVpad
\textDescrHead{Lumină poziție}(verde) Lumina de poziție este în funcțiune.
\stopSymVpad

\startSymVpad
\HL\NC \externalfigure[vpadSyWaterTemp][height=1.7\lH]
\SymVpad
\textDescrHead{Temperatură}(roșu) Temperatura lichidului hidraulic sau al motorului este prea ridicată. Informați atelierul de reparații.

+\:Fehlermeldung \# 700 sau \# 610
\stopSymVpad

\startSymVpad
\externalfigure[vpadWarningFilter][height=1.7\lH]
\SymVpad
\textDescrHead{Filtru înfundat}(roșu) Filtrul hidraulic combinat sau filtrul de aer este înfundat.

+\:Fehlermeldung \# 702 sau \# 851
\stopSymVpad

\startSymVpad
\externalfigure[vpadTSpray][height=1.7\lH]
\SymVpad
\textDescrHead{Pistol de apă}(galbenă) Pompa de apă de înaltă presiune pentru pistolul de apă este activată.

Întrerupător \textSymb{temoin_buse} în consola de tavan.
\stopSymVpad

\startSymVpad
\externalfigure[vpadTClear][height=1.7\lH]
\SymVpad
\textDescrHead{Mesaj eroare}(roșu) Un mesaj de eroare se află în memoria \Vpad. Apăsați tasta ~\textSymb{vpadClear}, pentru a afișa toate notificările înregistrate. Informați atelierul de reparații.
\stopSymVpad

% \stopcolumns
\stop % local group for temporary redefinition of \textDescrHead

\stopsection

\page [yes]


\startsection [title={Meniuri Vpad},
reference={vpad:menu}]

\start

\setupTABLE [background=color,
frame=off,
option=stretch,textwidth=\makeupwidth]

\setupTABLE [r] [each] [style=sans, background=color, bottomframe=on, framecolor=TableWhite, rulethickness=1.5pt]
\setupTABLE [r] [first][backgroundcolor=TableDark, style=sansbold]
\setupTABLE [r] [odd][backgroundcolor=TableMiddle]
\setupTABLE [r] [even] [backgroundcolor=TableLight]
\bTABLE [split=repeat]
\bTABLEhead
\bTR\bTD Meniu \eTD\bTD Denumire\index{Vpad+Afișaj} \eTD\bTD Funcție \eTD\eTR
\eTABLEhead

\bTABLEbody
\bTR\bTD \externalfigure [v:symbole:clear] \eTD\bTD Mesaj(e) de eroare \eTD\bTD Afișare mesaje de eroare memorate în Vpad și confirmare. \eTD\eTR
\bTR\bTD \framed[frame=off]{\externalfigure [v:symbole:beacon]\externalfigure [v:symbole:beacon:black]} \eTD\bTD Girofar \eTD\bTD Girofar pornit/oprit \eTD\eTR
\bTR\bTD \externalfigure [v:symbole:engine] \eTD\bTD Date în timp real \eTD\bTD Afișare date de funcționare în timp real pentru motor și sistem hidraulic\eTD\eTR
\bTR\bTD \externalfigure [v:symbole:oneTwoThree] \eTD\bTD Contor \eTD\bTD Afișaj contor ore de funcționare: Contor zile, contor anotimpuri, contor total\eTD\eTR
\bTR\bTD \externalfigure [v:symbole:serviceInfo] \eTD\bTD Interval de revizie \eTD\bTD Indică data, precum și orele de funcționare rămase până la următoarea revizie sau până la următorul service \eTD\eTR
\bTR\bTD \externalfigure [v:symbole:trash] \eTD\bTD Contor \eTD\bTD Resetați contorul sau resetați intervalul de service \eTD\eTR
\bTR\bTD \externalfigure [v:symbole:sunglasses] \eTD\bTD Mod ecran \eTD\bTD Comutare iluminat ecran între modul de \aW{zi} și \aW{de noapte} \eTD\eTR
\bTR\bTD \externalfigure [v:symbole:color] \eTD\bTD Luminozitate/contrast \eTD\bTD Setări pentru luminozitatea și contrastul ecranului \eTD\eTR
\bTR\bTD \externalfigure [v:symbole:select] \eTD\bTD Selectare \eTD\bTD Selectarea înregistrării marcate sau confirmarea unui mesaj de eroare \eTD\eTR
\bTR\bTD \externalfigure [v:symbole:return] \eTD\bTD Confirmare \eTD\bTD Confirmarea selectării \eTD\eTR
\bTR\bTD \framed[frame=off]{\externalfigure [v:symbole:up]\externalfigure [v:symbole:down]} \eTD\bTD Sus/jos, \\Pfeile \eTD\bTD Deplasare marcaj în sus/jos sau mărire/scădere valoare selectată \eTD\eTR
\bTR\bTD \externalfigure [v:symbole:rSignal] \eTD\bTD Semnal acustic de avertizare pentru mersul înapoi \eTD\bTD Activare/dezactivare semnal acustic de avertizare pentru mersul înapoi \eTD\eTR
\bTR\bTD \externalfigure [v:symbole:power] \eTD\bTD Deconectaţi ecranul \eTD\bTD Menţineţi butonul apăsat aproximativ 5 secunde, pentru a decupla ecranul Vpad. \eTD\eTR
\bTR\bTD \framed[frame=off]{\externalfigure [v:symbole:frontBrush]\externalfigure [v:symbole:frontBrush:black]}
\eTD\bTD A treia mătură\index{A treia mătură} (opțional) \eTD\bTD Cuplaţi cea de-a treia mătură.
Cea de-a treia mătură poate fi activată în modul descris la pagina \at[sec:using:frontBrush] . \eTD\eTR
\eTABLEbody
\eTABLE
\stop


% \startcolumns

\subsection{Alte simboluri pe ecranul Vpad}


\subsubsubject{Rezervă de apă proaspătă şi de reciclare}


\start % local group for temporary redefinition of \textDescrHead [TF]
\define[1]\textDescrHead{{\bf#1\fourperemspace}}


\startSymVpad
\externalfigure[sym:vpad:water]
\SymVpad
\textDescrHead{Nivel umplere apă proaspătă} Nivel umplere apă proaspătă insuficient (max. 190\,l; în spatele cabinei șoferului).
\stopSymVpad

\startSymVpad
\externalfigure[sym:vpad:rwater:yellow]
\SymVpad
\textDescrHead{Nivel umplere apă de reciclare}(galben) Nivel de umplere apă de reciclare sub schimbătorul de căldură. Nu are loc nici o răcire a fluidului hidraulic și nici o încălzire a sistemului de umezire a canalului de aspirație.
\stopSymVpad

\startSymVpad
\externalfigure[sym:vpad:rwater]
\SymVpad
\textDescrHead{Nivel umplere apă de reciclare} Nivel umplere apă reciclare insuficient (max. 140\,l; sub recipientul pentru gunoi).
\stopSymVpad


\subsubsubject{Sistem de aspirație} % nouveau

{\em Acest simbol este afişat doar atunci când mătura este dezactivată.}

\startSymVpad
\externalfigure[sym:vpad:sucker]
\SymVpad
\textDescrHead{Gură de aspirație} Sistem de aspirație\index{Gură de aspirație} activată:
Gura de aspiraţie este coborâtă, turbina este activată.
\stopSymVpad


\subsubsubject{Mături laterale} % nouveau

{\em Acest simbol este afişat când a treia mătură nu este activată.}

\startSymVpad
\externalfigure[sym:vpad:sideBrush:83]
\SymVpad
\textDescrHead{Mături laterale} Besen\index{Măturare}\index{Mături laterale } activate. Viteza de rotaţie (în \% viteza maximă de rotaţie)[V\low{max}]) este afişată sub simbol, descărcarea actuală a fiecărei mături este afişată deasupra simbolului (\type{ } = poziţie de plutire, 14 = descărcare maximă).

{\md Descărcare:} {\lt Cu cât descărcarea este mai mică, cu atât mai mare este presiunea exercitată de mături pe pardoseală.}
\stopSymVpad


\startSymVpad
\externalfigure[sym:vpad:sideBrush:float:60]
\SymVpad
\textDescrHead{Poziţia de plutire}(verde pe marginea inferioară)
Pentru a decupla descărcarea, menţineţi manetele timp de 2 s apăsate în faţă; mătura se află acum cu întreaga sa greutate pe pământ. Viteza de rotaţie a măturii este la 60\hairspace\% din V\low{max} (exemplu).
\stopSymVpad

\startSymVpad
\externalfigure[sym:vpad:sideBrush]
\SymVpad
\textDescrHead{Măturile laterale } Măturile sunt activate. Sunt scoase din funcţiune şi sunt ridicate.
\stopSymVpad


\subsubsubject{A treia mătură (opțional)} % nouveau

\startSymVpad
\externalfigure[sym:vpad:frontBrush]
\SymVpad
\textDescrHead{A treia mătură} A treia mătură\index{A 3-a. mătură} este activată. Viteza de rotaţie (în \%  din viteza maximă de rotație [V\low{max}]) este afişată sub simbol.
\stopSymVpad


\startSymVpad
\externalfigure[sym:vpad:frontBrush:left]
\SymVpad
\textDescrHead{Poziţia de plutire}(verde pe marginea inferioară)
Pentru a decupla descărcarea, menţineţi manetele timp de 2 s apăsate în faţă; mătura se află acum cu întreaga sa greutate pe pământ. Viteza de rotaţie a măturii este la 70\hairspace\% din V\low{max} (exemplu).

{\md Direcţia de rotaţie::} {\lt În partea superioară este afişată direcţia de rotaţie (săgeată neagră pe fundal galben).}
\stopSymVpad

\stop % local group for temporary redefinition of \textDescrHead

\stopsection


\page [yes]

\startsection[title={Reglarea luminozității ecranului},
reference={sec:vpad:brightness}]

Ecranul \Vpad\ poate fi utilizat în două trepte preconfigurate de luminozitate: Modul \aW{zi}~– \textSymb{vpadSunglasses}, luminozitate normală~– și modul \aW{noapte}~– \textSymb{vpadMoon}, luminozitate redusă.
Cu tasta \textSymb{vpadColor} puteți avea acces la diferiți parametri.

Pentru a modifica treptele preconfigurate de luminozitate, procedați după cum urmează:

\startSteps
\item Apăsați în mijlocul ecranului (Touchscreen), pentru a derula lista de meniuri de pe marginea inferioară a ecranului.
\item Apăsați pe simbolul \textSymb{vpadSunglasses} respectiv \textSymb{vpadMoon}, pentru a selecta modul pe care doriți să îl modificați.
\item Apăsați \textSymb{vpadColor}, pentru a afișa parametri.
\item Cu ajutorul simbolului săgeată, selectați~\textSymb{vpadUp}\textSymb{vpadDown} parametrul pe care doriți să îl modificați și selectați-l cu ~\textSymb{vpadSelect}.
\item Modificați valoarea cu ajutorul simbolului \textSymb{vpadMinus}\textSymb{vpadPlus}. Atenție, nu reduceți prea mult luminozitatea (\VpadOp{162} -255), astfel încât să nu mai puteți vedea nimic pe ecran!
\stopSteps
\blank [1*big]

\start
\setupcombinations[width=\textwidth]
\startcombination [3*1]
{\setups[VpadFramedFigureHome]% \VpadFramedFigureK pour bande noire
\VpadScreenConfig{
\VpadFoot{\VpadPictures{vpadGPS}{vpadTachygraph}{vpadSunglasses}{vpadColor}}}%
\framed{\null}}{Apăsați în mijlocul Touchscreen -ului}
{\setups[VpadFramedFigure]
\VpadScreenConfig{
\VpadFoot{\VpadPictures{vpadReturn}{vpadUp}{vpadDown}{vpadSelect}}}%
\framed{\bTABLE
\bTR\bTD \VpadOp{160} \eTD\eTR
\bTR\bTD [backgroundcolor=black,color=TableWhite] \VpadOp{162}\hfill 15 \eTD\eTR
\bTR\bTD \VpadOp{163}\hfill 180 \eTD\eTR
\bTR\bTD \VpadOp{164}\hfill 55 \eTD\eTR
\bTR\bTD \VpadOp{165}\hfill 3 \eTD\eTR
\eTABLE}}{Selectare cu \textSymb{vpadSelect}}
{\setups[VpadFramedFigure]% \VpadFramedFigureK pour bande noire
\VpadScreenConfig{
\VpadFoot{\VpadPictures{vpadReturn}{vpadMinus}{vpadPlus}{vpadNull}}}%
\framed[backgroundscreen=.9]{\bTABLE
\bTR\bTD \VpadOp{160} \eTD\eTR
\bTR\bTD \VpadOp{162}\hfill -80 \eTD\eTR
\bTR\bTD \VpadOp{163}\hfill 180 \eTD\eTR
\bTR\bTD \VpadOp{164}\hfill 55 \eTD\eTR
\bTR\bTD \VpadOp{165}\hfill 3 \eTD\eTR
\eTABLE}}{Modificare valoare cu \textSymb{vpadMinus}\textSymb{vpadPlus}}
\stopcombination
\stop
\blank [1*big]

\startSteps [continue]
\item Confirmați valoarea cu \textSymb{vpadReturn}.
\item Apăsați încă o dată simbolul \textSymb{vpadReturn}, pentru a vă întoarce la ecranul principal.
\stopSteps

\stopsection

\page [yes]


\startsection[title={Contor ore de funcționare și kilometri},
reference={vpad:compteurs}]

Softul \Vpad\ dispune de trei perioade diferite de măsurare~– \aW{zi}, \aW{sezon}, \aW{total}~–, în care pot funcționa diferite contoare, cum ar fi pentru \aW{Distanța străbătută}, \aW{orele de funcționare} (motor sau perie), \aW{timp de lucru} (per șofer).

Pentru a citi contoarele sau pentru a le reseta, procedați după cum urmează:

\startSteps
\item Apăsați în mijlocul Touchscreen-ului, pentru a derula lista de meniuri.
\item Apăsați încă o dată simbolul \textSymb{vpadOneTwoThree}, pentru a afișa contorul pentru zile.
\item Cu ajutorul simbolurilor înainte/inapoi~\textSymb{vpadBW}\textSymb{vpadFW} puteți comuta la contorul total sau pentru sezoane.
\item Apăsați \textSymb{vpadTrash}, pentru a reseta contorul afișat.
\item Într-o fereastră de dialog sunteți solicitat să confirmați resetarea.
\stopSteps
\blank [1*big]

\start
\setupcombinations[width=\textwidth]
\startcombination [3*1]
{\setups[VpadFramedFigure]% \VpadFramedFigureK pour bande noire
\VpadScreenConfig{
\VpadFoot{\VpadPictures{vpadOneTwoThree}{vpadTachygraph}{vpadSunglasses}{vpadColor}}}%
\framed{\bTABLE
\bTR\bTD \VpadOp{120} \eTD\eTR
\bTR\bTD \VpadOp{123}\hfill 87.4\,h \eTD\eTR
\bTR\bTD \VpadOp{125}\hfill 62.0\,h \eTD\eTR
\bTR\bTD \VpadOp{126}\hfill 240.2\,km \eTD\eTR
\bTR\bTD \VpadOp{124}\hfill 901.9\,km \eTD\eTR
\bTR\bTD \VpadOp{127}\hfill 2,1\,l/h \eTD\eTR
\eTABLE}}{Apăsați simbolul~\textSymb{vpadOneTwoThree}, la final~\textSymb{vpadBW} sau~\textSymb{vpadFW}}
{\setups[VpadFramedFigure]
\VpadScreenConfig{
\VpadFoot{\VpadPictures{vpadReturn}{vpadBW}{vpadFW}{vpadTrash}}}%
\framed{\bTABLE
\bTR\bTD \VpadOp{121} \eTD\eTR
\bTR\bTD \VpadOp{123}\hfill 522.0\,h \eTD\eTR
\bTR\bTD \VpadOp{125}\hfill 662.8\,h \eTD\eTR
\bTR\bTD \VpadOp{126}\hfill 1605.5\,km \eTD\eTR
\bTR\bTD \VpadOp{124}\hfill 2608.4\,km \eTD\eTR
\bTR\bTD \VpadOp{127}\hfill 2,0\,l/h \eTD\eTR
\eTABLE}}{Resetați contorul cu \textSymb{vpadTrash} }
{\setups[VpadFramedFigure]% \VpadFramedFigureK pour bande noire
\VpadScreenConfig{
\VpadFoot{\VpadPictures{vpadReturn}{vpadTrash}{vpadNull}{vpadNull}}}%
\framed{\bTABLE
\bTR\bTD \VpadOp{121} \eTD\eTR
\bTR\bTD \null \eTD\eTR
\bTR\bTD \VpadOp{136} \eTD\eTR
\bTR\bTD \null \eTD\eTR
\bTR\bTD \VpadOp{137} \eTD\eTR
\eTABLE}}{Confirmați cu \textSymb{vpadTrash}}
\stopcombination
\stop
\blank [1*big]

\startSteps [continue]
\item Dacă este necesar, introduceți parola și confirmați apoi resetarea cu ajutorul simbolului \textSymb{vpadTrash}.
\item Apăsați simbolul \textSymb{vpadReturn}, pentru a vă întoarce la ecranul principal.
\stopSteps

\stopsection

\page [yes]

\startsection[title={Intervale de revizie},
reference={vpad:maintenance}]

Planul de revizii pentru \sdeux\ prevede două tipuri de revizii: revizii regulate și service (asigurat de un atelier de reparații, stabilit de comun acord cu \boschung-Kundendienst ).

Pentru a citi contoarele sau pentru a le reseta, procedați după cum urmează:
\startSteps
\item Apăsați în mijlocul Touchscreen-ului, pentru a derula lista de meniuri.
\item Apăsați simbolul \textSymb{vpadServiceInfo}, pentru a afișa contorul pentru zile.
\item Cu ajutorul simbolului săgeată~\textSymb{vpadUp}\textSymb{vpadDown} treceți la intervalul dorit.
\item Apăsați simbolul ~\textSymb{vpadTrash}, pentru a reseta un interval. Introduceți parola cu ajutorul ~\textSymb{vpadPlus}\textSymb{vpadMinus} și confirmați cu~\textSymb{vpadSelect}).
\item Într-o fereastră de dialog sunteți solicitat să confirmați resetarea.
\stopSteps
\blank [1*big]

\start
\setupcombinations[width=\textwidth]
\startcombination [3*1]
{\setups[VpadFramedFigure]% \VpadFramedFigureK pour bande noire
\VpadScreenConfig{
\VpadFoot{\VpadPictures{vpadReturn}{vpadNull}{vpadNull}{vpadTrash}}}%
\framed{\bTABLE
\bTR\bTD[nc=2] \VpadOp{190} \eTD\eTR
\bTR\bTD \VpadOp{191}\eTD\bTD \VpadOp{195}\hfill 600\,h \eTD\eTR % [backgroundcolor=black,color=TableWhite]
\bTR\bTD \VpadOp{192}\eTD\bTD \VpadOp{195}\hfill 600\,h \eTD\eTR
\bTR\bTD \VpadOp{193}\eTD\bTD \VpadOp{195}\hfill 2400\,h \eTD\eTR
\eTABLE}}{Apăsați simbolul ~\textSymb{vpadTrash}, pentru a reseta un interval}
{\setups[VpadFramedFigure]
\VpadScreenConfig{
\VpadFoot{\VpadPictures{vpadReturn}{vpadMinus}{vpadPlus}{vpadSelect}}}%
\framed{\bTABLE
\bTR\bTD \VpadOp{190} \eTD\eTR
\bTR\bTD \hfill 2014-03-31 \eTD\eTR
\bTR\bTD \null \eTD\eTR
\bTR\bTD \null \eTD\eTR
\bTR\bTD \null \eTD\eTR
\bTR\bTD \null \eTD\eTR
\bTR\bTD \VpadOp{002}\hfill 0000 \eTD\eTR
\eTABLE}}{Introduceți parola (codul de cifre)}
{\setups[VpadFramedFigure]% \VpadFramedFigureK pour bande noire
\VpadScreenConfig{
\VpadFoot{\VpadPictures{vpadReturn}{vpadUp}{vpadDown}{vpadSelect}}}%
\framed{\bTABLE
\bTR\bTD \VpadOp{190} \eTD\eTR
\bTR\bTD[backgroundcolor=black,color=TableWhite] \VpadOp{041}\eTD\eTR % [backgroundcolor=black,color=TableWhite]
\bTR\bTD \VpadOp{042} \eTD\eTR
\bTR\bTD \VpadOp{043} \eTD\eTR
\eTABLE}}{Selectați și confirmați cu~\textSymb{vpadSelect}}
\stopcombination
\stop
\blank [1*big]

\startSteps [continue]
\item Confirmați resetarea cu ajutorul simbolului~\textSymb{vpadSelect}.
\item Apăsați simbolul \textSymb{vpadReturn}, pentru a vă întoarce la ecranul principal.
\stopSteps

\stopsection

\page [yes]


\startsection[title={Managementul erorilor prin Vpad},
reference={vpad:error}]


Dispozitivul \Vpad\ afișează erorile\index{Vpad+Mesaje erori}, care sunt diagnosticate de sistemele electronice de comandă și sunt transmise prin CAN-Bus.
Atunci când este înregistrată o eroare minoră, se aprinde simbolul~\textSymb{VpadTClear} (roșu).
Când este vorba de o eroare de o prioritate ridicată, se aprinde un simbol~\textSymb{VpadTClear} și suplimentar se aude un semnal acustic.
Pentru a finaliza alarma, trebuie confirmat mesajul de eroare (ca \aW{luat la cunoștință}).

Pentru a citi și confirma mesajele de eroare, procedați după cum urmează:

\startSteps
\item Apăsați simbolul~\textSymb{vpadClear} de pe ecranul \Vpad.
\item Apăsați simbolul~\textSymb{vpadClear}, pentru a confirma mesajul selectat.
\item În afară de mesajul confirmat apare un simbol \aW{\#}, care marchează mesajul ca fiind \aW{luat la cunoștință}, iar marcajul trece la următorul mesaj (dacă acesta există).
\item După ce toate mesajele au fost confirmate, afișajul revine la ecranul principal.
\stopSteps
\blank [1*big]

\start
\setupcombinations[width=\textwidth]
\startcombination [3*1]
{\setups[VpadFramedFigure]% \VpadFramedFigureK pour bande noire
\VpadScreenConfig{
\VpadFoot{\VpadPictures{vpadReturn}{vpadUp}{vpadDown}{vpadSelect}}}%
\framed{\bTABLE
\bTR\bTD \VpadEr{000} \eTD\eTR
\bTR\bTD [backgroundcolor=black,color=TableWhite] \VpadEr{851a} \eTD\eTR
\bTR\bTD \VpadEr{902} \eTD\eTR
\eTABLE}}{Afișarea mesajelor}
{\setups[VpadFramedFigure]
\VpadScreenConfig{
\VpadFoot{\VpadPictures{vpadReturn}{vpadUp}{vpadDown}{vpadSelect}}}%
\framed{\bTABLE
\bTR\bTD \VpadEr{000} \eTD\eTR
\bTR\bTD [backgroundcolor=black,color=TableWhite] \VpadEr{851} \eTD\eTR
\bTR\bTD \VpadEr{902} \eTD\eTR
\eTABLE}}{Confirmați cu~\textSymb{vpadClear}}
{\setups[VpadFramedFigureHome]% \VpadFramedFigureK pour bande noire
\VpadScreenConfig{
\VpadFoot{\VpadPictures{vpadClear}{vpadBeacon}{vpadBeam}{vpadEngine}}}%
\framed{\null}}{Înapoi la ecranul principal}
\stopcombination
\stop
\blank [1*big]

\startSteps [continue]
\item Pentru a afișa din nou mesajele, apăsați din nou simbolul~\textSymb{vpadClear}. Mesajele de eroare sunt șterse de \Vpad\ doar după ce problema a fost remediată.
\stopSteps


\subsection{Cele mai frecvente mesaje de eroare (cu căutarea defecțiunii)}

\subsubsubject{\VpadEr{604}} % {\#\ 604 Pression huile moteur basse}

+ \textSymb{vpadTEnginOilPressure}~– Opriți imediat motorul. Verificați nivelul de ulei, informați atelierul specializat.


\subsubsubject{\VpadEr{609}} % {\#\ 609 Température eau refroidissement moteur haute}

+ \textSymb{vpadSyWaterTemp}~– Întrerupeți activitatea. Lăsați motorul să meargă fără încărcătură și observați evoluția temperaturii:

Dacă temperatura scade, verificați nivelul de umplere cu lichid de răcire, precum și starea radiatorului.
În cazul în care nivelul de umplere este corect, iar starea radiatorului este bună, pentru diagnosticarea suplimentară mergeți cu atenție la un atelier de reparații.

\subsubsubject{\VpadEr{610}} % {\#\ 610 Température eau refroidissement moteur trop haute}

+ \textSymb{vpadSyWaterTemp}~– Întrerupeți activitatea. Verificați nivelul de umplere cu lichid de răcire și ulei de motor, informați imediat atelierul de reparații.


\subsubsubject{\VpadEr{650}} % {\#\ 650 Se rendre à un garage}

+ \textSymb{vpadWarningService}~– informați imediat atelierul de reparații.
% \VpadEr{651} % {\#\ 651 Moteur en mode urgence}


\subsubsubject{\VpadEr{652}} % {\#\ 652 Inspection véhicule}
% \VpadEr{653} % {\#\ 653 Grand service moteur}

+ \textSymb{vpadWarningService}~– Următoarea revizie regulată este scadentă. Consultați planul de revizii și faceți programare la atelierul de reparații.


\subsubsubject{\VpadEr{700}} % {\#\ 700 Température d'huile hydraulique}

+ \textSymb{vpadSyWaterTemp}~– Întrerupeți activitatea. Lăsați motorul să meargă fără încărcătură și observați evoluția temperaturii:

Dacă temperatura scade, verificați nivelul de umplere cu lichid de răcire, precum și starea radiatorului.
În cazul în care nivelul de umplere este corect, iar starea radiatorului este bună, pentru diagnosticarea suplimentară mergeți cu atenție la un atelier de reparații.


\subsubsubject{\VpadEr{702}} % {\#\ 702 Filtre d'huile hydraulique}

+ \textSymb{vpadWarningFilter}~– Filtrul hidraulic de retur și/sau filtrul de aspirație este înfundat. Înlocuiți imediat elementul filtrant.
% \VpadEr{703} % {\#\ 703 Vidange d'huile hydraulique}


\subsubsubject{\VpadEr{800}} % {\#\ 800 Interrupteur d'urgence actionné}

+ \textSymb{vpadTClear}~– Ați acționat întrerupătorul pentru situațiile de avarie. Opriți sistemul de aprindere și porniți din nou motorul, pentru a șterge eroarea.


\subsubsubject{\VpadEr{801}} % {\#\ 905 Frein à main actionné}

Recipientul de gunoi este ridicat sau nu este complet coborât. Viteza autovehiculului este limitată la 5\,km/h, atâta timp cât recipientul pentru gunoi nu este coborât.

\subsubsubject{\VpadEr{851}} % {\#\ 851 Filtre à air}

+ \textSymb{vpadWarningFilter}~– Filtrul de aer este înfundat. Înlocuiți imediat elementul filtrant.


\subsubsubject{\VpadEr{902}} % {\#\ 902 Pression de freinage}

+ \textSymb{vpadTBrakeError}~– Presiunea de frânare nu este suficientă. Întrerupeți activitatea și informați atelierul de reparații.
% \VpadEr{904} % {\#\ 904 Interrupteur de direction d'avancement}


\subsubsubject{\VpadEr{905}} % {\#\ 905 Frein à main actionné}

+ \textSymb{vpadTBrakePark}~– Frâna de staționare nu este complet eliberată. Viteza autovehiculului este limitată la 5\,km/h, atâta timp cât frâna de staționare este încă activată.


\stopsection

\stopchapter

\stopcomponent
