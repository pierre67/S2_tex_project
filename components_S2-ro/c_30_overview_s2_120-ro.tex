\startcomponent c_30_overview_s2_120-ro
\product prd_ba_s2_120-ro

\chapter{Privire de ansamblu asupra autovehiculului}

\setups [pagestyle:marginless]


\placefig [here] [] {Privire de ansamblu asupra părții stângi a autovehiculului}
{\externalfigure [overview:side:left:ro]}


\page [yes]


\placefig [here] [] {Privire de ansamblu asupra părții drepte a autovehiculului}
{\externalfigure [overview:side:right:ro]}

\page [yes]

\setups [pagestyle:normal]


\section{Generalități}

\placefig[margin][p4_vue_01]{\sdeux\ în momentul transferului}
{%
\startcombination [1*3]
{\externalfigure[overview:vhc:01]}{}
{\externalfigure[overview:vhc:02]}{}
{\externalfigure[overview:vhc:03]}{}
\stopcombination}

Prin intermediul mașinii de măturat \BosFull{sdeux}, Boschung oferă o materializare a întregii experiențe și competențe acumulate de-a lungul a zeci de ani de activitate, în cursul unei colaborări strânse cu clienții și partenerii săi de nădejde.
Cerințele autorităților locale și ale prestatorilor de servicii au cresc enorm în această perioadă, mai ales din punctul de vedere al mobilității și versatilității. Proiectanții de la \sdeux\ au acceptat această provocare, ghidați de necesitățile clienților și impulsionați de propunerile vizionare de optimizare ale serviciului clienți din cadrul companiei Boschung.
Din această sinteză de orientare spre clienți și de implementare consecventă a experienței acumulate a rezultat \sdeux\.


\subsection{Tehnologia inovativă}

Mașina compactă de măturat \BosFull{sdeux} se remarcă în clasa sa mai ales prin greutatea sa redusă (2300\,kg), capacitatea sa ridicată (recipient de gunoi din clasa de 2,0-m\high{3}), dimensiunile sale compacte (lățime 1,15\,m) și punctul de lucru deosebit de ergonomic pentru șoferul autovehiculului.

Datorită construcției sale suple, \sdeux\ poate fi numită o mașină de măturat \quotation{Universală}pentru străzi și trotuare în orașe și sate. Motorul său diesel extrem de puternic coroborat cu mecanismul compact de acționare de tip hidrostatic (hidromotor cu piston radial pe roțile din față)garantează în orice moment o mobilitate ridicată, indiferent de locul de intervenție și de gradul de umplere al recipientului de gunoi.

Pompele hidraulice sunt acționate de un motor diesel de tip \aW{VW 2.0 CDI} potrivit normei Euro-V. Acesta asigură un moment de rotație de 285\,Nm la 1750~rotații și o putere maximă de 75\,kW la 3000~rotații. În acest mod, mașina poate fi utilizată în mod eficient chiar și la turație mică~– și astfel cu o poluare fonică redusă~. \sdeux\ dispune în versiunea în serie de un filtru pentru particulă.


\section{Inovațiile în serviciul clientului}

Direcția articulată a \sdeux\ asigură un cerc de viraj redus și astfel de o mobilitate maximă. Materialele speciale, precum Domex®, și proiectarea autovehiculului bazată în întregime pe CAD asigură o sarcină utilă demnă de luat în seamă de 1200\,kg.

\placefig[margin][overview:cab:frontright]{\sdeux\ Pregătit de lucru}
{\externalfigure[overview:cab:twoleft][width=\Bildwidth]}

Cabina șoferului cu geamuri de jur împrejur dispune de două scaune confortabile, echipate cu centură de siguranță în 3 puncte. Opțional, \sdeux\ poate fi echipată cu instalație de aer condiționat.

Cu viteza sa maximă de 40\,km/h, autovehiculul se integrează fără probleme în circulație. Datorită suspensiilor față și spate deosebit de confortabile, chiar și cel mai dificil drum poate fi traversat confortabil și în condiții de siguranță.

Agregatul de măturat~– montat pe două brațe articulate~– se află în întregime în câmpul vizual al operatorului, iar gura de aspirația este poziționat în fața axei frontale. O mătură frontală oscilantă este disponibilă ca dotare suplimentară.

\page [yes]


\subsection{Cabina șoferului confortabilă și amortizată sonor}

Cabina șoferului\index{Cabina șoferului} a mașinii \sdeux\ dispune de un volan pe dreapta și este concepută pentru două locuri. Aceasta este izolată împotriva zgomotelor și este montată pe un silentbloc, izolat împotriva vibrațiilor.

Ușile și pardoseala sunt din sticlă, asigurând astfel un câmp de vizibilitate cuprinzător. Geamul de protecție împotriva vântului se întindea de-a lungul întregii părți frontale și asigură astfel o vizibilitate asupra activității măturilor.

Scaunul șoferului dispune de o suspensie mecanică sau~– opțional~– pneumatică. Scaunul șoferului și pasagerului sunt montate pe șine de alunecare reglabile.


\subsubsubject{Ergonomie}

\startfigtext[right][overview:joy:sideview]{Consolă de comandă}
{\externalfigure[overview:joy:top]}
Consola multifuncțională, din partea stângă a scaunului șoferului, face ca toate funcțiile elementare să poată fi acționate cu mâna. Cele două mături pot fi comandate independent, cu ajutorul a două joystick-uri, folosind degetul mare și degetul arătător. Mecanismele de acționare pentru mături și pentru mătura frontală (opțiune), pentru turația motorului, a tempomatelor, etc se află de asemenea pe consola multifuncțională.
\stopfigtext

Am La marginea inferioară a câmpului de vizibilitate a șoferului se află un Touchscreen, care indică în timp real toate informațiile importante pentru funcțiile mașinii, fără a afecta vizibilitatea spre exterior.

\placefig[margin][overview:vhc:left]{\sdeux\ în fața unor ruine istorice}
% \placefig[margin][overview:vhc:left]{\sdeux\ sur site historique}
{\externalfigure[overview:vhc:left]}

\page [yes]


\subsubsubject{Post de comandă}

Maneta de\index{Post de comandă} selectare a treptelor de viteză (\quotation{schimbătorul de viteză}) se aflăîn dreapta volanului; există două viteze de mers înainte și una de mers înapoi. În exteriorul manetei de selectare a vitezelor se află un buton pentru comutarea între cele două moduri de lucru \aW{Lucru} și \aW{Mers}. Mașina \sdeux\ nu trebuie oprită pentru realizarea comutării. (Mai multe informații în capitolul \about[sec:using:work], \atpage[sec:using:work].)

\placefig[margin][fig:overview:steeringwheel]{Post de comandă}
{\externalfigure[overview:driver:place]}

În timpul mersului cu spatele, se aprinde monitorul camerei pentru mersul cu spatele și se aude un semnal acustic de avertizare (care poate fi dezactivat prin Vpad).

Maneta multifuncțională de pe partea stângă a volanului cuprinde maneta pentru ștergătoarele de parbriz (două trepte și interval), precum și claxonul luminos și acustic.

În capitolul \about[chap:using] de la \atpage[chap:using] găsiți detalii despre acestea și despre celelalte funcții ale \sdeux.

\page [yes]

\setups[pagestyle:marginless]


\subsection[overview:brushsystem]{Sistem de măturare și de aspirare}

\subsubsubject{Mături}

\startfigtext[left][fig:overview:steeringwheel]{Sistem de măturare și de aspirare}
{\externalfigure[system:brush]}
Măturile\index{Măturatul} sunt poziționate pe capete reglabile, care la rândul lor sunt montate pe brațe articulate. Praful produs în momentul operațiunii de măturare este atenuat prin pulverizarea cu apă: Fiecare mătură este dotată cu două duze, care își extrag apa din rezervorul cu apă proaspătă sau cu apă de reciclare.

Un întrerupător\index{Aspirare} de pe consola multifuncțională activează concomitent măturile și pompa de apă.\footnote{Pentru pompa de apă, a se vedea capitolul \in[chap:using] \about[chap:using], mai ales \about[sec:using:work], \atpage[sec:using:work].}
Pozițiile măturilor, precum și înclinarea lor transversală și longitudinală pot fi controlate direct, cun ajutorul Joystick-ului corespunzător de pe consola multifuncțională.
\stopfigtext

Măturile sunt protejate prin intermediul unul sistem anti-coliziune mecanic și hidraulic.


\subsubsubject{Gura de aspirație}

În poziția de lucru (coborâtă) gura de aspirație atinge 4~role și acoperă în întregime suprafața dintre cele două mături. Datorită poziției sale \quotation{remorcate}, în cazul coliziunii cu diferite obstacole, este protejată permanentă împotriva deteriorărilor mecanice. La mersul cu spatele, gura de aspirație este ridicată automat.

O manșetă groasă, substituibilă din cauciuc asigură închiderea etanșă față de suprafața părții carosabile. O clapetă cu sistem electro-hidraulic de comandă de pe partea frontală a gurii de aspirație permite colectarea deșeurilor grosiere.


\subsubsubject{Recipient de gunoi}

Recipientul de gunoi din aluminiu poate fi înclinat până la 50° și poate fi ridicat până la o înălțime de 1,5\,m (înălțime de rulare). În el se află venind de jos canalul de aspirație, cu un diametru de deschidere de 180\,mm.

Depresiune de admisie este generată de o turbină de înaltă putere, care este montată orizonat, în recipientul de gunoi. Aceasta dispune de o clapetă de revizie pentru curățenie și controlul vizual.

În clapeta de închidere a recipientului de gunoi se află două grilaje de aspirație din inox. Acestea pot fi îndepărtate pentru curățenie, fără a fi necesare nici un fel de instrumente de lucru. Clapeta de închidere poate fi deblocată și deschisă manual.

Cu ajutorul unei clapete care poate fi deschisă manual, fluxul de aer poate fi comutat fără dificultăți între canalul de aspirație și furtunul de aspirație manual (dotare suplimentară).


\subsection{Sistem de umezire}

\subsubsubject{Sistem de apă proaspătă}

Rezervorul din \index{Măturare+Umezire} din fontă se află în poziție verticală în spatele cabinei șoferului. Acesta are o capacitate\index{Rezervor+apă proaspătă} de 190\,l.

O pompă electrică (6,5\,l/min) transportă apa la duza de stropire de deasupra fiecărei mături (inclusiv a treia mătură opțională).


\subsubsubject{Reciclare apă uzată}

Apa uzată curge prin microperforațiile pereților interiori ai recipientului de gunoi, pentru a se scurge apoi prin clapeta de reciclare în rezervorul de reciclare aflat dedesubt. Rezervorul de apă reciclată\index{Rezervor+apă reciclată} are o capacitate de 140\,l.

O pompă hidraulică submersibilă transportă apa spre duzele de stropire, în interiorul gurii și canalului de aspirație.


\testpage [8]
\subsubsubject{Rezervor de apă reciclată}

Rezervorul de apă reciclată dispune de un schimbător de căldură pentru apă sau lichidul hidraulic, care îndeplinește o dublă funcție:

\startitemize[width=35mm,style=\md, command={\setupwhitespace[small]}]
\sym{Funcție în timpul verii} Apa transportă căldura lichidului hidraulic prin convecție în pereții de aluminiu ai rezervorului, de unde aceasta este emanată în aerul ambiental.

\sym{Funcți în timpul iernii} Lichidul hidraulic încălzește apa din rezervor. În acest fel, este posibilă pulverizarea cu apă a canalului și gurii de aspirații, chiar și la temperaturi sub punctul de înghețare.
\stopitemize


\subsubsubject{Supravegherea nivelului de umplere cu apă}

\startitemize[width=35mm,style=\md, command={\setupwhitespace[small]}]
\sym{Apa proaspătă} În cazul unui nivel de umplere insuficient apare simbolul ~\textSymb{vpad_water} pe display-ul Vpad-ului.
\sym{Apa de reciclare} Atunci când nivelul apei de reciclare se află sub cel al schimbătorului de căldură (vezi mai sus), apare simbolul~\textSymb{vpad_rwater_orange} (galben) pe display-ul Vpad-ului. În cazul unui nivel de umplere insuficient, apare simbolul~\textSymb{vpad_rwater} (roșu).
\stopitemize


\subsubsubject{Anvelope late (opțional)}

Presiunea aplicată pe teren\index{Anvelope late} corespunde presiunii din anvelope. Cu presiune de 1,8\,bar se atinge o presiune aplicată pe teren de 18\,N/cm². Cu toate acestea, sarcina portantă a anvelopei pentru sarcina pe axe garantată nu mai este atinsă. Cu 1,8\,bar, la 40\,km/h poate fi garantată o sarcină pe osii de doar 1495\,kg. În cazul în care presiunea din anvelope este selectată altfel decât 3.0\,bar, responsabilitatea revine proprietarului autovehiculului.

\subsubsubject{Afișaj suprasarcină (opțional)}

În cazul în care autovehiculul este supraîncărcat\index{Afișaj suprasarcină}, pe Vpad apare un mesaj. Supraîncărcare este stabilită cu ajutorul unui senzor unghiular poziționat pe axa din spate. Din fabrică, afișajul de supraîncărcare este reglat la 3500\,kg, trebuie însă evitat un câmp de toleranță al acestei valori. Această setare de 3500\,kg poate fi modificată de o firmă de specialitate.

\page [yes]
\setups[pagestyle:normal]


\section{Identificarea autovehiculului}

\subsection{Plăcuță de identificare a autovehiculului}

Plăcuța de identificare a autovehiculului\index{Identificarea +autovehiculului} se află în cabina șoferului, în fața consolei, sub scaunul pasagerului(a se vedea \inF[fig:identity:location], \atpage[fig:identity:location]).


\subsection{Cod și serie motor}

Codul motorului se află pe plăcuța de identificare a motorului (etichetă), pe conducta metalică a circuitului de răcire, în fața pe motor (ridicați recipientul de gunoi).

Seria motorului este gravată pe motor (\inF[identity:engine:number]). Aceasta constă din nouă date alfanumerice: Primele trei reprezintă codul motorului, iar celelalte șase seria motorului.


\placefig[margin][idvhc]{Plăcuța de identificare a autovehiculului}
{\externalfigure[s2:id:plaque]}

\placefig[margin][identity:engine:code]{Plăcuța de identificare a motorului}
{\externalfigure[engine:id:code]}

\placefig[margin][identity:engine:number]{Serie motor}
{\externalfigure[engine:id:number]}

\page [yes]


\subsection [sec:plateWheel]{Plăcuța de identificare a roților}

Plăcuța de identificare a jantelor și anvelopelor\index{Anvelope+Presiune} se află în cabina șoferului\index{Jante+Dimensiuni}, sub scaunul pasagerului.


\subsection{Serie de identificare șasiu}

Seria de identificare șasiu\index{Identificare+Serie identificare șasiu} (serie șasiu) este imprimată în partea dreaptă a autovehiculului, sub cabina șoferului, pe șasiu.


\subsection{\symbol[europe][CEsign]conformitate și marcaj}

Marcajul comunitar de conformitate~\symbol[europe][CEsign] se află în cabina șoferului, în fața consolei, sub scaunul pasagerului.

 \sdeux\ corespunde cerințelor de bază privind siguranța și sănătatea, prevăzută de Directiva pentru mașini\index{Certificat+Conformitate CE}\index{Directiva pentru mașini} 2006/42/CE\footnote{Directiva 2006/42/CE a Parlamentului European și a Consiliului din 17.~Mai 2006}.
% \textrule

\placefig[margin][idpneus]{Presiune anvelope}
{\externalfigure[identity:tires]}

\placefig[margin][fig:identity:location]{Plăcuțe de identificare}
{\externalfigure[identity:location]}

\page [yes]
\setups [pagestyle:marginless]


\startsection[title={Date tehnice},
reference={donnees_techniques}]

\subsection [sec:measurement] {Dimensiuni autovehicul}

\placefig[here][fig:measurement]{\select{caption}{Lățime~– Mături în poziție de repaus sau în funcțiune~–, Lungimea și înălțimea autovehiculului}{Dimensiuni autovehicul}}
{\Framed{\externalfigure[s2:measurement]}}

\page [yes]

\placefig[here][fig:measurement]{\select{caption}{Înălțimea autovehiculului cu recipientul pentru mizerie ridicat}{Înălțimea autovehiculului}}
{\Framed{\externalfigure[s2:measurement:02]}}

\page [yes]

\starttabulate [|lBw(45mm)|p|l|rw(35mm)|]
\FL
\NC Grupa\index{Dimensiuni} \NC \bf Dimensiune \NC \bf Unitate\NC \bf Valoare \NC\NR
\ML
\NC Dimensiuni autovehicul \NC Lungime (peste tot) \NC \unite{mm} \NC 4588,00 \NC\NR
\NC\NC Lungime cu 3.\,mături \NC \unite{mm} \NC 5020,00 \NC\NR
\NC\NC Lățimea autovehiculului\NC \unite{mm} \NC 1150,00 \NC\NR
\NC\NC Lățimea autovehiculului (peste tot) \NC \unite{mm} \NC 1575,00 \NC\NR
\NC\NC Înălțime fără girofar \NC \unite{mm} \NC 1990,00 \NC\NR
\NC\NC Ampatament \NC \unite{mm} \NC 1740,00 \NC\NR
\NC\NC Ecartament \NC \unite{mm} \NC 894,00 \NC\NR
\ML
\NC Lățime măturat \NC Mături standard \NC \unite{mm} \NC 2300,00 \NC\NR
\NC\NC Cu 3.\,mături \NC \unite{mm} \NC 2600,00 \NC\NR
\NC\NC Diametru mături \NC \unite{mm} \NC 800,00 \NC\NR
\NC\NC Lățime gură de aspirație \NC \unite{mm} \NC 800,00 \NC\NR
\ML
\NC Distribuție sarcină \NC Greutate proprie\note[weight:empty] Axă frontală \NC \unite{kg} \NC cca. 1100,00 \NC\NR
\NC\NC Greutate proprie\note[weight:empty] Axă posterioară \NC \unite{kg} \NC ca. 1200,00 \NC\NR
\NC\NC Greutate proprie\note[weight:empty] \NC \unite{kg} \NC cca. 2300,00 \NC\NR
\NC\NC Greutate maximă admisă \NC \unite{kg} \NC 3500,00 \NC\NR
\LL
\stoptabulate


\subsection{Rază ecartament și rază de virare}

\starttabulate [|lBw(45mm)|p|l|rw(35mm)|]
\FL
\NC Dimensiunde\index{Dimensiune} \NC \bf Dimensiune \NC \bf Unitate \NC \bf Valoare \NC\NR
\ML
\NC Rază ecartament\index{Rază ecratament}\index{Dimensiune+Rază ecartament} \NC Rază minimă de întoarcere cu mătură \NC \unite{mm} \NC 3325,00 \NC\NR
\ML
\NC Rază de virare \NC exterior \NC \unite{mm} \NC 3425,00 până la 3850,00 \NC\NR
\NC\NC interior \NC \unite{mm} \NC 2025,00 până la 1675,00 \NC\NR
\LL
\stoptabulate

%% TODO en/de/fr: Footnote on preceeding page
\footnotetext[weight:empty]{Configurație standard, cu șofer (cca. 75\,kg).}

\placefig[here][pict:steerin_sweeping:radius]{Cerc de bracaj/Cerc de viraj și rază de viraj}
{\externalfigure[steerin_sweeping:radius]}

\page [yes]


\subsection{Roți și anvelope}

{\sla Dimensiuni standard}

\starttabulate[|lBw(45mm)|p|rw(55mm)|]
\FL
\NC Componente \NC \bf Echipare \NC \bf Valoare \NC\NR
\ML
\NC Anvelope \NC dimensiuni standard \NC 205/70 R 15 C \NC\NR
\ML
\NC Jante \NC Dimensiuni standard \NC 6J\;×\;15 H2 ET 60 \NC\NR
\ML
\NC Presiune anvelope\index{Presiune anvelope} \NC Standard, față/spate \NC 4,5/5,8\,bar \NC\NR
\LL
\stoptabulate

{\sla Anvelope late}

\starttabulate[|lBw(45mm)|p|rw(55mm)|]
\FL
\NC Componente \NC \bf Echipare \NC \bf Valoare \NC\NR
\ML
\NC Anvelope\index{Anvelope late} \NC Anvelope late \NC 275/60 R 15 107H \NC\NR
\ML
\NC Jante \NC Anvelope late \NC 8LB\;×\;15 ET 30 \NC\NR
\ML
\NC Presiune anvelope\index{Presiune anvelope} \NC Standard, față/spate \NC 3,0/3,0\,bar \NC\NR
\LL
\stoptabulate


\subsection{Motor diesel}

\starttabulate [|lBw(45mm)|l|rp|]
\FL
\NC \bf Grupă\index{Motor diesel+Identificare} \NC \bf Parametri \NC \bf valoare\NC\NR
\ML
\NC Tip motor \NC \NC VW CJDA TDI 2.0 – 475 NE \NC\NR
\NC Generalități \NC Ciclul de funcționare \NC În 4 timpi \NC\NR
\NC\NC Număr cilindri \unite{n} \NC 4 \NC\NR
\NC\NC orificiu x cursă \unite{mm} \NC 81\;×\;95,5 \NC\NR
\NC\NC Cilindree totală \unite{cm\high{3}} \NC 1968 \NC\NR
\NC\NC Ventil per cilindru \NC 4 \NC\NR
\NC\NC Ordinea comandării ventilelor \NC 1-3-4-2 \NC\NR
\NC\NC Cea mai scăzută turație în gol \unite{min\high{−1}} \NC 830 +50/−25 \NC\NR
\NC Putere/moment de rotație \NC Turație maximă \unite{min\high{−1}} \NC 3400 \NC\NR
\NC\NC Putere maximă \unite{kW} la \unite{min\high{−1}} \NC 75 la 3000 \NC\NR
\NC\NC Moment maxim de rotație \unite{Nm} la \unite{min\high{−1}} \NC 285 bei 1750 \NC\NR
\NC Consum specific\index{Motor diesel+consum} \NC Combustibil \unite{g/kWh} \NC 224 (la putere maximă) \NC\NR
\NC\NC Ulei \unite{g/kWh} \NC 0,22 \NC\NR
\NC Instalație de carburant \NC Sistem de injecție \NC Injecție directă \quote{Common Rail} \NC\NR
\NC\NC Alimentare carburant \NC Pompă cu roată dințată \NC\NR
\NC\NC Încărcare \NC Da \NC\NR
\NC\NC Răcire aer de alimentare \NC Da \NC\NR
\NC\NC Presiune de încărcare \unite{mbar} \NC 1300\NC\NR
\NC Circuit de gresare\index{Motor diesel+Gresare} \NC Tip \NC Ungere forțată cu schimbător de apă/ulei \NC\NR
\NC\NC Alimentare \NC Pompă rotor \NC\NR
\NC\NC Consum ulei \unite{Litri/20\,h} \NC <\:0,1 \NC\NR
\NC Circuit de răcire\index{Motor diesel+Răcire} \NC Capacitate totală \unite{l} \NC cca. 12 \NC\NR
\NC\NC Presiune etalon hdruck Vas de expansiune \unite{bar} \NC 1,4 \NC\NR
\NC\NC Termostat (orificiu) \unite{°C} \NC 87 \NC\NR
\NC\NC Termostat (plin) \unite{°C} \NC 102 \NC\NR
\NC Gaze de eșapament \NC Filtru de particule \NC Da \NC\NR
\NC\NC Tratament gaze de eșapament \NC Da \NC\NR
\NC\NC Standard \NC Euro 5 \NC\NR
\LL
\stoptabulate


\subsection{Capacitate de trafic}

\starttabulate[|lBw(45mm)|p|l|rw(35mm)|]
\FL
\NC Capacitate de trafic\index{Capacitate de trafic} \NC \bf Configurație \NC \bf Unitate \NC \bf Valoare \NC\NR
\ML
\NC Viteză \NC \aW{Mod de}lucru \NC \unite{km/h} \NC 0 până la 18 (fără trepte) \NC\NR
\NC\NC \aW{Mod de}conducere \NC \unite{km/h} \NC 0 până 40 \NC\NR
\ML
\NC Limitarea vitezei \NC Reglabil \NC \unite{km/h} \NC 0 până la 25 \NC\NR
\LL
\stoptabulate


\subsection{Instalație electrică}

{\starttabulate [|lw(65mm)|p|rw(30mm)|]
\FL
\NC \bf Grupă \NC \bf Componente \NC \bf Valoare \NC\NR
\ML
\NC Baterie \NC Acumulator cu plumb \NC 12\,V 75\,Ah \NC\NR
\NC Alimentare cu curent \NC Alternator \NC 14,8\,V 140\,A \NC\NR
\NC Declanșator \NC Putere \NC 1,8\,kW \NC\NR
\NC Echipament audio \NC Conexiune radio\index{Conexiune radio} și difuzor\index{Difuzor} \NC dotare în serie \NC\NR
% \NC Sécurité et surveillance \NC Tachygraphe\index{tachygraphe} \NC En option \NC\NR
% \NC\NC Enregistreur de fin de parcours\index{fin de parcours} \NC En option \NC\NR
\NC Sistem de iluminat/semnalizare față \NC lumină de poziție \NC 12\,V 5\,W \NC\NR
\NC\NC Lumină fază scurtă \NC H7, 12\,V 55\,W \NC\NR
\NC\NC Far de lucru \NC G886, 12\,V 55\,W \NC\NR
\NC\NC Lumină intermitentă \NC 12\,V 21\,W \NC\NR
\NC Sisteme de iluminat/semnalizare spate \NC Lumini de stop combinate \NC 12\,V 5/21\,W \NC\NR
\NC\NC Lumină intermitentă \NC 12\,V 21\,W \NC\NR
\NC\NC Lumini de mers cu spatele \NC 12\,V 21\,W \NC\NR
\NC\NC Lumini pentru numărul de înmatriculare \NC 12\,V 5\,W \NC\NR
\NC Lumină suplimentară \NC Girofar \NC H1, 12\,V 55\,W \NC\NR
\LL
\stoptabulate
}
\stopsection

\stopcomponent

