\startcomponent c_10_safety_s2_120-ro
\product prd_ba_s2_120-ro

\marking[chapter]{Marcaje de securitate}


\chapter{Marcaje de securitate}

\setups[pagestyle:marginless]

\section{Noua procedură europeană de etichetare a substanțelor periculoase}

{\em Rombic cu fundal alb și margini de culoare roșie.}\par\blank[1*medium]
{\em Din 2008, la nivelul UE se aplică așa numitul Regulament CLP \index{Regulamentul CLP} cu noul sistem de etichetare a substanțelor și produselor periculoase.}\par\null

\startSymList \GHSgeneric
\SymList
\textDescrHead{Risc la adresa sănătății }
Avertizează cu privire la \index{Riscurile la adresa sănătății} posibilele riscuri la adresa sănătății, care nu pot duce la deces sau la afecțiuni grave de sănătate. Aici se înscriu iritarea pielii sau producerea unei alergii. Simbolul este utilizat și ca avertisment pentru alte posibile pericole, precum inflamabilitatea.\par
Înlocuiește:\crlf \HAZOcross\ sau \HAZOpoison\ sau \PHgeneric
\stopSymList

\startSymList \GHSbody
\SymList
\textDescrHead{Riscuri mari la adresa sănătății; în cazul copiilor, poate duce chiar și la deces} Produsele pot aduce grave daune sănătății. Acest simbol avertizează și cu privire la riscurile la adresa sănătății\index{Riscuri+ femeile însărcinate} pentru femeile însărcinate, cu privire la efectele cancerigene\index{Pericol+ substanțe cu efecte cancerigene} și la alte riscuri mari la adresa sănătății. Produsele trebuie folosite cu precauție.\par
Înlocuiește:\crlf \HAZOcross\ sau \HAZOpoison\
\stopSymList

\startSymList \GHSbomb
\SymList
\textDescrHead{Explozivi}
În momentul reacției, substanțele, amestecurile și produsele explozive \index{Pericol+ de explozie} instabile cu explozivi\index{Produse explozive} au o acțiune expansibilă, care poate duce la distrugeri considerabile; în cazul utilizării inadecvate, există risc de deces.\par
Înlocuiește:\crlf \HAZObomb\
\stopSymList


\startSymList \GHSpoison
\SymList
\textDescrHead{Intoxicații}
Prin contactul cu pielea, prin inspirare\index{Pericol +intoxicare} sau prin înghițire chiar și în cantitate mică, \index{ produsele toxice} pot duce la intoxicații grave sau chiar letale. Nu permiteți contactul direct.\par
Înlocuiește:\crlf \HAZOpoison\
\stopSymList

\startSymList \GHSfire
\SymList
\textDescrHead{Produsele cu un grad scăzut sau ridicat de inflamabilitate}
Produsele\index{ Pericol +de incendiu} se aprind repede în apropierea căldurii sau a flăcărilor. Spray-urile cu acest marcaj nu trebuie pulverizate pe suprafețele încinse sau în apropierea flăcărilor deschise.\par
Înlocuiește:\crlf \HAZOfire\ sau \HAZOfirebis\
\stopSymList

\startSymList \GHSenvironment
\SymList
\textDescrHead{Pericol pentru animale și mediu}
Produsele\index{ Protejarea mediului} pot produce daunele pe termen scurt sau lung \index{Substanțe toxice}. Acestea pot ucide organismele care trăiesc în apă (\eG\ peștii) sau pe termen lung pot afecta mediul. Nu aruncați astfel de produse în canalizare sau în gunoiul menajer!\par
Înlocuiește:\crlf \HAZOenvironment\
\stopSymList

\startSymList \GHScorrosive
\SymList
\textDescrHead{Riscuri pentru piele sau ochi}
Chiar și în urma unui contact de scurtă durată, produsele \index{Pericol+de vătămare la nivelul pielii}\index{Pericol+de vătămare a ochilor} pot produce daune la nivelul pielii și pot cauza apariția unor leziuni sau pot afecta ochii pe termen lung. În momentul utilizării, protejați-vă pielea și ochii!\par
Înlocuiește:\crlf \HAZOcross\ sau \HAZOcorrosive
\stopSymList

\testpage [5]


\section{Avertismente}

{\em Text negru pe fundal galben}\par\null

\startSymList \PHgeneric
\SymList
\textDescrHead{Avertisment general}
Indică\index{Pericol+avertisment}\index{general} un pericol iminent, în urma căruia dumneavoastră sau alte persoane v-ați putea vătăma.
\crlf\null
\stopSymList

\startSymList \PHpoison
\SymList
\textDescrHead{Avertisment împotriva substanțelor toxice}
În urma contactului cu pielea, a inspirării sau înghițirii, substanțele toxice\index{Pericol+intoxicare} pot cauza afecțiuni acute sau cronice considerabile sau chiar decesul.
\stopSymList

\startSymList \PHfire
\SymList
\textDescrHead{Avertisment împotriva produselor inflamabile}
Evitați flăcările deschise și formarea de scântei \index{Pericol +de incendiu}. Produsele sunt ușor inflamabile sau poate contribui la propagarea incendiului. Fumatul interzis!
\stopSymList

\startSymList \PHexplosive
\SymList
\textDescrHead{Avertisment împotriva produselor explozibile}
Substanțele sau amestecurile solide, lichide sau gelatinoase, care în urma izbiturilor, a frecării, acțiunii focului, căldurii sau în alte situații similare pot exploda.\index{Pericol+de explozie} Fumatul interzis!
\stopSymList

\startSymList \PHcrushing
\SymList
\textDescrHead{Avertisment privind pericolul de strivire}
Indică o zonă\index{Pericol+de strivire} în care, din cauza componentelor mecanice mobile, există pericol de strivire. Stați la distanță de astfel de zonă, atâta timp cât dispozitivul se află în mișcare.
\stopSymList

\startSymList \PHhand
\SymList
\textDescrHead{Avertisment privind pericolul de vătămare a mâinilor}
Există pericolul ca mâinile\index{Pericol+de strivire} sau alte părți ale corpului\index{Pericol+de rănire a mâinilor} să fie strivite, \eG\ în timpul înclinării cabinei șoferului sau a rampei de acces.
\stopSymList

\startSymList \PHentangle
\SymList
\textDescrHead{Avertisment privind rotirea în sens opus / privind pericolul de tragere }
Există pericolul ca membrele superioare\index{Pericol+de tragere} să fie prinse de părțile rotative și să fie trase. Stați la distanță, atâta timp cât dispozitivul se află în mișcare.
\stopSymList

\startSymList \PHcorrosive
\SymList
\textDescrHead{Avertisment împotriva produselor iritante}
Lucrați cu atenție\index{Pericol+substanțe iritante}, purtați echipament de protecție adecvat (mănuși, ochelari de protecție, uniformă de protecție).
\stopSymList

\startSymList \PHhot
\SymList
\textDescrHead{Avertisment cu privire la suprafețele fierbinți}
Nu vă apropiați de componente sau de dispozitivul de lucru\index{Pericol+ardere} fără a avea cunoștințe suficiente; purtați mănuși.
\stopSymList

\startSymList \PHvoltage
\SymList
\textDescrHead{Avertisment cu privire la tensiunea electrică periculoasă}
Nu atingeți cu obiecte metalice\index{Pericol+tensiune electrică}.
Pericol de vătămare sau ardere în cazul unui scurt circuit!
\stopSymList

\startSymList \PHfalling
\SymList
\textDescrHead{Avertisment privind pericolul de prăbușire}
În această zonă trebuie să fiți deosebit de atent\index{Pericol+de prăbușire}, purtați încălțăminte adecvată de lucru (cu talpă anti-derapantă, rezistentă la hidrocarburi).
\stopSymList

\startSymList \PHbattery
\SymList
\textDescrHead{Avertisment privind pericolul reprezentat de baterii} Atrage atenția asupra pericolelor care pot apărea la încărcarea bateriilor (baterii cu plumb acid)\index{Pericol+baterii}, mai ales din cauza hidrogenului gazos eliberat și din cauza acidului sulfuric din baterii.
\stopSymList

\startSymList \PHremote
\SymList
\textDescrHead{Avertisment privind pornirea automată}
Avertizează privind pericolul reprezentat de\index{Pericol+pornire automată} posibila pornire automată sau telecomandată a unui echipament.
\stopSymList

% \startSymList \PHquetschgefahr
% \SymList
% \textDescrHead{Risque d’écrasement}
% Risque d’écrasement\index{risque d’écrasement}.
% \stopSymList
% % NOTE: Doppelt! (auch Bilddatei)
%
% % NOTE: Evtl. Folgendes als Ersatz für oben?

% \startSymList\PHhandcrushed
% \SymList
% \textDescrHead{Gefahr von Handquetschungen}
% Es besteht\index{Gefahr+Quetschung} die Gefahr, dass Hände oder Finger
% gequetscht werden. Nähern Sie die Hände nicht an, ohne die Gefahr
% identifiziert und beseitigt zu haben.
% \stopSymList

\startSymList \PHhandfoot
\SymList
\textDescrHead{Avertisment privind componentele mobile}
Atrage atenția cu privire la componentele utilajelor/autovehiculelor aflate în mișcare.
\index{Pericol+componente în mișcare}.
\stopSymList

\startSymList \PHnarrowed
\SymList
\textDescrHead{Pericol privind calea de rulare îngustată}
Cale de rulare\index{Pericol+lățimea autovehiculului} îngustată.
% Denken Sie an die Breite des Fahrzeugs.
\stopSymList

\testpage [5]


\section{Semn de interzicere}

{\em Rotund cu fundal alb, margine roșie și o line transversală}
\par\null


\startSymList \PPfire
\SymList
\textDescrHead{Focul, mijloacele de iluminat cu flacără deschisă și fumatul sunt interzise} Flăcările deschise\index{Fumatul+și focul interzise} și jăraticul sub orice formă sunt interzise(\eG\ țigări,chibriturile, lumânările; și generarea de scântei de orice tip).
\stopSymList

\startSymList \PPentry
\SymList
\textDescrHead{Interzis accesul persoanelor neautorizate}
Persoanele\index{Accesul+interzis} neautorizate nu pot intra în această zonă și nici nu se pot apropia de aceasta.
\stopSymList

\startSymList \PPphone
\SymList
\textDescrHead{Interzisă utilizarea telefonului mobil}
Telefoanele mobile\index{Telefoane mobile+interzise} și orice alte dispozitive, care emit radiații electromagnetice, trebuie scoase din funcțiune. Radiațiile electromagnetice pot cauza disfuncționalități ale sistemului electronic al echipamentului.
\stopSymList

\startSymList \PPspray
\SymList
\textDescrHead{Pulverizarea cu apă interzisă}
Nu îndreptați niciodată jetul de apă sau vapori\index{Jetul de apă sau vapori+interzis} în direcția componentelor sau echipamentelor sensibile (\eG\ senzori, elemente de comandă, sistemul de injecție, etc.).
\stopSymList

\startSymList \PPchildren
\SymList
\textDescrHead{Țineți copiii la distanță}
Indicație\index{Accesul copiilor+interzis} privind un pericol deosebit pentru copii. În general se aplică următorul principiu: Copiilor nu le este permis accesul în apropierea unui utilaj aflat în funcțiune, nici măcar în timpul lucrărilor de revizie.
\stopSymList

\startSymList \PPwater
\SymList
\textDescrHead{Nu este apă potabilă}
Nu beți apa din rezervor\index{Interzis+Nu este apă potabilă}. Posibil pericol de intoxicare.
\stopSymList

% \page [yes]


\section{Certificări pentru mediu}

\startSymList \PSrecycle
\SymList
\textDescrHead{Reciclare}
Prevederi specifice pentru evacuarea regulamentară a anumitor deșeuri.
\stopSymList

\startSymList \PSwelt
\SymList
\textDescrHead{Protejarea mediului înconjurător}
Mențiune privind normele în vigoare privind protecția mediului.
\stopSymList

\startSymList \PStrash[width=\PictoHeight,height=,]
\SymList
\textDescrHead{Deșeurile trebuie eliminate în conformitate cu normele în vigoare}
Pentru anumite deșeuri, \eG\ bateriile cu plumb, sunt valabile prevederi speciale privind eliminarea.
\stopSymList


\testpage[12]


\section{Simboluri grafice}


{\em Rotunde cu fundal albastru}\par\null

\startSymList \PMgeneric
\SymList
\textDescrHead{Simboluri grafice generale}
Aceste simboluri pot fi folosite doar împreună cu un alt simbol suplimentar, care precizează recomandarea corespunzătoare.
\stopSymList


\startSymList \PMrtfm
\SymList
\textDescrHead{Respectați instrucțiunile de utilizare}
Înainte de punerea în funcțiune, se recomandă citirea instrucțiunilor de utilizare pe această temă\index{Citiți instrucțiunile de utilizare}, despre un aparat sau un produs. Instrucțiunile de utilizare trebuie păstrate la îndemână, în cabina șoferului.
\stopSymList

\startSymList \PMproteyes
\SymList
\textDescrHead{Utilizați echipamentul de protecție pentru ochi}
În cazul lucrărilor la care există pericol de vătămare la nivelul ochilor, trebuie purtat echipament de protecție corespunzător\index{Echipament de protecție pentru ochi}.
\stopSymList

\startSymList \PMprothands
\SymList
\textDescrHead{Utilizați echipamentul de protecție pentru mâini}
În cazul lucrărilor la care există pericol de vătămare la nivelul mâinilor, trebuie să purtați mănuși de protecție\index{Utilizați echipament de protecție pentru mâini}.
\stopSymList

\startSymList \PMprotears
\SymList
\textDescrHead{Utilizați echipament pentru protejarea auzului}
Echipamentul pentru protejarea auzului\index{Pericol+pentru auz} este obligatorie (\eG în apropierea unui ventilator sau turbine aflate în funcțiune).
\stopSymList

\startSymList \PMsafetybelt
\SymList
\textDescrHead{Utilizați centura de siguranță} Pentru siguranța dumneavoastră\index{Centura de siguranță} fixați centura de siguranță.
\stopSymList

\section{Indicatoare suplimentare}

% \adaptlayout[height=+5mm]{{{

% \startSymList \SETshoe
% \SymList
% \textDescrHead{Port de chaussures de sécurité obligatoire}
% Le port de chaussures de sécurité est obligatoire\index{chaussures de sécurité}.
% \stopSymList
%
% \startSymList \SETglasses
% \SymList
% \textDescrHead{Port de lunettes des protection obligatoire}
% Le port de lunettes est obligatoire\index{lunette de protection}.
% \stopSymList
%
% \startSymList \SEToreillettes
% \SymList
% \textDescrHead{Port de casque obligatoire}
% Le port d’un casque de protection est \index{casque} obligatoire.
% \stopSymList
%
% \startSymList \SETgloves
% \SymList
% \textDescrHead{Port de gants de protection obligatoire}
% Le port de gants de protection est obligatoire\index{gants}.
% \stopSymList
%
% \startSymList \SETmainecrase
% \SymList
% \textDescrHead{Risque d’écrasement}
% Danger pour les mains\index{écrasement} et les pieds.
% \stopSymList
%
% \startSymList \SETgetriebe
% \SymList
% \textDescrHead{Risque de happement}
% Risque de happement par\index{happement} des pièces en rotation.
% \stopSymList
%
% \startSymList \SETradkeil
% \SymList
% \textDescrHead{Cale de roue}
% Sécuriser le véhicule contre toute mise\index{Cale de roue} en marche involontaire.
% \stopSymList
%}}}

\startSymList \SETfirstaid
\SymList
\textDescrHead{Primul ajutor}
Indică locul de păstrare al echipamentului de prim ajutor. Informarea rapidă a serviciului de salvare reprezintă o componentă de bază a primului ajutor.\index{Telefon de urgență}\index{primul ajutor} Înregistrați aici numărul dumneavoastră pentru cazurile de urgență:
\fillinrules[n=1]{\bf
\framed[align=right,frame=off,offset=none,width=30mm]{Serviciu de salvare}}
\fillinrules[n=1]{\bf
\framed[align=right,frame=off,offset=none,width=30mm]{Poliția}}
\fillinrules[n=1]{\bf
\framed[align=right,frame=off,offset=none,width=30mm]{Pompieri}}
\stopSymList

\startSymList \SETbrandschutzzeichen
\SymList
\textDescrHead{Extinctoare}
Anumite echipamente sunt dotate cu unul sau mai multe extinctoare\index{Extinctoare}. De regulă, acestea necesită o întreținere specială; mai multe informații în acest sens găsiți pe aparat sau în instrucțiunile de utilizare ale echipamentului.
\stopSymList


\page[yes]

\section{Cei trei pași ai intervenției de ajutor}
% NOTE [tf]: Shouldn't be in this book, IMO

\starttextbackground [CB]
\textDescrHead{Asigurați locul accidentului și persoanele afectate}
\startitemize
\item Verificați siguranța locului accidentului și asigurați-vă că nu mai pot apărea alte pericole.
\stopitemize
\textDescrHead{Evaluați starea persoanelor rănite}
\startitemize
\item Verificați dacă persoanele rănite sunt conștiente și pot respira normal. Dacă este cazul, eliberați-le căile respiratorii.
\stopitemize
\textDescrHead{Informații echipele de salvare}
\startitemize Apelul dumneavoastră trebuie să conțină următoarele informații:\par
\item Numărul de telefon la care puteți fi contactat.
\item Tipul incidentul (boală, accident).
\item Riscurile existente(incendiu, explozie, pericol de prăbușire etc.).
\item Locul exact unde a avut loc incidentul.
\item Numărul persoanelor rănite și starea acestora.
\item Măsurile de ajutor care au fost deja aplicate.
\item Răspundeți la celelalte întrebări care vă sunt adresate.
\stopitemize
\stoptextbackground

\stopcomponent
