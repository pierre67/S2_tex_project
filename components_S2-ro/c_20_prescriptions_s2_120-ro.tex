\startcomponent c_20_prescriptions_s2_120-ro
\product prd_ba_s2_120-ro


\chapter [safety:risques] {Normele de siguranță}

\setups [pagestyle:marginless]


\section{Indicații de bază}

\subsubject{Bazele legale}

Accidentele pot avea o serie de consecințe grave atât pentru angajator, cât și pentru angajat. Din acest motiv, dorim să vă reamintim obligațiile celor două părți:\note[prescription:user:right].

Angajatorul este obligat ca înainte de a încredința angajatului mașina de măturat, să respecte următoarele puncte:

\startSteps
\item Fiecare șofer de autovehicul trebuie să fie instruit în domeniul conducerii autovehiculelor. Certificatul de studii trebuie să fie prezentat spre verificare.
\item Fiecare șofer trebuie să dispună de un permis formal de conducere a autovehiculul. Acesta trebuie emis doar atunci când sunt respectate următoarele trei condiții:
\startitemize [2]
\item Angajatul trebuie să treacă un test medical de aptitudini efectuat de medicul unității.
\item Angajatul cunoaște locurile de desfășurare a operațiunilor de lucru și este familiarizat cu normele de siguranță privind autovehiculul, valabile în punctul de lucru, norme care i-au sunt aduse la cunoștință de superiorii săi.
\item Angajatul a trecut un test de aptitudini, care confirmă că acesta dispune de cunoștințele necesare pentru conducerea autovehiculului.
\stopitemize
\stopSteps

În cazul în care viteza maximă a autovehiculului este mai mare de 25 km/h\note[prescription:user:right], autovehiculul trebuie înmatriculat, iar șoferul autovehiculului trebuie să dispună de următorul permis de conducere:
\startitemize
\item Permis de conducere categoria B\note[prescription:lisence] pentru autovehicule cu o greutate maximă admisă mai mică de 3,5 tone sau \item permis de conducere categoria C\note[prescription:lisence] pentru autovehicule cu o greutate maximă admisă de peste 3,5 tone.
\stopitemize

În cazul în care viteza maximă a autovehiculului este de 25 km/h, șoferul autovehiculului trebuie să cunoască cel puțin Codul Rutier în vigoare pentru circulația pe străzile și în locurile publice, chiar dacă pentru conducerea autovehiculului nu este necesar un permis de conducere\note[prescription:user:right] categoria B.

\footnotetext [prescription:user:right] {Obligațiile angajatorului și angajatului pot varia în funcție de țară și regiune. Familiarizați-vă cu normele în vigoare în țara, respectiv regiunea dumneavoastră.}

\footnotetext[prescription:lisence] {Directiva 2006/126/CE Parlamentului European şi a Consiliului din 20 decembrie 2006 privind permisele de conducere.}


\subsubject{Condiții de utilizare}

 \sdeux\ poate fi utilizat doar atunci când se află într-o stare optimă de funcționare. În plus, utilizatorul trebuie să respecte normele de siguranță și prevederile menționate în aceste instrucțiuni de utilizare. Disfuncționalitățile care afectează siguranța autovehiculului trebuie imediat înlăturate/remediate de către personalul de specialitate.
\blank [big]

\startSymList
\externalfigure [s2_inspection] [width=4.5em]
\SymList
{\md Întreținerea zilnică:}
După fiecare operațiune de lucru, supuneți autovehiculul unei verificări și reparați daunele și defecțiunile vizibile. În cazul unor defecțiuni sau disfuncționalități, informați imediat atelierul de specialitate. În cazul în care acest lucru nu este posibil, scoateți imediat autovehiculul din funcţiune și izolați locul incidentului.
\stopSymList


\subsubject{Utilizare în conformitate cu scopul prevăzut}

\sdeux\ este concepută pentru lucrările de curățenie și întreținere a străzilor, aleilor și piețelor. Orice utilizare în afara acestui cadru este considerată ca nefiind corespunzătoare. În acest caz, firma \boschung\ nu își asumă responsabilitatea pentru daunele apărute în acest fel. În cazul utilizării neconforme cu scopul prevăzut, utilizatorul își asumă întreaga responsabilitate. {\em Utilizarea în conformitate cu scopul prevăzut cuprinde și respectarea instrucțiunilor de siguranță și al planului de întreținere, incluse în prezentul Manual de utilizare.}


\section{Circulația pe drumurile publice}

\subsubject{Prevederi generale}

În afară de instrucțiunile de utilizare trebuie respectate toate normele general valabile, prevederile legale, precum și reglementările privind prevenirea accidentelor și pentru protejarea mediului.


\subsubject{Locul pasagerului}

Un/o pasager/ă are voie să se așeze doar pe scaunul prevăzut în acest sens, așa numitul {\em loc pentru} pasager.


\subsubject{Centura de siguranță}

\startSymList
% \externalfigure [prescription:safety:belt]
\PMbelt
\SymList
Șoferul și pasagerul din \sdeux\ trebuie să folosească centura de siguranță, atunci când se află în autovehicul.
\stopSymList


\subsubject{A vedea și a fi văzut}

\startSymList
\externalfigure [travaux_deviation] [width=3.5em]
\SymList
Asigurați-vă că sunteți vizibil, mai ales pe străzile cu trafic intens.

În cazul în care în timpul unei manevre de mers sau într-un anumit punct de lucru, șoferul autovehiculului nu are vizibilitate suficientă, trebuie să solicite ajutorul unei alte persoane, cu care să mențină contact vizual.
\stopSymList


\subsubject{Iluminatul și mijloace de semnalizare}

În funcție de prevederile Codului Rutier în vigoare, luminile de poziție și farurile autovehiculului trebuie pornite, dacă este cazul, și în timpul zilei.


\subsubject{Utilizarea telefoanelor mobile}

\startSymList
\PPphone
\SymList
Utilizarea unui telefon mobil sau a unei stații în timpul mersului pe drumurile publice sunt strict interzise, excepție făcând cazul în care autovehiculul este echipat cu un dispozitiv tip Handsfree.

Apelurile telefonice\index{Siguranță+Telefon mobil} la volan – chiar și cu dispozitiv Handsfree – afectează
în orice caz concentrarea asupra circulației rutiere.
\stopSymList


\section{Prevederi de întreținere}

\subsubject{Instrucțiuni de întreținere}

Înainte de începerea lucrărilor, personalul responsabil cu întreținerea trebuie să citească
Manualul de utilizare al \sdeux, mai ales capitolele privind siguranța și întreținere.


\subsubject{Calificări necesare}

\startSymList
\externalfigure [mecanicienne] [width=3.5em]
\SymList
Doar persoanele care dispun de cunoștințele necesare obținute în urma unui curs de instruire sunt autorizate să efectueze lucrările de revizie la \sdeux. Acest lucru se aplică mai ales în cazul lucrărilor la motor, la sistemul de frânare, la direcție și la instalația electrică și hidraulică.
\stopSymList


\testpage [6]
\subsubject{Supraveghere}

\startSymList
\externalfigure [mecanicien_hyerarchie] [width=3.5em]
\SymList
Persoanele care se află în perioada de pregătire - stagiu de pregătire sau ucenicie –, pot lucra cu autovehiculul
numai sub supravegherea unei persoane de specialitate. Verificați în mod aleator dacă personalul respectă indicațiile din Manualul de utilizare și normele de siguranță.
\stopSymList


\subsubject{Lucrările de sudură}

\startSymList
\externalfigure [pince_soudure2] [width=3.5em]
\SymList
Înainte de efectuarea lucrărilor de sudură la caroserie sau șasiu, acumulatorul și toate dispozitivele electronice de comandă trebuie neapărat decuplate.
\stopSymList

\subsubject{Curățarea autovehiculului}

\startSymList
\externalfigure [washer_pressure] [width=3.5em]
\SymList
Înainte de operațiunile de curățenie ale \sdeux\ citiți capitolul \about[sec:cleaning] \atpage[sec:cleaning], mai ales capitolul privind instrucțiunile de curățenie.
\stopSymList


\subsubject{Accesul la documentația autovehiculului}

\startSymList
\externalfigure [lecteur_1] [width=3.5em]%\PMrtfm
\SymList
În timpul operațiunilor de lucru, păstrați documentația autovehiculului întotdeauna la îndemână în cabina șoferului.
\stopSymList


\section{Prevederi speciale de utilizare}

\subsubject{Înălțimea autovehiculului}

\startSymList
\PPmaxheight
\SymList
În cazul lucrărilor/mersului în zone nedeschise (garaje subterane, pasaje subterane,etc ), asigurați-vă că înălțimea de traversare pentru \sdeux\ este suficientă (a se vedea \in{Capitolul}[sec:measurement], \atpage[sec:measurement]).
\stopSymList


\subsubject{Stabilitatea autovehiculului}

Evitați orice manevră care ar putea afecta stabilitatea autovehiculului. În cazul unei viteze ridicate în curbe, din cauza structurii sale și a centrului de greutate ridicat atunci când recipientul de gunoi este plin, \sdeux\ s-ar putea răsturna.


\subsubject{Deplasarea nedorită a autovehiculului}

Atunci când părăsiți autovehiculul, asigurați-l împotriva utilizării de către persoane neautorizate. În principiu, activați frâna de staționare înainte de a părăsi autovehiculul și, dacă este cazul, asigurați roțile cu pene de blocare.

\startbuffer [prescription:handbrake]
\starttextbackground [CB]
\startPictPar
\PPstop
\PictPar
{\md Trageți frâna de acționare!}În caz contrar, autovehiculul s-ar putea pune în mișcare, chiar și\index{Frână de staționare+Potențial pericol} pe o pantă abia sesizabilă, existând riscul de provoca un accident cu risc fatal pentru terțe persoane.

{\lt Din cauza sistemului de acționare de tip hidrostatic, în staționare, presiunea din circuitul hidraulic se reduce treptat, ceea ce aduce cu sine o reducere a forței de fixare a motorului. Din acest motiv este deosebit de important ca la părăsirea autovehiculului să trageți frâna de staționare.}
\stopPictPar
\stoptextbackground

\stopbuffer

\getbuffer [prescription:handbrake]


\testpage [6]
\subsubject{Recipient pentru gunoi}

\startbuffer [prescription:container:gravity]
\starttextbackground [CB]
\startPictPar
\PHgravite
\PictPar
{\md Pericol de accident:}
{\lt La înclinarea recipientului de gunoi în sus, centrul de greutate se transferă în partea superioară. În acest fel, crește pericolul de răsturnare a autovehiculului. De aceea, la înclinarea recipientului asigurați-vă că autovehiculul se află pe o suprafață orizontală și stabilă.}
\stopPictPar
\stoptextbackground

\stopbuffer

\getbuffer [prescription:container:gravity]


\startbuffer [prescription:container:tilt]
\starttextbackground [CB]
\startPictPar
\PHcrushing
\PictPar
{\md Pericol de accident:}
{\lt Nu desfășurați niciodată lucrări sub recipient, înainte de a fi fixat sistemele de siguranță la cilindrul hidraulic ridicător al recipientului de gunoi.}
\stopPictPar
\stoptextbackground

\stopbuffer

\getbuffer [prescription:container:tilt]


\stopcomponent
