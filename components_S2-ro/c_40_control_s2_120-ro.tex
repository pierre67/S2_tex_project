\startcomponent c_40_control_s2_120-ro
\product prd_ba_s2_120-ro


\startchapter [title={Elementele de comandă ale autovehiculului},
reference={chap:ctrl}]

\setups[pagestyle:marginless]

\placefig[here][fig:ctrl:cab:front]{Elemente de comandă}
{\externalfigure[ctrl:cab:front]}

\startcolumns [n=3]
\startLongleg
 \item Coloană de direcție (\in{§}[sec:steeringColumn])
 \item Reglare coloană de direcție
 \item Pedală de frână și de accelerație
 \item Computer de bord \Vpad SN (\inP[sec:vpad])
 \item Consolă de tavan (\inP[sec:ctrl:aux])
 \item Radio/MP3
\stopLongleg


\subsubsubject{Dotări opționale}

\startLongleg [continue]
 \item Monitor mers cu spatele
\stopLongleg
\stopcolumns

\startsection [title={Coloana de direcție},
reference={sec:steeringColumn}]

\subsection{Reglarea coloanei de direcție}

\textDescrHead{Înclinarea volanului} Apăsați pedala și \Ltwo simultan reglați unghiul de înclinare al coloanei de direcție. Eliberați pedala pentru a bloca la loc mecanismul coloanei de direcție.

\page[yes]
\setups [pagestyle:normal]


\subsection{Sisteme de iluminat și de semnalizare}

\placefig [margin] [fig:column:left] {Manetă multifuncțională și întrerupător rotativ}
{\externalfigure[ctrl:column:left]}

\placefig [margin] [fig:column:right] {Manetă de selectare a vitezelor de mers}
{\externalfigure[ctrl:column:right]}


\subsubsubject{Întrerupător rotativ}

\startitemize[width=1.7em]
\sym{\textSymb{com_lowlight}} Lumină fază scurtă (rotire \TorqueR).
\startitemize
\sym{1} Lumină de poziție
\sym{2} Lumină fază scurtă
\stopitemize
\stopitemize


\subsubsubject{Manetă multifuncțională}

\startitemize[width=1.7em]
\sym{\textSymb{com_lowlight}} {[}Nicht belegt{]}
\sym{\textSymb{com_light}} Claxon luminos (Apăsați scurt maneta în sus)
\sym{\textSymb{com_blink}} Indicator direcție de mers (maneta în sus/jos)
\sym{\textSymb{com_claxonArrow}} Claxon (apăsați butonul din exteriorul manetei)
\sym{\textSymb{com_wipper}} Ștergătoare de parbriz
\startitemize
\sym{J} Temporizare
\sym{O} Oprit
\sym{I} 1. treaptă de viteză
\sym{II} 2. treaptă de viteză
\stopitemize
\sym{\textSymb{com_washerArrow}} Instalație de spălare a parbrizelor (apăsați pe coroană, la capătul manetei).
\stopitemize


\subsubsubject{Manetă de selectare a vitezelor}

Funcțiile manetei de selectare a vitezelor de mers sunt descrise în detaliu în capitolul \about[chap:using], începând cu\atpage[sec:using:start].

\stopsection

\page [yes]


\startsection [title={Alte funcții},
reference={sec:ctrl:add}]


\subsection[sec:ctrl:aux]{Consolă de tavan}

{\sl Consola\index{Consolă de tavan} de tavan se află în partea din față a tavanului din cabină, pe partea șoferului.}
\placefig [margin] [fig:console:aux] {Consolă de tavan}
{\externalfigure[ctrl:console:aux]}


\placefig [margin] [fig:console:climat] {Încălzire și instalație de climatizare}
{\externalfigure[ctrl:console:climat]}


\startitemize [unpacked][width=1.7em]
\sym{\textBigSymb{temoin_retrochauffant}} Încălzire oglinzi exterioare
\sym{\textBigSymb{temoin_hazard}} Lumină intermitentă de avertizare
\sym{\textBigSymb{temoin_eclairage_L}} Farul de lucru
\stopitemize


\subsubsubject{Dotări opționale}

\startLeg [unpacked][width=1.7em]
\sym{\textBigSymb{temoin_buse}} Pompă de apă de înaltă presiune pentru pistolul de apă \crlf {\sl a se vedea \atpage[sec:using:water:spray]}
\sym{\textBigSymb{temoin_aspiration_manuelle}} Turbină pentru furtunul de aspirație manual \crlf {\sl a se vedea \atpage[sec:using:suction:hose]}
\stopLeg


\subsection[sec:ctrl:climat]{Încălzire și instalație de climatizare}

{\sl Această consolă\index{Consolă de încălzire} se află pe peretele din spate al cabinei șoferului, între scaune.}

\startitemize [unpacked][width=23mm]
\sym{\bf 0\quad I\quad II\quad III} întrerupător rotativ suflantă
\sym{\externalfigure[tirette_chauffage][height=1em]} Regulator de translație pentru temperatură
\stopitemize


\subsubsubject{Dotări opționale}

\startitemize [unpacked][width=1.7em]
\sym{\textBigSymb{temoin_climat_bk}} Instalație de climatizare
\stopitemize

\page [yes]

\setups [pagestyle:bigmargin]


\subsection[sec:ctrl:central]{Consolă centrală}

{\sl Consola\index{Consolă centrală} centrală se află între scaune.}

\placefig [margin] [fig:console:central] {Consolă centrală}
{\externalfigure[ctrl:console:central]}


\subsubsubject{Umezirea măturii}

\startLeg [unpacked][width=1.7em]
\sym{\textBigSymb{temoin_busebalais}} Pompă de apă de joasă tensiune\index{Pompă de apă} pentru sistemul de umezire\index{Pompă de apă+Umezire} al măturii. (Poziția 1: {\em Automatic}; Poziția 2: {\em Permanent})
\stopLeg


\subsubsubject{Înclinarea recipientului de murdărie}

\setupinmargin[right][style=normal]
\inright{%
\startitemize
\sym{\textSymb{mand_readtheoperatingmanual}} Vă rugăm să respectați instrucțiunile de folosire a frânei de staționare \atpage[sec:using:stop].
\stopitemize}

\startLeg [unpacked][width=1.7em]
\sym{\textBigSymb{temoin_kipp2}} Înclinarea recipientului de gunoi.
Pentru\index{Recipient de gunoi+Înclinare} a putea înclina recipientul de gunoi,
trebuie activată frâna de staționare, iar maneta de selectare a vitezelor trebuie fixată în poziția neutră.
\stopLeg


\subsubsubject{Oprire de urgență}

\starttextbackground [FC]
\startPictPar
\externalfigure[Emergency_Stop][Pict]
\PictPar
În cazuri de urgență\index{Oprire de urgență} toate echipamentele de aspirație și măturat, precum și mecanismele de acționare, pot fi scoase din funcțiune prin apăsarea întrerupătorului de avarie.
\stopPictPar
\stoptextbackground


\subsection[sec:foot:switch]{Întrerupător de picior}

\placefig [margin] [fig:foot:switch] {Întrerupător de picior}
{\vskip 60pt
\externalfigure[work:foot:switch]}

Cu ajutorul acestui întrerupător\index{Întrerupător de picior} de la piciorul coloanei de direcție (\inF[fig:foot:switch]) puteți coborî rapid mătura, atunci când este necesar (\eG\ în vârful unei pante, urmarea pe un trotuar).

\stopsection
\page[yes]
\setups [pagestyle:marginless]


\startsection[title={Consolă multifuncțională},
reference={ctrl:console:middle}]

\startlocalfootnotes

\startfigtext[left]{Consolă multifuncțională}
{\externalfigure[overview:joy:large]}


\subsubsubject{Joystick-uri}

\textDescrHead{Fără mătură frontală (sau mătură frontală dezactivată):}
Joystick-urile comandă independent unul de celălalt câte o mătură: Ridicare/coborâre~(\textSymb{joystick_aa}) sau stânga/dreapta~(\textSymb{joystick_gd}). Joystick-ul stâng controlează mătura stângă, iar cel drept controlează mătura dreaptă.\footnote{Pentru a putea modifica poziția măturii laterale la un autovehicul care este echipat cu mătură frontală, această mătură trebuie dezactivată (tasta~\textSymb{joy_key_frontbrush_act}).}

\textDescrHead{Cu mătură frontală:}
Cu ajutorul Joystick-ului stâng puteți ridica/coborî mătura frontală(\textSymb{joystick_aa}) și o puteți deplasa în direcția stânga/dreapta (\textSymb{joystick_gd}). Cu ajutorul joystickului drept înclinați mătura pe axa sa longitudinală~(\textSymb{joystick_aa}) și transversală~(\textSymb{joystick_gd}).

\placelocalfootnotes %[height=\textheight]
\stopfigtext
\stoplocalfootnotes
% \vfill

\adaptlayout [height=+5mm]


\subsubsubject{Taste laterale}

\startcolumns

\startPictList
\VPcltr
\PictList
Tempomat: Creșterea vitezei reglate
\stopPictList\vskip -3pt

\startPictList
\VPclbr
\PictList
Tempomat: Micșorarea vitezei reglate
\stopPictList\vskip -3pt

\startPictList
\VPcrtr
\PictList
Ridicare gură de aspirație
\stopPictList

\startPictList
\VPcrbr
\PictList
Coborâre gură de aspirație
\stopPictList\vskip -3pt

\startPictList
\VPcrtf
\PictList
Deschidere clapetă gunoi grosier (în față la gura de aspirație)
\stopPictList\vskip -3pt

\startPictList
\VPcrbf
\PictList
Închidere clapetă gunoi grosier
\stopPictList

\stopcolumns

\defineparagraphs[SymVpad][n=2,distance=4mm,rule=off,before={\page[preference]},after={\nobreak\hrule\blank [2*medium]}]
\setupparagraphs [SymVpad][1][width=4em,inner=\hfill]

\subsubsubject{Taste simboluri}

% \startcolumns

\startSymVpad
\externalfigure[joy:stop]
\SymVpad
\textDescrHead{Stop} Oprire echipament activat:

1\:× apăsați: 3.\,dezactivați mătura\crlf
2\:× apăsați: Dezactivați totul
\stopSymVpad

\startSymVpad
\externalfigure[joy:tempomat]
\SymVpad
\textDescrHead{Tempomat} Reglați tempomatele la viteza momentană și activați-le. Pentru dezactivare, acționați din nou tasta ~\textSymb{joy:tempomat} sau frânați. Accelerați/încetiniți cu tastele laterale.
\stopSymVpad

\startSymVpad
\externalfigure[joy:ftbrs:minus]
\SymVpad
\textDescrHead{Viteza măturii} Reducerea vitezei de rotație a măturii laterale sau a măturii frontale.
\stopSymVpad

\startSymVpad
\externalfigure[joy:ftbrs:plus]
\SymVpad
\textDescrHead{Viteza măturii} Creșterea vitezei de rotație a măturii laterale sau a măturii frontale.
\stopSymVpad

\startSymVpad
\externalfigure[joy:eng:minus]
\SymVpad
\textDescrHead{Turație motor} Reduceți turația motorului diesel.
\stopSymVpad

\startSymVpad
\externalfigure[joy:eng:plus]
\SymVpad
\textDescrHead{Turație motor} Măriți turația motorului diesel.
\stopSymVpad
\columnbreak

\startSymVpad
\externalfigure[joy:suc]
\SymVpad
\textDescrHead{Aspirație} Activați sistemul de aspirație: Gura de aspirație este coborâtă, turbina și pompa de apă de reciclare sunt puse în funcțiune.\note [recyclingwaterpump] \crlf
Apăsați~\textSymb{joy:stop} tasta stop pentru a dezactiva sistemul.
\stopSymVpad

\startSymVpad
\externalfigure[joy:sucbrs]
\SymVpad
\textDescrHead{Activați operațiunea de măturare/aspirație}sistemul de aspirare/măturare: Gura de aspirație este coborâtă, măturile laterale sunt coborâte și poziționate, turbina, mătura și pompa de apă de reciclare sunt puse în funcțiune.\note [recyclingwaterpump] \crlf
Apăsați~\textSymb{joy:stop} tasta stop pentru a dezactiva sistemul.
\stopSymVpad

\footnotetext[recyclingwaterpump]{Pompa cu apă proaspătă este de asemenea cuplată, atunci când întrerupătorul este poziționat ~\textBigSymb{temoin_busebalais} pe {\em Automatic} (a se vedea \in [sec:ctrl:central] la \atpage [sec:ctrl:central]).}
\startSymVpad
\externalfigure[joy:ftbrs:act]
\SymVpad
\textDescrHead{Mătură frontală activată} Activare/dezactivare mătură frontală.
%% NOTE @Andrew: Singular
\stopSymVpad

\startSymVpad
\externalfigure[joy:ftbrs:right]
\SymVpad
\textDescrHead{Mătură frontală stânga} Direcție de rotire pentru operațiunile cu mătura frontală pe partea stângă (direcția de rotație: sensul acelor de ceas).
\stopSymVpad

\startSymVpad
\externalfigure[joy:ftbrs:left]
\SymVpad
\textDescrHead{Mătură frontală dreapta} Direcție de rotire pentru operațiunile cu mătura frontală pe partea stângă (direcția de rotație: în sensul invers acelor de ceas).
\stopSymVpad

% \stopcolumns

\stopsection

\stopchapter

\stopcomponent
