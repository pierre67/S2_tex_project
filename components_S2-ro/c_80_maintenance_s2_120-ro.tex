\startcomponent c_80_maintenance_s2_120-ro
\product prd_ba_s2_120-ro

\startchapter [title={Întreţinerea şi menţinerea în stare de funcţionare},
reference={chap:maintenance}]

\setups[pagestyle:marginless]


\startsection [title={Instrucțiuni generale}]


\subsection{Protejarea mediului înconjurător}

\starttextbackground [FC]
\setupparagraphs [PictPar][1][width=2.45em,inner=\hfill]

\startPictPar
\Penvironment
\PictPar
\Boschung\ transpune în practică \index{Protejarea mediului înconjurător} normele de protecție a mediului. Pentru noi cauzele sunt foarte importante, astfel efectele procesului de producţie şi ale produselor asupra mediului înconjurător sunt incluse în procesul decizional din interiorul întreprinderii. Scopul nostru este economisirea resurselor şi prelucrarea atentă a elementelor necesare vieţii, a căror conservare este scopul omului şi al naturii. Prin respectarea anumitor reguli în timpul manipulării autovehiculului, puteți contribui la protejarea mediului înconjurător. Aici se înscrie și manipularea corectă și corespunzătoare a substanțelor și materialelor în cadrul operațiunilor de întreținere a autovehiculului (\eG\ eliminarea substanțelor chimice și a deșeurilor speciale).

Consumul de combustibil al motorului şi gradul de uzură a acestuia depinde de condiţiile de utilizare. Din acest considerent vă rugăm acordaţi atenţie următoarelor aspecte:

\startitemize
\item Nu încălziţi motorul lăsându-l să funcţioneze în gol.
\item Opriţi motorul pe durata timpilor de aşteptare necesari pentru buna funcţionare a acestuia.
\item Verificați regulat consumul de carburant.
\item {\em Lucrările de revizie trebuie efectuate în conformitate cu planul de lucru și executate de personal de specialitate.}
\stopitemize
\stopSymList
\stoptextbackground

\page [yes]


\subsection{Cerințe privind siguranța}

\startSymList
\PHgeneric
\SymList
Pentru\index{Revizie+Cerințe privind siguranța} a evita producerea de daune la autovehicule, precum și accidentele în timpul lucrărilor de revizie, este absolut obligatoriu să respectați următoarele cerințe privind siguranța. De asemenea, respectați cerințele generale privind siguranța (\about[safety:risques], \at{de la pagina}[safety:risques]).
\stopSymList

\starttextbackground [FC]
\startPictPar
\PMgeneric
\PictPar
\textDescrHead{Prevenirea accidentelor}
După fiecare lucrare de reparație sau revizie,\index{Prevenirea accidentelor} verificați starea autovehiculului. Asigurați-vă mai ales de faptul că toate componentele importante pentru asigurarea siguranței, precum și toate dispozitivele de iluminat și semnalizare funcționează în mod corespunzător, înainte să circulați pe drumurile publice.
\stopPictPar
\stoptextbackground
\blank [big]

\start
\setupparagraphs [SymList][1][width=6em,inner=\hfill]
\startSymList\PHcrushing\PHfalling\SymList
\textDescrHead{Stabilizarea autovehiculului}
Înainte de fiecare lucrare de revizie, autovehiculul trebuie asigurat împotriva oricăror mișcări nedorite: Poziționați schimbătorul de viteză în poziția {\em neutră}, activați frâna de staționare și asigurați autovehiculul cu cale pentru blocarea roților.
\stopSymList
\stop

\starttextbackground[CB]
\startPictPar\PHpoison\PictPar
\textDescrHead{Pornirea motorului}
În cazul în care\index{Pericol+intoxicare} trebuie să porniți motorul într-un loc care nu dispune de o aerisire suficientă, lăsați-l în funcțiune doar timpul necesar\index{Pericol+gaze de eșapament}, pentru a evita intoxicațiile cu monoxid de carbon.
\stopPictPar
\startitemize
\item Porniţi motorul numai dacă acumulatorii sunt conectaţi corespunzător.
\item Nu conectați bateria la cleme în timp ce motorul este în funcțiune.
\item Nu porniți motorul cu ajutorul unui dispozitiv auxiliar de pornire. În cazul în care\index{Încărcător+acumulator} acumulatorul trebuie încărcat cu încărcător rapid, acumulatorul trebuie scos în prealabil din autovehicul. Vă rugăm acordaţi atenţie instrucţiunilor de utilizare a încărcătorului rapid.
\stopitemize
\stoptextbackground

\page [bigpreference]


\subsubsection{Protejarea componentelor electronice}

\startitemize
\item Înainte de a începe\index{Sudură electrică} lucrările de sudură, separați cablul de la acumulator și conectați legătura la masă și cablul plus al acumulatorului.
\item Conectați și separați\index{Electronica} sistemele electronice de comandă doar atunci când acestea nu se află sub tensiune.
\item În cazul sistemului de alimentare cu curent electric\index{Dispozitiv de comandă} polaritatea greşită (\eG\ (prin acumulatori conectaţi necorespunzător) poate cauza distrugerea pieselor componente şi a aparatelor electronice.
\item La\index{Temperaturi ambientale+extreme} temperaturi ambientale de peste 80 °C (\eG\ într-o cameră de uscare), componentele/aparatele electronice trebuie evacuate.
\stopitemize


\subsubsection{Diagnosticare și măsurări}

\startitemize
\item Pentru lucrările de măsurare și diagnosticare, utilizați doar cabluri {\em adecvate} de verificare (\eG\ cablurile originale ale aparatului).
\item Telefoanele mobile\index{Telefoane mobile} și alte dispozitive similare pot afecta funcțiile autovehiculului, ale aparatului de diagnosticare și astfel desfășurarea în condiții de siguranță a procesului de funcționare.
\stopitemize


\subsubsection{Calificarea personalului}

\starttextbackground[CB]
\startPictPar
\PHgeneric
\PictPar
\textDescrHead{Pericol de accidente}
În cazul unei\index{Calificare+personal de întreținere} execuții inadevcate a lucrărilor de revizie, capacitatea de funcționare și siguranța autovehiculului pot fi afectate. O astfel de obstrucţionare creşte riscul producerii de accidente şi a suferirii de leziuni.

Astfel,\index{Calificare+atelier} dacă este necesară executarea unor lucrări de întreţinere sau de reparaţii la autovehicul, adresaţi-vă exclusiv unui atelier specializat, care dispune de cunoştinţele şi de instrumentele necesare.

În cazul în care aveți îndoieli \Boschung, adresaţi-vă Serviciului Clienţi al firmei.
\stopPictPar
\stoptextbackground

Autovehiculul \ProductId trebuie deservit, reparat şi întreţinut exclusiv de un personal calificat şi autorizat \Boschung Serviciul Clienţi al firmei.

Competenţele necesare pentru executarea lucrărilor de reparaţii şi de întreţinere vor fi definite \Boschung Serviciul Clienţi al firmei.


\subsubsection{Modificări și transformări}

\starttextbackground[CB]
\startPictPar
\PHgeneric
\PictPar
\textDescrHead{Pericol de accidente}
Toate\index{Modificări la autovehicul} modificările, pe care le efectuați la autovehiculul Urban||Sweeper~S2 îi pot afecta capacitatea de funcționare și siguranța operațională, contribuind astfel la creșterea riscului de producere a unui accident sau unor leziuni.
\stopPictPar

\startPictPar
\PMwarranty
\PictPar
Pentru daunele produse\index{Condiții+de garanție} prin intervenții sau modificări proprii la Urban||Sweeper~S2
sau un agregat, \Boschung nu asigură nici un fel de garanție sau culanță.
\stopPictPar
\stoptextbackground

\stopsection


\startsection [title={Combustibili și lubrifianți}, reference={sec:liquids}]


\subsection{Utilizarea corectă}

\starttextbackground[CB]
\startPictPar
\PHpoison
\PictPar
\textDescrHead{Pericol de vătămare și intoxicare}
Prin\index{Combustibili} contactul cu pielea\index{Lubrifianți} sau\index{Pericol+de intoxicare} înghțirea de combustibili sau lubrifianți,\index{Carburanți+Siguranță} crește considerabil riscul de intoxicare sau producere de leziuni. În timpul utilizării, depozitării și eliminării acestor produse respectați întotdeauna prevederile legale.
\stopPictPar
\stoptextbackground

\starttextbackground [FC]
\startPictPar
\PMproteyes\par
\PMprothands
\PictPar
În momentul manipulării combustibililor și lubrifianților,purtaţi întotdeauna îmbrăcăminte de protecţie și echipament corespunzător pentru protejarea căilor respiratorii. Evitaţi inhalarea vaporilor.
Evitaţi producerea de contacte între combustibili şi piele, ochi sau îmbrăcăminte. Suprafaţa de piele intrată în contact cu combustibilii trebuie curăţată imediat cu apă şi săpun. În cazul în care combustibilii intră în contact cu ochii, spălaţi-vă din abundență cu apă curată şi, dacă este necesar, apelaţi la un medic oftalmolog. Dacă înghiţiţi combustibili, vă rugăm apelaţi imediat la un medic.
\stopPictPar
\stoptextbackground

\startSymList
\PPchildren
\SymList
Preveniţi accesul copiilor la combustibili.
\stopSymList

\startSymList
\PPfire
\SymList
\textDescrHead{Pericol de incendiu}
Gradul înalt de inflamabilitate a combustibililor \index{Pericol+de foc} creşte riscul producerii unui incendiu în momentul mânuirii acestor substanţe. Fumatul, folosirea surselor de flacără \index{Fumatul interzis} şi de lumină deschisă sunt strict interzise pe durata procesului de manipulare a combustibililor..
\stopSymList

\starttextbackground [FC]
\startPictPar
\PMgeneric
\PictPar
Utilizați doar lubrifianți, care sunt potriviți pentru părțile componente ale Urban||Sweeper~S2.
Din acest motiv, utilizați doar produse verificate și omologate de \Boschung.
Acestea sunt enumerate în lista de produse consumabile \atpage[sec:liqquantities].
Aditivii\index{Aditivi} pentru lubrifianți nu sunt necesari. În cazul în care adăugați aditivi,
aceștia pot duce la anularea drepturilor de garanție\index{Condiții+garanție}.
Pentru mai multe informații, adresați-vă Serviciului clienți \Boschung.
\stopPictPar
\stoptextbackground

\starttextbackground [FC]
\startPictPar
\Penvironment
\PictPar
\textDescrHead{Protejarea mediului înconjurător}
La eliminarea\index{Eliminarea+lubrifianților} carburanților și\crlf lubrifianților\index{Protejarea mediului înconjurător} sau a obiectelor contaminate cu acestea (\eG\ filtre, cârpe), respectați\index{Eliminarea+carburanților} prevederile în vigoare privind protejarea mediului înconjurător.
\stopPictPar
\stoptextbackground

\page [yes]

\setups [pagestyle:normal]


\subsection[sec:liqquantities]{Specificații și cantități de umplere}

% \adaptlayout [height=+5mm]

Toate\index{Cantitate+combustibili}\index{Cantitate+lubrifianți}\index{Cantități+combustibili și lubrifianți}\index{Specificații+combustibili și lubrifianți} cantitățile precizate în următorul tabel au caracter orientativ. După fiecare schimb de combustibili/lubrifianți, trebuie verificat nivelul real de umplere și, dacă este cazul, cantitatea de umplere trebuie mărită sau redusă.
% \blank[big]

\placetable[margin][tab:glyco]{Protecție împotriva înghețului (\index{Protecție împotriva gerului}motor)}
{\noteF\startframedcontent[FrTabulate]
%\starttabulate[|Bp(80pt)|r|r|]
\starttabulate[|Bp|r|r|]
\NC Protecție împotriva înghețului până la {[}°C{]}\NC \bf \textminus 25 \NC \bf \textminus 40 \NC\NR
\NC Apă distilată [Vol.-\%] \NC 60 \NC 40 \NC\NR
\NC Antigel \break [Vol.-\%] \NC 40 \NC {\em max.} 60 \NC\NR
\stoptabulate\stopframedcontent\endgraf
Atenţie: La o fracțiune de volum de mai mult de 60\hairspace\percent\ antigel {\em, nivelul de protecție} împotriva înghețului scade și puterea de răcire se diminuează!}

\placefig[margin][fig:hydrgauge]{\select{caption}{Indicator de nivel lichid hidraulic (partea stângă a autovehiculului)}{Indicator de nivel lichid hidraulic}}
{\externalfigure[main:hy:level_temp]
\noteF Nivelul de umplere al rezervorului hidraulic poate fi observat prin geamul de control și trebuie {\em verificat} zilnic.}

\vskip -8pt
\start
\define [1] \TableSmallSymb {\externalfigure[#1][height=4ex]}
\define\UC\emptY
\pagereference[page:table:liquids]

\setupTABLE [frame=off,style={\ssx\setupinterlinespace[line=.86\lH]},background=color, option=stretch, split=repeat]
\setupTABLE [r] [each] [topframe=on,
framecolor=TableWhite,
% rulethickness=.8pt
]

\setupTABLE [c] [odd] [backgroundcolor=TableMiddle]
\setupTABLE [c] [even] [backgroundcolor=TableLight]
\setupTABLE [c] [1][width=30mm]
\setupTABLE [c] [2][width=20mm]
\setupTABLE [c] [4][width=28mm]
\setupTABLE [c] [last] [width=10mm]
\setupTABLE [r] [first] [topframe=off,style={\bfx\setupinterlinespace[line=.95\lH]},
% backgroundcolor=TableDark
]
\setupTABLE [r] [2][framecolor=black]

\bTABLE

\bTABLEhead
\bTR
\bTC Grupa \eTC
\bTC Categoria \eTC
\bTC Clasificare \eTC
\bTC Produs\note[Produkt] \eTC
\bTC Cantitate \eTC
\eTR
\eTABLEhead

% TODO diverses adaptations largeur et textes 1re colonne

\bTABLEbody
\bTR \bTD Motor diesel \eTD
\bTD Ulei motor\eTD
\bTD \liqC{SAE 5W-30}; \liqC{VW 507.00}\eTD
\bTD Total Quartz INEO Long Life \eTD
\bTD 4,3\,l\eTD
\eTR
\bTR \bTD Circuit hidraulic \eTD
\bTD Ulei ATF \eTD
\bTD \liqC{dexron iii} \eTD
\bTD Total Equiviz ZS 46 (Rezervor cca. 40\,l) \eTD
\bTD cca. 50\,l\eTD
\eTR
\bTR \bTD Circuit hidraulic (opțiune {\em Bio})\eTD
\bTD Ulei ATF \eTD
\bTD \liqC{dexron iii} \eTD
\bTD Total Biohydran TMP SE 46 \eTD
\bTD cca. 50\,l\eTD
\eTR
\bTR \bTD Ventil electromagnetic: Miez de bobină \eTD
\bTD Lubrifiant\eTD
\bTD Vaselină pe bază de cupru \eTD
\bTD \emptY\eTD
\bTD n. B.\note[Bedarf] \eTD
\eTR
\bTR \bTD Diverse \eTD % Diverse: Încuietori, dispozitive mecanice pentru uși, pedală de frână
\bTD Lubrifiant\eTD
\bTD Universal-Spray\eTD
\bTD \emptY\eTD
\bTD n. B.\note[Bedarf] \eTD
\eTR
\bTR \bTD Sistem de gresare centralizată \eTD
\bTD Unsoare universală\eTD % Unsoare universală pentru lagăre
\bTD \liqC{nlgi} 2 \eTD
\bTD Total Multis EP 2\eTD
\bTD n. B.\note[Bedarf] \eTD
\eTR
\bTR \bTD Sistem de răcire \eTD
\bTD Antigel\eTD % Antigel/inhibitori de coroziune
\bTD TL VW 774 F/G; max. 60\hairspace\% vol.\eTD
\bTD G12+/G12++ (roz/violet)\eTD
\bTD cca. 14\,l \eTD
\eTR
\bTR \bTD Pompă de apă de mare presiune \eTD
\bTD Ulei motor\eTD
\bTD \liqC{SAE 10W-40}; \liqC{api cf – acea e6}\eTD
\bTD Total Rubia TIR 8900 \eTD
\bTD 0,29\,l \eTD
\eTR
\bTR \bTD Sistem de aer condiționat \eTD
\bTD Agent frigorific\eTD
\bTD + 20 ml ulei POE\eTD
\bTD R 134a\eTD
\bTD 700\,g\eTD
\eTR
\bTR \bTD Spălătoare de parbriz \eTD
\bTD [nc=2] Apă și concentrat pentru spălarea parbrizului, {\em S} vara, {\em W} iarna; respectați raportul de amestec \eTD
\bTD Comerț cu amănuntul \eTD
\bTD n. B.\note[Bedarf] \eTD
\eTR
\eTABLEbody

\eTABLE
\stop

\footnotetext[Bedarf]{{\it n. B.} în funcție de necesități, potrivit instrucțiunilor}
\footnotetext[Produkt]{Produse utilizate de \Boschung. Alte produse care corespund specificațiilor pot fi de asemenea utilizate.}

\stopsection

\page [yes]

\setups [pagestyle:marginless]


\startsection [title={Mentenanța motorului diesel}, reference={sec:workshop:vw}]


\subsection [sSec:vw:diagTool]{Sistem de diagnosticare On-Board}

\startregister[index][reg:main:vw]{Mentenanța+motorului diesel} Sistemul de comandă al motorului (J623) este dotat cu o memorie de stocare a erorilor. În cazul în care apar defecțiuni la nivelul senzorilor sau a componentelor monitorizate, acestea sunt stocate în memorie, precizându-se și tipul de eroare.

După evaluarea informației,\index{Diagnosticarea+motorului diesel} sistemul de comandă al motorului diferențiază diferite clase de erori și le memorează până la ștergerea conținutului memoriei de erori.

Erorile care apar doar {\em sporadic} sunt afișate cu mențiunea {\em SP}. Cauza erorilor sporadice poate fi \eG\ un contact slăbit sau o întrerupere de scurtă durată a conexiunii. În cazul în care apare în cursul a 50 de porniri o eroare sporadică nu mai apare, aceasta este ștearsă automat din eroare.

În cazul în care sunt identificate erori care afectează buna funcționare a motorului, pe ecranul Vpad se aprinde simbolul de control {\em Diagnosticare motor} \textSymb{vpadWarningEngine1}.

Erorile stocate pot fi citite cu sistemul de diagnosticare a autovehiculului, precum și cu sistemul de măsurare și informare {\em VAS 5051/B}.

După remedierea defecțiunii, erorile stocate trebuie șterse.


\subsubsection[sSec:vw:diagTool:connect]{Punerea în funcțiune a sistemului de diagnosticare}

\starttextbackground [FC]
\startPictPar
\PMgeneric
\PictPar
Informații detaliate despre sistemul de diagnosticare a autovehiculului VAS 5051/B sunt disponibile în Manualul de utilizare al sistemului.

Puteți utiliza și alte sisteme compatibile de diagnosticare, \eG\ {\em DiagRA}.
\stopPictPar
\stoptextbackground

\page [yes]


\subsubsubsubject{Condiții}

\startitemize
\item Siguranțele trebuie să fie în bună stare de funcționare.
\item Tensiunea acumulatorului trebuie să fie mai mare de 11,5 V.
\item Toți consumatorii electrici trebuie să fie opriţi.
\item Conexiunea de masă trebuie realizată în mod corespunzător.
\stopitemize


\subsubsubsubject{Procedura}

\startSteps
\item Introduceți ștecherul cablului de diagnosticare VAS 5051B/1 în conexiunea de diagnosticare.
\item În funcție de funcționare, fie porniți sistemul de aprindere sau porniți motorul.
\stopSteps

\subsubsubsubject{Selectați modul de funcționare}

\startSteps [continue]
\item Apăsați pe Display, pe butonul {\em Diagnosticare individuală a autovehiculului}.
\stopSteps


\subsubsubsubject{Selectați sistemul autovehiculului}

\startSteps [continue]
\item Apăsați pe Display, pe butonul {\em 01- Electronică motor}.
\stopSteps

Pe ecran este afișată acum seria de identificare a sistemului de comandă și codul sistemului de comandă al motorului.

În cazul în care codurile nu sunt corecte, acestea trebuie verificate.


\subsubsubsubject{Selectați funcția de diagnosticare}

Pe ecran, vă sunt afișate toate funcțiile de diagnosticare disponibile.

\startSteps [continue]
\item Apăsați pe display pe butonul pentru funcția dorită.
\stopSteps



\subsection [sSec:vw:faultMemory]{Memorie stocare erori}


\subsubsection{Citire memorie stocare erori}

\subsubsubject{Flux de activitate}

\startSteps
\item Lăsați motorul să funcţioneze în gol.
\item Conectați VAS 5051/B an (a se vedea \in{§}[sSec:vw:diagTool:connect])
și selectați sistemul de comandă al motorului.
\item Selectați funcția de diagnosticare {\em 004-Conținut memorie stocare erori}.
\item Selectați funcția de diagnosticare {\em 004.01-Verificare conținut memorie stocare erori}.
\stopSteps

{\sla Doar când motorul nu pornește:}

\startitemize [2]
\item Porniți sistemul de aprindere.
\item În cazul în care nu există nici o eroare în sistemul de comandă al motorului, pe display apare {\em 0 erori identificate}.
\item În cazul în care sunt identificate erori în sistemul de comandă al motorului, acestea sunt afișate pe display unele sub altele.
\item Finalizați funcția de diagnosticare.
\item Opriți sistemul de aprindere.
\item Dacă este cazul, remediați eroarea afișată cu ajutorul tabelului de erori (a se vedea Documentația de service) și la final ștergeți erorile remediate.
\stopitemize

\starttextbackground [FC]
\startPictPar
\PMrtfm
\PictPar
Atunci când o eroare nu poate fi ștearsă, vă rugăm să vă adresați serviciului clienți \boschung.
\stopPictPar
\stoptextbackground


\subsubsubject{Erori statice}

În cazul în care în memoria de date există una sau mai multe erori statice, vă rugăm să vă adresați serviciului clienți al firmei Boschung, pentru a remedia această eroare cu ajutorul {\em Căutarea ghidată a erorii}.


\subsubsubject{Erori sporadice}

În cazul în care în memorie apar doar erori sporadice sau indicații și nu sunt constate nici un fel de disfuncționalități ale sistemului electronic al autovehiculului, erorile stocate pot fi șterse:

\startSteps [continue]
\item Apăsați încă o dată tasta {\em Mai departe/următorul} \inframed[strut=local]{>}, pentru a ajunge la planul de verificări.
\item Pentru a finaliza operațiunea de căutare asistată a erorii,
apăsați tasta {\em Skip/Salt} și apoi {\em Finalizare}.
\stopSteps

Acum sunt verificate toate erorile salvate în memorie.

Într-o fereastră este confirmat faptul că toate erorile sporadice au fost șterse.
% Das Diagnoseprotokoll wird automatisch (online) verschickt.

Testarea sistemului autovehiculului este astfel finalizată.


\subsubsection[sSec:vw:faultMemory:errase]{Ștergerea erorilor memorate}

\subsubsubject{Flux de activitate}

{\sla Condiții:}

\startitemize [2]
\item Toate erorile trebuie corectate, iar cauza erorilor trebuie îndepărtată.
\stopitemize

\page [yes]


{\sla Procedura}

\starttextbackground [FC]
\startPictPar
\PMrtfm
\PictPar
După remedierea erorii, memoria trebuie din nou verificată și, la final, curățată:
\stopPictPar
\stoptextbackground

\startSteps
\item Lăsați motorul să funcţioneze în gol.
\item Conectați VAS 5051/B (a se vedea \in{§}[sSec:vw:diagTool:connect]) și selectați sistemul de comandă al motorului.
\item Selectați funcția de diagnosticare {\em 004-Verificare conținut memorie}.
\item Selectați funcția de diagnosticare {\em 004.10-Ștergere conținut memorie}.
\stopSteps

\starttextbackground [FC]
\startPictPar
\PMrtfm
\PictPar
În cazul în care memoria nu poate fi curățată, înseamnă că există încă o eroare, care trebuie remediată.
\stopPictPar
\stoptextbackground

\startSteps [continue]
\item Finalizați funcția de diagnosticare.
\item Opriți sistemul de aprindere.
\stopSteps


\subsection [sSec:vw:lub] {Ungerea motorului diesel}

\subsubsection [ssSec:vw:oilLevel] {Verificați nivelul de ulei de motor}

\starttextbackground [FC]
\startPictPar
\PMrtfm
\PictPar
Nivelul de ulei\index{Nivel+ulei de motor} nu trebuie în nici un caz să depășească marcajul {\em Max.}. În caz contrar\index{Nivel umplere+ulei de motor} există riscul de producere de defecțiuni la catalizator.
\stopPictPar
\stoptextbackground

\startSteps
\item Opriți motorul și așteptați cel puțin 3 minute pentru ca uleiul să poată curge înapoi în baia de ulei.
\item Scoateți tija de măsurare și ștergeți-o; introduceți la loc tija până în punctul de oprire.
\item Scoateți din nou tija de măsurare și evaluați nivelul de ulei:

\startfigtext[right][fig:vw:gauge]{Verificați nivelul de ulei}
{\externalfigure[VW_Oil_Gauge][width=50mm]}
\startitemize [A]
\item Nivelul maxim de umplere; nu trebuie adăugată o cantitate suplimentară de ulei.
\item Nivel de umplere suficient; poate fi adăugat {\em ulei} până la atingerea marcajului {\em A}.
\item Nivel de umplere insuficient; trebuie adăugat {\em ulei}, până când nivelul de umplere ajunge în zona {\em B}.
\stopitemize
{\em În cazul unui nivel de umplere peste marcajul {\em A} există riscul de producere de defecțiuni la nivelul catalizatorului.}
\stopfigtext
\stopSteps


\subsubsection [ssSec:vw:oilDraining] {Schimb ulei de motor}

\starttextbackground [FC]
\startPictPar
\PMrtfm
\PictPar
Filtrul pentru uleiul de motor S2 trebuie montat în poziție verticală. Aceasta înseamnă că filtrul {\em trebuie} schimbat înainte de schimbul de ulei. Prin scoaterea elementului de filtrare, se deschide o supapă, iar uleiul din carcasa filtrului curge automat în carterul motorului.
\stopPictPar
\stoptextbackground

\startSteps
\item Poziționați un recipient de captare adecvat\index{Schimb de ulei+motor diesel} sub motor.
\item Deșurubați șurubul de golire pentru ulei\index{Schimb+ ulei de motor} și lăsați uleiul să se scurgă.
\stopSteps

\starttextbackground [FC]
\startPictPar
\PMrtfm
\PictPar
Aveți grijă ca recipientul să poată cuprinde întreaga cantitate de ulei uzat.
Specificațiile necesare pentru ulei și cantitatea de umplere sunt prezentate în \in{§}[sec:liqquantities].

Șurubul de golire pentru ulei este prevăzut cu un inel de etanșare indispensabil. De aceea, șurubul de golire pentru ulei trebuie întotdeauna înlocuit.
\stopPictPar
\stoptextbackground

\startSteps [continue]
\item Înșurubați un nou șurub cu inel de etanșare (\TorqueR 30 Nm).
\item Umpleți cu ulei cu specificațiile corespunzătoare (a se vedea \in{§}[sec:liqquantities]).
\stopSteps


\subsubsection [ssSec:vw:oilFilter] {Înlocuiți filtrul pentru uleiul de motor}

\starttextbackground [FC]
\startPictPar
\PMrtfm
\PictPar
\startitemize [1]
\item Respectați prevederile\index{Filtrul de ulei+motor diesel} în vigoare privind eliminarea și reciclarea.
\item Schimbați\index{Filtru ulei+motor diesel} filtrele {\em înainte de} a realiza schimbul de ulei (a se vedea \in{§}[ssSec:vw:oilDraining]).
\item Înainte de montaj, ungeți ușor garnitura noului filtrului.
\stopitemize
\stopPictPar
\stoptextbackground

\startfigtext[right][fig:vw:oilFilter]{Filtru de ulei}
{\externalfigure[VW_OilFilter_03][width=50mm]}
\startSteps
\item Deșurubați capacul \Lone\ carcasei de filtru, folosind o cheie adecvată.
\item Curățați suprafețele de etanșare ale capacului și carcasei filtrului.
\item Înlocuiți elementul filtrant \Lthree.
\item Înlocuiți garniturile de tip O-Ring \Ltwo\ și \Lfour.
\item Înșurubați capacul la locul pe carcasa filtrului (\TorqueR 25 Nm).
\stopSteps



%\subsubsubject{Données techniques}
%
%
%\hangDescr{Couple de serrage du couvercle:} \TorqueR 25 Nm.
%
%\hangDescr{Huile moteur prescrite:} Selon tableau \atpage[sec:liqquantities].
%% NOTE: Redundant [tf]

\stopfigtext



\subsubsection [ssSec:vw:oilreplenish] {Umpleți cu ulei de motor}

\starttextbackground [FC]
\startPictPar
\PMrtfm
\PictPar
\startitemize [1]
\item Înainte de scoaterea capacului,\index{Ulei de motor} ștergeți cu o cârpă {\em ștuțul} de umplere.
\item Adăugați\index{Umpleți cu ulei+pentru motorul diesel} doar ulei care corespunde specificațiilor prescrise.
\item Umpleți treptat, în cantități mici.
\item Pentru a evita umplerea în exces, după fiecare reumplere așteptați ca uleiul să curgă în baia de ulei, până la marcajul tijei de măsurare (a se vedea \in{§}[ssSec:vw:oilLevel]).
\stopitemize
\stopPictPar
\stoptextbackground

\startfigtext[right][fig:vw:oilFilter]{Realimentați cu ulei}
{\externalfigure[s2_bouchonRemplissage][width=50mm]}
\startSteps
\item Scoateți tija de măsurare pentru ulei aproximativ 10 cm, astfel încât la umplere, să poată fi eliminat aerul.
\item Deschideți orificiul de umplere.
\item Umpleți cu ulei, respectând indicațiile de mai sus.
\item Închideți cu atenție orificiul de umplere.
\item Porniți motorul.
\item Efectuați un control pentru verificarea nivelului de umplere (a se vedea \in{§}[ssSec:vw:oilLevel]).
\stopSteps

\stopfigtext


\subsection [sSec:vw:fuel] {Sistemul de alimentare cu carburant}

\subsubsection [ssSec:vw:fuelFilter] {Înlocuire filtru carburant}

\starttextbackground [FC]
\startPictPar
\PMrtfm
\PictPar
\startitemize [1]
\item Respectați prevederile\index{Filtrul combustibil+motor diesel} în vigoare privind eliminarea și reciclarea.
\item Nu scoateți conductele de combustibil din partea de sus a filtrului.
\item Nu aplicați nici o forță asupra punctelor de fixare ale conductelor de combustibil; în caz contrar, poate fi prejudiciată partea de sus a filtrului.
\stopitemize
\stopPictPar
\stoptextbackground

\startfigtext[right][fig:vw:oilFilter]{Filtru combustibil}
{\externalfigure[s2_fuelFilter_location][width=50mm]}

{\sla Pregătire:}

Carcasa\index{Filtru de combustibil} filtrului de combustibil este fixată în fața motorului, pe partea dreaptă a șasiului.
Îndepărtați cele două șuruburi de fixare cu ajutorul unei chei cu crichet de 10 mm și a unei chei inelare de 10 mm.

\stopfigtext


\page [yes]

\setups [pagestyle:normal]

{\sla Procedura}

\startLongsteps
\item Îndepărtați toate șuruburile din partea superioară a filtrului. Scoateți partea superioară a filtrului.
\stopLongsteps

\starttextbackground [FC]
\startPictPar
\PMrtfm
\PictPar
Ridicați partea superioară a filtrului. În cazul în care este necesar, utilizați în acest sens o șurubelniță la canalul de montaj (\in{\LAa, fig.}[fig:fuelfilter:detach]) și îndepărtați partea superioară a filtrului.
\stopPictPar
\stoptextbackground

\placefig [margin] [fig:fuelfilter:detach]{Scoateți filtrul pentru combustibil}
{\externalfigure[fuelfilter:detach]}

\placefig [margin] [fig:fuelfilter:explosion]{Filtru pentru combustibil}
{\externalfigure[fuelfilter:explosion]}

\startLongsteps [continue]
\item Scoateți elementul filtrant din partea inferioară a filtrului.
\item Îndepărtați garnitura de etanșare (\in{\Ltwo, fig.}[fig:fuelfilter:explosion]) din partea superioară a filtrului.
\item Curățați cu atenție partea superioară și inferioară a filtrului.
\item Introduceți un element filtrant nou în partea inferioară a filtrului.
\item Umeziți o nouă garnitură (\in{\Ltwo, fig.}[fig:fuelfilter:explosion]) cu un pic de combustibil și aplicați-o în partea superioară a filtrului.
\item Așezați partea superioară pe partea inferioară a filtrului în mod corespunzător și apăsați cu aceeași putere, astfel încât aceasta să se poziționeze în mod corespunzător de jur împrejur.
\item Înșurubați manual toate șuruburile din partea inferioară {\em și superioară} a filtrului. Strângeți apoi toate șuruburile cu momentul de strângere prescris (\TorqueR 5 Nm).
\stopLongsteps

% \subsubsubject{Données techniques}
%
% \hangDescr{Couple de serrage des vis de fixation du couvercle:} \TorqueR 5 Nm.
%% NOTE: redundant [tf]

\startLongsteps [continue]
\item Porniți sistemul de aprindere pentru a ventila sistemul; porniți motorul și lăsați-l să meargă în gol timp de 1 până la 2 minute.
\item Ștergeți erorile din memorie în modul descris în \atpage[sSec:vw:faultMemory:errase].
\stopLongsteps


\subsection [sSec:vw:cooling] {Sistem de răcire}

\starttextbackground [FC]
\startPictPar
\PMrtfm
\PictPar
\startitemize [1]
\item Utilizați\index{Răcire+motor diesel} doar agent refrigerant conform specificațiilor prescrise (a se vedea Tabelul \atpage[sec:liqquantities]).
\item Pentru\index{Agent refrigerant} a asigura protecția împotriva înghețului și coroziunii, agentul refrigerant poate fi diluat doar cu apă distilată, potrivit tabelului de mai jos.
\item Nu umpleți niciodată cu apă circuitul pentru agent refrigerant, pentru că în acest mod este afectat nivelul de protecție împotriva înghețului și coroziunii.
\stopitemize
\stopPictPar
\stoptextbackground


\subsubsection [sSec:vw:coolingLevel] {Nivel agent refrigerant}

\placefig [margin] [fig:coolant:level] {Nivel agent refrigerant}
{\externalfigure[coolant:level]}


\placefig [margin] [fig:refractometer] {Refractometru VW T 10007}
{\externalfigure[coolant:refractometer]}

\placefig [margin] [fig:antifreeze] {Verificarea densității antigelului}
{\externalfigure[coolant:antifreeze]}


\startSteps
\item Ridicați recipientul pentru gunoi și fixați ștuțul de siguranță.
\item Verificați\index{Nivel umplere+agent refrigerant} nivelul de umplere cu agent refrigerant din vasul de expansiune: Trebuie să fie deasupra marcajului {\em min}.
\stopSteps

\start
\define [1] \TableSmallSymb {\externalfigure[#1][height=4ex]}
\define\UC\emptY
\pagereference[page:table:liquids]


\setupTABLE [frame=off,style={\ssx\setupinterlinespace[line=.86\lH]},background=color,
option=stretch,
split=repeat]
\setupTABLE [r] [each] [topframe=on,
framecolor=TableWhite,
% rulethickness=.8pt
]

\setupTABLE [c] [odd] [backgroundcolor=TableMiddle]
\setupTABLE [c] [even] [backgroundcolor=TableLight]
\setupTABLE [r] [first] [topframe=off,style={\bfx\setupinterlinespace[line=.95\lH]},
% backgroundcolor=TableDark
]
\setupTABLE [r] [2][framecolor=black]

\bTABLE

\bTABLEhead
\bTR
\bTC Antigel până la … \eTC
\bTC Fracțiune de volum G12\hairspace ++\eTC
\bTC Vol. antigel \eTC
\bTC Vol. apă distilată\eTC
\eTR
\eTABLEhead

\bTABLEbody
\bTR \bTD \textminus 25 °C \eTD
\bTD 40\hairspace\% \eTD
\bTD 3,8 l \eTD
\bTD 4,2 l \eTD
\eTR
\bTR \bTD \textminus 35 °C \eTD
\bTD 50\hairspace\% \eTD
\bTD 4,0 l \eTD
\bTD 4,0 l \eTD
\eTR
\bTR \bTD \textminus 40 °C \eTD
\bTD 60\hairspace\% \eTD
\bTD 4,2 l \eTD
\bTD 3,8 l \eTD
\eTR
\eTABLEbody

\eTABLE
\stop

\adaptlayout [height=+20pt]
\subsubsection [sSec:vw:coolingFreeze] {Nivel agent refrigerant}

Verificați\index{Densitatea antigelului} densitatea antigelului cu ajutorul unui refractometru adecvat (a se vedea \in{fig.}[fig:refractometer]: VW T 10007).
Respectați scala 1: G12\hairspace ++ (a se vedea \in{fig.}[fig:antifreeze]).

\page [yes]


\subsection [sSec:vw:airFilter] {Alimentarea cu aer}

Filtrul pentru aer este accesibil prin ușa de revizie din spate, de pe partea dreaptă a autovehiculului (a se vedea \in{fig.}[fig:airFilter]).

\placefig [margin] [fig:airFilter] {Filtrul de aer al motorului}
{\externalfigure[vw:air:filter]
\noteF
\startLeg
\item Eclisă de siguranță
\item Partea inferioară a carcasei
\item Orificiu de ventilație
\item Senzor de presiune
\stopLeg}


\subsubsubject{Condiții de utilizare}

Mașinile de măturat sunt frecvent utilizate în medii cu mari acumulări de praf.
Din acest motiv verificarea și curățarea săptămânală a filtrelor de aer este deosebit de importantă.
A se vedea și \about[table:scheduleweekly], \atpage[table:scheduleweekly].
Dacă este necesar, filtrul de aer trebuie înlocuit.


\subsubsubject{Autodiagnosticare}

Conducta de admisie dispune de un senzor de presiune (\Lfour, \in{fig.}[fig:airFilter]),
prin care pot fi stabilite pierderile de încărcare\footnote{Debit de aer redus ca urmare a permeabilității scăzute la aer a filtrului.} prin filtru.
Atunci când filtrul de aer este încărcat, pe displayul Vpad apare simbolul de control \textSymb{vpadWarningFilter}
și mesajul afișat este \VpadEr{851} înregistrat.


\subsubsubject{Întreținere/înlocuire}

\startSteps
\item Trageți în jos eclisa \Lone de siguranță (\in{fig.}[fig:airFilter]).
\item Rotiți partea inferioară a carcasei în sensul invers acelor de ceas \Ltwo și scoateți-o.
\item Scoateți elementul filtrant și verificați-l. În cazul în care este necesar, înlocuiți-l.
\item Curățați partea interioară a filtrului și asamblați filtrul de aer în ordinea inversă.
\stopSteps

\page [yes]


\subsection [sSec:vw:belt] {Curele de transmisie}

Cureaua\index{Motor diesel+curea de transmisie} de transmise preia mișcarea volantei arborelui cotit și o transmite spre alternator și compresor (dotare opțională).
Un\index{Curea de transmisie} element de menținere a tensiunii în ultimul segment (dintre alternator și arborele cotit) menține curea de transmisie sub tensiune.


\subsubsection [sSec:belt:change] {Înlocuirea curelei de transmisie}

\placefig [margin] [fig:belt:tool] {Dorn de blocare VW T 10060 A}
{\externalfigure[vw:belt:tool]}

\placefig [margin] [fig:belt:overview] {Element de menținere a tensiunii}
{\externalfigure[vw:belt:overview]}

\placefig [margin] [fig:belt:tens] {Punct de inserare al dornului de blocare}
{\externalfigure[vw:belt:tens]}


\subsubsubject{Cu compresor de climă}


{\sla Instrument de lucru special:}

Dorn de blocare {\em VW T 10060 A} pentru fixarea elementului de menținere a tensiunii.

\startSteps
\item Marcați direcția de rulare a curelei de transmisie.
\item Balansați în sensul acelor de ceas brațul elementului de menținere a tensiunii, folosind o cheie inelară (\in {fig.}[fig:belt:overview]).
\item Aliniați orificiile (a se vedea săgețile, \in {fig.}[fig:belt:tens]) și blocați elementul de menținere a tensiunii folosind dornul de blocare.
\item Scoateți cureaua de transmisie.
\stopSteps

Montajul curelei de transmisie se realizează în ordinea inversă.

\starttextbackground [FC]
\startPictPar
\PMrtfm
\PictPar
\startitemize [1]
\item Respectați direcția de rulare a curelei de transmisie.
\item Respectați poziția corectă a curelei pe disc.
\item Porniți motorul și verificați modul de rulare al curelei.
\stopitemize
\stopPictPar
\stoptextbackground


\subsubsubject{Fără compresor de climă}

{\sla Materiale necesare:}

Kit de reparații, constând din instrucțiuni de reparații, curea de transmisie și instrument special de lucru.\footnote{A se vedea catalogul cu piese de schimb, de la {\em Piese de întreținere}.}

\startSteps
\item Separați cureaua de transmisie.
\item Urmați pașii ulteriori din Instrucțiunile de reparații.
\stopSteps

\starttextbackground [FC]
\startPictPar
\PMrtfm
\PictPar
\startitemize [1]
\item Respectați poziția corectă a curelei pe disc.
\item Porniți motorul și verificați modul de rulare al curelei.
\stopitemize
\stopPictPar
\stoptextbackground


\subsubsection [sSec:belt:tens] {Înlocuirea elementului de menținere a tensiunii}

{\sla Doar pentru versiunea cu compresor de climă}

\blank [medium]

\placefig [margin] [fig:belt:tens:change] {Înlocuirea elementului de menținere a tensiunii}
{\externalfigure[vw:belt:tens:change]
\noteF
\startLeg
\item Element de menținere a tensiunii
\item Șurub de siguranță
\stopLeg

{\bf Moment de strângere}

Șurub de siguranță:

\TorqueR 20 Nm\:+ ½ rotație (180°).}

\startSteps
\item Demontați cureaua de transmisie în modul descris (a se vedea \atpage[sSec:belt:change]).
\item Demontați componentele periferice (în funcție de echipare).
\item Deșurubați șurubul de siguranță (\in{\Ltwo, fig.}[fig:belt:tens:change]).
\stopSteps

Montajul elementului de menținere a tensiunii se realizează în ordinea inversă.

\starttextbackground [FC]
\startPictPar
\PMrtfm
\PictPar
\startitemize [1]
\item După montaj, utilizați neapărat un șurub de siguranță nou.
\item Momentul de strângere: A se vedea \in{fig.}[fig:belt:tens:change].
\stopitemize
\stopPictPar
\stoptextbackground

\stopregister[index][reg:main:vw]

\stopsection

\page[yes]


\setups[pagestyle:marginless]


\startsection[title={Sistem hidraulic},
reference={sec:hydraulic}]

\starttextbackground [FC]
% \startfiguretext[left,none]{}
% {\externalfigure[toni_melangeur][width=30mm]}

\startSymPar
\externalfigure[toni_melangeur][width=4em]
\SymPar
\textDescrHead{Reciclarea combustibililor}
Lubrifianții și combustibilii uzați nu pot fi eliminați sau incinerați în natură.

Lubrifianții uzați nu trebuie evacuați în sistemele de canalizare și nu pot fi eliminați în gunoiul menajer.

Lubrifianții uzați nu trebuie amestecați cu alte fluide, pentru că există pericolul să se formeze substanțe toxice sau alte produse greu de eliminat.
\stopSymPar
\stoptextbackground
\blank [big]

% \starthangaround{\PMgeneric}
% \textDescrHead{Qualification du personnel}
% Toute intervention sur l’installation hydraulique de votre véhicule ne peut être réalisée que par une personne dument qualifiée, ou par un service reconnu par \boschung.
% \stophangaround
% \blank[big]

\startSymList
\PHgeneric
\SymList
\textDescrHead{Curățenia} Sistemul hidraulic reacționează foarte repede la impuritățile din ulei. De aceea, este foarte important să se lucreze într-un mediu salubru.
\stopSymList

\startSymList
\PHhot
\SymList
\textDescrHead{Pericol de stropire}
Înainte de lucrările la sistemul hidraulic al \sdeux\ trebuie eliminată presiunea reziduală din fiecare circuit hidraulic. Picăturile fierbinți de ulei pot provoca leziuni.
\stopSymList

\startSymList
\PHhand
\SymList
\textDescrHead{Pericol de strivire}
Recipientul pentru gunoi trebuie neapărat coborât sau asigurat în mod mecanic cu ajutorul ștuțul de siguranță înainte de începerea lucrărilor la sistemul hidraulic al \sdeux.
\stopSymList

\startSymList
\PImano
\SymList
\textDescrHead{Măsurarea presiunii}
Pentru măsurarea presiunii uleiului hidraulic, montați un manometru la una dintre {\em conexiunile}
de măsurare ale circuitului. Aveți grijă ca manometrul să prezinte un interval de măsurare corespunzător.
\stopSymList

\page [yes]

\setups[pagestyle:normal]

\subsection{Intervale de revizie}

\start

\setupTABLE [frame=off,
style={\ssx\setupinterlinespace[line=.93\lH]},
background=color,
option=stretch,
split=repeat]
\setupTABLE [r] [each] [
topframe=on,
framecolor=white,
backgroundcolor=TableLight,
% rulethickness=.8pt,
]

% \setupTABLE [c] [odd] [backgroundcolor=TableMiddle]
% \setupTABLE [c] [even] [backgroundcolor=TableLight]
\setupTABLE [c] [1][ % width=30mm,
style={\bfx\setupinterlinespace[line=.93\lH]},
]
\setupTABLE [r] [first] [topframe=off,
style={\bfx\setupinterlinespace[line=.93\lH]},
backgroundcolor=TableMiddle,
]
% \setupTABLE [r] [2][style={\ssBfx\setupinterlinespace[line=.93\lH]}]


\bTABLE

\bTABLEhead
\bTR\bTD Intervale \eTD\bTD lucrări de revizie \eTD\eTR
\eTABLEhead

\bTABLEbody
\bTR\bTD Verificare scurgeri \eTD\bTD zilnic \eTD\eTR
\bTR\bTD Verificare nivel ulei hidraulic \eTD\bTD zilnic \eTD\eTR
\bTR\bTD Verificarea stării țevilor/furtunurilor hidraulice; dacă este cazul înlocuirea acestora \eTD\bTD 600 h / 12 luni \eTD\eTR
\bTR\bTD Înlocuire filtru retur ulei hidraulic și filtru de aspirație \eTD\bTD 600 h / 12 luni \eTD\eTR
\bTR\bTD Ungeți miezul bobinei ventilului electromagnetic cu vaselină pe bază de cupru \eTD\bTD 600 h / 12 luni \eTD\eTR
\bTR\bTD Schimb ulei hidraulic \eTD\bTD 1200 h / 24 luni \eTD\eTR
\eTABLEbody
\eTABLE
\stop


\subsection[niveau_hydrau]{Nivel umplere}

\placefig[margin][fig:hydraulic:level]{Nivel umplere fluide hidraulice}
{\externalfigure[hydraulic:level]
\noteF
\startLeg
\item Nivel optim de umplere
\stopLeg}

Un geam de control transparent\index{Nivel umplere+fluide hidraulice}\index{Revizie+sistem hidraulic} facilitează verificarea nivelului de umplere cu ulei hidraulic.
În momentul în care nivelul uleiului hidraulic a scăzut trebuie stabilită cauza, înainte ca nivelul să fie recompletat. Respectați intervalele de schimb prescrise (tabelul de mai sus) și specificațiile pentru fluidul hidraulic (Tabelul \at{pagina}[sec:liqquantities]).


\subsubsection{Completare nivel fluid hidraulic}

Completați nivelul de umplere cu fluid hidraulic până când geamul de control din mijloc este complet acoperit.
Porniți motorul și, dacă este cazul, completați cu fluid hidraulic până când este atins nivelul necesar de umplere.


\subsection{Schimb lichid hidraulic}

Cantitatea de umplere și specificațiile necesare pentru fluidul hidraulic sunt disponibile în tabelul \at{de la pagina}[sec:liqquantities].

\startSteps
\item Deschideți orificiul de alimentare al rezervorului hidraulic.
\item Goliți rezervorul cu ajutorul unui aspirator-recuperator de ulei sau îndepărtați șurubul de golire.

Șurubul de golire se află în spatele rezervorului hidraulic, în fața roții anterioare din partea stângă (\in{fig.}[fig:hydraulic:fluidDrain]).
\item Alimentați cu fluid hidraulic până când geamul de control din mijloc este complet acoperit.
Porniți motorul și, dacă este cazul, completați cu fluid hidraulic până când este atins nivelul necesar de umplere.
\stopSteps

\placefig[margin][fig:hydraulic:fluidDrain]{Șurub de golire}
{\externalfigure[hydraulic:fluidDrain]}


\placefig[margin][fig:hydraulic:returnFilter]{Filtru hidraulic}
{\externalfigure[hydraulic:returnFilter]}

\subsection[filtres:nettoyage]{Filtru de retur și filtru de aspirație}

\startSteps
\item Ridicați recipientul de gunoi și fixați ștuțul de siguranță.
\item Scoateți capacul filtrului de la rezervorul hidraulic (\in{fig.}[fig:hydraulic:returnFilter]).
\item Înlocuiți\index{Filtru+ulei hidraulic} elementul filtrant cu unul nou.
\item Umeziți o garnitură nouă de tip O-Ring cu un pic de fluid hidraulic și montați-o.
\item Înșurubați la loc capacul (\TorqueR cca. 20 Nm).
\stopSteps

\page [yes]


\subsection[sec:solenoid]{Ungeți ventilul electromagnetic}

\placefig[margin][graissage_bobine]{Ungeți ventilul electromagnetic}
{\externalfigure[graissage_bobine][M]
\noteF
\startLeg
\item Bobina ventilului electromagnetic
\item Miezul bobinei
\stopLeg}

Umezeala și reziduurile de sare, care ajung în miezul bobinei electromagnetice, duc la corodarea miezului. Miezurile bobinei trebuie unse o dată pe an cu vaselină pe bază de cupru. Vaselina trebuie să fie rezistentă la coroziune, apă și la temperaturi de până la 50 °C:
\startSteps
\item Demontați bobina ventilului electromagnetic (\in{\Lone, fig.}[graissage_bobine]).
\item Ungeți miezul (\in{\Ltwo, fig.}[graissage_bobine]) cu vaselina specială specificată și montați la loc bobina.
\stopSteps


\subsection{Înlocuirea furtunurilor}

Cauciucul de protecție\index{Intervale înlocuire+furtunuri} și materialul textil de întărire al furtunurilor sunt supuse unei uzuri naturale. Din acest motiv, furtunurile sistemului hidraulic trebuie neapărat înlocuite la intervalele prescrise, chiar dacă nu există {\em nici o} deteriorare vizibilă.

Asigurați-vă că furtunurile sunt fixate corect în autovehicul pentru a evita o uzură timpurie ca urmare a procesului de frecare. Acestea trebuie poziționate la o distanță suficientă de celelalte componente, astfel încât să fie evitate deteriorările cauzate în urma frecării și a vibrațiilor.

\stopsection

\page [yes]

\setups [pagestyle:bigmargin]


\startsection[title={Sistem de frânare},
reference={sec:brake}]

\placefig[margin][fig:brake:rear]{Frână cu tambur}
{\startcombination [1*2]
{\externalfigure[brake:wheelHub]}{\slx Butucul roții din spate}
{\externalfigure[brake:drum]}{\slx Mecanism și garnituri de frână}
\stopcombination}

Tamburii de frână \Lfour\ trebuie demontați la fiecare revizie regulată, mecanismul de frânare \Lseven trebuie curățat și garniturile \Lfive, \Lsix\ trebuie supuse unui control vizual (\in{fig.}[fig:brake:rear]).


\subsubject {Demontare}

\startSteps
\item Puneți autovehiculul pe o platformă adecvată și ridicați roțile.
\item Demontați roțile.
\stopSteps


{\sla Demontare frâne roți față}

\startSteps [continue]
\item Demontați tamburul de frână \Lfour.
\stopSteps

{\sla Demontare frâne roți spate}

\startSteps [continue]
\item Scoateți capacul de protecție \Lone de la butuc.
\item Îndepărtați șurubul \Ltwo și scoateți bailagul.
\item Deșurubați piulița butucului \Lthree folosind o șurubelniță cu clichet.
\item Scoateți butucul cu tamburul de frână.
\stopSteps


\subsubject {Remontarea}

Montați la loc tamburul de frână în ordinea inversă. Strângeți piulițele butucului roții din spate \Lthree cu momentul de strângere prescris de 190\,Nm.

\stopsection

\page [yes]

\setups [pagestyle:normal]


\startsection[title={Verificarea și întreținerea anvelopelor},
reference={sec:pneumatiques}]

Anvelopele\index{Întreținerea+anvelopelor} trebuie să fie permanent într-o stare corespunzătoare, pentru a-și putea îndeplini cele două funcții de bază: o bună aderență și un comportament de frânare adecvat. Uzura deosebit de ridicată și presiunea greșită, mai ales presiunea prea scăzută, sunt factori importanți care pot cauza producerea de accidente.


\subsection{Puncte relevante pentru siguranță}

\subsubsection{Verificarea uzurii}

Uzura anvelopelor trebuie verificată cu ajutorul indicatorilor de uzură, care se află într-o canelură a benzii de rulare (\in{fig.}[pneususure]).
Deficiențele la nivelul anvelopelor și cauzele acestora pot fi stabilite cu ajutorul unei verificări vizuale:

\placefig[margin][pneususure]{Verificarea uzurii}
{\Framed{\externalfigure[pneusUsure][M]}}

\placefig[margin][pneusdomages]{Anvelope deteriorate}
{\Framed{\externalfigure[pneusDomages][M]}}

\startitemize
\item Uzură pe părțile laterale ale suprafeței de rulare: Presiune prea scăzută.
\item Uzură accentuată în mijloc: Presiunea prea ridicată.
\item Uzură asimetrică pe părțile laterale ale anvelopelor: Axa frontală (ecartament, geometria axelor) reglată greșit.
\item Fisuri în suprafața de rulare: Anvelope prea vechi; cu timpul, cauciucul anvelopelor devine mai dur și mai fisurabil (\in{fig.}[pneusdomages]).
\stopitemize

\starttextbackground[CB]
\startPictPar
\PHgeneric
\PictPar
\textDescrHead{Riscuri din cauza anvelopelor uzate}
Anvelopele uzate nu își mai duc la îndeplinire funcțiile, mai ales în ceea ce privește devierea apei și a nămolului; calea de rulare se prelungește, iar comportamentul de mers se înrăutățește. Anvelopele uzate alunecă mai ușor, mai ales în caz de umiditate. Riscul ca anvelopele să își piardă aderența crește.
\stopPictPar
\stoptextbackground


\subsubsection{Presiunea din anvelope}

Presiunea prescrisă din anvelope este notată pe plăcuța de identificare pentru roți, din partea frontală de pe consolă, dinspre partea pasagerului (a se vedea \atpage [sec:plateWheel]).

Chiar dacă\index{Anvelope+Presiune} anvelopele se află într-o stare bună, odată cu trecerea timpului acestea încep să piardă mai mult sau mai puțin aer din ce în ce mai repede (cu cât autovehiculul este mai utilizat, cu atât mai pare este pierderea de presiune). De aceea, presiunea din anvelope trebuie verificată lunar, atunci când anvelopele nu sunt încălzite. În cazul în care verificați presiunea atunci când anvelopele sunt încălzite, trebuie să adăugați 0,3 bar la presiunea prescrisă.

\start
\setupcombinations[M]
\placefig[margin][pneuspression]{Presiunea din anvelope}
{\Framed{\externalfigure[pneusPression][M]}
\noteF
\startLeg
\item Presiune corectă
\item Presiune prea ridicată
\item Presiune prea scăzută
\stopLeg
Presiunea prescrisă din anvelope este precizată pe plăcuța de identificare a roților, din cabina șoferului, pe partea pasagerului.}
\stop

\starttextbackground[CB]
\startPictPar
\PHgeneric
\PictPar
\textDescrHead{Riscuri din cauza presiunii prea scăzute din anvelope}
O anvelopă se poate fisura atunci când presiunea este prea scăzută. Anvelopele se pot comprima mai mult atunci când nu este umflat suficient sau când autovehiculul este supra-încărcat. Cauciucul se supra-încălzește și există riscul ca într-o curbă, componentele anvelopei să se desprindă.
\stopPictPar
\stoptextbackground

\stopsection

\page [yes]

\setups[pagestyle:marginless]


\startsection[title={Șasiu},
reference={main:chassis}]

\subsection{Sisteme de fixare a componentelor}

{\em Relevante din punctul de vedere al siguranței}

La fiecare revizie, trebuie verificată poziția corectă a șuruburilor de fixare a anumitor componente relevante din punctul de vedere al siguranței, inclusiv verificarea momentelor de strângere prescrise. Această situație este valabilă mai ales pentru sistemul de direcție articulată și pentru axe.

\blank [big]

\startfigtext [left] [fig:frontAxle:fixing] {Axa frontală}
{\externalfigure [frontAxle:fixing]}
{\sla Dispozitivele de fixare ale axei frontale}
\startLeg
\item Fixarea foii de arc: \TorqueR 150 Nm
\item Fixarea unităților de tractare: \TorqueR 78 Nm
\stopLeg

{\sla Dispozitivele de fixare ale axei posterioare}
\startLeg
\item Fixarea foii de arc: \TorqueR 150 Nm
\stopLeg

\stopfigtext

\start

\setupTABLE [frame=off,style={\ssx\setupinterlinespace[line=.93\lH]},background=color, option=stretch, split=repeat]

\setupTABLE [r] [each] [topframe=on,
framecolor=white,
% rulethickness=.8pt
]

\setupTABLE [c] [odd] [backgroundcolor=TableMiddle]
\setupTABLE [c] [even] [backgroundcolor=TableLight]
\setupTABLE [c] [1][style={\bfx\setupinterlinespace[line=.93\lH]}]
\setupTABLE [r] [first] [topframe=off,style={\bfx\setupinterlinespace[line=.93\lH]},
]
% \setupTABLE [r] [2][style={\bfx\setupinterlinespace[line=.93\lH]}]


\bTABLE

\bTABLEhead
\bTR [backgroundcolor=TableDark] \bTD [nc=3] Momente de strângere \eTD\eTR
% \bTR\bTD Position \eTD\bTD Type de vis \eTD\bTD Couple \eTD\eTR
\eTABLEhead

\bTABLEbody
\bTR\bTD Motoare de tracțiune stânga/dreapta \eTD\bTD M12\:×\:35 8.8 \eTD\bTD 78 Nm \eTD\eTR
%% NOTE @Andrew: das sind Hydraulikmotoren
\bTR\bTD Pompă de lucru \eTD\bTD M16\:×\:40 100 \eTD\bTD 330 Nm \eTD\eTR
\bTR\bTD Pompă de acționare \eTD\bTD M12\:×\:40 100 \eTD\bTD 130 Nm \eTD\eTR
\bTR\bTD Foi de arc față/spate \eTD\bTD M16\:×\:90/160 8.8 \eTD\bTD 150 Nm \eTD\eTR
% \bTR\bTD Fixation du système oscillant \eTD\bTD M12\:×\:40 8.8 \eTD\bTD 78 Nm \eTD\eTR
\bTR\bTD Fixarea recipientului de gunoi \eTD\bTD M10\:×\:30 Verbus Ripp 100 \eTD\bTD 80 Nm \eTD\eTR
\bTR\bTD Piulițe roți \eTD\bTD M14\:×\:1,5 \eTD\bTD 180 Nm \eTD\eTR
\bTR\bTD Fixarea măturii frontale \eTD\bTD M16\:×\:40 100 \eTD\bTD 180 Nm \eTD\eTR
\eTABLEbody
\eTABLE
\stop


\stopsection

\page [yes]


\startmode [main:centralLubrication]

\startsection[title={Sistem de gresare centralizată},
reference={main:graissageCentral}]


\subsection{Descrierea modulului de comandă}

\sdeux\ poate fi echipată cu\index{Sistem de gresare centralizată} un sistem de gresare centralizată (opțional). Sistemul de gresare centralizată alimentează cu lubrifiant, la intervale regulate de timp, fiecare punct de ungere al autovehiculului.

\startfigtext [left] [vogel_affichage] {Modul de afișare}
{\externalfigure[vogel_base2][W50]}
\blank
\startLeg
\item Display cu 7 poziții: Valori și stare de funcționare
\item \LED: Sistem în pauză (mod de funcționare Standby)
\item \LED: Pompă în funcțiune
\item \LED: Mod de comandă a sistemului prin intermediul temporizatorului ciclic
\item \LED: Supravegherea sistemului prin intermediul întrerupătorului manometric
\item \LED: Mesaj eroare
\item Taste ecran:
\startLeg [R]
\item Activare display
\item Afișare valori
\item Modificare valori
\stopLeg
\item Tastă pentru schimbarea modului de funcționare; confirmarea valorilor
\item Declanșarea unui ciclu de gresare intermediară
\stopLeg
\stopfigtext

Sistemul de gresare centralizată cuprinde pompa de lubrifiant, recipientul transparent pentru lubrifiant de pe partea stângă a șasiului și modulul de comandă din sistemul electronic central.
% \blank
\page [yes]


\subsubsubject{Afișajul și tastele modului de comandă}

\start

\setupTABLE [frame=off,style={\ssx\setupinterlinespace[line=.93\lH]},background=color, option=stretch, split=repeat]

\setupTABLE [r] [each] [topframe=on,
framecolor=white,
% rulethickness=.8pt
]

\setupTABLE [c] [odd] [backgroundcolor=TableMiddle]
\setupTABLE [c] [even] [backgroundcolor=TableLight]
\setupTABLE [c] [1][width=9mm,style={\bfx\setupinterlinespace[line=.93\lH]}]
\setupTABLE [r] [first] [topframe=off,style={\bfx\setupinterlinespace[line=.93\lH]},
]
% \setupTABLE [r] [2][style={\bfx\setupinterlinespace[line=.93\lH]}]


\bTABLE
\bTABLEhead
% \bTR [backgroundcolor=TableDark]
% \bTD [nc=4] Anzeige und Tasten des Steuermoduls \eTD\eTR
\bTR\bTD Pos. \eTD
\bTD \LED \eTD\bTD Mod de afișaj \eTD
\bTD Mod de programare \eTD\eTR
\eTABLEhead

\bTABLEbody
\bTR\bTD 2 \eTD
\bTD Stare de funcționare {\em pauză}\hskip.5em\null \eTD
\bTD Sistemul se află în modul Standby\hskip.5em\null \eTD %
\bTD Intervalul de pauză poate fi modificat \eTD\eTR
\bTR\bTD 3 \eTD
\bTD Stare de funcționare {\em Contact} \eTD
\bTD Pompa lucrează \eTD
\bTD Timpul de lucru poate fi modificat \eTD\eTR
\bTR\bTD 4 \eTD
\bTD Control sistem {\em CS} \eTD
\bTD Cu temporizatorul extern \eTD
\bTD Modul de control poate fi dezactivat sau modificat \eTD\eTR
\bTR\bTD 5 \eTD
\bTD Control sistem {\em PS} \eTD
\bTD Cu întrerupător manometric extern \eTD
\bTD Modul de control poate fi dezactivat sau modificat \eTD\eTR
\bTR\bTD 6 \eTD
\bTD Defecțiune {\em Fault} \eTD
\bTD [nc=2] Există o disfuncționalitate. Cauza este afișată sub forma unui cod de eroare, după care se apasă tasta \textSymb{vogel_DK}. Execuția funcțiilor este întreruptă. \eTD\eTR
\bTR\bTD 7 \eTD
\bTD Taste cu săgeți \textSymb{vogelTop} \textSymb{vogelBottom} \eTD
\bTD [nc=2] \items[symbol=R]{Activarea displayului, verificarea parametrilor (mod de afișaj),reglarea valorii afișate (I) (mod de programare)}
\eTD\eTR
\bTR\bTD 8 \eTD
\bTD Tasta \textSymb{vogelSet} \eTD
\bTD [nc=2] Comutare între modul de afișaj și programare sau confirmarea valorilor inserate. \eTD\eTR
\bTR\bTD 9 \eTD
\bTD Tasta \textSymb{vogel_DK} \eTD
\bTD [nc=2] Atunci când aparatul se află în starea {\em Pauză}, prin acționarea tastei se declanșează ciclul intermediar de gresare. Mesajele de erori sunt confirmate și șterse. \eTD\eTR
\eTABLEbody
\eTABLE
\stop
\vfill

\startfigtext [left] [vogel_touches]{Mod de afișare}
{\externalfigure[vogel_base][width=50mm]}
\textDescrHead{Mod de afișare} Apăsați scurt tastele cu săgeți \textSymb{vogelTop} \textSymb{vogelBottom}, pentru a activa displayul cu 7 poziții \textSymb{led_huit}. Prin reapăsarea tastei \textSymb{vogelTop} pot fi afișați diferiți parametri urmați de valorile lor. Modul {\em Afișaj} poate fi recunoscut după \LED\char"2060s permanent aprins (\in{2 până la 6, fig.}[vogel_affichage]).
\blank [medium]
\textDescrHead{Mod de programare} Pentru modificarea valorilor, apăsați tasta \textSymb{vogelSet} timp de două secunde, pentru a comuta în modul {\em Programare}: \LED\char"2060s se aprind. Apăsați scurt tasta \textSymb{vogelSet}, pentru\index{Sistem de gresare centralizată+Programare} a modifica modul de afișaj, apoi modificați valorile dorite cu ajutorul tastelor \textSymb{vogelTop} \textSymb{vogelBottom}. Confirmați cu\index{Sistem de gresare centralizată+afișaj} tasta \textSymb{vogelSet}.
\stopfigtext

\page [yes]


\subsection{Sub-meniuri în modul {\em Afișaj}}

\vskip -9pt

\adaptlayout [height=+5mm]

\startcolumns[balance=no]\stdfontsemicn

\startSymVogel
\externalfigure[vogel_tpa][width=26mm]
\SymVogel
\textDescrHead{Perioadă de pauză [h]} Apăsați tasta \textSymb{vogelTop}, pentru a afișa valorile programate.
\stopSymVogel

\startSymVogel
\externalfigure[vogel_068][width=26mm]
\SymVogel
\textDescrHead{Timp de pauză rămas [h]} Timpul încă rămas până la următorul ciclu de gresare.
\stopSymVogel

\startSymVogel
\externalfigure[vogel_090][width=26mm]
\SymVogel
\textDescrHead{Timp de pauză total [h]} Timpul de pauză total între două cicluri.
\stopSymVogel

\startSymVogel
\externalfigure[vogel_tco][width=26mm]
\SymVogel
\textDescrHead{Timp de gresare [min]} Apăsați \textSymb{vogelTop}, pentru a afișa valorile programate.
\stopSymVogel

\startSymVogel
\externalfigure[vogel_tirets][width=26mm]
\SymVogel
\textDescrHead{Aparat în Standby} Afișajul nu este posibil pentru că aparatul este în Standby (pauză).
\stopSymVogel

\startSymVogel
\externalfigure[vogel_026][width=26mm]
\SymVogel
\textDescrHead{Timp de gresare [min]} Durata unei operațiuni de gresare.
\stopSymVogel

\startSymVogel
\externalfigure[vogel_cop][width=26mm]
\SymVogel
\textDescrHead{Controlul sistemului} Apăsați \textSymb{vogelTop}, pentru a afișa valorile programate.
\stopSymVogel

\startSymVogel
\externalfigure[vogel_off][width=26mm]
\SymVogel
\textDescrHead{Mod control} \hfill PS: Întrerupător manometric;\crlf CS: Temporizator; OFF: dezactivat.
\stopSymVogel

\startSymVogel
\externalfigure[vogel_0h][width=26mm]
\SymVogel
\textDescrHead{Ore de funcționare} Apăsați \textSymb{vogelTop}, pentru a afișa valorile în două faze.
\stopSymVogel

\startSymVogel
\externalfigure[vogel_005][width=26mm]
\SymVogel
\textDescrHead{Partea 1: 005} Timpul de funcționare este afișat în două părți; la partea 2 cu tasta \textSymb{vogelTop}.
\stopSymVogel

\startSymVogel
\externalfigure[vogel_338][width=26mm]
\SymVogel
\textDescrHead{Partea 2: 33,8} Partea a 2 a a cifrei este 33,8; rezultă un timp de funcționare de 533,8 h.
\stopSymVogel

\startSymVogel
\externalfigure[vogel_fh][width=26mm]
\SymVogel
\textDescrHead{Timp de eroare} Apăsați \textSymb{vogelTop}, pentru a afișa valorile în două faze.
\stopSymVogel

\startSymVogel
\externalfigure[vogel_000][width=26mm]
\SymVogel
\textDescrHead{Partea 1: 000} Timpul de eroare este afișat în două părți; \crlf la partea 2 cu tasta \textSymb{vogelTop}.
\stopSymVogel

\startSymVogel
\externalfigure[vogel_338][width=26mm]
\SymVogel
\textDescrHead{Partea 2: 33,8} Partea a 2 a a cifrei este 33,8; rezultă un timp de eroare de 33,8 h.
\stopSymVogel

\stopcolumns

\stopsection


\page [yes]

\stopmode % central lubrication

\setups [pagestyle:marginless]


\startsection[title={Schemă de ungere pentru gresarea manuală},
reference={sec:grasing:plan}]

\starttextbackground [FC]
\startPictPar
\PMgeneric
\PictPar
Punctele de ungere (\in{fig.}[fig:greasing:plan]) precizate în schema de ungere trebuie gresate la intervale regulate de timp. O gresare la intervale regulate de timp este indispensabilă, pentru a garanta pe termen lung {\em reducerea riscului de frecare} și pentru a ține la distanță umiditatea și alte substanțe corozive.
\stopPictPar
\stoptextbackground

\blank [big]

\start

\setupcombinations [width=\textwidth]

\placefig[here][fig:greasing:plan]{Schema de ungere a autovehiculului}
{\startcombination [3*1]
{\externalfigure[frame:steering:greasing]}{\ssx Direcția articulată și mecanismul de pendulare}
{\externalfigure[frame:axles:greasing]}{\ssx Axe}
{\externalfigure[frame:sucMouth:greasing]}{\ssx Gură de aspirație}
\stopcombination}

\stop

\vfill

\startLeg [columns,three]
\item Cilindru ridicător al direcției articulate\crlf {\sl 2 nipluri de la fiecare cilindru}
\item Lagăr direcție articulată\crlf {\sl 2 nipluri de ungere pe partea stângă}
\columnbreak
\item Lagăr mecanism de pendulare\crlf {\sl 1 niplu de ungere în fața rezervorului}
\item Arcuri cu foi\crlf {\sl 2 nipluri de ungere pentru fiecare foaie de arc}
\columnbreak
\item Gură de aspirație\crlf {\sl 1 niplu de ungere pentru fiecare roată}
\item Gură de aspirație\crlf {\sl 1 niplu de ungere pe brațul extensibil}
\stopLeg



\page [yes]


\setups [pagestyle:bigmargin]


\subsubject{Gresarea recipientului pentru gunoi}

Recipientul pentru gunoi dispune de 6 puncte de ungere (2\:×\:4), care trebuie unse săptămânal.

\blank [big]


\placefig[here][fig:greasing:container]{Mecanismul de ridicare a recipientului}
{\externalfigure[container:mechanisme]}


\placelegende [margin,none]{}
{{\sla Legendă:}

\startLeg
\item Lagărul stâng al recipientului (2\:×)
\item Lagărul drept al recipientului (2\:×)
\item Cilindru hidraulic stâng (sus)
\item Cilindru hidraulic stâng (jos)

{\em Ca în cazul cilindrului drept (punctul \in[greasing:point;hide]).}
\item Cilindru hidraulic drept (sus)
\item [greasing:point;hide] Cilindru hidraulic drept(jos)
\stopLeg}

\stopsection

\page [yes]



\startsection[title={Instalație electrică},
reference={sec:main:electric}]

\subsection{Instalație electrică centrală în șasiu}

\startbuffer [fuses:preventive]
\starttextbackground [CB]
\startPictPar
\PHvoltage
\PictPar
\textDescrHead{Cerințe privind siguranța}
Respectați cerințele în materie de siguranță\index{Siguranțe+Șasiu} menționate\index{Releuri+Șasiu} în acest manual: Înlocuiți întotdeauna siguranțele doar cu siguranțe cu numărul prescris de amperi; scoateți bijuteriile metalice înainte să lucrați la \index{Instalație electrică} instalația electrică (inele,brățări etc.).
\stopPictPar
\stoptextbackground
\stopbuffer


\subsubsubject{Siguranțe MIDI}

\starttabulate[|l|r|p|]
\HL
\NC\md F 1 \NC 5 A \NC lămpi de stop, {\em +\:15} OBD \NC\NR
\NC\md F 2 \NC 5 A \NC {\em +\:15} comandă motor \NC\NR
\NC\md F 3 \NC 7,5 A \NC {\em +\:30} comandă motor și OBD \NC\NR
\NC\md F 4 \NC 20 A \NC pompă carburant \NC\NR
\NC\md F 5 \NC 20 A \NC {\em D\:+} alternator, {\em +\:15} releu K 1 \NC\NR
\NC\md F 6 \NC 5 A \NC comandă motor \NC\NR
\NC\md F 7 \NC 10 A\NC tratarea gazelor de eșapament de la motor \NC\NR
\NC\md F 8 \NC 20 A \NC sistem electronic motor (comandă) \NC\NR
\NC\md F 9 \NC 15 A \NC tratarea gazelor de eșapament de la motor, pompă de carburant, preîncălzire \NC\NR
\NC\md F 10\NC 30 A \NC comandă motor \NC\NR
\NC\md F 11\NC 5 A \NC lumini pentru mersul înapoi \NC\NR
%% NOTE @Andrew: Singular
\HL
\stoptabulate

\placefig [margin] [fig:electric:power:rear] {Instalația electrică centrală în șasiu}
{\externalfigure [electric:power:rear]
\noteF
\startKleg
\sym{K 1} Sistem electronice de comandă motor
\sym{K 2} pompă de carburant
\sym{K 3} eliberarea demarorului
\sym{K 4} lămpi de stop
\sym{K 5} {[}Rezerve{]}
\sym{K 6} Lumini pentru mersul înapoi
\sym{K 7} Instalație de pre-încălzire
\stopKleg
}


\subsubsubject{Siguranțe MAXI}

% \startcolumns [n=2]
\starttabulate[|l|r|p|]
\HL
\NC\md F 15 \NC 50 A \NC Alimentare principală a instalației electrice centrale \NC\NR
\HL
\stoptabulate

\page [yes]

\setups[pagestyle:marginless]


\subsection{Instalația electrică centrală în cabina șoferului}

\startcolumns[rule=on]

\placefig [bottom] [fig:fuse:cab] {Siguranțe și releuri în cabina șoferului}
{\externalfigure [electric:power:front]}

\columnbreak

\subsubsubject{Releuri}

\vskip -12pt

\index{Siguranțe+cabina șoferului}\index{Releuri+cabina șpferului}

\starttabulate[|lB|p|]
\NC K 2\NC compresor de climă\NC\NR
\NC K 3\NC compresor de climă\NC\NR
\NC K 4\NC Pompă electrică de apă\NC\NR
\NC K 5\NC Girofar\NC\NR
\NC K 10 \NC Sursă de frecvență semnalizare\NC\NR
\NC K 11 \NC Lumină de fază scurtă\NC\NR
\NC K 12 \NC Lumină de fază lungă {[}Rezervă{]} \NC\NR
\NC K 13 \NC far de lucru\NC\NR
\NC K 14 \NC Temporizare ștergătoare\NC\NR
\stoptabulate

\vskip -24pt

\placefig [bottom] [fig:fuse:access] {Clapetă de acces la instalația electrică centrală}
{\externalfigure [electric:power:cabin]}

\stopcolumns

\page [yes]


\subsubsubject{Siguranțe MINI}

\startcolumns[rule=on]
% \setuptabulate[frame=on]
%\placetable[here][tab:fuses:cab]{Fusibles dans la cabine}
%{\noteF
\starttabulate[|lB|r|p|]
\NC F 1 \NC 3 A \NC Lumină de poziție stânga \NC\NR
\NC F 2 \NC 3 A \NC Lumină de poziție dreapta \NC\NR
\NC F 3 \NC 7,5 A \NC Lumină de fază scurtă stânga \NC\NR
\NC F 4 \NC 7,5 A \NC Lumină de fază scurtă dreapta \NC\NR
\NC F 5 \NC 7,5 A \NC Lumină de fază lungă stânga {[}Rezervă{]} \NC\NR
\NC F 6 \NC 7,5 A \NC Lumină de fază lungă dreapta {[}Rezervă{]} \NC\NR
\NC F 7 \NC 10 A \NC Far de lucru sus \NC\NR
%% NOTE @Andrew: Plural
\NC F 8 \NC 10 A \NC Far de lucru jos (Rezervă) \NC\NR
%% NOTE @Andrew: Plural
\NC F 9 \NC 10 A \NC Mătură frontală \NC\NR
\NC F 10 \NC 10 A \NC Ștergătoare de parbriz \NC\NR
\NC F 11 \NC 5 A \NC Întrerupător sistem de iluminat și instalație de iluminat intermitent de semnalizare \NC\NR
\NC F 12 \NC 5 A \NC {[}Rezervă{]} \NC\NR
\NC F 13 \NC 10 A \NC încălzire oglinzi exterioare \NC\NR
\NC F 14 \NC 7,5 A \NC {\em +\:15} Radio și cameră video \NC\NR
\NC F 15 \NC 10 A \NC {\em +\:30} instalație de iluminat intermitent de semnalizare \NC\NR
\NC F 16 \NC 5 A \NC Lumină coloană de direcție \NC\NR
\NC F 17 \NC 7,5 A \NC {\em +\:30} Radio, întrerupător lumină și sistem de iluminat intern \NC\NR
\NC F 18 \NC — \NC {[}liber{]} \NC\NR
\NC F 19 \NC 20 A \NC {\em +\:30} RC 12 față \NC\NR
\NC F 20 \NC 20 A \NC {\em +\:30} RC 12 spate \NC\NR
\NC F 21 \NC 15 A \NC priză 12-V \NC\NR
\NC F 22 \NC 5 A \NC cheie de contact, consolă multifuncțională, Vpad \NC\NR
\NC F 23 \NC 5 A \NC oprire de avarie, consolă centrală, RC 12 față \NC\NR
\NC F 24 \NC 5 A \NC oprire de avarie, consolă centrală, RC 12 spate \NC\NR
\NC F 25 \NC 2 A \NC {\em +\:15} RC 12 față \NC\NR
\NC F 26 \NC 2 A \NC {\em +\:15} RC 12 spate \NC\NR
\NC F 27 \NC 25 A \NC Suflantă de încălzire \NC\NR
\NC F 28 \NC 10 A \NC compresor de climă, instalație de gresare centralizată \NC\NR
\NC F 29 \NC 25 A \NC condensator de climă \NC\NR
\NC F 30 \NC 5 A \NC Termostat instalație de climatizare \NC\NR
\NC F 31 \NC 5 A \NC {\em +\:15} Consolă multifuncțională/Vpad \NC\NR
\NC F 32 \NC 15 A \NC pompă electrică de apă, girofar \NC\NR
\NC F 33 \NC — \NC {[}liber{]} \NC\NR
\NC F 34 \NC — \NC {[}liber{]} \NC\NR
\NC F 35 \NC — \NC {[}liber{]} \NC\NR
\NC F 36 \NC — \NC {[}liber{]} \NC\NR
\stoptabulate
\stopcolumns

\page [yes]

\setups [pagestyle:bigmargin]


\subsection[sec:lighting]{Sistem de iluminare și de semnalizare}


\placefig [here] [fig:lighting] {Sistemul de iluminare și de semnalizare al autovehiculului}
{\externalfigure [vhc:electric:lighting]}

\placelegende [margin,none]{}{%
\vskip 30pt
{\sla Legendă:}\noteF
\startLongleg
\item Lumină de poziție\hfill 12 V–5 W
\item Lumină de fază scurtă\hfill H7 12 V–55 W
\item Semnalizare\hfill orange 12 V–21 W
\item {\stdfontsemicn Far de lucru}\hfill G886 12 V–55 W
\item Indicator direcția de mers\hfill 12 V–21 W
\item Lumini de mers înapoi/lămpi de stop\hfill 12 V–5/21 W
\item Lumini de mers înapoi\hfill 12 V–21 W
\item {[}Liber{]}
\item Iluminat al numărului de înmatriculare\hfill 12 V–5 W
\item Girofar\hfill H1 12 V–55 W
\stopLongleg}

\subsubsubject{Reglarea farurilor}

\placefig [margin] [fig:lighting:adjustment] {Fascicul luminos 5 m}
{\externalfigure [vhc:lighting:adjustment]
\noteF
\startitemize
\sym{H\low{1}} înălțimea filamentului: 100 cm
\sym{H\low{2}} corectare la 2\hairspace\%: 10 cm
\stopitemize}

{\md Condiții:} Recipient apă= curată/apă de reciclare plin, șofer la volan.

Orientarea farurilor se realizează în prealabil în fabrică. Înălțimea și înclinația fasciculului de lumină pot fi reglate prin rotirea manetei din material plastic.

În cazul în care în timpul unei verificări se constată că reglarea trebuie modificată, slăbiți șurubul de siguranță și corectați înclinația astfel încât aceasta să corespundă prevederilor legale (a se vedea \in{fig.}[fig:lighting:adjustment]). Strângeți la loc șurubul de siguranță.

\page [yes]
\setups [pagestyle:marginless]


\subsection[sec:battcheck]{Acumulator}

\subsubsection{Cerințe privind siguranța}

\startSymList
\PPfire
\SymList
\textDescrHead{Pericol de explozie}
La încărcarea\index{Acumulator+Instrucțiuni de siguranță}\index{Pericol+explozie} acumulatorilor se formează gaz exploziv\index{Gaz detonant} detonant. Încărcaţi acumulatorii numai în spaţii bine ventilate! Evitaţi producerea de scântei! Nu folosiţi surse de flacără şi de lumină deschisă şi nu fumaţi în apropierea acumulatorului!
\stopSymList

\startSymList
\PHvoltage
\SymList
\textDescrHead{Pericol de scurtcircuit}
Dacă\index{Întreținere+acumulator} borna pozitivă a acumulatorului conectat intră în contact cu componentele autovehiculului, \index{Pericol+incendiu}\index{Pericol+explozie} există risc de producere a unui scurt circuit. În aces fel amestecul de gaze emis din acumulator poate exploda, iar dumneavoastră și alte persoane puteți suferi leziuni grave.

\startitemize
\item Nu amplasați obiecte metalice sau unelte de lucru pe acumulator.
\item La decuplarea acumulatorului, scoateți întotdeauna mai întâi borna negativă și apoi borna pozitivă.
\item La conectarea acumulatorului, cuplați întotdeauna borna pozitivă, apoi pe cea de negativă.
\item Nu desfaceți sau scoateți bornele de conexiune atunci când motorul este în funcțiune.
\stopitemize
\stopSymList


\startSymList
\PHcorrosive
\SymList
\textDescrHead{Pericol de coroziune cutanată}
Purtați\index{Pericol+de coroziune cutanată} ochelari de protecție și mănuși rezistente la acizi. Lichidul din baterie conține aproximativ 27\percent acid sulfuric (H\low{2}SO\low{4}) și de aceea să cauzeze coroziuni cutanate. Neutralizați\index{Acumulator+Pericol}\index{Lichid+de acumulator} lichidul din acumulator care a intrat în contact cu pielea, folosind o soluție din natron și curățați cu apă proaspătă. În cazul în care lichidul din baterie ajunge în ochi, spălați-vă ochii cu multă apă rece și consultați imediat un medic.
\stopSymList

\startSymList
\startcombination[1*2]
{\PHcorrosive}{}
{\PHfire}{}
\stopcombination
\SymList
\textDescrHead{Depozitarea acumulatorilor}
Depozitați întotdeauna\index{Depozitare+acumulatori} acumulatorii în mod corespunzător. În caz contrar lichidul din acumulatori se poate scurge și poate cauza coroziuni cutanate sau - în reacție cu alte substanțe- poate duce la producerea de incendii. \par\null\par\null
\stopSymList

\testpage [16]

\starttextbackground [FC]
\setupparagraphs [PictPar][1][width=2.4em,inner=\hfill]

\startPictPar
\PMproteyes
\PictPar
\textDescrHead{Ochelari de protecție}
Pe\index{Pericol+vătămare la nivelul ochilor} durata procesului de amestecare a acidului cu apa este posibil ca soluţia rezultată să intre în contact cu ochii.. Dacă acidul intră în contact cu ochii, spălaţi-vă ochii imediat cu apă curată şi apelaţi la un medic!
\stopPictPar
\blank [small]

\startPictPar
\PMrtfm
\PictPar
\textDescrHead{Documentație}
Înainte de manipularea acumulatorilor vă rugăm citiţi instrucţiunile de siguranţă, măsurile de protecţie şi modalităţile prezentate în acest ghid de utilizare.
\stopPictPar
\blank [small]

\startPictPar
\PStrash
\PictPar
\textDescrHead{Protejarea mediului înconjurător}
Acumulatorii\index{Protejarea mediului înconjurător} conțin substanțe toxice. Acumulatorii vechi trebuie eliminaţi separat de deşeurile casnice. Eliminaţi acumulatorii în mod ecologic. Predaţi acumulatorii la un atelier specializat sau la o unitate de preluare a acestora.

Acumulatorii trebuie transportaţi şi depozitaţi în mod corespunzător. În timpul transportului, acumulatorii trebuie asigurați împotriva răsturnării! Există pericolul scurgerii acizilor din orificiile de ventilare a dopurilor de închidere, ceea ce va cauza poluarea mediului înconjurător.
\stopPictPar
\stoptextbackground

\page [yes]

\setups[pagestyle:normal]


\subsubsection{Sfaturi practice}

Asiguraţi acumulatorilor o durată de viaţă maximă, încărcându-le permanent în mod corespunzător.

O\index{Durata de viață+a acumulatorilor} încărcare de întreținere a acumulatorului în cursul unor perioade mai lungi de staționare a autovehiculului prelungește nu doar durata de viață a autovehiculului, ci, în același timp, garantează o disponibilitate de utilizare.

\placefig[margin][fig:batterycompartment]{\select{caption}{Compartiment acumulator (clapetă de revizie)}{compartiment acumulator}}
{\externalfigure[batt:compartment]}


\subsubsection{Întreținere}

În cazul acumulatorului, \sdeux\ este vorba de un {\em acumulator cu plumb} care nu necesită întreținere. În afară de menținerea stării de încărcare și de curățare, acumulatorul nu necesită alte măsuri de întreținere.

\startitemize
\item Asigurați-vă că polul bateriei este curat și uscat. Curățați polul ușor cu o vaselină antiacidă.
\item Acumulatorii, care\index{Încărcare+acumulator} prezintă o tensiune de repaos mai mică de\index{Acumulator+tensiune în repaos} 12,4 V, trebuie reîncărcate.
\stopitemize

\placefig[margin][fig:bclean]{Curățarea polului}
{\externalfigure[batt:clean]
\noteF
Utilizați\index{Curățarea+acumulatorului}\index{Curățarea+acumulatorului} apă caldă pentru a îndepărta praful de culoare albă apărut prin coroziune. În cazul în care un pol este acoperit cu rugină, decuplați cablul acumulatorului și curățați polul cu o perie de sârmă. Aplicați peste pol o peliculă subțire de vaselină.}


\subsubsection[sec:battery:switch]{Utilizarea separatorului de la acumulator}

{\sl Nu este recomandabilă acționarea regulată a separatorului materiei (de exemplu zilnic)!}

\startSteps
\item Opriți\index{separator baterie} sistemul de aprindere și așteptați aproximativ un minut.
\item Deschideți compartimentul pentru acumulator (\inF[fig:batterycompartment]).
\item Apăsați pe butonul roșu al separatorului acumulatorului pentru a întrerupe circuitul electric.
\item Pentru a conecta din nou circuitul electric, rotiți separatorul într-o rotație de ¼ în sensul acelor de ceas.
\stopSteps

\stopsection

\page [yes]


\setups[pagestyle:marginless]

\section[sec:cleaning]{Curățarea autovehiculului}

Înainte de\startregister[index][vhc:lavage]{Întreținere+Curățenie} curățarea propriu zisă, îndepărtați nămolul și mizeria grosieră de pe caroserie folosind apă din abundență. Curățați nu doar suprafețele laterale, ci și carcasa roților și partea inferioară a autovehiculului.

Mai ales în timpul iernii, autovehiculul trebuie curățat temeinic, pentru a elimina \index{Coroziune+Prevenire} resturile de sare pentru deszăpezire.

\starttextbackground [FC]
\startPictPar
\PHgeneric
\PictPar
\textDescrHead{Evitați daunele produse din cauza apei}
Nu curățați autovehiculul niciodată cu ajutorul {\em tunurilor de apă} (\eG\ ale pompierilor) sau {\em produselor de curățare pe bază de hidrocarburi}. Atunci când lucrați cu un aparat de curățare cu aburi de înaltă presiuni respectați instrucțiunile aferente menționate în cele ce urmează.
\stopPictPar
\blank[small]

\startPictPar
\pTwo[monde]
\PictPar
\textDescrHead{Protejarea mediului înconjurător}
Curățarea unui autovehicul poate cauza poluarea mediului. Curățați autovehiculul doar într-un punct de lucru\index{Protejarea mediului} prevăzut cu separator de ulei. Respectați prevederile în vigoare privind protejarea mediului.
\stopPictPar
\blank[small]

\startPictPar
\PMwarranty
\PictPar
\textDescrHead{Curățați în mod corespunzător!}
Firma \BosFull{boschung} nu își asumă răspunderea și nici nu oferă vreo garanție pentru daunele apărute ca urmare a nerespectării prevederilor privind curățarea autovehiculului.
\stopPictPar
\stoptextbackground


\subsection{Aparate de curățat cu înaltă presiune}

Pentru curățarea la înaltă presiune a autovehiculului\index{Curățare+înaltă presiune} este recomandată utilizarea unui aparat obișnuit de curățare cu înaltă presiune.

La curățarea cu înaltă presiune, trebuie respectate următoarele puncte:

\startitemize
\item Presiune de lucru maximum 50\,bar
\item Duza din bandă de oţel în unghi de stropire de 25°
\item Distanță de stropire de minimum 80\,cm
\item Temperatura apei maximum 40\,°C
\item Țineți cont de instrucțiunile menționate în secțiune \about[reiMi], \atpage[reiMi].
\stopitemize

În cazul nerespectării\index{Deteriorea+stratului de vopsea} acestor instrucțiuni se poate ajunge la deterioarea stratului de vopsea și a protecției\index{Deterioarea+stratului de vopsea} anticorozive.

Respectați și instrucțiunile de utilizare și dispozițiile privind siguranța ale aparatului de curățat cu înaltă presiune.

\starttextbackground[FC]
\startPictPar\PPspray\PictPar
În cazul curățării cu înaltă presiune, apa poate pătrunde în anumite locuri unde poate cauza defecțiuni. Nu orientaţi jetul de apă spre părţile sensibile şi spre echipamente, în special spre:
\stopPictPar

\startitemize
\item Senzori, conexiuni şi racorduri electrice
\item Demaratoare, alternatoare, sistem de injecție
\item Ventile electromagnetice
\item Orificii de ventilație
\item Componente mecanice încă încălzite
\item Etichetele cu mesaje informative, de avertizare şi de semnalizare a pericolelor
\item Aparate electronice de comandă
\stopitemize

\textDescrHead{Curățarea motorului}
Evitați neapărat intrarea apei în orificiile de aspirație, aerisire și evacuare a aerului uzat. În cazul curățării la înaltă presiune, nu îndepărtați jetul direct spre componentele și cablurile electrice. Nu îndreptați jetul direct spre instalația de injecție! După curățarea motorului, conservați motorului; în acest caz, protejați cureaua de transmisie împotriva produsului de conservare.
\stoptextbackground

\starttextbackground [FC]
\setupparagraphs [PictPar][1][width=6em,inner=\hfill]
\startPictPar
\framed[frame=off,offset=none]{\PMproteyes\PMprotears}
\PictPar
\textDescrHead{Apă reziduală}
În timpul operațiunilor de curățare, în anumite locuri ale autovehiculului se formează acumulări de apă (\eG\ în compartimentului blocului motor sau al cutiei de viteze); aceasta trebuie îndepărtată cu ajutorul aerului comprimat. Asigurați-vă că în cazul utilizării aerului comprimat, purtați echipament de protecție adecvat, iar instalație respectă normele în vigoare în materie de siguranță (Multi-duză).
\stopPictPar
\stoptextbackground


\subsubsection[reiMi]{Produse adecvate de curățare}

Utilizați\index{Produse de curățare} exclusiv produse de curățare care prezintă următoarele caracteristici:

\startitemize
\item Fără efect de fracțiune
\item Valoare PH de 6–7
\item Fără dizolvanți
\stopitemize

Pentru eliminarea petelor dure de pe suprafeţele lăcuite folosiţi cu atenţie benzină de extracţie sau alcool sanitar, dar în nici un caz solvenţi. Îndepărtaţi urmele de solvenţi de pe suprafeţele lăcuite. Curățarea cu benzină a pieselor din material plastic poate duce la fisuri și modificări de culoare!

Curățați suprafețele cu\index{Curățare+etichete} etichete de avertizare și cu mesaje informative cu apă curată, folosind un burete moale.

Evitați pătrunderea apei în componentele electrice: Nu utilizați perii electrice pentru curățarea carcasei sistemului de iluminare intermitentă sau a lămpilor, ci un burete sau un material textil moale.

\starttextbackground [CB]
\startPictPar
\GHSgeneric\par
\GHSfire
\PictPar
\textDescrHead{Pericol din cauza substanțelor chimice}
Produsele de curățare pot reprezenta un risc pentru sănătate și siguranță (substanțe ușor inflamabile) Respectați prevederile de siguranță în vigoare pentru produsele de curățare utilizate; respectați fișele tehnice și cu date de siguranță ale produselor utilizate.
\stopPictPar
\stoptextbackground

\stopregister[index][vhc:lavage]


\page [yes]


\setups [pagestyle:bigmargin]

\startsection [title={Reglarea gurii de aspirație},
reference={sec:main:suctionMouth}]


Distanța optimă dintre\index{Gură de aspirație+Reglare} suprafața părții carosabile și șina din material plastic a gurii de aspirației este de 8 mm. Pentru a controla sau regla distanța, utilizați cele trei șabloane/calibre de reglare, pe care le găsiți în cutia de unele (cabina șoferului, partea șoferului).


\placefig [margin] [fig:suctionMouth] {Reglarea gurii de aspirație}
{\Framed{\externalfigure [suctionMouth:adjust]}}

\placeNote[][service_picto]{}{%
\noteF
\starttextrule{\PHasphyxie\enskip Pericol de intoxicare și de asfixiere \enskip}
{\md Indicație:} În timpul lucrărilor de reglare, motorul autovehiculului trebuie să fie în funcțiune, pentru a putea menține gura de aspirație în poziție de plutire. Pentru a evita riscul de intoxicație sau asfixiere, trebuie utilizată obligatoriu o instalație de aspirație a gazelor de eșapament, respectiv lucrările trebuie executate doar într-un spațiu foarte bine aerisit.
\stoptextrule}

\startSteps
\item Poziționați autovehiculul într-un spațiu bine aerisit, pre o suprafață orizontală și plană.
\item Activați\index{Aspirarea} modul {\em de lucru} (apăsați butonul din partea exterioară manetei de selectare a vitezelor de mers).

Lăsați motorul să meargă în gol. (Apăsați tasta~\textSymb{joy_key_engine_decrease} de pe consola multifuncțională pentru a reduce turația motorului.)
\item Trageți frâna de staționare și asigurați roțile din spate cu câte o pană.
\item Apăsați tasta~\textSymb{joy_key_suction}, pentru a coborî gura de aspirație.
\item Poziționați cele trei calibre de reglare~\LAa\ sub șina de material plastic a gurii de aspirare, în modul prezentat în imagine.
\item [sucMouth:adjust]Slăbiți șuruburile de fixare~\Lone\ și de reglare~\Ltwo\ de la fiecare roată; cele patru roți coboară pe suprafața unde amplasat autovehiculul.
\item Strângeți la loc șuruburile~\Lone\ și ~\Ltwo\ și apoi îndepărtați cele trei calibre de reglare.
\item Ridicați/coborâți gura de aspirare și verificați modul de reglare cu ajutorul calibrului. În cazul în care reglarea nu este corespunzătoare, repetați procedura de reglare de la punctul~\in[sucMouth:adjust].

\stopSteps


\stopsection
\stopchapter
\stopcomponent


