\startcomponent c_80_maintenance_s2_130-sl
\product prd_ba_s2_130-sl

\startchapter [title={Vzdrževanje in servisiranje},
							reference={chap:maintenance}]

\setups[pagestyle:marginless]


\startsection [title={Splošni napotki}]


\subsection{Varstvo okolja}

\starttextbackground [FC]
\setupparagraphs [PictPar][1][width=2.45em,inner=\hfill]

\startPictPar
\Penvironment
\PictPar
	\Boschung\ izvaja varstvo okolja\index{varstvo okolja} v praksi. Usmerjamo se
	po vzroki in pri podjetnih odločitvah vselej upoštevamo vse učinke proizvodnega
	postopka in izdelka na okolje.
	Cilji so varčna uporaba virov in blagodejno ravnanje z
	naravnimi eksistenčnimi osnovami, ki so namenjeni ohranjanju človeka in narave.
	Z upoštevanjem določenih pravil pri obratovanju vozila lahko
	prispevate k varstvu okolja. Sem sodi tudi primerno in
	pravilno ravnanje s snovmi in materiali v okviru
	vzdrževanja vozila (\eG\ odlaganje kemikalij in posebnih odpadkov).

	Poraba goriva in obraba motorja sta odvisna od
	obratovalnih pogojev. Zato vas prosimo, da ste pozorni na nekatere točke:

\startitemize
	\item Motor pustite delovati v prostem teku, da se segreje.
	\item Med čakalnimi časi zaradi obratovanja motor zaustavite.
	\item Redno preverjajte porabo goriva.
	\item {\em Vzdrževalna dela naj vam opravijo v usposobljeni delavnici
	v skladu z vzdrževalnim načrtom.}
\stopitemize
\stopSymList
\stoptextbackground

\page [yes]


\subsection{Varnostni predpisi}

\startSymList
\PHgeneric
\SymList
Da\index{vzdrževanje+varnostni predpisi} preprečite škodo na vozilu in agregatih
ter nesreče pri vzdrževanju je nujno, da
upoštevate naslednje varnostne predpise. Upoštevajte tudi
splošne varnostne predpise (\about[safety:risques], \at{od
strani}[safety:risques]).
\stopSymList

\starttextbackground [FC]
\startPictPar
\PMgeneric
\PictPar
\textDescrHead{Preprečevanje nesreč}
	Po vsakem vzdrževanju ali popravilu preverite\index{preprečevanje nesreč} stanje
	vozila. Predvsem se pred vožnjo v javnem prometu prepričajte, ali
	da vse komponente, bistvene za varnost, ter osvetljava
	in signalna oprema brezhibno
	delujejo.
\stopPictPar
\stoptextbackground
\blank [big]

\start
\setupparagraphs [SymList][1][width=6em,inner=\hfill]
\startSymList\PHcrushing\PHfalling\SymList
\textDescrHead{Stabilizacija vozila}
	Pred vsakim vzdrževanjem je treba vozilo zavarovati
	pred nenadzorovanim premikanjem: Izbirno ročico za stopnjo vožnje postavite na
	\aW{nevtralni položaj}, aktivirajte ročno zavoro in vozilo
	zavarujte s kolesnimi zagozdami.
\stopSymList
\stop

\starttextbackground[CB]
\startPictPar\PHpoison\PictPar
\textDescrHead{Zagon motorja}
	Če \index{nevarnost+zastrupitev} je treba motor zagnati v slabo prezračevanem
	prostoru, ga pustite delovati le tako dolgo, kot je potrebno\index{nevarnost+izpušni plini,}
	da preprečite zastrupitev z ogljikovim monoksidom.
	\stopPictPar
\startitemize
	\item Motor zaganjajte samo s pravilno priključeno baterijo.
	\item Baterije nikoli ne odklopite pri vključenem motorju.
	\item Motorja nikoli ne zaganjajte s pomočjo naprave za pomoč pri zagonu.
	Če\index{baterija+polnilnik} je treba baterijo polniti z napravo
	za hitro polnjenje, jo je treba najprej ločiti od vozila. Upoštevajte tudi
	obratovalne predpise naprave za hitro polnjenje.
\stopitemize
\stoptextbackground

\page [bigpreference]


\subsubsection{Zaščita elektronskih komponent}

\startitemize
	\item Pred\index{električno varjenje} začetkom varjenja
	ločite baterijski kabel od baterije in sklenite pozitivni
	in masni kabel.
	\item Elektronske krmilne naprave\index{elektronika} priključite in ločite samo,
	ko niso pod napetostjo.
	\item Napačna polarnost\index{krmilna naprava} v električnem napajanju (\eG\
	zaradi napačno priključene baterije) lahko uniči elektronske dele
	in naprave.
	\item Pri\index{okoljska temperatura+ekstremna} okoljskih temperaturah nad
	80\,°C (\eG\ v suhi komori) je treba elektronske dele|/|naprave
	odstraniti.
\stopitemize


\subsubsection{Diagnoza in meritve}

\startitemize
	\item Za merilna in diagnostična dela uporabljajte samo preizkuševalne kable, primerne za {\em}
	(\eG\ originalni kabli naprave).
	\item Mobilni telefoni\index{mobilni telefon} in primerljive radijske naprave lahko
	motijo funkcije naprave, diagnostične naprave ter posledično
	obratovalno varnost.
\stopitemize


\subsubsection{Kvalifikacija osebja}

\starttextbackground[CB]
\startPictPar
\PHgeneric
\PictPar
\textDescrHead{Nevarnost nesreče}
Pri\index{kvalifikacija+vzdrževalno osebje} nepravilni izvedbi
vzdrževalnih del lahko pride do motenj v delovanju in varnosti
vozila. To predstavlja povišano tveganje nesreč
in poškodb.

Za vzdrževanje in popravila se obrnite\index{kvalifikacija+delavnica}
na kvalificirano delavnico, ki ima potrebno znanje
in orodje.

V dvomih se obrnite na službo za stranke podjetja \Boschung.
\stopPictPar
\stoptextbackground

Vozilo \ProductId lahko upravlja, vzdržuje ali popravlja izključno kvalificirano
\Boschung osebje ali osebje, ki ga je usposobila služba za stranke
podjetja.

Odgovornosti za upravljanje, servisiranje in popravila
\Boschung določi služba za stranke podjetja.


\subsubsection{Spremembe in preureditve}

\starttextbackground[CB]
\startPictPar
\PHgeneric
\PictPar
\textDescrHead{Nevarnost nesreče}
Vse\index{sprememba na vozilu} spremembe, ki jih samovoljno
naredite na vozilu, lahko vplivajo na delovanje in obratovalno
\ProductId varnost vozila in posledično vodijo do nesreč
in poškodb.
\stopPictPar

\startPictPar
\PMwarranty
\PictPar
Za škodo, nastalo\index{garancija+pogoji} zaradi samovoljnih posegov
ali sprememb na vozilu \ProductId ali katerem od agregatov, podjetje
\Boschung ne prevzema nobenega jamstva ali odgovornosti.
\stopPictPar
\stoptextbackground

\stopsection


\startsection [title={Obratovalne snovi in maziva}, reference={sec:liquids}]


\subsection{Pravilno ravnanje}

\starttextbackground[CB]
\startPictPar
\PHpoison
\PictPar
\textDescrHead{Nevarnost poškodb in zastrupitve}
Zaradi\index{gorivo} stik s kožo\index{maziv}
ali\index{nevarnost+zastrupitev} zaužitja obratovalnih snovi in
maziv lahko\index{gorivo+varnost} pride do znatnih poškodb
ali zastrupitev. Pri ravnanju, skladiščenju in odlaganju teh snovi vedno
upoštevajte zakonske predpise.
\stopPictPar
\stoptextbackground

\starttextbackground [FC]
\startPictPar
\PMproteyes\par
\PMprothands
\PictPar
Pri ravnanju z obratovalnimi snovmi in mazivi vedno nosite ustrezna
zaščitna oblačila in dihalno zaščito. Preprečite vdihavanje hlapov.
Preprečite vsakršen stik s kožo, očmi ali oblačili. Mesta na koži,
ki so prišla v stik z obratovalnimi snovmi, takoj
sperite z vodo in milom.  Če vam obratovalne snovi pridejo v oči, si jih
sperite z veliko čiste vode in pojdite k
okulistu. Po zaužitju obratovalnih snovi je treba takoj
poiskati zdravniško pomoč!
\stopPictPar
\stoptextbackground

\startSymList
\PPchildren
\SymList
Obratovalne snovi hranite otrokom nedostopno.
\stopSymList

\startSymList
\PPfire
\SymList
\textDescrHead{Nevarnost požara}
	Zaradi\index{nevarnost+požar} visoke vnetljivosti obratovalnih snovi
	se pri ravnanju z njimi poveča nevarnost požara.  Pri ravnanju z obratovalnimi snovmi so kajenje,
	ogenj\index{prepoved kajenja} in odprta svetloba
	strogo prepovedani.
\stopSymList

\starttextbackground [FC]
\startPictPar
\PMgeneric
\PictPar
Uporabljati je dovoljeno samo maziva, primerna za \ProductId
komponente, uporabljene v vozilu. Zato uporabljajte samo preverjene
\Boschung\ in odobrene izdelke. Te najdete na
seznamu obratovalnih snovi \atpage[sec:liqquantities]. Dodatki\index{dodatki} za
maziva niso potrebni. Če boste dodali dodatke, lahko
to vodi do izničenja garancije\index{garancija+pogoji}.
Za nadaljnje informacije se obrnite na službo za stranke podjetja \Boschung.
\stopPictPar
\stoptextbackground

\starttextbackground [FC]
\startPictPar
\Penvironment
\PictPar
\textDescrHead{Varstvo okolja}
Pri \index{maziva+odstranjevanje} odstranjevanju obratovalnih snovi
in\crlf maziv\index{varstvo okolja} ali predmetov, kontaminiranih z njimi
(\eG\ filtri, krpe),
upoštevajte določila za varstvo okolja\index{obratovalne snovi+odstranjevanje}.
\stopPictPar
\stoptextbackground

\page [yes]

\setups [pagestyle:normal]


\subsection[sec:liqquantities]{Specifikacije in polnilne količine}

% Vse polnilne količine v naslednji tabeli so orientacijske vrednosti.
% Po vsaki\index{obratovalne snovi+polnilna količina}\index{maziva+polnilna količina}
% menjavi obratovalne snovi|/|maziva je treba\index{polnilne količine+obratovalne snovi in maziva}
% preveriti\index{specifikacije+obratovalne snovi in maziva} dejanski nivo napolnjenosti
% in po potrebi sredstvo doliti ali ga odtočiti.
% \blank[big]

\placetable[margin][tab:glyco]{Zaščita proti zamrznitvi (\index{zaščita proti zamrznitvi}motor)}
{\noteF\startframedcontent[FrTabulate]
%\starttabulate[|Bp(80pt)|r|r|]
\starttabulate[|Bp|r|r|]
\NC Zaščita proti zamrznitvi do {[}°C{]}\NC \bf \textminus 25 \NC \bf \textminus 40 \NC\NR
\NC Destil. voda [Vol.||\%] \NC 60 \NC 40 \NC\NR
\NC Sredstvo proti zamrznitvi \break [Vol.||\%] \NC 40 \NC {\em najv.} 60 \NC\NR
\stoptabulate\stopframedcontent\endgraf
Pozor: Pri deležu prostornine nad 60\hairspace\percent\
sredstvo proti zamrznitvi {\em zaščita proti zamrznitvi pade}, hladilni učinek pa se poslabša!}

\placefig[margin][fig:hydrgauge]{\select{caption}{Prikaz nivoja
Hidravlična tekočina (leva stran vozila)}{Prikaz nivoja
Hidravlična tekočina}}
{\externalfigure[main:hy:level_temp]
\noteF Nivo polnosti hidravličnega rezervoarja je mogoče odčitati na kontrolnem okencu
in ga je treba {\em dnevno} preverjati.}

\vskip -8pt
\start
\define [1] \TableSmallSymb {\externalfigure[#1][height=4ex]}
\define\UC\emptY
\pagereference[page:table:liquids]

\setupTABLE	[frame=off,style={\ssx\setupinterlinespace[line=.86\lH]},background=color,
			option=stretch,
			split=repeat]
\setupTABLE	[r]	[each]	[topframe=on,
						framecolor=TableWhite,
						% rulethickness=.8pt
						]

\setupTABLE	[c]	[odd]	[backgroundcolor=TableMiddle]
\setupTABLE	[c]	[even]	[backgroundcolor=TableLight]
\setupTABLE	[c]	[1]		[width=30mm]
\setupTABLE	[c]	[2]		[width=20mm]
\setupTABLE	[c]	[4]		[width=25mm]
\setupTABLE	[c]	[last]	[width=10mm]
\setupTABLE	[r] [first]	[topframe=off,style={\bfx\setupinterlinespace[line=.95\lH]},
						% backgroundcolor=TableDark
						]
\setupTABLE	[r]	[2]		[framecolor=black]

\bTABLE

\bTABLEhead
	\bTR
	\bTC Skupina \eTC
	\bTC Kategorija \eTC
	\bTC Klasifikacija \eTC
	\bTC Izdelek\note[Produkt] \eTC
	\bTC Količina \eTC
	\eTR
\eTABLEhead

\bTABLEbody
 \bTR \bTD	Dizelski motor \eTD
  \bTD Motorno olje\eTD
  \bTD \liqC{SAE 5W-30}; \liqC{VW\,507.00}\eTD
  \bTD Total Quartz INEO Long Life \eTD
  \bTD	4,3\,l\eTD
  \eTR
 \bTR \bTD	Hidravlični obtok \eTD
  \bTD Hidravlično olje \eTD
  \bTD \liqC{ISO VG 46} \eTD
  \bTD  Total Equiviz ZS 46 (rezervoar pribl. 40\,l) \eTD
  \bTD pribl. 50\,l\eTD
  \eTR
 \bTR \bTD Hidravlični krog (dodatna možnost~\aW{bio})\eTD
  \bTD Hidravlično olje \eTD
  \bTD \liqC{ISO VG 46} \eTD
  \bTD  Total Biohydran TMP SE 46\eTD
  \bTD pribl. 50\,l\eTD
  \eTR
 \bTR \bTD	Magnetni ventili: Jedra tuljav \eTD
  \bTD Mazivo\eTD
  \bTD Bakrena mast \eTD
  \bTD \emptY\eTD
  \bTD po\, potrebi\note[Bedarf] \eTD
  \eTR
 \bTR \bTD	Razno: ključanice, mehanika vrat, zavorna stopalka \eTD
  \bTD Mazivo\eTD
  \bTD Univerzalno razpršilo\eTD
  \bTD \emptY\eTD
  \bTD po\, potrebi\note[Bedarf] \eTD
  \eTR
 \bTR \bTD	Sistem za centralno mazanje \eTD
  \bTD Univerzalna ležajna mast\eTD
  \bTD \liqC{nlgi}~2 \eTD
  \bTD Total Multis EP~2\eTD
  \bTD po\, potrebi\note[Bedarf] \eTD
  \eTR
 \bTR \bTD	Hladilni sistem \eTD
  \bTD Sredstvo proti zamrzovanju|/|koroziji\eTD
  \bTD TL VW 774 F/G; najv. 60\hairspace\% vol.\eTD
  \bTD G12+|/|G12++ (roza|/|vijolično)\eTD
  \bTD pribl. 14\,l \eTD
  \eTR
 \bTR \bTD	Visokotlačna vodna črpalka \eTD
  \bTD Motorno olje\eTD
  \bTD \liqC{SAE 10W-40}; \liqC{api cf~– acea e6}\eTD
  \bTD Total Rubia TIR 8900 \eTD
  \bTD 0,29\,l \eTD
  \eTR
 \bTR \bTD	Klimatska naprava \eTD
  \bTD Hladilno sredstvo\eTD
  \bTD + 20\,ml POE||olje\eTD
  \bTD R\,134a\eTD
  \bTD	700\,g\eTD
  \eTR
 \bTR \bTD	Naprava za pranje stekel  \eTD
	\bTD [nc=2] Voda in koncentrat za pranje oken, \aW{S}~poletje, \aW{W}~zima, upoštevajte razmerje mešanice \eTD
	\bTD Maloprodaja \eTD
  \bTD po\, potrebi\note[Bedarf] \eTD
  \eTR
\eTABLEbody

\eTABLE
\stop

\footnotetext[Bedarf]{{\it po\, potrebi} po potrebi, skladno z vsakokratnimi
navodili}
\footnotetext[Produkt]{Uporabljeni izdelki znamke \Boschung\ . Uporabljate lahko tudi druge izdelke, ki ustrezajo tem specifikacijam.}

\stopsection

\page [yes]

\setups [pagestyle:marginless]


\startsection [title={Vzdrževanje dizelskega motorja},
							reference={sec:workshop:vw}]


\subsection [sSec:vw:diagTool]{Diagnostični sistem v vozilu}

Krmilna naprava motorja (J623) \startregister[index][reg:main:vw]{vzdrževanje+dizelski motor} je opremljena s pomnilnikom napak.
Če se v nadzorovanih tipalih oz. sestavnih delih pojavijo motnje, se te skupaj z vrsto napake shranijo v pomnilnik napak.

Krmilna naprava\index{dizelski motor+diagnoza} po analizi informacij razlikuje med različnimi razredi napak in jih hrani, dokler ne izbrišete vsebine pomnilnika napak.

Napake, ki se pojavijo le {\em občasno}, se prikažejo z dodatkom \aW{SP}. Vzrok za občasne napake je lahko \eG\ zrahljani stik ali kratkotrajna prekinitev voda. Če se v 50 zagonih motorja občasna napaka več ne pojavi, se izbriše iz pomnilnika napak.

Če sistem zazna napake, ki vplivajo na delovanje motorja, na zaslonu Vpada zasveti kontrolni simbol \aW{diagnoza motorja}~\textSymb{vpadWarningEngine1}.

Shranjene napake je mogoče odčitati s sistemom za diagnozo, meritve in informacije vozila \aW{VAS\,5051/B}.

Ko odpravite napako ali napake, je treba pomnilnik napak izbrisati.


\subsubsection[sSec:vw:diagTool:connect]{Zagon diagnostičnega sistema}

\starttextbackground [FC]
\startPictPar
\PMgeneric
\PictPar
Podrobne informacije o diagnostičnem sistemu vozila VAS\,5051/B najdete v navodilih za uporabo sistema.

Uporabljate lahko tudi druge združljive diagnostične sisteme, \eG\ \aW{DiagRA}.
\stopPictPar
\stoptextbackground

\page [yes]


\subsubsubsubject{Pogoji}

\startitemize
\item Varovalke morajo biti v redu.
\item Baterijska napetost mora biti večja od 11,5\,V.
\item Vsi električni porabniki morajo biti izključeni.
\item Priključek mase mora biti v redu.
\stopitemize


\subsubsubsubject{Način postopanja}

\startSteps
\item Vtaknite vtič diagnostičnega voda VAS\,5051B/1 v diagnostični priključek.
\item Glede na funkcijo vključite vžig ali zaženite motor.
\stopSteps

\subsubsubsubject{Izbor obratovalnega načina}

\startSteps [continue]
\item Na zaslonu pritisnite na polje \aW{Samodiagnoza vozila}.
\stopSteps


\subsubsubsubject{Izbor sistema vozila}

\startSteps [continue]
\item Na zaslonu pritisnite na polje \aW{01-Elektronika motorja}.
\stopSteps

Na zaslonu se sedaj prikaže identifikacija krmilnih naprav ter kodiranje krmilne naprave motorja.

Če se kode ne ujemajo, je treba preveriti kodiranje krmilne naprave.


\subsubsubsubject{Izbor diagnostične funkcije}

Na zaslonu so prikazane vse možne diagnostične funkcije.

\startSteps [continue]
\item Na zaslonu pritisnite na polje želene funkcije.
\stopSteps



\subsection [sSec:vw:faultMemory]{Pomnilnik napak}


\subsubsection{Branje pomnilnika napak}

\subsubsubject{Potek dela}

\startSteps
\item Motor pustite delovati v prostem teku.
\item Priključite VAS\,5051/B (glejte \in{odsek}[sSec:vw:diagTool:connect])
in izberite krmilno napravo motorja.
\item Izberite diagnostično funkcijo \aW{004-Vsebina pomnilnika napak}.
\item Izberite diagnostično funkcijo \aW{004.01-Poizvedba pomnilnika napak}.
\stopSteps

{\sla Samo če se motor ne zažene:}

\startitemize [2]
\item Vključite vžig.
\item Če v krmilni napravi motorja ni shranjene nobene napake, se na zaslonu prikaže \aW{0~zaznanih napak}.
\item Če so v krmilni napravi motorja shranjene napake, se bodo prikazale na zaslonu ena pod drugo.
\item Zaključite diagnostično funkcijo.
\item Izključite vžig.
\item Po potrebi odpravite prikazane napake s pomočjo tabele napak (glejte servisno dokumentacijo) in nato izbrišite pomnilnik napak.
\stopitemize

\starttextbackground [FC]
\startPictPar
\PMrtfm
\PictPar
Če katere napake ni mogoče izbrisati, se obrnite na službo za stranke podjetja \boschung.
\stopPictPar
\stoptextbackground


\subsubsubject{Statične napake}

Če je v pomnilniku napak ena ali več statičnih napak, se obrnite na službo za stranke podjetja Boschung, da vam te napake odpravi s pomočjo \aW{vodenega iskanja napak}.


\subsubsubject{Občasne napake}

Če so v pomnilniku napak shranjene izključno občasne napake ali napotki in ni ugotovljenih napak v delovanju elektronskega sistema vozila, lahko pomnilnik napak izbrišete:

\startSteps [continue]
\item Ponovno pritisnite tipko \aW{Naprej}~\inframed[strut=local]{>}, da pridete v kontrolni načrt.
\item Če želite zaključiti vodeno iskanje napak, pritisnite tipko \aW{Skok} in nato \aW{Zaključi}.
\stopSteps

Sedaj ponovno sledi prikaz vseh napak v pomnilniku.

V enem oknu nato potrdite, da želite izbrisati vse občasne napake.
% Das Diagnoseprotokoll wird automatisch (online) verschickt.

Preizkus sistema vozila je s tem končan.


\subsubsection[sSec:vw:faultMemory:errase]{Brisanje pomnilnika napak}

\subsubsubject{Potek dela}

{\sla Pogoji:}

\startitemize [2]
\item Vse napake in vzroki napak morajo biti odpravljeni.
\stopitemize

\page [yes]


{\sla Način postopanja:}

\starttextbackground [FC]
\startPictPar
\PMrtfm
\PictPar
Ko odpravite napake, je treba ponovno priklicati pomnilnik napak in ga izbrisati:
\stopPictPar
\stoptextbackground

\startSteps
\item Motor pustite delovati v prostem teku.
\item Priključite VAS\,5051/B (glejte \in{odsek}[sSec:vw:diagTool:connect])
in izberite krmilno napravo motorja.
\item Izberite diagnostično funkcijo \aW{004-Poizvedba pomnilnika napak}.
\item Izberite diagnostično funkcijo \aW{004.10-Brisanje pomnilnika napak}.
\stopSteps

\starttextbackground [FC]
\startPictPar
\PMrtfm
\PictPar
Če pomnilnika napak ni mogoče izbrisati, to pomeni, da obstaja še kakšna napaka, ki jo je treba odpraviti.
\stopPictPar
\stoptextbackground

\startSteps [continue]
\item Zaključite diagnostično funkcijo.
\item Izključite vžig.
\stopSteps


\subsection [sSec:vw:lub] {Mazanje dizelskega motorja}

\subsubsection [ssSec:vw:oilLevel] {Preverjanje nivoja motornega olja}

\starttextbackground [FC]
\startPictPar
\PMrtfm
\PictPar
Nivo olja\index{motorno olje+nivo} nikakor ne sme presegati oznake \aW{Max.}. Sicer obstaja\index{nivo polnosti+motorno olje} nevarnost okvare katalizatorja.
\stopPictPar
\stoptextbackground

\startSteps
\item Zaustavite motor in počakajte najmanj 3~minute, da lahko olje odteče nazaj v oljno kad.
\item Izvlecite merilno palico in jo obrišite, nato pa znova vstavite do konca v odprtino.
\item Palico znova izvlecite in preverite nivo olja:

\startfigtext[right][fig:vw:gauge]{Odčitanje nivoja olja}
{\externalfigure[VW_Oil_Gauge][width=50mm]}
\startitemize [A]
\item Največji nivo polnosti; olja ni dovoljeno doliti.
\item Zadosten nivo polnosti; olje {\em lahko} dolijete do oznake~\aW{A}.
\item Nezadosten nivo polnosti; olje je {\em treba} doliti, dokler nivo polnosti ne doseže območja \aW{B}.
\stopitemize
{\em Pri nivoju polnosti nad oznako~\aW{A} obstaja nevarnost okvare katalizatorja.}
\stopfigtext
\stopSteps


\subsubsection [ssSec:vw:oilDraining] {Menjava motornega olja}

\starttextbackground [FC]
\startPictPar
\PMrtfm
\PictPar
Filter motornega olja vozila S2 je montiran pokončno. To pomeni, da je treba filter zamenjati {\em pred} zamenjavo olja. Ko odstranite filtrirni vložek, se odpre ventil in olje v ohišju filtra samodejno steče v ohišje ročice.
\stopPictPar
\stoptextbackground

\startSteps
\item Pod motor podstavite primerno\index{dizelski motor+menjava olja} prestrezno posodo.
\item Odvijte izpustni vijak za olje\index{motorno olje+menjava} in iztočite olje.
\stopSteps

\starttextbackground [FC]
\startPictPar
\PMrtfm
\PictPar
Prestrezna posoda mora imeti zadostno prostornino za celotno količino starega olja.
Potrebno specifikacijo olja in polnilno količino najdete v \in{odseku}[sec:liqquantities].

Izpustni vijak za olje je opremljen s tesnilom, ki ga ni mogoče izgubiti. Izpustni vijak za olje je zato treba vedno zamenjati.
\stopPictPar
\stoptextbackground

\startSteps [continue]
\item Privijte nov izpustni vijak za olje s tesnilnim obročem (\TorqueR~30\,Nm).
\item Nalijte motorno olje ustrezne specifikacije (glejte \in{odsek}[sec:liqquantities]).
\stopSteps


\subsubsection [ssSec:vw:oilFilter] {Zamenjava filtra motornega olja}

\starttextbackground [FC]
\startPictPar
\PMrtfm
\PictPar
\startitemize [1]
\item Upoštevajte\index{dizelski motor+filter za olje} predpise za odstranjevanje in recikliranje.
\item Filter zamenjajte \index{filter za olje+dizelski motor}{\em pred} zamenjavo olja (glejte \in{odsek}[ssSec:vw:oilDraining]).
\item Pred montažo na rahlo naoljite tesnilo novega filtra.
\stopitemize
\stopPictPar
\stoptextbackground

\startfigtext[right][fig:vw:oilFilter]{Filter za olje}
{\externalfigure[VW_OilFilter_03][width=50mm]}
\startSteps
\item S primernim vijačnim ključem odvijte pokrov~\Lone\ ohišja filtra.
\item Očistite tesnilne površine pokrova in ohišja filtra.
\item Zamenjajte filtrirni vložek \Lthree.
\item Zamenjajte tesnilna obročka \Ltwo\ in \Lfour.
\item Ponovno privijte pokrov na ohišje filtra (\TorqueR~25\,Nm).
\stopSteps



%\subsubsubject{Données techniques}
%
%
%\hangDescr{Couple de serrage du couvercle:} \TorqueR~25\,Nm.
%
%\hangDescr{Huile moteur prescrite:} Selon tableau \atpage[sec:liqquantities].
%% NOTE: Redundant [tf]

\stopfigtext



\subsubsection [ssSec:vw:oilreplenish] {Dolivanje motornega olja}

\starttextbackground [FC]
\startPictPar
\PMrtfm
\PictPar
\startitemize [1]
\item S krpo očistite polnilne nastavke, \index{motorno olje} {\em preden} odstranite pokrov.
\item Dolijte\index{dizelski motor+dolivanje olja} izključno olje, ki ustreza predpisani specifikaciji.
\item Dolivajte postopoma v majhnih količinah.
\item Da preprečite prekomerno napolnjenje, po vsakem dolivanju malce počakajte, da lahko olje v kadi za motorno olje steče do oznake na merilni palici (glejte \in{odsek}[ssSec:vw:oilLevel]).
\stopitemize
\stopPictPar
\stoptextbackground

\startfigtext[right][fig:vw:oilFilter]{Dolivanje olja}
{\externalfigure[s2_bouchonRemplissage][width=50mm]}
\startSteps
\item Izvlecite merilno palico za pribl. 10~cm, tako da lahko pri dolivanju uhaja zrak.
\item Odprite polnilno odprtino.
\item Dolijte olje in pri tem upoštevajte zgornje predpise.
\item Skrbno zaprite polnilno odprtino.
\item Zaženite motor.
\item Preverite nivo olja (glejte \in{odsek}[ssSec:vw:oilLevel]).
\stopSteps

\stopfigtext


\subsection [sSec:vw:fuel] {Sistem za oskrbo z gorivom}

\subsubsection [ssSec:vw:fuelFilter] {Zamenjava filtra za gorivo}

\starttextbackground [FC]
\startPictPar
\PMrtfm
\PictPar
\startitemize [1]
\item Upoštevajte\index{dizelski motor+filter za gorivo} zakonske predpise za odstranjevanje in recikliranje posebnih odpadkov.
\item Ne odstranite vodov za gorivo z zgornjega dela filtra.
\item Ne vlecite za pritrdilne točke vodov za gorivo, ker lahko pride do poškodb zgornjega dela filtra.
\stopitemize
\stopPictPar
\stoptextbackground

\startfigtext[right][fig:vw:oilFilter]{Filter za gorivo}
{\externalfigure[s2_fuelFilter_location][width=50mm]}

{\sla Priprava:}

Ohišje filtra za gorivo\index{filter za gorivo} je pritrjeno pred motorjem na desni strani šasije.
S pomočjo 10-milimetrskega matičnega ključa in 10-milimetrskega očesnega ključa odstranite pritrdilna vijaka.

\stopfigtext


\page [yes]

\setups [pagestyle:normal]

{\sla Način postopanja:}

\startLongsteps
\item Odstranite vse vijake zgornjega dela filtra. Snemite zgornji del filtra.
\stopLongsteps

\starttextbackground [FC]
\startPictPar
\PMrtfm
\PictPar
Dvignite zgornji del filtra. Po potrebi pri tem na montažni utor (\in{\LAa, sl.}[fig:fuelfilter:detach]) položite kotni izvijač in iztaknite zgornji del.
\stopPictPar
\stoptextbackground

\placefig [margin] [fig:fuelfilter:detach]{Odstranjevanje filtra za gorivo}
{\externalfigure[fuelfilter:detach]}

\placefig [margin] [fig:fuelfilter:explosion]{Filter za gorivo}
{\externalfigure[fuelfilter:explosion]}

\startLongsteps [continue]
\item Izvlecite filtrski vložek iz spodnjega dela filtra.
\item Snemite tesnilo (\in{\Ltwo, sl.}[fig:fuelfilter:explosion]) z zgornjega dela filtra.
\item Skrbno očistite spodnji in zgornji del filtra.
\item V spodnji del filtra vstavite nov filtrski vložek.
\item Namažite novo tesnilo (\in{\Ltwo, sl.}[fig:fuelfilter:explosion]) z nekaj goriva in ga vstavite v zgornji del.
\item Namestite zgornji del na spodnji del filtra in ju enakomerno stisnite, tako da bo zgornji del nalegal po celotnem obsegu na spodnjega.
\item Ponovno {\em ročno} privijte zgornji in spodnji del z vsemi vijaki. Nato navzkrižno privijte vse vijake s predpisanim zateznim momentom (\TorqueR~5\,Nm) an.
\stopLongsteps

% \subsubsubject{Données techniques}
%
% \hangDescr{Couple de serrage des vis de fixation du couvercle:} \TorqueR 5\,Nm.
%% NOTE: redundant [tf]

\startLongsteps [continue]
\item Vključite vžig, da odzračite sistem; zaženite motor in ga pustite delovati 1~do 2~minuti v število vrtljajev prostega teka.
\item Izbrišite pomnilnik napak, kot je opisano na \atpage[sSec:vw:faultMemory:errase].
\stopLongsteps


\subsection [sSec:vw:cooling] {Hladilni sistem}

\starttextbackground [FC]
\startPictPar
\PMrtfm
\PictPar
\startitemize [1]
\item Uporabljajte samo\index{dizelski motor+hlajenje} hladilno sredstvo predpisane specifikacije (\atpage[sec:liqquantities]).
\item Za\index{hladilno sredstvo} zagotovitev zaščite pred zamrznitvijo in korozijo, je dovoljeno hladilno sredstvo redčiti izključno z destilirano vodo in v skladu s spodnjo tabelo.
\item Krogotoka hladilne tekočine nikoli ne polnite z vodo, ker to vpliva na zaščito pred zamrznitvijo in korozijo.
\stopitemize
\stopPictPar
\stoptextbackground


\subsubsection [sSec:vw:coolingLevel] {Nivo hladilnega sredstva}

\placefig [margin] [fig:coolant:level] {Nivo hladilnega sredstva}
{\externalfigure[coolant:level]}


\placefig [margin] [fig:refractometer] {Refraktometer VW~T\,10007}
{\externalfigure[coolant:refractometer]}

\placefig [margin] [fig:antifreeze] {Preverjanje gostote sredstva proti zamrznitvi}
{\externalfigure[coolant:antifreeze]}


\startSteps
\item Dvignite posodo za umazanijo in namestite varnostni podpornik.
\item Preverite\index{nivo polnosti+hladilno sredstvo} nivo polnosti hladilnega sredstva v raztezalni posodi: nivo mora biti nad oznako \aW{min}.
\stopSteps

\start
\define [1] \TableSmallSymb {\externalfigure[#1][height=4ex]}
\define\UC\emptY
\pagereference[page:table:liquids]


\setupTABLE	[frame=off,style={\ssx\setupinterlinespace[line=.86\lH]},background=color,
			option=stretch,
			split=repeat]
\setupTABLE	[r]	[each]	[topframe=on,
						framecolor=TableWhite,
						% rulethickness=.8pt
						]

\setupTABLE	[c]	[odd]	[backgroundcolor=TableMiddle]
\setupTABLE	[c]	[even]	[backgroundcolor=TableLight]
\setupTABLE	[r] [first]	[topframe=off,style={\bfx\setupinterlinespace[line=.95\lH]},
						% backgroundcolor=TableDark
						]
\setupTABLE	[r]	[2]		[framecolor=black]

\bTABLE

\bTABLEhead
 \bTR
 \bTC Zaščita proti zamrznitvi do ... \eTC
 \bTC Delež G12\hairspace ++\eTC
 \bTC Vol. sredstva proti zamrzovanju \eTC
 \bTC Vol. destilirane vode \eTC
 \eTR
\eTABLEhead

\bTABLEbody
 \bTR \bTD \textminus 25\,°C \eTD
  \bTD 40\hairspace\% \eTD
  \bTD 3,8\,l \eTD
  \bTD 4,2\,l \eTD
  \eTR
 \bTR \bTD \textminus 35\,°C \eTD
  \bTD 50\hairspace\% \eTD
  \bTD 4,0\,l \eTD
  \bTD 4,0\,l \eTD
  \eTR
 \bTR \bTD \textminus 40\,°C \eTD
  \bTD 60\hairspace\%  \eTD
  \bTD 4,2\,l \eTD
  \bTD 3,8\,l \eTD
  \eTR
\eTABLEbody

\eTABLE
\stop

\adaptlayout [height=+20pt]
\subsubsection [sSec:vw:coolingFreeze] {Nivo hladilnega sredstva}

Preverite\index{gostota sredstva proti zamrznitvi} gostoto sredstva proti zamrznitvi s primernim refraktometrom (glejte \in{sl.}[fig:refractometer]: VW T\,10007).
Opazujte skalo~1: G12\hairspace ++ (glejte \in{sl.}[fig:antifreeze]).

\page [yes]


\subsection [sSec:vw:airFilter] {Oskrba z zrakom}

Zračni filter je dostopen skozi zadnja vzdrževalna vrata na desni strani vozila (glejte \in{sl.}[fig:airFilter]).

\placefig [margin] [fig:airFilter] {Zračni filter motorja}
{\externalfigure[vw:air:filter]
\noteF
\startLeg
\item Varnostni zapah
\item Spodnji del ohišja
\item Odzračevalna odprtina
\item Tlačno tipalo
\stopLeg}


\subsubsubject{Pogoji uporabe}

Pometalno vozilo se pogosto uporablja v zelo prašnih okoljih. Zaradi tega je treba zračni filter preverjati in čistiti tedensko. Glejte tudi \about[table:scheduleweekly], \atpage[table:scheduleweekly]. Po potrebi je treba zračni filter zamenjati.


\subsubsubject{Samodiagnoza}

Sesalni vod ima tlačno tipalo (\Lfour, \in{sl.}[fig:airFilter]), prek katerega je mogoče zaznati padce tlaka\footnote{zmanjšan pretok zraka zaradi manjše prehodnosti zraka skozi filter} v filtru.
Če je zračni filter zamašen, na zaslonu Vpada zasveti kontrolni simbol~\textSymb{vpadWarningFilter} in prikaže se sporočilo o napaki \VpadEr{851}.


\subsubsubject{Servisiranje/zamenjava}

\startSteps
\item Povlecite varnostni zapah~\Lone navzdol (\in{sl.}[fig:airFilter]).
\item Obrnite spodnji del ohišja~\Ltwo v nasprotni smeri urinega kazalca in ga snemite.
\item Odstranite filtrski vložek in ga preglejte. Po potrebi ga zamenjajte.
\item Očistite notranjost filtra in zračni filter sestavite v obratnem zaporedju.
\stopSteps

\page [yes]


\subsection [sSec:vw:belt] {Klinasti jermen}

Klinasti jermen\index{dizelski motor+klinasti jermen} prenaša gibanje vztrajnika ročične gredi na generator in klimatski kompresor (dodatna oprema).
Napenjalni element\index{klinasti jermen} v zadnjem segmentu (med generatorjem in ročično gredjo) ohranja jermen napet.


\subsubsection [sSec:belt:change] {Zamenjava klinastega jermena}

\placefig [margin] [fig:belt:tool] {Poravnalni trn VW T\,10060\,A}
{\externalfigure[vw:belt:tool]}

\placefig [margin] [fig:belt:overview] {Napenjalni element}
{\externalfigure[vw:belt:overview]}

\placefig [margin] [fig:belt:tens] {Mesto namestitve poravnalnega trna}
{\externalfigure[vw:belt:tens]}


\subsubsubject{S klimatskim kompresorjem}


{\sla Potrebno posebno orodje:}

Poravnalni trn \aW{VW T\,10060\,A} za držanje napenjalnega elementa.

\startSteps
\item Označite smer vrtenja klinastega jermena.
\item Z rebrastim očesnim ključem obrnite ročico napenjalnega elementa v smeri urinega kazalca (\in {sl.}[fig:belt:overview]).
\item Spravite izvrtine (glejte puščice, \in {sl.}[fig:belt:tens]) do pokrivala in fiksirajte napenjalni element s poravnalnim trnom.
\item Snemite klinasti jermen.
\stopSteps

Montaža klinastega jermena poteka v obratnem zaporedju.

\starttextbackground [FC]
\startPictPar
\PMrtfm
\PictPar
\startitemize [1]
\item Bodite pozorni na smer vrtenja klinastega jermena.
\item Bodite pozorni na pravilni sed jermena v jermenicah.
\item Zaženite motor in preverite tek jermena.
\stopitemize
\stopPictPar
\stoptextbackground


\subsubsubject{Brez klimatskega kompresorja}

{\sla Potrebni material:}

pribor za popravilo, sestavljen iz navodil za popravilo, klinastega jermena in posebnega orodja.\footnote{Glejte katalog nadomestnih delov pod \aW{Vzdrževalni deli}.}

\startSteps
\item Prerežite klinasti jermen.
\item Sledite nadaljnjim korakom v navodilih za popravilo.
\stopSteps

\starttextbackground [FC]
\startPictPar
\PMrtfm
\PictPar
\startitemize [1]
\item Bodite pozorni na pravilni sed jermena v jermenicah.
\item Zaženite motor in preverite tek jermena.
\stopitemize
\stopPictPar
\stoptextbackground


\subsubsection [sSec:belt:tens] {Zamenjava napenjalnega elementa}

{\sla Samo pri izvedbi s klimatskim kompresorjem.}

\blank [medium]

\placefig [margin] [fig:belt:tens:change] {Zamenjava napenjalnega elementa}
{\externalfigure[vw:belt:tens:change]
\noteF
\startLeg
\item Napenjalni element
\item Varovalni vijak
\stopLeg

{\bf Zatezni moment}

Varovalni vijak:

\TorqueR~20\,Nm\:+\,½~obrata (180°).}

\startSteps
\item Demontirajte klinasti jermen, kot je opisano (\atpage[sSec:belt:change]).
\item Demontirajte periferne dele (glejte na opremo).
\item Odvijte varovalni vijak (\in{\Ltwo, sl.}[fig:belt:tens:change]).
\stopSteps

Montaža napenjalnega elementa poteka v obratnem zaporedju.

\starttextbackground [FC]
\startPictPar
\PMrtfm
\PictPar
\startitemize [1]
\item Po montaži obvezno uporabite nov varovalni vijak.
\item Zatezni moment: Glejte \in{sl.}[fig:belt:tens:change].
\stopitemize
\stopPictPar
\stoptextbackground

\stopregister[index][reg:main:vw]

\stopsection

\page[yes]


\setups[pagestyle:marginless]


\startsection[title={Hidravlični sistem},
							reference={sec:hydraulic}]

\starttextbackground [FC]
% \startfiguretext[left,none]{}
% {\externalfigure[toni_melangeur][width=30mm]}

\startSymPar
\externalfigure[toni_melangeur][width=4em]
\SymPar
\textDescrHead{Recikliranje obratovalnih sredstev}
Rabljena obratovalna sredstva in maziva ni dovoljeno odstraniti v naravi
niti jih sežgati.

Rabljena maziva ni dovoljeno zlivati v kanalizacijo niti v
naravo, prav tako pa jih ni dovoljeno odstraniti med gospodinjske odpadke.

Rabljenih maziv ni dovoljeno mešati z drugimi tekočinami,
ker obstaja nevarnost nastanka strupenih snovi ali težko odstranljivih snovi.
\stopSymPar
\stoptextbackground
\blank [big]

% \starthangaround{\PMgeneric}
% \textDescrHead{Qualification du personnel}
% Toute intervention sur l’installation hydraulique de votre véhicule ne peut être réalisée que par une personne dument qualifiée, ou par un service reconnu par \boschung.
% \stophangaround
% \blank[big]

\startSymList
\PHgeneric
\SymList
\textDescrHead{Čistoča} Hidravlični sistem zelo občutljivo reagira na
nečistoče v olju. Zato je pomembno, da delate v povsem čistem
okolju.
\stopSymList

\startSymList
\PHhot
\SymList
\textDescrHead{Nevarnost brizganja}
Pred delom na hidravličnem sistemu vozila \sdeux\ je treba sprostiti preostali tlak
v vsakokratnem hidravličnem krogu. Vroči curki olja lahko
vodijo do opeklin.
\stopSymList

\startSymList
\PHhand
\SymList
\textDescrHead{Nevarnost zmečkanin}
Posodo za umazanijo je treba obvezno spustiti ali mehansko zavarovati
z varnostno podporo, preden začnete z deli na hidravličnem sistemu
vozila \sdeux\ .
\stopSymList

\startSymList
\PImano
\SymList
\textDescrHead{Merjenje tlaka}
Za merjenje tlaka, na enega od
\aW{mini merilnih} priključkov kroga priključite manometer. Pazite, da bo manometer prikazoval
ustrezno merilno območje.
\stopSymList

\page [yes]

\setups[pagestyle:normal]

\subsection{Vzdrževalni intervali}

\start

	\setupTABLE	[frame=off,
				style={\ssx\setupinterlinespace[line=.93\lH]},
				background=color,
				option=stretch,
				split=repeat]
	\setupTABLE	[r]	[each]	[
							topframe=on,
							framecolor=white,
							backgroundcolor=TableLight,
							% rulethickness=.8pt,
							]

	% \setupTABLE	[c]	[odd]	[backgroundcolor=TableMiddle]
	% \setupTABLE	[c]	[even]	[backgroundcolor=TableLight]
	\setupTABLE	[c]	[1]		[ % width=30mm,
							style={\bfx\setupinterlinespace[line=.93\lH]},
							]
	\setupTABLE	[r] [first]	[topframe=off,
							style={\bfx\setupinterlinespace[line=.93\lH]},
							backgroundcolor=TableMiddle,
							]
	% \setupTABLE	[r]	[2]		[style={\ssBfx\setupinterlinespace[line=.93\lH]}]


\bTABLE

\bTABLEhead
\bTR\bTD Vzdrževalno delo \eTD\bTD Interval \eTD\eTR
\eTABLEhead

\bTABLEbody
\bTR\bTD Preverite tesnjenje. \eTD\bTD Dnevno \eTD\eTR
\bTR\bTD Preverite nivo hidravličnega olja. \eTD\bTD Dnevno \eTD\eTR
\bTR\bTD Preverite stanje hidravličnih vodov|/|gibkih cevi in jih po potrebi zamenjajte. \eTD\bTD 600\,h / 12~mesecev \eTD\eTR
\bTR\bTD Zamenjajte povratni filter za hidravlično olje in sesalni filter. \eTD\bTD 600\,h / 12~mesecev \eTD\eTR
\bTR\bTD Z bakreno mastjo namažite jedra tuljav magnetnih ventilov. \eTD\bTD 600\,h / 12~mesecev \eTD\eTR
\bTR\bTD Zamenjajte hidravlično olje. \eTD\bTD 1200\,h / 24~mesecev \eTD\eTR
\eTABLEbody
\eTABLE
\stop


\subsection[niveau_hydrau]{Nivo polnosti}

\placefig[margin][fig:hydraulic:level]{Nivo polnosti hidravlične tekočine}
{\externalfigure[hydraulic:level]
\noteF
\startLeg
\item Optimalen nivo polnosti
\stopLeg}

Prozorno
kontrolno okence\index{nivo polnosti+hidravlična tekočina}\index{vzdrževanje+hidravlični sistem}
omogoča preverjanje nivoja hidravličnega olja.
Če nivo hidravličnega olja pade, je treba poiskati
vzrok, preden olje dolijete. Upoštevajte predpisane
intervale zamenjave (zgornja tabela) in specifikacije za
hidravlično tekočino (tabela \at{stran}[sec:liqquantities]).


\subsubsection{Dolivanje hidravlične tekočine}

Dolijte hidravlično tekočino, dokler srednje kontrolno okence ne bo povsem prekrito.
Zaženite motor in po potrebi dolijte še tekočine, dokler ne dosežete
želenega nivoja polnosti.


\subsection{Zamenjava hidravlične tekočine}

Polnilno tekočino in potrebne specifikacije hidravlične tekočine najdete v
tabeli na \at{strani}[sec:liqquantities].

\startSteps
\item Odprite odprtino za dolivanje na hidravličnem rezervoarju.
\item S pomočjo naprave za izsesovanje olja izpraznite rezervoar ali pa odstranite izpustni vijak.

Izpustni vijak je spodaj na hidravličnem rezervoarju pred levim zadnjim kolesom (\in{sl.}[fig:hydraulic:fluidDrain]).
\item Dolijte hidravlično tekočino, dokler srednje kontrolno okence ne bo povsem prekrito.
Zaženite motor in po potrebi dolijte še tekočine, dokler ne dosežete potrebnega nivoja polnosti.
\stopSteps

\placefig[margin][fig:hydraulic:fluidDrain]{Izpustni vijak}
{\externalfigure[hydraulic:fluidDrain]}


\placefig[margin][fig:hydraulic:returnFilter]{Hidravlični filter}
{\externalfigure[hydraulic:returnFilter]}

\subsection[filtres:nettoyage]{Povratni in sesalni filter}

\startSteps
\item Dvignite posodo za umazanijo in namestite varnostni podpornik.
\item Snemite pokrov filtra na hidravličnem rezervoarju (\in{sl.}[fig:hydraulic:returnFilter]).
\item Zamenjajte\index{filter za olje+hidravlični} filtrski vložek z novim.
\item Namažite nov tesnilni obroček z nekaj hidravlične tekočine in ga namestite.
\item Z dvema rokama ponovno privijte pokrov (\TorqueR~pribl.~20\,Nm).
\stopSteps

\page [yes]


\subsection[sec:solenoid]{Mazanje magnetnih ventilov}

\placefig[margin][graissage_bobine]{Mazanje magnetnih ventilov}
{\externalfigure[graissage_bobine][M]
\noteF
\startLeg
\item Tuljava magnetnega ventila
\item Jedro tuljave
\stopLeg}

Vlaga in ostanki soli, ki zaidejo v jedro elektromagnetne tuljave,
vodijo do korozije jeder. Jedra tuljave je zato treba enkrat
letno namazati z bakreno mastjo. Mast mora biti obstojna proti koroziji, vodi in
temperaturam do 50\,°C:
\startSteps
\item Demontirajte tuljavo magnetnega ventila (\in{\Lone, sl.}[graissage_bobine]).
\item Namažite jedro (\in{\Ltwo, sl.}[graissage_bobine]) s predpisano
posebno mastjo in montirajte tuljavo nazaj.
\stopSteps


\subsection{Zamenjava gibkih cevi}

Zaščitna guma\index{gibke cevi+intervali zamenjave} in ojačevalna tkanina
gibkih cevi se z leto obrabijo. Zato je treba
gibke cevi hidravličnega sistema obvezno menjavati v predpisanih
intervalih, četudi ni {\em vidnih} poškodb.

Bodite pozorni, da so gibke cevi pravilno pritrjene na vozilo,
da preprečite predčasno obrabo zaradi trenja. Imeti morajo
zadosten razmak do drugih sestavnih delov, da preprečite
škodo zaradi trenja in vibracij.

\stopsection

\page [yes]

\setups [pagestyle:bigmargin]


\startsection[title={Zavorni sistem},
							reference={sec:brake}]

\placefig[margin][fig:brake:rear]{Bobnasta zavora}
{\startcombination [1*2]
{\externalfigure[brake:wheelHub]}{\slx Pesto zadnjega kolesa}
{\externalfigure[brake:drum]}{\slx Mehanizem in zavorne garniture}
\stopcombination}

Zavorne bobne~\Lfour\ je treba demontirati pri vsakem rednem vzdrževanju,
očistiti zavorni mehanizem~\Lseven\  ter pregledati zavorne garniture~\Lfive
\Lsix\ (\in{sl.}[fig:brake:rear]).


\subsubject {Demontaža}

\startSteps
\item Zapeljite vozilo na primeren dvižni oder in dvignite kolesa.
\item Snemite kolesa.
\stopSteps


{\sla Demontaža zavor sprednjih koles}

\startSteps [continue]
\item Demontirajte zavorni boben~\Lfour\ .
\stopSteps

{\sla Demontaža zavor zadnjih koles}

\startSteps [continue]
\item Snemite pokrov~\Lone\  s pesta.
\item Odstranite vijak~\Ltwo\  in snemite vmesni element.
\item Odvijte matico pesta~\Lthree\  z matičnim ključem.
\item Snemite pokrov z zavornim bobnom.
\stopSteps


\subsubject {Ponovna montaža}

Zavorne bobne montirajte v obratnem zaporedju. Zategnite
matice pest zadnjih koles~\Lthree\  s predpisanim
zateznim momentom 190\,Nm an.

\stopsection

\page [yes]

\setups [pagestyle:normal]


\startsection[title={Pregled in vzdrževanje pnevmatik},
							reference={sec:pneumatiques}]

Pnevmatike\index{pnevmatike+vzdrževanje}  mora biti vedno v brezhibnem stanju,
da lahko izpolnjujejo svoji dve glavni funkciji: dobra oprijemljivost in
brezhibno zaviranje. Nedovoljeno visoka obraba in napačen polnilni tlak,
predvsem prenizki tlak, so pomembni dejavniki nesreč.


\subsection{Točke, bistvene za varnost}

\subsubsection{Kontrola obrabe}

Obrabo pnevmatik je treba preveriti na osnovi pokazateljev obrabe, ki so
v profilu pnevmatike (\in{sl.}[pneususure]).
Nepravilnosti na pnevmatiki in njihove vzroke je mogoče ugotoviti s pomočjo vizualnega pregleda:

\placefig[margin][pneususure]{Kontrola obrabe}
{\Framed{\externalfigure[pneusUsure][M]}}

\placefig[margin][pneusdomages]{Poškodovane pnevmatike}
{\Framed{\externalfigure[pneusDomages][M]}}

\startitemize
\item Obraba na straneh tekalne površine: Prenizek polnilni tlak.
\item Ojačena obraba na sredini: Prevelik polnilni tlak.
\item Asimetrična obraba na straneh pnevmatike: Napačno nastavljena sprednja os (osna razdalja,
geometrija osi).
\item Razpoke v tekalni površini: Pnevmatike so prestare; guma pnevmatike s časom postane
bolj trda in razpokana (\in{sl.}[pneusdomages]).
\stopitemize

\starttextbackground[CB]
\startPictPar
\PHgeneric
\PictPar
\textDescrHead{Tveganja zaradi obrabljenih pnevmatik.}
Obrabljena pnevmatika ne izpolnjuje več svoje funkcije predvsem
glede odvajanje vode in blata, zavorna pot se podaljša,
vedenje med vožnjo pa poslabša. Obrabljena pnevmatika hitreje zdrsi, predvsem
v mokrih pogojih. Nevarnost, da pnevmatika izgubi sprijemljivost s podlago, narašča.
\stopPictPar
\stoptextbackground


\subsubsection{Polnilni tlak pnevmatik}

Predpisan polnilni tlak pnevmatik je naveden na tipski ploščici pnevmatik spredaj na
konzoli na sovoznikovi strani (\atpage [sec:plateWheel]).

Četudi\index{pnevmatike+polnilni tlak} so pnevmatike v dobrem stanju,
s časom hitreje začnejo izgubljati zrak (več
kot vozilo uporabljate, večja je izguba tlaka). Zato je
treba tlak v pnevmatikah preverjati mesečno pri hladnih pnevmatikah. Če tlak
preverjate pri toplih pnevmatikah, je treba k predpisanemu tlaku prišteti 0,3\,
bara.

\start
\setupcombinations[M]
\placefig[margin][pneuspression]{Polnilni tlak pnevmatik}
{\Framed{\externalfigure[pneusPression][M]}
\noteF
\startLeg
\item Pravilni tlak
\item Previsoki tlak
\item Prenizki tlak
\stopLeg
Predpisan tlak v pnevmatikah je naveden na tipski ploščici koles v voznikovi kabini
na sovoznikovi strani.}
\stop

\starttextbackground[CB]
\startPictPar
\PHgeneric
\PictPar
\textDescrHead{Nevarnosti zaradi prenizkega tlaka v pnevmatikah}
Pri prenizkem tlaku lahko pnevmatika poči. Pnevmatike se
bolj stisnejo, če niso dovolj napolnjene ali če je
vozilo prekomerno natovorjeno. Guma se pri tem segreje in deli pnevmatike
lahko v zavoju odstopijo.
\stopPictPar
\stoptextbackground

\stopsection

\page [yes]

\setups[pagestyle:marginless]


\startsection[title={Šasija},
			  reference={main:chassis}]

\subsection{Za varnost bistvena pritrditev komponent}

Pri vsakem vzdrževanju je treba preveriti pravilno pritrditev za varnost bistvenih varnostnih vijakov določenih komponent in njihove zatezne momente. To velja predvsem za zgibni sistem in osi.

\blank [big]

\startfigtext [left] [fig:frontAxle:fixing] {Sprednja os}
{\externalfigure [frontAxle:fixing]}
{\sla Pritrdilni elementi sprednje osi}
\startLeg
\item Pritrditev lista vzmeti: \TorqueR~150\,Nm
\item Pritrditev vlečnih enot: \TorqueR~78\,Nm
\stopLeg

{\sla Pritrdilni elementi zadnje osi}
\startLeg
\item Pritrditev lista vzmeti: \TorqueR~150\,Nm
\stopLeg

\stopfigtext

\start

\setupTABLE	[frame=off,style={\ssx\setupinterlinespace[line=.93\lH]},background=color,
			option=stretch,
			split=repeat]

\setupTABLE	[r]	[each]	[topframe=on,
						framecolor=white,
						% rulethickness=.8pt
						]

\setupTABLE	[c]	[odd]	[backgroundcolor=TableMiddle]
\setupTABLE	[c]	[even]	[backgroundcolor=TableLight]
\setupTABLE	[c]	[1]		[style={\bfx\setupinterlinespace[line=.93\lH]}]
\setupTABLE	[r] [first]	[topframe=off,style={\bfx\setupinterlinespace[line=.93\lH]},
						]
% \setupTABLE	[r]	[2]		[style={\bfx\setupinterlinespace[line=.93\lH]}]


\bTABLE

\bTABLEhead
\bTR [backgroundcolor=TableDark] \bTD [nc=3] Zatezni momenti \eTD\eTR
% \bTR\bTD Position \eTD\bTD Type de vis \eTD\bTD Couple \eTD\eTR
\eTABLEhead

\bTABLEbody
\bTR\bTD Pogonski motorji levo/desno \eTD\bTD M12\:×\:35~8,8 \eTD\bTD 78\,Nm \eTD\eTR
%% NOTE @Andrew: das sind Hydraulikmotoren
\bTR\bTD Delovna črpalka \eTD\bTD M16\:×\:40~100 \eTD\bTD 330\,Nm \eTD\eTR
\bTR\bTD Pogonska črpalka \eTD\bTD M12\:×\:40~100 \eTD\bTD 130\,Nm \eTD\eTR
\bTR\bTD Listi vzmeti spredaj|/|zadaj \eTD\bTD M16\:×\:90|/|160~8,8 \eTD\bTD 150\,Nm \eTD\eTR
% \bTR\bTD Fixation du système oscillant \eTD\bTD M12\:×\:40~8.8 \eTD\bTD 78\,Nm \eTD\eTR
\bTR\bTD Pritrditev posode za umazanijo \eTD\bTD M10\:×\:30 Verbus Ripp~100 \eTD\bTD 80\,Nm \eTD\eTR
\bTR\bTD Kolesne matice \eTD\bTD M14\:×\:1,5 \eTD\bTD 180\,Nm \eTD\eTR
\bTR\bTD Pritrditev sprednje metle \eTD\bTD M16\:×\:40~100 \eTD\bTD 180\,Nm \eTD\eTR
\eTABLEbody
\eTABLE
\stop


\stopsection

\page [yes]


\startmode [main:centralLubrication]

\startsection[title={Sistem za centralno mazanje},
							reference={main:graissageCentral}]


\subsection{Opis krmilnega modula}

Vozilo \sdeux\ je lahko opremljeno s\index{sistem za centralno mazanje} sistemom za
centralno mazanje (dodatna možnost). Sistem za centralno mazanje v rednih intervalih oskrbuje
vsako mazalno točko vozila z mazivom.

\startfigtext [left] [vogel_affichage] {Prikazovalni modul}
{\externalfigure[vogel_base2][W50]}
\blank
\startLeg
	\item 7-mestni zaslon: vrednosti in obratovalno stanje
	\item \LED: sistem v premoru (stanje pripravljenosti)
	\item \LED: črpalka obratuje
	\item \LED: krmiljenje sistema s cikličnim stikalom
	\item \LED: nadzor sistema s tlačnim stikalom
	\item \LED: sporočilo o napaki
	\item Zaslonske tipke:
	\startLeg [R]
	\item Vklop zaslona
	\item Prikaz vrednosti
	\item Sprememba vrednosti
	\stopLeg
	\item Tipka za spremembo obratovalnega načina; potrditev vrednosti
	\item Sprožitev ciklusa vmesnega mazanja
\stopLeg
\stopfigtext

Sistem za centralno mazanje zajema tlačilko za mazivo, prozorni
zbiralnik za mazivo na levi strani šasije in krmilni modul v
centralni elektroniki.
% \blank
\page [yes]


\subsubsubject{Prikaz in tipke krmilnega modula}

\start

\setupTABLE	[frame=off,style={\ssx\setupinterlinespace[line=.93\lH]},background=color,
			option=stretch,
			split=repeat]

\setupTABLE	[r]	[each]	[topframe=on,
						framecolor=white,
						% rulethickness=.8pt
						]

\setupTABLE	[c]	[odd]	[backgroundcolor=TableMiddle]
\setupTABLE	[c]	[even]	[backgroundcolor=TableLight]
\setupTABLE	[c]	[1]		[width=9mm,style={\bfx\setupinterlinespace[line=.93\lH]}]
\setupTABLE	[r] [first]	[topframe=off,style={\bfx\setupinterlinespace[line=.93\lH]},
						]
% \setupTABLE	[r]	[2]		[style={\bfx\setupinterlinespace[line=.93\lH]}]


\bTABLE
\bTABLEhead
% \bTR [backgroundcolor=TableDark]
% \bTD [nc=4] Anzeige und Tasten des Steuermoduls \eTD\eTR
\bTR\bTD Poz. \eTD
\bTD \LED \eTD\bTD Način prikaza \eTD
\bTD Način programiranja \eTD\eTR
\eTABLEhead

\bTABLEbody
	\bTR\bTD 2 \eTD
	\bTD Obratovalno stanje {\em Pause}\hskip .5em\null \eTD
	\bTD Naprava je v stanju pripravljenost\hskip .5em\null \eTD % ||Betrieb
	\bTD Čas premora je mogoče spremeniti. \eTD\eTR
	\bTR\bTD 3 \eTD
	\bTD Obratovalno stanje {\em Contact} \eTD
	\bTD Črpalka deluje. \eTD
	\bTD Delovni čas je mogoče spremeniti. \eTD\eTR
	\bTR\bTD 4 \eTD
	\bTD Kontrola sistema {\em CS} \eTD
	\bTD Z zunanjim cikličnim stikalom. \eTD
	\bTD Kontrolni način je mogoče izključiti ali spremeniti. \eTD\eTR
	\bTR\bTD 5 \eTD
	\bTD Kontrola sistema {\em PS} \eTD
	\bTD Z zunanjim tlačnim stikalom. \eTD
	\bTD Kontrolni način je mogoče izključiti ali spremeniti. \eTD\eTR
	\bTR\bTD 6 \eTD
	\bTD Motnja {\em Fault} \eTD
	\bTD [nc=2] Prišlo je do motnje v delovanju. Vzrok je prikazan v
	obliki kode napake, ko pritisnete tipko~\textSymb{vogel_DK}
	. Izvajanje funkcije se prekine. \eTD\eTR
	\bTR\bTD 7 \eTD
	\bTD Puščične tipke \textSymb{vogelTop}~\textSymb{vogelBottom} \eTD
	\bTD [nc=2] \items[symbol=R]{Vklop zaslona, priklic parametrov
	(način prikaza), nastavitev prikazane (I) vrednosti (način programiranja)}
	\eTD\eTR
	\bTR\bTD 8 \eTD
	\bTD Tipka \textSymb{vogelSet} \eTD
	\bTD [nc=2] Preklapljanje med načinom prikaza in programiranja ali
	potrditev vnesenih vrednosti. \eTD\eTR
	\bTR\bTD 9 \eTD
	\bTD Tipka \textSymb{vogel_DK} \eTD
	\bTD [nc=2] Če je naprava v stanju premora {\em Pause}, se ob pritisku
	tipke sproži ciklus vmesnega mazanja. Sporočila o napaki
	se potrdijo in izbrišejo. \eTD\eTR
\eTABLEbody
\eTABLE
\stop
\vfill

\startfigtext [left] [vogel_touches]{Prikazovalni modul}
{\externalfigure[vogel_base][width=50mm]}
\textDescrHead{Način prikaza} Na kratko pritisnite eno od
puščičnih tipk~\textSymb{vogelTop}~\textSymb{vogelBottom}, da vključite
7-mestni zaslon~\textSymb{led_huit}. S ponovnim pritiskom
tipke~\textSymb{vogelTop} je mogoče prikazati različne parametre in njihove
vrednosti. Način {\em Prikaz} je mogoče prepoznati po tem,
da vedno svetijo indikatorji \LED\char"2060 (\in{2~do 6, sl.}[vogel_affichage]).
\blank [medium]
\textDescrHead{Način programiranja} Za spremembo vrednosti pritisnite
in za najm. 2~ sekundi
zadržite tipko~\textSymb{vogelSet}, da preklopite v način {\em}
Programiranje. Indikatorji \LED\char"2060 utripajo. Na kratko pritisnite tipko~\textSymb{vogelSet},
da spremenite\index{sistem za centralno mazanje+programiranje} prikaz, nato
pa s tipkama~\textSymb{vogelTop}~\textSymb{vogelBottom}
spremenite želeno vrednost. Potrdite s tipko \index{sistem za centralno mazanje+prikaz} tipko~\textSymb{vogelSet}
.
\stopfigtext

\page [yes]


\subsection{Podmeniju v načinu {\em Prikaz}}

\vskip -9pt

\adaptlayout [height=+5mm]

\startcolumns[balance=no]\stdfontsemicn

\startSymVogel
\externalfigure[vogel_tpa][width=26mm]
\SymVogel
\textDescrHead{Čas premora [h]} Pritisnite tipko~\textSymb{vogelTop}, da
prikažete programirane vrednosti.
\stopSymVogel

\startSymVogel
\externalfigure[vogel_068][width=26mm]
\SymVogel
\textDescrHead{Preostali čas premora [h]} Še preostali čas do
naslednjega mazalnega ciklusa.
\stopSymVogel

\startSymVogel
\externalfigure[vogel_090][width=26mm]
\SymVogel
\textDescrHead{Skupni čas premora [h]} Skupni čas premora med dvema ciklusoma.
\stopSymVogel

\startSymVogel
\externalfigure[vogel_tco][width=26mm]
\SymVogel
\textDescrHead{Čas mazanja [min]} Pritisnite~\textSymb{vogelTop}, da
prikažete programirane vrednosti.
\stopSymVogel

\startSymVogel
\externalfigure[vogel_tirets][width=26mm]
\SymVogel
\textDescrHead{Naprava v stanju pripravljenosti} Prikaz ni mogoč, ker je naprava v
stanju pripravljenosti (premor).
\stopSymVogel

\startSymVogel
\externalfigure[vogel_026][width=26mm]
\SymVogel
\textDescrHead{Čas mazanja [min]} Trajanje mazanja.
\stopSymVogel

\startSymVogel
\externalfigure[vogel_cop][width=26mm]
\SymVogel
\textDescrHead{Kontrola sistema} Pritisnite tipko~\textSymb{vogelTop}, da
prikažete programirane vrednosti.
\stopSymVogel

\startSymVogel
\externalfigure[vogel_off][width=26mm]
\SymVogel
\textDescrHead{Kontrolni način} \hfill PS:~tlačno stikalo;\crlf
CS:~ciklično stikalo; OFF:~izključeno.
\stopSymVogel

\startSymVogel
\externalfigure[vogel_0h][width=26mm]
\SymVogel
\textDescrHead{Obratovalne ure} Pritisnite tipko~\textSymb{vogelTop}, da prikažete
vrednost v dveh korakih.
\stopSymVogel

\startSymVogel
\externalfigure[vogel_005][width=26mm]
\SymVogel
\textDescrHead{1. del: 005} Obratovalni čas se prikaže v dveh delih; na
2.~ del s tipko~\textSymb{vogelTop}.
\stopSymVogel

\startSymVogel
\externalfigure[vogel_338][width=26mm]
\SymVogel
\textDescrHead{2. del: 33,8} 2.~ del številke je 33,8; skupaj znese
obratovalni čas 533,8\,h.
\stopSymVogel

\startSymVogel
\externalfigure[vogel_fh][width=26mm]
\SymVogel
\textDescrHead{Čas napake} Pritisnite tipko~\textSymb{vogelTop}, da prikažete vrednost
v dveh korakih.
\stopSymVogel

\startSymVogel
\externalfigure[vogel_000][width=26mm]
\SymVogel
\textDescrHead{1. del: 000} Čas napake se prikaže v dveh delih;\crlf
na 2.~ del s tipko~\textSymb{vogelTop}.
\stopSymVogel

\startSymVogel
\externalfigure[vogel_338][width=26mm]
\SymVogel
\textDescrHead{2. del: 33,8} 2.~ del številke je 33,8; skupaj znese
čas napake 33,8\,h.
\stopSymVogel

\stopcolumns

\stopsection


\page [yes]

\stopmode % central lubrication

\setups [pagestyle:marginless]


\startsection[title={Mazalni načrt za ročno mazanje},
							reference={sec:grasing:plan}]

\starttextbackground [FC]
\startPictPar
\PMgeneric
\PictPar
Mazalna mesta, navedena v mazalnem načrtu (\in{sl.}[fig:greasing:plan]),
je treba redno mazati. Redno mazanje
je obvezno za zagotovitev trajnega {\em zmanjšanja trenja}
ter preprečevanje vdora vlage in drugih korozivnih snovi.
\stopPictPar
\stoptextbackground

\blank [big]

\start

\setupcombinations [width=\textwidth]

\placefig[here][fig:greasing:plan]{Mazalni načrt vozila}
{\startcombination [3*1]
{\externalfigure[frame:steering:greasing]}{\ssx Zgibno krmilje in nihalni mehanizem}
{\externalfigure[frame:axles:greasing]}{\ssx Osi}
{\externalfigure[frame:sucMouth:greasing]}{\ssx Sesalno ustje}
\stopcombination}

\stop

\vfill

\startLeg [columns,three]
\item Dvižni valj zgibnega krmilja\crlf {\sl 2 mazalki na valj}
\item Ležaj zgibnega krmilja\crlf {\sl 2 mazalki na levi strani}
\columnbreak
\item Ležaj nihalnega mehanizma\crlf {\sl 1 mazalka pred rezervoarjem}
\item Listi vzmeti\crlf {\sl 2 mazalki na list vzmeti}
\columnbreak
\item Sesalno ustje\crlf {\sl 1 mazalka na kolo}
\item Sesalno ustje\crlf {\sl 1 mazalka na vlečni roki}
\stopLeg



\page [yes]


\setups [pagestyle:bigmargin]


\subsubject{Mazanje posode za umazanijo}

Posoda za umazanijo ima 6~ mazalnih točk (2\:×\:4), ki jih je treba mazati tedensko.

\blank [big]


\placefig[here][fig:greasing:container]{Dvižni mehanizem posode}
{\externalfigure[container:mechanisme]}


\placelegende [margin,none]{}
{{\sla Legenda:}

\startLeg
\item Levi ležaj posode (2\:×)
\item Desni ležaj posode (2\:×)
\item Levi hidravlični valj (zgoraj)
\item Levi hidravlični valj (spodaj)

{\em Kot desni valj (točka \in[greasing:point;hide]).}
\item Desni hidravlični valj (zgoraj)
\item [greasing:point;hide] Desni hidravlični valj (spodaj)
\stopLeg}

\stopsection

\page [yes]



\startsection[title={Električni sistem},
							reference={sec:main:electric}]

\subsection{Centralna elektrika v šasiji}

\startbuffer [fuses:preventive]
\starttextbackground [CB]
\startPictPar
\PHvoltage
\PictPar
\textDescrHead{Varnostni predpisi}
Upoštevajte varnostne predpise v\index{varovalke+šasija}
teh\index{rele+šasija} navodilih: Varovalke vedno zamenjajte samo
z varovalkami s predpisano napetostjo; pred delom na
električnem\index{električni sistem} sistemu, snemite ves kovinski nakit (prstane,
zapestnice itd.).
\stopPictPar
\stoptextbackground
\stopbuffer


\subsubsubject{Varovalke MIDI}

\starttabulate[|l|r|p|]
\HL
\NC\md F\,1 \NC 5\,A  \NC zavorna luč, \aW{+\:15} OBD \NC\NR
\NC\md F\,2 \NC 5\,A  \NC \aW{+\:15} krmilje motorja \NC\NR
\NC\md F\,3 \NC 7,5\,A \NC \aW{+\:30} krmilje motorja in OBD \NC\NR
\NC\md F\,4 \NC 20\,A \NC črpalka za gorivo \NC\NR
\NC\md F\,5 \NC 20\,A \NC \aW{D\:+} generator, \aW{+\:15} rele K\,1 \NC\NR
\NC\md F\,6 \NC 5\,A \NC krmilje motorja \NC\NR
\NC\md F\,7 \NC 10\,A\NC priprava izpušnih plinov motorja \NC\NR
\NC\md F\,8 \NC 20\,A \NC elektronika motorja (krmiljenje) \NC\NR
\NC\md F\,9 \NC 15\,A \NC priprava izpušnih plinov motorja, črpalka za gorivo, razžarevanje \NC\NR
\NC\md F\,10\NC 30\,A \NC krmilje motorja \NC\NR
\NC\md F\,11\NC 5\,A \NC vzvratna luč \NC\NR
%% NOTE @Andrew: Singular
\HL
\stoptabulate

\placefig [margin] [fig:electric:power:rear] {Centralna elektrika v šasiji}
{\externalfigure [electric:power:rear]
\noteF
\startKleg
\sym{K\,1} elektronska krmilna naprava motorja
\sym{K\,2} črpalka za gorivo
\sym{K\,3} sprostitev zaganjalnika
\sym{K\,4} zavorne luči
\sym{K\,5} {[}rezerva{]}
\sym{K\,6} vzvratna luč
\sym{K\,7} sistem za razžarevanje
\stopKleg
}


\subsubsubject{Varovalke MAXI}

% \startcolumns [n=2]
\starttabulate[|l|r|p|]
\HL
\NC\md F\,15 \NC 50\,A \NC glavno napajanje centralne elektrike \NC\NR
\HL
\stoptabulate

\page [yes]

\setups[pagestyle:marginless]


\subsection{Centralna elektrika v voznikovi kabini}

\startcolumns[rule=on]

\placefig [bottom] [fig:fuse:cab] {Varovalke in releji v voznikovi kabini}
{\externalfigure [electric:power:front]}

\columnbreak

\subsubsubject{Rele}

\vskip -12pt

\index{varovalke+voznikova kabina}\index{rele+voznikova kabina}

\starttabulate[|lB|p|]
\NC K\,2 	\NC klimatski kompresor\NC\NR
\NC K\,3 	\NC klimatski kompresor\NC\NR
\NC K\,4 	\NC električna vodna črpalka\NC\NR
\NC K\,5 	\NC rotacijska luč\NC\NR
\NC K\,10 \NC frekvenčni dajalnik utripanja\NC\NR
\NC K\,11 \NC zasenčena luč\NC\NR
\NC K\,12 \NC dolga luč {[}rezerva{]} \NC\NR
\NC K\,13 \NC delovni žaromet\NC\NR
\NC K\,14 \NC intervalno delovanje brisalca\NC\NR
\stoptabulate

\vskip -24pt

\placefig [bottom] [fig:fuse:access] {Dostopna loputa do centralne elektrike}
{\externalfigure [electric:power:cabin]}

\stopcolumns

\page [yes]


\subsubsubject{Varovalke MINI}

\startcolumns[rule=on]
% \setuptabulate[frame=on]
%\placetable[here][tab:fuses:cab]{Fusibles dans la cabine}
%{\noteF
\starttabulate[|lB|r|p|]
\NC F\,1  \NC 3\,A \NC pozicijska luč levo \NC\NR
\NC F\,2  \NC 3\,A \NC pozicijska luč desno \NC\NR
\NC F\,3  \NC 7,5\,A \NC zasenčena luč levo \NC\NR
\NC F\,4  \NC 7,5\,A \NC zasenčena luč desno \NC\NR
\NC F\,5  \NC 7,5\,A \NC dolga luč levo {[}rezerva{]} \NC\NR
\NC F\,6  \NC 7,5\,A \NC dolga luč desno {[}rezerva{]} \NC\NR
\NC F\,7  \NC 10\,A \NC delovni žaromet zgoraj \NC\NR
%% NOTE @Andrew: Plural
\NC F\,8  \NC 10\,A \NC delovni žaromet spodaj (rezerva) \NC\NR
%% NOTE @Andrew: Plural
\NC F\,9  \NC 10\,A \NC sprednja metla \NC\NR
\NC F\,10 \NC 10\,A \NC brisalec \NC\NR
\NC F\,11 \NC 5\,A \NC stikalo za osvetlitev in opozorilne utripalnike \NC\NR
\NC F\,12 \NC 5\,A \NC {[}rezerva{]} \NC\NR
\NC F\,13 \NC 10\,A \NC ogrevanje zunanjih ogledal \NC\NR
\NC F\,14 \NC 7,5\,A \NC \aW{+\:15} radio in kamera \NC\NR
\NC F\,15 \NC 10\,A \NC \aW{+\:30} opozorilne utripalke \NC\NR
\NC F\,16 \NC 5\,A \NC osvetlitev krmilnega droga \NC\NR
\NC F\,17 \NC 7,5\,A \NC \aW{+\:30} radio, svetlobno stikalo in notranja osvetlitev \NC\NR
\NC F\,18 \NC — \NC {[}prosto{]} \NC\NR
\NC F\,19 \NC 20\,A \NC \aW{+\:30} RC\,12 spredaj \NC\NR
\NC F\,20 \NC 20\,A \NC \aW{+\:30} RC\,12 zadaj \NC\NR
\NC F\,21 \NC 15\,A \NC 12-voltna vtičnica \NC\NR
\NC F\,22 \NC 5\,A \NC ključ za vžig, večfunkcijska konzola, Vpad \NC\NR
\NC F\,23 \NC 5\,A \NC izklop v sili, sredinska konzola, RC\,12 spredaj \NC\NR
\NC F\,24 \NC 5\,A \NC izklop v sili, sredinska konzola, RC\,12 zadaj \NC\NR
\NC F\,25 \NC 2\,A \NC \aW{+\:15} RC\,12 spredaj \NC\NR
\NC F\,26 \NC 2\,A \NC \aW{+\:15} RC\,12 zadaj \NC\NR
\NC F\,27 \NC 25\,A \NC ventilator ogrevanja \NC\NR
\NC F\,28 \NC 10\,A \NC klimatski kompresor, sistem za centralno mazanje \NC\NR
\NC F\,29 \NC 25\,A \NC kondenzator klimatske naprave \NC\NR
\NC F\,30 \NC 5\,A \NC termostat klimatske naprave \NC\NR
\NC F\,31 \NC 5\,A \NC \aW{+\:15} večfunkcijska konzola|/|Vpad \NC\NR
\NC F\,32 \NC 15\,A \NC električna vodna črpalka, rotacijskal luč \NC\NR
\NC F\,33 \NC — \NC {[}prosto{]} \NC\NR
\NC F\,34 \NC — \NC {[}prosto{]} \NC\NR
\NC F\,35 \NC — \NC {[}prosto{]} \NC\NR
\NC F\,36 \NC — \NC {[}prosto{]} \NC\NR
\stoptabulate
\stopcolumns

\page [yes]

\setups [pagestyle:bigmargin]


\subsection[sec:lighting]{Osvetlitev in signalna oprema}


\placefig [here] [fig:lighting] {Osvetlitev in signalna oprema vozila}
{\externalfigure [vhc:electric:lighting]}

\placelegende [margin,none]{}{%
\vskip 30pt
{\sla Legenda:}
\startLongleg
\item Pozicijske luči\hfill 12\,V–5\,W
\item Zasenčene luči\hfill H7~12\,V–55\,W
\item Smernik\hfill oranžen 12\,V–21\,W
\item {\stdfontsemicn Delovni žaromet}\hfill G886~12\,V–55\,W
\item Smernik\hfill 12\,V–21\,W
\item Vzvratne|/|zavorne luči\hfill 12\,V–5|/|21\,W
\item Vzvratni žaromet\hfill 12\,V–21\,W
\item {[}prosto{]}
\item Osvetlitev registrske tablice\hfill 12\,V–5\,W
\item Rotacijska luč\hfill H1~12\,V–55\,W
\stopLongleg}

\subsubsubject{Nastavitev žarometov}

\placefig [margin] [fig:lighting:adjustment] {Svetlobni žarek pri 5\,m}
{\externalfigure [vhc:lighting:adjustment]
\startitemize
\sym{H\low{1}} Višina svetlobne niti: 100\,cm
\sym{H\low{2}} Popravek pri 2\hairspace\%: 10\,cm
\stopitemize}

{\md Pogoji:} Posoda za svežo|/|reciklirano vodo je polna, voznik je za krmilom.

Žarometi so tovarniško poravnani. Višino in naklon svetlobnega žarka je mogoče nastaviti z obračanjem plastičnega držala.

Če se v sklopu pregleda izkaže, da je treba nastavitev spremeniti,
odvijte varnostni vijak in popravite nagib tako,
da bo ustrezal zakonskim predpisom (glejte
\in{sl.}[fig:lighting:adjustment]). Ponovno privijte varnostni vijak.

\page [yes]
\setups [pagestyle:marginless]


\subsection[sec:battcheck]{Baterija }

\subsubsection{Varnostni predpisi}

\startSymList
\PPfire
\SymList
\textDescrHead{Nevarnost eksplozije}
Pri\index{baterija+varnostni predpisi}\index{nevarnost+eksplozija} polnjenju
baterij se tvori eksplozivni\index{pokalni plin} pokalni plin. Baterije
polnite samo v dobro prezračevanih prostorih! Preprečite iskrenje!
V bližini baterije ne ravnajte z ognjem, odprto svetlobo in ne
kadite.
\stopSymList

\startSymList
\PHvoltage
\SymList
\textDescrHead{Nevarnost kratkega stika}
Če se\index{baterija+vzdrževanje} pozitivna sponka priključene baterije dotakne
delov vozila,
obstaja\index{nevarnost+požar}\index{nevarnost+eksplozija} nevarnost kratkega stika.
Pri tem lahko pride do eksplozije mešanice plinov, ki uhaja iz baterije, in
lahko pride do hudih telesnih poškodb oseb v bližini.

\startitemize
\item Na baterijo ne odlagajte nobenih kovinskih predmetov ali orodja.
\item Pri odklapljanju baterije vedno najprej odklopite negativno
in nato pozitivno sponko.
\item Pri priklapljanju baterije vedno najprej priklopite pozitivno
in nato negativno sponko.
\item Pri vključenem motorju ne rahljajte ali snemite priključnih sponk
baterije.
\stopitemize
\stopSymList


\startSymList
\PHcorrosive
\SymList
\textDescrHead{Nevarnost poškodb}
Uporabljajte\index{nevarnost+razjede} zaščitna očala in zaščitne rokavice,
odporne proti kislini. Baterijska tekočina je pribl. 27-odstotna
žveplova kislina (H\low{2}SO\low{4}) in lahko zato povzroči razjede.
Nevtralizirajte\index{baterija+nevarnost}\index{baterijska+tekočina}
baterijsko tekočino, ki vam pride na kožo, z raztopino iz
natrona z dvojnim ogljikovim dioksidom in jo sperite s čisto vodo.  Če
vam baterijska tekočina pride v oči, jih sperite z veliko hladne
vode in takoj pojdite k zdravniku.
\stopSymList

\startSymList
\startcombination[1*2]
 {\PHcorrosive}{}
 {\PHfire}{}
 \stopcombination
\SymList
\textDescrHead{Skladiščenje baterij}
Baterije\index{baterija+skladiščenje} vedno skladiščite pokončno. Drugače lahko
pride do iztekanja baterijske tekočine in razjed ali, pri reakciji z
drugimi snovmi~, do požarov. \par\null\par\null
\stopSymList

\testpage [16]

\starttextbackground [FC]
\setupparagraphs [PictPar][1][width=2.4em,inner=\hfill]

\startPictPar
\PMproteyes
\PictPar
\textDescrHead{Zaščitna očala}
Pri\index{nevarnost+poškodbe oči} mešanju vode in kisline vam lahko
tekočina brizgne v oči. Če vam kislina brizgne v oči, si jih
takoj sperite s čisto vodo in nemudoma pojdite k zdravniku!
\stopPictPar
\blank [small]

\startPictPar
\PMrtfm
\PictPar
\textDescrHead{Dokumentacija}
Pri ravnanju z baterijami je treba obvezno upoštevati varnostne napotke,
zaščitne ukrepe in načine postopanja, opisane v teh navodilih
za uporabo.
\stopPictPar
\blank [small]

\startPictPar
\PStrash
\PictPar
\textDescrHead{Varstvo okolja}
Baterije\index{varstvo okolja} vsebujejo škodljive snovi. Starih baterij
nikoli ne mečite med gospodinjske odpadke. Baterije odstranite na okolju prijazen način. Vrnite jih v
delavnico ali jih odnesite na zbirališče za stare baterije.

Polne baterije vedno transportirajte in skladiščite pokončno. Pri transportu je treba
baterije zavarovati, da se ne morejo prevrniti. Iz odzračevalnih odprtin zapornih
čepov lahko izteka baterijska tekočina in pride v okolje.
\stopPictPar
\stoptextbackground

\page [yes]

\setups[pagestyle:normal]


\subsubsection{Praktični nasveti}

Za najdaljšo življenjsko dobo baterijo mora ta biti po možnosti vedno povsem napolnjena.

Pri daljšem mirovanju vozila lahko z\index{baterija+življenjska doba} vzdrževalnim polnjenjem baterije
podaljšate življenjsko dobo baterije in
celo zagotovite stalno zagonsko pripravljenost vozila.

\placefig[margin][fig:batterycompartment]{\select{caption}{prostor za vgradnjo baterije (vzdrževalna loputa)}{prostor za vgradnjo baterije}}
{\externalfigure[batt:compartment]}


\subsubsection{Servisiranje}

Pri bateriji vozila \sdeux\ gre za svinčeni akumulator, ki {\em ne potrebuje vzdrževanja}. Razen vzdrževanja napolnjenega stanja in čiščenja niso potrebni nadaljnji vzdrževalni ukrepi na bateriji.

\startitemize
\item Pazite, da bodo poli baterije vedno čisti in suhi. Pole na rahlo namažite z nekaj masti, ki odbija umazanijo.
\item Baterije, ki imajo\index{baterija+polnjenje} mirovni tok
nižji\index{baterija+mirovni tok} od 12,4\,V, napolnite.
\stopitemize

\placefig[margin][fig:bclean]{Čiščenje polov}
{\externalfigure[batt:clean]
\noteF
Uporabite\index{baterija+čiščenje}\index{čiščenje+baterij} toplo
vodo, da odstranite beli prašek, ki nastane zaradi korozije. Če
je kateri pol zarjavel, odklopite baterijski kabel in pol
očistite z žičnato krtačo. Nato pol premažite s tankim slojem masti.}


\subsubsection[sec:battery:switch]{Uporaba ločilnega stikala baterije}

{\sl Redna uporaba ločilnega stikala baterije ni priporočljiva (npr. dnevno)!}

\startSteps
\item Izključite\index{ločilno stikalo baterije} vžig in nato počakajte pribl. 1~minuto.
\item Odprite prostor za vgradnjo baterije (\inF[fig:batterycompartment]).
\item Pritisnite na rdeči gumb ločilnega stikala baterije, da prekinete električni tokokrog.
\item Za ponovni priklop električnega tokokroga obrnite ločilno stikalo baterije za ¼~obrata v smeri urinega kazalca.
\stopSteps

\stopsection

\page [yes]


\setups[pagestyle:marginless]

\section[sec:cleaning]{Čiščenje vozila}

Pred dejanskim čiščenjem sperite\startregister[index][vhc:lavage]{vzdrževanje+čiščenje}
grobo blato in umazanijo s karoserije z
vodo. Pri tem ne očistite le stranskih površin, temveč tudi
ohišje koles in spodnjo stran vozila.

Temeljito pranje vozila je pomembno predvsem pozimi, da z njega
odstranite visoko korozivne\index{korozija+preprečevanje} ostanke soli za posipanje cest.

\starttextbackground [FC]
\startPictPar
\PHgeneric
\PictPar
\textDescrHead{Preprečite škodo zaradi vode}
Vozila nikoli ne čistite z {\em vodnimi topovi} (\eG\
gasilcev) ali {\em napravami za hladno čiščenje na osnovi ogljikovodika.} Pri uporabi
visokotlačnega parnega čistilnika upoštevajte ustrezne
predpise v nadaljevanju.
\stopPictPar
\blank[small]

\startPictPar
\pTwo[monde]
\PictPar
\textDescrHead{Varstvo okolja}
Čiščenje vozila lahko vodi do velike obremenitve okolja.
Vozilo čistite samo na mestu, \index{varstvo okolja}
opremljenem z izločevalnikom olja. Upoštevajte veljavna
določila za varstvo okolja.
\stopPictPar
\blank[small]

\startPictPar
\PMwarranty
\PictPar
\textDescrHead{Čistite strokovno!}
Za škodo, nastalo zaradi neupoštevanja predpisov za čiščenje,
ni mogoče do podjetja \BosFull{boschung} uveljavljati pravic iz jamstva ali
garancije.
\stopPictPar
\stoptextbackground


\subsection{Visokotlačno čiščenje}

Za visokotlačno čiščenje\index{čiščenje+visoki tlak} vozila je primerna
običajna visokotlačna čistilna naprava.

Pri visokotlačnem čiščenju je treba upoštevati naslednje točke:

\startitemize
	\item Najv. delovni tlak 50\,bar
	\item Šoba s ploskim curkom in brizgalnim kotom 25°
	\item Najm. brizgalni razmik 80\,cm
	\item Najv. temperatura vode 40\,°C
	\item Upoštevajte odsek \about[reiMi], \atpage[reiMi].
\stopitemize

Pri neupoštevanju teh\index{lak+škoda} predpisov lahko pride do poškodb
laka in protikorozijske zaščite\index{škoda+lak}.

Upoštevajte tudi navodila za uporabo in varnostne predpise
visokotlačne čistilne naprave.

\starttextbackground[FC]
\startPictPar\PPspray\PictPar
Pri visokotlačnem čiščenju lahko pride do vdora vode in
posledične škode. Zato vodnega curka nikoli ne usmerjajte na občutljive
dele in naprave:
\stopPictPar

\startitemize
	\item tipala, električne povezave in priključki
	\item zaganjalnik, generator, vbrizgalni sistem
	\item magnetni ventili
	\item prezračevalne odprtine
	\item še ne ohlajene mehanske komponente
	\item nalepke z napotki, opozorili in nevarnostmi
	\item elektronske krmilne naprave
\stopitemize

\textDescrHead{Pranje motorja}
Obvezno preprečite vod vode v sesalne, prezračevalne in
odzračevalne odprtine. Pri visokotlačnem čiščenju curka nikoli ne usmerite neposredno
na električne dele in vode. Curka ne usmerjajte na vbrizgalno
napravo! Po pranju motorja le-tega konservirajte, pri čemer jermen
zaščitite pred sredstvi za konserviranje.
\stoptextbackground

\starttextbackground [FC]
\setupparagraphs [PictPar][1][width=6em,inner=\hfill]
\startPictPar
\framed[frame=off,offset=none]{\PMproteyes\PMprotears}
\PictPar
\textDescrHead{Preostala voda}
	Med čiščenjem se na določenih mestih vozila kopiči
	voda (\eG\ v kotanjah motornega bloka ali menjalnika); to je treba
	odstraniti s pomočjo stisnjenega zraka. Upoštevajte, da je treba pri
	ravnanju s stisnjenim zrakom nositi ustrezno zaščitno opremo in da
	mora naprava ustrezati veljavnim varnostnim predpisom (več šob).
\stopPictPar
\stoptextbackground


\subsubsection[reiMi]{Primerna čistilna sredstva}

Uporabljajte\index{čistilna sredstva} izključno čistilna sredstva, ki imajo
naslednje lastnosti:

\startitemize
	\item ne drgnejo
	\item pH-vrednost 6–7
	\item brez topila
\stopitemize

Za odstranjevanje trdovratnih madežev lahko na majhnih madežih
premišljeno uporabite pralni bencin ali špirit, nikakor pa drugih topil.
Odstranite ostanke topila z laka. čiščenje
plastičnih delov z bencinom lahko vodi do razpok ali razbarvanja!

Površine z\index{čiščenje+nalepke} opozorilnimi nalepkami ali
nalepkami z napotki čistite s čisto vodo in mehko gobo.

Preprečite vdor vode v električne komponente: Za čiščenje
ohišja smernikov in luči ne uporabljajte krtače za avto, ampak
mehko krpo ali gobo.

\starttextbackground [CB]
\startPictPar
\GHSgeneric\par
\GHSfire
\PictPar
\textDescrHead{Nevarnost zaradi kemikalij}
Čistilna sredstva predstavljajo določena zdravstvena in varnostna tveganja (hitro
gorljive snovi). Upoštevajte varnostne predpise, ki veljajo
za uporabljeno čistilno sredstvo; upoštevajte podatkovne
liste in liste z nevarnostmi uporabljenih sredstev.
\stopPictPar
\stoptextbackground

\stopregister[index][vhc:lavage]


\page [yes]


\setups [pagestyle:bigmargin]

\startsection	[title={Nastavitev sesalnega ustja},
				 reference={sec:main:suctionMouth}]


Optimalna razdalja\index{sesalno ustje+nastavitev} med površino ceste in plastično tirnico sesalnega ustja je 8\,mm.
Da preverite oz. nastavite razdaljo, uporabite tri nastavitvena merila, ki jih najdete v zaboju za orodje (voznikova kabina, voznikova stran).


\placefig [margin] [fig:suctionMouth] {Nastavitev sesalnega ustja}
{\Framed{\externalfigure [suctionMouth:adjust]}}

\placeNote[][service_picto]{}{%
\noteF
\starttextrule{\PHasphyxie\enskip Nevarnost zastrupitve in zadušitve \enskip}
{\md Napotek:} Med nastavitvenimi deli mora biti motor vozila vključen, da lahko drži sesalno ustje v lebdečem položaju. Da izključite nevarnost zastrupitve ali zadušitve, je treba zato obvezno uporabljati odsesovalni sistem za izpušne pline oz. dela izvajati izključno na zelo dobro prezračevanem mestu.
\stoptextrule}

\startSteps
\item Vozilo parkirajte na dobro prezračevanem mestu na vodoravno in plosko površino.
\item Vključite\index{odsesovanje} \aW{delovni} način (pritisnite gumb na zunanji strani izbirne ročice za stopnjo vožnje).

Motor pustite delovati v prostem teku. (Pritisnite tipko~\textSymb{joy_key_engine_decrease} na večfunkcijski konzoli, da zmanjšate število vrtljajev motorja.)
\item Zategnite ročno zavoro in zadnja kolesa podložite s klado.
\item Pritisnite tipko~\textSymb{joy_key_suction}, da spustite sesalno ustje.
\item Nastavitvena merila~\LAa\  postavite pod plastično vodilo sesalnega ustja, kot je prikazano na sliki.
\item [sucMouth:adjust]Odvijte pritrdilne~\Lone\  in nastavitvene vijake~\Ltwo\ vsakega kolesa, da se bodo spustila na tla.
\item Ponovno privijte vijake~\Lone\  in ~\Ltwo\  in odstranite nastavitvena merila.
\item Dvignite|/|spustite sesalno ustje in z nastavitvenimi merili preverite nastavitev. Če nastavitev še ni povsem v redu, ponovite nastavitveni postopek od točke~\in[sucMouth:adjust].

\stopSteps


\stopsection
\stopchapter
\stopcomponent

