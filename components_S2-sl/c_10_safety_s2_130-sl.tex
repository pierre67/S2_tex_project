\startcomponent c_10_safety_s2_130-sl
\product prd_ba_s2_130-sl

\marking[chapter]{Varnostni znaki}


\chapter{Varnostni znaki}

\setups[pagestyle:marginless]

\section{Nova evropska označba za nevarne snovi}

{\em Rombaste oblike z belim ozadjem in rdečim
robom.}\par\blank[1*medium]
{\em Od leta 2008 velja v EU t. i.
uredba CLP\index{Uredba CLP} z novimi opozorilnimi znaki za nevarne
snovi in izdelke.}\par\null

\startSymList \GHSgeneric
\SymList
	\textDescrHead{Nevarnost za zdravje}
	Opozarja pred\index{nevarnost za zdravje} nevarnostmi za zdravje, ki povzročijo
	smrt ali hude zdravstvene težave. Sem sodijo
	draženje kože in alergične reakcije. Simbol se uporablja tudi kot opozorilo
	pred drugimi nevarnostmi, kot so vnetja.\par
	Nadomesti:\crlf \HAZOcross\ ali \HAZOpoison\ ali \PHgeneric
\stopSymList

\startSymList \GHSbody
\SymList
	\textDescrHead{Hude nevarnosti za zdravje; predvsem pri otrocih lahko
	vodijo tudi do smrti}
	Izdelki lahko povzročijo hude zdravstvene težave. Ta simbol opozarja
	tudi pred nevarnostmi\index{nevarnost+nosečnost} za nosečnost,
	rakotvornimi učinki\index{nevarnost+rakotvorne snovi} in podobno
	hudimi tveganji za zdravje. Izdelke je treba uporabljati previdno.\par
	Nadomesti:\crlf \HAZOcross\ ali \HAZOpoison\
\stopSymList

\startSymList \GHSbomb
\SymList
	\textDescrHead{Eksplozivne snovi}
	Nestabilne eksplozivne\index{nevarnost+eksplozija} snovi,
	mešanice in produkti v eksplozivnih snoveh\index{eksplozivne snovi} imajo
	pri svoji reakciji močno ekspanzijski učinek, ki lahko povzroči veliko
	uničenje; pri nepravilnem ravnanju obstaja življenjska nevarnost.\par
	Nadomesti:\crlf \HAZObomb\
\stopSymList


\startSymList \GHSpoison
\SymList
	\textDescrHead{Zastrupitev}
	Izdelki\index{nevarnost+zastrupitev} lahko že v majhni količini na koži,
	pri vdihavanju\index{strupene snovi} ali zaužitju povzročijo hude ali celo
	smrtne zastrupitve. Ne dovolite neposrednega stika.\par
	Nadomesti:\crlf \HAZOpoison\
\stopSymList

\startSymList \GHSfire
\SymList
	\textDescrHead{Hitro ali zelo vnetljivo}
	Izdelki\index{nevarnost+ogenj} se v bližini vročine ali plamena
	hitro zanetijo. Razpršil s to oznako nikakor ni dovoljeno pršiti na
	vroče površini ali v bližini ognja.\par
	Nadomesti:\crlf \HAZOfire\ ali \HAZOfirebis\
\stopSymList

\startSymList \GHSenvironment
\SymList
	\textDescrHead{Nevarnost za živali in okolje}
	Izdelki\index{varstvo okolja} lahko v okolju povzročijo
	dolgoročno ali kratkoročno škodo\index{strupene snovi}. Organizme,
	ki živijo v vodi (\eG\ ribe) lahko ubijejo in imajo zelo dolgoročne
	škodljive vplive na okolje. V nobenem primeru ne zlivajte v odpadne vode ali gospodinjske odpadke!\par
	Nadomesti:\crlf \HAZOenvironment\
\stopSymList

\startSymList \GHScorrosive
\SymList
	\textDescrHead{Nevarnost za kožo ali oči}
	Izdelki\index{nevarnost+poškodbe kože}\index{nevarnost+poškodbe oči} lahko že po kratkem
	stiku poškodujejo kožo ali povzročijo nastanek brazgotin oz. trajne
	poškodbe oči. Pri uporabi si zaščitite kožo in oči!\par
	Nadomesti:\crlf \HAZOcross\ ali \HAZOcorrosive
\stopSymList

\page [yes]


\section{Opozorilni znak}

{\em Črni napis na rumenem ozadju}\par\null

\startSymList \PHgeneric
\SymList
	\textDescrHead{Splošni opozorilni znak}
	Opozarja\index{nevarnost+splošno}\index{opozorilni znak} na neposredno
	pretečo nevarnost, pri kateri lahko pride do telesnih poškodb.
	\crlf\null
\stopSymList

\startSymList \PHpoison
\SymList
	\textDescrHead{Opozorilo pred strupenimi snovmi}
	Strupene snovi\index{nevarnost+zastrupitev} lahko
	pri stiku s kožo, vdihavanju ali zaužitju povzročijo akutne
	ali celo kronične hude zdravstvene težave ali celo smrt.
\stopSymList

\startSymList \PHfire
\SymList
	\textDescrHead{Opozorilo pred snovmi, nevarnimi za požar}
	Preprečite odprt ogenj in iskrenje\index{nevarnost+ogenj}. Snov
	je hitro vnetljiva ali pa lahko pospešuje ogenj. Prepovedano kaditi!
\stopSymList

\startSymList \PHexplosive
\SymList
	\textDescrHead{Opozorilo pred eksplozijsko nevarnimi snovmi}
	Trde, tekoče ali gelaste snovi ali pripravki, ki lahko ob učinkovanju
	udarcev, trenja, ognja, vroče ipd.\,,
	eksplodirajo.\index{nevarnost+eksplozija} Prepovedano kaditi!
\stopSymList

\startSymList \PHcrushing
\SymList
	\textDescrHead{Opozorilo zaradi nevarnosti zmečkanin}
	Opozarja na območje\index{nevarnost+zmečkanina}, v katerem zaradi
	premikajočih se mehanskih delov obstaja nevarnost zmečkanin. Ne približujte se tem območjem,
	dokler je naprava vključena.
\stopSymList

\startSymList \PHhand
\SymList
	\textDescrHead{Opozorilo pred poškodbami rok}
	Med nagibanjem voznikove kabine ali nakladalnega mostu obstaja nevarnost, da\index{nevarnost+zmečkanine} vam zmečka
	dlani ali druge dele\index{nevarnost+poškodbe rok} telesa.
	.
\stopSymList

\startSymList \PHentangle
\SymList
	\textDescrHead{Opozorilo pred nasproti vrtečimi se kolesi/nevarnostjo vpotega}
	Obstaja nevarnost, da vam vrteči se deli\index{nevarnos+vpoteg} vpotegnejo
	okončine. Ne približujte se tem delom,
	dokler je naprava vključena.
\stopSymList

\startSymList \PHcorrosive
\SymList
	\textDescrHead{Opozorilo pred jedkimi snovmi}
	Previdno\index{nevarnost+jedke snovi} ravnajte in uporabljajte ustrezno
	osebno zaščitno opremo (rokavice, zaščitna očala, zaščitna oblačila).
\stopSymList

\startSymList \PHhot
\SymList
	\textDescrHead{Opozorilo pred vročo površino}
	Delu ali pripravi\index{nevarnost+opekline}
	se ne približujte, če nimate zadostnega znanja; nosite rokavice.
\stopSymList

\startSymList \PHvoltage
\SymList
	\textDescrHead{Opozorilo pred nevarno električno napetostjo}
	S kovinskimi predmeti\index{nevarnost+se ne dotikajte električne napetosti}.
	Nevarnost poškodb ali opeklin pri kratkem stiku!
\stopSymList

\startSymList \PHfalling
\SymList
	\textDescrHead{Opozorilo pred nevarnostjo padca}
	Na tem območju bodite še posebej previdni\index{nevarnost+padec}
	in nosite ustrezno obutev (s protizdrsnimi podplati, odpornimi proti ogljikovodiku).
\stopSymList

\startSymList \PHbattery
\SymList
	\textDescrHead{Opozorilo pred nevarnostjo zaradi baterij} Opozarja na nevarnosti, ki
	se pojavijo pri polnjenju baterij (svinčenega akumulatorja)\index{nevarnost+baterija},
	predvsem zaradi uhajanja vodikovega plina ter žveplove kisline, ki
	je v bateriji.
\stopSymList

\startSymList \PHremote
\SymList
	\textDescrHead{Opozorilo pred samodejnim zagonom}
	Opozarja pred\index{nevarnost+samodejni zagon} možnim samodejnim ali
	daljinsko upravljalnim zagonom naprave.
\stopSymList

% \startSymList \PHquetschgefahr
% \SymList
% \textDescrHead{Risque d’écrasement}
% Risque d’écrasement\index{risque d’écrasement}.
% \stopSymList
% % NOTE: Doppelt! (auch Bilddatei)
%
% % NOTE: Evtl. Folgendes als Ersatz für oben?

% \startSymList\PHhandcrushed
% \SymList
	% \textDescrHead{Gefahr von Handquetschungen}
	% Es besteht\index{Gefahr+Quetschung} die Gefahr, dass Hände oder Finger
	% gequetscht werden. Nähern Sie die Hände nicht an, ohne die Gefahr
	% identifiziert und beseitigt zu haben.
% \stopSymList

\startSymList \PHhandfoot
\SymList
\textDescrHead{Opozorilo pred premikajočimi se deli}
Opozarja pred premikajočimi se deli\index{Nevarnost+Premikajoči se deli} stroja|/|vozila.
\stopSymList

\startSymList \PHnarrowed
\SymList
	\textDescrHead{Opozorilo pred zožanjem proge}
	Zožanje\index{nevarnost+širina vozila} vozišče.
	% Denken Sie an die Breite des Fahrzeugs.
\stopSymList

\page [yes]


\section{Znaki prepovedi}

{\em Okrogli z belim ozadjem, rdečim robom in diagonalno črto}
\par\null


\startSymList \PPfire
\SymList
	\textDescrHead{Prepoved ognja, odprte svetlobe in kajenja} Odprti
	ogenj\index{prepoved+kajenje, ogenj} in žerjavica vsake oblike so prepovedani (\eG\
	goreče cigarete, vžigalice, sveče, tudi kakršno koli iskrenje).
\stopSymList

\startSymList \PPentry
\SymList
	\textDescrHead{Dostop nepooblaščenim prepovedan}
	Nepooblaščene\index{prepoved+dostop} osebe nimajo dostopa
	na to območje oz. se mu ne smejo približati.
\stopSymList

\startSymList \PPphone
\SymList
	\textDescrHead{Prepovedana uporaba mobilnih telefonov}
	Mobilne telefone\index{prepoved+mobilni telefon} in vse naprave, ki
	oddajajo elektromagnetno sevanje, je treba izključiti. Elektromagnetno
	sevanje lahko povzroči motnje v delovanju
	elektronike naprave.
\stopSymList

\startSymList \PPspray
\SymList
	\textDescrHead{Prepovedano brizganje z vodo}
	Vodnega ali parnega curka\index{prepoved+vodni curek, para} nikoli
	ne usmerjajte na občutljive del ali naprave (\eG\ tipala, krmilniki,
	vbrizgalne naprave itd.).
\stopSymList

\startSymList \PPchildren
\SymList
	\textDescrHead{Otrokom ne dovoliti v bližino}
	Opozorilo\index{prepoved+otroci} na posebno nevarnost za otroke. Splošno
	velja: otroci se ne smejo približati vključenemu stroju,
	tudi ne med vzdrževanjem.
\stopSymList

\startSymList \PPwater
\SymList
	\textDescrHead{Ni pitna voda}
	Ne piti vode iz rezervoarja\index{prepoved+ni pitna voda}. Obstaja
	nevarnost zastrupitve.
\stopSymList

% \page [yes]


\section{Znak za varstvo okolja}

\startSymList \PSrecycle
\SymList
	\textDescrHead{Recikliranje}
	Posebni predpisi za pravilno odlaganje določenih odpadkov.
\stopSymList

\startSymList \PSwelt
\SymList
	\textDescrHead{Varstvo okolja}
	Opozorilo na veljavna določila za varstvo okolja.
\stopSymList

\startSymList \PStrash[width=\PictoHeight,height=,]
\SymList
	\textDescrHead{Pravilno odlagajte odpadke}
	Za določene odpadke, \eG\ svinčene akumulatorje veljajo posebni
	predpisi za odlaganje.
\stopSymList


\testpage[12]


\section{Znaki zapovedi}


{\em Okrogli z modrim ozadjem}\par\null

\startSymList \PMgeneric
\SymList
	\textDescrHead{Splošni znaki zapovedi}
	Ta znak se lahko uporablja samo v povezavi z dodatnim znakom, ki
	natančno določa zapoved.
\stopSymList


\startSymList \PMrtfm
\SymList
	\textDescrHead{Upoštevajte navodila za uporabo}
	Pred začetkom uporabe obvezno preberite\index{prebrati navodila za uporabo}
	napotke o tej temi, določeni napravi
	ali izdelku. Navodila za uporabo morajo biti spravljena v voznikovi
	kabini.
\stopSymList

\startSymList \PMproteyes
\SymList
	\textDescrHead{Uporabljajte zaščito za oči}
	Pri delih, pri katerih obstaja nevarnost poškodb oči, je treba
	nositi zaščito za oči\index{zaščita za oči}.
\stopSymList

\startSymList \PMprothands
\SymList
	\textDescrHead{Uporabljajte zaščito za roke}
	Pri delih, pri katerih lahko pride do poškodb rok, je treba
	nositi zaščitne rokavice\index{uporaba zaščite za roke}.
\stopSymList

\startSymList \PMprotears
\SymList
	\textDescrHead{Uporabljajte glušnike}
	Uporabljati je treba glušnike\index{nevarnost+sluh} (\eG v bližini
	vključenega ventilatorja ali turbine).
\stopSymList

\startSymList \PMsafetybelt
\SymList
	\textDescrHead{Uporabljajte varnostni pas} Zaradi lastne varnosti\index{varnostni pas} si pripnite
	varnostni pas.
\stopSymList

\section{Dodatni znaki}

% \adaptlayout[height=+5mm]                                                 {{{

% \startSymList \SETshoe
% \SymList
% \textDescrHead{Port de chaussures de sécurité obligatoire}
% Le port de chaussures de sécurité est obligatoire\index{chaussures de sécurité}.
% \stopSymList
%
% \startSymList \SETglasses
% \SymList
% \textDescrHead{Port de lunettes des protection obligatoire}
% Le port de lunettes est obligatoire\index{lunette de protection}.
% \stopSymList
%
% \startSymList \SEToreillettes
% \SymList
% \textDescrHead{Port de casque obligatoire}
% Le port d’un casque de protection est \index{casque} obligatoire.
% \stopSymList
%
% \startSymList \SETgloves
% \SymList
% \textDescrHead{Port de gants de protection obligatoire}
% Le port de gants de protection est obligatoire\index{gants}.
% \stopSymList
%
% \startSymList \SETmainecrase
% \SymList
% \textDescrHead{Risque d’écrasement}
% Danger pour les mains\index{écrasement} et les pieds.
% \stopSymList
%
% \startSymList \SETgetriebe
% \SymList
% \textDescrHead{Risque de happement}
% Risque de happement par\index{happement} des pièces en rotation.
% \stopSymList
%
% \startSymList \SETradkeil
% \SymList
% \textDescrHead{Cale de roue}
% Sécuriser le véhicule contre toute mise\index{Cale de roue} en marche involontaire.
% \stopSymList
%}}}

\startSymList \SETfirstaid
\SymList
\textDescrHead{Prva pomoč}
	Prikazuje mesto, kjer je spravljena oprema za prvo pomoč. Pomemben del prve pomoči
	je hitro obveščanje reševalne službe.\index{Prva
	pomoč}\index{Klic v sili} Sem vnesite svoje številke za klic v sili:
	\fillinrules[n=1]{\bf
	\framed[align=right,frame=off,offset=none,width=30mm]{Reševalna služba}}
\fillinrules[n=1]{\bf
\framed[align=right,frame=off,offset=none,width=30mm]{Policija}}
\fillinrules[n=1]{\bf
\framed[align=right,frame=off,offset=none,width=30mm]{Gasilci}}
\stopSymList

\startSymList \SETbrandschutzzeichen
\SymList
\textDescrHead{Gasilni aparat}
	Določene naprave so opremljene z enim ali več gasilnimi aparati\index{gasilni aparat}
	. Ti praviloma potrebujejo posebno vzdrževanje; nadaljnje
	informacije o tem najdete na napravi ali v navodilih za
	uporabo naprave.
\stopSymList


\page[yes]

\section{Trije koraki nudenja pomoči}
% NOTE [tf]: Shouldn't be in this book, IMO

\starttextbackground [CB]
\textDescrHead{Zavarujte mesto nesreče in prizadete osebe.}
\startitemize
\item  Preverite varnost mesta nesreče in zagotovite, da ne more priti do dodatnih nevarnosti.
\stopitemize
\textDescrHead{Ocenite stanje poškodovanih.}
\startitemize
\item  Preverite, ali so poškodovani pri zavesti in normalno dihajo.
Po potrebi jim sprostite dihalne poti.
\stopitemize
\textDescrHead{Obvestite reševalno službo.}
\startitemize Vaš klic v sili mora zajemati naslednje informacije:\par
	\item telefonsko številko, na kateri ste dosegljivi;
	\item vrsto nesreče (bolezen, nesreča);
	\item obstoječa tveganja (Požar, eksplozija, nevarnost porušenja itd.);
	\item natančno mesto dogodka;
	\item število poškodovanih in njihova stanja;
	\item že uvedeni reševalni ukrepi.
	\item Odgovorite na ostala vprašanja, ki vam jih morebiti zastavijo.
\stopitemize
\stoptextbackground

\stopcomponent

