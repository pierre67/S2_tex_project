\startcomponent c_60_work_s2_130-sl
\product prd_ba_s2_130-sl


\startchapter [title={S2 v vsakdanu},
							reference={chap:using}]

\setups [pagestyle:marginless]


% \placefig[margin][fig:ignition:key]{Clé de contact}
% {\externalfigure [work:ignition:key]}
\startregister[index][chap:using]{Zagon}

\startsection [title={Zagon},
							reference={sec:using:start}]


\startSteps
\item Zagotovite, da so bili redni pregledi in vzdrževanja izvedene po predpisih.
\item S ključem za vžig zaženite motor: vključite vžig, nato ključ obrnite naprej v desno in ga zadržite, dokler se motor ne zažene (mogoče samo, če je izbirna ročica za stopnjo vožnje na nevtralnem položaju).
\stopSteps

\start
\setupcombinations [width=\textwidth]

\placefig[here][fig:select:drive]{Izbirna ročica za stopnjo vožnje}
{\startcombination [2*1]
{\externalfigure [work:select:fDrive]}{Izbirna ročica na položaju {\em vožnja naprej}}
{\externalfigure [work:select:rDrive]}{Izbirna ročica na položaju {\em vzvratna vožnja}}
\stopcombination}
\stop


\startSteps [continue]
\item Obrnite stikalo izbirne ročice za stopnjo vožnje, da v načinu {\em vožnje} prestavite v stopnjo vožnje:
\startitemize [R]
\item prva stopnja
\item druga stopnja (samodejno obratovanje; samodejni zagon po prvi stopnji)
\stopitemize

ali pa pritisnite gumb na zunanji strani ročice, da vključite|/|izključite {\em delovni} način.
\stopSteps

\startbuffer [work:config]
\starttextbackground [FC]
\startPictPar
\PMrtfm
\PictPar
V delovnem načinu je na voljo samo prva stopnja vožnje, motor pa se vrti s 1300\,min\high{\textminus 1}.

S tipkama~\textSymb{joy_key_engine_increase} in~\textSymb{joy_key_engine_decrease} na večfunkcijski konzoli lahko upravljate število vrtljajev motorja.
\stopPictPar
\stoptextbackground
\stopbuffer

\getbuffer [work:config]

\startSteps [continue]
\item Potisnite izbirno ročico za stopnjo vožnje navzgor in naprej (vožnja naprej) oz. navzgor in nazaj (vzvratna vožnja). Glejte zgornje slike.
\item Preden pospešite, sprostite ročno zavoro.
\stopSteps

\starttextbackground [FC]
\startPictPar
\PMrtfm
\PictPar
{\md Ročno zavoro je treba povsem sprostiti!} Položaj ročice za ročno zavoro nadzira elektronsko tipalo: če ročna zavora ni povsem sproščena, je hitrost vožnje omejena na 5\,km/h.
\stopPictPar
\stoptextbackground

\startSteps [continue]
\item Počasi pritisnite na stopalko za vožnjo, da speljete vozilo.
\stopSteps


%% NOTE: New text [2014-04-29]:
\subsection [sSec:suctionClap] {Loputa sesalnega kanala}

Sesalni sistem proizvaja zračni rok ali od sesalnega ustja ali ročne sesalne gibke cevi (dodatna možnost) do posode za umazanijo.

Loputa, ki jo je treba prestaviti ročno(\inF[fig:suctionClap], \atpage[fig:suctionClap]), omogoča preklop zračnega toka med sesalnim ustjem in ročno sesalno gibko cevjo.

\placefig [here] [fig:suctionClap] {Loputa sesalnega kanala}
{\startcombination [2*1]
{\externalfigure [work:suctionClap:open]}{Odprt sesalni kanal}
{\externalfigure [work:suctionClap:closed]}{Zaprt sesalni kanal}
\stopcombination}

V normalnem obratovanju~– dela s sesalnim ustjem~– mora biti sesalni kanal odprt (preklopna ročica kaže navzgor).

Da lahko uporabite ročno sesalno gibko cev, mora biti sesalni kanal zaprt (preklopna ročica kaže navzdol). Na ta način je zračni tok usmerjen skozi ročno sesalno gibko cev.
%% End new text

\stopsection


\startsection [title={Izklop},
							reference={sec:using:stop}]

\index{Izklop}

\startSteps
\item Aktivirajte ročno zavoro (ročica med sedežema) in izbirno ročico za stopnjo vožnje postavite na {\em nevtralni} položaj.
\item Izvedite potrebna kontrolna dela,~ dnevne in morebitne tedenske kontrole~, kot je opisano na \atpage[table:scheduledaily].
\stopSteps

\getbuffer [prescription:handbrake]

\stopsection


\startsection [title={Pometanje in sesanje},
							reference={sec:using:work}]

\startSteps
\item Zagon\index{pometanje} vozila izvedite, kot je opisano v  \in{§}[sec:using:start], \atpage[sec:using:start].
\item Aktivirajte\index{sesanje} {\em delovni }način (gumb na zunanji strani izbirne ročice za stopnjo vožnje).
\stopSteps

% \getbuffer [work:config]
%% NOTE: outcommented by PB

\startSteps [continue]
\item Pritisnite tipko~\textSymb{joy_key_suction_brush}, da vključite turbino in metle.

{\md Različica:} {\lt Pritisnite tipko~\textSymb{joy_key_suction}, če želite delati samo s sesalnim ustjem.}

\item S pomočjo tipk~\textSymb{joy_key_frontbrush_increase}\textSymb{joy_key_frontbrush_decrease} večfunkcijske konzole nastavite hitrost vrtljajev metel.

\item S pomočjo ustrezne krmilne ročice metle postavite na položaj za optimalno delovno širino.
\stopSteps

\vfill

\start
\setupcombinations [width=\textwidth]

\placefig[here][fig:brush:position]{Postavitev metel}
{\startcombination [2*1]
{\externalfigure [work:brushes:enlarge]}{Metla navzven|/|navznoter}
{\externalfigure [work:brush:left:raise]}{Metla gor|/|dol}
\stopcombination}
\stop

\page [yes]


\subsubsubject{Vlaženje metel in sesalnega kanala}

Pritisnite\index{pometanje+vlaženje} stikalo~\textSymb{temoin_busebalais} med sedežema:

{\md Položaj 1:} vodna črpalka deluje samodejno, dokler je metla aktivirana.

{\md Položaj 2:} vodna črpalka stalno deluje. (Uporabno \eG\ za nastavitvena dela.)


\subsubsubject{Groba umazanija}

\startSteps [continue]
\item Če obstaja nevarnost, da bi večji kosi smeti (\eG\ PET||plastenke) zamašili sesalno ustje, s pomočjo stranskih tipk na večfunkcijski konzoli  odprite\index{loputa za grobo umazanijo} loputo za grobo umazanijo,~ če pa to ne zadošča~, pa začasno dvignite\index{sesalno ustje+groba umazanija} sesalno ustje.
\stopSteps

\start
\setupcombinations [width=\textwidth]

\placefig[here][fig:suctionMouth:clap]{Ravnanje z grobo umazanijo}
{\startcombination [2*1]
{\externalfigure [work:suction:open]}{Odpiranje lopute za grobo umazanijo}
{\externalfigure [work:suction:raise]}{Začasno dviganje sesalnega ustja}
\stopcombination}
\stop

\stopsection


\startsection [title={Praznjenje posode za umazanijo},
							reference={sec:using:container}]

\startSteps
\item Zapeljite\index{posoda za umazanijo+praznjenje} vozilo na primeren prostor, kjer lahko izpraznite posodo. Pri tem morate upoštevati veljavna določila za varnost okolja.
\item Aktivirajte ročno zavoro in izbirno ročico za stopnjo vožnje postavite na {\em nevtralni} položaj. (Potrebno za sprostitev stikala za nagibanje ||posode.)
\stopSteps

\getbuffer [prescription:container:gravity]

\startSteps [continue]
\item Sprostite in odprite zaporno loputo posode za umazanijo.
\item Pritisnite stikalo~\textSymb{temoin_kipp2} (sredinska konzola med sedežema), da nagnete posodo za umazanijo.
\item Ko je posoda prazna, sperite njeno notranjost z vodnim curkom. Pri tem lahko uporabite integrirano vodno pištolo (dodatna oprema).
\stopSteps

\start
\setupcombinations [width=\textwidth]
\placefig[here][fig:brush:adjust]{Ravnanje s posodo za umazanijo}
{\startcombination [3*1]
{\externalfigure [container:cover:unlock]}{Zaklepanje zaporne lopute }
{\externalfigure [container:safety:unlocked]}{Varnostna prečka}
{\externalfigure [container:safety:locked]}{Varnostna prečka je zapahnjena}
\stopcombination}
\stop

\startSteps [continue]
\item Preverite|/|očistite tesnila in naležne površine tesnil posode, sistema za recikliranje in sesalnega kanala.
\stopSteps

\getbuffer [prescription:container:tilt]

\startSteps [continue]
\item Pritisnite stikalo~\textSymb{temoin_kipp2}, da spustite posodo za umazanijo. (Po potrebi pred tem odstranite varnostne prečke s hidravličnih valjev.)
\item Zaklenite zaporno loputo posode za umazanijo.
\stopSteps

\stopsection


\startsection [title={Ročna sesalna gibka cev},
							reference={sec:using:suction:hose}]

Vozilo \sdeux\ je lahko dodatno opremljeno \index{ročna sesalna gibka cev} z ročno sesalno gibko cevjo. Ta je pritrjena na zaporno loputo posode za umazanijo in je enostavna za uporabo.

{\sla Pogoji:}

Posoda za umazanijo je povsem spuščena; vozilo \sdeux\ je v {\em delovnem} načinu. (Glejte \in{§}[sec:using:start], \atpage[sec:using:start].)

\startfigtext[left][fig:using:suction:hose]{Ročna sesalna gibka cev}
{\externalfigure[work:suction:hose]}
\startSteps
\item Pritisnite tipko~\textSymb{temoin_aspiration_manuelle} na stropni konzoli, da vključite sesalni sistem.
\item Zategnite ročno zavoro, preden zapustite voznikovo kabino.
\item Zaprite sesalni kanal z loputo sesalnega kanala. (Glejte \in{§}[sSec:suctionClap], \atpage[sSec:suctionClap].)
\item Povlecite ročno sesalno gibko cev na ustju iz držala in začnite z delom.
\item Po koncu dela ponovno pritisnite tipko~\textSymb{temoin_aspiration_manuelle}, da izključite sesalni sistem.
\stopSteps
\stopfigtext

\stopsection

\page [yes]

\setups[pagestyle:normal]


\startsection [title={Visokotlačna vodna pištola},
							reference={sec:using:water:spray}]

Vozilo \sdeux\ je lahko dodatno\index{vodna pištola} opremljeno z visokotlačno vodno pištolo. Vodna pištola je pritrjena v zadnjih desnih vzdrževalnih vratih in povezana z 10-metrskim||cevnim navijalnikom~ na nasprotni strani vozila~.

Pri uporabi vodne pištole postopajte kot sledi:

{\sla Pogoji:}

V rezervoarju za svežo vodo je dovolj vode; \sdeux\ je v {\em delovnem} načinu. (Glejte \in{§}[sec:using:start], \atpage[sec:using:start].)

\placefig[margin][fig:using:water:spray]{Visokotlačna vodna pištola}
{\externalfigure[work:water:spray]}

\startSteps
\item Pritisnite tipko~\textSymb{temoin_buse} na stropni konzoli, da vključite visokotlačno vodno pištolo.
\item Zategnite ročno zavoro, preden zapustite voznikovo kabino.
\item Odprite zadnja desna vzdrževalna vrata in ven vzemite vodno pištolo.
\item Odvijte toliko gibke cevi, kot je potrebno, in začnite z delom.
\item Po koncu dela ponovno pritisnite tipko~\textSymb{temoin_buse}, da izključite visokotlačno vodno črpalko.
\item Na kratko povlecite za gibko cev, da sprostite blokade in navijete gibko cev.
\item Vodno pištolo znova pritrdite v njeno držalo in zaprite vzdrževalna vrata.
\stopSteps

\stopsection

\page [yes]


\setups [pagestyle:marginless]


\startsection [title={Delo s tretjo metlo (dodatna možnost)},
							reference={sec:using:frontBrush},
							]

\startSteps
\item Zaženite\index{pometanje} vozilo, kot je opisano v \in{odseku}[sec:using:start] \atpage[sec:using:start].
\item Aktivirajte\index{3.\,metla} {\em delovni} način (gumb na zunanji strani izbirne ročice za stopnjo vožnje).
\stopSteps

% \getbuffer [work:config]

\startSteps [continue]
\item Preverite, ali je tretja metla aktivirana na zaslonu enote Vpad
(\textSymb{vpadFrontBrush} \textSymb{vpadFrontBrushK}, \atpage[vpad:menu]).
\item Pritisnite tipko~\textSymb{joy_key_frontbrush_act}, da aktivirate hidravliko tretje metle.
\item Pritisnite tipko~\textSymb{joy_key_frontbrush_left} ali~\textSymb{joy_key_frontbrush_right}, da tretjo metlo začnete vrteti v želeno smer.

\item S pomočjo tipk~\textSymb{joy_key_frontbrush_increase} in~\textSymb{joy_key_frontbrush_decrease} na večfunkcijski konzoli nastavite hitrost vrtenja.

\item S krmilno ročico postavite metle, kot je prikazano na spodnjih slikah.

\stopSteps

{\md Napotek:} {\lt Da lahko postavite stranske metle, je treba s tipko~\textSymb{joy_key_frontbrush_act} izključiti hidravliko tretje metle.}
\vfill

\start
\setupcombinations [width=\textwidth]

\placefig[here][fig:brush:position]{Postavitev tretje metle}
{\startcombination [2*1]
{\externalfigure [work:frontBrush:move]}{Navzgor|/|navzdol; v levo|/|desno}
{\externalfigure [work:frontBrush:incline]}{Prečno|/|vzdolžno nagibanje}
\stopcombination}
\stop



\stopsection


\stopregister[index][chap:using]

\stopchapter
\stopcomponent
