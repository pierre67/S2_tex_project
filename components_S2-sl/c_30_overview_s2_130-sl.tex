\startcomponent c_30_overview_s2_130-sl
\product prd_ba_s2_130-sl

\chapter{Pregled vozila}

\setups [pagestyle:marginless]


\placefig [here] [] {Pregled leve strani vozila}
{\externalfigure [overview:side:left:sl]}


\page [yes]


\placefig [here] [] {Pregled desne strani vozila}
{\externalfigure [overview:side:right:sl]}

\page [yes]

\setups [pagestyle:normal]


\section{Splošno}

\placefig[margin][p4_vue_01]{\sdeux\ o prevozu}
{%
\startcombination [1*3]
{\externalfigure[overview:vhc:01]}{}
{\externalfigure[overview:vhc:02]}{}
{\externalfigure[overview:vhc:03]}{}
\stopcombination}

Pometalno vozilo \BosFull{sdeux} je rezultat izkušenj in sposobnosti, ki si jih je podjetje Boschung pridobilo z desetletji neprestanega sodelovanja s svojimi zvestimi strankami in partnerji.
Zahteve skupnosti in izvajalcev so se v tem času glede mobilnosti in vsestranskosti močno povečale. Razvijalci naprave \sdeux\ so si zastavili ta izziv, usmerjen k potrebam kupcev in podprt s predlogi za izboljšanje službe za stranke podjetja Boschung.
Iz te sinteze usmerjenosti h kupcem in dosledne realizacije pridobljenih praktičnih izkušenj izhaja \sdeux.


\subsection{Inovativna tehnologija}

Kompaktno pometalno vozilo \BosFull{sdeux} se v svojem razredu odlikuje s svojo izredno majhno težo (2300\,kg), visoko zmogljivostjo (posoda za umazanijo razreda 2,0-m\high{3}), kompaktnimi dimenzijami (širina 1,15\,m) in posebno ergonomičnem delovnem mestu za voznika.

Zaradi ozke izvedbe je \sdeux\ idealen \quotation{povsod} pometalni stroj za ceste in pločnike v mestih in vaseh. Njegov močan dizelski motor v povezavi s kompaktnim hidrostatičnim pogonom (radialni bat||hidromotorji na sprednji kolesi) vedno zagotavlja najvišjo mobilnost, neodvisno od lastnosti mesta uporabe ali stopnje napolnjenosti posode za umazanijo.

Hidravlične črpalke poganja dizelski motor tipa \aW{VW 2.0 CDI} v skladu s standardom Euro-V. Navor motorja je 285\,Nm pri 1750~obratih, doseže pa lahko največjo moč 75\,kW pri 3000~obratih. Zato je mogoče stroj učinkovito uporabljati že pri nizkem številu vrtljajev motorja~– in tako z nižjo obremenitvijo zaradi hrupa~. \sdeux\ ima serijsko vgrajen filter za drobne delce.


\section{Inovacije v storitvi za kupca}

Upogibno krmilje naprave \sdeux\ omogoča majhen obračalni krog in s tem maksimalno gibljivost. Posebni materiali, kot je Domex®, in povsem na CAD-u||zasnovan razvoj vozila omogočata izjemno nosilnost 1200\,kg.

\placefig[margin][overview:cab:frontright]{\sdeux\ Pripravljen za uporabo}
{\externalfigure[overview:cab:twoleft][width=\Bildwidth]}

V celoti zastekljena voznikova kabina ima dva udobna sedeža, opremljena s tritočkovnima varnostnima pasovoma. Naprava \sdeux\ je lahko po želji opremljena s klimatsko napravo.

S svojo najvišjo hitrostjo 40\,km/h se vozilo brez težav vključi v mestni promet. S pomočjo udobnega vzmetenja sprednje in zadnje osi je mogoča varna in udobna vožnja še po tako slabi poti.

Pometalni agregat~– montiran na dveh zgibnih rokah~– je povsem v zornem polju upravljavca, sesalno ustje pa je nameščeno dobro vidno pred sprednjo osjo. Sprednja metla, ki se vrti v obe smeri, je na voljo kot dodatna oprema.

\page [yes]


\subsection{Zvočno izolirana in udobna voznikova kabina}

Voznikova kabina\index{voznikova kabina} vozila \sdeux\ ime volan na desni in je zasnovana za dve osebi. Je zvočno izolirana in montirana na amortiziranih tihih blokih.

Vrata in dno sta zastekljena, kar omogoča široko zorno polje. Vetrobransko steklo se razteza po celotni čelni strani vozila in tako omogoča neoviran pogled na delo metel.

Voznikov sedež ima mehansko ali~– opcijsko~– pnevmatsko vzmetenje. Voznikov in sovoznikov sedež sta nameščena na nastavljivih drsnih vodilih.


\subsubsubject{Ergonomija}

\startfigtext[right][overview:joy:sideview]{Upravljalna konzola}
{\externalfigure[overview:joy:top]}
Večfunkcijska konzola levo od voznikovega sedeža omogoča enoročno upravljanje vseh osnovnih funkcij. Z dvema krmilnima ročicama je mogoče s palcem in kazalcem krmiliti vsako metlo posebej. Stikala za metle in sprednjo metlo (dodatna možnost), število vrtljajev motorja, tempomat itd. so prav tako na večfunkcijski konzoli.
\stopfigtext

Na spodnjem robu zornega polja voznika stroja je zaslon na dotik, ki prikazuje vse pomembne informacije o funkcijah stroja v resničnem času, ne da bi oviral pogled navzven.

\placefig[margin][overview:vhc:left]{\sdeux\ Pred zgodovinskimi zidovi}
% \placefig[margin][overview:vhc:left]{\sdeux\ sur site historique}
{\externalfigure[overview:vhc:left]}

\page [yes]


\subsubsubject{Voznikova kabina}

\index{voznikova kabina} Izbirna ročica za stopnjo vožnje (\quotation{menjalnik}) je desno od krmilnega droga, na voljo pa sta dve prestavi za vožnjo naprej in ena za vzvratno vožnjo. Na zunanji strani izbirne ročice za stopnjo vožnje je gumb za preklapljanje med načinoma dela \aW{delo} in \aW{vožnja}. Za preklop načina dela je treba \sdeux\  zaustaviti. (Več o tem v poglavju \about[sec:using:work], \atpage[sec:using:work].)

\placefig[margin][fig:overview:steeringwheel]{Voznikova kabina}
{\externalfigure[overview:driver:place]}

Pri vzvratni vožnji se vključi monitor vzvratne kamere in oglaša se opozorilni signal (izključite ga lahko prek Vpada).

Večfunkcijska ročica na levi strani krmilnega droga vsebuje stikalo za brisalec (dve stopnji in interval) ter stikalo za svetlobno in zvočno hupo.

V poglavju \about[chap:using] od strani \atpage[chap:using] naprej najdete podrobnosti o teh in drugih funkcijah vozila \sdeux.

\page [yes]

\setups[pagestyle:marginless]


\subsection[overview:brushsystem]{Pometalna in sesalna priprava}

\subsubsubject{Metla}

\startfigtext[left][fig:overview:steeringwheel]{Pometalna|/|sesalna priprava}
{\externalfigure[system:brush]}
Metle\index{pometanje} imajo poravnalne glave, ki so nameščene na zgibnih rokah. Prah, ki se dviga pri pometanju, se veže ob pršenju z vodo: vsaka metla je opremljena z dvema šobama, ki črpata vodo iz posode za svežo vodo ali recikliranje vode.

Stikalo\index{sesanje} na večfunkcijski konzoli istočasno aktivira metle in vodno črpalko.\footnote{Za vodno črpalko glejte poglavje \in[chap:using] \about[chap:using], predvsem pa \about[sec:using:work], \atpage[sec:using:work].}
Položaje metel ter njihov prečni in vzdolžni nagib je mogoče upravljati neposredno prek krmilne ročice na večfunkcijski konzoli.
\stopfigtext

Metle so zaščitene z mehanskim in hidravličnim sistemom proti trčenju.


\subsubsubject{Sesalno ustje}

Na delovnem položaju (spuščeno) sesalno ustje počiva na 4~kolesih in povsem prekriva površino med razmaknjenima metlama. V svojem \quotation{vlečenem} položaju je pri trčenjih z ovirami zaščiteno pred mehanskimi poškodbami. Pri vzvratni vožnji se sesalno ustje samodejno dvigne.

Debel, zamenljiv gumijasti rob poskrbi za zatesnitev s talno površino. Elektro||hidravlično krmilna loputa na sprednji strani sesalnega ustja omogoča sesanje večje umazanije.


\subsubsubject{Posoda za umazanijo}

Aluminijasto posodo za umazanijo je mogoče dvigniti do 50° in na višino 1,5\,m (praznilna višina). V njej je, napeljan od spodaj, nameščen sesalni kanal s premerom odprtine 180\,mm.

Sesalni podtlak se proizvaja z visoko zmogljivo turbino, ki je nameščena vodoravno v posodi za umazanijo. Posoda ima vzdrževalno loputo za čiščenje in vizualni pregled.

V zaporni loputi posode za umazanijo sta dve sesalni rešetki iz legiranega jekla. Ti je mogoče pri čiščenju odpreti brez orodja. Zaporno loputo je mogoče odpahniti in odpreti z roko.

Z loputo, ki jo je mogoče ročno preklopiti, je mogoče enostavno preklopiti zračni tok med sesalnim kanalom in ročno sesalno gibko cevjo (dodatna oprema).


\subsection{Vlažilna naprava}

\subsubsubject{Sistem za svežo vodo}

Rezervoar vozila\index{pometanje+vlaženje} iz PE||litine je nameščen v pokončnem položaju za voznikovo kabino. Njegova prostornina\index{rezervoar za svežo vodo+} je 190\,l.

Električna črpalka (6,5\,l/min) črpa vodo do razpršilnih šob nad vsako metlo (vključno z dodatno tretjo metlo).


\subsubsubject{Recikliranje onesnažene vode}

Onesnažena voda teče skozi mikroperforacije notranje stene rezervoarja za onesnaženo vodo in nato kozi reciklirno loputo v spodaj nameščen rezervoar za reciklirano vodo. Prostornina \index{rezervoar za reciklirano vodo+} rezervoarja za reciklirano vodo je 140\,l.

Hidravlična potopna črpalka črpa vodo do razpršilnih šob v notranjosti sesalnega ustja in sesalnega kanala.


\testpage [8]
\subsubsubject{Rezervoar za reciklirano vodo}

Rezervoar za reciklirano vodo ima toplotni izmenjevalnik za vodo in hidravlično tekočino z dvojno funkcijo:

\startitemize[width=35mm,style=\md, command={\setupwhitespace[small]}]
\sym{Funkcija poleti} Voda prevaja toploto hidravlične tekočine prek konvekcije do aluminijastih sten rezervoarja, od koder se oddaja v okoljski zrak.

\sym{Funkcija pozimi} Hidravlična tekočina segreva vodo v rezervoarju. To omogoča
pršenje sesalnega kanala kot tudi sesalnega ustja tudi pri temperaturah malce pod zmrziščem.
\stopitemize


\subsubsubject{Nadzor stanja polnosti vode}

\startitemize[width=35mm,style=\md, command={\setupwhitespace[small]}]
\sym{Sveža voda} Pri nezadostnem stanju polnosti se na zaslonu Vpada prikaže simbol~\textSymb{vpad_water}.
\sym{Reciklirana voda} Če je nivo polnosti v reciklirnem rezervoarju pod toplotnim izmenjevalnikom (glejte zgoraj) se na zaslonu Vpada prikaže simbol~\textSymb{vpad_rwater_orange} (rumen). Pri nezadostnem stanju polnosti se prikaže simbol~\textSymb{vpad_rwater} (rdeč).
\stopitemize


\subsubsubject{Široke pnevmatike (dodatna možnost)}

Pritisk ob tla\index{široke pnevmatike} ustreza tlaku v pnevmatikah. S tlakom v pnevmatikah 1,8\,bar je mogoče doseči pritisk ob tla 18\,N/cm². Kljub temu se nosilnost pnevmatik za zagotovljeno osno obremenitev več ne doseže. Z 1,8\,bari je mogoče pri 40\,km/h zagotoviti le še osno obremenitev 1495\,kg. Če za tlak v pnevmatikah izberete drugačno vrednost razen 3,0\,bare, odgovornost prevzemate sami.

\subsubsubject{Prikaz preobremenitve (dodatna možnost)}

Pri preobremenitvi vozila\index{prikaz preobremenitve} se na Vpadu prikaže sporočilo. Preobremenitev se določa s tipalom nagiba zadnje osi. Serijsko je prikaz preobremenitve nastavljen na 3500\,kg, vendar je treba preprečiti tolerančno polje te vrednosti. Nastavitev 3500\,kg vam lahko spremeni strokovnjak.

\page [yes]
\setups[pagestyle:normal]


\section{Identifikacija vozila}

\subsection{Tipska ploščica vozila}

Tipska ploščica vozila\index{identifikacija+vozilo} je v
voznikovi kabini nasproti konzole pod sovoznikovim sedežem (glejte \inF[fig:identity:location], \atpage[fig:identity:location]).


\subsection{Koda in številka motorja}

Koda motorja je na tipski ploščici motorja (nalepka), na nagrbančenem kovinskem vodu hladilnega krogotoka, spredaj na motorju (dvignite posodo za umazanijo).

Številka motorja je vgravirana na motorju (\inF[identity:engine:number]). Sestavljena je iz devetih alfanumeričnih znakov: prvi trije so koda motorja, naslednjih šest pa serijska številka motorja.


\placefig[margin][idvhc]{Tipska ploščica vozila}
{\externalfigure[s2:id:plaque]}

\placefig[margin][identity:engine:code]{Tipska ploščica motorja}
{\externalfigure[engine:id:code]}

\placefig[margin][identity:engine:number]{Številka motorja}
{\externalfigure[engine:id:number]}

\page [yes]


\subsection [sec:plateWheel]{Tipska ploščica koles}

Tipska ploščica platišč in pnevmatik\index{pnevmatike+polnilni tlak} je v voznikovi kabini\index{platišča+dimenzije} pod sovoznikovim sedežem.


\subsection{Številka podvozja}

Številka podvozja\index{identifikacija+številka podvozja} (številka šasije) je pritrjena na desni strani vozila pod voznikovo kabino na podvozju.


\subsection{\symbol[europe][CEsign]Skladnost in označba}

~\symbol[europe][CEsign]-znak za skladnost vozila je v voznikovi kabini nasproti konzole pod sovoznikovim sedežem.

Vozilo \sdeux\ je v skladu z osnovnimi varnostnimi in zdravstvenimi zahtevami Direktive o strojih\index{certifikat+CE-skladnost}\index{Direktiva o strojih} 2006/42/ES\footnote{Direktiva 2006/42/ES Evropskega parlamenta in Sveta z dne 17.~maja 2006}.
% \textrule

\placefig[margin][idpneus]{Polnilni tlak pnevmatik}
{\externalfigure[identity:tires]}

\placefig[margin][fig:identity:location]{Tipske ploščice}
{\externalfigure[identity:location]}

\page [yes]
\setups [pagestyle:marginless]


\startsection[title={Tehnični podatki},
							reference={donnees_techniques}]

\subsection [sec:measurement] {Mere vozila}

\placefig[here][fig:measurement]{\select{caption}{Širina~– metle v mirovanju ali iztegnjene~–, dolžina in višina vozila}{mere vozila}}
{\Framed{\externalfigure[s2:measurement]}}

\page [yes]

\placefig[here][fig:measurement]{\select{caption}{Višina vozila z dvignjeno posodo za umazanijo}{višina vozila}}
{\Framed{\externalfigure[s2:measurement:02]}}

\page [yes]

\starttabulate [|lBw(45mm)|p|l|rw(35mm)|]
\FL
\NC Skupina\index{mere} \NC \bf Mera \NC \bf Enota \NC \bf Vrednost \NC\NR
\ML
\NC Mere vozila \NC Dolžina (celotna) \NC \unite{mm} \NC 4588,00 \NC\NR
\NC\NC Dolžina s 3.\,metlo \NC \unite{mm} \NC 5020,00 \NC\NR
\NC\NC Širina vozila \NC \unite{mm} \NC 1150,00 \NC\NR
\NC\NC Širina vozila (celotna) \NC \unite{mm} \NC 1575,00 \NC\NR
\NC\NC Višina brez rotacijske luči \NC \unite{mm} \NC 1990,00 \NC\NR
\NC\NC Medosje \NC \unite{mm} \NC 1740,00 \NC\NR
\NC\NC Osna razdalja \NC \unite{mm} \NC 894,00 \NC\NR
\ML
\NC Pometalna širina \NC Standardna metla \NC \unite{mm} \NC 2300,00 \NC\NR
\NC\NC S 3.\,metlo \NC \unite{mm} \NC 2600,00 \NC\NR
\NC\NC Premer metle \NC \unite{mm} \NC 800,00 \NC\NR
\NC\NC Širina sesalnega ustja \NC \unite{mm} \NC 800,00 \NC\NR
\ML
\NC Razporeditev bremena \NC Prazna teža\note[weight:empty] sprednja os \NC \unite{kg} \NC pribl. 1100,00 \NC\NR
\NC\NC Prazna teža\note[weight:empty] zadnja os \NC \unite{kg} \NC pribl. 1200,00 \NC\NR
\NC\NC Prazna teža\note[weight:empty] \NC \unite{kg} \NC pribl. 2300,00 \NC\NR
\NC\NC Dovoljena skupna teža \NC \unite{kg} \NC 3500,00 \NC\NR
\LL
\stoptabulate


\subsection{Polmer osne razdalje in pometanja}

\starttabulate [|lBw(45mm)|p|l|rw(35mm)|]
\FL
\NC Dimenzija\index{Dimenzije} \NC \bf Mera \NC \bf Enota \NC \bf Vrednost \NC\NR
\ML
\NC Polmer osne razdalje\index{polmer osne razdalje}\index{mera+polmer osne razdalje} \NC Najmanjši polmer obračanja z metlo \NC \unite{mm}	\NC 3325,00 \NC\NR
\ML
\NC Polmer pometanja \NC zunanji \NC \unite{mm} \NC 3425,00 do 3850,00 \NC\NR
\NC\NC notranji \NC \unite{mm} \NC 2025,00 do 1675,00 \NC\NR
\LL
\stoptabulate

%% TODO en/de/fr: Footnote on preceeding page
\footnotetext[weight:empty]{Standardna konfiguracija, z voznikom (pribl. 75\,kg).}

\placefig[here][pict:steerin_sweeping:radius]{Polmer medosne razdalje|/|krog obračanja in polmer pometanja}
{\externalfigure[steerin_sweeping:radius]}

\page [yes]


\subsection{Kolesa in pnevmatike}

{\sla Standardne dimenzije}

\starttabulate[|lBw(45mm)|p|rw(55mm)|]
\FL
\NC Komponenta \NC \bf Oprema \NC \bf Vrednost \NC\NR
\ML
\NC Pnevmatike \NC Standardne dimenzije \NC 205/70 R 15 C \NC\NR
\ML
\NC Platišča \NC Standardne dimenzije \NC 6J\;×\;15 H2 ET 60 \NC\NR
\ML
\NC Polnilni tlak pnevmatik\index{polnilni tlak pnevmatik} \NC Standardni, spredaj|/|zadaj \NC 4,5|/|5,8\,bar \NC\NR
\LL
\stoptabulate

{\sla Široke pnevmatike}

\starttabulate[|lBw(45mm)|p|rw(55mm)|]
\FL
\NC Komponenta \NC \bf Oprema \NC \bf Vrednost \NC\NR
\ML
\NC Pnevmatike\index{široke pnevmatike} \NC Široke pnevmatike \NC 275/60 R 15 107H \NC\NR
\ML
\NC Platišča \NC Široke pnevmatike \NC 8LB\;×\;15 ET 30 \NC\NR
\ML
\NC Polnilni tlak pnevmatik\index{polnilni tlak pnevmatik} \NC Standardni, spredaj|/|zadaj \NC 3,0|/|3,0\,bar \NC\NR
\LL
\stoptabulate


\subsection{Dizelski motor}

\starttabulate [|lBw(45mm)|l|rp|]
\FL
\NC \bf Skupina\index{dizelski motor+identifikacija} \NC \bf Parameter \NC \bf Vrednost\NC\NR
\ML
\NC Tip motorja \NC \NC VW CJDA TDI 2.0 – 475 NE \NC\NR
\NC Splošno \NC 	Število taktov \NC štiritaktni \NC\NR
\NC\NC Število valjev \unite{n} \NC 4 \NC\NR
\NC\NC Izvrtina x hod \unite{mm} \NC 81\;×\;95,5 \NC\NR
\NC\NC Skupna delovna prostornina \unite{cm\high{3}} \NC 1968 \NC\NR
\NC\NC Ventili na valj \NC 4 \NC\NR
\NC\NC Zaporedje krmiljenja ventilov \NC 1-3-4-2 \NC\NR
\NC\NC Najnižje število vrtljajev v prostem teku \unite{min\high{−1}} \NC 830 +50/−25 \NC\NR
\NC Moč|/|navor \NC Najv. št. vrtljajev \unite{min\high{−1}} \NC 3400 \NC\NR
\NC\NC Najv. moč \unite{kW} pri \unite{min\high{−1}} \NC 75 pri 3000 \NC\NR
\NC\NC Najv. navor \unite{Nm} pri \unite{min\high{−1}} \NC 285 pri 1750 \NC\NR
\NC Specifična poraba\index{dizelski motor+poraba} \NC Gorivo \unite{g/kWh} \NC 224 (pri najv. moči) \NC\NR
\NC\NC Olje \unite{g/kWh} \NC 0,22 \NC\NR
\NC Sistem za gorivo \NC Vbrizgalni sistem \NC Sistem s skupnim vodom \NC\NR
\NC\NC Oskrba z gorivom \NC Zobniška črpalka \NC\NR
\NC\NC Polnjenje \NC Da \NC\NR
\NC\NC Hlajenje s polnilnim zrakom \NC Da \NC\NR
\NC\NC Polnilni tlak \unite{mbar} \NC 1300\NC\NR
\NC Mazalni krog\index{dizelski motor+mazanje} \NC Tip \NC Z izmenjevalnikom olja|/|vode \NC\NR
\NC\NC Napajanje \NC Črpalka rotorja \NC\NR
\NC\NC Poraba olja \unite{liter/20\,h} \NC <\:0,1 \NC\NR
\NC Hladilni krog\index{dizelski motor+hlajenje} \NC Skupna kapaciteta \unite{l} \NC pribl. 12 \NC\NR
\NC\NC Umerjalni tlak raztezalne posode \unite{bar} \NC 1,4 \NC\NR
\NC\NC Termostat (odprtina) \unite{°C} \NC 87 \NC\NR
\NC\NC Termostat (poln) \unite{°C} \NC 102 \NC\NR
\NC Izpušni plini \NC Filter za drobne delce \NC Da \NC\NR
\NC\NC Priprava izpušnih plinov \NC Da \NC\NR
\NC\NC Standard \NC Euro 5 \NC\NR
\LL
\stoptabulate


\subsection{Vozne zmogljivosti}

\starttabulate[|lBw(45mm)|p|l|rw(35mm)|]
\FL
\NC Vozna zmogljivost\index{vozne zmogljivosti} \NC \bf Konfiguracija \NC \bf Enota \NC \bf Vrednost \NC\NR
\ML
\NC Hitrost \NC \aW{Delovni }način \NC \unite{km/h} \NC 0 do 18 (brezstopenjsko) \NC\NR
\NC\NC \aW{Način }vožnje \NC \unite{km/h} \NC 0 do 40 \NC\NR
\ML
\NC Omejitev hitrosti \NC nastavljiva \NC \unite{km/h} \NC 0 do 25 \NC\NR
\LL
\stoptabulate


\subsection{Električni sistem}

{\starttabulate [|lw(65mm)|p|rw(30mm)|]
\FL
\NC \bf Skupina \NC \bf Komponenta \NC \bf Vrednost \NC\NR
\ML
\NC Baterija \NC Svinčeni akumulator \NC 12\,V 75\,Ah \NC\NR
\NC Električno napajanje \NC Generator \NC 14,8\,V 140\,A \NC\NR
\NC Zaganjalnik \NC Moč \NC 1,8\,kW \NC\NR
\NC Avdio oprema \NC Priključek za radio\index{priključek za radio} in zvočniki\index{zvočniki} \NC Serijska oprema \NC\NR
% \NC Sécurité et surveillance \NC Tachygraphe\index{tachygraphe} \NC En option \NC\NR
% \NC\NC Enregistreur de fin de parcours\index{fin de parcours} \NC En option \NC\NR
\NC Osvetlitev/signalna oprema spredaj \NC Pozicijska luč \NC 12\,V 5\,W \NC\NR
\NC\NC Zasenčena luč \NC H7, 12\,V 55\,W \NC\NR
\NC\NC Delovni žaromet \NC G886, 12\,V 55\,W \NC\NR
\NC\NC Smerniki \NC 12\,V 21\,W \NC\NR
\NC Osvetlitev/signalna oprema zadaj \NC Kombinirane zavorne luči \NC 12\,V 5/21\,W \NC\NR
\NC\NC Smerniki \NC 12\,V 21\,W \NC\NR
\NC\NC Vzvratne luči \NC 12\,V 21\,W \NC\NR
\NC\NC Osvetlitev registrske tablice \NC 12\,V 5\,W \NC\NR
\NC Dodatna osvetljava \NC Rotacijska luč \NC H1, 12\,V 55\,W \NC\NR
\LL
\stoptabulate
}
\stopsection

\stopcomponent
