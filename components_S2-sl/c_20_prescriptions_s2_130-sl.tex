\startcomponent c_20_prescriptions_s2_130-sl
\product prd_ba_s2_130-sl


\chapter [safety:risques] {Varnostni predpisi}

\setups [pagestyle:marginless]


\section{Osnovni napotki}

\subsubject{Zakonska podlaga}

Nesreče lahko imajo hude posledice tako za delodajalca kot tudi za zaposlene. Zato bi ponovno radi navedli obveznosti obeh:\note[prescription:user:right].

Delodajalec mora, preden zaposlenega seznani z upravljanje pometalnega stroja, upoštevati naslednje točke:

\startSteps
\item Vsak voznik mora biti usposobljen za upravljanje vozila. Imeti mora dokazilo o usposabljanju.
\item Vsak voznik vozila mora imeti uradno vozniško dovoljenje. Tega je dovoljeno izdati samo, če so izpolnjeni naslednji trije pogoji:
\startitemize [2]
\item Zaposleni je opravil zdravniški pregled oči pri obratnem zdravniku.
\item Zaposleni pozna danosti dela in je seznanjen z vsemi varnostnimi predpisi na mestu uporabe vozila, ki mu jih preda nadrejeni.
\item Zaposleni je opravil preizkus primernosti, ki potrjuje, da ima potrebna znanja za upravljanje vozila.
\stopitemize
\stopSteps

Če je največja hitrost vozila več kot 25\,km/h\note[prescription:user:right], mora imeti vozilo uradno dovoljenje, voznik vozila pa mora imeti naslednje vozniško dovoljenje:
\startitemize
\item vozniško dovoljenje kategorije B\note[prescription:lisence] za vozila z največjo dovoljeno skupno težo 3,5~ton oz.
\item vozniško dovoljenje kategorije C\note[prescription:lisence] za vozila z dovoljeno skupno težo nad 3,5~tone.
\stopitemize

Če je največja hitrost vozila 25\,km/h, mora voznik vozila poznati najmanj cestnoprometne predpise, ki veljajo na javnih cestah in mestih, četudi za vožnjo vozila ne potrebuje vozniškega dovoljenja kategorije B\note[prescription:user:right].

\footnotetext [prescription:user:right] {Obveznosti delojemalca in zaposlenega se lahko razlikujejo glede na državo ali področje. Seznanite se z veljavnimi predpisi v svoji državi oz. na svojem področju.}

\footnotetext[prescription:lisence] {Direktiva 2006/126/ES Evropskega parlamenta in Sveta z dne 20.~decembra 2006 o vozniškem izpitu.}


\subsubject{Pogoji uporabe}

Napravo \sdeux\ je dovoljeno uporabljati samo, če je v brezhibnem stanju. Poleg tega mora upravljavec upoštevati varnostne napotke in predpise, vsebovane v pričujočih navodilih za uporabo. Motnje v delovanju, ki vplivajo na varnost, vam mora nemudoma odpraviti|/|popraviti strokovno podjetje.
\blank [big]

\startSymList
\externalfigure [s2_inspection] [width=4.5em]
\SymList
{\md Dnevno vzdrževanje:}
Po vsaki uporabi je treba vozilo pregledati in popraviti vidne poškodbe in okvare. V primeru škode ali motenj v delovanju vozila nemudoma obvestite delavnico. Če to ni mogoče, je treba vozilo takoj zaustaviti in zavarovati na mestu okvare.
\stopSymList


\subsubject{Pravilna uporaba}

\sdeux\ je predviden za čiščenje in vzdrževanje cest, poti in trgov. Vsaka drugačna uporaba velja kot nepravilna. Posledično podjetje \boschung\ ne prevzema nobene odgovornosti za škode, nastale zaradi neupoštevanja tega navodila. Pri nepravilni uporabi vso odgovornost za škode prevzema upravljavec. {\em K pravilni uporabi sodi tudi upoštevanje varnostnih napotkov in vzdrževalnega načrta kot tudi upoštevanje pričujočih navodil za uporabo.}


\section{Vožnja po javnih cestah}

\subsubject{Splošni predpisi}

Poleg obratovalnih navodil je treba upoštevati tudi splošno veljavna pravila, veljavne zakonske in druge predpise in določila za preprečevanje nesreč in varstvo okolja.


\subsubject{Sovoznikovo mesto}

Sovoznica~/sovoznik se lahko prevaža na za to predvidenem
sedežu, t .i.{\em sovoznikovem sedežu}.


\subsubject{Varnostni pas}

\startSymList
% \externalfigure [prescription:safety:belt]
\PMbelt
\SymList
Voznik in sovoznik \sdeux\ se morata v skladu s cestnoprometnimi predpisi pripeti
z varnostnim pasom, ko se usedeta v vozilo.
\stopSymList


\subsubject{Videti in biti viden}

\startSymList
\externalfigure [travaux_deviation] [width=3.5em]
\SymList
Poskrbite, da boste dobro vidni, predvsem na zelo prometnih cestah.

Če voznik vozila pri določenem manevru ali med določenim delom nima zadostne vidljivost, mora za pomoč prosite dodatno osebo, s katero bo v stalnem vidnem stiku.
\stopSymList


\subsubject{Osvetlitev in signalna sredstva}

Glede na veljavne cestnoprometne predpise je treba po potrebi tudi podnevi
imeti vključene žaromete in|/|ali rotacijsko luč vozila.


\subsubject{Uporaba mobilnih telefonov}

\startSymList
\PPphone
\SymList
Uporaba mobilnega telefona ali radijske naprave med vožnjo po javnih cestah je prepovedana, razen če je vozilo opremljeno z napravo za prostoročno telefoniranje.

Telefoniranje\index{varnost+mobilni telefon} za volanom~– tudi z napravo za prostoročno telefoniranje~– v vsakem primeru moti koncentracijo v cestnem prometu.
\stopSymList


\section{Vzdrževalni predpisi}

\subsubject{Vzdrževalni napotki}

Pred začetkom del mora vzdrževalno osebje prebrati navodila za uporabo za \sdeux, predvsem odseka o varnosti in vzdrževanju.


\subsubject{Potrebne kvalifikacije}

\startSymList
\externalfigure [mecanicienne] [width=3.5em]
\SymList
Samo osebe, ki so si z ustreznim šolanjem pridobile potrebno znanje, lahko izvajajo vzdrževalna dela na napravi \sdeux\ . To velja predvsem za dela na motorju, zavornem sistemu, krmilju ter električnem in hidravličnem sistemu.
\stopSymList


\testpage [6]
\subsubject{Nadzor}

\startSymList
\externalfigure [mecanicien_hyerarchie] [width=3.5em]
\SymList
Osebe, ki se izobražujejo~– praksa ali pripravništvo~–, lahko na vozilu delajo samo pod nadzorom strokovne osebe. Z naključnim preverjanjem preverite, ali osebje upošteva navodila za uporabo in varnostne predpise.
\stopSymList


\subsubject{Varjenje}

\startSymList
\externalfigure [pince_soudure2] [width=3.5em]
\SymList
Pred začetkom varjenja na karoseriji ali šasiji je
treba obvezno odklopiti baterijo in vse elektronske krmilne naprave.
\stopSymList

\subsubject{Čiščenje vozila}

\startSymList
\externalfigure [washer_pressure] [width=3.5em]
\SymList
Pred čiščenjem naprave \sdeux\ preberite poglavje \about[sec:cleaning] na strani \atpage[sec:cleaning], predvsem odsek o predpisih za čiščenje.
\stopSymList


\subsubject{Dostopnost dokumentacije vozila}

\startSymList
\externalfigure [lecteur_1] [width=3.5em]%\PMrtfm
\SymList
Med uporabo naj bo dokumentacija vozila vedno enostavno dostopna v voznikovi kabini.
\stopSymList


\section{Posebna določila za uporabo}

\subsubject{Višina vozila}

\startSymList
\PPmaxheight
\SymList
Pri delu|/|vožnji po nejavnih površinah (parkirne garaže, podvozi, električne napeljave itd.) se vedno prepričajte, ali je višina zadostna za \sdeux\  (glejte \in{poglavje}[sec:measurement], \atpage[sec:measurement]).
\stopSymList


\subsubject{Stabilnost vozila}

Preprečite morebitne manevre, ki bi lahko vplivali na stabilnost vozila. Pri povišani hitrosti v zavojih se lahko \sdeux\ zaradi svoje ozke konstrukcije in povišanega težišča pri polni posodi za umazanijo prevrne.


\subsubject{Neželeni premiki vozila}

Če zapustite vozilo, ga zavarujte pred nepooblaščenimi osebami. Preden zapustite vozilo, aktivirajte ročno zavoro in kolesa podstavite s kladami.

\startbuffer [prescription:handbrake]
\starttextbackground [CB]
\startPictPar
\PPstop
\PictPar
{\md Čvrsto zategnite ročno zavoro!} Drugače se lahko vozilo nenadzorovano začne premikati tudi\index{ročna zavora+možnost nevarnosti} na komaj zaznavnih klancih ter povzroči nesrečo z nevarnostjo smrtnih poškodb drugih oseb.

{\lt Zaradi hidrostatičnega pogonskega sistema se v mirovanju postopoma zniža tlak v hidravličnem krogu, kar povzroči zmanjšanje zadrževalne sile motorja. Zaradi tega je zelo pomembno, da zategnete ročno zavoro vedno, ko zapustite vozilo.}
\stopPictPar
\stoptextbackground

\stopbuffer

\getbuffer [prescription:handbrake]


\testpage [6]
\subsubject{Posoda za umazanijo}

\startbuffer [prescription:container:gravity]
\starttextbackground [CB]
\startPictPar
\PHgravite
\PictPar
{\md Nevarnost nesreče:}
{\lt Pri dviganju posode za umazanijo se težišče premakne navzgor. Pri tem se poveča nevarnost prevračanja vozila. Zato pri nagibanju posode za umazanijo pazite, da je vozilo vodoravno in na čvrsti podlagi.}
\stopPictPar
\stoptextbackground

\stopbuffer

\getbuffer [prescription:container:gravity]


\startbuffer [prescription:container:tilt]
\starttextbackground [CB]
\startPictPar
\PHcrushing
\PictPar
{\md Nevarnost nesreče:}
{\lt Nikoli ne izvajajte del pod posodo za umazanijo, dokler na hidravlične dvižne valje posode za umazanijo ne namestite varnostne prečke.}
\stopPictPar
\stoptextbackground

\stopbuffer

\getbuffer [prescription:container:tilt]


\stopcomponent
