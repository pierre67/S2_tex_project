\startcomponent c_45_vpad_s2_130-sl
\product prd_ba_s2_130-sl

\startchapter[title={Računalnik v kabini (Vpad)},
							reference={sec:vpad}]

\setups[pagestyle:marginless]


\startsection[title={Opis Vpada},
							reference={vpad:description}]

\startfigtext [left] {Vpad SN v voznikovi kabini}
{\externalfigure[vpad:inside:view]}
\textDescrHead{Inovativen, inteligenten … } \Vpad\ je bil zasnovan za krmiljenje
agregatov na komunalnem področju, katerih tehnologija postaja vedno
bolj zahtevna in ki nudijo najrazličnejše
funkcije.
Z enoto \Vpad\ je upravljavcu na voljo sistem, ki ni omejen
samo na prikaz informacij~– vizualno ali zvočno~– o vseh
delovnih postopkih in postopkih stroja.
Kaj predvsem odlikuje enoto \Vpad\ in kje postavlja nova merila so
intuitivno vodenje uporabnika, upravljalna ergonomija in logika ukazov.

Zahvaljujoč svojim raznolikim funkcijam je mogoče \Vpad\ uporabljati izredno prilagodljivo,
zaradi česar je veliko več kot zgolj elektronska krmilna enota.
\stopfigtext

\textDescrHead{… univerzalno} Združljivost in prilagodljivost sta bila pri
razvoju enote \Vpad\ v ospredju:
kot modularno krmilno enoto jo je mogoče prilagoditi individualnim lokalnim danostim in
različicam opreme, zaradi njenih številnih elektronskih
vmesnikov in poti za prenos podatkov~– vse do WLAN-a~–, pa so odprte
vse možnosti.
Enota \Vpad\ deluje z najsodobnejšo elektroniko z 32-bitno tehnologijo in
operacijskim sistemom v resničnem času.
\vfill


\startfigtext[left]{Večfunkcijska konzola}
{\externalfigure[console:topview]}
\textDescrHead{… in modularno} Zradi modularnosti ima enota \Vpad\
izjemno prednost:
tako je mogoče različico~, ki se uporablja serijsko v \sdeux, postopoma
kadar koli razširiti z nadaljnjimi komponentami, kot sta npr. modem ali konzola (glejte
sliko).
Modularnost ni omejena samo na strojno opremo, saj je mogoče sistem tudi
programsko zelo razširiti ter prilagoditi spreminjajočim se potrebam.

Večfunkcijska konzola enote \sdeux\ je visoko razviti vmesnik med
upravljavcem in strojem. Celotni pometalni|/|sesalni sistem je mogoče krmiliti prek te konzole.
\stopfigtext

\page [yes]


\subsection[vpad:home]{Glavni zaslon}

%% Note: outcommented by PB
% \placefig[left][fig:vpad:engineData]{Accueil mode transport}
% {\scale[sx=1.5,sy=1.5]
% {\setups[VpadFramedFigureHome]
% \VpadScreenConfig{
% \VpadFoot{\VpadPictures{vpadClear}{vpadBeacon}{vpadEngine}{vpadSignal}}}
% \framed{\null}}
% }


\start

\setupcombinations[width=\textwidth]

\placefig [here][fig:vpad:engineData]{Glavni zaslon}
{\startcombination [2*1]
{\setups[VpadFramedFigureHome]% \VpadFramedFigureK pour bande noire
\VpadScreenConfig{
\VpadFoot{\VpadPictures{vpadClear}{vpadBeacon}{vpadEngine}{vpadSignal}}}%
\scale[sx=1.5,sy=1.5]{\framed{\null}}}{\aW{Način }za vožnjo}
{\setups[VpadFramedFigureWork]% \VpadFramedFigureK pour bande noire
\VpadScreenConfig{
\VpadFoot{\VpadPictures{vpadClear}{vpadBeacon}{vpadEngine}{vpadSignal}}}%
\scale[sx=1.5,sy=1.5]{\framed{\null}}}{\aW{Delovni }način}
\stopcombination}

\stop

\blank [1*big]

Glavni zaslon enote \Vpad\ zajema vse
potrebne elemente za nadzor vseh funkcij vozila \sdeux.

Na zgornjem območju so kontrolni prikazi.

Sredinsko območje v resničnem času med\, drugim prikazuje naslednje podatke:
hitrost, število vrtljajev in temperatura motorja, nivo polnosti goriva, nivo polnosti vode za recikliranje itd.

Način \aW{Vožnja} je prikazan z zajčkom~\textSymb{transport_mode}, način \aW{Delo} pa z želvo~\textSymb{working_mode}.

Menijska vrstica v spodnjem robu prikazuje razpoložljive menije: S pritiskom na sredino zaslona na dotik se prikažejo dodatni meniji.

\page [yes]


\defineparagraphs[SymVpad][n=2,distance=4mm,rule=off,before={\page[preference]},
							after={\nobreak\hrule\blank [2*medium]\page[preference]}]
\setupparagraphs [SymVpad][1][width=4em,inner=\hfill]


\subsection{Kontrolni prikaz na zaslonu Vpad} % nouveau


\start % local group for temporary redefinition of \textDescrHead [TF]
\define[1]\textDescrHead{{\bf#1\fourperemspace}}

\startSymVpad
\externalfigure[vpadTEnginOilPressure][height=1.7\lH]
\SymVpad
\textDescrHead{Tlak motornega olja}(rdeča) Prenizek tlak motornega olja. Nemudoma zaustavite
motor.

+\:Sporočilo o napaki \# 604
\stopSymVpad

\startSymVpad
\externalfigure[vpadWarningBattery][height=1.7\lH]
\SymVpad
\textDescrHead{Napolnjenost baterije}(rdeča) Prenizek polnilni tok baterije. Obvestite delavnico.
\stopSymVpad


\startSymVpad
\externalfigure[vpadWarningEngine1][height=1.7\lH]
\SymVpad
\textDescrHead{Diagnoza motorja}(rumena) Napaka v krmiljenju motorja. Obvestite delavnico.
\stopSymVpad


\startSymVpad
\externalfigure[vpadWarningService][height=1.7\lH]
\SymVpad
\textDescrHead{Obiščite delavnico}(rumena) Potrebno je redno vzdrževanje vozila.
(glejte \about [sec:schedule] \atpage [sec:schedule])
ali pa je prišlo do napake motorja (potrebna delavnica).

+\:Sporočila o napakah \# 650 do \# 653 ali \# 703
\stopSymVpad


\startSymVpad
\externalfigure[vpadTDPF][height=1.7\lH]
\SymVpad
\textDescrHead{Filter za drobne delce}(rumena) Regeneracija filtra za drobne delce se začne takoj, ko obratovalno stanje to omogoča.

{\md Napotek:} {\lt Po možnosti motorja {\em ne} izključite, dokler prikaz sveti!}
\stopSymVpad


\startSymVpad
\externalfigure[vpadTBrakeError][height=1.7\lH]
\SymVpad
\textDescrHead{Zavorni sistem}(rdeča) Napaka v zavornem sistemu. Obvestite delavnico.

+\:Sporočilo o napaki \# 902
\stopSymVpad


\startSymVpad
\externalfigure[vpadTBrakePark][height=1.7\lH]
\SymVpad
\textDescrHead{Ročna zavora}(rdeča) Aktivirana je ročna zavora
vozila.

+\:Sporočilo o napaki \# 905
\stopSymVpad

\startSymVpad
\externalfigure[vpadTEngineHeating][height=1.7\lH]
\SymVpad
\textDescrHead{Sistem za razžarevanje}(rumena) Poteka razžarevanje motorja.

Utripajoča lučka prikazuje, da je v pomnilniku dogodkov bila registrirana napaka.
\stopSymVpad


\startSymVpad
\externalfigure[vpadTFuelReserve][height=1.7\lH]
\SymVpad
\textDescrHead{Nivo polnosti goriva}(rumena) Nivo polnosti goriva je zelo nizek
(rezerva).
\stopSymVpad

\startSymVpad
\externalfigure[vpadTBlink][height=1.7\lH]
\SymVpad
\textDescrHead{Opozorilne utripalke}(zelena) Aktivirane so opozorilne utripalke.
\stopSymVpad

\startSymVpad
\externalfigure[vpadTLowBeam][height=1.7\lH]
\SymVpad
\textDescrHead{Pozicijska luč}(zelena) Vključena je pozicijska luč.
\stopSymVpad

\startSymVpad
\HL\NC \externalfigure[vpadSyWaterTemp][height=1.7\lH]
\SymVpad
\textDescrHead{Temperatura}(rdeča) Previsoka temperatura hidravlične tekočine ali motorja. Obvestite delavnico.

+\:Sporočilo o napaki \# 700 ali \# 610
\stopSymVpad

\startSymVpad
\externalfigure[vpadWarningFilter][height=1.7\lH]
\SymVpad
\textDescrHead{Zamašen filter}(rdeča) Kombiniran hidravlični filter ali zračni filter je zamašen.

+\:Sporočilo o napaki \# 702 ali \# 851
\stopSymVpad

\startSymVpad
\externalfigure[vpadTSpray][height=1.7\lH]
\SymVpad
\textDescrHead{Vodna pištola}(rumena) Vključena je visokotlačna črpalka za vodno pištolo.

Stikalo \textSymb{temoin_buse} na stropni konzoli.
\stopSymVpad

\startSymVpad
\externalfigure[vpadTClear][height=1.7\lH]
\SymVpad
\textDescrHead{Sporočilo o napaki}(rdeča) V pomnilniku enote \Vpad je shranjeno sporočilo o napaki. Pritisnite tipko~\textSymb{vpadClear}, da prikažete vsa shranjena sporočila. Obvestite delavnico.
\stopSymVpad


\stop % local group for temporary redefinition of \textDescrHead

\stopsection

\page [yes]


\startsection [title={Meniju enote Vpad},
				reference={vpad:menu}]

\start

\setupTABLE [background=color,
			frame=off,
			option=stretch,textwidth=\makeupwidth]

\setupTABLE [r] [each] [style=sans, background=color, bottomframe=on, framecolor=TableWhite, rulethickness=1.5pt]
\setupTABLE [r] [first][backgroundcolor=TableDark, style=sansbold]
\setupTABLE [r] [odd]  [backgroundcolor=TableMiddle]
\setupTABLE [r] [even] [backgroundcolor=TableLight]
\bTABLE [split=repeat]
\bTABLEhead
\bTR\bTD Meni \eTD\bTD Oznaka\index{Vpad+Prikaz} \eTD\bTD Funkcija \eTD\eTR
\eTABLEhead

\bTABLEbody
\bTR\bTD \externalfigure [v:symbole:clear] \eTD\bTD Sporočila(o) o napaki \eTD\bTD Prikažite in potrdite sporočila o napakah, shranjena v enoti Vpad. \eTD\eTR
\bTR\bTD \framed[frame=off]{\externalfigure [v:symbole:beacon]\externalfigure [v:symbole:beacon:black]} \eTD\bTD Rotacijska luč \eTD\bTD Vklop|/|izklop rotacijske luči. \eTD\eTR
\bTR\bTD \externalfigure [v:symbole:engine] \eTD\bTD Podatki v resničnem času \eTD\bTD Prikaži obratovalne podatke motorja in hidravlike v resničnem času.\eTD\eTR
\bTR\bTD \externalfigure [v:symbole:oneTwoThree] \eTD\bTD Števec \eTD\bTD Prikaz števca obratovalnih ur: dnevni števec, sezonski števec, skupni števec\eTD\eTR
\bTR\bTD \externalfigure [v:symbole:serviceInfo] \eTD\bTD Vzdrževalni interval \eTD\bTD Prikaže datum in preostale obratovalne ure do naslednjega vzdrževanja ali naslednjega velikega servisa. \eTD\eTR
\bTR\bTD \externalfigure [v:symbole:trash] \eTD\bTD Števec \eTD\bTD Ponastavitev števca ali servisnega intervala \eTD\eTR
\bTR\bTD \externalfigure [v:symbole:sunglasses] \eTD\bTD Način zaslona \eTD\bTD Preklapljanje osvetlitve zaslona med \aW{dnevno} in \aW{nočno osvetlitvijo.} \eTD\eTR
\bTR\bTD \externalfigure [v:symbole:color] \eTD\bTD Svetlost|/|kontrast \eTD\bTD Nastavitve za svetlost in kontrast zaslona. \eTD\eTR
\bTR\bTD \externalfigure [v:symbole:select] \eTD\bTD Izbor \eTD\bTD Izbor označenega vnosa ali potrditev sporočila o napaki. \eTD\eTR
\bTR\bTD \externalfigure [v:symbole:return] \eTD\bTD Potrditev \eTD\bTD Potrditev izbora. \eTD\eTR
\bTR\bTD \framed[frame=off]{\externalfigure [v:symbole:up]\externalfigure [v:symbole:down]} \eTD\bTD Gor|/|dol, \\puščice \eTD\bTD Premik oznake navzgor|/|navzdol ali povečanje|/|zmanjšanje izbrane vrednosti. \eTD\eTR
\bTR\bTD \externalfigure [v:symbole:rSignal] \eTD\bTD Opozorilni signal za vzvratno vožnjo \eTD\bTD Vklop|/|izklop opozorilnega signala za vzvratno vožnjo. \eTD\eTR
\bTR\bTD \externalfigure [v:symbole:power] \eTD\bTD Izklop zaslona \eTD\bTD Pridržite pribl. 5\,s, da izključite zaslon enote Vpad. \eTD\eTR
\bTR\bTD \framed[frame=off]{\externalfigure [v:symbole:frontBrush]\externalfigure [v:symbole:frontBrush:black]}
\eTD\bTD Tretja metla\index{3.\,metla} (dodatna možnost) \eTD\bTD Sprostitev tretje metle.
Tretjo metlo je sedaj mogoče aktivirati, kot je opisano na strani~\at[sec:using:frontBrush]. \eTD\eTR
\eTABLEbody
\eTABLE
\stop


\subsection{Ostali simboli na zaslonu Vpad}


\subsubsubject{Zaloga sveže in reciklirane vode}


\start % local group for temporary redefinition of \textDescrHead [TF]
\define[1]\textDescrHead{{\bf#1\fourperemspace}}

\startSymVpad
\externalfigure[sym:vpad:water]
\SymVpad
\textDescrHead{Nivo polnosti sveže vode} Nivo polnosti sveže vode ne zadošča (najv. 190\,l; za voznikovo kabino).
\stopSymVpad

\startSymVpad
\externalfigure[sym:vpad:rwater:yellow]
\SymVpad
\textDescrHead{Nivo polnosti reciklirane vode}(rumena) Nivo polnosti reciklirane vode je pod toplotnim izmenjevalnikom. Hidravlično tekočina se ne hladi in vlažilni sistem sesalnega kanala se ne segreva.
\stopSymVpad

\startSymVpad
\externalfigure[sym:vpad:rwater]
\SymVpad
\textDescrHead{Nivo polnosti reciklirane vode}(rdeča) Nivo polnosti reciklirane vode ne zadošča (najv. 140\,l; pod posodo za umazanijo).
\stopSymVpad


\subsubsubject{Sesalni sistem} % nouveau

{\em Ta simbol je prikazan samo, če so metle izključene.}

\startSymVpad
\externalfigure[sym:vpad:sucker]
\SymVpad
\textDescrHead{Sesalno ustje} Sesalno ustje\index{sesalno ustje} je aktivirano:
sesalno ustje je spuščeno in turbina je aktivirana.
\stopSymVpad


\subsubsubject{Stranska metla} % nouveau

{\em Ta simbol je prikazan samo, če tretja metla ni aktivirana.}

\startSymVpad
\externalfigure[sym:vpad:sideBrush:83]
\SymVpad
\textDescrHead{Stranska metla} metla\index{pometanje}\index{stranska metla} je aktivirana. Hitrost vrtenja (v \% najv. hitrosti vrtenja [V\low{max}]) je prikazana pod simbolom, trenutna razbremenitev vsakokratne metle pa je prikazana nad simbolom (\type{~}~= lebdeči položaj, 14~= največja razbremenitev).

{\md Razbremenitev:} {\lt Nižja, kot je razbremenitev, višji je pritisk metle ob tla.}
\stopSymVpad


\startSymVpad
\externalfigure[sym:vpad:sideBrush:float:60]
\SymVpad
\textDescrHead{Lebdeči položaj}(zeleno na spodnjem robu)
Za izklop razbremenitve krmilno ročico za pribl. 2\,s pritisnite naprej; metla se bo sedaj s celotno lastno težo odložila na tla. Hitrost vrtenja metle je pri 60\hairspace\% od V\low{max} (primer).
\stopSymVpad

\startSymVpad
\externalfigure[sym:vpad:sideBrush]
\SymVpad
\textDescrHead{Stranski metli}Metli sta aktivirani. Metli mirujeta in sta dvignjeni.
\stopSymVpad


\subsubsubject{Tretja metla (dodatna možnost)} % nouveau

\startSymVpad
\externalfigure[sym:vpad:frontBrush]
\SymVpad
\textDescrHead{Tretja metla} Tretja metla\index{3.\,metla} je aktivirana. Hitrost vrtenja (v \% od največje hitrosti vrtenja [V\low{max}]) je prikazana pod simbolom.
\stopSymVpad


\startSymVpad
\externalfigure[sym:vpad:frontBrush:left]
\SymVpad
\textDescrHead{Lebdeči položaj}(zeleno na spodnjem robu)
Za izklop razbremenitve krmilno ročico za pribl. 2\,s pritisnite naprej; metla se bo sedaj s celotno lastno težo odložila na tla. Hitrost vrtenja metle je pri 70\hairspace\% od V\low{max} (primer).

{\md Smer vrtenja:} {\lt Na zgornjem robu je prikazana smer vrtenja (črna puščica na rumenem ozadju).}
\stopSymVpad

\stopsection

\stop % local group for temporary redefinition of \textDescrHead

\page [yes]


\startsection[title={Nastavitev svetlosti zaslona},
              reference={sec:vpad:brightness}]

Zaslon enote \Vpad\ je mogoče uporabljati v dveh vnaprej konfiguriranih stopnjah
svetlosti: način \aW{Dan}~– \textSymb{vpadSunglasses}, normalna
svetlost~– in način \aW{Noč}~– \textSymb{vpadMoon}, zmanjšana svetlost.
S tipko \textSymb{vpadColor} lahko dostopate do različnih
parametrov.

Če želite spremeniti vnaprej nastavljene stopnje svetlosti, postopajte kot sledi:

\startSteps
\item Pritisnite na sredino zaslona na dotik, da se lahko pomikate po menijski vrstici na spodnjem robu zaslona.
\item Pritisnite na simbol \textSymb{vpadSunglasses} oz.
\textSymb{vpadMoon}, da izberete način, ki ga želite spremeniti.
\item Pritisnite \textSymb{vpadColor}, da prikažete parametre.
\item S pomočjo
puščičnih simbolov~\textSymb{vpadUp}\textSymb{vpadDown} označite parameter, ki ga želite
spremeniti, in ga izberite z~\textSymb{vpadSelect}.
\item Vrednost spremenite s pomočjo simbolov
\textSymb{vpadMinus}\textSymb{vpadPlus}. Previdno, svetlosti ne zmanjšajte
preveč (\VpadOp{162} -255), da ne bi bilo več mogoče na zaslonu
prepoznati nič!
\stopSteps
\blank [1*big]

\start
\setupcombinations[width=\textwidth]
\startcombination [3*1]
{\setups[VpadFramedFigureHome]% \VpadFramedFigureK pour bande noire
\VpadScreenConfig{
\VpadFoot{\VpadPictures{vpadOneTwoThree}{vpadServiceInfo}{vpadSunglasses}{vpadColor}}}%
\framed{\null}}{Pritisnite na sredino zaslona na dotik}
{\setups[VpadFramedFigure]
\VpadScreenConfig{
\VpadFoot{\VpadPictures{vpadReturn}{vpadUp}{vpadDown}{vpadSelect}}}%
\framed{\bTABLE
\bTR\bTD \VpadOp{160} \eTD\eTR
\bTR\bTD [backgroundcolor=black,color=TableWhite] \VpadOp{162}\hfill 15 \eTD\eTR
\bTR\bTD \VpadOp{163}\hfill 180 \eTD\eTR
\bTR\bTD \VpadOp{164}\hfill 55 \eTD\eTR
\bTR\bTD \VpadOp{165}\hfill 3 \eTD\eTR
\eTABLE}}{Izberite z \textSymb{vpadSelect}}
{\setups[VpadFramedFigure]% \VpadFramedFigureK pour bande noire
\VpadScreenConfig{
\VpadFoot{\VpadPictures{vpadReturn}{vpadMinus}{vpadPlus}{vpadNull}}}%
\framed[backgroundscreen=.9]{\bTABLE
\bTR\bTD \VpadOp{160} \eTD\eTR
\bTR\bTD \VpadOp{162}\hfill -80 \eTD\eTR
\bTR\bTD \VpadOp{163}\hfill 180 \eTD\eTR
\bTR\bTD \VpadOp{164}\hfill 55 \eTD\eTR
\bTR\bTD \VpadOp{165}\hfill 3 \eTD\eTR
\eTABLE}}{Vrednost spremenite z \textSymb{vpadMinus}\textSymb{vpadPlus}}
\stopcombination
\stop
\blank [1*big]

\startSteps [continue]
\item Vrednost potrdite s tipko \textSymb{vpadReturn}.
\item Ponovno pritisnite na simbol \textSymb{vpadReturn}, da se
vrnete nazaj na glavni zaslon.
\stopSteps

\stopsection

\page [yes]


\startsection[title={Števec obratovalnih ur in kilometrov},
							reference={vpad:compteurs}]

Programska oprema enote \Vpad\ ima tri različna merilna obdobja: \aW{dan},
\aW{sezono}, \aW{skupaj}, v katerih lahko delujejo različni števci, kot so npr.
\aW{števec prevoženih kilometrov}, \aW{obratovalne ure} (motor ali krtača),
\aW{delovni čas} (na voznika).

Če želite odčitati števce ali jih ponastaviti, postopajte kot sledi:

\startSteps
\item Pritisnite na sredino zaslona na dotik, da se lahko
pomikate po menijski vrstici.
\item Pritisnite na simbol \textSymb{vpadOneTwoThree}, da prikažete
dnevni števec.
\item S simbolom nazaj|/|naprej||~\textSymb{vpadBW}\textSymb{vpadFW}
lahko preklopite na skupni ali sezonski števec.
\item Pritisnite tipko \textSymb{vpadTrash}, da ponastavite prikazani
števec.
\item V pogovornem oknu boste pozvani, da potrdite
ponastavitev.
\stopSteps
\blank [1*big]

\start
\setupcombinations[width=\textwidth]
\startcombination [3*1]
{\setups[VpadFramedFigure]% \VpadFramedFigureK pour bande noire
\VpadScreenConfig{
\VpadFoot{\VpadPictures{vpadOneTwoThree}{vpadServiceInfo}{vpadSunglasses}{vpadColor}}}%
\framed{\bTABLE
\bTR\bTD \VpadOp{120} \eTD\eTR
\bTR\bTD \VpadOp{123}\hfill 87,4\,h \eTD\eTR
\bTR\bTD \VpadOp{125}\hfill 62,0\,h \eTD\eTR
\bTR\bTD \VpadOp{126}\hfill 240,2\,km \eTD\eTR
\bTR\bTD \VpadOp{124}\hfill 901,9\,km \eTD\eTR
\bTR\bTD \VpadOp{127}\hfill 2,1\,l/h \eTD\eTR
\eTABLE}}{Pritisnite simbol~\textSymb{vpadOneTwoThree}, nato pa~\textSymb{vpadBW} oder~\textSymb{vpadFW}}
{\setups[VpadFramedFigure]
\VpadScreenConfig{
\VpadFoot{\VpadPictures{vpadReturn}{vpadBW}{vpadFW}{vpadTrash}}}%
\framed{\bTABLE
\bTR\bTD \VpadOp{121} \eTD\eTR
\bTR\bTD \VpadOp{123}\hfill 522,0\,h \eTD\eTR
\bTR\bTD \VpadOp{125}\hfill 662,8\,h \eTD\eTR
\bTR\bTD \VpadOp{126}\hfill 1605,5\,km \eTD\eTR
\bTR\bTD \VpadOp{124}\hfill 2608,4\,km \eTD\eTR
\bTR\bTD \VpadOp{127}\hfill 2,0\,l/h \eTD\eTR
\eTABLE}}{Ponastavite števec s tipko \textSymb{vpadTrash}.}
{\setups[VpadFramedFigure]% \VpadFramedFigureK pour bande noire
\VpadScreenConfig{
\VpadFoot{\VpadPictures{vpadReturn}{vpadTrash}{vpadNull}{vpadNull}}}%
\framed{\bTABLE
\bTR\bTD \VpadOp{121} \eTD\eTR
\bTR\bTD \null \eTD\eTR
\bTR\bTD \VpadOp{136} \eTD\eTR
\bTR\bTD \null \eTD\eTR
\bTR\bTD \VpadOp{137} \eTD\eTR
\eTABLE}}{Potrdite s tipko \textSymb{vpadTrash}}
\stopcombination
\stop
\blank [1*big]

\startSteps [continue]
\item Po potrebi vnesite geslo in nato potrdite ponastavitev s pomočjo simbola \textSymb{vpadTrash}.
\item Pritisnite na simbol \textSymb{vpadReturn}, da se vrnete na glavni zaslon.
\stopSteps

\stopsection

\page [yes]

\startsection[title={Vzdrževalni intervali},
							reference={vpad:maintenance}]

Vzdrževalni načrt vozila \sdeux\ pozna dve osnovni vrsti vzdrževanja: redno vzdrževanje in velik servis (s strani delavnice s pooblastilom službe za stranke \boschung).

Če želite odčitati števce ali jih ponastaviti, postopajte kot sledi:
\startSteps
\item Pritisnite na sredino zaslona na dotik, da se lahko
pomikate po menijski vrstici.
\item Pritisnite na simbol \textSymb{vpadServiceInfo}, da se prikažejo
vzdrževalni intervali.
\item S pomočjo
puščičnih simbolov~\textSymb{vpadUp}\textSymb{vpadDown} preklopite v želeni interval.
\item Pritisnite na simbol~\textSymb{vpadTrash}, da ponastavite
interval. S pomočjo
~\textSymb{vpadPlus}\textSymb{vpadMinus} vnesite geslo in ga potrdite
s tipko~\textSymb{vpadSelect}).
\item V pogovornem oknu boste pozvani, da potrdite
ponastavitev.
\stopSteps
\blank [1*big]

\start
\setupcombinations[width=\textwidth]
\startcombination [3*1]
{\setups[VpadFramedFigure]% \VpadFramedFigureK pour bande noire
\VpadScreenConfig{
\VpadFoot{\VpadPictures{vpadReturn}{vpadNull}{vpadNull}{vpadTrash}}}%
\framed{\bTABLE
\bTR\bTD[nc=2] \VpadOp{190} \eTD\eTR
\bTR\bTD \VpadOp{191}\eTD\bTD \VpadOp{195}\hfill 600\,h \eTD\eTR % [backgroundcolor=black,color=TableWhite]
\bTR\bTD \VpadOp{192}\eTD\bTD \VpadOp{195}\hfill 600\,h \eTD\eTR
\bTR\bTD \VpadOp{193}\eTD\bTD \VpadOp{195}\hfill 2400\,h \eTD\eTR
\eTABLE}}{Pritisnite na simbol~\textSymb{vpadTrash}, da ponastavite
interval.}
{\setups[VpadFramedFigure]
\VpadScreenConfig{
\VpadFoot{\VpadPictures{vpadReturn}{vpadMinus}{vpadPlus}{vpadSelect}}}%
\framed{\bTABLE
\bTR\bTD \VpadOp{190} \eTD\eTR
\bTR\bTD \hfill 31.3.2014 \eTD\eTR
\bTR\bTD \null \eTD\eTR
\bTR\bTD \null \eTD\eTR
\bTR\bTD \null \eTD\eTR
\bTR\bTD \null \eTD\eTR
\bTR\bTD \VpadOp{002}\hfill 0000 \eTD\eTR
\eTABLE}}{Vnesite geslo (številčna koda)}
{\setups[VpadFramedFigure]% \VpadFramedFigureK pour bande noire
\VpadScreenConfig{
\VpadFoot{\VpadPictures{vpadReturn}{vpadUp}{vpadDown}{vpadSelect}}}%
\framed{\bTABLE
\bTR\bTD \VpadOp{190} \eTD\eTR
\bTR\bTD[backgroundcolor=black,color=TableWhite] \VpadOp{041}\eTD\eTR % [backgroundcolor=black,color=TableWhite]
\bTR\bTD \VpadOp{042} \eTD\eTR
\bTR\bTD \VpadOp{043} \eTD\eTR
\eTABLE}}{Izberite in potrdite s tipko~\textSymb{vpadSelect}}
\stopcombination
\stop
\blank [1*big]

\startSteps [continue]
\item Ponastavitev potrdite s pomočjo simbola~\textSymb{vpadSelect}.
\item Pritisnite na simbol \textSymb{vpadReturn}, da se vrnete na glavni zaslon.
\stopSteps

\stopsection

\page [yes]


\startsection[title={Upravljanje napak prek enote Vpad},
							reference={vpad:error}]


Enota \Vpad\ prikaže napako\index{Vpad+sporočila o napakah}, ki so jo diagnosticirali elektronski krmilni sistemi in poslala vodila CAN.
Če je bila registrirana srednje težka napaka, zasveti simbol~\textSymb{VpadTClear} (rdeč).
Če gre za napako višje prioritete, sveti simbol~\textSymb{VpadTClear}. Dodatno pa se oglaša alarmni zvok.
Da odpravite alarm, je treba potrditi sporočilo
o napaki (potrditi kot \aW{vzeto na znanje}).

Da preberete in potrdite sporočila o napakah, postopajte kot sledi:

\startSteps
\item Pritisnite simbol~\textSymb{vpadClear} na zaslonu enote \Vpad.
\item Pritisnite simbol~\textSymb{vpadClear}, da potrdite izbrano sporočilo.
\item Poleg potrjenega sporočila se sedaj prikaže simbol \aW{\#}, ki
sporočilo označi kot \aW{vzeto na znanje}, oznaka pa preskoči
na naslednje sporočilo (če obstaja).
\item Ko potrdite vsa sporočila, se prikaz vrne na
glavni zaslon.
\stopSteps
\blank [1*big]

\start
\setupcombinations[width=\textwidth]
\startcombination [3*1]
{\setups[VpadFramedFigure]% \VpadFramedFigureK pour bande noire
\VpadScreenConfig{
\VpadFoot{\VpadPictures{vpadReturn}{vpadUp}{vpadDown}{vpadSelect}}}%
\framed{\bTABLE
\bTR\bTD \VpadEr{000} \eTD\eTR
\bTR\bTD [backgroundcolor=black,color=TableWhite] \VpadEr{851a} \eTD\eTR
\bTR\bTD \VpadEr{902} \eTD\eTR
\eTABLE}}{Prikaz sporočil}
{\setups[VpadFramedFigure]
\VpadScreenConfig{
\VpadFoot{\VpadPictures{vpadReturn}{vpadUp}{vpadDown}{vpadSelect}}}%
\framed{\bTABLE
\bTR\bTD \VpadEr{000} \eTD\eTR
\bTR\bTD [backgroundcolor=black,color=TableWhite] \VpadEr{851} \eTD\eTR
\bTR\bTD \VpadEr{902} \eTD\eTR
\eTABLE}}{Potrdite s tipko~\textSymb{vpadClear}}
{\setups[VpadFramedFigureHome]% \VpadFramedFigureK pour bande noire
\VpadScreenConfig{
\VpadFoot{\VpadPictures{vpadClear}{vpadBeacon}{vpadBeam}{vpadEngine}}}%
\framed{\null}}{Nazaj na glavni zaslon}
\stopcombination
\stop
\blank [1*big]

\startSteps [continue]
\item Za ponovni prikaz sporočil, pritisnite na simbol~\textSymb{vpadClear}. Sporočila o napakah se izbrišejo z enote \Vpad\
šele, ko odpravite vzrok težave.
\stopSteps


\subsection{Najpogostejša sporočila o napakah (z iskanjem napak)}

\subsubsubject{\VpadEr{604}} % {\#\ 604	Pression huile moteur basse}

+ \textSymb{vpadTEnginOilPressure}~– Nemudoma izključite motor. Preverite nivo olja in obvestite delavnico.


\subsubsubject{\VpadEr{609}} % {\#\ 609	Température eau refroidissement moteur haute}

+ \textSymb{vpadSyWaterTemp}~– Prekinite z delom. Motor pustite delovati brez obremenitve in opazujte razvoj temperature:

če temperatura pade, preverite nivoje polnosti hladilne tekočine, motornega olja in hidravlične tekočine ter stanje hladilnika.
Če so nivoji polnosti in hladilnik v redu, previdno nadaljujte z diagnozo napake v delavnici.

\subsubsubject{\VpadEr{610}} % {\#\ 610	Température eau refroidissement moteur trop haute}

+ \textSymb{vpadSyWaterTemp}~– Prekinite z delom. Preverite nivoje polnosti hladilne tekočine in motornega olja in nemudoma obvestite delavnico.


\subsubsubject{\VpadEr{650}} % {\#\ 650	Se rendre à un garage}

+ \textSymb{vpadWarningService}~– Nemudoma obvestite delavnico.
% \VpadEr{651} % {\#\ 651	Moteur en mode urgence}


\subsubsubject{\VpadEr{652}} % {\#\ 652	Inspection véhicule}
% \VpadEr{653} % {\#\ 653	Grand service moteur}

+ \textSymb{vpadWarningService}~– Čas je za naslednje redno vzdrževanje. Poglejte v vzdrževalni načrt in z delavnico dogovorite termin.


\subsubsubject{\VpadEr{700}} % {\#\ 700	Température d'huile hydraulique}

+ \textSymb{vpadSyWaterTemp}~– Prekinite z delom. Motor pustite delovati brez obremenitve in opazujte razvoj temperature:

če temperatura pade, preverite nivoje polnosti hladilne tekočine, motornega olja in hidravlične tekočine ter stanje hladilnika.
Če so nivoji polnosti in hladilnik v redu, previdno nadaljujte z diagnozo napake v delavnici.


\subsubsubject{\VpadEr{702}} % {\#\ 702	Filtre d'huile hydraulique}

+ \textSymb{vpadWarningFilter}~– Hidravlični||povratni in|/|ali sesalni filter je zamašen. Nemudoma zamenjajte filtrski vložek.
% \VpadEr{703} % {\#\ 703	Vidange d'huile hydraulique}


\subsubsubject{\VpadEr{800}} % {\#\ 800	Interrupteur d'urgence actionné}

+ \textSymb{vpadTClear}~– Pritisnili ste stikalo za izklop v sili. Izključite vžig in znova zaženite motor, da izbrišete sporočilo.


\subsubsubject{\VpadEr{801}} % {\#\ 905	Frein à main actionné}

Posoda za umazanijo je dvignjena ali pa ni povsem spuščena. Hitrost vozila je omejena na 5\,km/h, dokler posode za umazanijo ne spustite.

\subsubsubject{\VpadEr{851}} % {\#\ 851	Filtre à air}

+ \textSymb{vpadWarningFilter}~– Zračni filter je zamašen. Nemudoma zamenjajte filtrski vložek.


\subsubsubject{\VpadEr{902}} % {\#\ 902	Pression de freinage}

+ \textSymb{vpadTBrakeError}~– Nezadosten zavorni tlak. Takoj prekinite z delom in nemudoma obvestite delavnico.
% \VpadEr{904} % {\#\ 904	Interrupteur de direction d'avancement}


\subsubsubject{\VpadEr{905}} % {\#\ 905	Frein à main actionné}

+ \textSymb{vpadTBrakePark}~– Ročna zavora ni povsem sproščena. Hitrost vozila je omejena na 5\,km/h, dokler ne sprostite ročne zavore.


\stopsection

\stopchapter

\stopcomponent













