\startcomponent c_20_prescriptions_s2_120-de
\product prd_ba_s2_120-de


\chapter [safety:risques] {Sicherheitsvorschriften}

\setups [pagestyle:marginless]


\section{Grundlegende Anweisungen}

\subsubject{Gesetzliche Grundlagen}

Unfälle können schwerwiegende Konsequenzen nach sich ziehen, sowohl für den Arbeitgeber als auch für den Angestellten. Wir möchten die Pflichten beider nochmals ins Gedächtnis rufen:\note[prescription:user:right].

Der Arbeitgeber ist verpflichtet, bevor er einen Angestellten mit der Bedienung der Kehrmaschine vertraut, die folgenden Punkte zu beachten:

\startSteps
\item Jeder Fahrzeugführer muss zum Führen des Fahrzeugs ausgebildet worden sein. Der Ausbildungsnachweis muss vorhanden sein.
\item Jeder Fahrzeugführer muss über eine formelle Fahrerlaubnis verfügen. Diese darf nur ausgestellt werden, wenn die folgenden drei Bedingungen erfüllt sind:
\startitemize [2]
\item Der Angestellte hat einen medizinischen Eignungstest durch den Betriebsarzt bestanden.
\item Der Angestellte kennt die Örtlichkeiten der Arbeitseinsätze und ist mit allen Sicherheitsvorschriften den Einsatzort des Fahrzeugs betreffend, welche ihm von seinem Vorgesetzten übermittelt wurden, vertraut.
\item Der Angestellte hat einen Eignungstest bestanden, welcher die zum Führen des Fahrzeugs nötigen Kenntnisse bescheinigt.
\stopitemize
\stopSteps

Wenn die Höchstgeschwindigkeit des Fahrzeugs mehr als 25\,km/h beträgt\note[prescription:user:right], muss das Fahrzeug amtlich zugelassen sein und der Fahrzeugführer muss sich im Besitz folgender Fahrerlaubnis befinden:
\startitemize
\item Fahrerlaubnis der Klasse B\note[prescription:lisence] für Fahrzeuge mit einem zulässigen Gesamtgewicht von weniger als 3,5~Tonnen bzw.
\item Fahrerlaubnis der Klasse C\note[prescription:lisence] für Fahrzeuge mit einem zulässigen Gesamtgewicht von mehr als 3,5~Tonnen.
\stopitemize

Wenn die Höchstgeschwindigkeit des Fahrzeugs 25\,km/h beträgt, muss der Fahrzeugführer mindestens die auf öffentlichen Straßen und Plätzen gültige Straßenverkehrsordnung kennen, selbst wenn zum Führen des Fahrzeugs keine Fahrerlaubnis der Klasse B\note[prescription:user:right] erforderlich ist.

\footnotetext [prescription:user:right] {Die Verpflichtungen von Arbeitgeber und Angestelltem können je nach Land oder Region variieren. Machen Sie sich mit den in Ihrem Land bzw. Ihrer Region geltenden Vorschriften vertraut.}

\footnotetext[prescription:lisence] {Richtlinie 2006/126/EG des Europäischen Parlaments und des Rates vom 20.~Dezember 2006 über den Führerschein.}


\subsubject{Benutzungsbedingungen}

Die \sdeux\ darf ausschließlich benutzt werden, wenn sie sich in einwandfreiem Betriebszustand befindet. Darüber hinaus hat der Bediener die in der vorliegenden Betriebsanleitung enthaltenen Sicherheitsanweisungen und Vorschriften einzuhalten. Funktionsstörungen, die die Sicherheit beeinträchtigen, müssen unverzüglich durch einen geeigneten Fachbetrieb beseitigt|/|repariert werden.
\blank [big]

\startSymList
\externalfigure [s2_inspection] [width=4.5em]
\SymList
{\md Tägliche Wartung:}
Unterziehen Sie das Fahrzeug nach jedem Arbeitseinsatz einer Inspektion und reparieren Sie sichtbare Schäden und Defekte. Informieren Sie im Fall von Schäden oder Funktionsstörungen des Fahrzeugs unverzüglich die Fachwerkstatt. Ist dies nicht möglich, halten Sie das Fahrzeug unverzüglich an und sichern Sie den Pannenort.
\stopSymList


\subsubject{Bestimmungsgemäße Benutzung}

Die \sdeux\ ist für Reinigungs- und Unterhaltsarbeiten von Straßen, Wegen und Plätzen konzipiert. Jegliche Benutzung außerhalb dieses Rahmens gilt als nicht bestimmungsgemäß. Infolgedessen weist die Firma \boschung\ jegliche Verantwortung für dadurch entstandene Schäden zurück. Bei unzweckgemäßer Benutzung hat allein der Bediener die Folgen zu verantworten. {\em Die bestimmungsgemäße Benutzung umfasst ebenfalls die Einhaltung von Sicherheitsanweisungen und Wartungsplan, enthalten in der vorliegenden Betriebsanleitung.}


\section{Fahren auf öffentlichen Straßen}

\subsubject{Allgemeine Vorschriften}

Neben den Betriebsanweisungen sind alle allgemein gültigen Regeln, die geltenden gesetzlichen und anderen Vorschriften und Bestimmungen zur Unfallverhütung und zum Umweltschutz einzuhalten.


\subsubject{Beifahrerplatz}

Eine Beifahrerin~/ ein Beifahrer darf auf dem zu diesem Zweck vorgesehenen
Sitz, dem sog. {\em Beifahrersitz}, Platz nehmen.


\subsubject{Sicherheitsgurt}

\startSymList
% \externalfigure [prescription:safety:belt]
\PMbelt
\SymList
Fahrer und Beifahrer der \sdeux\ müssen gemäß der geltenden Straßenverkehrsordnung den
Sicherheitsgurt anlegen, wenn sie im Fahrzeug Platz nehmen.
\stopSymList


\subsubject{Sehen und gesehen werden}

\startSymList
\externalfigure [travaux_deviation] [width=3.5em]
\SymList
Sorgen Sie dafür, dass Sie gut sichtbar sind, insbesondere auf viel befahrenen Straßen.

Wenn der Fahrzeugführer bei einem bestimmten Fahrmanöver oder einer bestimmten Einsatztätigkeit nicht über ausreichende Sicht verfügt, muss er die Hilfe einer Hilfsperson, mit der er Sichtkontakt aufrechterhält, in Anspruch nehmen.
\stopSymList


\subsubject{Beleuchtung und Signalmittel}

In Abhängigkeit von der geltenden Straßenverkehrsordnung sind ggf. auch tagsüber
Scheinwerfer und|/|oder Rundumkennleuchte des Fahrzeugs einzuschalten.


\subsubject{Benutzung von Mobiltelefonen}

\startSymList
\PPphone
\SymList
Die Verwendung eines Mobiltelefons oder Funkgeräts während der Fahrt auf öffentlichen Straßen ist untersagt, außer, das Fahrzeug ist mit einer Freisprechvorrichtung ausgestattet.

Telefonieren\index{Sicherheit+Mobiltelefon} am Steuer~– auch mit Freisprechvorrichtung~– beeinträchtigt in jedem Fall die Konzentration auf den Straßenverkehr.
\stopSymList


\section{Wartungsvorschriften}

\subsubject{Wartungsanweisungen}

Das Wartungspersonal muss vor Aufnahme der Arbeiten die Betriebsanleitung der \sdeux, insbesondere die Abschnitte zu Sicherheit und Wartung, gelesen haben.


\subsubject{Erforderliche Qualifikationen}

\startSymList
\externalfigure [mecanicienne] [width=3.5em]
\SymList
Nur Personen, die in einer geeigneten Schulung die erforderlichen Kenntnisse erlangt haben, sind befugt, Wartungsarbeiten an der \sdeux\ vorzunehmen. Dies gilt insbesondere für Arbeiten am Motor, am Bremssystem, an der Lenkung und an der Elektro- und Hydraulikanlage.
\stopSymList


\testpage [6]
\subsubject{Aufsicht}

\startSymList
\externalfigure [mecanicien_hyerarchie] [width=3.5em]
\SymList
Personen, die sich in der Ausbildung~– Praktikum oder Lehre~– befinden, dürfen nur unter Aufsicht einer Fachperson am Fahrzeug arbeiten. Überprüfen Sie stichprobenartig, ob das Personal die Betriebsanleitung und die Sicherheitsvorschriften einhält.
\stopSymList


\subsubject{Schweißarbeiten}

\startSymList
\externalfigure [pince_soudure2] [width=3.5em]
\SymList
Vor der Durchführung von Schweißarbeiten an Karosserie oder Chassis müssen die
Batterie und alle elektronischen Steuergeräte unbedingt abgeklemmt werden.
\stopSymList

\subsubject{Fahrzeugreinigung}

\startSymList
\externalfigure [washer_pressure] [width=3.5em]
\SymList
Lesen Sie vor der Reinigung der \sdeux\ den Abschnitt \about[sec:cleaning] ab \atpage[sec:cleaning], insbesondere den Abschnitt zu den Reinigungsvorschriften.
\stopSymList


\subsubject{Zugänglichkeit der Fahrzeugdokumentation}

\startSymList
\externalfigure [lecteur_1] [width=3.5em]%\PMrtfm
\SymList
Bewahren Sie bei Einsätzen die Fahrzeugdokumentation stets leicht zugänglich im Führerhaus des Fahrzeugs auf.
\stopSymList


\section{Besondere Benutzungsbestimmungen}

\subsubject{Fahrzeughöhe}

\startSymList
\PPmaxheight
\SymList
Vergewissern Sie sich bei Arbeiten|/|Fahrten in nicht offenem Gelände (Tiefgaragen, Unterführungen, Stromleitungen etc.) stets, dass die Durchfahrtshöhe für die \sdeux\ ausreichend ist (siehe \in{Abschnitt}[sec:measurement], \atpage[sec:measurement]).
\stopSymList


\subsubject{Stabilität des Fahrzeugs}

Vermeiden Sie jegliche Manöver, die die Stabilität des Fahrzeugs beeinträchtigen könnten. Bei erhöhter Geschwindigkeit in Kurven könnte die \sdeux\ aufgrund ihrer schmalen Bauweise und ihres erhöhten Schwerpunkts bei vollem Schmutzbehälter kippen.


\subsubject{Ungewollte Fahrzeugbewegung}

Wenn Sie das Fahrzeug verlassen, sichern Sie es gegen die Benutzung durch unbefugte Personen. Aktivieren Sie grundsätzlich die Feststellbremse, bevor Sie das Fahrzeug verlassen; sichern sie die Räder ggf. mit Keilen.

\startbuffer [prescription:handbrake]
\starttextbackground [CB]
\startPictPar
\PPstop
\PictPar
{\md Ziehen Sie die Feststellbremse fest an!} Andernfalls kann sich das Fahrzeug ungewollt in Bewegung setzen, selbst\index{Feststellbremse+Gefahrenpotenzial} auf kaum wahrnehmbarem Gefälle, und einen Unfall mit der Gefahr tödlicher Verletzungen Dritter verursachen.

{\lt Durch das hydrostatische Antriebssystem reduziert sich im Stillstand schrittweise der Druck im Hydraulikkreis, was eine Verringerung der Haltekraft des Motors nach sich zieht. Aus diesem Grund ist es besonders wichtig, die Feststellbremse beim Verlassen des Fahrzeugs stets fest anzuziehen.}
\stopPictPar
\stoptextbackground

\stopbuffer

\getbuffer [prescription:handbrake]


\testpage [6]
\subsubject{Schmutzbehälter}

\startbuffer [prescription:container:gravity]
\starttextbackground [CB]
\startPictPar
\PHgravite
\PictPar
{\md Unfallgefahr:}
{\lt Beim Hochkippen des Schmutzbehälters verlagert sich der Schwerpunkt nach oben. Hierdurch erhöht sich die Kippgefahr des Fahrzeugs. Achten Sie deshalb beim Kippen des Schmutzbehälters darauf, dass sich das Fahrzeug auf waagerechtem und festem Untergrund befindet.}
\stopPictPar
\stoptextbackground

\stopbuffer

\getbuffer [prescription:container:gravity]


\startbuffer [prescription:container:tilt]
\starttextbackground [CB]
\startPictPar
\PHcrushing
\PictPar
{\md Unfallgefahr:}
{\lt Führen Sie nie Arbeiten unter dem Schmutzbehälter aus, bevor Sie die Sicherungsstreben an den hydraulischen Hebezylindern des Schmutzbehälters angebracht haben.}
\stopPictPar
\stoptextbackground

\stopbuffer

\getbuffer [prescription:container:tilt]


\stopcomponent
