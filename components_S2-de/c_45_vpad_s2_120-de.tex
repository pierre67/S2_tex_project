\startcomponent c_45_vpad_s2_120-de
\product prd_ba_s2_120-de

\startchapter[title={Bordcomputer (Vpad)},
							reference={sec:vpad}]

\setups[pagestyle:marginless]


\startsection[title={Beschreibung des Vpad},
							reference={vpad:description}]

\startfigtext [left] {Das Vpad SN im Führerstand}
{\externalfigure[vpad:inside:view]}
\textDescrHead{Innovativ, intelligent … } Das \Vpad\ wurde zur Steuerung von
Aggregaten aus dem Kommunalbereich konzipiert, deren Technologie immer komplexer
geworden ist und die eine Fülle verschiedenster Funktionen zur Verfügung
stellen.
Mit dem \Vpad\ hat der Bediener ein System an der Hand, das sich nicht darauf
beschränkt, in Echtzeit Informationen~– visuell oder akustisch~– über sämtliche
Arbeits- und Maschinenvorgänge zu liefern.
Was das \Vpad\ vor allem auszeichnet und wo es neue Maßstäbe setzt, ist die
intuitive Benutzerführung, Bedienerergonomie und Befehlslogik.

Dank seiner Funktionsvielfalt kann das \Vpad\ überaus flexibel genutzt werden,
und wird dadurch zu viel mehr als einer bloßen elektronischen Steuereinheit.
\stopfigtext

\textDescrHead{… universell} Kompatibilität und Flexibilität standen bei der
Entwicklung des \Vpad\ im Focus:
Als modulare Steuereinheit lässt es sich individuell an lokale Gegebenheiten und
Ausstattungsvarianten anpassen, und durch seine zahlreichen elektronischen
Schnittstellen und Wege der Datenübertragung~– bis hin zum WLAN~– stehen alle
Möglichkeiten offen.
Das \Vpad\ arbeitet mit modernster Elektronik mit 32-bit||Technologie und
Echtzeit||Betriebssystem.
\vfill


\startfigtext[left]{Multifunktionskonsole}
{\externalfigure[console:topview]}
\textDescrHead{… und modular} Durch seine Modularität verfügt des \Vpad\ über
einen enormen Vorteil:
So kann die in der \sdeux\ serienmäßig zum Einsatz kommende Version~SN schrittweise
durch weitere Komponenten, wie beispielsweise ein Modem oder eine Konsole (siehe
Abbildung), jederzeit erweitert werden.
Die Modularität beschränkt sich nicht auf die Hardware, auch softwareseitig ist
das System in hohem Maße erweiterbar und an sich ändernde Bedürfnisse anpassbar.

Die Multifunktionskonsole der \sdeux\ ist eine hochentwickelte Schnittstelle
zwischen Bediener und Maschine. Das gesamte Kehr-|/|Saugsystem lässt sich über diese Konsole steuern.
\stopfigtext

\page [yes]


\subsection[vpad:home]{Hauptbildschirm}

%% Note: outcommented by PB
% \placefig[left][fig:vpad:engineData]{Accueil mode transport}
% {\scale[sx=1.5,sy=1.5]
% {\setups[VpadFramedFigureHome]
% \VpadScreenConfig{
% \VpadFoot{\VpadPictures{vpadClear}{vpadBeacon}{vpadEngine}{vpadSignal}}}
% \framed{\null}}
% }


\start

\setupcombinations[width=\textwidth]

\placefig [here][fig:vpad:engineData]{Hauptbildschirm}
{\startcombination [2*1]
{\setups[VpadFramedFigureHome]% \VpadFramedFigureK pour bande noire
\VpadScreenConfig{
\VpadFoot{\VpadPictures{vpadClear}{vpadBeacon}{vpadEngine}{vpadSignal}}}%
\scale[sx=1.5,sy=1.5]{\framed{\null}}}{\aW{Fahr}modus}
{\setups[VpadFramedFigureWork]% \VpadFramedFigureK pour bande noire
\VpadScreenConfig{
\VpadFoot{\VpadPictures{vpadClear}{vpadBeacon}{vpadEngine}{vpadSignal}}}%
\scale[sx=1.5,sy=1.5]{\framed{\null}}}{\aW{Arbeits}modus}
\stopcombination}

\stop

\blank [1*big]

Der Hauptbildschirm des \Vpad\ umfasst alle
nötigen Elemente zur Überwachung sämtlicher Funktionen der \sdeux.

Im oberen Bereich befinden sich die Kontrollanzeigen.

Der mittlere Bereich zeigt in Echtzeit u.\,a. folgende Daten an:
Geschwindigkeit, Drehzahl und Temperatur des Motors, Kraftstofffüllstand, Füllstand des Recyclingwassers etc.

Der Modus \aW{Fahren} wird durch einen Hasen~\textSymb{transport_mode} symbolisiert, der Modus \aW{Arbeit} durch eine Schildkröte~\textSymb{working_mode}.

Die Menüleiste am unteren Rand zeigt die verfügbaren Menüs an: Drücken Sie in die Mitte des berührungsempfindlichen Bildschirms (Touchscreens), um zusätzliche Menüs anzuzeigen.

\page [yes]


\defineparagraphs[SymVpad][n=2,distance=4mm,rule=off,before={\page[preference]},
							after={\nobreak\hrule\blank [2*medium]\page[preference]}]
\setupparagraphs [SymVpad][1][width=4em,inner=\hfill]


\subsection{Kontrollanzeigen auf dem Vpad-Bildschirm} % nouveau


\start % local group for temporary redefinition of \textDescrHead [TF]
\define[1]\textDescrHead{{\bf#1\fourperemspace}}

\startSymVpad
\externalfigure[vpadTEnginOilPressure][height=1.7\lH]
\SymVpad
\textDescrHead{Motoröldruck}(rot) Motoröldruck zu gering. Stellen Sie den Motor
unverzüglich ab.

+\:Fehlermeldung \# 604
\stopSymVpad

\startSymVpad
\externalfigure[vpadWarningBattery][height=1.7\lH]
\SymVpad
\textDescrHead{Batterieladung}(rot) Batterieladestrom zu gering. Verständigen Sie die Werkstatt.
\stopSymVpad


\startSymVpad
\externalfigure[vpadWarningEngine1][height=1.7\lH]
\SymVpad
\textDescrHead{Motordiagnose}(gelb) Fehler in der Motorsteuerung. Verständigen Sie die Werkstatt.
\stopSymVpad


\startSymVpad
\externalfigure[vpadWarningService][height=1.7\lH]
\SymVpad
\textDescrHead{Werkstatt aufsuchen}(gelb) Reguläre Fahrzeugwartung fällig
(siehe \about [sec:schedule] \atpage [sec:schedule])
oder ein Motorfehler wurde registriert (Fachwerkstatt erforderlich).

+\:Fehlermeldungen \# 650 bis \# 653, oder \# 703
\stopSymVpad


\startSymVpad
\externalfigure[vpadTDPF][height=1.7\lH]
\SymVpad
\textDescrHead{Partikelfilter}(gelb) Regenerierung des Partikelfilters wird gestartet, sobald es der Betriebzustand erlaubt.

{\md Hinweis:} {\lt Stellen Sie wenn möglich den Motor {\em nicht} ab, solange diese Anzeige aufleuchtet!}
\stopSymVpad


\startSymVpad
\externalfigure[vpadTBrakeError][height=1.7\lH]
\SymVpad
\textDescrHead{Bremssystem}(rot) Fehler im Bremssystem. Verständigen Sie die Werkstatt.

+\:Fehlermeldung \# 902
\stopSymVpad


\startSymVpad
\externalfigure[vpadTBrakePark][height=1.7\lH]
\SymVpad
\textDescrHead{Feststellbremse}(rot) Feststellbremse des Fahrzeugs ist
aktiviert.

+\:Fehlermeldung \# 905
\stopSymVpad

\startSymVpad
\externalfigure[vpadTEngineHeating][height=1.7\lH]
\SymVpad
\textDescrHead{Vorglühanlage}(gelb) Motor wird vorgeglüht.

Eine blinkende Lampe zeigt an, dass ein Fehler im Ereignisspeicher registriert wurde.
\stopSymVpad


\startSymVpad
\externalfigure[vpadTFuelReserve][height=1.7\lH]
\SymVpad
\textDescrHead{Kraftstofffüllstand}(gelb) Kraftstofffüllstand ist sehr niedrig
(Reserve).
\stopSymVpad

\startSymVpad
\externalfigure[vpadTBlink][height=1.7\lH]
\SymVpad
\textDescrHead{Warnblinkanlage}(grün) Warnblinkanlage ist aktiviert.
\stopSymVpad

\startSymVpad
\externalfigure[vpadTLowBeam][height=1.7\lH]
\SymVpad
\textDescrHead{Standlicht}(grün) Standlicht ist eingeschaltet.
\stopSymVpad

\startSymVpad
\HL\NC \externalfigure[vpadSyWaterTemp][height=1.7\lH]
\SymVpad
\textDescrHead{Temperatur}(rot) Temperatur der Hydraulikflüssigkeit oder des Motors zu hoch. Verständigen Sie die Werkstatt.

+\:Fehlermeldung \# 700 oder \# 610
\stopSymVpad

\startSymVpad
\externalfigure[vpadWarningFilter][height=1.7\lH]
\SymVpad
\textDescrHead{Filter zugesetzt}(rot) Der kombinierte Hydraulikfilter oder der Luftfilter ist zugesetzt.

+\:Fehlermeldung \# 702 oder \# 851
\stopSymVpad

\startSymVpad
\externalfigure[vpadTSpray][height=1.7\lH]
\SymVpad
\textDescrHead{Wasserpistole}(gelb) Hochdruckwasserpumpe für die Wasserpistole ist aktiviert.

Schalter \textSymb{temoin_buse} in der Deckenkonsole.
\stopSymVpad

\startSymVpad
\externalfigure[vpadTClear][height=1.7\lH]
\SymVpad
\textDescrHead{Fehlermeldung}(rot) Eine Fehlermeldung befindet sich im Speicher des \Vpad. Drücken Sie die Taste~\textSymb{vpadClear}, um alle registrierten Nachrichten anzuzeigen. Verständigen Sie die Werkstatt.
\stopSymVpad


\stop % local group for temporary redefinition of \textDescrHead

\stopsection

\page [yes]


\startsection [title={Die Menüs des Vpad},
				reference={vpad:menu}]

\start

\setupTABLE [background=color,
			frame=off,
			option=stretch,textwidth=\makeupwidth]

\setupTABLE [r] [each] [style=sans, background=color, bottomframe=on, framecolor=TableWhite, rulethickness=1.5pt]
\setupTABLE [r] [first][backgroundcolor=TableDark, style=sansbold]
\setupTABLE [r] [odd]  [backgroundcolor=TableMiddle]
\setupTABLE [r] [even] [backgroundcolor=TableLight]
\bTABLE [split=repeat]
\bTABLEhead
\bTR\bTD Menü \eTD\bTD Bezeichnung\index{Vpad+Anzeige} \eTD\bTD Funktion \eTD\eTR
\eTABLEhead

\bTABLEbody
\bTR\bTD \externalfigure [v:symbole:clear] \eTD\bTD Fehlermeldung(en) \eTD\bTD Die im Vpad verzeichneten Fehlermeldungen anzeigen und quittieren (bestätigen). \eTD\eTR
\bTR\bTD \framed[frame=off]{\externalfigure [v:symbole:beacon]\externalfigure [v:symbole:beacon:black]} \eTD\bTD Rundumkennleuchte \eTD\bTD Rundumkennleuchte ein|/|aus \eTD\eTR
\bTR\bTD \externalfigure [v:symbole:engine] \eTD\bTD Echzeitdaten \eTD\bTD Echtzeitbetriebsdaten von Motor und Hydraulik anzeigen\eTD\eTR
\bTR\bTD \externalfigure [v:symbole:oneTwoThree] \eTD\bTD Zähler \eTD\bTD Anzeige der Betriebsstundenzähler: Tageszähler, Saisonzähler, Gesamtzähler\eTD\eTR
\bTR\bTD \externalfigure [v:symbole:serviceInfo] \eTD\bTD Wartungsintervall \eTD\bTD Zeigt das Datum sowie die verbleibenden Betriebsstunden bis zur nächsten Wartung oder bis zum nächsten großen Service an \eTD\eTR
\bTR\bTD \externalfigure [v:symbole:trash] \eTD\bTD Zähler \eTD\bTD Zähler zurücksetzen oder Serviceintervall zurücksetzen \eTD\eTR
\bTR\bTD \externalfigure [v:symbole:sunglasses] \eTD\bTD Bildschirmmodus \eTD\bTD Art der Bildschirmbeleuchtung umschalten zwischen \aW{Tag} und \aW{Nacht} \eTD\eTR
\bTR\bTD \externalfigure [v:symbole:color] \eTD\bTD Helligkeit|/|Kontrast \eTD\bTD Einstellungen für Helligkeit und Kontrast des Bildschirms \eTD\eTR
\bTR\bTD \externalfigure [v:symbole:select] \eTD\bTD Auswahl \eTD\bTD Auswählen des markierten Eintrags oder quittieren einer Fehlermeldung \eTD\eTR
\bTR\bTD \externalfigure [v:symbole:return] \eTD\bTD Bestätigung \eTD\bTD Bestätigen der Auswahl \eTD\eTR
\bTR\bTD \framed[frame=off]{\externalfigure [v:symbole:up]\externalfigure [v:symbole:down]} \eTD\bTD Auf|/|Ab, \\Pfeile \eTD\bTD Markierung nach oben|/|unten verschieben oder ausgewählten Wert erhöhen|/|verringern \eTD\eTR
\bTR\bTD \externalfigure [v:symbole:rSignal] \eTD\bTD Rückfahrwarnton \eTD\bTD Akustisches Rückfahrwarnsignal aktivieren|/|deaktivieren \eTD\eTR
\bTR\bTD \externalfigure [v:symbole:power] \eTD\bTD Bildschirm ausschalten \eTD\bTD Etwa 5\,s gedrückt halten, um den Bildschirm des Vpad auszuschalten. \eTD\eTR
\bTR\bTD \framed[frame=off]{\externalfigure [v:symbole:frontBrush]\externalfigure [v:symbole:frontBrush:black]}
\eTD\bTD Dritter Besen\index{3.\,Besen} (Option) \eTD\bTD Dritten Besen freischalten.
Der dritte Besen kann nun aktiviert werden wie auf Seite~\at[sec:using:frontBrush] beschrieben. \eTD\eTR
\eTABLEbody
\eTABLE
\stop


\subsection{Weitere Symbole auf dem Vpad-Bildschirm}


\subsubsubject{Frischwasser- und Recyclingwasservorrat}


\start % local group for temporary redefinition of \textDescrHead [TF]
\define[1]\textDescrHead{{\bf#1\fourperemspace}}

\startSymVpad
\externalfigure[sym:vpad:water]
\SymVpad
\textDescrHead{Füllstand Frischwasser} Füllstand des Frischwassers nicht ausreichend (max. 190\,l; hinter dem Führerhaus).
\stopSymVpad

\startSymVpad
\externalfigure[sym:vpad:rwater:yellow]
\SymVpad
\textDescrHead{Füllstand Recyclingwasser}(gelb) Füllstand des Recyclingwassers unterhalb des Wärmetauschers. Es erfolgt keine Kühlung der Hydraulikflüssigkeit und keine Anwärmung des Befeuchtungssystems des Saugkanals.
\stopSymVpad

\startSymVpad
\externalfigure[sym:vpad:rwater]
\SymVpad
\textDescrHead{Füllstand Recyclingwasser}(rot) Füllstand des Recyclingwassers nicht ausreichend (max. 140\,l; unter dem Schmutzbehälter).
\stopSymVpad


\subsubsubject{Saugsystem} % nouveau

{\em Dieses Symbol wird nur angezeigt, wenn die Besen deaktiviert sind.}

\startSymVpad
\externalfigure[sym:vpad:sucker]
\SymVpad
\textDescrHead{Saugmund} Saugsystem\index{Saugmund} aktiviert:
Saugmund ist abgesenkt und Turbine ist aktiviert.
\stopSymVpad


\subsubsubject{Seitenbesen} % nouveau

{\em Dieses Symbol wird nur angezeigt, wenn der dritte Besen nicht aktiviert ist.}

\startSymVpad
\externalfigure[sym:vpad:sideBrush:83]
\SymVpad
\textDescrHead{Seitenbesen} Besen\index{Kehren}\index{Seitenbesen} aktiviert. Die Rotationsgeschwindigkeit (in \% der max. Rotatationsgeschwindigkeit [V\low{max}]) wird unterhalb des Symbols angezeigt, die aktuelle Entlastung des jeweiligen Besens wird oberhalb  des Symbols angezeigt (\type{~}~= Schwimmstellung, 14~= Maximale Entlastung).

{\md Entlastung:} {\lt Je niedriger die Entlastung, desto höher der Druck der Besen auf den Boden.}
\stopSymVpad


\startSymVpad
\externalfigure[sym:vpad:sideBrush:float:60]
\SymVpad
\textDescrHead{Schwimmstellung}(grün am unteren Rand)
Um die Entlastung auszuschalten, halten Sie den Joystick für etwa 2\,s nach vorn gedrückt; der Besen liegt nun mit seinem ganzen Eigengewicht auf dem Boden auf. Die Rotationsgeschwindigkeit der Besen ist bei 60\hairspace\% der V\low{max} (Beispiel).
\stopSymVpad

\startSymVpad
\externalfigure[sym:vpad:sideBrush]
\SymVpad
\textDescrHead{Seitenbesen} Die Besen sind aktiviert. Sie stehen still und sind angehoben.
\stopSymVpad


\subsubsubject{Dritter Besen (Option)} % nouveau

\startSymVpad
\externalfigure[sym:vpad:frontBrush]
\SymVpad
\textDescrHead{Dritter Besen} Der dritte Besen\index{3.\,Besen} ist aktiviert. Die Rotatationsgeschwindigkeit (in \% der max. Rotatationsgeschwindigkeit [V\low{max}]) wird unterhalb des Symbols angezeigt.
\stopSymVpad


\startSymVpad
\externalfigure[sym:vpad:frontBrush:left]
\SymVpad
\textDescrHead{Schwimmstellung}(grün am unteren Rand)
Um die Entlastung auszuschalten, halten Sie den Joystick für etwa 2\,s nach vorn gedrückt; der Besen liegt nun mit seinem ganzen Eigengewicht auf dem Boden auf. Die Rotationsgeschwindigkeit der Besen ist bei 70\hairspace\% der V\low{max} (Beispiel).

{\md Rotationsrichtung:} {\lt Am oberen Rand wird die Rotationsrichtung angezeigt (schwarzer Pfeil auf gelbem Hintergrund).}
\stopSymVpad

\stopsection

\stop % local group for temporary redefinition of \textDescrHead

\page [yes]


\startsection[title={Einstellen der Bildschirmhelligkeit},
              reference={sec:vpad:brightness}]

Der Bildschirm des \Vpad\ kann in zwei vorkonfigurierten Helligkeitsstufen
betrieben werden: Modus \aW{Tag}~– \textSymb{vpadSunglasses}, normale
Helligkeit~– und Modus \aW{Nacht}~– \textSymb{vpadMoon}, reduzierte Helligkeit.
Mit der Taste \textSymb{vpadColor} können Sie auf verschiedene
Parameter zugreifen.

Um die vorkonfigurierten Helligkeitsstufen abzuändern, gehen Sie so vor:

\startSteps
\item Drücken Sie auf die Mitte des berührungsempfindlichen Bildschirms (Touchscreen), um durch die Menüleiste am unteren Bildschirmrand zu scrollen.
\item Drücken Sie auf das Symbol \textSymb{vpadSunglasses} bzw.
\textSymb{vpadMoon}, um den Modus auszuwählen, den Sie abändern möchten.
\item Drücken Sie \textSymb{vpadColor}, um die Parameter anzuzeigen.
\item Markieren Sie mithilfe der
Pfeilsymbole~\textSymb{vpadUp}\textSymb{vpadDown} den Parameter, den Sie ändern
möchten, und wählen Sie ihn mit~\textSymb{vpadSelect} aus.
\item Ändern Sie den Wert mithilfe der Symbole
\textSymb{vpadMinus}\textSymb{vpadPlus}. Vorsicht, reduzieren Sie die Helligkeit
nicht so stark (\VpadOp{162} -255), dass Sie nichts mehr auf dem Bildschirm
erkennen können!
\stopSteps
\blank [1*big]

\start
\setupcombinations[width=\textwidth]
\startcombination [3*1]
{\setups[VpadFramedFigureHome]% \VpadFramedFigureK pour bande noire
\VpadScreenConfig{
\VpadFoot{\VpadPictures{vpadOneTwoThree}{vpadServiceInfo}{vpadSunglasses}{vpadColor}}}%
\framed{\null}}{Drücken Sie mittig auf den Touchscreen}
{\setups[VpadFramedFigure]
\VpadScreenConfig{
\VpadFoot{\VpadPictures{vpadReturn}{vpadUp}{vpadDown}{vpadSelect}}}%
\framed{\bTABLE
\bTR\bTD \VpadOp{160} \eTD\eTR
\bTR\bTD [backgroundcolor=black,color=TableWhite] \VpadOp{162}\hfill 15 \eTD\eTR
\bTR\bTD \VpadOp{163}\hfill 180 \eTD\eTR
\bTR\bTD \VpadOp{164}\hfill 55 \eTD\eTR
\bTR\bTD \VpadOp{165}\hfill 3 \eTD\eTR
\eTABLE}}{Auswählen mit \textSymb{vpadSelect}}
{\setups[VpadFramedFigure]% \VpadFramedFigureK pour bande noire
\VpadScreenConfig{
\VpadFoot{\VpadPictures{vpadReturn}{vpadMinus}{vpadPlus}{vpadNull}}}%
\framed[backgroundscreen=.9]{\bTABLE
\bTR\bTD \VpadOp{160} \eTD\eTR
\bTR\bTD \VpadOp{162}\hfill -80 \eTD\eTR
\bTR\bTD \VpadOp{163}\hfill 180 \eTD\eTR
\bTR\bTD \VpadOp{164}\hfill 55 \eTD\eTR
\bTR\bTD \VpadOp{165}\hfill 3 \eTD\eTR
\eTABLE}}{Wert ändern mit \textSymb{vpadMinus}\textSymb{vpadPlus}}
\stopcombination
\stop
\blank [1*big]

\startSteps [continue]
\item Bestätigen Sie den Wert mit \textSymb{vpadReturn}.
\item Drücken Sie nochmals das Symbol \textSymb{vpadReturn}, um zum
Hauptbildschirm zurückzugelangen.
\stopSteps

\stopsection

\page [yes]


\startsection[title={Betriebsstunden- und Kilometerzähler},
							reference={vpad:compteurs}]

Die Software des \Vpad\ verfügt über drei verschiedene Messperioden~– \aW{Tag},
\aW{Saison}, \aW{Gesamt}~–, in denen verschiedene Zähler laufen können, wie
\aW{Zurückgelegte Strecke}, \aW{Betriebsstunden} (Motor oder Bürste),
\aW{Arbeitszeit} (pro Fahrer).

Um die Zähler abzulesen oder sie zurückzusetzen, gehen Sie so vor:

\startSteps
\item Drücken Sie auf die Mitte des Touchscreens, um
durch die Menüleiste zu scrollen.
\item Drücken Sie auf das Symbol \textSymb{vpadOneTwoThree}, um den Tageszähler
anzuzeigen.
\item Mithilfe der Zurück-|/|Vor||Symbole~\textSymb{vpadBW}\textSymb{vpadFW}
können Sie zum Gesamt- oder Saisonzähler wechseln.
\item Drücken Sie \textSymb{vpadTrash}, um den angezeigten Zähler
zurückzusetzen.
\item In einem Dialogfenster werden Sie aufgefordert, das Zurücksetzen zu
bestätigen.
\stopSteps
\blank [1*big]

\start
\setupcombinations[width=\textwidth]
\startcombination [3*1]
{\setups[VpadFramedFigure]% \VpadFramedFigureK pour bande noire
\VpadScreenConfig{
\VpadFoot{\VpadPictures{vpadOneTwoThree}{vpadServiceInfo}{vpadSunglasses}{vpadColor}}}%
\framed{\bTABLE
\bTR\bTD \VpadOp{120} \eTD\eTR
\bTR\bTD \VpadOp{123}\hfill 87.4\,h \eTD\eTR
\bTR\bTD \VpadOp{125}\hfill 62.0\,h \eTD\eTR
\bTR\bTD \VpadOp{126}\hfill 240.2\,km \eTD\eTR
\bTR\bTD \VpadOp{124}\hfill 901.9\,km \eTD\eTR
\bTR\bTD \VpadOp{127}\hfill 2,1\,l/h \eTD\eTR
\eTABLE}}{Drücken Sie das Symbol~\textSymb{vpadOneTwoThree}, anschließend~\textSymb{vpadBW} oder~\textSymb{vpadFW}}
{\setups[VpadFramedFigure]
\VpadScreenConfig{
\VpadFoot{\VpadPictures{vpadReturn}{vpadBW}{vpadFW}{vpadTrash}}}%
\framed{\bTABLE
\bTR\bTD \VpadOp{121} \eTD\eTR
\bTR\bTD \VpadOp{123}\hfill 522.0\,h \eTD\eTR
\bTR\bTD \VpadOp{125}\hfill 662.8\,h \eTD\eTR
\bTR\bTD \VpadOp{126}\hfill 1605.5\,km \eTD\eTR
\bTR\bTD \VpadOp{124}\hfill 2608.4\,km \eTD\eTR
\bTR\bTD \VpadOp{127}\hfill 2,0\,l/h \eTD\eTR
\eTABLE}}{Setzen Sie den Zähler mit \textSymb{vpadTrash} zurück}
{\setups[VpadFramedFigure]% \VpadFramedFigureK pour bande noire
\VpadScreenConfig{
\VpadFoot{\VpadPictures{vpadReturn}{vpadTrash}{vpadNull}{vpadNull}}}%
\framed{\bTABLE
\bTR\bTD \VpadOp{121} \eTD\eTR
\bTR\bTD \null \eTD\eTR
\bTR\bTD \VpadOp{136} \eTD\eTR
\bTR\bTD \null \eTD\eTR
\bTR\bTD \VpadOp{137} \eTD\eTR
\eTABLE}}{Bestätigen Sie mit \textSymb{vpadTrash}}
\stopcombination
\stop
\blank [1*big]

\startSteps [continue]
\item Geben Sie, wenn nötig, das Passwort ein, und bestätigen Sie dann das Zurücksetzen mithilfe des Symbols \textSymb{vpadTrash}.
\item Drücken Sie das Symbol \textSymb{vpadReturn}, um zum Hauptbildschirm zurückzugelangen.
\stopSteps

\stopsection

\page [yes]

\startsection[title={Wartungsintervalle},
							reference={vpad:maintenance}]

Der Wartungsplan der \sdeux\ kennt zwei Grundarten der Wartung: die reguläre Wartung und den großen Service (durch eine mit dem \boschung||Kundendienst vereinbarte Fachwerkstatt).

Um die Zähler abzulesen oder zurückzusetzen, gehen Sie so vor:
\startSteps
\item Drücken Sie auf die Mitte des Touchscreens, um
durch die Menüleiste zu scrollen.
\item Drücken Sie das Symbol \textSymb{vpadServiceInfo}, um die
Wartungsintervalle anzuzeigen.
\item Wechseln Sie mithilfe der
Pfeilsymbole~\textSymb{vpadUp}\textSymb{vpadDown} zum gewünschten Intervall.
\item Drücken Sie das Symbol~\textSymb{vpadTrash}, um ein Intervall
zurückzusetzen. Geben Sie das Passwort mithilfe
von~\textSymb{vpadPlus}\textSymb{vpadMinus} ein und bestätigen Sie
mit~\textSymb{vpadSelect}).
\item In einem Dialogfenster werden Sie aufgefordert, das Zurücksetzen zu
bestätigen.
\stopSteps
\blank [1*big]

\start
\setupcombinations[width=\textwidth]
\startcombination [3*1]
{\setups[VpadFramedFigure]% \VpadFramedFigureK pour bande noire
\VpadScreenConfig{
\VpadFoot{\VpadPictures{vpadReturn}{vpadNull}{vpadNull}{vpadTrash}}}%
\framed{\bTABLE
\bTR\bTD[nc=2] \VpadOp{190} \eTD\eTR
\bTR\bTD \VpadOp{191}\eTD\bTD \VpadOp{195}\hfill 600\,h \eTD\eTR % [backgroundcolor=black,color=TableWhite]
\bTR\bTD \VpadOp{192}\eTD\bTD \VpadOp{195}\hfill 600\,h \eTD\eTR
\bTR\bTD \VpadOp{193}\eTD\bTD \VpadOp{195}\hfill 2400\,h \eTD\eTR
\eTABLE}}{Drücken Sie das Symbol~\textSymb{vpadTrash}, um ein Intervall
zurückzusetzen}
{\setups[VpadFramedFigure]
\VpadScreenConfig{
\VpadFoot{\VpadPictures{vpadReturn}{vpadMinus}{vpadPlus}{vpadSelect}}}%
\framed{\bTABLE
\bTR\bTD \VpadOp{190} \eTD\eTR
\bTR\bTD \hfill 2014-03-31 \eTD\eTR
\bTR\bTD \null \eTD\eTR
\bTR\bTD \null \eTD\eTR
\bTR\bTD \null \eTD\eTR
\bTR\bTD \null \eTD\eTR
\bTR\bTD \VpadOp{002}\hfill 0000 \eTD\eTR
\eTABLE}}{Geben Sie das Passwort ein (Zahlencode)}
{\setups[VpadFramedFigure]% \VpadFramedFigureK pour bande noire
\VpadScreenConfig{
\VpadFoot{\VpadPictures{vpadReturn}{vpadUp}{vpadDown}{vpadSelect}}}%
\framed{\bTABLE
\bTR\bTD \VpadOp{190} \eTD\eTR
\bTR\bTD[backgroundcolor=black,color=TableWhite] \VpadOp{041}\eTD\eTR % [backgroundcolor=black,color=TableWhite]
\bTR\bTD \VpadOp{042} \eTD\eTR
\bTR\bTD \VpadOp{043} \eTD\eTR
\eTABLE}}{Wählen Sie aus und bestätigen Sie mit~\textSymb{vpadSelect}}
\stopcombination
\stop
\blank [1*big]

\startSteps [continue]
\item Bestätigen Sie das Zurücksetzen mithilfe des Symbols~\textSymb{vpadSelect}.
\item Drücken Sie das Symbol \textSymb{vpadReturn}, um zum Hauptbildschirm zurückzugelangen.
\stopSteps

\stopsection

\page [yes]


\startsection[title={Fehlermanagement über das Vpad},
							reference={vpad:error}]


Das \Vpad\ zeigt Fehler an\index{Vpad+Fehlermeldungen}, welche von den elektronischen Steuersystemen diagnostiziert und vom  CAN||Bus übermittelt wurden.
Wenn ein minderschwerer Fehler registriert wird, leuchtet des Symbol~\textSymb{VpadTClear} (rot).
Wenn es sich um einen Fehler hoher Priorität handelt, leuchtet des Symbol~\textSymb{VpadTClear} und zusätzlich ertönt ein Alarmton.
Um den Alarm zu beenden, muss die Fehlermeldung
quittiert (als \aW{zur Kenntnis genommen} bestätigt) werden.

Um Fehlermeldungen zu lesen und zu quittieren, gehen Sie so vor:

\startSteps
\item Drücken Sie das Symbol~\textSymb{vpadClear} auf dem Bildschirm des \Vpad.
\item Drücken Sie das Symbol~\textSymb{vpadClear}, um die ausgewählte Meldung zu quittieren.
\item Neben der quittierten Meldung erscheint nun ein \aW{\#}||Symbol, welches
die Meldung als \aW{zur Kenntnis genommen} kennzeichnet, und die Markierung
springt zur nächsten Meldung (soweit vorhanden).
\item Nachdem alle Meldungen quittiert wurden, kehrt die Anzeige zum
Hauptbildschirm zurück.
\stopSteps
\blank [1*big]

\start
\setupcombinations[width=\textwidth]
\startcombination [3*1]
{\setups[VpadFramedFigure]% \VpadFramedFigureK pour bande noire
\VpadScreenConfig{
\VpadFoot{\VpadPictures{vpadReturn}{vpadUp}{vpadDown}{vpadSelect}}}%
\framed{\bTABLE
\bTR\bTD \VpadEr{000} \eTD\eTR
\bTR\bTD [backgroundcolor=black,color=TableWhite] \VpadEr{851a} \eTD\eTR
\bTR\bTD \VpadEr{902} \eTD\eTR
\eTABLE}}{Anzeige der Meldungen}
{\setups[VpadFramedFigure]
\VpadScreenConfig{
\VpadFoot{\VpadPictures{vpadReturn}{vpadUp}{vpadDown}{vpadSelect}}}%
\framed{\bTABLE
\bTR\bTD \VpadEr{000} \eTD\eTR
\bTR\bTD [backgroundcolor=black,color=TableWhite] \VpadEr{851} \eTD\eTR
\bTR\bTD \VpadEr{902} \eTD\eTR
\eTABLE}}{Quittieren Sie mit~\textSymb{vpadClear}}
{\setups[VpadFramedFigureHome]% \VpadFramedFigureK pour bande noire
\VpadScreenConfig{
\VpadFoot{\VpadPictures{vpadClear}{vpadBeacon}{vpadBeam}{vpadEngine}}}%
\framed{\null}}{Zurück zum Hauptbildschirm}
\stopcombination
\stop
\blank [1*big]

\startSteps [continue]
\item Um die Meldungen erneut anzuzeigen, drücken Sie das Symbol~\textSymb{vpadClear}. Fehlermeldungen werden erst dann vom \Vpad\
gelöscht, wenn die Ursache des Problems beseitigt wurde.
\stopSteps


\subsection{Die häufigsten Fehlermeldungen (Mit Störungssuche)}

\subsubsubject{\VpadEr{604}} % {\#\ 604	Pression huile moteur basse}

+ \textSymb{vpadTEnginOilPressure}~– Schalten Sie den Motor unverzüglich aus. Überprüfen Sie den Ölstand, verständigen Sie die Werkstatt.


\subsubsubject{\VpadEr{609}} % {\#\ 609	Température eau refroidissement moteur haute}

+ \textSymb{vpadSyWaterTemp}~– Brechen Sie Ihre Arbeit ab. Lassen Sie den Motor ohne Last weiterlaufen und beobachten Sie die Temperaturentwicklung:

Wenn die Temperatur sinkt, überprüfen Sie die Füllstände von Kühlflüssigkeit, Motoröl und Hydraulikflüssigkeit sowie den Zustand des Kühlers.
Wenn die Füllstände und der Kühler in Ordnung sind, fahren Sie zur weiteren Fehlerdiagnose vorsichtig in die Werkstatt.

\subsubsubject{\VpadEr{610}} % {\#\ 610	Température eau refroidissement moteur trop haute}

+ \textSymb{vpadSyWaterTemp}~– Brechen Sie Ihre Arbeit ab. Überprüfen Sie die Füllstände von Kühlflüssigkeit und Motoröl, verständigen Sie unverzüglich die Werkstatt.


\subsubsubject{\VpadEr{650}} % {\#\ 650	Se rendre à un garage}

+ \textSymb{vpadWarningService}~– Verständigen Sie umgehend Ihre Werkstatt.
% \VpadEr{651} % {\#\ 651	Moteur en mode urgence}


\subsubsubject{\VpadEr{652}} % {\#\ 652	Inspection véhicule}
% \VpadEr{653} % {\#\ 653	Grand service moteur}

+ \textSymb{vpadWarningService}~– Die nächste reguläre Wartung ist fällig. Konsultieren Sie den Wartungsplan und machen Sie einen Termin mit Ihrer Werkstatt aus.


\subsubsubject{\VpadEr{700}} % {\#\ 700	Température d'huile hydraulique}

+ \textSymb{vpadSyWaterTemp}~– Brechen Sie Ihre Arbeit ab. Lassen Sie den Motor ohne Last weiterlaufen und beobachten Sie die Temperaturentwicklung:

Wenn die Temperatur sinkt, überprüfen Sie die Füllstände von Kühlflüssigkeit, Motoröl und Hydraulikflüssigkeit sowie den Zustand des Kühlers.
Wenn die Füllstände und der Kühler in Ordnung sind, fahren Sie zur weiteren Fehlerdiagnose vorsichtig in die Werkstatt.


\subsubsubject{\VpadEr{702}} % {\#\ 702	Filtre d'huile hydraulique}

+ \textSymb{vpadWarningFilter}~– Der Hydraulik||Rücklauf- und|/|oder Ansaugfilter ist zugesetzt. Ersetzen Sie umgehend das Filterelement.
% \VpadEr{703} % {\#\ 703	Vidange d'huile hydraulique}


\subsubsubject{\VpadEr{800}} % {\#\ 800	Interrupteur d'urgence actionné}

+ \textSymb{vpadTClear}~– Sie haben den Not||Aus||Schalter betätigt. Schalten Sie die Zündung aus und starten Sie den Motor neu, um die Meldung zu löschen.


\subsubsubject{\VpadEr{801}} % {\#\ 905	Frein à main actionné}

Der Schmutzbehälter ist angehoben oder nicht komplett gesenkt. Die Geschwindigkeit des Fahrzeugs ist auf 5\,km/h begrenzt, solange der Schmutzbehälter nicht gesenkt ist.

\subsubsubject{\VpadEr{851}} % {\#\ 851	Filtre à air}

+ \textSymb{vpadWarningFilter}~– Der Luftfilter ist zugesetzt. Ersetzen Sie umgehend das Filterelement.


\subsubsubject{\VpadEr{902}} % {\#\ 902	Pression de freinage}

+ \textSymb{vpadTBrakeError}~– Der Bremsdruck ist nicht ausreichend. Brechen Sie Ihre Arbeit ab und verständigen Sie umgehend die Werkstatt.
% \VpadEr{904} % {\#\ 904	Interrupteur de direction d'avancement}


\subsubsubject{\VpadEr{905}} % {\#\ 905	Frein à main actionné}

+ \textSymb{vpadTBrakePark}~– Die Feststellbremse ist nicht vollständig gelöst. Die Geschwindigkeit des Fahrzeugs ist auf 5\,km/h begrenzt, solange die Feststellbremse nicht gelöst ist.


\stopsection

\stopchapter

\stopcomponent













