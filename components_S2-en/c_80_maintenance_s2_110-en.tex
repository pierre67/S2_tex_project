\startcomponent c_80_maintenance_s2_095-en
\product prd_ba_s2_095-en

\startchapter [title={Maintenance and servicing},
							reference={chap:maintenance}]

\setups[pagestyle:marginless]


\startsection [title={General information}]


\subsection{Environmental protection}

\starttextbackground [FC]
\setupparagraphs [PictPar][1][width=2.45em,inner=\hfill]

\startPictPar
\Penvironment
\PictPar
\Boschung\ puts environmental protection\index{Environmental protection} into practice. We focus on the root causes and, when making business decisions, we take into account all the various ways that the production process and the product can impact on the environment. Our goals are to use resources sparingly and operate with due consideration for the natural environment which has to be protected for the benefit of mankind and nature. By adhering to certain rules when operating the vehicle you can also help to protect the environment. This also includes the adequate and proper handling of substances and materials during the course of vehicle maintenance (\eG\ disposing of chemicals and special waste).

How the vehicle is operated has a major impact on both fuel consumption and engine wear. We would therefore ask you to observe a number points:

\startitemize
\item Do not allow the engine to warm up at idling speed.
\item Switch off the engine during breaks in operation.
\item Regularly check fuel consumption.
\item {\em Have maintenance work carried out in line with the maintenance schedule and by a competent specialist workshop.}
\stopitemize
\stopPictPar
\stoptextbackground

\page [yes]


\subsection{Safety requirements}

\startSymList
\PHgeneric
\SymList
To\index{Maintenance+Safety requirements} prevent vehicle and assembly damage and accidents during maintenance work, it is essential to observe the following safety requirements. Please also note the general safety requirements (\about[safety:risques], \at{p.}[safety:risques] ff.).
\stopSymList

\startSymList
\PMgeneric
\SymList
\textDescrHead{Accident prevention}
Check\index{Accident prevention}  the condition of the vehicle after every maintenance or repair activity. In particular, ensure that all safety||relevant components, lights and signals are fully functional before driving on public roads.
\stopSymList

\start
\setupparagraphs [SymList][1][width=6em,inner=\hfill]
\startSymList\PHcrushing\PHfalling\SymList
\textDescrHead{Stabilizing the vehicle}
The vehicle must be secured against unintentional movement before any maintenance activity is undertaken: Move the gearshift to {\em neutral}, apply the parking brake and secure the vehicle with wheel blocks.
\stopSymList
\stop


\starttextbackground [CB]
\startPictPar \PHpoison \PictPar
\textDescrHead{Starting the engine}
If\index{Hazard+Poisoning} you need to start the engine in a poorly ventilated location, allow it to run for only as long as\index{Hazard+Exhaust fumes} necessary in order to prevent carbon monoxide poisoning.
\stopPictPar

\startitemize
\item Start the engine only when the battery is correctly connected.
\item Never disconnect the battery while the engine is running.
\item Do not start the engine using a jump||start device.
If\index{Battery+Charger} the battery is to be charged using a quick||charger, it should be disconnected from the vehicle first. Observe the user instructions for the quick||charger.
\stopitemize
\stoptextbackground

\page [yes]


\subsubsection{Protecting electronic components}

\startitemize
\item Before\index{Arc welding} starting any welding work, disconnect the battery cable from the battery and connect the {\em positive} and {\em ground} cables together.
\item Only connect or disconnect\index{Electronics} electronic control units when they are not live.
\item Incorrect\index{Control unit} polarity in the power supply (\eG\ because the batteries are not connected correctly) can destroy electronic components and devices.
\item Electronic components/devices must be removed when\index{Ambient temperature+Extreme} ambient temperatures exceed 80\,°C (\eG\ in a drying chamber).
\stopitemize


\subsubsection{Diagnostics and measurements}

\startitemize
\item Only use {\em suitable} test cables (\eG\ the original device cables) for measurement and diagnostic tasks.
\item Cellphones\index{Cellphone} and similar radio devices can affect the function of the vehicle, the diagnostic device and therefore operational safety.
\stopitemize


%%%%%%%%%%%%%%%%%%%%%%%%%%%%%%%%%%%%%%%%%%%%%%%%%%%%%%%%%%%%%%%%%%%%%%%%%%%%%%%%%%%%%%%%%
\subsubsection{Qualification of personnel}

\starttextbackground[CB]
\startPictPar
\PHgeneric
\PictPar
\textDescrHead{Accident hazard}
If\index{Qualification+Service personnel} maintenance and repair work is not properly executed, the functionality and safety of the vehicle can be compromised. This brings an increased risk of accident and injury.

Employ\index{Qualification+Workshop} a qualified specialist workshop that possesses the skills and equipment needed to execute the maintenance and repair work professionally.

In case of doubt, contact \Boschung\ Customer Services.
\stopPictPar
\stoptextbackground

% \page [yes]

The \ProductId must be operated, serviced and repaired only by qualified personnel and personnel trained by \Boschung.

\Boschung\ Customer Services issues authorization for operating, servicing and repair work.


\subsubsection{Alterations and conversions}

\starttextbackground[CB]
\startPictPar
\PHgeneric
\PictPar
\textDescrHead{Accident hazard}
Any\index{Changes to the vehicle} alterations that you undertake independently on the vehicle could impair the functionality and operational safety of the \ProductId and thus result in an unforeseeable risk of accident and injury.
\stopPictPar

\startPictPar
\PMwarranty
\PictPar
\Boschung\ assumes no warranty or goodwill\index{Warranty+Conditions} for damage caused by unauthorized tampering or modifications on the \ProductId or an assembly.
\stopPictPar
\stoptextbackground

\stopsection


\startsection [title={Operating fluids and lubricants}, reference={sec:liquids}]

\subsection{Correct handling}

\starttextbackground[CB]
\startPictPar
\PHpoison
\PictPar
\textDescrHead{\index{Hazard+Poisoning}Risk of injury or poisoning}
Operating\index{Fuel} fluids and lubricants\index{Lubricants} can\index{Fuel+Safety} cause injuries or poisoning when they come into contact with skin or are swallowed.
Always observe the legal requirements when working with, storing or disposing of these substances.
\stopPictPar
\stoptextbackground

\starttextbackground [FC]
\startPictPar
\PMproteyes\par
\PMprothands
\PictPar
Always wear appropriate protective clothing and respiratory protection when handling operating fluids and lubricants. Try not to inhale fumes. Avoid all contact with skin, eyes or clothing. Wash any areas of skin that come into contact with operating fluids with water and soap immediately. If operating fluids come into contact with the eyes, rinse out thoroughly with plenty of clean water and seek medical attention. If operating fluids are swallowed, seek medical attention immediately!
\stopPictPar
\stoptextbackground

\startSymList
\PPchildren
\SymList
Keep all operating fluids out of reach of children.
\stopSymList

\startSymList
\PPfire
\SymList
\textDescrHead{Fire hazard}
Due to their high flammability,\index{Hazard+Fire} the handling of operating fluids represents a significant fire hazard. Smoking, fire\index{No smoking} and naked flames are strictly forbidden when handling operating fluids.
\stopSymList

\starttextbackground [FC]
\startPictPar
\PMgeneric
\PictPar
Only lubricants suitable for the components used in the \ProductId are permitted to be used. Consequently, you should only use products that have been expressly approved by \Boschung. These can be found in the list of operating fluids on \atpage[page:table:liquids]. Additives\index{Additives} for lubricants are not required. The use of additives can invalidate your warranty\index{Warranty+Conditions}. For further information, contact \Boschung\ Customer Services.
\stopPictPar
\stoptextbackground


\starttextbackground [FC]
\startPictPar
\Penvironment
\PictPar
\textDescrHead{Environmental protection}
When disposing\index{Lubricants+Disposal} of operating fluids and lubricants\index{Environmental protection} or objects that are contaminated with these substances (\eG\ filters, cloths),
observe\index{Operating fluids+Disposal} environmental protection regulations.
\stopPictPar
\stoptextbackground


\page [yes]

\setups [pagestyle:normal]


\subsection[sec:liqquantities]{Specifications and fill volumes}

All\index{Operating fluids+Fill volumes}\index{Lubricants+Fill volumes}\index{Fill volumes+Operating fluids and lubricants}\index{Specifications+Operating fluids and lubricants} fill volumes stated in the table below are guideline values. After changing an operating fluid/lubricant, always check the fill level and top up or reduce the fill volume if necessary.
% \blank[small]

\placetable[margin][tab:glyco]{Anti||freeze (\index{Anti||freeze}engine)}
{\noteF\startframedcontent[FrTabulate]
\starttabulate[|Bp|r|r|]
\NC Anti||freeze protection \NC  \NC \NC\NR
\NC \emspace to {[}°C{]}\NC \bf \textminus 25 \NC \bf \textminus 40 \NC\NR
\NC Distil. water [Vol. \%] \NC 60 \NC 40 \NC\NR
\NC Anti||freeze [Vol. \%] \NC 40 \NC {\em max.} 60 \NC\NR
\stoptabulate\stopframedcontent\endgraf
Caution: If the volume of anti||freeze exceeds 60\hairspace\percent, protection against freezing is {\em reduced} and cooling performance is impaired!}

\placefig[margin][fig:hydrgauge]{\select{caption}{Level indicator for hydraulic fluid (left||hand side of vehicle)}{Level indicator for hydraulic fluid}}
{\externalfigure[main:hy:level_temp]
\noteF The fill level in the hydraulic tank can be read off from the level indicator and should be checked {\em daily}.}

\vskip -8pt
\start
\define [1] \TableSmallSymb {\externalfigure[#1][height=4ex]}
\define\UC\emptY
\pagereference[page:table:liquids]


\setupTABLE [frame=off,style={\ssx\setupinterlinespace[line=.86\lH]},background=color,
option=stretch,
split=repeat]
\setupTABLE [r] [each] [topframe=on,
framecolor=TableWhite,
% rulethickness=.8pt
]

\setupTABLE [c] [odd] [backgroundcolor=TableMiddle]
\setupTABLE [c] [even] [backgroundcolor=TableLight]
\setupTABLE [c] [1] [width=30mm]
\setupTABLE [c] [2] [width=25mm]
\setupTABLE [c] [4] [width=25mm]
\setupTABLE [c] [last] [width=13mm]
\setupTABLE [r] [first] [topframe=off,style={\bfx\setupinterlinespace [line=.95\lH]},
% backgroundcolor=TableDark
]
\setupTABLE [r] [2] [framecolor=black]

\bTABLE

\bTABLEhead
 \bTR
 \bTC Group \eTC
 \bTC Category \eTC
 \bTC Classification \eTC
 \bTC Product\note[Produkt] \eTC
 \bTC Volume \eTC
 \eTR
\eTABLEhead
%
\bTABLEbody
 \bTR \bTD Engine \eTD
 \bTD Engine oil\eTD
 \bTD \liqC{SAE 5W-30}; \liqC{VW\,507.00}\eTD
 \bTD Total Quartz INEO Long Life \eTD
 \bTD 4.3\,l\eTD
 \eTR
 \bTR \bTD Hydraulic circuit \eTD
  \bTD Hydraulic oil \eTD
  \bTD \liqC{ISO VG 46} \eTD
  \bTD  Total Equiviz ZS 46 (tank approx. 40\,l) \eTD
  \bTD approx. 50\,l\eTD
  \eTR
 \bTR \bTD Hydraulic circuit \break (\aW{Bio} option)\eTD
  \bTD Hydraulic oil \eTD
  \bTD \liqC{ISO VG 46} \eTD
  \bTD Total Biohydran TMP SE 46 \eTD
  \bTD approx. 50\,l\eTD
  \eTR
 \bTR \bTD Solenoid valves: Coil cores \eTD
  \bTD Lubricant\eTD
  \bTD Copper grease \eTD
  \bTD \emptY\eTD
  \bTD a.r.\note[Bedarf] \eTD
  \eTR
 \bTR \bTD Various: Locks, door mechanics, brake pedal \eTD
  \bTD Lubricant\eTD
  \bTD Universal spray\eTD
  \bTD \emptY\eTD
  \bTD a.r.\note[Bedarf] \eTD
  \eTR
 \bTR \bTD Central lubrication system \eTD
  \bTD Universal bearing grease\eTD
  \bTD \liqC{nlgi}~2\eTD
  \bTD Total Multis EP~2\eTD
  \bTD a.r.\note[Bedarf] \eTD
  \eTR
 \bTR \bTD Cooling system \eTD
  \bTD Anti||freeze/corrosion inhibitor\eTD
  \bTD TL VW 774 F/G; max. 60\hairspace\% vol.\eTD
  \bTD G12+|/|G12++ (rose|/|violet)\eTD
  \bTD approx. 14\,l \eTD
  \eTR
 \bTR \bTD High||pressure water pump \eTD
  \bTD Engine oil\eTD
  \bTD \liqC{SAE 10W-40}; \liqC{api cf~– acea e6}\eTD
  \bTD Total Rubia TIR 8900 \eTD
  \bTD	0.29\,l \eTD
  \eTR
 \bTR \bTD Air||conditioning system \eTD
  \bTD Refrigerant\eTD
  \bTD + 20\,ml POE oil\eTD
  \bTD R\,134a\eTD
  \bTD	700\,g\eTD
  \eTR
 \bTR \bTD Windshield washer system \eTD
	\bTD [nc=2] Water and windshield wash concentrate, \aW{S}~summer, \aW{W}~winter; note mixing ratio \eTD
	\bTD Retail \eTD
  \bTD a.r.\note[Bedarf] \eTD
  \eTR
\eTABLEbody

\eTABLE
\stop

\footnotetext[Bedarf]{{\it a.r.} As required, in line with the relevant directions}
\footnotetext[Produkt]{Products used by \Boschung. Other products that meet the specifications can also be used.}

\stopsection

\page [yes]

\setups [pagestyle:marginless]


\startsection [title={Diesel engine maintenance},
reference={sec:workshop:vw}]


\subsection [sSec:vw:diagTool]{On|-|board diagnostic system}

The\startregister[index][reg:main:vw]{Maintenance+Diesel engine} engine control unit (J623) is equipped with a fault memory.
If malfunctions occur in the sensors or components that are being monitored, they are stored in the fault memory with an indication of the type of fault.

After evaluating the information, the\index{Diesel engine+Diagnostics} engine control unit differentiates between the various types of fault and stores them until they are deleted from the fault memory.

Faults that only occur {\em sporadically}, are displayed with the suffix \aW{SP}. Sporadic faults can be caused by \eG\ a loose contact or a temporary open circuit in wiring. If a sporadic fault does not reoccur over the course of 50 engine starts, it is deleted from the fault memory.

If faults that influence engine operation are detected, the \aW{Engine diagnosis}~\textSymb{vpadWarningEngine1} monitoring icon lights up on the Vpad screen.

The stored faults can be read out using the vehicle diagnosis, testing and information system \aW{VAS 5051/B}.

Once the fault or faults have been rectified, the fault memory must be cleared.


\subsubsection[sSec:vw:diagTool:connect]{Starting the vehicle diagnostics system}

\starttextbackground [FC]
\startPictPar
\PMgeneric
\PictPar
Detailed information on the VAS 5051/B vehicle diagnosis system can be found in the user manual for the system.

You can also use other compatible diagnostics systems, \eG\ \aW{DiagRA}.
\stopPictPar
\stoptextbackground

\page [yes]


\subsubsubsubject{Requirements}

\startitemize
\item The fuses must be OK.
\item Battery voltage must be more than 11.5\,V.
\item All electrical consumers must be switched off.
\item The earth connection must be OK.
\stopitemize


\subsubsubsubject{Procedure}

\startSteps
\item Insert the connector of the VAS 5051B/1 diagnosis lead into the diagnosis interface.
\item Depending on the function, either switch on the ignition or start the engine.
\stopSteps

\subsubsubsubject{Selecting the operating mode}

\startSteps [continue]
\item Press the button on the display for \aW{Vehicle self-diagnosis}.
\stopSteps


\subsubsubsubject{Selecting the vehicle system}

\startSteps [continue]
\item Press the button on the display for \aW{01-Engine electronics}.
\stopSteps

 The display shows the control unit identification and the coding for the engine control unit.

If the codes do not match, the control unit coding will have to be checked.


\subsubsubsubject{Selecting the diagnosis function}

All of the available diagnosis functions are shown in the display.

\startSteps [continue]
\item Press the button on the display for the desired function.
\stopSteps



\subsection [sSec:vw:faultMemory]{Fault memory}


\subsubsection{Reading out the fault memory}

\subsubsubject{Procedure}

\startSteps
\item Run the engine at idling speed.
\item Connect the VAS 5051/B (see \in{Section}[sSec:vw:diagTool:connect]) and select the engine control unit.
\item Select the diagnosis function \aW{004-Contents of fault memory}.
\item Select the diagnosis function \aW{004.01-Read fault memory}.
\stopSteps

{\sla Only if the engine does not start:}

\startitemize [2]
\item Switch on the ignition.
\item If no faults are stored in the engine control unit, the display shows \aW{0 faults detected}.
\item If faults are stored in the engine control unit, these are shown in the display in succession.
\item Close the diagnosis function.
\item Switch the ignition off.
\item Rectify any potential faults using the fault table (see service documentation) and then clear the fault memory.
\stopitemize

\starttextbackground [FC]
\startPictPar
\PMrtfm
\PictPar
If one of the faults cannot be deleted, please contact \boschung\ Customer Services.
\stopPictPar
\stoptextbackground


\subsubsubject{Static faults}

If one or more static faults are stored in the memory, please contact Boschung Customer Services to rectify the faults with the aid of \aW{guided fault finding}.


\subsubsubject{Sporadic faults}

If the fault memory contains only sporadic faults or notes, and no malfunctions have been identified in the electronic vehicle system, the fault memory can be cleared:

\startSteps [continue]
\item Press the \aW{continue} \inframed[strut=local]{>} button again to access the test schedule.
\item To exit guided fault finding, press the \aW{skip} and then \aW{close} buttons.
\stopSteps

All fault memories are then read out again.

A window appears, confirming that all sporadic faults have been deleted. The diagnosis protocol is automatically sent (online).

This completes the vehicle system test.


\subsubsection[sSec:vw:faultMemory:errase]{Clearing the fault memory}

\subsubsubject{Procedure}

{\sla Requirements:}

\startitemize [2]
\item All faults rectified and the causes of the faults resolved.
\stopitemize

{\sla Procedure:}

\starttextbackground [FC]
\startPictPar
\PMrtfm
\PictPar
Once the faults have been rectified, the fault memory must be read out again and then cleared:
\stopPictPar
\stoptextbackground

\startSteps
\item Run the engine at idling speed.
\item Connect the VAS 5051/B (see \in{Section}[sSec:vw:diagTool:connect]) and select the engine control unit.
\item Select the diagnosis function \aW{004-Read fault memory}.
\item Select the diagnosis function \aW{004.10-Erase fault memory}.
\stopSteps

\starttextbackground [FC]
\startPictPar
\PMrtfm
\PictPar
If the fault memory cannot be cleared, a fault is still present and must be rectified.
\stopPictPar
\stoptextbackground

\startSteps [continue]
\item Close the diagnosis function.
\item Switch the ignition off.
\stopSteps


\subsection [sSec:vw:lub] {Engine lubrication}

\subsubsection [ssSec:vw:oilLevel] {Checking engine oil fill level}

\starttextbackground [FC]
\startPictPar
\PMrtfm
\PictPar
The\index{Engine oil+Fill level} engine oil fill level must not under any circumstances exceed the \aW{Max.} mark. If it does, there is a risk that the\index{Fill level+Engine oil} catalytic converter could be damaged.
\stopPictPar
\stoptextbackground

\startSteps
\item Switch off the engine and wait for at least 3 minutes to allow the oil to flow back to the oil pan.
\item Pull out the dipstick and wipe it clean, then reinsert it up to the stop.
\item Pull the dipstick back out and assess the oil fill level:
\startfigtext[right][fig:vw:gauge]{Reading off the oil fill level}
{\externalfigure[VW_Oil_Gauge][width=50mm]}
\startitemize [A]
\item Maximum fill level: Do not add any more oil.
\item Fill level adequate: Oil {\em can} be added, up to the \aW{A} mark.
\item Fill level too low: Oil {\em must} be added to bring the fill level into the \aW{B} range.
\stopitemize
{\em If the fill level goes over the \aW{A} mark, there is a risk of damage to the catalytic converter.}
\stopfigtext
\stopSteps


\subsubsection [ssSec:vw:oilDraining] {Changing the engine oil}

\starttextbackground [FC]
\startPictPar
\PMrtfm
\PictPar
The engine oil filter on the S2 is mounted upright. As a result, the filter must be replaced {\em before} changing the oil. When the filter element is removed, a valve opens and the oil in the filter housing automatically flows into the crankcase.
\stopPictPar
\stoptextbackground

\startSteps
\item Place a suitably sized container\index{Diesel engine+Oil change} under the engine.
\item Unscrew the oil drain plug\index{Engine oil+Oil change} and allow the oil to drain out.
\stopSteps

\starttextbackground [FC]
\startPictPar
\PMrtfm
\PictPar
Ensure that the container is large enough to hold all the used oil from the engine.
You can find the relevant oil specifications and fill volume in \in{Section}[sec:liqquantities].

The oil drain plug is fitted with a non-detachable seal. As a result, the oil drain plug must be replaced every time the oil is changed.
\stopPictPar
\stoptextbackground

\startSteps [continue]
\item Screw a new oil drain plug with seal into place (\TorqueR~30\,Nm).
\item Add an engine oil that meets the relevant specifications (see \in{Section}[sec:liqquantities]).
\stopSteps


\subsubsection [ssSec:vw:oilFilter] {Replacing the engine oil filter}

\starttextbackground [FC]
\startPictPar
\PMrtfm
\PictPar
\startitemize [1]
\item Observe\index{Diesel engine+Oil filter} the applicable disposal and recycling laws and regulations.
\item Replace\index{Oil filter+Diesel engine} the filter {\em before} changing the oil (see \in{Section}[ssSec:vw:oilDraining]).
\item Lightly oil the seal of the new filter before installing it.
\stopitemize
\stopPictPar
\stoptextbackground

\startfigtext[right][fig:vw:oilFilter]{Oil filter}
{\externalfigure[VW_OilFilter_03][width=50mm]}
\startSteps
\item Use a suitable wrench to unscrew the lid~\Lone\ of the filter housing.
\item Clean the sealing surfaces on the lid and filter housing.
\item Replace filter element~\Lthree.
\item Replace O-rings~\Ltwo\ and~\Lfour.
\item Screw the lid of the filter housing back in place (\TorqueR~25\,Nm).
\stopSteps



%\subsubsubject{Données techniques}
%
%
%\hangDescr{Couple de serrage du couvercle:} \TorqueR~25\,Nm.
%
%\hangDescr{Huile moteur prescrite:} Selon tableau \atpage[sec:liqquantities].
%% NOTE: Redundant [tf]

\stopfigtext



\subsubsection [ssSec:vw:oilreplenish] {Topping up engine oil}

\starttextbackground [FC]
\startPictPar
\PMrtfm
\PictPar
\startitemize [1]
\item Clean\index{Engine oil} the fill opening with a cloth {\em before} removing the lid.
\item Only add\index{Diesel engine+Adding oil} oil that meets the stated specifications.
\item Add the oil gradually, in small quantities.
\item To prevent overfilling, wait a little after each time you add oil to allow the oil to flow into the engine oil pan and up to the mark on the dipstick (see \in{Section}[ssSec:vw:oilLevel]).
\stopitemize
\stopPictPar
\stoptextbackground

\startfigtext[right][fig:vw:oilFilter]{Topping up the oil}
{\externalfigure[s2_bouchonRemplissage][width=50mm]}
\startSteps
\item Pull the oil dipstick out by around 10\,cm so that air can escape while you are adding oil.
\item Open the filler opening.
\item Add the oil, taking care to observe the instructions set out above.
\item Carefully close the filler opening.
\item Start the engine.
\item Carry out a fill||level check (see \in{Section}[ssSec:vw:oilLevel]).
\stopSteps

\stopfigtext


\subsection [sSec:vw:fuel] {Fuel supply system}

\subsubsection [ssSec:vw:fuelFilter] {Changing the fuel filter}

\starttextbackground [FC]
\startPictPar
\PMrtfm
\PictPar
\startitemize [1]
\item Observe\index{Diesel engine+Fuel filter} the applicable disposal and recycling laws and regulations for special waste.
\item Do not remove the fuel lines from the upper part of the filter.
\item Do not pull the fixing points of the fuel lines, otherwise the upper part of the filter could be damaged.
\stopitemize
\stopPictPar
\stoptextbackground

\startfigtext[right][fig:vw:oilFilter]{Replace fuel filter}
{\externalfigure[s2_fuelFilter_location][width=50mm]}

{\sla Preparations:}

The\index{Fuel filter} fuel filter housing is mounted in front of the engine, on the right||hand side of the chassis.
Use a 10\,mm socket wrench and a 10\,mm ring wrench to remove the two fixing bolts.

\stopfigtext


\page [yes]

\setups [pagestyle:normal]

{\sla Procedure:}

\startLongsteps
\item Remove all the bolts on the upper part of the filter. Remove the upper part of the filter.
\stopLongsteps

\starttextbackground [FC]
\startPictPar
\PMrtfm
\PictPar
Lift the upper part off. If necessary, use an angled screwdriver at the assembly groove (\in{\LAa, Fig.}[fig:fuelfilter:detach]), turning the screwdriver to unseat the upper part.
\stopPictPar
\stoptextbackground

\placefig [margin] [fig:fuelfilter:detach]{Remove the fuel filter}
{\externalfigure[fuelfilter:detach]}

\placefig [margin] [fig:fuelfilter:explosion]{Fuel filter}
{\externalfigure[fuelfilter:explosion]}

\startLongsteps [continue]
\item Pull the filter element out of the lower part of the filter.
\item Remove the seal (\in{\Ltwo, Fig.}[fig:fuelfilter:explosion]) from the upper part of the filter.
\item Carefully clean the upper and lower parts of the filter.
\item Insert a new filter element in the lower part of the filter.
\item Wet a new seal (\in{\Ltwo, Fig.}[fig:fuelfilter:explosion]) with a little fuel and insert it into the upper part.
\item Place the upper part onto the lower part of the filter, ensure the two parts are correctly aligned and apply even pressure to ensure the upper part is in contact all the way around.
\item Use all the bolts to screw the upper and lower parts back together, tightening {\em by hand}. Then tighten all the bolts alternately, crosswise, with the specified tightening torque (\TorqueR~5\,Nm).
\stopLongsteps

% \subsubsubject{Données techniques}
%
% \hangDescr{Couple de serrage des vis de fixation du couvercle:} \TorqueR~5\,Nm.
%% NOTE: redundant [tf]

\startLongsteps [continue]
\item Switch on the ignition to vent the system, start the engine and run it at idling speed for 1 to 2 minutes.
\item Clear the fault memory as described on \atpage[sSec:vw:faultMemory:errase] .
\stopLongsteps


\subsection [sSec:vw:cooling] {Cooling system}

\starttextbackground [FC]
\startPictPar
\PMrtfm
\PictPar
\startitemize [1]
\item Use\index{Diesel engine+Cooling} only coolants that meet the stated specifications (see table, \atpage[sec:liqquantities]).
\item In\index{Coolant} order to ensure frost and corrosion protection, the coolant must be diluted only with distilled water and in line with the table below.
\item Never fill the coolant circuit with water, as doing so would impair frost and corrosion protection.
\stopitemize
\stopPictPar
\stoptextbackground


\subsubsection [sSec:vw:coolingLevel] {Coolant fill level}

\placefig [margin] [fig:coolant:level] {Coolant fill level}
{\externalfigure[coolant:level]}


\placefig [margin] [fig:refractometer] {Refractometer VW T 10007}
{\externalfigure[coolant:refractometer]}

\placefig [margin] [fig:antifreeze] {Checking the density of anti||freeze}
{\externalfigure[coolant:antifreeze]}


\startSteps
\item Raise the dirt hopper and place the safety props to the hopper’s lifting cylinders.
\item Determine\index{Fill level+Coolant} the coolant fill level in the expansion tank: It must be above the \aW{min} mark.
\stopSteps

\start
\define [1] \TableSmallSymb {\externalfigure[#1][height=4ex]}
\define\UC\emptY
\pagereference[page:table:liquids]


\setupTABLE [frame=off,style={\ssx\setupinterlinespace[line=.86\lH]},background=color,
option=stretch,
split=repeat]
\setupTABLE [r] [each] [topframe=on,
framecolor=TableWhite,
% rulethickness=.8pt
]

\setupTABLE [c] [odd] [backgroundcolor=TableMiddle]
\setupTABLE [c] [even] [backgroundcolor=TableLight]
\setupTABLE [r] [first] [topframe=off,style={\bfx\setupinterlinespace[line=.95\lH]},
% backgroundcolor=TableDark
]
\setupTABLE [r] [2][framecolor=black]

\bTABLE

\bTABLEhead
 \bTR
 \bTC Anti||freeze protection to … \eTC
 \bTC Concentr. G12\hairspace ++\eTC
 \bTC Vol. anti||freeze\eTC
 \bTC Vol. distilled water\eTC
 \eTR
\eTABLEhead

\bTABLEbody
 \bTR \bTD \textminus 25\,°C \eTD
 \bTD 40\hairspace\% \eTD
 \bTD 3.8\,l \eTD
 \bTD 4.2\,l \eTD
 \eTR
 \bTR \bTD \textminus 35\,°C \eTD
 \bTD 50\hairspace\% \eTD
 \bTD 4.0\,l \eTD
 \bTD 4.0\,l \eTD
 \eTR
 \bTR \bTD \textminus 40\,°C \eTD
 \bTD 60\hairspace\% \eTD
 \bTD 4.2\,l \eTD
 \bTD 3.8\,l \eTD
 \eTR
\eTABLEbody

\eTABLE
\stop

\adaptlayout [height=+20pt]
\subsubsection [sSec:vw:coolingFreeze] {Coolant fill level}

Use a suitable refractometer\index{Anti||freeze density} to check the density of the anti||freeze (see \in{Fig.}[fig:refractometer]: VW T 10007).
Note scale 1: G12\hairspace ++ (see \in{Fig.}[fig:antifreeze]).

\page [yes]


\subsection [sSec:vw:airFilter] {Air intake}

The air filter is accessed via the rear maintenance hatch on the right||hand side of the vehicle (see \in{Fig.}[fig:airFilter]).

\placefig [margin] [fig:airFilter] {Engine air filter}
{\externalfigure[vw:air:filter]
\noteF
\startLeg
\item Locking clip
\item Lower part of housing
\item Vent opening
\item Pressure sensor
\stopLeg}


\subsubsubject{Deployment conditions}

Sweeper vehicles are often used in very dusty environments. Consequently, it is important to check and clean the air filter on a weekly basis. See also \about[table:scheduleweekly], \atpage[table:scheduleweekly]. The air filter must be replaced when necessary.


\subsubsubject{Automatic diagnostics}

The intake line incorporates a pressure sensor (\Lfour, \in{Fig.}[fig:airFilter]) that is used to identify charge air losses\footnote{Reduced air flow due to a drop in the filter’s air permeability.} in the filter.
If the air filter is clogged, the monitoring icon~\textSymb{vpadWarningFilter} lights up on the Vpad screen and the error message \aW{\VpadEr{851}} is registered.


\subsubsubject{Maintenance/replacement}

\startSteps
\item Pull locking clip~\Lone down (\in{Fig.}[fig:airFilter]).
\item Turn the lower part of the housing ~\Ltwo counterclockwise and remove it.
\item Remove the filter element and check it. If necessary, replace it.
\item Clean the inside of the filter and refit the air filter by following the above sequence in reverse.
\stopSteps

\page [yes]


\subsection [sSec:vw:belt] {Poly V-belt}

The\index{Diesel engine+Poly V-belt} poly V-belt transmits the rotation of the crankshaft flywheel to the alternator and A/C compressor (optional equipment).
A\index{Poly V-belt} tensioning element in the final segment (between the alternator and crankshaft) keeps the belt under tension.

% \adaptlayout [+20pt]
\subsubsection [sSec:belt:change] {Replacing the poly V-belt}

\placefig [margin] [fig:belt:tool] {Locking pin VW T\,10060\,A}
{\externalfigure[vw:belt:tool]}

\placefig [margin] [fig:belt:overview] {Tensioning element}
{\externalfigure[vw:belt:overview]}

% \placefig [margin] [fig:belt:tens] {Point of insertion for locking pin}
\placefig [margin] [fig:belt:tens] {Insertion point for locking pin}
{\externalfigure[vw:belt:tens]}


\subsubsubject{With A/C compressor}

{\sla Special tool required:}

Locking pin \aW{VW T\,10060\,A} to hold the tensioning element.

\startSteps
\item Mark the direction of rotation on the poly V-belt.
\item Use a cranked ring wrench to swing the tensioning element clockwise (\in {Fig.}[fig:belt:overview]).
\item Align the holes (see arrows, \in {Fig.}[fig:belt:tens]) and use the locking pin to lock the tensioning element in place.
\item Remove the poly V-belt.
\stopSteps

Follow the same sequence in reverse to install the poly V-belt.

\starttextbackground [FC]
\startPictPar
\PMrtfm
\PictPar
\startitemize [1]
\item Ensure that the direction of rotation on the poly V-belt is correct.
\item Ensure that the belt is correctly seated in the pulleys.
\item Start the engine and check the belt is running correctly.
\stopitemize
\stopPictPar
\stoptextbackground


\testpage [8]
\subsubsubject{Without A/C compressor}

{\sla Material required:}

Repair kit, comprising repair guide, poly V-belt and special tool.\footnote{See spare parts catalog, under \aW{Service parts}.}

\startSteps
\item Cut through the poly V-belt.
\item Follow the steps in the repair guide.
\stopSteps

\starttextbackground [FC]
\startPictPar
\PMrtfm
\PictPar
\startitemize [1]
\item Ensure that the belt is correctly seated in the pulleys.
\item Start the engine and check the belt is running correctly.
\stopitemize
\stopPictPar
\stoptextbackground


\subsubsection [sSec:belt:tens] {Replacing the tensioning element}

{\sla Only for configurations with A/C compressor}

\blank [medium]

\placefig [margin] [fig:belt:tens:change] {Replacing the tensioning element}
{\externalfigure[vw:belt:tens:change]
\noteF
\startLeg
\item Tensioning element
\item Securing bolt
\stopLeg

{\bf Tightening torque}

Securing bolt:

\TorqueR~20\,Nm\:+\,½~turn (180°).}

\startSteps
\item Remove the poly V-belt as described above (see \atpage[sSec:belt:change]).
\item Remove peripheral parts (depending on configuration).
\item Unscrew the securing bolt (\in{\Ltwo, Fig.}[fig:belt:tens:change]).
\stopSteps

Follow the same sequence in reverse to install the tensioning element.

\starttextbackground [FC]
\startPictPar
\PMrtfm
\PictPar
\startitemize [1]
\item Always use a new securing bolt after installation.
\item Tightening torque: See \in{Fig.}[fig:belt:tens:change].
\stopitemize
\stopPictPar
\stoptextbackground

\stopregister[index][reg:main:vw]

\stopsection

\page [yes]
\setups[pagestyle:marginless]


\startsection[title={Hydraulic system},
reference={sec:hydraulic}]

\starttextbackground [FC]
% \startfiguretext[left,none]{}
% {\externalfigure[toni_melangeur][width=30mm]}

\startSymPar
\externalfigure[toni_melangeur][width=4em]
\SymPar
\textDescrHead{Recycling operating fluids}
Used operating fluids and lubricants must be neither dumped nor incinerated.

Used lubricants must not be disposed of in the wastewater network nor drained into the natural environment. They must not be disposed of with domestic waste.

Used lubricants must not be mixed with other liquids, as doing so could produce toxic substances or substances that are difficult to dispose of.
\stopSymPar
\stoptextbackground
\blank [big]

% \starthangaround{\PMgeneric}
% \textDescrHead{Qualification du personnel}
% Toute intervention sur l’installation hydraulique de votre véhicule ne peut être réalisée que par une personne dument qualifiée, ou par un service reconnu par \boschung.
% \stophangaround
% \blank[big]

\startSymList
\PHgeneric
\SymList
\textDescrHead{Cleanliness} The hydraulic system is highly sensitive to any impurities in oil. It is therefore important to work in a totally clean environment.
\stopSymList

\startSymList
\PHhot
\SymList
\textDescrHead{Spray hazard}
Relieve the residual pressure in the relevant hydraulic circuit before commencing any work on the hydraulic system of the \sdeux\ . Hot oil that sprays out of the system can cause burns.
\stopSymList

\startSymList
\PHhand
\SymList
\textDescrHead{Crush hazard}
Always lower the dump bed or secure it mechanically with the
safety prop before commencing any work on the hydraulic system of the \sdeux\ .
\stopSymList

\startSymList
\PImano
\SymList
\textDescrHead{Pressure measurement}
To measure hydraulic pressure, attach a manometer to one of the \aW{Minimess} monitoring connections for the circuit. Ensure that the manometer has an appropriate measuring range.
\stopSymList

\page [yes]

\setups[pagestyle:normal]

\subsection{Maintenance intervals}

\start

 \setupTABLE [frame=off,
style={\ssx\setupinterlinespace[line=.93\lH]},
background=color,
option=stretch,
split=repeat]
 \setupTABLE [r] [each] [
topframe=on,
framecolor=white,
backgroundcolor=TableLight,
% rulethickness=.8pt,
]

 % \setupTABLE [c] [odd] [backgroundcolor=TableMiddle]
 % \setupTABLE [c] [even] [backgroundcolor=TableLight]
 \setupTABLE [c] [1][ % width=30mm,
style={\bfx\setupinterlinespace[line=.93\lH]},
]
 \setupTABLE [r] [first] [topframe=off,
style={\bfx\setupinterlinespace[line=.93\lH]},
backgroundcolor=TableMiddle,
]
 % \setupTABLE [r] [2][style={\ssBfx\setupinterlinespace[line=.93\lH]}]


\bTABLE

\bTABLEhead
\bTR\bTD Maintenance work\eTD\bTD Interval \eTD\eTR
\eTABLEhead

\bTABLEbody
\bTR\bTD Check for leaks \eTD\bTD Daily \eTD\eTR
\bTR\bTD Check hydraulic oil fill level \eTD\bTD Daily \eTD\eTR
\bTR\bTD Check the condition of the hydraulic lines/hoses; replace if necessary \eTD\bTD 600\,h / 12~months \eTD\eTR
\bTR\bTD Change hydraulic oil return and intake filter \eTD\bTD 600\,h / 12~months \eTD\eTR
\bTR\bTD Grease the coil cores of the solenoid valves with copper grease \eTD\bTD 600\,h / 12~months \eTD\eTR
\bTR\bTD Change the hydraulic oil \eTD\bTD 1200\,h / 24~months\eTD\eTR
\eTABLEbody
\eTABLE
\stop


\subsection[niveau_hydrau]{Fill level}

\placefig[margin][fig:hydraulic:level]{Hydraulic fluid fill level}
{\externalfigure[hydraulic:level]
\noteF
\startLeg
\item Optimum fill level
\stopLeg}

A transparent level indicator\index{Fill level+Hydraulic fluid}\index{Maintenance+Hydraulic system} can be used to check the hydraulic oil fill level.
If the hydraulic oil fill level has dropped, the cause must be identified before the fill level is topped up. Comply with the stated change intervals (table above) and specifications for the hydraulic fluid (see table, \at{p.}[sec:liqquantities]).


\subsubsection{Topping up hydraulic fluid}

Add hydraulic fluid until the fill level reaches the middle of the level indicator.
Start the engine and, if necessary, top up the oil until the required fill level is obtained.


\subsection{Changing hydraulic fluid}

The fill volume and required specifications for the hydraulic fluid can be found in the table on \at{p.}[sec:liqquantities].

\startSteps
\item Open the filler cap of the hydraulic tank.
\item Use an oil extraction device or remove the drain plug to empty the tank.

The drain plug is located on the bottom of the hydraulic tank, in front of the left rear wheel (\in{Fig.}[fig:hydraulic:fluidDrain]).
\item Add hydraulic fluid until the fill level reaches the middle of the level indicator.
Start the engine and, if necessary, top up the oil until the required fill level is obtained.
\stopSteps

\placefig[margin][fig:hydraulic:fluidDrain]{Drain plug}
{\externalfigure[hydraulic:fluidDrain]}


\placefig[margin][fig:hydraulic:returnFilter]{Hydraulic filter}
{\externalfigure[hydraulic:returnFilter]}

\subsection[filtres:nettoyage]{Return and intake filter}

\startSteps
\item Raise the dirt hopper and put the safety prop in place.
\item Remove the lid from the filter on the hydraulic tank (\in{Fig.}[fig:hydraulic:returnFilter]).
\item Replace\index{Oil filter+Hydraulic oil} the filter element with a new one.
\item Wet a new O-ring seal with a little hydraulic fluid and fix it in place.
\item Use two hands to screw the lid back into place (\TorqueR~approx.~20\,Nm).
\stopSteps

\page [yes]


\subsection[sec:solenoid]{Lubricating the solenoid valves}

\placefig[margin][graissage_bobine]{Lubricating the solenoid valves}
{\externalfigure[graissage_bobine][M]
\noteF
\startLeg
\item Solenoid valve coil
\item Coil core
\stopLeg}

Moisture and salt residues that penetrate into the core of the electromagnetic coils can cause the cores to corrode. The coil cores must be greased once a year with copper grease. The grease must be corrosion-, water- and temperature||resistant up to 50\,°C:
\startSteps
\item Remove the coil of the solenoid valve (\in{\Lone, Fig.}[graissage_bobine]).
\item Lubricate the core (\in{\Ltwo, Fig.}[graissage_bobine]) with the specified special grease and reinstall the coil.
\stopSteps


\subsection{Replacing hoses}

The protective rubber\index{Hoses+Maintenance intervals} and fabric reinforcement of the hoses are subject to a natural aging process. As a result, the hoses of the hydraulic system must be replaced at the specified intervals, even if there is no {\em visible} sign of damage.

Ensure that the hoses are fastened to the vehicle correctly, so as to prevent premature wear due to friction. They must be spaced far enough away from other components to prevent damage that could be caused by friction and vibration.

\stopsection

\page [yes]

\setups [pagestyle:bigmargin]


\startsection[title={Braking system},
reference={sec:brake}]

\placefig[margin][fig:brake:rear]{Drum brake}
{\startcombination [1*2]
{\externalfigure[brake:wheelHub]}{\slx Rear wheel hub}
{\externalfigure[brake:drum]}{\slx Mechanism and brake fittings}
\stopcombination}

The brake drums~\Lfour\ must be disassembled during all regular maintenance work, the braking mechanism~\Lseven\ must be cleaned and the brake fittings~\Lfive, \Lsix\ must be visually inspected (\in{Fig.}[fig:brake:rear]).


\subsubject {Disassembly}

\startSteps
\item Drive the vehicle onto a suitable lifting platform and lift the wheels.
\item Remove the wheels.
\stopSteps


{\sla Disassembling the front wheel brakes}

\startSteps [continue]
\item Remove the brake drum~\Lfour\ .
\stopSteps

{\sla Disassembling the rear wheel brakes}

\startSteps [continue]
\item Remove the cover~\Lone\ from the hub.
\item Remove the bolt~\Ltwo\ and take off the isolator.
\item Use a socket wrench to unscrew the hub retaining nut~\Lthree\ .
\item Remove the hub with the brake drum.
\stopSteps


\subsubject {Reassembly}

Reassemble the brake drum in the reverse order. Tighten the nuts of the rear wheel hubs~\Lthree\ with the specified tightening torque of 190\,Nm.

\stopsection

\page [yes]

\setups [pagestyle:normal]


\startsection[title={Checking and maintaining the tires},
reference={sec:pneumatiques}]

The tires\index{Tires+Maintenance} must always be in perfect condition, so that they can fulfil their primary functions~– good road holding and perfect braking performance. Key factors that contribute to accidents include excessive wear and incorrect tire pressure, particularly tire pressures that are too low.


\subsection{Safety|-|relevant points}

\subsubsection{Checking wear}

Use the wear indicators located in the tread (\in{Fig.}[pneususure]) to check tire wear.
A visual check can be used to identify any problems with the tires and their causes:

\placefig[margin][pneususure]{Checking wear}
{\Framed{\externalfigure[pneusUsure][M]}}

\placefig[margin][pneusdomages]{Tire damage}
{\Framed{\externalfigure[pneusDomages][M]}}

\startitemize
\item Wear at the sides of the tread area: Tire pressure too low.
\item Intensified wear in the middle: Tire pressure too high.
\item Asymmetrical wear on the sides of the tire: Front axle (track, axle geometry) incorrectly set.
\item Cracks in the tread: Tires too old~– the rubber in tires becomes brittle over time and starts to crack (\in{Fig.}[pneusdomages]).
\stopitemize

\starttextbackground[CB]
\startPictPar
\PHgeneric
\PictPar
\textDescrHead{Risks associated with worn tires}
A worn tire no longer fulfils its functions, particularly in terms of water and mud displacement~– the braking distance is extended and driving performance impaired. A worn tire slips more easily, particularly in wet conditions. There is a greater risk that the tires will lose their ability to hold the road.
\stopPictPar
\stoptextbackground


\subsubsection{Tire pressure}

The prescribed tire pressure is indicated on the wheel specification plate at the front of the console, on the passenger side (see \atpage [sec:plateWheel]).

Even\index{Tires+Pressure} when the tires are in good condition, air escapes from them over time at varying rates (the more the vehicle is driven, the quicker the tire pressure will drop). Tire pressure should therefore be checked on a monthly basis and when the tires are cold. If you check the tire pressure when the tires are warm, you will need to add 0.3\,bar to the prescribed tire pressure.

\start
\setupcombinations[M]
\placefig[margin][pneuspression]{Tire pressure}
{\Framed{\externalfigure[pneusPression][M]}
\noteF
\startLeg
\item Correct pressure
\item Pressure too high
\item Pressure too low
\stopLeg
The prescribed tire pressure is indicated on the wheel specification plate at the front of the console, on the passenger side.}
\stop

\starttextbackground[CB]
\startPictPar
\PHgeneric
\PictPar
\textDescrHead{Hazards when tire pressure is too low}
A tire can rupture when its pressure is too low. When a tire has not been pumped up enough, or when the vehicle is overloaded, the tire becomes more compressed. This causes the rubber to get hot and parts of the tire may separate when taking a corner.
\stopPictPar
\stoptextbackground

\stopsection

\page [yes]

\setups[pagestyle:marginless]


\startsection[title={Chassis},
reference={main:chassis}]

\subsection{Safety||relevant fixings for components}

During all maintenance work, it is important to check that the safety||relevant fixing bolts of certain components are correctly seated and tightened with the specified tightening torque. This applies in particular to the articulated steering system and axles.

\blank [big]

\startfigtext [left] [fig:frontAxle:fixing] {Front axle}
{\externalfigure [frontAxle:fixing]}
{\sla Front axle fixings}
\startLeg
\item Spring leaf fixing: \TorqueR\ 150\,Nm
\item Traction unit fixing: \TorqueR\ 78\,Nm
\stopLeg

{\sla Rear axle fixings}
\startLeg
\item Spring leaf fixing: \TorqueR\ 150\,Nm
\stopLeg

\stopfigtext

\start

\setupTABLE [frame=off,style={\ssx\setupinterlinespace[line=.93\lH]},background=color,
option=stretch,
split=repeat]

\setupTABLE [r] [each] [topframe=on,
framecolor=white,
% rulethickness=.8pt
]

\setupTABLE [c] [odd] [backgroundcolor=TableMiddle]
\setupTABLE [c] [even] [backgroundcolor=TableLight]
\setupTABLE [c] [1][style={\bfx\setupinterlinespace[line=.93\lH]}]
\setupTABLE [r] [first] [topframe=off,style={\bfx\setupinterlinespace[line=.93\lH]},
]
% \setupTABLE [r] [2][style={\bfx\setupinterlinespace[line=.93\lH]}]


\bTABLE

\bTABLEhead
\bTR [backgroundcolor=TableDark] \bTD [nc=3] Tightening torques \eTD\eTR
% \bTR\bTD Position \eTD\bTD Type de vis \eTD\bTD Couple \eTD\eTR
\eTABLEhead

\bTABLEbody
\bTR\bTD Drive motors, left/right \eTD\bTD M12\:×\:35~8.8 \eTD\bTD 78\,Nm \eTD\eTR
%% NOTE @Andrew: das sind Hydraulikmotoren
\bTR\bTD Working pump \eTD\bTD M16\:×\:40~100 \eTD\bTD 330\,Nm \eTD\eTR
\bTR\bTD Drive pump \eTD\bTD M12\:×\:40~100 \eTD\bTD 130\,Nm \eTD\eTR
\bTR\bTD Spring leaves, front/back \eTD\bTD M16\:×\:90/160~8.8 \eTD\bTD 150\,Nm \eTD\eTR
% \bTR\bTD Fixation du système oscillant \eTD\bTD M12\:×\:40~8.8 \eTD\bTD 78\,Nm \eTD\eTR
\bTR\bTD Dirt hopper fixing \eTD\bTD M10\:×\:30 Verbus Ripp~100 \eTD\bTD 80\,Nm \eTD\eTR
\bTR\bTD Wheel nuts \eTD\bTD M14\:×\:1.5 \eTD\bTD 180\,Nm \eTD\eTR
\bTR\bTD Front broom fixing \eTD\bTD M16\:×\:40~100 \eTD\bTD 180\,Nm \eTD\eTR
\eTABLEbody
\eTABLE
\stop


\stopsection


\page [yes]

\startmode [main:centralLub]

\startsection[title={Central lubrication system},
							reference={main:graissageCentral}]



\subsection{Description of the control module}

The \sdeux\ can be equipped with\index{Central lubrication system} a central lubrication system (optional). The central lubrication system supplies all lubrication points of the vehicle with lubricant at regular intervals.

\startfigtext [left] [vogel_affichage] {Display module}
{\externalfigure[vogel_base2][W50]}
\blank
\startLeg
\item 7-segment display: Values and operating status
\item \LED: System paused (standby operation)
\item \LED: Pump in operation
\item \LED: System controlled by cycle switch
\item \LED: System monitored by pressure switch
\item \LED: Error message
\item Scroll buttons:
\startLeg [R]
\item Activates the display
\item Displays the values
\item Changes the values
\stopLeg
\item Button for changing the operating mode; confirms values
\item Triggers interim lubricant cycle
\stopLeg
\stopfigtext

The central lubrication system includes a pump, the transparent lubricant container on the left side of the chassis and the control module in the central electronics system.\par

\page [yes]


\start

\setupTABLE[frame=off,style={\ssx\setupinterlinespace[line=.93\lH]},background=color,
option=stretch,
split=repeat]
\setupTABLE[r][each][
topframe=on,framecolor=white]

\setupTABLE[c][odd][backgroundcolor=TableMiddle]
\setupTABLE[c][even][backgroundcolor=TableLight]
\setupTABLE[c][1][width=9mm,style={\bfx\setupinterlinespace[line=.93\lH]}]
\setupTABLE[r] [first][topframe=off,style={\bfx\setupinterlinespace[line=.93\lH]}]
\setupTABLE[r][2][style={\bfx\setupinterlinespace[line=.93\lH]}]


\bTABLE

\bTABLEhead
\bTR [backgroundcolor=TableDark]
\bTD [nc=4] Display and buttons of the control module \eTD\eTR
\bTR\bTD Pos. \eTD
\bTD \LED \eTD\bTD Display mode \eTD
\bTD Programming mode \eTD\eTR
\eTABLEhead

\bTABLEbody
\bTR\bTD 2 \eTD
\bTD Operating status {\em Pause} \eTD
\bTD The system is in standby operation \eTD
\bTD The pause time can be changed \eTD\eTR
\bTR\bTD 3 \eTD
\bTD Operating status {\em Contact} \eTD
\bTD The pump is working \eTD
\bTD The work duration can be changed \eTD\eTR
\bTR\bTD 4 \eTD
\bTD System control {\em CS} \eTD
\bTD With the external cycle switch \eTD
\bTD The control mode can be deactivated or changed \eTD\eTR
\bTR\bTD 5 \eTD
\bTD System monitoring {\em PS} \eTD
\bTD With the external pressure switch \eTD
\bTD The control mode can be deactivated or changed \eTD\eTR
\bTR\bTD 6 \eTD
\bTD Malfunction {\em Fault} \eTD
\bTD [nc=2] There is a malfunction. The cause can be displayed in the form of an error code after pressing the \textSymb{vogel_DK}~button. Execution of the functions will be interrupted. \eTD\eTR
\bTR\bTD 7 \eTD
\bTD Arrow buttons \textSymb{vogelTop}~\textSymb{vogelBottom} \eTD
\bTD [nc=2] \items[symbol=R]{Activates the display,Queries the parameters (display mode),Sets the displayed (I) value (programming mode)}
\eTD\eTR
\bTR\bTD 8 \eTD
\bTD ​\textSymb{vogelSet} button \eTD
\bTD [nc=2] Switches between display and programming mode or confirms the values entered. \eTD\eTR
\bTR\bTD 9 \eTD
\bTD ​\textSymb{vogel_DK} button \eTD
\bTD [nc=2] If the device is in {\em Pause} mode, pressing this button initiates the interim lubricant cycle. The error messages are confirmed and deleted. \eTD\eTR
\eTABLEbody
\eTABLE
\stop
\vfill

\startfigtext [left] [vogel_touches]{Display module}
{\externalfigure[vogel_base][width=50mm]}
\textDescrHead{Display mode} Briefly press one of the arrow buttons~\textSymb{vogelTop}~\textSymb{vogelBottom} to activate the 7-segment display~\textSymb{led_huit}. Press the \textSymb{vogelTop}~button again to display the different parameters followed by their values. When {\em Display} mode is active the \LED\char"2060s stay lit (\in{2~to 6, Fig.}[vogel_affichage]).
\blank [medium]
\textDescrHead{Programming mode} To change the values press the \textSymb{vogelSet}~button for at least 2~seconds to switch to {\em Programming} mode: The \LED\char"2060s will flash. Briefly press the \textSymb{vogelSet}~button to change the\index{Central lubrication system+Programming} display, then change the required value using the \textSymb{vogelTop}~\textSymb{vogelBottom}~buttons. Confirm by pressing\index{Central lubrication system+Display} the \textSymb{vogelSet}~button.
\stopfigtext


\subsection{Submenus in {\em Display} mode}

\vskip -9pt

\adaptlayout [height=+5mm]

\startcolumns[balance=no]\stdfontsemicn
\startSymVogel
\externalfigure[vogel_tpa][width=26mm]
\SymVogel
\textDescrHead{Pause time [h]} Press~\textSymb{vogelTop} to display the programmed values.
\stopSymVogel

\startSymVogel
\externalfigure[vogel_068][width=26mm]
\SymVogel
\textDescrHead{Remaining pause time [h]} Remaining time to the next lubrication cycle.
\stopSymVogel

\startSymVogel
\externalfigure[vogel_090][width=26mm]
\SymVogel
\textDescrHead{Total pause time [h]} Total pause time between two cycles.
\stopSymVogel

\startSymVogel
\externalfigure[vogel_tco][width=26mm]
\SymVogel
\textDescrHead{Lubrication time [min]} Press~\textSymb{vogelTop} to display the programmed values.
\stopSymVogel

\startSymVogel
\externalfigure[vogel_tirets][width=26mm]
\SymVogel
\textDescrHead{Device in standby} Display not possible because device is in standby (pause).
\stopSymVogel

\startSymVogel
\externalfigure[vogel_026][width=26mm]
\SymVogel
\textDescrHead{Lubrication time [min]} Duration of a lubrication process.
\stopSymVogel

\startSymVogel
\externalfigure[vogel_cop][width=26mm]
\SymVogel
\textDescrHead{System check} Press~\textSymb{vogelTop} to display the programmed value.
\stopSymVogel

\startSymVogel
\externalfigure[vogel_off][width=26mm]
\SymVogel
%% TODO: line break / spacing
\textDescrHead{Control mode} \emspace PS:~pressure switch; CS:~cycle switch; OFF:~deactivated.
\stopSymVogel

\startSymVogel
\externalfigure[vogel_0h][width=26mm]
\SymVogel
\textDescrHead{Operating hours} Press~\textSymb{vogelTop} to display the value in two steps.
\stopSymVogel

\startSymVogel
\externalfigure[vogel_005][width=26mm]
\SymVogel
\textDescrHead{Part 1: 005} The operating time is displayed in two parts. Press~\textSymb{vogelTop} to view part~2.
\stopSymVogel

\startSymVogel
\externalfigure[vogel_338][width=26mm]
\SymVogel
\textDescrHead{Part 2: 33.8} The 2nd~part of the number is 33.8, resulting in a total operating time of 533.8\,h.
\stopSymVogel

\startSymVogel
\externalfigure[vogel_fh][width=26mm]
\SymVogel
\textDescrHead{Error time} Press~\textSymb{vogelTop} to display the value in two steps.
\stopSymVogel

\startSymVogel
\externalfigure[vogel_000][width=26mm]
\SymVogel
\textDescrHead{Part 1: 000} The error time is displayed in two parts. Press~\textSymb{vogelTop} to view part~2.
\stopSymVogel

\startSymVogel
\externalfigure[vogel_338][width=26mm]
\SymVogel
\textDescrHead{Part 2: 33.8} The 2nd~part of the number is 33.8, resulting in a total error time of 33.8\,h.
\stopSymVogel

\stopcolumns

\stopsection

\page [yes]

\stopmode % central lubrication

\setups [pagestyle:marginless]


\startsection[title={Lubricating by hand},
reference={sec:grasing:plan}]

\starttextbackground [FC]
\startPictPar
\PMgeneric
\PictPar
The lubricating points indicated in the lubricating plan (\in{Fig.}[fig:greasing:plan]) must be lubricated on a regular basis. Regular lubrication is essential to ensure {\em friction can be kept to a minimum} over the long term and to keep out moisture and other corrosive substances.
\stopPictPar
\stoptextbackground

\blank [big]

\start

\setupcombinations [width=\textwidth]

\placefig[here][fig:greasing:plan]{Vehicle lubricating plan}
{\startcombination [3*1]
{\externalfigure[frame:steering:greasing]}{\ssx Articulated steering and pendulum mechanism}
{\externalfigure[frame:axles:greasing]}{\ssx Axles}
{\externalfigure[frame:sucMouth:greasing]}{\ssx Suction mouth}
\stopcombination}

\stop

\vfill

\startLeg [columns,three]
\item Lifting cylinder of the articulated steering\crlf {\sl 2 lubricating nipples per cylinder}
\item Bearings of articulated steering\crlf {\sl 2 lubricating nipples on the left side}
\columnbreak
\item Bearings of the pendulum mechanism\crlf {\sl 1 lubricating nipple in front of the tank}
\item Leaf springs\crlf {\sl 2 lubricating nipples per leaf spring}
\columnbreak
\item Suction mouth\crlf {\sl 1 lubricating nipple per wheel}
\item Suction mouth\crlf {\sl 1 lubricating nipple on the traction arm}
\stopLeg


\page [yes]
\setups [pagestyle:bigmargin]


\subsubject{Lubricating the dirt hopper}

The dirt hopper has 6~lubricating points (2\:×\:3) that must be lubricated on a weekly basis.

\blank [big]


\placefig[here][fig:greasing:container]{Lifting mechanism of hopper} {\externalfigure[container:mechanisme]}


\placelegende [margin,none]{}
{{\sla Key:}

\startLeg
\item Left hopper bearing
\item Right hopper bearing
\item Left hydraulic cylinder (top)
\item Left hydraulic cylinder (bottom)

{\em Same as right cylinder (Item~\in[greasing:point;hide]).}
\item Right hydraulic cylinder (top)
\item [greasing:point;hide] Right hydraulic cylinder (bottom)
\stopLeg}

\stopsection
\page [yes]

\setups [pagestyle:bigmargin]


\startsection[title={Electrical system},
reference={sec:main:electric}]

\subsection{Central electrical system in the chassis}

\startbuffer [fuses:preventive]
\starttextbackground [CB]
\startPictPar
\PHvoltage
\PictPar
\textDescrHead{Safety requirements}
Please observe the safety requirements in\index{Fuses+Chassis} this\index{Relays+Chassis} guide: Always replace fuses with fuses that have the specified amperage; remove metal jewelry before working on the electrical\index{Electrical system} system (rings, bracelets, etc.).
\stopPictPar
\stoptextbackground
\stopbuffer


\subsubsubject{MIDI fuses}


\starttabulate[|l|r|p|]
\HL
\NC\md F 1 \NC 5\,A \NC Brake light, \aW{+\:15} OBD \NC\NR
\NC\md F 2 \NC 5\,A \NC \aW{+\:15} Engine control\NC\NR
\NC\md F 3 \NC 7.5\,A \NC \aW{+\:30} Engine control and OBD \NC\NR
\NC\md F 4 \NC 20\,A \NC Fuel pump\NC\NR
\NC\md F 5 \NC 20\,A \NC \aW{D\:+} Alternator, \aW{+\:15} relay K 1 \NC\NR
\NC\md F 6 \NC 5\,A \NC Engine control\NC\NR
\NC\md F 7 \NC 10\,A\NC Engine exhaust gas treatment\NC\NR
\NC\md F 8 \NC 20\,A \NC Engine electronics (control) \NC\NR
\NC\md F 9 \NC 15\,A \NC Engine exhaust gas treatment, supply unit, pre||heating \NC\NR
\NC\md F 10\NC 30\,A \NC Engine control\NC\NR
\NC\md F 11\NC 5\,A \NC Working light, rear \NC\NR
\HL
\stoptabulate

\placefig [margin] [fig:electric:power:rear] {Central electrical system in chassis}
{\externalfigure [electric:power:rear]
\noteF
\startKleg
\sym{K 1} Electronic engine control unit
\sym{K 2} Fuel pump
\sym{K 3} Release for starter
\sym{K 4} Brake lights
\sym{K 5} {[}Reserve{]}
\sym{K 6} Working light, rear
\sym{K 7} Pre||heating system
\stopKleg
}


\subsubsubject{MAXI fuses}

% \startcolumns [n=2]
\starttabulate[|l|r|p|]
\HL
\NC\md F 15 \NC 50\,A \NC Main supply unit of the central electrical system\NC\NR
\HL
\stoptabulate

\page [yes]

\setups[pagestyle:marginless]


\subsection{Central electrical system in driver’s cabin}

\startcolumns[rule=on]

\placefig [bottom] [fig:fuse:cab] {Fuses and relays in driver’s cabin}
{\externalfigure [electric:power:front]}

% \vfill

\subsubsubject{Relays}

\index{Fuses+Driver’s cabin}\index{Relays+Driver’s cabin}

\starttabulate[|lB|p|]
\NC K 2\NC A/C compressor\NC\NR
\NC K 3\NC A/C compressor\NC\NR
\NC K 4\NC Electrical water pump\NC\NR
\NC K 5\NC Beacon light\NC\NR
\NC K 10 \NC Turn signal load equalizer\NC\NR
\NC K 11 \NC Low||beam headlight\NC\NR
\NC K 12 \NC High||beam headlight {[}Reserve{]} \NC\NR
\NC K 13 \NC Working lights\NC\NR
\NC K 14 \NC Windshield wiper, intermittent mode\NC\NR
\stoptabulate

\vskip -24pt

\placefig [bottom] [fig:fuse:access] {Access hatch to central electrical system}
{\externalfigure [electric:power:cabin]}

\stopcolumns

\page [yes]


\subsubsubject{MINI fuses}


\startcolumns[rule=on]
% \setuptabulate[frame=on]
%\placetable[here][tab:fuses:cab]{Fusibles dans la cabine}
%{\noteF
\starttabulate[|lB|r|p|]
\NC F 1 \NC 3\,A \NC Parking light, left \NC\NR
\NC F 2 \NC 3\,A \NC Parking light, right \NC\NR
\NC F 3 \NC 7.5\,A \NC Low||beam headlight, left \NC\NR
\NC F 4 \NC 7.5\,A \NC Low||beam headlight, right \NC\NR
\NC F 5 \NC 7.5\,A \NC High||beam headlight, left {[}Reserve{]} \NC\NR
\NC F 6 \NC 7.5\,A \NC High||beam headlight, right {[}Reserve{]} \NC\NR
\NC F 7 \NC 10\,A \NC Working lights, top \NC\NR
\NC F 8 \NC 10\,A \NC Working lights, bottom (reserve) \NC\NR
\NC F 9 \NC — \NC {[}Unassigned{]} \NC\NR
\NC F 10 \NC 10\,A \NC Windshield wiper\NC\NR
\NC F 11 \NC 5\,A \NC Switch, lights and hazard warning lights \NC\NR
\NC F 12 \NC 5\,A \NC {[}Reserve{]} \NC\NR
\NC F 13 \NC 10\,A \NC External mirror heating \NC\NR
\NC F 14 \NC 7.5\,A \NC \aW{+\:15} Radio and camera \NC\NR
\NC F 15 \NC 10\,A \NC \aW{+\:30} Hazard warning lights\NC\NR
\NC F 16 \NC 5\,A \NC Steering column lighting \NC\NR
\NC F 17 \NC 7.5\,A \NC \aW{+\:30} Radio and interior lighting \NC\NR
\NC F 18 \NC — \NC {[}Unassigned{]} \NC\NR
\NC F 19 \NC 20\,A \NC \aW{+\:30} RC 12, front \NC\NR
\NC F 20 \NC 20\,A \NC \aW{+\:30} RC 12, rear \NC\NR
\NC F 21 \NC 15\,A \NC 12\,V socket \NC\NR
\NC F 22 \NC 5\,A \NC Ignition, multifunction console, Vpad \NC\NR
\NC F 23 \NC 5\,A \NC Emergency shut-down, center console, RC 12, front \NC\NR
\NC F 24 \NC 5\,A \NC Emergency shut-down, center console, RC 12, rear \NC\NR
\NC F 25 \NC 2\,A \NC \aW{+\:15} RC 12, front \NC\NR
\NC F 26 \NC 2\,A \NC \aW{+\:15} RC 12, rear \NC\NR
\NC F 27 \NC 15\,A \NC Heating fan \NC\NR
\NC F 28 \NC 10\,A \NC A/C compressor, central lubrication system\NC\NR
\NC F 29 \NC 15\,A \NC A/C condenser\NC\NR
\NC F 30 \NC 5\,A \NC Thermostat, air||conditioning system \NC\NR
\NC F 31 \NC 5\,A \NC \aW{+\:15} Multifunction console/Vpad \NC\NR
\NC F 32 \NC 15\,A \NC Electrical water pump, beacon light \NC\NR
\NC F 33 \NC — \NC {[}Unassigned{]} \NC\NR
\NC F 34 \NC — \NC {[}Unassigned{]} \NC\NR
\NC F 35 \NC — \NC {[}Unassigned{]} \NC\NR
\NC F 36 \NC — \NC {[}Unassigned{]} \NC\NR
\stoptabulate
\stopcolumns

\page [yes]

\setups [pagestyle:bigmargin]


\subsection[sec:lighting]{Lights and signals}


\placefig [here] [fig:lighting] {Vehicle lights and signals}
{\externalfigure [vhc:electric:lighting]}

\placelegende [margin,none]{}{%
\vskip 30pt
{\sla Key:}
\startLongleg
\item Parking lights\hfill 12\,V~– 5\,W
\item Low-beam headlights\hfill H7 12\,V~– 55\,W
\item Turn signals\hfill orange 12\,V~– 21\,W
\item Working lights\hfill G886 12\,V~– 55\,W
\item Turn signals\hfill 12\,V~– 21\,W
\item Rear/brake lights\hfill 12\,V~– 5/21\,W
\item Reversing headlights\hfill 12\,V~– 21\,W
\item {[}Free{]}
\item License plate lighting\hfill 12\,V~– 5\,W
\item Beacon light\hfill H1 12\,V~– 55\,W
\stopLongleg}

\subsubsubject{Adjusting the headlights}

\placefig [margin] [fig:lighting:adjustment] {Beam at 5\,m}
{\externalfigure [vhc:lighting:adjustment]
\startitemize
\sym{H\low{1}} Height of the filament: 100\,cm
\sym{H\low{2}} Correction at 2\hairspace\%: 10\,cm
\stopitemize}

{\md Requirements:} Freshwater/recycling water tank full, driver at wheel.

Headlight alignment is set at the factory. The height and incline of the beam can be changed by pivoting the plastic holder.

If it is established during an inspection that the setting needs to be adjusted, remove the securing bolt and correct the incline to comply with the applicable statutory regulations (see \in{Fig.}[fig:lighting:adjustment]). Retighten the securing bolt.

\page [yes]
\setups [pagestyle:marginless]


\subsection[sec:battcheck]{Battery}

\subsubsection{Safety instructions}

\startSymList
\PPfire
\SymList
\textDescrHead{Explosion hazard}
The\index{Battery+Safety instructions}\index{Hazard+Explosion} charging of batteries produces explosive\index{Oxyhydrogen gas} oxyhydrogen gas. Only charge batteries in a sufficiently ventilated area! Avoid sparking!
Keep the battery away from fire and naked flames and do not smoke near it.
\stopSymList

\startSymList
\PHvoltage
\SymList
\textDescrHead{Risk of short circuit}
If\index{Battery+Service} the positive terminal of the connected battery comes into contact with vehicle parts there\index{Hazard+Fire}\index{Hazard+Explosion} is a risk of a short circuit.
This can ignite the explosive gas mixture emitted by the battery, seriously injuring you and others.

\startitemize
\item Do not place any metal objects or tools on the battery.
\item When disconnecting the battery, always disconnect the negative terminals first and then the positive terminals.
\item When reconnecting the battery, always connect the positive terminals first and then the negative terminals.
\item Do not loosen or disconnect the connection terminals of the battery while the engine is running.
\stopitemize
\stopSymList


\startSymList
\PHcorrosive
\SymList
\textDescrHead{Corrosive}
Wear\index{Hazard+Acid burn} safety goggles and acid||proof safety gloves. Battery fluid is approx. 27\hairspace\percent\ sulfuric acid (H\low{2}SO\low{4}) and can therefore cause acid burns. Neutralize\index{Battery+Hazard}\index{Battery+Fluid} battery fluid on the skin with a solution of bicarbonate of soda and rinse with clean water. If battery fluid gets into the eyes, rinse with large amounts of cold water and seek medical attention immediately.
\stopSymList

\page [yes]


\startSymList
  \startcombination[1*2]
    {\PHcorrosive}{}
    {\PHfire}{}
  \stopcombination
\SymList
\textDescrHead{Storing batteries}
Always store batteries\index{Battery+Storage} upright. Otherwise battery fluid may leak and cause acid burns or~– if it reacts with other substances~– fire. \par\null\par\null
\stopSymList

\starttextbackground [FC]
\setupparagraphs [PictPar][1][width=2.4em,inner=\hfill]

\startPictPar
\PMproteyes
\PictPar
\textDescrHead{Protective goggles}
Mixing\index{Hazard+Eye injury} water and acid can lead to fluid splashing into the eyes.
If acid splashes into the eyes, rinse the eyes with clean water and seek medical attention immediately!
\stopPictPar
\blank [small]

\startPictPar
\PMrtfm
\PictPar
\textDescrHead{Documentation}
Observe the safety instructions, protective measures and procedures set out in this User Manual when working with batteries.
\stopPictPar
\blank [small]

\startPictPar
\PStrash
\PictPar
\textDescrHead{Environmental protection}
Batteries\index{Environmental protection} contain harmful substances. Never dispose of old batteries as household waste.
Dispose of batteries in an environmentally friendly way. Hand them in to a specialist workshop or a returns service for used batteries.

Always transport and store filled batteries in an upright position. Batteries have to be secured against falling over during transport.
Battery fluid can leak from the vent holes of the vent plugs and get into the environment.
\stopPictPar
\stoptextbackground

\page [yes]

\setups[pagestyle:normal]


\subsubsection{Practical advice}

Batteries must always be fully charged wherever possible in order to maximize their service life.

When the vehicle is out of use for long periods,\index{Battery+Service life} conservation charging both extends the service life of the batteries and ensures the vehicle is always ready for use.

\placefig[margin][fig:batterycompartment]{\select{caption}{Battery compartment}{Battery compartment}}
{\externalfigure[batt:compartment]}


\subsubsection{Maintenance}

The battery of the \sdeux\ is a {\em maintenance||free} lead||acid battery. The battery requires no maintenance work other than charging and cleaning.

\startitemize
\item Ensure that the terminals of the battery are always clean and dry. Apply a thin coating of acid||resistant grease to the terminals.
\item Recharge any batteries with\index{Battery+Charging} an open-circuit voltage of\index{Battery+Open-circuit voltage} less than 12.4\,V.
\stopitemize

\placefig[margin][fig:bclean]{Cleaning the terminals}
{\externalfigure[batt:clean]
\noteF
Use\index{Battery+Cleaning}\index{Cleaning+Batteries} warm water to remove the white powder resulting from corrosion. If a terminal is corroded, disconnect the battery cable and clean the terminal with a wire brush. Then apply a thin layer of grease to the terminals.}


\subsubsection[sec:battery:switch]{How to use the battery disconnect switch}

%% TODO: check break
{\sl It is not advisable to use the battery disconnect switch on a regular basis (\eG\ daily)!}

\startSteps
\item Turn\index{Battery disconnect switch} the ignition off and then wait for around 1~minute.
\item Open the battery compartment (\inF[fig:batterycompartment]).
\item Press the red knob of the battery disconnect switch to interrupt the electric circuit.
\item To reconnect the electric circuit, turn the battery disconnect switch ¼~turn clockwise.
\stopSteps

%\starttextbackground [FC]
%\startPictPar
%\PMgeneric
%\PictPar
%The battery disconnect switch is intended as a means of temporarily disconnecting the battery to allow certain maintenance and repair work to be undertaken. It is not advisable to use the battery disconnect switch on a regular basis (\eG\ daily): Certain electronic components should have a constant power supply, otherwise error messages can be logged in the fault memory.
%\stopPictPar
%\stoptextbackground
%
\stopsection
\page [yes]


\setups[pagestyle:marginless]

\section[sec:cleaning]{Cleaning the vehicle}

Before cleaning the vehicle, remove\startregister[index][vhc:lavage]{Maintenance+Cleaning} mud and dirt from the vehicle body using plenty of water.
Be sure to wash not only the side panels, but also the wheel wells and the underside of the vehicle.\par

During winter, it is particularly important to wash the \ProductId thoroughly to remove highly corrosive\index{Corrosion+Prevention} road salt residue.

\starttextbackground [FC]
\startPictPar
\PHgeneric
\PictPar
\textDescrHead{Preventing water damage}
Never clean the vehicle using {\em water cannons} (\eG\ from the fire department) or {\em hydrocarbon||based cold cleaners.} Observe the corresponding regulations when using a pressure washer.
\stopPictPar
\blank [small]

\startPictPar
\PSwelt
\PictPar
\textDescrHead{Environmental protection}
Cleaning a vehicle can have a serious environmental impact.
Clean the \ProductId only in a\index{Environmental protection} location equipped with an oil separator. Observe the locally applicable environmental regulations.
\stopPictPar
\blank [small]

\startPictPar
\PMwarranty
\PictPar
\textDescrHead{Appropriate cleaning}
No liability or warranty claims can be brought against \BosFull{boschung} for damages which result from failure to comply with the cleaning instructions.
\stopPictPar
\stoptextbackground


\subsection{Using a pressure washer}

A conventional pressure washer can be used to clean\index{Cleaning+Pressure washer} the vehicle.

The following aspects have to be observed when using a pressure washer:

\startitemize
\item Maximum working pressure: 50\,bar
\item Flat spray nozzle with a spraying angle of 25°
\item Spraying distance at least 80\,cm
\item Maximum water temperature 40\,°C
\item See \about[reiMi] \atpage[reiMi].
\stopitemize

Failure to comply with these\index{Paintwork+Damage} instructions can result in damage to paintwork and corrosion protection\index{Damage+Paintwork}.

Please also observe operating instructions and the safety instructions for your pressure washer.

\starttextbackground [FC]
\startPictPar
\PPspray
\PictPar
When using a pressure washer, water can get into places where it can cause damage. The water jet should therefore never be directed at sensitive parts and devices:
\stopPictPar

\startitemize
\item Sensors and electrical connections
\item Starter, alternator, injection system
\item Solenoid valves
\item Vent openings
\item Mechanical components that have not cooled down
\item Information, warning and hazard labels
\item Electronic control units
\stopitemize

\textDescrHead{Washing the engine}
It is extremely important to stop water getting into suction and ventilation openings. Never direct the jet of pressure washers at electrical components and lines. Do not direct the jet at the injection system! After cleaning the engine, apply a preservative, but ensure the belt drive does not come into contact with the preservative agent.
\stoptextbackground

\starttextbackground [FC]
\setupparagraphs [PictPar][1][width=6em,inner=\hfill]
\startPictPar
\framed[frame=off,offset=none]{\PMproteyes\PMprotears}
\PictPar
\textDescrHead{Residual Water}
Water will collect in certain locations on the vehicle after cleaning (\eG\ in the hollows of the engine block or the gearbox). This water must be removed with compressed air. Please note that adequate protective clothing must be worn when working with compressed air and that the system has to comply with the applicable safety requirements (multi nozzle).
\stopPictPar
\stoptextbackground


\page [yes]
\subsubsection[reiMi]{Suitable cleaning agents}

Use\index{Cleaning agents} only cleaning agents which have the following properties:

\startitemize
\item No abrasive effect
\item pH value of 6–7
\item Solvent||free
\stopitemize

To remove stubborn marks from small areas of paintwork, carefully apply cleaner's naphtha or white spirit, but no other solvents.
Remove all solvent residues from paintwork. Cleaning plastic parts with cleaner's naphtha can cause cracks or discolorations!

Clean surfaces with\index{Cleaning+Labels} warning or information labels using clean water and a soft sponge.

Prevent water from penetrating into electrical components: Do not use car brushes to clean turn||signal and light housings. Use a soft cloth or sponge instead.

\starttextbackground [CB]
\setupparagraphs [PictPar][1][width=6em,inner=\hfill]
\startPictPar
\GHSgeneric\GHSfire
\PictPar
\textDescrHead{Chemical hazard}
Cleaning agents can pose health and safety risks (highly flammable substances). Observe the safety instructions for the cleaning agents used. Observe the hazard and safety data sheets for the agents used.
\stopPictPar
\stoptextbackground

\stopregister[index][vhc:lavage]

\page [yes]


\setups [pagestyle:bigmargin]

\startsection [title={Adjusting the suction mouth},
reference={sec:main:suctionMouth}]


The optimum\index{Suction mouth+Adjusting} gap between the ground surface and the rubber lip of the suction mouth is 10\,mm.
Use the three setting gauges in the tool box (driver’s cabin, driver’s side) to check and/or adjust the gap.


\placefig [margin] [fig:suctionMouth] {Adjusting the suction mouth}
{\Framed{\externalfigure [suctionMouth:adjust]}}

\placeNote[][service_picto]{}{%
\noteF
\starttextrule{\PHasphyxie\enskip Toxic and suffocation hazard \enskip}
{\md Important:} To ensure the suction mouth can be held in the float position, the vehicle engine must be running while adjustments are being made. To eliminate the risk of poisoning and suffocation, it is essential to use an exhaust gas extraction system and|/|or ensure work is carried out exclusively in a very well ventilated location.
\stoptextrule}

\startSteps
\item Park the vehicle in a well ventilated location on level and even ground.
\item Activate\index{Vacuum} \aW{working} mode (button on the end of the gearshift).

Run the engine at idling speed (press the \textSymb{joy_key_engine_decrease}~button on the multifunction console to reduce the engine speed).
\item Apply the parking brake and secure each rear wheel with a wheel block.
\item Press the \textSymb{joy_key_suction}~button to lower the suction mouth.
\item Place the three setting gauges~\LAa\ under the rubber lip of the suction mouth, as shown in the figure.
\item [sucMouth:adjust]Loosen the fixing~\Lone\ and adjustment~\Ltwo\ bolts of each wheel~– the four wheels will sink down to the ground.
\item Tighten bolts~\Lone\ and~\Ltwo then remove the three setting gauges.
\item Raise|/|lower the suction mouth and check the setting with the setting gauges. If the setting is still not quite right, repeat the adjustment procedure, starting from point~\in[sucMouth:adjust].

\stopSteps


\stopsection

\stopchapter
\stopcomponent
