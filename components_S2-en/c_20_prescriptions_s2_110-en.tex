\startcomponent c_20_prescriptions_s2_095-en
\product prd_ba_s2_095-en


\chapter [safety:risques] {Safety requirements}

\setups [pagestyle:marginless]


\section{Basic directions}

\subsubject{Legal principles}

Accidents can have serious consequences for both employers and employees. We would like to briefly draw attention to the responsibilities of both parties:\note[prescription:user:right].

Before entrusting the operation of the Sweeper to an employee, the employer must observe the following points:

\startSteps
\item Every vehicle driver must have completed the requisite training for driving the vehicle. Training documentation must be available.
\item Every vehicle driver must have an official driver license. This license can be issued only when the following three conditions have been met:
\startitemize [2]
\item The employee has passed a medical suitability test conducted by the company medical officer.
\item The employee is familiar with the deployment sites and with all the safety requirements communicated to him by his superior that are relevant to the vehicle’s working environment.
\item The employee has passed an aptitude test that confirms he has the knowledge and skills required to drive the vehicle.
\stopitemize
\stopSteps

If the vehicle’s maximum speed is over 25\,km/h (approx. 15 mph)\note[prescription:user:right], the vehicle must be officially registered and the vehicle driver must hold the following driver license:
\startitemize
\item Category B driver license (EU)\note[prescription:lisence] for vehicles weighing up to 3.5~metric tons or
\item Category C driver license (EU)\note[prescription:lisence] for vehicles weighing over 3.5~metric tons.
\stopitemize

If the vehicle’s maximum speed is 25\,km/h (approx. 15 mph), the vehicle driver must at the very least be familiar with the traffic regulations that apply on public roads, even if a Category B\note[prescription:user:right] driver license is not required to drive the vehicle.

\footnotetext [prescription:user:right] {The obligations of employers and employees may vary in different countries and regions. Please ensure you are familiar with the regulations and laws that apply in your country and region.}

\footnotetext[prescription:lisence] {Directive 2006/126/EC of the European Parliament and of the Council of 20~December 2006 on driving licences.}


\subsubject{Conditions of use}

The \sdeux\ may be used only when in perfect working order. The operator must also observe the safety instructions and guidelines contained in this User Manual. Malfunctions that compromise safety must be rectified|/|repaired immediately by a suitably qualified specialist workshop.
\blank [big]

\startSymList
\externalfigure [s2_inspection] [width=4.5em]
\SymList
{\md Daily maintenance:}
Inspect the vehicle after each deployment and repair any visible damage and defects. Contact a specialist workshop immediately if you notice vehicle damage or malfunctions. If this is not possible, stop the vehicle immediately and secure the location.
\stopSymList


\subsubject{Appropriate usage}

The \sdeux\ is designed for the cleaning and upkeep of roads, sidewalks and other spaces. Using it for any other purposes is classed as improper usage. \boschung\ bears no responsibility whatsoever for any damage or injury caused in such way. The user alone is responsible for all consequences arising from such usage. {\em Appropriate usage also includes compliance with the safety instructions and maintenance schedule included in this User Manual.}


\section{Driving on public roads}

\subsubject{General directions}

In addition to directions in the User Manual, all generally applicable traffic regulations and other applicable regulations relating to accident prevention and environmental protection must be observed.


\subsubject{Passenger seat}

Any passenger traveling on the vehicle must sit on the seat intended for this purpose, i.e. the {\em passenger seat}.


\subsubject{Safety belt}

\startSymList
% \externalfigure [prescription:safety:belt]
\PMbelt
\SymList
The driver and passenger of the \sdeux\ must put on their safety belts when taking their place in the vehicle, as required under the applicable traffic regulations.
\stopSymList


\subsubject{Seeing and being seen}

\startSymList
\externalfigure [travaux_deviation] [width=3.5em]
\SymList
Ensure that you are clearly visible, particularly when on a busy road.

If the vehicle driver does not have a clear enough view when undertaking a specific driving maneuver or deployment operation, he must seek the help of a second person with whom he can maintain constant eye contact.
\stopSymList


\subsubject{Lights and signals}

Depending on the applicable traffic regulations, it may be necessary to keep the vehicle’s headlights and beacon light on during the day.


\subsubject{Using cellphones}

\startSymList
\PPphone
\SymList
It is forbidden to use a cellphone or radio device while driving the vehicle on public roads, unless the vehicle is equipped with a hands||free system.

Using a telephone\index{Safety+Cellphone} at the wheel~– even with a hands||free system~– will always impair a driver’s ability to concentrate on traffic.
\stopSymList


\section{Maintenance instructions}

\subsubject{Maintenance directions}

Before commencing work on the \sdeux, maintenance personnel must read the User Manual, particularly the sections on safety and maintenance.


\subsubject{Qualifications required}

\startSymList
\externalfigure [mecanicienne] [width=3.5em]
\SymList
Only individuals who have completed appropriate training and have the required know||how are permitted to carry out maintenance work on the \sdeux. This applies in particular to work on the engine, braking system, steering, and electrical and hydraulic installations.
\stopSymList


\testpage [6]
\subsubject{Supervision}

\startSymList
\externalfigure [mecanicien_hyerarchie] [width=3.5em]
\SymList
Individuals who are undergoing training~– an internship or apprenticeship~– are permitted to carry out work on the machine only under the supervision of a fully qualified individual. You should conduct random checks to ensure that personnel are complying with directions in the User Manual and safety requirements.
\stopSymList


\subsubject{Welding}

\startSymList
\externalfigure [pince_soudure2] [width=3.5em]
\SymList
The battery and all electronic control units must be disconnected before any welding work is carried out on the bodywork or chassis.
\stopSymList

\subsubject{Vehicle cleaning}

\startSymList
\externalfigure [washer_pressure] [width=3.5em]
\SymList
Before cleaning the \sdeux\ read \about[sec:cleaning] , \atpage[sec:cleaning], particularly the section on cleaning instructions.
\stopSymList


\subsubject{Accessibility of vehicle documentation}

\startSymList
\externalfigure [lecteur_1] [width=3.5em]%\PMrtfm
\SymList
Ensure that the vehicle documentation is always stowed in an easily accessible location in the driver’s cabin during deployments.
\stopSymList


\section{Special instructions}

\subsubject{Vehicle height}

\startSymList
\PPmaxheight
\SymList
When working|/|driving in areas that are not open ground (underground parking lots, subways, power lines, etc.), always ensure there is sufficient clearance for the \sdeux\  (see\in{ Section}[sec:measurement], \atpage[sec:measurement]).
\stopSymList


\subsubject{Stability of the vehicle}

Avoid any maneuver that could compromise the stability of the vehicle. Due to its narrow design and high center of gravity when the dirt hopper is full, the \sdeux\ could tip over when taking bends at high speed.


\subsubject{Unwanted vehicle movement}

On leaving the vehicle, ensure that it is secured against use by unauthorized persons. Always apply the parking brake before exiting the vehicle; if necessary, secure the wheels with blocks.

\startbuffer [prescription:handbrake]
\starttextbackground [CB]
\startPictPar
\PPstop
\PictPar
{\md Apply the parking brake fully!} If you do not, the vehicle could start to move on its own,\index{Parking brake+Hazard potential} even on virtually imperceptible gradients and cause an accident that could result in fatal injuries to third parties.

{\lt Because the vehicle features a hydrostatic drive system, the pressure in the hydraulic circuit gradually declines when the vehicle is stationary, which causes the holding force of the engine to drop. Consequently, it is particularly important to ensure that you always apply the parking brake fully when leaving the vehicle.}
\stopPictPar
\stoptextbackground

\stopbuffer

\getbuffer [prescription:handbrake]


\testpage [6]
\subsubject{Dirt hopper}

\startbuffer [prescription:container:gravity]
\starttextbackground [CB]
\startPictPar
\PHgravite
\PictPar
{\md Accident hazard:}
{\lt When the dirt hopper is raised, the vehicle’s center of gravity is higher. This increases the risk that the vehicle could tip over. Always ensure the vehicle is on even and solid ground before raising the dirt hopper.}
\stopPictPar
\stoptextbackground

\stopbuffer

\getbuffer [prescription:container:gravity]


\startbuffer [prescription:container:tilt]
\starttextbackground [CB]
\startPictPar
\PHcrushing
\PictPar
{\md Accident hazard:}
{\lt Never work underneath the dirt hopper without first fitting the safety props to the hopper’s hydraulic lifting cylinders.}
\stopPictPar
\stoptextbackground

\stopbuffer

\getbuffer [prescription:container:tilt]


\stopcomponent

