\startcomponent c_60_work_s2_095-en
\product prd_ba_s2_095-en


\startchapter [title={The S2 in day||to||day use},
reference={chap:using}]

\setups [pagestyle:marginless]


% \placefig[margin][fig:ignition:key]{Clé de contact}
% {\externalfigure [work:ignition:key]}
\startregister[index][chap:using]{Starting up}

\startsection [title={Starting up},
reference={sec:using:start}]


\startSteps
\item Ensure that the standard checks and maintenance work have been carried out in line with directions.
\item Use the ignition key to start the engine: Switch on the ignition then continue to turn the key in a clockwise direction and hold it in that position until the engine starts.
\stopSteps

\start
\setupcombinations [width=\textwidth]

\placefig[here][fig:select:drive]{Gearshift}
{\startcombination [2*1]
{\externalfigure [work:select:fDrive]}{Gearshift set to \aW{forward}}
{\externalfigure [work:select:rDrive]}{Gearshift set to \aW{reverse}}
\stopcombination}
\stop


\startSteps [continue]
\item When in \aW{driving} mode, turn the gearshift to select a gear:
\startitemize [R]
\item First gear (automatically selected on starting the engine)
\item Second gear
\stopitemize

Or press in the button on the end of the lever to activate/deactivate \aW{working} mode.
\stopSteps

\startbuffer [work:config]
\starttextbackground [FC]
\startPictPar
\PMrtfm
\PictPar
Only first gear is available when in working mode, and the engine runs at 1300\,rpm.

Use the \textSymb{joy_key_engine_increase} and \textSymb{joy_key_engine_decrease}~buttons on the multifunction console to control engine speed.
\stopPictPar
\stoptextbackground
\stopbuffer

\getbuffer [work:config]

\startSteps [continue]
\item Push the gearshift up and forward (driving forward) or up and back (reverse). See figures above.
\item Release the parking brake before accelerating.
\stopSteps

\starttextbackground [FC]
\startPictPar
\PMrtfm
\PictPar
{\md Release the parking brake fully!} An electronic sensor monitors the position of the parking brake lever: If the parking brake has not been fully released, the vehicle’s speed is restricted to 5\,km/h (approx. 3 mph).
\stopPictPar
\stoptextbackground

\startSteps [continue]
\item Slowly depress the gas pedal to get the vehicle moving.
\stopSteps


%% NOTE: New text [2014-04-29]:
\subsection [sSec:suctionClap] {Suction duct flap}

The suction system generates an airflow from the suction mouth or manual suction hose (optional accessory) to the dirt hopper.

A hand||operated flap (\inF[fig:suctionClap], \atpage[fig:suctionClap]) is used to switch the airflow between the suction mouth and manual suction hose.

\placefig [here] [fig:suctionClap] {Suction duct flap}
{\startcombination [2*1]
{\externalfigure [work:suctionClap:open]}{Suction duct open}
{\externalfigure [work:suctionClap:closed]}{Suction duct closed}
\stopcombination}

When operating as normal~– working with the suction mouth~– the suction duct has to be open (switching lever up).

The suction duct must be closed (switching lever down) in order to use the manual suction hose. This redirects the airflow through the manual suction hose.
%% End new text

\stopsection



\startsection [title={Shutting down},
reference={sec:using:stop}]

\index{Shutting down}

\startSteps
\item Apply the parking brake (lever between seats) and move the gearshift to the \aW{neutral} position.
\item Carry out the requisite checks~– daily and, if necessary, weekly checks~– as described on \atpage[table:scheduledaily] .
\stopSteps

\getbuffer [prescription:handbrake]

\stopsection

\page [yes]
\startsection [title={Deploying the sweeping and/or vacuum systems},
reference={sec:using:work}]

\startSteps
\item Start up the vehicle\index{Sweeping} as described in \in{Section}[sec:using:start], \atpage[sec:using:start].
\item Activate\index{Vacuum} \aW{working} mode (button on the end of the gearshift).
\stopSteps

% \getbuffer [work:config]
%% NOTE: outcommented by PB

\startSteps [continue]
\item Press the \textSymb{joy_key_suction_brush}~button to switch on the turbine and brooms.

{\md Variant:} {\lt Press the \textSymb{joy_key_suction}~button to work with the suction mouth only.}

\item Use the \textSymb{joy_key_frontbrush_increase}\textSymb{joy_key_frontbrush_decrease}~buttons on the multifunction console to set the rotary speed of the brooms.

\item Use the relevant joystick to position each broom to achieve the optimum working width.
\stopSteps

\vfill

\start
\setupcombinations [width=\textwidth]

\placefig[here][fig:brush:position]{Positioning the brooms}
{\startcombination [2*1]
{\externalfigure [work:brushes:enlarge]}{Moving brooms in/out}
{\externalfigure [work:brush:left:raise]}{Moving brooms up/down}
\stopcombination}
\stop

\page [yes]


\subsubsubject{The dampening system for brooms and suction duct}

Actuate\index{Sweeping+Dampening system} the \textSymb{temoin_busebalais}~switch between the seats:

{\md Position 1:} The water pump runs automatically as long as the brooms are active.

{\md Position 2:} The water pump runs continuously (useful \eG\ when making adjustments).


\subsubsubject{Large items of waste}

\startSteps [continue]
\item If there is a risk that larger items of waste (\eG\ PET bottles) could block the suction mouth, open\index{Large waste flap} the large waste flap using the buttons on the side of the multifunction console or~– if that is insufficient~– lift the\index{Suction mouth+Large items of waste} suction mouth temporarily.
\stopSteps

\start
\setupcombinations [width=\textwidth]

\placefig[here][fig:suctionMouth:clap]{Dealing with large items of waste}
{\startcombination [2*1]
{\externalfigure [work:suction:open]}{Opening the large waste flap}
{\externalfigure [work:suction:raise]}{Temporarily lifting the suction mouth}
\stopcombination}
\stop

\stopsection


\startsection [title={Emptying the dirt hopper},
reference={sec:using:container}]

\startSteps
\item Drive\index{Dirt hopper+Emptying} the vehicle to a location suitable for emptying the dirt hopper. Ensure the applicable environmental protection regulations are observed.
\item Apply the parking brake and move the gearshift to the \aW{neutral} position
(necessary to release the hopper tipping switch).
% \item Apply the parking brake (necessary to release the hopper tipping switch).
\stopSteps

\getbuffer [prescription:container:gravity]

\startSteps [continue]
\item Unlock and open the hatch on the dirt hopper.
\item Actuate the \textSymb{temoin_kipp2}~switch (center console, between the seats) to tip the dirt hopper.
\item Once the hopper has been emptied, wash the inside of it with a jet of water. You can use the integrated water gun to do this (optional equipment).
\stopSteps

\start
\setupcombinations [width=\textwidth]
\placefig[here][fig:brush:adjust]{Working with the dirt hopper}
{\startcombination [3*1]
{\externalfigure [container:cover:unlock]}{Unlocking the hatch}
{\externalfigure [container:safety:unlocked]}{Safety prop}
{\externalfigure [container:safety:locked]}{Safety prop locked in place}
\stopcombination}
\stop

\startSteps [continue]
\item Check/clean the seals and contact areas of the seals on the hopper, recycling system and suction duct.
\stopSteps

\getbuffer [prescription:container:tilt]

\startSteps [continue]
\item Actuate the \textSymb{temoin_kipp2}~switch to lower the dirt hopper (only after removing the safety props from the hydraulic cylinders).
\item Lock the hopper hatch.
\stopSteps

\stopsection


\startsection [title={Manual suction hose},
reference={sec:using:suction:hose}]

The \sdeux\ can be fitted with an optional\index{Manual suction hose} manual suction hose. It is mounted on the dirt hopper hatch and is easy to operate.

{\sla Requirements:}

The dirt hopper must be fully lowered; the \sdeux\ must be in \aW{working} mode (see \in{Section}[sec:using:start], \atpage[sec:using:start]).

\startfigtext[left][fig:using:suction:hose]{Manual suction hose}
{\externalfigure[work:suction:hose]}
\startSteps
\item Press the \textSymb{temoin_aspiration_manuelle}~switch on the overhead console to activate the suction system.
\item Apply the parking brake fully before leaving the driver’s cabin.
\item Use the suction duct flap to close the suction duct (see \in{§}[sSec:suctionClap], \atpage[sSec:suctionClap]).
\item Holding the mouthpiece, pull the manual suction hose out of its mount and commence clearing work.
\item Once your work is complete, press the \textSymb{temoin_aspiration_manuelle}~switch again to deactivate the suction system.
\stopSteps
\stopfigtext

\stopsection

\page [yes]

\setups[pagestyle:normal]


\startsection [title={High||pressure water gun},
reference={sec:using:water:spray}]

The \sdeux\ can be fitted with an optional\index{water gun} high||pressure water gun. The water gun is mounted on the rear, right||hand maintenance hatch and is connected to a 10||meter hose reel on the opposite side of the vehicle.

Follow the steps below to use the water gun:

{\sla Requirements:}

The freshwater tank must contain sufficient water; the \sdeux\ must be in \aW{working} mode (see \in{Section}[sec:using:start], \atpage[sec:using:start]).

\placefig[margin][fig:using:water:spray]{High||pressure water gun}
{\externalfigure[work:water:spray]}

\startSteps

\item Press the \textSymb{temoin_buse}~switch on the overhead console to activate the high||pressure water pump.
\item Apply the parking brake fully before leaving the driver’s cabin.
\item Open the rear, right||hand maintenance hatch and remove the water gun.
\item Unroll as much of the hose as you need and commence clearing work.
\item Once your work is complete, press the \textSymb{temoin_buse}~switch again to deactivate the high||pressure water pump.
\item Pull on the hose briefly to release the locking mechanism and reel the hose back in.
\item Replace the water gun on its mount and close the maintenance hatch.
\stopSteps

\stopsection

\page [yes]


\setups [pagestyle:marginless]


\startsection [title={Working with the third broom (optional)},
reference={sec:using:frontBrush},
]

\startSteps
\item Start up\index{Sweeping} the vehicle as described in \in{Section}[sec:using:start] \atpage[sec:using:start] .
\item Select\index{3rd broom} \aW{working} mode (button on the end of the gearshift).
\stopSteps

% \getbuffer [work:config]

\startSteps [continue]
\item Ensure that the third broom shows up as activated on the Vpad screen (see \textSymb{vpadFrontBrush} \textSymb{vpadFrontBrushK}, \atpage[vpad:menu]).
\item Press the \textSymb{joy_key_frontbrush_act}~button to activate the hydraulics of the third broom.
\item Press the \textSymb{joy_key_frontbrush_left} or \textSymb{joy_key_frontbrush_right}~button to start the third broom turning in the desired direction.

\item Use the \textSymb{joy_key_frontbrush_increase} and \textSymb{joy_key_frontbrush_decrease}~buttons on the multifunction console to set the rotary speed.

\item Use the joystick to position the broom, as shown in the diagrams below.

\stopSteps

{\md Important:} {\lt Before the side brooms can be positioned, the hydraulics of the third broom must be deactivated by pressing the \textSymb{joy_key_frontbrush_act}~button.}
\vfill

\start
\setupcombinations [width=\textwidth]

\placefig[here][fig:brush:position]{Positioning the third broom}
{\startcombination [2*1]
{\externalfigure [work:frontBrush:move]}{Move up|/|down; left|/|right}
{\externalfigure [work:frontBrush:incline]}{Tip left|/|right; tip forward|/|back}
\stopcombination}
\stop

\stopsection

\stopregister[index][chap:using]

\stopchapter
\stopcomponent

