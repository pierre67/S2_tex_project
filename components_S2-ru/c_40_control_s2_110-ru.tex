\startcomponent c_40_control_s2_110-ru
\product prd_ba_s2_110-ru


\startchapter [title={Элементы управления транспортного средства},
reference={chap:ctrl}]

\setups[pagestyle:marginless]

\placefig[here][fig:ctrl:cab:front]{Органы управления}
{\externalfigure[ctrl:cab:front]}

\startcolumns [n=3]
\startLongleg
\item Рулевая колонка (\in{§}[sec:steeringColumn])
\item Регулировка рулевой колонки
\item Педаль газа и тормоза
\item Бортовой компьютер \Vpad~SN (\inP[sec:vpad])
\item Потолочная панель (\inP[sec:ctrl:aux])
\stopLongleg


\subsubsubject{Дополнительное оснащение}

\startLongleg [continue]
\item Монитор заднего хода
\item Радио/MP3
\stopLongleg
\stopcolumns

\startsection [title={Рулевая колонка},
reference={sec:steeringColumn}]

\subsection{Регулировка рулевой колонки}

\textDescrHead{Наклон рулевой колонки} Нажмите педаль~\Ltwo и одновременно отрегулируйте наклон рулевой колонки. Для блокировки механизма колонки отпустите педаль.

\page[yes]
\setups [pagestyle:normal]


\subsection{Освещение и сигнальные устройства}

\placefig [margin] [fig:column:left] {Многофункциональный рычаг и поворотный переключатель}
{\externalfigure[ctrl:column:left]}

\placefig [margin] [fig:column:right] {Рычаг выбора передачи}
{\externalfigure[ctrl:column:right]}


\subsubsubject{Поворотный переключатель}

\startitemize[width=1.7em]
\sym{\textSymb{com_lowlight}} Ближний свет (повернуть~\TorqueR).
\startitemize
\sym{1} Стояночный свет
\sym{2} Ближний свет
\stopitemize
\stopitemize


\subsubsubject{Многофункциональный рычаг}

\startitemize[width=1.7em]
\sym{\textSymb{com_lowlight}} {[}не используется{]}
\sym{\textSymb{com_light}} Световой сигнал (перевести рычаг вверх на короткое время)
\sym{\textSymb{com_blink}} Индикатор направления движения (перевести рычаг вперед/назад)
\sym{\textSymb{com_claxonArrow}} Звуковой сигнал (нажать кнопку снаружи рычага)
\sym{\textSymb{com_wipper}} Стеклоочистители
\startitemize
\sym{J} Прерывистое включение
\sym{O} Выкл.
\sym{I} 1-я\,передача
\sym{II} 2-я\,передача
\stopitemize
\sym{\textSymb{com_washerArrow}} Стеклоомыватели (нажать на венец на конце рычага).
\stopitemize


\subsubsubject{Рычаг переключения передач}

Функции рычага переключения передач подробно описаны в главе~\about[chap:using], начиная с~\atpage[sec:using:start].

\stopsection

\page [yes]


\startsection [title={Другие функции},
reference={sec:ctrl:add}]


\subsection[sec:ctrl:aux]{Потолочная консоль}

{\sl \index{Потолочная консоль} Потолочная консоль находится спереди на потолке кабины со стороны водителя.}
\placefig [margin] [fig:console:aux] {Потолочная консоль}
{\externalfigure[ctrl:console:aux]}


\placefig [margin] [fig:console:climat] {Система отопления и кондиционер}
{\externalfigure[ctrl:console:climat]}


\startitemize [unpacked][width=1.7em]
\sym{\textBigSymb{temoin_retrochauffant}} Обогрев наружного зеркала
\sym{\textBigSymb{temoin_hazard}} Система аварийной световой сигнализации
\sym{\textBigSymb{temoin_eclairage_L}} Рабочие прожектора
\stopitemize


\subsubsubject{Дополнительное оснащение}

\startLeg [unpacked][width=1.7em]
\sym{\textBigSymb{temoin_buse}} Насос высокого давления для водяного пистолета \crlf {\sl см. \atpage[sec:using:water:spray]}
\sym{\textBigSymb{temoin_aspiration_manuelle}} Турбина для ручного всасывающего шланга \crlf {\sl см. \atpage[sec:using:suction:hose]}
\stopLeg


\subsection[sec:ctrl:climat]{Система отопления и кондиционер}

{\sl Эта консоль\index{Консоль системы отопления} находится на задней стенке кабины между сиденьями.}

\startitemize [unpacked][width=23mm]
\sym{\bf 0\quad I\quad II\quad III} Поворотный переключатель вентилятора
\sym{\externalfigure[tirette_chauffage][height=1em]} Регулятор температуры
\stopitemize


\subsubsubject{Дополнительное оснащение}

\startitemize [unpacked][width=1.7em]
\sym{\textBigSymb{temoin_climat_bk}} Кондиционер
\stopitemize

\page [yes]

\setups [pagestyle:bigmargin]


\subsection[sec:ctrl:central]{Средняя консоль}

{\sl \index{Средняя консоль} Средняя консоль находится между сиденьями.}

\placefig [margin] [fig:console:central] {Средняя консоль}
{\externalfigure[ctrl:console:central]}


\subsubsubject{Смачивание щеток}

\startLeg [unpacked][width=1.7em]
\sym{\textBigSymb{temoin_busebalais}} Насос низкого давления\index{Водяной насос} для системы увлажнения\index{Водяной насос+Befeuchtung} der Besen. (Позиция~1: \aW{автоматически}; позиция~2: \aW{постоянно})
\stopLeg


\subsubsubject{Опрокидывание бункера для отходов}

\setupinmargin[right][style=normal]
\inright{%
\startitemize
\sym{\textSymb{mand_readtheoperatingmanual}} См. инструкцию по использованию стояночного тормоза на \atpage[sec:using:stop].
\stopitemize}

\startLeg [unpacked][width=1.7em]
\sym{\textBigSymb{temoin_kipp2}} Опрокидывание бункера для отходов. Чтобы\index{Бункер для отходов+Опрокидывание} опрокинуть бункер, нужно поставить машину на стояночный тормоз и привести рычаг переключения передач в нейтральное положение.
\stopLeg


\subsubsubject{Аварийное выключение}

\starttextbackground [FC]
\startPictPar
\externalfigure[Emergency_Stop][Pict]
\PictPar
В случае аварии\index{Аварийное выключение} все всасывающие и уборочные устройства можно отключить, нажав кнопку аварийного выключения.
\stopPictPar
\stoptextbackground


\subsection[sec:foot:switch]{Педаль}

\placefig [margin] [fig:foot:switch] {Педаль}
{\vskip 60pt
\externalfigure[work:foot:switch]}

С помощью\index{Педаль} этой педали у основания рулевой колонки (\inF[fig:foot:switch]) можно быстро и просто опустить щетки при необходимости (\eG\ на вершине холма или въезде на тротуар).

\stopsection
\page[yes]
\setups [pagestyle:marginless]


\startsection[title={Многофункциональная консоль},
reference={ctrl:console:middle}]

\startlocalfootnotes

\startfigtext[left]{Многофункциональная консоль}
{\externalfigure[overview:joy:large]}


\subsubsubject{Джойстики}

\textDescrHead{Без передней щетки (или передняя щетка деактивирована):}\crlf
Джойстики управляют одной щеткой каждый независимо друг от друга: подъем/опускание~(\textSymb{joystick_aa}) или влево/вправо~(\textSymb{joystick_gd}). Левый джойстик управляет левой щеткой, правый – правой щеткой.\footnote{Чтобы у машины, оснащенной передней щеткой (опция), изменить положение боковых щеток, переднюю щетку нужно деактивровать (кнопка~\textSymb{joy_key_frontbrush_act}).}

\textDescrHead{С передней щеткой:}
Левым джойстиком можно поднять или опустить переднюю щетку (\textSymb{joystick_aa}), а также переместить ее влево или вправо (\textSymb{joystick_gd}). Правый джойстик предназначен для наклона щетки по продольной~(\textSymb{joystick_aa}) и поперечной оси~(\textSymb{joystick_gd}).

\placelocalfootnotes %[height=\textheight]
\stopfigtext
\stoplocalfootnotes
\vfill

\subsubsubject{Боковые кнопки}

\startcolumns

\startPictList
\VPcltr
\PictList
Круиз-контроль: Повышение настроенной скорости
\stopPictList\vskip -3pt

\startPictList
\VPclbr
\PictList
Круиз-контроль: Уменьшение настроенной скорости
\stopPictList\vskip -3pt

\startPictList
\VPcrtr
\PictList
Поднять патрубок всасывания
\stopPictList

\startPictList
\VPcrbr
\PictList
Опустить патрубок всасывания
\stopPictList\vskip -3pt

\startPictList
\VPcrtf
\PictList
Открыть заслонку для крупного мусора (спереди на патрубке всасывания)
\stopPictList\vskip -3pt

\startPictList
\VPcrbf
\PictList
Закрыть заслонку для крупного мусора
\stopPictList

\stopcolumns


\subsubsubject{Кнопки с пиктограммами}

\startcolumns

\setupparagraphs [SymVpad][n=2,distance=2mm,rule=off,before={\page[preference]},after={\nobreak\hrule\vskip -9pt}]
\setupparagraphs [SymVpad][1][width=3em,inner=\hfill]

\startSymVpad
\externalfigure[joy:stop]
\SymVpad
\textDescrHead{Стоп} Остановить активное устройство:

1\:× нажатие: деактивировать 3-ю\,щетку\crlf
2\:× нажатия: деактивировать все
\stopSymVpad

\startSymVpad
\externalfigure[joy:tempomat]
\SymVpad
\textDescrHead{Круиз-контроль} Установка круиз-контроля на текущую скорость и активация. Для деактивации еще раз нажать кнопку~\textSymb{joy:tempomat} или притормозить. Для ускорения и замедления использовать боковые кнопки.
\stopSymVpad

\startSymVpad
\externalfigure[joy:ftbrs:minus]
\SymVpad
\textDescrHead{Скорость щеток} Уменьшение скорости вращения боковых щеток или передней щетки.
\stopSymVpad

\startSymVpad
\externalfigure[joy:ftbrs:plus]
\SymVpad
\textDescrHead{Скорость щеток} Увеличение скорости вращения боковых щеток или передней щетки.
\stopSymVpad

\startSymVpad
\externalfigure[joy:eng:minus]
\SymVpad
\textDescrHead{Частота вращения двигателя} Уменьшение частоты вращения дизельного двигателя.
\stopSymVpad

\startSymVpad
\externalfigure[joy:eng:plus]
\SymVpad
\textDescrHead{Частота вращения двигателя} Увеличение частоты вращения дизельного двигателя.
\stopSymVpad
\columnbreak

\startSymVpad
\externalfigure[joy:suc]
\SymVpad
\textDescrHead{Система всасывания} Активация системы всасывания: Патрубок всасывания опускается, включаются турбина и водяной насос системы регенерации.\note [recyclingwaterpump]
Чтобы деактивировать систему, нажмите кнопку останова~\textSymb{joy:stop}.
\stopSymVpad

\startSymVpad
\externalfigure[joy:sucbrs]
\SymVpad
\textDescrHead{Уборка/всасывание}Активация системы всасывания и уборки: Патрубок всасывания опускаются, боковые щетки опускаются и устанавливаются в нужное положение, включаются турбина, щетки и водяной насос системы регенерации.\note [recyclingwaterpump]
Чтобы деактивировать систему, нажмите кнопку останова~\textSymb{joy:stop}.
\stopSymVpad

\footnotetext[recyclingwaterpump]{Насос чистой воды также включается, если переключатель~\textBigSymb{temoin_busebalais} установлен в положение \aW{автоматически} (см. \in [sec:ctrl:central] на \atpage [sec:ctrl:central]).}
\startSymVpad
\externalfigure[joy:ftbrs:act]
\SymVpad
\textDescrHead{Передняя щетка активирована} Активация и деактивация передней щетки.
%% NOTE @Andrew: Singular
\stopSymVpad

\startSymVpad
\externalfigure[joy:ftbrs:right]
\SymVpad
\textDescrHead{Передняя щетка слева} Направление вращения для работы с передней щеткой на левой стороне (направление вращения: по часовой стрелке).
\stopSymVpad

\startSymVpad
\externalfigure[joy:ftbrs:left]
\SymVpad
\textDescrHead{Передняя щетка справа} Направление вращения для работы с передней щеткой на правой стороне (направление вращения: против часовой стрелки).
\stopSymVpad

\stopcolumns

\stopsection

\stopchapter

\stopcomponent













