\startsection[title={Система централизованной смазки},
reference={main:graissageCentral}]
%
%
\subsection{Описание управляющего модуля}
%
Машину \sdeux\ можно оснастить\index{Система централизованной смазки} системой централизованной смазки (опция). Система централизованной смазки периодически распределяет смазочное средство по всем точкам смазки.
%
\startfigtext [left] [vogel_affichage] {Модуль индикации}
{\externalfigure[vogel_base2][W50]}
\blank
\startLeg
\item 7-значный дисплей: Значения и рабочее состояние
\item \LED: Перерыв в работе системы (режим ожидания)
\item \LED: Насос в рабочем режиме
\item \LED: Управление системой с помощью переключателя цикла
\item \LED: Контроль системы с помощью переключателя давления
\item \LED: Сообщение об ошибке
\item Кнопки прокрутки:
\startLeg [R]
\item Включить дисплей
\item Показать значения
\item Изменить значения
\stopLeg
\item Кнопка смены режима; подтверждение значений параметров
\item Запуск цикла промежуточной смазки
\stopLeg
\stopfigtext
%
Система централизованной смазки включает смазочный насос, прозрачный резервуар смазки на левой стороне шасси и управляющий модуль в центральном блоке электрической системы.
% \blank
\page [yes]
%
%
\subsubsubject{Индикация и кнопки управляющего модуля}
%
\start
\stdfontsemicn
%
\setupTABLE [frame=off,style={\ssx\setupinterlinespace[line=.93\lH]},background=color,
option=stretch,
split=repeat]
%
\setupTABLE [r] [each] [topframe=on,framecolor=white]
%
\setupTABLE [c] [odd] [backgroundcolor=TableMiddle]
\setupTABLE [c] [even] [backgroundcolor=TableLight]
\setupTABLE [c] [1][width=9mm,style={\bfx\setupinterlinespace[line=.93\lH]}]
\setupTABLE [r] [first] [topframe=off,style={\bfx\setupinterlinespace[line=.93\lH]},
]
%
%
\bTABLE
\bTABLEhead
\bTR\bTD Поз. \eTD
\bTD Светодиод \eTD\bTD Режим индикации \eTD
\bTD Режим программирования \eTD\eTR
\eTABLEhead
%
\bTABLEbody
\bTR\bTD 2 \eTD
\bTD Рабочее состояние {\em Временное отключение}\hskip.5em\null \eTD
\bTD Система находится в режиме ожидания\hskip.5em\null \eTD %
\bTD Длительность прерывания может регулироваться \eTD\eTR
\bTR\bTD 3 \eTD
\bTD Рабочее состояние {\em Контакт} \eTD
\bTD Работает насос\eTD
\bTD Время работы может регулироваться \eTD\eTR
\bTR\bTD 4 \eTD
\bTD Контроль системы {\em CS} \eTD
\bTD С внешним выключателем цикла \eTD
\bTD Режим контроля можно выключить или изменить \eTD\eTR
\bTR\bTD 5 \eTD
\bTD Контроль системы {\em PS} \eTD
\bTD С внешним выключателем, срабатывающим от давления \eTD
\bTD Режим контроля можно выключить или изменить \eTD\eTR
\bTR\bTD 6 \eTD
\bTD Неисправность {\em Сбой} \eTD
\bTD [nc=2] Возникла функциональная неисправность. Причина может выдаваться на дисплей в форме кода ошибки после того, как будет нажата кнопка \textSymb{vogel_DK}. Выполнение функций прекращается. \eTD\eTR
\bTR\bTD 7 \eTD
\bTD Кнопки со стрелками \textSymb{vogelTop} \textSymb{vogelBottom} \eTD
\bTD [nc=2] \items[symbol=R]{Включение дисплея, опрос параметров (режим индикации), регулировка показанного (I) параметра (режим программирования)}
\eTD\eTR
\bTR\bTD 8 \eTD
\bTD Кнопка \textSymb{vogelSet} \eTD
\bTD [nc=2] Переключение между режимами индикации и программирования или подтверждение введенных значений. \eTD\eTR
\bTR\bTD 9 \eTD
\bTD Кнопка \textSymb{vogel_DK} \eTD
\bTD [nc=2] Если инструмент находится в состоянии {\em Пауза}, при нажатии на кнопку запускается цикл промежуточной смазки. После подтверждения сообщения об ошибке, оно удаляется. \eTD\eTR
\eTABLEbody
\eTABLE
\stop
\vfill
%
\startfigtext [left] [vogel_touches]{Модуль индикации}
{\externalfigure[vogel_base][width=50mm]}
\textDescrHead{Режим индикации} Коротко нажать на одну из кнопок со стрелками \textSymb{vogelTop} \textSymb{vogelBottom}, чтобы включить 7-значный дисплей \textSymb{led_huit}. Следующим нажатием на кнопку \textSymb{vogelTop} можно вывести на дисплей различные параметры с их значениями. Режим {\em индикации} распознается по непрерывному свечению \LED\char"2060s (\in{2 – 6, рис.}[vogel_affichage]).
\blank [medium]
\textDescrHead{Режим программирования} Для изменения значений нажмите и держите, по крайней мере 2 секунды кнопку \textSymb{vogelSet}, чтобы переключиться в режим {\em программирования}: Мигают\LED\char"2060s. Коротко нажать на кнопку \textSymb{vogelSet}, чтобы\index{Централизованная смазка+Программирование} переключить индикацию, а затем изменить желаемые значения с помощью кнопок \textSymb{vogelTop} \textSymb{vogelBottom}. Нажать\index{Централизованная смазка+Индикация} кнопку \textSymb{vogelSet} для подтверждения.
\stopfigtext
%
\page [yes]
%
%
\subsection{Подменю в режиме {\em Индикация}}
%
\vskip -9pt
%
\adaptlayout [height=+5mm]
%
\startcolumns[balance=no]\stdfontsemicn
%
\startSymVogel
\externalfigure[vogel_tpa][width=26mm]
\SymVogel
\textDescrHead{Время прерывания [h]} Нажмите кнопку \textSymb{vogelTop}, чтобы отобразить запрограммированные значения.
\stopSymVogel
%
\startSymVogel
\externalfigure[vogel_068][width=26mm]
\SymVogel
\textDescrHead{Остающееся время прерывания [h]} Время, остающееся до следующего цикла смазки.
\stopSymVogel
%
\startSymVogel
\externalfigure[vogel_090][width=26mm]
\SymVogel
\textDescrHead{Общее время прерывания [h]} Общее время отключения между двумя циклами.
\stopSymVogel
%
\startSymVogel
\externalfigure[vogel_tco][width=26mm]
\SymVogel
\textDescrHead{Время смазки [min]} Нажмите кнопку \textSymb{vogelTop}, чтобы отобразить запрограммированные значения.
\stopSymVogel
%
\startSymVogel
\externalfigure[vogel_tirets][width=26mm]
\SymVogel
\textDescrHead{Инструмент в режиме ожидания} Индикация невозможна, так как инструмент находится в режиме ожидания (прерывание).
\stopSymVogel
%
\startSymVogel
\externalfigure[vogel_026][width=26mm]
\SymVogel
\textDescrHead{Время смазки [min]} Длительность процесса смазки.
\stopSymVogel
%
\startSymVogel
\externalfigure[vogel_cop][width=26mm]
\SymVogel
\textDescrHead{Контроль системы} Нажмите \textSymb{vogelTop}, чтобы вывести на дисплей запрограммированные значения.
\stopSymVogel
%
\startSymVogel
\externalfigure[vogel_off][width=26mm]
\SymVogel
\textDescrHead{Режим контроля} \hfill PS: манометрический выключатель;\crlf
CS: переключатель циклов; OFF: деактивирован.
\stopSymVogel
%
\startSymVogel
\externalfigure[vogel_0h][width=26mm]
\SymVogel
\textDescrHead{Рабочие часы} Нажмите \textSymb{vogelTop}, чтобы отобразить значение по частям.
\stopSymVogel
%
\startSymVogel
\externalfigure[vogel_005][width=26mm]
\SymVogel
\textDescrHead{Часть 1: 005} Рабочее время отображается двумя частями; для перехода к части 2 нажмите \textSymb{vogelTop}.
\stopSymVogel
%
\startSymVogel
\externalfigure[vogel_338][width=26mm]
\SymVogel
\textDescrHead{Часть 2: 33,8} 2-я часть числа 33,8; в совокупности получается рабочее время 533,8 ч.
\stopSymVogel
%
\startSymVogel
\externalfigure[vogel_fh][width=26mm]
\SymVogel
\textDescrHead{Время неисправности} Нажмите \textSymb{vogelTop}, чтобы отобразить значение по частям.
\stopSymVogel
%
\startSymVogel
\externalfigure[vogel_000][width=26mm]
\SymVogel
\textDescrHead{Часть 1: 000} Время сбоя отображается двумя частями;\crlf
к части 2 посредством \textSymb{vogelTop}.
\stopSymVogel
%
\startSymVogel
\externalfigure[vogel_338][width=26mm]
\SymVogel
\textDescrHead{Часть 2: 33,8} 2-ячасть числа 33,8; в совокупности получается время сбоя 33,8 ч.
\stopSymVogel
%
\stopcolumns
\stopsection
\page [yes]
