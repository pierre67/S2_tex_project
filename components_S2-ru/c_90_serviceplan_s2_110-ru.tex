\startcomponent c_90_serviceplan_s2_110-ru
\product prd_ba_s2_110-ru


\chapter[sec:schedule]{План техобслуживания}

\startregister[index][wartplan]{План техобслуживания+Машина}


\section{Общие замечания}

\placeNote[][service_picto]{}{%
\noteF
\starttextrule{\Pmtcheck\enskip Контроль\enskip}
Контроль, который может проводиться\index{Квалификация+Персонал техобслуживания} производственным персоналом транспортного средства без посторонней помощи. В любом случае, требуются базовые знания об автомобилях.
\stoptextrule \blank[line]

\starttextrule{\Pmtpro\enskip Сервисные работы\enskip}
Регулярные работы по обслуживанию, которые должны проводиться в рамках ухода. Техобслуживание должно проводиться квалифицированным персоналом авторизованной мастерской с использованием требуемых инструментов.
\stoptextrule \blank[line]

\starttextrule{\Pmtspecial\enskip Специальные работы\enskip}
Специальные\index{Специальные работы} Специальные работы по обслуживанию, которые могут выполняться лишь работником, прошедшим специальное, признаваемое гарантийной службой \BoschungNote.
\stoptextrule \blank[line]

\starttextrule{\Pmtvisual\enskip Визуальный осмотр\enskip}
Проверка\index{Визуальный контроль} путем визуального контроля. О выявленных неисправностях и повреждениях следует сообщать сотруднику, ответственному за техобслуживание.
\stoptextrule
}

При всех работах по техобслуживанию следует различать три уровня квалификации (требования к персоналу и оснащению) и четыре типа техобслуживания (вида работ).

\startTwoPar
{\em Уровни квалификации:} \par \blank [medium]
\start \setupwhitespace [none]
\symDescr{\textSymb{maint:check}} Контроль \par
\symDescr{\textSymb{maint:pro}} Сервисные работы \par
\symDescr{\textSymb{maint:special}} Специальные работы \par
\stop
\TwoPar
{\em Типы техобслуживания:} \par \blank [medium]
\start \setupwhitespace [none]
\symDescr{\textSymb{maint:visual}} Визуальный контроль \par
\symDescr{\textSymb{maint:function}} Контроль функционирования \par
\symDescr{\textSymb{maint:level}} Контроль уровня наполнения \par
\symDescr{\textSymb{maint:exchange}} Замена рабочих материалов \par
\symDescr{\textSymb{maint:generic}} Прочие работы \par
\stop
\stopTwoPar

\handItem{См. пояснения в крайних столбцах.}


\placeNote[][service_picto]{}{%
\noteF
\starttextrule{\Pmtfunction\enskip Контроль функционирования\enskip}
Проверки, которые\index{Контроль+Функция} выходят за рамки осмотра, \eG\ функциональный контроль тормозов, проверка состояния узлов, ручной контроль креплений.
\stoptextrule \blank[big]

\starttextrule{\Pmtlevel\enskip Контроль уровня наполнения\enskip}
Контроль\index{Контроль+Уровень наполнения} уровня наполнения включает проверку~– визуально или с помощью щупа~– и, при необходимости, наполнение соответствующей рабочей жидкостью. {\em К этому относится и система централизованной смазки.} Прочитайте также раздел \about[sec:liqquantities], где вы можете найти данные по качеству и количеству используемых рабочих материалов.
\stoptextrule \blank[big]

\starttextrule{\Pmtexchange\enskip Замена рабочей среды\enskip}
Это\index{Рабочие материалы+Замена} включает замену рабочего материала и контроль уровня наполнения. Данные по качеству и количеству рабочих материалов можно найти в \atpage[sec:liqquantities].
\stoptextrule \blank[big]

\starttextrule{\Pmtgeneric\enskip Прочие работы\enskip}
Различные работы по техобслуживанию, выполняемые надлежащим образом в соответствии с инструкцией.
\stoptextrule
}%

\blank[big]
\starttextbackground[CB]
\setupparagraphs [PictPar][1][width=6em,inner=\hfill]
\startPictPar\PMgeneric~\Penvironment\PictPar
При выполнении любых операций контроля и техобслуживания следует соблюдать правила техники безопасности и охраны окружающей среды. \BosFull{boschung} не несет никакой ответственности за травмы и материальный ущерб, возникшие вследствие несоблюдения этих предписаний.
\stopPictPar
\stoptextbackground
\page


%%%%%%%%%%%%%%%%%%%%%%%%%%%%%%%%%%%%%%%%%%%%%%%%%%%%%%%%%%%%%%%%%%%%%%%%%%%%%%
\section{План технического обслуживания машины}

Обслуживание\index{Контроль, периодический} машины состоит из регулярных работ по техобслуживанию, периодических проверок (контроля), а также разового сервиса\footnote{Все указанные данные в часах следует понимать, как рабочие часы}:

\start

\stdfontcond

\starttextbackground[FC]
\starttabulate [|w(6mm)|w(26mm)B|w(18mm)B|p({\dimexpr\textwidth-(50mm+3.5em)\relax})|]
\NC \Pmtpro\NC Работы по \NC 50\,ч\NC первое сервисное обслуживание через 50\,ч\NC \NR
\NC\NC техобслуживанию\NC 600\,ч\NC Регулярное обслуживание через каждые 600\,ч / 12 месяцев\NC \NR
\NC\NC\NC 1200\,ч\NC Регулярное обслуживание через каждые 1200\,ч / 2 года\NC \NR
\NC\NC\NC 2400\,ч \NC Регулярное обслуживание через каждые 2400\,ч / 4 года\NC \NR
\NC\NC\NC 4800\,ч \NC Регулярное обслуживание через каждые 4800\,ч / 8 года\NC \NR
\NC \Pmtcheck\NC Список проверок\NC Ежедневно \NC В течение всего сезона работ \NC \NR
\NC\NC\NC Еженедельно\NC В течение всего сезона работ\NC\NR

\stoptabulate
\stoptextbackground

\stop

\blank [big]

Нижеследующий план техобслуживания относится к базовому транспортному средству.
Соблюдайте также план техобслуживания агрегатов (\eG\ уборочной машины, бункера) со \at{страницы}[sec:schedaggr].

При двойном указании периодичности обслуживания (\eG\ \quotation{каждые 600\,ч~/ 12~месяцев})
приоритет имеет срок, наступающий раньше.

Отдельные планы техобслуживания~– за исключением разового обслуживание после 50~часов~–
должны выполняться нарастающим итогом: Через каждые 1200~рабочих часов должно проводиться
600-часовое обслуживание {\em и} 1200-часовое обслуживание и т.д.

\page [yes]

\setup[pagestyle:marginless]

\start
\setup[tbl:schedule]

\subsection[table:scheduledaily]{Ежедневный контроль}

\bTABLE
\bTABLEhead
\bTR \bTD Вид\eTD \bTD Ежедневный контроль\eTD \bTD \Tcheck \eTD \bTD Ссылка\eTD \eTR
\eTABLEhead
\bTABLEbody
\bTR \bTD \Tgen \eTD \bTD Очистить транспортное средство\eTD \bTD \Tcheck \eTD \bTD \inP[sec:cleaning]\eTD \eTR
\bTR \bTD \Tvis \eTD \bTD Проверить транспортное средство на предмет возможных повреждений\eTD \bTD \Tcheck \eTD \bTD \emptY\eTD \eTR
\bTR \bTD \Tvis \eTD \bTD Проверить на наличие течей\eTD \bTD \Tcheck \eTD \bTD \emptY\eTD \eTR
\bTR \bTD \Tlev \eTD \bTD Проверить уровень масла в дизельном двигателе (щупом!)\eTD \bTD \Tcheck \eTD \bTD \inP[ssSec:vw:oilLevel]\eTD \eTR
\bTR \bTD \Tlev \eTD \bTD Проверить уровень охлаждающей жидкости в дизельном двигателе\eTD \bTD \Tcheck \eTD \bTD \inP[sSec:vw:cooling] \eTD \eTR
\bTR \bTD \Tlev \eTD \bTD Проверить уровень гидравлической жидкости (смотровое стекло на бачке)\eTD \bTD \Tcheck \eTD \bTD \inP[sec:hydraulic]\eTD \eTR
\bTR \bTD \Tlev \eTD \bTD Проверить уровень топлива\eTD \bTD \Tcheck \eTD \bTD \emptY\eTD \eTR
\bTR \bTD \Tlev \eTD \bTD Проверить уровень жидкости в стеклоомывателе\eTD \bTD \Tcheck \eTD \bTD \inP[sec:liqquantities]\eTD \eTR
\bTR \bTD \Tfun \eTD \bTD Проверить работоспособность индикаторов и освещения приборной панели и панели с органами контроля \eTD \bTD \Tcheck \eTD \bTD \emptY\eTD \eTR
\bTR \bTD \Tfun \eTD \bTD Проверить работоспособность стояночного тормоза\eTD \bTD \Tcheck \eTD \bTD \emptY\eTD \eTR
\bTR \bTD \Tfun \eTD \bTD Функциональный контроль оборудования освещения и сигнализации\eTD \bTD \Tcheck \eTD \bTD \inP[sec:lighting]\eTD \eTR
\eTABLEbody
\eTABLE

\testpage [8]
\subsection[table:scheduleweekly]{Еженедельный контроль}

\bTABLE
\bTABLEhead
\bTR \bTD Вид \eTD \bTD Еженедельный контроль \eTD \bTD \Tcheck \eTD \bTD Ссылка \eTD \eTR
\eTABLEhead
\bTABLEbody
\bTR \bTD \Tfun \eTD \bTD Проверить давление в шинах (см. на заводской табличке колес транспортного средства)\eTD \bTD \Tcheck \eTD \bTD \inP[sec:pneumatiques]\eTD \eTR
\bTR \bTD \Tgen \eTD \bTD Проверить патрон воздушного фильтра и, при необходимости, очистить\eTD \bTD \Tcheck \eTD \bTD \inP[sSec:vw:airFilter]\eTD \eTR
\bTR \bTD \Tgen \eTD \bTD Очистите фильтр для пресной воды (позади кабиной водителя) \eTD \bTD \Tcheck \eTD \bTD \emptY \eTD \eTR
\bTR \bTD \Tgen \eTD \bTD Нанести смазку во всех точках (шасси, шарнирное сочленение) \eTD \bTD \Tcheck \eTD \bTD \inP[sec:grasing:plan] \eTD \eTR
\eTABLEbody
\eTABLE
%%%%%%%%%%%%%%%%%%%%%%%%%%%%%%%%%%%%%%%%%%%%%%%%%%%%%%%%%%%%%%%%%%%%%%%%%%%%%%


\subsection [sec:50h]{Обслуживание через 50\,ч~– один раз}

\bTABLE
\bTABLEhead
\bTR \bTD Вид\eTD \bTD Обслуживание после 50\,ч~– один раз\eTD \bTD Q.\eTD \bTD Ссылка\eTD \eTR
\eTABLEhead
\bTABLEbody
\bTR \bTD \Tlev \eTD \bTD Проверить уровень охлаждающей жидкости в дизельном двигателе\eTD \bTD \Tcheck \eTD \bTD \inP[sSec:vw:cooling]\eTD \eTR
\bTR \bTD \Tgen \eTD \bTD Очистить патрон воздушного фильтра (при необходимости, заменить\eTD \bTD \Tcheck \eTD \bTD \inP[sSec:vw:airFilter]\eTD \eTR
\bTR \bTD \Tgen \eTD \bTD Проверить натяжение поликлинового ремня дизельного двигателя, при необходимости подтянуть\eTD \bTD \Tpro\eTD \bTD \inP[sSec:vw:belt]\eTD \eTR
\bTR \bTD \Tgen \eTD \bTD Заменить фильтр в обратной магистрали гидросистемы и всасывающий фильтр\eTD \bTD \Tcheck\eTD \bTD \inP[sec:hydraulic]\eTD \eTR
\bTR \bTD \Tlev \eTD \bTD Проверить уровень гидравлической жидкости (смотровое стекло на бачке)\eTD \bTD \Tcheck \eTD \bTD \inP[sec:hydraulic]\eTD \eTR
\bTR \bTD \Tlev \eTD \bTD Система централизованной смазки (опция): Проверить запас и консистенцию смазки\eTD \bTD \Tpro\eTD \bTD \inP[main:graissageCentral]\eTD \eTR
\bTR \bTD \Tgen \eTD \bTD Нанести смазку во всех точках (шасси, шарнирное сочленение)\eTD \bTD \Tcheck \eTD \bTD \inP[sec:grasing:plan] \eTD \eTR
\bTR \bTD \Tlev \eTD \bTD Проверить уровень жидкости для омывания стекол\eTD \bTD \Tcheck \eTD \bTD \inP[sec:liqquantities]\eTD \eTR
\bTR \bTD \Tfun \eTD \bTD Проверить крепления кабины водителя\eTD \bTD \Tcheck \eTD \bTD \emptY \eTD \eTR
\bTR \bTD \Tfun \eTD \bTD Проверить крепеж двигателя на шасси\eTD \bTD \Tcheck \eTD \bTD \emptY\eTD \eTR
\bTR \bTD \Tfun \eTD \bTD Проверить крепеж насосов на двигателе\eTD \bTD \Tcheck \eTD \bTD \emptY\eTD \eTR
\bTR \bTD \Tfun \eTD \bTD Проверить крепления комбинированного охладителя\eTD \bTD \Tcheck \eTD \bTD \emptY\eTD \eTR
\bTR \bTD \Tfun \eTD \bTD Проверить крепеж осей\eTD \bTD \Tcheck \eTD \bTD \emptY\eTD \eTR
\bTR \bTD \Tfun \eTD \bTD Проверить колеса\index{Момент затяжки+Колеса} на предмет прочной посадки (момент затяжки: 180\,Нм)\eTD \bTD \Tcheck \eTD \bTD \emptY\eTD \eTR
\bTR \bTD \Tfun \eTD \bTD Проверить давление в шинах (см. на заводской табличке колес транспортного средства)\eTD \bTD \Tcheck \eTD \bTD \inP[sec:pneumatiques]\eTD \eTR
\bTR \bTD \Tgen \eTD \bTD Проверить/отрегулировать ход рычага стояночного тормоза (Потяните до 5 зуба)\eTD \bTD \Tpro \eTD \bTD \emptY \eTD \eTR
\bTR \bTD \Tgen \eTD \bTD Проверить состояние батареи; очистить полюса/клеммы\eTD \bTD \Tcheck \eTD \bTD \inP[sec:battcheck] \eTD \eTR
\bTR \bTD \Tgen \eTD \bTD Проверить настройку фар согласно ПДД, отрегулировать при необходимости\eTD \bTD \Tcheck\eTD \bTD \inP[sec:lighting]\eTD \eTR
\bTR \bTD \Tgen \eTD \bTD Смазать сердечники катушек магнитных клапанов медной смазкой\eTD \bTD \Tspecial \eTD \bTD \inP[sec:hydraulic]\eTD \eTR
\bTR \bTD \Tgen \eTD \bTD Просмотреть память событий (Vpad и блока управления двигателем); при необходимости – устранить причины ошибок\eTD \bTD \Tcheck \eTD \bTD \inP[sSec:vw:faultMemory], \inP[vpad:error] \eTD \eTR
\eTABLEbody
\eTABLE

%%%%%%%%%%%%%%%%%%%%%%%%%%%%%%%%%%%%%%%%%%%%%%%%%%%%%%%%%%%%%%%%%%%%%%%%%%%%%%


\subsection {Обслуживание через 600\,ч / 12 месяцев}

\bTABLE
\bTABLEhead
\bTR \bTD Вид\eTD \bTD Обслуживание через каждые 600\,ч / 12 месяцев\eTD \bTD Q.\eTD \bTD Ссылка\eTD \eTR
\eTABLEhead
\bTABLEbody
\bTR \bTD \Tchg \eTD \bTD Заменить масло в дизельном двигателе\eTD \bTD \Tpro\eTD \bTD \inP[ssSec:vw:oilDraining]\eTD \eTR
\bTR \bTD \Tgen \eTD \bTD Заменить масляный фильтр двигателя\eTD \bTD \Tpro\eTD \bTD \inP[ssSec:vw:oilFilter]\eTD \eTR
\bTR \bTD \Tgen \eTD \bTD Заменить топливный фильтр\eTD \bTD \Tpro\eTD \bTD \inP[ssSec:vw:fuelFilter]\eTD \eTR
\bTR \bTD \Tgen \eTD \bTD Проверить герметичность двигателя и компонентов в моторном отсеке\eTD \bTD \Tpro \eTD \bTD \emptY \eTD \eTR
\bTR \bTD \Tgen \eTD \bTD Проверить герметичность и крепление выхлопной системы\eTD \bTD \Tpro \eTD \bTD \emptY \eTD \eTR
\bTR \bTD \Tgen \eTD \bTD Проверить состояние и натяжение поликлинового ремня дизельного двигателя, при необходимости – подтянуть или заменить\eTD \bTD \Tpro\eTD \bTD \inP[sSec:vw:belt]\eTD \eTR
\bTR \bTD \Tlev \eTD \bTD Проверить уровень охлаждающей жидкости в дизельном двигателе\eTD \bTD \Tcheck\eTD \bTD \inP[sSec:vw:cooling]\eTD \eTR
\bTR \bTD \Tgen \eTD \bTD Заменить патрон воздушного фильтра\eTD \bTD \Tcheck\eTD \bTD \emptY\eTD \eTR
\bTR \bTD \Tgen \eTD \bTD Заменить фильтр в обратной магистрали гидросистемы и всасывающий фильтр\eTD \bTD \Tcheck\eTD \bTD \inP[sec:hydraulic]\eTD \eTR
\bTR \bTD \Tlev \eTD \bTD Проверить уровень гидравлической жидкости в баке\eTD \bTD \Tcheck\eTD \bTD \inP[sec:hydraulic]\eTD \eTR
% \bTR \bTD \Tlev \eTD \bTD Система централизованной смазки (опция): Проверить запас и консистенцию смазки\eTD \bTD \Tpro\eTD \bTD \inP[main:graissageCentral]\eTD \eTR
\bTR \bTD \Tgen \eTD \bTD Нанести смазку во всех точках (шасси, шарнирное сочленение)\eTD \bTD \Tcheck \eTD \bTD \inP[sec:grasing:plan] \eTD \eTR
\bTR \bTD \Tlev \eTD \bTD Проверить уровень жидкости для омывания стекол\eTD \bTD \Tcheck\eTD \bTD \inP[sec:liquids]\eTD \eTR
\bTR \bTD \Tfun \eTD \bTD Проверить крепления кабины водителя\eTD \bTD \Tcheck \eTD \bTD \emptY \eTD \eTR
\bTR \bTD \Tfun \eTD \bTD Проверить крепеж двигателя на шасси\eTD \bTD \Tcheck \eTD \bTD \emptY\eTD \eTR
\bTR \bTD \Tfun \eTD \bTD Проверить крепеж насосов на двигателе\eTD \bTD \Tcheck \eTD \bTD \emptY\eTD \eTR
\bTR \bTD \Tfun \eTD \bTD Проверить крепления комбинированного охладителя\eTD \bTD \Tcheck \eTD \bTD \emptY\eTD \eTR
\bTR \bTD \Tfun \eTD \bTD Проверить крепеж осей\eTD \bTD \Tcheck \eTD \bTD \emptY\eTD \eTR
\bTR \bTD \Tfun \eTD \bTD Рулевое управление: Проверка зазора шарового шарнира \eTD \bTD \Tcheck \eTD \bTD \emptY \eTD \eTR
\bTR \bTD \Tfun \eTD \bTD Рулевое управление: проверить герметичность баллона \eTD \bTD \Tcheck \eTD \bTD \emptY \eTD \eTR
\bTR \bTD \Tfun \eTD \bTD Шарнирная система рулевого: проверить винты, пластины и блокировки смазки линии \eTD \bTD \Tcheck \eTD \bTD \emptY \eTD \eTR
\bTR \bTD \Tgen \eTD \bTD Проверить/отрегулировать ход рычага стояночного тормоза (Потяните до 5 зуба)\eTD \bTD \Tpro \eTD \bTD \emptY \eTD \eTR
\bTR \bTD \Tgen \eTD \bTD Проверить/очистить тормозные барабаны и колодки; очистить тормозной механизм\eTD \bTD \Tpro \eTD \bTD \inP[sec:brake] \eTD \eTR
\bTR \bTD \Tfun \eTD \bTD Проверить колеса\index{Момент затяжки+Колеса} на предмет прочной посадки (момент затяжки: 180\,Нм)\eTD \bTD \Tcheck \eTD \bTD \emptY\eTD \eTR
\bTR \bTD \Tfun \eTD \bTD Проверить давление в шинах (см. на заводской табличке колес транспортного средства)\eTD \bTD \Tcheck \eTD \bTD \inP[sec:pneumatiques]\eTD \eTR
\bTR \bTD \Tgen \eTD \bTD Проверить состояние батареи; очистить полюса/клеммы\eTD \bTD \Tcheck \eTD \bTD \inP[sec:battcheck] \eTD \eTR
\bTR \bTD \Tgen \eTD \bTD Проверить настройку фар согласно ПДД, отрегулировать при необходимости\eTD \bTD \Tcheck\eTD \bTD \inP[sec:lighting]\eTD \eTR
\bTR \bTD \Tgen \eTD \bTD Фильтры кабины: удалите и проверьте оба фильтра, очистите или замените их при необходимости \eTD \bTD \Tcheck \eTD \bTD \emptY \eTD \eTR
\bTR \bTD \Tgen \eTD \bTD Смазать сердечники катушек магнитных клапанов медной смазкой\eTD \bTD \Tpro \eTD \bTD \inP[sec:hydraulic]\eTD \eTR
% \bTR \bTD \Tgen \eTD \bTD Проверить антикоррозионную защиту, при необходимости – исправить/заменить\eTD \bTD \Tspecial \eTD \bTD \inP[sec:anticorrosion]\eTD \eTR
\bTR \bTD \Tgen \eTD \bTD Просмотреть память событий (Vpad и блока управления двигателем); при необходимости – устранить причины ошибок\eTD \bTD \Tcheck \eTD \bTD \inP[sSec:vw:faultMemory], \inP[vpad:error] \eTD \eTR
\eTABLEbody
\eTABLE

%%%%%%%%%%%%%%%%%%%%%%%%%%%%%%%%%%%%%%%%%%%%%%%%%%%%%%%%%%%%%%%%%%%%%%%%%%%%%%
\subsection {Обслуживание через каждые 1200\,ч / 2 года}
\bTABLE
\bTABLEhead
\bTR \bTD Вид\eTD \bTD Обслуживание через каждые 1200\,ч / 2 года\eTD \bTD Q.\eTD \bTD Ссылка\eTD \eTR
\eTABLEhead
\bTABLEbody
\bTR \bTD \Tchg \eTD \bTD Заменить гидравлическое масло (в баке)\eTD \bTD \Tpro \eTD \bTD \inP[sec:hydraulic] \eTD \eTR
\bTR \bTD \Tgen \eTD \bTD Заменить хладагент (R134a) кондиционера\eTD \bTD \Tspecial \eTD \bTD \emptY\eTD \eTR
\eTABLEbody
\eTABLE

%%%%%%%%%%%%%%%%%%%%%%%%%%%%%%%%%%%%%%%%%%%%%%%%%%%%%%%%%%%%%%%%%%%%%%%%%%%%%%
\subsection {Обслуживание через каждые 2400\,ч / 4 года}
\bTABLE

\bTABLEhead
\bTR \bTD Вид\eTD \bTD Обслуживание через каждые 2400\,ч / 4 года\eTD \bTD Q.\eTD \bTD Ссылка\eTD \eTR
\eTABLEhead
\bTABLEbody
\bTR \bTD \Tgen \eTD \bTD Заменить зубчатый ремень дизельного двигателя \eTD \bTD \Tspecial \eTD \bTD \emptY \eTD \eTR
\eTABLEbody
\eTABLE

%%%%%%%%%%%%%%%%%%%%%%%%%%%%%%%%%%%%%%%%%%%%%%%%%%%%%%%%%%%%%%%%%%%%%%%%%%%%%%

\subsection {Обслуживание через каждые 4800\,ч / 8 года}
\bTABLE

\bTABLEhead
\bTR \bTD Вид\eTD \bTD Обслуживание через каждые 4800\,ч / 8 года\eTD \bTD Q.\eTD \bTD Ссылка\eTD \eTR
\eTABLEhead
\bTABLEbody
\bTR \bTD \Tgen \eTD \bTD Проверить гидравлические шланги, при необходимости – заменить\eTD \bTD \Tpro \eTD \bTD \inP[sec:hydraulic] \eTD \eTR
\bTR \bTD \Tgen \eTD \bTD Заменить водяной насос (одновременно с зубчатым ремнем) \eTD \bTD \Tspecial \eTD \bTD \emptY \eTD \eTR
\eTABLEbody
\eTABLE

\stopregister[index][wartplan]

\page [yes]


\section[sec:schedaggr]{Обслуживание агрегатов}

Обслуживание\startregister[index][wartplanAgg]{План техобслуживания+Агрегаты} агрегатов включает регулярные работы по техобслуживанию\index{Периодический контроль} а также ежедневные и еженедельные проверки:

\starttextbackground[FC]
\starttabulate [|w(6mm)|w(31mm)B|w(22mm)B|p({\dimexpr\textwidth-(59mm+3.5em)\relax})|]
\NC \Pmtpro\NC Работы по \NC 50\,ч\NC первое сервисное обслуживание через 50\,ч\NC \NR
\NC\NC техобслуживанию\NC 600\,ч\NC Регулярное обслуживание через каждые 600\,ч / 12 месяцев\NC \NR
\NC \Pmtcheck\NC Список проверок\NC Ежедневно \NC В течение всего сезона работ \NC \NR
\NC\NC\NC Еженедельно\NC В течение всего сезона работ\NC\NR
\stoptabulate

\stoptextbackground
\blank [big]

Нижеследующий план техобслуживания относится к стандартным агрегатам, которыми обычно оснащается \sdeux\. Для специальных агрегатов, которые не описаны в данном руководстве, следует соблюдать план техобслуживания соответствующего агрегата.


\subsection[table:scheduledaily]{Ежедневный контроль}

\bTABLE
\bTABLEhead
\bTR \bTD Вид\eTD \bTD Ежедневный контроль\eTD \bTD Q.\eTD \bTD Ссылка\eTD \eTR
\eTABLEhead
\bTABLEbody
\bTR \bTD \Tgen \eTD \bTD Очистить бункер и систему регенерации \eTD \bTD \Tcheck \eTD \bTD \inP[sec:cleaning] \eTD \eTR
\bTR \bTD \Tgen \eTD \bTD Очистить патрубок и канал всасывания \eTD \bTD \Tcheck \eTD \bTD \inP[sec:cleaning] \eTD \eTR
\eTABLEbody
\eTABLE

\subsection[table:scheduledaily]{Еженедельный контроль}

\bTABLE
\bTABLEhead
\bTR \bTD Тип \eTD \bTD Еженедельный контроль \eTD \bTD Q. \eTD \bTD Ссылка \eTD \eTR
\eTABLEhead
\bTABLEbody
\bTR \bTD \Tgen \eTD \bTD Проверить износ линейной и передней щеток (опция) \eTD \bTD \Tcheck \eTD \bTD \emptY \eTD \eTR
\bTR \bTD \Tgen \eTD \bTD Проверить заслонку и резиновую кромку патрубка всасывания \eTD \bTD \Tcheck \eTD \bTD \emptY \eTD \eTR
\bTR \bTD \Tgen \eTD \bTD Нанести смазку во всех точках (бункер, щетки, патрубок всасывания) \eTD \bTD \Tcheck \eTD \bTD \inP[sec:grasing:plan] \eTD \eTR
\eTABLEbody
\eTABLE


\subsection {Обслуживание через 50\,ч~– один раз}

\bTABLE
\bTABLEhead
\bTR \bTD
Art \eTD \bTD Обслуживание после 50\,ч~– разовое \eTD \bTD Q. \eTD \bTD Ссылка \eTD \eTR
\eTABLEhead
\bTABLEbody
\bTR \bTD \Tfun \eTD \bTD Проверить крепеж щеток \eTD \bTD \Tcheck \eTD \bTD \emptY \eTD \eTR
\bTR \bTD \Tgen \eTD \bTD Отрегулировать заслонку и резиновую кромку патрубка всасывания \eTD \bTD \Tpro \eTD \bTD \emptY \eTD \eTR
\eTABLEbody
\eTABLE


\subsection {Обслуживание через 600\,ч/12 месяцев}


\bTABLE
\bTABLEhead
\bTR \bTD Вид \eTD \bTD Обслуживание через 600\,ч/12 месяцев\eTD \bTD Q. \eTD \bTD Ссылка \eTD \eTR
\eTABLEhead
\bTABLEbody
\bTR \bTD \Tfun \eTD \bTD Проверить крепеж щеток \eTD \bTD \Tcheck \eTD \bTD \emptY \eTD \eTR
\bTR \bTD \Tgen \eTD \bTD Отрегулировать заслонку и резиновую кромку патрубка всасывания \eTD \bTD \Tpro \eTD \bTD \inP[sec:main:suctionMouth] \eTD \eTR
\bTR \bTD \Tchg \eTD \bTD Замена масла водяного насоса высокого давления (дополнительное оборудование) \eTD \bTD \Tpro \eTD \bTD \emptY \eTD \eTR
\eTABLEbody
\eTABLE

\stopregister[index][wartplanAgg]
\stop


\stopcomponent
% vim: fdm=indent

