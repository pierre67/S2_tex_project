\startcomponent c_20_prescriptions_s2_110-ru
\product prd_ba_s2_110-ru

\chapter [safety:risques] {Правила техники безопасности}

\setups [pagestyle:marginless]


\section{Основные инструкции}

\subsubject{Законные основания}

Несчастные случаи могут привести к серьезным последствиям как для работодателя, так и для сотрудников. Мы хотим еще раз напомнить об их обязанностях:\note[prescription:user:right].

Прежде чем поручить одному из сотрудников управление уборочной машиной, работодатель обязан принять во внимание следующее:

\startSteps
\item Каждый водитель машины должен обладать квалификацией, необходимой для управления машиной. Требуется наличие подтверждающего документа.
\item Каждый водитель машины должен обладать формальным допуском. Он может быть выдан исключительно при соблюдении следующих трех условий.
\startitemize [2]
\item Сотрудник прошел медицинскую проверку профпригодности у врача предприятия.
\item Сотрудник получил от своего начальника всю информацию о местах работы и знает все правила техники безопасности, относящиеся к месту эксплуатации машины.
\item Сотрудник прошел проверку профпригодности, подтверждающую наличие знаний, которые необходимы для управления машиной.
\stopitemize
\stopSteps

Если максимальная скорость машины превышает 25\,км/ч\note[prescription:user:right], машина должна получить официальный допуск к эксплуатации, а водитель должен обладать правами следующей категории:
\startitemize
\item водительские права категории B\note[prescription:lisence] для транспортных средств с максимально допустимой массой до 3,5~тонн или
\item водительские права категории С\note[prescription:lisence] для транспортных средств с максимально допустимой массой более 3,5~тонн.
\stopitemize

Если максимальная скорость машины составляет 25\,км/ч, водитель должен как минимум знать правила движения по общественным дорогам и улицам, даже если для вождения машины не требуются права категории B\note[prescription:user:right].

\footnotetext [prescription:user:right] {Обязательства работодателя и сотрудников могут варьироваться в зависимости от страны или региона. Ознакомьтесь с предписаниями, действующими в вашей стране и регионе.}

\footnotetext[prescription:lisence] {Директива 2006/126/EG Европарламента и Совета Европы от 20~декабря 2006 г. о водительских правах.}


\subsubject{Условия использования}

Использовать машину \sdeux\ можно только в том случае, если она находится в безупречном состоянии. Оператор должен соблюдать инструкции и правила техники безопасности, содержащиеся в настоящем руководстве по эксплуатации. Неисправности, отрицательно влияющие на безопасность, должны незамедлительно устраняться на подходящем специализированном предприятии.
\blank [big]

\startSymList
\externalfigure [s2_inspection] [width=4.5em]
\SymList
{\md Ежедневное техобслуживание:}
После каждого использования проводите инспекцию машины и устраняйте видимые повреждения и дефекты. При обнаружении повреждений или сбоев немедленно поставьте в известность специализированную мастерскую. Если это невозможно, немедленно остановите машину и оградите место аварии.
\stopSymList


\subsubject{Использование по назначению}

Машина \sdeux\ предназначена для очистки и ухода за улицами, дорогами и площадями. Любое применение вне этих рамок считается использованием не по назначению. В связи с этим фирма \boschung\ исключает любую ответственность за возникший в результате этого ущерб. При ненадлежащем использовании пользователь несет единоличную ответственность за все связанные с этим последствия. {\em Использование по назначению также включает соблюдение инструкций по технике безопасности и сроков техобслуживания, содержащихся в данном руководстве.}


\section{Передвижение по общественным дорогам}

\subsubject{Общие предписания}

В дополнение к инструкциям по эксплуатации должны соблюдаться общепринятые правила, действующие законодательные и другие правила и нормы по предотвращению несчастных случаев, а также по защите окружающей среды.


\subsubject{Место помощника водителя}

Помощник водителя~ может занять соответствующее место рядом с водителем, так называемое {\em место помощника}.


\subsubject{Ремень безопасности}

\startSymList
% \externalfigure [prescription:safety:belt]
\PMbelt
\SymList
Водитель и помощник водителя машины \sdeux\ должны при посадке в машину всегда пристегивать ремни безопасности в соответствии с действующими правилами дорожного движения.
\stopSymList


\subsubject{Видеть и быть видимым}

\startSymList
\externalfigure [travaux_deviation] [width=3.5em]
\SymList
Убедитесь, что вас хорошо видно, особенно на улицах с сильным движением.

Если при выполнении определенного маневра или работы водитель не имеет достаточного обзора, он должен прибегнуть к помощи другого человека, с которым он должен постоянно поддерживать зрительный контакт.
\stopSymList


\subsubject{Освещение и сигнальные устройства}

В зависимости от действующих ПДД фары и (или) проблесковый маячок машины должны быть включены и днем.


\subsubject{Использование мобильных телефонов}

\startSymList
\PPphone
\SymList
Во время движения по улице общего пользования пользоваться мобильным или радиотелефоном запрещено, если только машина не оснащена головной гарнитурой.

В любом случае, во время разговора по телефону\index{Безопасность+Мобильный телефон} за рулем~– даже пр наличии гарнитуры~– снижается концентрация на дороге.
\stopSymList


\section{Правила техобслуживания}

\subsubject{Инструкции по техобслуживанию}

Обслуживающий персонал должен прочитать руководство по эксплуатации машины \sdeux, в частности, разделы, посвященные технике безопасности и техобслуживанию.


\subsubject{Требуемая квалификация}

\startSymList
\externalfigure [mecanicienne] [width=3.5em]
\SymList
Право заниматься техобслуживанием \sdeux\ имеют только лица, получившие необходимые знания в рамках соответствующего обучения. Это относится, прежде всего, ко всем работам на двигателе, тормозной системе, рулевом управлении, а также на электрическом и гидравлическом оборудовании.
\stopSymList


\testpage [6]
\subsubject{Надзор}

\startSymList
\externalfigure [mecanicien_hyerarchie] [width=3.5em]
\SymList
Лица, получающие образование~– находящиеся на практике или обучении~,– имеют право работать с машиной только под надзором специалиста. Проведите выборочную проверку того, знают ли сотрудники руководство по эксплуатации и правила техники безопасности.
\stopSymList


\subsubject{Сварочные работы}

\startSymList
\externalfigure [pince_soudure2] [width=3.5em]
\SymList
Перед проведением сварочных работ на кузове или шасси, необходимо обязательно отсоединить аккумуляторную батарею и все электронные устройства управления.
\stopSymList

\testpage [6]
\subsubject{Очистка машины}

\startSymList
\externalfigure [washer_pressure] [width=3.5em]
\SymList
Перед очисткой \sdeux\ прочитайте раздел\about[sec:cleaning] \atpage[sec:cleaning], в частности, инструкцию по очистке.
\stopSymList


\subsubject{Доступность документации на машину}

\startSymList
\externalfigure [lecteur_1] [width=3.5em]%\PMrtfm
\SymList
Во время работы документация должна постоянно находиться в кабине водителя машины в легко доступном месте.
\stopSymList


\section{Специальные инструкции по эксплуатации}

\subsubject{Высота машины}

\startSymList
\PPmaxheight
\SymList
Во время работы или движения не на открытой местности (в подземных гаражах, путепроводах, под ЛЭП и т.д.) обязательно убедитесь, что габарит по высоте достаточен для проезда \sdeux\ (см. \in{раздел}[sec:measurement], \atpage[sec:measurement]).
\stopSymList


\subsubject{Устойчивость машины}

Не совершайте маневры, которые могут нарушить устойчивость машины. В случае прохождения поворотов на высокой скорости машина \sdeux\ может опрокинуться из-за узкой конструкции и высокого центра тяжести при полном бункере для грязи.


\subsubject{Непредвиденное движение машины}

При выходе из транспортного средства примите меры против его использования не уполномоченными на это лицами. Всегда ставьте машину на стояночный тормоз, прежде чем выйти из нее; при необходимости подложите под колеса противооткатные клинья.

\startbuffer [prescription:handbrake]
\starttextbackground [CB]
\startPictPar
\PPstop
\PictPar
{\md Надежно поставьте машину на стояночный тормоз!} В противном случае машина может неожиданно прийти в движение – даже на небольшом уклоне –\index{Стояночный тормоз+Потенциальная опасность} и стать причиной аварии с потенциально летальными травмами.

{\lt Благодаря гидростатической приводной системе при простое давление в гидравлическом контуре постепенно снижается, что приводит к уменьшению удерживающей силы двигателя. По этой причине очень важно при выходе из машины надежно поставить ее на стояночный тормоз.}
\stopPictPar
\stoptextbackground

\stopbuffer

\getbuffer [prescription:handbrake]


\testpage [6]
\subsubject{Бункер для отходов}

\startbuffer [prescription:container:gravity]
\starttextbackground [CB]
\startPictPar
\PHgravite
\PictPar
{\md Опасность несчастного случая:}
{\lt При опрокидывании бункера центр тяжести смещается вверх. В результате увеличивается опасность опрокидывания машины. По этой причине при опрокидывании бункера машина должна стоять на горизонтальной и прочной поверхности.}
\stopPictPar
\stoptextbackground

\stopbuffer

\getbuffer [prescription:container:gravity]


\startbuffer [prescription:container:tilt]
\starttextbackground [CB]
\startPictPar
\PHcrushing
\PictPar
{\md Опасность несчастного случая:}
{\lt Перед началом работ под бункером обязательно установите защитные подпорки на гидравлические подъемные цилиндры бункера.}
\stopPictPar
\stoptextbackground

\stopbuffer

\getbuffer [prescription:container:tilt]


\stopcomponent


