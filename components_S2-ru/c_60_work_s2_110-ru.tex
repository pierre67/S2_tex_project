\startcomponent c_60_work_s2_110-ru
\product prd_ba_s2_110-ru

\startchapter [title={Машина S2 в повседневной жизни},
reference={chap:using}]

\setups [pagestyle:marginless]


% \placefig[margin][fig:ignition:key]{Clé de contact}
% {\externalfigure [work:ignition:key]}
\startregister[index][chap:using]{Ввод в эксплуатацию}

\startsection [title={Ввод в эксплуатацию},
reference={sec:using:start}]


\startSteps
\item Убедитесь, что регулярные проверки и обслуживание были проведены надлежащим образом.
\item Включите двигатель ключом зажигания: Включите зажигание, затем поверните ключ дальше по часовой стрелке и удерживайте, пока двигатель не запустится (это возможно только, если рычаг переключения передач находится в нейтральном положении).
\stopSteps

\start
\setupcombinations [width=\textwidth]

\placefig[here][fig:select:drive]{Рычаг переключения передач}
{\startcombination [2*1]
{\externalfigure [work:select:fDrive]}{Рычаг в положении \aW{Движение вперед}}
{\externalfigure [work:select:rDrive]}{Рычаг в положении \aW{Движение назад}}
\stopcombination}
\stop


\startSteps [continue]
\item Поверните переключатель рычага, чтобы в режиме \aW{движения} выбрать передачу:
\startitemize [R]
\item Первая передача
\item Вторая передача (автоматический режим; автоматический запуск на первой передаче)
\stopitemize

Или нажмите на наружную кнопку на рычаге, чтобы активировать или деактивировать \aW{рабочий} режим.
\stopSteps

\startbuffer [work:config]
\starttextbackground [FC]
\startPictPar
\PMrtfm
\PictPar
В рабочем режиме доступна только первая передача, частота вращения двигателя 1300\,мин\high{\textminus 1}.

Частотой вращения двигателя можно управлять кнопками~\textSymb{joy_key_engine_increase} и~\textSymb{joy_key_engine_decrease} многофункциональной консоли.
\stopPictPar
\stoptextbackground
\stopbuffer

\getbuffer [work:config]

\startSteps [continue]
\item Переместите рычаг переключения передач вверх и вперед (для движения вперед) или вверх и назад (для движения назад). См. иллюстрацию вверху.
\item Перед ускорением отпустите стояночный тормоз.
\stopSteps

\starttextbackground [FC]
\startPictPar
\PMrtfm
\PictPar
{\md Освободите стояночный тормоз полностью!} Положение рычага стояночного тормоза контролируется электронным датчиком: Если стояночный тормоз освобожден не полностью, скорость движения ограничена 5\,км/ч.
\stopPictPar
\stoptextbackground

\startSteps [continue]
\item Медленно нажмите на педаль акселератора, чтобы начать движение.
\stopSteps


%% NOTE: New text [2014-04-29]:
\subsection [sSec:suctionClap] {Заслонка канала всасывания}

Система всасывания создает поток воздуха от патрубка всасывания или ручного шланга (опция) к бункеру для грязи.

С помощью управляемой вручную заслонки (\inF[fig:suctionClap], \atpage[fig:suctionClap]) воздушный поток можно переключать между патрубком всасывания и ручным шлангом.

В обычном режим~– при работе с патрубком~– канал всасывания должен быть открыт (рычаг переключения направлен вверх).

Чтобы можно было использовать шланг, канал всасывания следует закрыть (рычаг переключения направлен вниз). Поток воздуха будет направляться по шлангу.

\testpage [8]

\placefig [here] [fig:suctionClap] {Заслонка канала всасывания}
{\startcombination [2*1]
{\externalfigure [work:suctionClap:open]}{Канал всасывания открыт}
{\externalfigure [work:suctionClap:closed]}{Канал всасывания закрыт}
\stopcombination}

%% End new text

\stopsection


\startsection [title={Прекращение эксплуатации},
reference={sec:using:stop}]

\index{Прекращение эксплуатации}

\startSteps
\item Поставьте машину на ручной тормоз (рычагом между сиденьями) и приведите рычаг переключения передач в \aW{нейтральное} положение.
\item Проведите необходимые проверки~– ежедневные и при необходимости еженедельные~– в соответствии с описанием в разделе \atpage[table:scheduledaily].
\stopSteps

\getbuffer [prescription:handbrake]

\stopsection


\startsection [title={Уборка и всасывание},
reference={sec:using:work}]

\startSteps
\item Проведите\index{Уборка} ввод машины в эксплуатацию в соответствии с описанием в \in{§}[sec:using:start], \atpage[sec:using:start].
\item Активируйте\index{Всасывание} \aW{рабочий}режим (кнопка на внешней стороне рычага переключения передач).
\stopSteps

% \getbuffer [work:config]
%% NOTE: outcommented by PB

\startSteps [continue]
\item Нажмите кнопку~\textSymb{joy_key_suction_brush}, чтобы включить турбину и щетки.

{\md Вариант:} {\lt Нажмите кнопку~\textSymb{joy_key_suction}, чтобы работать только с патрубком всасывания.}

\item Настройте скорость вращения щеток кнопками~\textSymb{joy_key_frontbrush_increase}\textSymb{joy_key_frontbrush_decrease} на многофункциональной консоли.

\item С помощью соответствующих джойстиков приведите щетки в такое положение, чтобы получить оптимальную рабочую ширину.
\stopSteps

\vfill

\start
\setupcombinations [width=\textwidth]

\placefig[here][fig:brush:position]{Позиционирование щеток}
{\startcombination [2*1]
{\externalfigure [work:brushes:enlarge]}{Перемещение щеток наружу/внутрь}
{\externalfigure [work:brush:left:raise]}{Перемещение щеток вверх/вниз}
\stopcombination}
\stop

\page [yes]


\subsubsubject{Смачивание щеток и канала всасывания}

Нажмите\index{Уборка+Смачивание} переключатель~\textSymb{temoin_busebalais} между сиденьями:

{\md Положение 1:} Водяной насос работает автоматически, пока щетки активированы.

{\md Положение 2:} Водяной насос работает постоянно. (Полезно \eG\ для работ по настройке.)


\subsubsubject{Крупный мусор}

\startSteps [continue]
\item Если существует опасность, что крупные объекты (\eG\ бутылки из ПЭТ) заблокируют патрубок всасывания, откройте\index{Заслонка для крупного мусора} заслонку для крупного мусора боковыми кнопками на многофункциональной консоли или~– если этого недостаточно~– поднимите\index{Патрубок всасывания+Крупный мусор} патрубок на некоторое время.
\stopSteps

\start
\setupcombinations [width=\textwidth]

\placefig[here][fig:suctionMouth:clap]{Действия с крупным мусором}
{\startcombination [2*1]
{\externalfigure [work:suction:open]}{Открытие заслонки для крупного мусора}
{\externalfigure [work:suction:raise]}{Временный подъем патрубка всасывания}
\stopcombination}
\stop

\stopsection


\startsection [title={Опорожнение грязевого резервуара},
reference={sec:using:container}]

\startSteps
\item Отведите\index{Бункер для грязи+Опорожнение} машину в подходящее место для опорожнения бункера. Соблюдайте действующие правила охраны окружающей среды.
\item Включите стояночный тормоз. (Это необходимо для разблокирования переключателя опрокидывания бункера.)
\stopSteps

\getbuffer [prescription:container:gravity]

\startSteps [continue]
\item Разблокируйте и откройте заслонку грязевого бункера.
\item Чтобы опрокинуть бункер, нажмите переключатель~\textSymb{temoin_kipp2} (средняя консоль, между сиденьями).
\item После опорожнения вымойте бункер изнутри струей воды. Для этого можно использовать водяной пистолет (дополнительное оснащение).
\stopSteps

\start
\setupcombinations [width=\textwidth]
\placefig[here][fig:brush:adjust]{Обращение с грязевым бункером}
{\startcombination [3*1]
{\externalfigure [container:cover:unlock]}{Блокировка заслонки}
{\externalfigure [container:safety:unlocked]}{Защитная подпорка}
{\externalfigure [container:safety:locked]}{Защитная подпорка заблокирована}
\stopcombination}
\stop

\startSteps [continue]
\item Проверьте и очистите уплотнения и поверхности прилегания уплотнений бункера, системы регенерации и канала всасывания.
\stopSteps

\getbuffer [prescription:container:tilt]

\startSteps [continue]
\item Нажмите переключатель~\textSymb{temoin_kipp2}, чтобы опустить бункер. (Предварительно уберите защитные распорки с гидравлических цилиндров.)
\item Заблокируйте заслонку грязевого бункера.
\stopSteps

\stopsection


\startsection [title={Ручной всасывающий шланг},
reference={sec:using:suction:hose}]

Машина \sdeux\ может быть дополнительно\index{Ручной всасывающий шланг} оснащена ручным всасывающим шлангом. Он зафиксирован на заслонке бункера; использовать его несложно.

{\sla Необходимые условия:}

Бункер для грязи полностью опущен; \sdeux\ находится в \aW{рабочем} режиме. (См. \in{§}[sec:using:start], \atpage[sec:using:start].)

\startfigtext[left][fig:using:suction:hose]{Ручной всасывающий шланг}
{\externalfigure[work:suction:hose]}
\startSteps
\item Нажмите кнопку~\textSymb{temoin_aspiration_manuelle} на потолочной консоли, чтобы активировать систему всасывания.
\item Прежде чем выйти из кабины, надежно поставьте машину на ручной тормоз.
\item Закройте канал всасывания заслонкой. (См. \in{§}[sSec:suctionClap], \atpage[sSec:suctionClap].)
\item Вытащите шланг из крепления за насадку и приступайте к работе.
\item По окончании работы снова нажмите кнопку~\textSymb{temoin_aspiration_manuelle}, чтобы выключить систему всасывания.
\stopSteps
\stopfigtext

\stopsection

\page [yes]

\setups[pagestyle:normal]


\startsection [title={Водяной пистолет высокого давления},
reference={sec:using:water:spray}]

Машина \sdeux\ может быть дополнительно\index{Водяной пистолет} оснащена водяным пистолетом. Водяной пистолет закреплен на задней правой дверцей для техобслуживания и соединен с барабаном со шлангом длиной 10 м~, закрепленным на противоположной стороне машины~.

Чтобы использовать водяной пистолет, сделайте следующее:

{\sla Необходимые условия:}

В баке чистой воды есть достаточно воды; \sdeux\ находится в \aW{рабочем}режиме. (См. \in{§}[sec:using:start], \atpage[sec:using:start].)

\placefig[margin][fig:using:water:spray]{Водяной пистолет высокого давления}
{\externalfigure[work:water:spray]}

\startSteps
\item Нажмите кнопку~\textSymb{temoin_buse} на потолочной консоли, чтобы активировать водяной пистолет.
\item Прежде чем выйти из кабины, надежно поставьте машину на ручной тормоз.
\item Откройте заднюю правую дверцу для техобслуживания и выньте пистолет.
\item Размотайте шланг на нужную длину и начинайте работу.
\item По окончании работы снова нажмите кнопку~\textSymb{temoin_buse}, чтобы выключить водяной пистолет.
\item Чтобы открыть блокиратор и смотать шланг, коротко потяните за него.
\item Закрепите пистолет в держателе и закройте дверцу для техобслуживания.
\stopSteps

\stopsection
\stopregister[index][chap:using]

\stopchapter
\stopcomponent

