\startcomponent c_30_overview_s2_110-ru
\product prd_ba_s2_110-ru

\startchapter [title={Обзор машины}]

\setups [pagestyle:marginless]


\placefig [here] [] {Обзор левой стороны}
{\externalfigure [overview:side:left:ru]}


\page [yes]


\placefig [here] [] {Обзор правой стороны}
{\externalfigure [overview:side:right:ru]}

\page [yes]

\setups [pagestyle:normal]


\startsection [title={Общие сведения}]

\placefig[margin][p4_vue_01]{\sdeux\ Транспортировка}
{%
\startcombination [1*3]
{\externalfigure[overview:vhc:01]}{}
{\externalfigure[overview:vhc:02]}{}
{\externalfigure[overview:vhc:03]}{}
\stopcombination}

В уборочной машине \BosFull{sdeux} компания Boschung воплотила все свои знания и опыт, приобретенные за десятилетия сотрудничества с постоянными клиентами и партнерами.
За это время требования муниципалитетов и компаний к мобильности и универсальности выросли невероятно. Проектировщики \sdeux\ смогли выполнить эти требования с учетом пожеланий заказчиков и предложений службы поддержки Boschung.
Базой для создания машины \sdeux\ стала именно ориентация на потребности клиентов и последовательная реализация практических знаний.


\subsection{Инновационная технология}

Компактная уборочная машина \BosFull{sdeux} отличается небольшой для своего класса массой (2300\,кг), большой вместимостью бункера (класса 2,0 м\high{3}), компактностью (ширина 1,15\,м) и особенно эргономичным местом водителя.

Благодаря компактной конструкции машину \sdeux\ можно использовать для уборки \quotation{любых} дорог и тротуаров в городах и деревнях.  Ее мощный дизельный двигатель в сочетании с компактным гидростатическим приводом (радиально-поршневые гидродвигатели на передних колесах) постоянно обеспечивает высочайшую мобильность вне зависимости от условий эксплуатации и степени заполнения бункера для грязи.

Гидравлические насосы приводятся в действие дизельным двигателем типа \aW{VW 2.0 CDI}, соответствующим стандарту Евро-5. Момент вращения двигателя составляет 285\,Нм при 1750~об/мин, а максимальная мощность 75\,кВт при 3000~об/мин. Благодаря этому машину можно эффективно использовать и при низкой частоте вращения двигателя~– и, следовательно, низком уровне шума~. Машина \sdeux\ серийно оснащается пылевым фильтром.

\stopsection


\startsection [title={Инновации на службе клиентов}]

Шарнирное сочленение \sdeux\ обеспечивает небольшой радиус поворота и, следовательно, максимальную подвижность. Благодаря использованию специальных материалов, например, Domex®, и проектированию полностью на базе CAD полезная нагрузка машины составляет целых 1200\,кг.

\placefig[margin][overview:cab:frontright]{\sdeux\ Готов к работе}
{\externalfigure[overview:cab:twoleft][width=\Bildwidth]}

В кабине с полным остеклением установлены два удобных сиденья с трехточечными ремнями безопасности. Машину \sdeux\ также можно оснастить кондиционером.

Так как максимальная скорость составляет 40\,км/ч, машины без проблем вписывается в уличное движение. Благодаря качественной подвеске переднего и заднего моста машина надежно и комфортно движется даже по самой плохой дороге.

Уборочный агрегат,~установленный на двух шарнирных консолях,~находится полностью в поле зрения водителя, а патрубок всасывания смонтирован перед передним мостом в хорошо видном месте. В качестве дополнительного оснащения предлагается передняя щетка, поворачиваемая в двух направлениях.

\page [yes]


\subsection{Звукоизолирующая и комфортабельная кабина водителя}

Кабина водителя\index{Кабина водителя} машины \sdeux\ с правосторонним рулем рассчитана на двух человек. Она имеет звукоизоляцию и смонтирована на амортизированных сайлент-блоках.

Двери и пол остеклены, что обеспечивает полный обзор. Ветровое стекло занимает всю переднюю часть кабины, благодаря чему щетки видны без помех.

Сиденье водителя оснащено механическим или~– в качестве опции~– пневматическим амортизатором. Оба сиденья установлены на регулируемых направляющих скольжения.


\subsubsubject{Эргономика}

\startfigtext[right][overview:joy:sideview]{Консоль управления}
{\externalfigure[overview:joy:top]}
Все базовыми функциями можно управлять одной рукой на многофункциональной консоли, находящейся слева от сиденья водителя. Обеими щетками можно управлять по отдельности большим и указательным пальцем с помощью двух джойстиков. Переключатели боковых и передней щетки (опция), частоты вращения двигателя, круиз-контроля и т.п. также находятся на многофункциональной консоли.
\stopfigtext

В нижней части поля зрения водителя находится сенсорный экран, на котором в реальном времени отображается вся важная информация о работе машины, причем считывать ее можно, не теряя обзор ситуации на дороге.

\placefig[margin][overview:vhc:left]{\sdeux\ Перед историческими стенами}
% \placefig[margin][overview:vhc:left]{\sdeux\ sur site historique}
{\externalfigure[overview:vhc:left]}

\page [yes]


\subsubsubject{Место водителя}

\index{Место водителя}Рычаг переключения передач (\quotation{скоростей}) находится на рулевой колонке справа; имеются две передачи для движения вперед и одна задняя передача. Снаружи на рычаге есть кнопка переключения двух режимов: \aW{рабочего} и режима \aW{движения}. Останавливать \sdeux\ для переключения не нужно. (См. также главу \about[sec:using:work], \atpage[sec:using:work].)

\placefig[margin][fig:overview:steeringwheel]{Место водителя}
{\externalfigure[overview:driver:place]}

При движении назад включается монитор камеры заднего вида, и раздается звуковой сигнал (его можно выключить на панели Vpad).

Многофункциональный рычаг на левой стороне рулевой колонки предназначен для управления стеклоочистителями (две ступени и интервал), а также светом и звуковым сигналом.

В главе \about[chap:using] , начиная с \atpage[chap:using] , приведены подробные сведения об этих и других функциях \sdeux.

\page [yes]

\setups[pagestyle:marginless]


\subsection[overview:brushsystem]{Уборочное и всасывающее приспособление}

\subsubsubject{Щетки}

\startfigtext[left][fig:overview:steeringwheel]{Уборочное и всасывающее приспособление}
{\externalfigure[system:brush]}
Щетки\index{Уборка} находятся на регулируемых головках, установленных на шарнирных консолях. Пыль, поднятая при уборке, опрыскивается водой: каждая щетка оснащена распылителем, вода к которому подается из бака чистой воды или бака системы регенерации.

Переключатель\index{Всасывание} на многофункциональной консоли включает щетки и водяной насос одновременно.\footnote{Информация о водяном насосе приведена в главе \in[chap:using] \about[chap:using], в частности, \about[sec:using:work], \atpage[sec:using:work].}
Положением, а также продольным и поперечным наклоном щеток можно управлять непосредственно соответствующим джойстиком на многофункциональной консоли.
\stopfigtext

Для защиты щеток используется механическая и гидравлическая система предотвращения столкновений.


\subsubsubject{Патрубок всасывания}

В рабочем (опущенном) положении патрубок всасывания опирается на 4~ролика и полностью закрывает пространство между двумя раздвинутыми щетками. Благодаря \quotation{отведенному} назад положению он защищен от механических повреждений при столкновениях с механическими препятствиями. При движении назад патрубок поднимается автоматически.

Толстая сменная кромка из резины обеспечивает герметичный стык с поверхностью дороги. Заслонка на передней стороне патрубка с электро-гидравлическим управлением позволяет захватывать крупный мусор.


\subsubsubject{Бункер для отходов}

Алюминиевый бункер можно опрокидывать до положения под углом 55° и по высоте 1,5\,м (для опорожнения). Снизу в него входит канал всасывания с диаметром входного отверстия 180\,мм.

Разрежение создается турбиной высокой мощности, установленной в бункере горизонтально. Она оснащена заслонкой для очистки и осмотра.

В заслонке бункера находятся два отверстия всасывания с решетками из высококачественной стали. Для очистки их можно откинуть без использования инструментов. Заслонку можно отпереть и открыть вручную.

С помощью регулируемой вручную заслонки поток воздуха можно переключать между каналом всасывания и ручным всасывающим шлангом (дополнительное оснащение).


\subsection{Увлажняющее устройство}

\subsubsubject{Система чистой воды}

\index{Уборка+Увлажнение} Бак из акрилонитрил-бутадиенстирола смонтирован за кабиной водителя вертикально. Объем\index{Чистая вода+Бак} составляет 190\,л.

Электрический насос (10\,л/мин) подает воду к распылителям над щетками (включая дополнительную третью щетку).


\subsubsubject{Регенерация грязной воды}

Грязная вода проходит через микроотверстия во внутренних стенках бака грязной воды и стекает в нижний бак регенрации через заслонку. \index{Вода системы регенерации+Бак} Объем бака регенерации составляет 140\,л.

Электрический погружной насос (10\,л/мин) подает воду к распылителям внутри патрубка и канала всасывания.


\testpage [8]

\subsubsubject{Бак системы регенерации}

Бак системы регенерации оснащен теплообменником "вода-гидравлическая жидкость" двойного действия:

\startitemize[width=45mm,style=\md, command={\setupwhitespace[small]}]
\sym{Принцип действия летом} Благодаря конвекции вода передает тепло гидравлической жидкости стенкам алюминиевого бака, которые отдают его окружающему воздуху.

\sym{Принцип действия зимой} Гидравлическая жидкость нагревает воду в баке. Благодаря этому канал и патрубок всасывания можно опрыскивать водой и при температурах чуть ниже точки замерзания.
\stopitemize


\subsubsubject{Контроль уровня воды}

\startitemize[width=45mm,style=\md, command={\setupwhitespace[small]}]
\sym{Чистая вода} Если уровень недостаточен, на экране Vpad появляется пиктограмма ~\textSymb{vpad_water}.
\sym{Вода системы регенерации} Если уровень воды в баке опустится ниже теплообменника (см. выше), на экране Vpad появляется пиктограмма~\textSymb{vpad_rwater_orange} (желтая). Если уровень недостаточен, появляется пиктограмма~\textSymb{vpad_rwater} (красная).
\stopitemize

\subsubsubject{Breitbereifung (Option)}

Давление на грунт\index{широкие шины} соответствует давлению в шинах. При давлении в шинах 1,8\,бар достигается давление на грунт, равное 18\,Н/см². Однако в этом случае грузоподъемность шины будет недостаточной для гарантированной допустимой нагрузки на ось. При 1,8\,бар и 40\,км/ч можно гарантировать только 1495\,кг нагрузки на ось. Если давление в шинах будет отличаться от 3,0\,бар, всю ответственность будет нести владелец машины.

\subsubsubject{Индикатор перегрузки (опция)}

В случае перегрузки машины\index{Индикатор перегрузки} на Vpad появится сообщение. Перегрузка измеряется угловым датчиком на задней оси. По умолчанию индикатор установлен на 3500\,кг, отклонения от этого значения недопустимы. Изменить это значение (3500\,кг) может специализированное предприятие.

\stopsection

\page [yes]

\setups[pagestyle:normal]


\startsection [title={Идентификация машины}]

\subsection{Заводская табличка машины}

Заводская табличка машины\index{Идентификация+Машина} находится в кабине напротив консоли, под сиденьем пассажира (см. \inF[fig:identity:location], \atpage[fig:identity:location]).


\subsection{Код и номер двигателя}

Код двигателя указан на заводской табличке двигателя (наклейка), на изогнутой металлической трубке контура охлаждения, в передней части двигателя (нужно поднять бункер для отходов).

Номер выгравирован на двигателе (\inF[identity:engine:number]). Он состоит из девяти буквенно-цифровых знаков: Первые три означают код двигателя, а следующие шесть – серийный номер.


\placefig[margin][idvhc]{Заводская табличка машины}
{\externalfigure[s2:id:plaque]}

\placefig[margin][identity:engine:code]{Заводская табличка двигателя}
{\externalfigure[engine:id:code]}

\placefig[margin][identity:engine:number]{Номер двигателя}
{\externalfigure[engine:id:number]}

\page [yes]


\subsection [sec:plateWheel]{Заводская табличка колес}

Заводская табличка ободов и шин\index{Шины+Давление} находится в кабине\index{Ободы+Размеры}, под сиденьем пассажира.


\subsection{Номер шасси}

Номер шасси\index{Идентификация+Номер шасси} выбит на шасси справа под кабиной водителя.


Соответствие нормам \subsection{\symbol[europe][CEsign]и обозначение}

Знак соответствия~\symbol[europe][CEsign] находится в кабине, напротив консоли, под сиденьем пассажира.

Машина \sdeux\ отвечает базовым требованиям к безопасности и охране здоровья Директивы по машиностроению\index{Сертификат+Соответствие нормам ЕС}\index{Директива по машиностроению} 2006/42/EG\footnote{2006/42/EG Европейского Парламента и Совета Европы от 17~мая 2006}.
% \textrule

\placefig[margin][idpneus]{Давление в шинах}
{\externalfigure[identity:tires]}

\placefig[margin][fig:identity:location]{Заводские таблички}
{\externalfigure[identity:location]}

\stopsection

\page [yes]


\setups [pagestyle:marginless]


\startsection[title={Технические характеристики},
reference={donnees_techniques}]

\subsection [sec:measurement] {Габариты машины}

\placefig[here][fig:measurement]{\select{caption}{Ширина~– щетки в положении покоя или раздвинуты~–, Длина и высота машины}{Габариты машины}}
{\Framed{\externalfigure[s2:measurement]}}

\page [yes]

\placefig[here][fig:measurement]{\select{caption}{Высота машины с опрокинутым бункером}{Высота машины}}
{\Framed{\externalfigure[s2:measurement:02]}}

\page [yes]

\starttabulate [|lBw(45mm)|p|l|rw(35mm)|]
\FL
\NC Группа\index{Габариты} \NC \bf Габарит \NC \bf Единица измерения \NC \bf Значение \NC\NR
\ML
\NC Габариты машины \NC Длина (габаританя) \NC \unite{мм} \NC 4588,00 \NC\NR
\NC\NC Длина с 3-й\,щеткой \NC \unite{мм} \NC 5020,00 \NC\NR
\NC\NC Ширина машины \NC \unite{мм} \NC 1150,00 \NC\NR
\NC\NC Ширина машины (габаритная) \NC \unite{мм} \NC 1575,00 \NC\NR
\NC\NC Высота без проблескового маячка \NC \unite{мм} \NC 1990,00 \NC\NR
\NC\NC Колесная база \NC \unite{мм} \NC 1740,00 \NC\NR
\NC\NC Колея \NC \unite{мм} \NC 894,00 \NC\NR
\ML
\NC Ширина уборки \NC Стандартные щетки\NC \unite{мм} \NC 2300,00 \NC\NR
\NC\NC С 3-й\,щеткой \NC \unite{мм} \NC 2600,00 \NC\NR
\NC\NC Диаметр щетки \NC \unite{мм} \NC 800,00 \NC\NR
\NC\NC Ширина патрубка всасывания \NC \unite{мм} \NC 800,00 \NC\NR
\ML
\NC Распределение нагрузки \NC Масса в пустом виде\note[weight:empty] Передний мост \NC \unite{кг} \NC ок. 1100,00 \NC\NR
\NC\NC Масса в пустом виде\note[weight:empty] Задний мост \NC \unite{кг} \NC ок. 1200,00 \NC\NR
\NC\NC Масса в пустом виде\note[weight:empty] \NC \unite{кг} \NC ок. 2300,00 \NC\NR
\NC\NC Допуст. общая масса \NC \unite{кг} \NC 3500,00 \NC\NR
\LL
\stoptabulate


\subsection{Радиус колеи и радиус уборки}

\starttabulate [|lBw(45mm)|p|l|rw(35mm)|]
\FL
\NC Габарит\index{Габариты} \NC \bf Размер \NC \bf Единица измерения \NC \bf Значение \NC\NR
\ML
\NC Радиус колеи\index{Радиус колеи}\index{Размер+Радиус колеи} \NC Минимальный радиус поворота со щетками \NC \unite{мм} \NC 3325,00 \NC\NR
\ML
\NC Радиус уборки \NC наружный \NC \unite{мм} \NC 3425,00 – 3850,00 \NC\NR
\NC\NC внутренний \NC \unite{мм} \NC 2025,00 – 1675,00 \NC\NR
\LL
\stoptabulate

%% TODO en/de/fr: Footnote on preceeding page
\footnotetext[weight:empty]{Стандартная конфигурация, с водителем (ок. 75\,кг).}

\placefig[here][pict:steerin_sweeping:radius]{Радиус колеи/поворота и радиус уборки}
{\externalfigure[steerin_sweeping:radius]}

\page [yes]


\subsection{Колеса и шины}

{\sla Стандартные габариты}

\starttabulate[|lBw(45mm)|p|rw(55mm)|]
\FL
\NC Компонент \NC \bf Оснащение \NC \bf Значение \NC\NR
\ML
\NC Шины \NC Стандарт \NC 205/70 R 15 C \NC\NR
\ML
\NC Ободы \NC Стандарт \NC 6J\;×\;15 H2 ET 60 \NC\NR
\ML
\NC Давление в шинах\index{внутренний} \NC Стандарт, спереди/сзади \NC 4,5/5,8\,бар \NC\NR
\LL
\stoptabulate

{\sla Широкие шины}

\starttabulate[|lBw(45mm)|p|rw(55mm)|]
\FL
\NC Компонент \NC \bf Оснащение \NC \bf Значение \NC\NR
\ML
\NC Reifen\index{Широкие шины} \NC Широкие шины \NC 275/60 R 15 107H \NC\NR
\ML
\NC Ободы \NC Широкие шины \NC 8LB\;×\;15 ET 30 \NC\NR
\ML
\NC Давление в шинах\index{Давление в шинах} \NC Стандарт, спереди|/|сзади \NC 3,0|/|3,0\,бар \NC\NR
\LL
\stoptabulate


\subsection{Дизельный двигатель}

\starttabulate [|lBw(45mm)|l|rp|]
\FL
\NC \bf Группа\index{Дизельный двигатель+Идентификация} \NC \bf Параметр \NC \bf Значение\NC\NR
\ML
\NC Тип двигателя \NC \NC VW CJDA TDI 2.0 – 475 NE \NC\NR
\NC Общие сведения \NC Рабочий цикл \NC Четырехтактный \NC\NR
\NC\NC Кол-во цилиндров \unite{n} \NC 4 \NC\NR
\NC\NC Отверстие x ход \unite{мм} \NC 81\;×\;95,5 \NC\NR
\NC\NC Рабочий объем \unite{см\high{3}} \NC 1968 \NC\NR
\NC\NC Клапанов на цилиндр \NC 4 \NC\NR
\NC\NC Последовательность включения клапанов \NC 1-3-4-2 \NC\NR
\NC\NC Минимальная частота вращения холостого хода \unite{об/мин\high{}} \NC 830 +50/−25 \NC\NR
\NC Мощность/момент вращения \NC Макс. частота вращения \unite{об/мин\high{}} \NC 3400 \NC\NR
\NC\NC Макс. мощность \unite{кВт} при \unite{об/мин\high{}} \NC 75 – 3000 \NC\NR
\NC\NC Макс. момент вращения \unite{Нм} при \unite{об/мин\high{}} \NC 285 – 1750 \NC\NR
\NC Удельный расход\index{Дизельный двигатель+Расход} \NC Топливо \unite{г/кВтч} \NC 224 (при макс. мощности) \NC\NR
\NC\NC Масло \unite{г/кВтч} \NC 0,22 \NC\NR
\NC Топливная система \NC Инжекторная \NC Прямой впрыск \quote{Топливная система высокого давления} \NC\NR
\NC\NC Подача топлива \NC Шестеренчатый насос  \NC\NR
\NC\NC Наддув \NC Да \NC\NR
\NC\NC Охлаждение наддувочного воздуха \NC Да \NC\NR
\NC\NC Давление наддува \unite{мбар} \NC 1300\NC\NR
\NC Контур смазки\index{Дизельный двигатель+Смазка} \NC Тип \NC Система принудительной смазки с масляно-водяным теплообменником \NC\NR
\NC\NC Подача смазки \NC Роторный насос \NC\NR
\NC\NC Расход масла \unite{л/20\,ч} \NC <\:0,1 \NC\NR
\NC Контур охлаждения\index{Дизельный двигатель+Охлаждение} \NC Общий объем \unite{л} \NC ок. 12 \NC\NR
\NC\NC Контрольное давление в расширительном бачке \unite{бар} \NC 1,4 \NC\NR
\NC\NC Термостат (открытие) \unite{°C} \NC 87 \NC\NR
\NC\NC Термостат (заполнено) \unite{°C} \NC 102 \NC\NR
\NC Выхлопные газы \NC Пылевой фильтр \NC Да \NC\NR
\NC\NC Очистка ОГ \NC Да \NC\NR
\NC\NC Стандарт \NC Евро 5 \NC\NR
\LL
\stoptabulate


\subsection{Ходовые характеристики}

\starttabulate[|lBw(45mm)|p|l|rw(35mm)|]
\FL
\NC Ходовые характеристики\index{Ходовые характеристики} \NC \bf Конфигурация \NC \bf Единица измерения \NC \bf Значение \NC\NR
\ML
\NC Скорость \NC \aW{Рабочий} режим \NC \unite{км/ч} \NC 0 – 18 (плавное регулирование) \NC\NR
\NC\NC \aW{Режим движения} \NC \unite{км/ч} \NC 0 – 40 \NC\NR
\ML
\NC Ограничение скорости \NC Регулируется \NC \unite{км/ч} \NC 0 – 25 \NC\NR
\LL
\stoptabulate


\subsection{Электрическая система}

{\starttabulate [|lw(65mm)|p|rw(30mm)|]
\FL
\NC \bf Группа \NC \bf Компонент \NC \bf Значение \NC\NR
\ML
\NC Батарея \NC Свинцовый аккумулятор \NC 12\,В 63\,Ач \NC\NR
\NC Электропитание \NC Генератор \NC 14,8\,В 90\,A \NC\NR
\NC Стартер \NC Мощность \NC 1,8\,кВт \NC\NR
\NC Аудиосистема \NC Радио\index{Радио} и динамики\index{Динамики} \NC Серийное оснащение \NC\NR
% \NC Sécurité et surveillance \NC Tachygraphe\index{tachygraphe} \NC En option \NC\NR
% \NC\NC Enregistreur de fin de parcours\index{fin de parcours} \NC En option \NC\NR
\NC Освещение и сигнальные устройства спереди \NC Стояночный свет \NC 12\,В 5\,Вт \NC\NR
\NC\NC Ближний свет \NC H7, 12\,В 55\,Вт \NC\NR
\NC\NC Рабочие фары \NC G886, 12\,В 55\,Вт \NC\NR
\NC\NC Мигалки \NC 12\,В 21\,Вт \NC\NR
\NC Освещение и сигнальные устройства сзади \NC Комбинированные стоп-сигналы \NC 12\,В 5/21\,Вт \NC\NR
\NC\NC Мигалки \NC 12\,В 21\,Вт \NC\NR
\NC\NC Фонари заднего хода \NC 12\,В 21\,Вт \NC\NR
\NC\NC Освещение номерного знака \NC 12\,В 5\,Вт \NC\NR
\NC Дополнительное освещение \NC Проблесковый маячок \NC H1, 12\,В 55\,Вт \NC\NR
\LL
\stoptabulate
}
\stopsection

\stopchapter

\stopcomponent

