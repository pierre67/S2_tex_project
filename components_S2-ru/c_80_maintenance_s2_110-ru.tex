\startcomponent c_80_maintenance_s2_110-ru
\product prd_ba_s2_110-ru

\startchapter [title={Техобслуживание и ремонт},
reference={chap:maintenance}]

\setups[pagestyle:marginless]


\startsection [title={Общие указания}]


\subsection{Охрана окружающей среды}

\starttextbackground [FC]
\setupparagraphs [PictPar][1][width=2.45em,inner=\hfill]

\startPictPar
\Penvironment
\PictPar
\Boschung\ применяет на практике концепцию\index{Охрана окружающей среды} охраны окружающей среды. Мы устанавливаем причины и при принятии решений рассматриваем все воздействия производственного процесса и продукта на окружающую среду. Целью при этом являются эффективное использование ресурсов и бережное обращение с основами природы, что служит сохранению человечества. Соблюдение определенных правил эксплуатации транспортного средства способствует охране окружающей среды. Это включает в себя разумное и правильное обращение с веществами и материалами в рамках технического обслуживания транспортных средств (\eG\ утилизация химикатов и опасных отходов).

Расход топлива и износ двигателя зависят от условий эксплуатации. Поэтому мы просим соблюдать следующие требования:

\startitemize
\item Не прогревайте двигатель на холостых оборотах.
\item Выключайте двигатель во время простоев по производственным причинам.
\item Регулярно проверяйте расход топлива.
\item {\em Проводите работы по техобслуживанию в соответствии с планом техобслуживания в компетентной специализированной мастерской.}
\stopitemize
\stopSymList
\stoptextbackground

\page [yes]


\subsection{Правила техники безопасности}

\startSymList
\PHgeneric
\SymList
\index{Техобслуживание+Правила техники безопасности} Для предотвращение поломок транспортных средств и агрегатов, а также несчастных случаев во время техобслуживания, необходимо соблюдать нижеследующие правила техники безопасности. Соблюдайте также общие правила техники безопасности (\about[safety:risques], \at{стр.}[safety:risques] и далее).
\stopSymList

\starttextbackground [FC]
\startPictPar
\PMgeneric
\PictPar
\textDescrHead{Правила техники безопасности}
После каждого техобслуживания или ремонта проводите контроль\index{Предотвращение несчастных случаев} состояния транспортного средства. Прежде чем выезжать на дорогу общего пользования убедитесь, в частности, в том, что все компоненты, относящиеся к обеспечению безопасности, а также устройства освещения и сигнализации, работают должным образом.
\stopPictPar
\stoptextbackground
\blank [big]

\start
\setupparagraphs [SymList][1][width=6em,inner=\hfill]
\startSymList\PHcrushing\PHfalling\SymList
\textDescrHead{Стабилизация транспортного средства}
Каждый раз перед проведением работ по техобслуживанию транспортное средство должно быть защищено от непредвиденного перемещения: Установите рычаг выбора передачи на \aW{нейтраль}, включите стояночный тормоз и зафиксируйте транспортное средство колесными клиньями.
\stopSymList
\stop

\starttextbackground[CB]
\startPictPar\PHpoison\PictPar
\textDescrHead{Запуск двигателя}
Если\index{Опасность+Отравление} вам нужно запустить двигатель в плохо проветриваемом месте, оставляйте его работающим не дольше, чем это необходимо\index{Опасность+Выхлопные газы}, чтобы предотвратить отравление угарным газом.
\stopPictPar
\startitemize
\item Запускайте двигатель только с правильно подключенной батареей.
\item Запрещается отсоединять батарею при работающем двигателе.
\item Запрещается запускать двигатель с помощью внешнего устройства запуска двигателя.
Если\index{Батарея+Зарядное устройство} нужно зарядить батарею зарядным устройством, перед этим необходимо отсоединить ее от транспортного средства. Соблюдайте инструкцию по эксплуатации зарядного устройства.
\stopitemize
\stoptextbackground

\page [bigpreference]

\subsubsection{Защита электронных компонентов}

\startitemize
\item Перед\index{Электросварка} началом сварочных работ отсоедините кабели от батареи и соедините кабели плюса и массы между собой.
\item Присоединяйте\index{Электроника} и отсоединяйте электронные устройства лишь тогда, когда они не находятся под напряжением.
\item Неправильная\index{Контроллер} полярность электропитания (\eG\ в результате неправильного подключения батарей) может привести к разрушению электронных узлов и приборов.
\item При\index{Окружающая температура+экстремально} окружающих температурах выше 80 °C (\eG\ например, в сушильной камере) необходимо удалять электронные компоненты/приборы.
\stopitemize


\subsubsection{Диагностика и измерения}

\startitemize
\item При операциях измерения и диагностики используйте только {\em подходящие} контрольные кабели (\eG\ оригинальные кабели приборов).
\item Мобильные телефоны\index{Мобильный телефон} и другие подобные радиоустройства могут приводить к ухудшению функций транспортного средства, контроллера диагностики и, тем самым, к снижению эксплуатационной надежности.
\stopitemize


%%%%%%%%%%%%%%%%%%%%%%%%%%%%%%%%%%%%%%%%%%%%%%%%%%%%%%%%%%%%%%%%%%%%%%%%%%%%%%%%%%%%%%%%%

\subsubsection{Квалификация персонала}

\starttextbackground[CB]
\startPictPar
\PHgeneric
\PictPar
\textDescrHead{Опасность несчастного случая}
При\index{Квалификация+Персонал техобслуживания} неквалифицированном проведении работ по техобслуживанию возможно ухудшение работоспособности и безопасности транспортного средства. Это влечет за собой повышенный риск несчастных случаев и травм.

Для проведения\index{Квалификация+Мастерская} техобслуживания и ремонта обращайтесь к специалистам специализированной мастерской, обладающим необходимыми знаниями и инструментами.

В случае сомнений обращайтесь в гарантийную службу \Boschung.
\stopPictPar
\stoptextbackground

% \page [yes]

\ProductId должен обслуживаться, управляться и ремонтироваться только квалифицированным персоналом, обученным в гарантийной службе \Boschung.

Полномочия для управления, обслуживания и ремонта выдаются гарантийной службой \Boschung.

%\adaptlayout [height=+5mm]


\subsubsection{Изменения и модификации}

\starttextbackground[CB]
\startPictPar
\PHgeneric
\PictPar
\textDescrHead{Опасность несчастного случая}
Любые\index{Модификация транспортного средства} самовольные модификации транспортного средства могут привести к ухудшению работоспособности и эксплуатационной надежности \ProductId и, тем самым, повлечь за собой непредсказуемый риск аварий и травм.
\stopPictPar

\startPictPar
\PMwarranty
\PictPar
В случае повреждений, возникших вследствие\index{Гарантия+Условия} самовольных операций или модификаций \ProductId или одного из агрегатов, \Boschung\ не несет никакой ответственности по гарантии и не делает никаких уступок.
\stopPictPar
\stoptextbackground

\stopsection


\startsection [title={Рабочие материалы и смазочные средства}, reference={sec:liquids}]


\subsection{Правильное обращение}

\starttextbackground[CB]
\startPictPar
\PHpoison
\PictPar
\textDescrHead{Опасность травмы и отравления}
\index{Топливо} Попадание на кожу\index{Смазочное средство} или\index{Опасность+Отравление} проглатывание рабочих материалов и смазочных средств может\index{Топливо+Безопасность} приводить к серьезным травмам или отравлениям. При хранении, обращении и утилизации этих материалов всегда соблюдайте официальные предписания.
\stopPictPar
\stoptextbackground

\starttextbackground [FC]
\startPictPar
\PMproteyes\par
\PMprothands
\PictPar
При работе с рабочими и смазочными материалами всегда надевайте соответствующую защитную одежду и средство защиты органов дыхания. Избегайте ингаляции паров.
Избегайте попадания материалов на кожу, в глаза или на одежду. Немедленно очищайте водой и мылом участки кожи, контактировавшие с рабочими материалами. При попадании рабочего материала в глаза, промойте их обильно чистой водой и обратитесь к глазному врачу. При проглатывании рабочих материалов следует немедленно обратиться к врачу!
\stopPictPar
\stoptextbackground

\startSymList
\PPchildren
\SymList
Рабочие материалы должны храниться вне досягаемости для детей.
\stopSymList

\startSymList
\PPfire
\SymList
\textDescrHead{Пожароопасность}
Ввиду\index{Опасность+Огонь} высокой воспламеняемости рабочих материалов при обращении с ними существует повышенный риск пожара. При обращении с рабочими материалами строго запрещается курить, подносить огонь\index{Запрещение курения} и незащищенные источники света.
\stopSymList


%% TODO; en
\starttextbackground [FC]
\startPictPar
\PMgeneric
\PictPar
Разрешается использовать только смазочные материалы, подходящие для компонентов \ProductId. Поэтому используйте только проверенные и разрешенные компанией \Boschung\ продукты. Вы можете найти их в списке рабочих материалов \atpage[sec:liqquantities]. Присадки\index{Присадки} для смазочных средств не требуются. Использование присадок может привести к аннулированию гарантии\index{Гарантия+Условия}.
Для подробной информации обращайтесь в гарантийную службу \Boschung.
\stopPictPar
\stoptextbackground

\starttextbackground [FC]
\startPictPar
\Penvironment
\PictPar
\textDescrHead{Охрана окружающей среды}
При утилизации\index{Смазочное средство+Утилизация} рабочих и смазочных материалов\crlf \index{Охрана окружающей среды} или загрязненных ими объектов (\eG\ фильтры, тряпки), обращайте внимание\index{Рабочие материалы+Утилизация} на соблюдение правил охраны окружающей среды.
\stopPictPar
\stoptextbackground

\page [yes]

\setups [pagestyle:normal]


\subsection[sec:liqquantities]{Спецификации и количества наполнения}

Все количества наполнения\index{Рабочие материалы+Количество наполнения}\index{Смазочное средство+Количество наполнения}\index{Количество наполнения+Рабочие материалы и смазочные средства}\index{Спецификации+Рабочие материалы и смазочные средства}, представленные в нижеследующей таблице, являются ориентировочными. После каждой замены рабочих материалов и смазочных средств необходимо контролировать фактический уровень наполнения и, при необходимости, принять меры к увеличению или уменьшению уровня.

\placetable[margin][tab:glyco]{Защита от мороза (\index{Защита от мороза}двигатель)}
{\noteF\startframedcontent[FrTabulate]
%\starttabulate[|Bp(80pt)|r|r|]
\starttabulate[|Bp|r|r|]
\NC Защита от замерзания до {[}°C{]}\NC \bf \textminus 25 \NC \bf \textminus 40 \NC\NR
\NC Дистилл. вода [Vol.-\%] \NC 60 \NC 40 \NC\NR
\NC Антифриз \break [Vol.-\%] \NC 40 \NC {\em макс.} 60 \NC\NR
\stoptabulate\stopframedcontent\endgraf
Внимание: При объемном проценте антифриза более 60\hairspace\percent\ {\em снижается} степень защиты от замерзания и ухудшается эффективность охлаждения!}

\placefig[margin][fig:hydrgauge]{\select{caption}{Индикатор уровня гидравлической жидкости (на левой стороне транспортного средства)}{Индикатор уровня гидравлической жидкости}}
{\externalfigure[main:hy:level_temp]
\noteF Уровень наполнения гидравлического резервуара просматривается через смотровое стекло и подлежит{\em ежедневному} контролю.}

%\placetable[here,split][tbl:liquids]{Spécifications et volumes de remplissage des consommables}
%{\readfile{tbl_jb-fr_liquids}{}{\Warn}}


\vskip -8pt
\start
\stdfontcond
\define [1] \TableSmallSymb {\externalfigure[#1][height=4ex]}
\define\UC\emptY
\pagereference[page:table:liquids]

\setupTABLE [frame=off,style={\ssx\setupinterlinespace[line=.86\lH]},background=color,
option=stretch,
split=repeat]
\setupTABLE [r] [each] [topframe=on,
framecolor=TableWhite,
% rulethickness=.8pt
]

\setupTABLE [c] [odd] [backgroundcolor=TableMiddle]
\setupTABLE [c] [even] [backgroundcolor=TableLight]
\setupTABLE [c] [1][width=30mm]
\setupTABLE [c] [2][width=20mm]
\setupTABLE [c] [4][width=28mm]
\setupTABLE [c] [last] [width=10mm]
\setupTABLE [r] [first] [topframe=off,style={\bfx\setupinterlinespace[line=.95\lH]},
% backgroundcolor=TableDark
]
\setupTABLE [r] [2][framecolor=black]

\bTABLE

\bTABLEhead
\bTR
\bTC Группа \eTC
\bTC Категория \eTC
\bTC Классификация \eTC
\bTC Продукт\note[Produkt] \eTC
\bTC Количество \eTC
\eTR
\eTABLEhead

\bTABLEbody
\bTR \bTD Дизельный двигатель \eTD
\bTD Моторное масло\eTD
\bTD \liqC{SAE 5W-30}; \liqC{VW 507.00}\eTD
\bTD Total Quartz INEO Long Life \eTD
\bTD 4,3 l\eTD
\eTR
\bTR \bTD Гидравлический контур \eTD
\bTD ATF-Öl \eTD
\bTD \liqC{dexron iii} \eTD
\bTD Total Equiviz ZS 46 (бак ок. 40 л) \eTD
\bTD ок. 50 л\eTD
\eTR
\bTR \bTD Гидравлический контур (опция у \aW{Bio})\eTD
\bTD ATF-Öl \eTD
\bTD \liqC{dexron iii} \eTD
\bTD Total Biohydran TMP SE 46\eTD
\bTD ок. 50 л\eTD
\eTR
\bTR \bTD Магнитные клапаны: Сердечники катушек \eTD
\bTD Смазочное средство\eTD
\bTD Медная консистентная смазка \eTD
\bTD \emptY\eTD
\bTD по потр.\note[Bedarf] \eTD
\eTR
\bTR \bTD Разное: замки, дверной механизм, педаль тормоза \eTD
\bTD Смазочное средство\eTD
\bTD Универсальный спрей\eTD
\bTD \emptY\eTD
\bTD по потр.\note[Bedarf] \eTD
\eTR
\bTR \bTD Система централизованной смазки \eTD
\bTD Универсальная подшипниковая смазка\eTD
\bTD \liqC{nlg2}\eTD
\bTD Total Multis EP 2\eTD
\bTD по потр.\note[Bedarf] \eTD
\eTR
\bTR \bTD Система охлаждения \eTD
\bTD Антифриз/антикоррозионное средство\eTD
\bTD TL VW 774 F/G; макс. 60\hairspace\% по объему\eTD
\bTD G12+/G12++ (розовый/фиолетовый)\eTD
\bTD ок. 14 л\eTD
\eTR
\bTR \bTD Водяной насос высокого давления \eTD
\bTD Моторное масло\eTD
\bTD \liqC{SAE 10W-40}; \liqC{api cf – acea e6}\eTD
\bTD Total Rubia TIR 8900 \eTD
\bTD 0,29 л\eTD
\eTR
\bTR \bTD Кондиционер \eTD
\bTD Хладагент\eTD
\bTD + 20 мл масла POE\eTD
\bTD R 134a\eTD
\bTD 700 г\eTD
\eTR
\bTR \bTD Стеклоомыватель \eTD
\bTD [nc=2] Вода и концентрат моющей жидкости для стекла, \aW{S} лето, \aW{W} зима с подходящим соотношением смеси \eTD
\bTD Розничная торговля \eTD
\bTD по потр.\note[Bedarf] \eTD
\eTR
\eTABLEbody

\eTABLE

\stop
\footnotetext[Bedarf]{{\it по потр.} по потребности, в соответствии с инструкцией}
\footnotetext[Produkt]{Используемые \Boschung\ продукты. Допускается также использование других продуктов, соответствующих спецификациям.}

\stopsection

\page [yes]

\setups [pagestyle:marginless]


\startsection [title={Техобслуживание дизельного двигателя},
reference={sec:workshop:vw},
]


\subsection [sSec:vw:diagTool]{Бортовая система диагностики}

\startregister[index][reg:main:vw]{Техобслуживание+Дизельный двигатель} блок управления двигателем (J623) оснащен памятью ошибок.
При возникновении ошибок контролируемых датчиков или компонентов, информация сохраняется в памяти вместе с указанием вида ошибки.

\index{Дизельный двигатель+Диагностика} блок управления двигателем классифицирует ошибки по классам и хранит их вплоть до очистки памяти ошибок.

Ошибки, возникающие лишь {\em спорадически}, отображаются с обозначением \aW{SP}. Причиной спорадических ошибок может быть \eG\ неплотный контакт или короткий сбой на линии. Если спорадическая ошибка более не возникнет в течение 50 пусков двигателя, она будет удалена из памяти.

При обнаружении ошибок, влияющих на характер работы двигателя, на экране Vpad загорится символ \aW{Диагностика двигателя} \textSymb{vpadWarningEngine1}.

Просмотреть сохраненные сообщения об ошибках можно в системе диагностики, измерения и информации \aW{VAS 5051/B}.

После устранения ошибок память следует очистить.


\subsubsection[sSec:vw:diagTool:connect]{Ввод системы диагностики в эксплуатацию}

\starttextbackground [FC]
\startPictPar
\PMgeneric
\PictPar
Подробные сведения о системе диагностики машины VAS 5051/B приведены в соответствующем руководстве.

Можно использовать и другие совместимые системы диагностики, \eG\ \aW{DiagRA}.
\stopPictPar
\stoptextbackground

\page [yes]


\subsubsubsubject{Необходимые условия}

\startitemize
\item Предохранители должны быть в порядке.
\item Напряжение аккумулятора должно превышать 11,5 В.
\item Все потребители электроэнергии должны быть отключены.
\item Соединение с массой должно быть в порядке.
\stopitemize


\subsubsubsubject{Способ действия}

\startSteps
\item Вставьте штекер кабеля диагностики VAS 5051B/1 в соответствующий разъем.
\item В зависимости от функции включите зажигание или запустите двигатель.
\stopSteps

\subsubsubsubject{Выбор режима работы}

\startSteps [continue]
\item На дисплее нажмите кнопку \aW{Самодиагностика машины}.
\stopSteps


\subsubsubsubject{Выбор системы машины}

\startSteps [continue]
\item На дисплее нажмите кнопку \aW{01-Электроника двигателя}.
\stopSteps

На дисплее появится идентификационная информация блоков управления и код блока управления двигателя.

Если коды не совпадают, следует проверить кодировку блоков управления.


\subsubsubsubject{Выбор функции диагностики}

На дисплее отображаются все доступные функции диагностики.

\startSteps [continue]
\item На дисплее нажмите кнопку нужной функции.
\stopSteps



\subsection [sSec:vw:faultMemory]{Память ошибок}


\subsubsection{Просмотр памяти ошибок}

\subsubsubject{Порядок действий}

\startSteps
\item Включите двигатель на холостом ходу.
\item Подключите VAS 5051/B (см. \in{раздел}[sSec:vw:diagTool:connect]) и выберите блок управления двигателя.
\item Выберите функцию диагностики \aW{004-Память ошибок}.
\item Выберите функцию диагностики \aW{004.01-Запрос памяти ошибок}.
\stopSteps

{\sla Только если двигатель не запустится:}

\startitemize [2]
\item Включите зажигание.
\item Если в памяти нет ошибок блока управления, на дисплее появится \aW{Обнаружено 0 ошибок}.
\item Если в памяти есть ошибки блока управления, на дисплее они будут отображаться подряд.
\item Завершите диагностику.
\item Выключите зажигание.
\item Устраните ошибки в соответствии с указаниями в таблице (см. сервисную документацию) и очистите память ошибок.
\stopitemize

\starttextbackground [FC]
\startPictPar
\PMrtfm
\PictPar
Если удалить одну из ошибок невозможно, обратитесь в гарантийную службу \boschung.
\stopPictPar
\stoptextbackground


\subsubsubject{Статические ошибки}

Если в памяти есть одна или несколько статических ошибок, обратитесь в гарантийную службу Boschung, чтобы устранить их в рамках \aW{поиска ошибок с поддержкой}.


\subsubsubject{Спорадические ошибки}

Если в памяти есть только спорадические ошибки или указания и сбои в электронике машины не обнаружены, память ошибок можно очистить:

\startSteps [continue]
\item Еще раз нажмите кнопку \aW{Далее} \inframed[strut=local]{>}, чтобы перейти к плану проверки.
\item Чтобы завершить поиск ошибок с поддержкой, нажмите кнопку \aW{Переход} и \aW{Завершить}.
\stopSteps

Во все памяти ошибок будут еще раз направлены запросы.

Откроется окно с сообщением, что все спорадические ошибки удалены. Протокол диагностики будет отправлен автоматически (онлайн).

Тест системы завершен.


\subsubsection[sSec:vw:faultMemory:errase]{Очистка памяти ошибок}

\subsubsubject{Порядок действий}

{\sla Необходимые условия:}

\startitemize [2]
\item Все ошибки и их причины устранены.
\stopitemize

\page [yes]


{\sla Способ действия:}

\starttextbackground [FC]
\startPictPar
\PMrtfm
\PictPar
После устранения ошибок нужно снова запросить и затем очистить память ошибок:
\stopPictPar
\stoptextbackground

\startSteps
\item Включите двигатель на холостом ходу.
\item Подключите VAS 5051/B (см. \in{раздел}[sSec:vw:diagTool:connect]) и выберите блок управления двигателя.
\item Выберите функцию диагностики \aW{004-Запрос памяти ошибок}.
\item Выберите функцию диагностики \aW{004.10-Очистка памяти ошибок}.
\stopSteps

\starttextbackground [FC]
\startPictPar
\PMrtfm
\PictPar
Если очистить память нельзя, значит имеется ошибка, которую еще следует устранить.
\stopPictPar
\stoptextbackground

\startSteps [continue]
\item Завершите диагностику.
\item Выключите зажигание.
\stopSteps


\subsection [sSec:vw:lub] {Смазка дизельного двигателя}

\subsubsection [ssSec:vw:oilLevel] {Проверка уровня масла в двигателе}

\starttextbackground [FC]
\startPictPar
\PMrtfm
\PictPar
\index{Моторное масло+Уровень} Уровень моторного масла ни в коем случае не должен превысить отметку \aW{Макс.}. В противном случае возможно \index{Уровень+Моторное масло} повреждение катализатора.
\stopPictPar
\stoptextbackground

\startSteps
\item Выключите двигатель и подождите минимум 3 минуты, чтобы масло стекло обратно в поддон.
\item Выньте щуп и протрите начисто; снова вставьте щуп до упора.
\item Выньте щуп и проверьте уровень масла:

\startfigtext[right][fig:vw:gauge]{Проверка уровня масла}
{\externalfigure[VW_Oil_Gauge][width=50mm]}
\startitemize [A]
\item Максимальный уровень; доливать масло нельзя.
\item Достаточный уровень; масло {\em можно} долить до отметки \aW{A}.
\item Недостаточный уровень; масло {\em нужно} долить до отметки \aW{B}.
\stopitemize
{\em Если уровень находится выше отметки \aW{A}, есть опасность повреждения катализатора.}

\stopfigtext
\stopSteps


\subsubsection [ssSec:vw:oilDraining] {Замена моторного масла}

\starttextbackground [FC]
\startPictPar
\PMrtfm
\PictPar
Фильтр моторного масла в S2 смонтирован вертикально. Это значит, что фильтр нужно менять {\em до} замены масла. Если вынуть фильтрующий элемент, откроется клапан, и масло из корпуса фильтра автоматически стечет в картер.
\stopPictPar
\stoptextbackground

\startSteps
\item Поставьте под двигатель подходящую емкость\index{Дизельный двигатель+Замена масла}.
\item Выверните пробку сливного отверстия\index{Моторное масло+Замена} и слейте масло.
\stopSteps

\starttextbackground [FC]
\startPictPar
\PMrtfm
\PictPar
Учтите, что объем емкости должен соответствовать объему старого масла.
Спецификация и количество масла указаны в \in{разделе}[sec:liqquantities].

На пробке сливного отверстия есть несъемное уплотнительное кольцо. Поэтому пробку нужно менять каждый раз
\stopPictPar
\stoptextbackground

\startSteps [continue]
\item Вверните новую пробку с уплотнительным кольцом (\TorqueR 30 Нм).
\item Залейте масло согласно спецификации (см. \in{раздел}[sec:liqquantities]).
\stopSteps


\subsubsection [ssSec:vw:oilFilter] {Замена масляного фильтра двигателя}

\starttextbackground [FC]
\startPictPar
\PMrtfm
\PictPar
\startitemize [1]
\item Следуйте\index{Дизельный двигатель+Масляный фильтр} инструкциям по утилизации и вторичной переработке.
\item Замените\index{Масляный фильтр+Дизельный двигатель} фильтр {\em до} замены масла (см. \in{раздел}[ssSec:vw:oilDraining]).
\item Перед установкой слегка смажьте уплотнение нового фильтра.
\stopitemize
\stopPictPar
\stoptextbackground

\startfigtext[right][fig:vw:oilFilter]{Масляный фильтр}
{\externalfigure[VW_OilFilter_03][width=50mm]}
\startSteps
\item Отвинтите крышку \Lone\ фильтрующего элемента подходящей отверткой.
\item Очистите уплотняющие поверхности крышки и корпуса фильтра.
\item Замените фильтрующий элемент \Lthree.
\item Замените уплотняющие кольца \Ltwo\ и \Lfour.
\item Привинтите крышку к корпусу фильтра (\TorqueR 25 Нм).
\stopSteps



%\subsubsubject{Données techniques}
%
%
%\hangDescr{Couple de serrage du couvercle:} \TorqueR 25 Nm.
%
%\hangDescr{Huile moteur prescrite:} Selon tableau \atpage[sec:liqquantities].
%% NOTE: Redundant [tf]

\stopfigtext



\subsubsection [ssSec:vw:oilreplenish] {Дозаправка моторного масла}

\starttextbackground [FC]
\startPictPar
\PMrtfm
\PictPar
\startitemize [1]
\item Очистите\index{Моторное масло} тряпкой наливной патрубок {\em до} снятия крышки.
\item Заправляйте\index{Дизельный двигатель+Дозаправка масла} только то масло, которое соответствует спецификации.
\item Наливайте масло постепенно небольшими порциями.
\item Чтобы избежать перелива, после заправки каждой порции подождите немного, чтобы масло могло стечь в поддон до отметки на щупе (см. \in{раздел}[ssSec:vw:oilLevel]).
\stopitemize
\stopPictPar
\stoptextbackground

\startfigtext[right][fig:vw:oilFilter]{Доливка масла}
{\externalfigure[s2_bouchonRemplissage][width=50mm]}
\startSteps
\item Выньте измерительный щуп примерно на 10 см, чтобы при заправке мог выходить воздух.
\item Откройте наливное отверстие.
\item Залейте масло в соответствии с инструкцией выше.
\item Тщательно закройте наливное отверстие.
\item Включите двигатель.
\item Проверьте уровень. (См. \in{раздел}[ssSec:vw:oilLevel].)
\stopSteps

\stopfigtext


\subsection [sSec:vw:fuel] {Система подачи топлива}

\subsubsection [ssSec:vw:fuelFilter] {Замена топливного фильтра}

\starttextbackground [FC]
\startPictPar
\PMrtfm
\PictPar
\startitemize [1]
\item Следуйте\index{Дизельный двигатель+Топливный фильтр} требованиям закона по утилизации и вторичной переработке опасных отходов.
\item Не отсоединяйте топливопроводы в верхней части фильтра.
\item Не тяните за крепления топливопроводов; в противном случае верхняя частью фильтра может быть повреждена.
\stopitemize
\stopPictPar
\stoptextbackground

\startfigtext[right][fig:vw:oilFilter]{Топливный фильтр}
{\externalfigure[s2_fuelFilter_location][width=50mm]}

{\sla Подготовка:}

\index{Топливный фильтр} Корпус топливного фильтра закреплен перед двигателем на правой стороне шасси.
Выверните два крепежных винта 10-миллиметровым торцовым ключом и 10-миллиметровым накидным ключом.

\stopfigtext


\page [yes]

\setups [pagestyle:normal]

{\sla Способ действия:}

\startLongsteps
\item Выверните все винты из верхней части фильтра. Снимите верхнюю часть фильтра.
\stopLongsteps

\starttextbackground [FC]
\startPictPar
\PMrtfm
\PictPar
Поднимите верхнюю часть. Если необходимо, вставьте угловую отвертку в монтажную канавку (\in{\LAa, рис.}[fig:fuelfilter:detach]) и раскачайте верхнюю часть.
\stopPictPar
\stoptextbackground

\placefig [margin] [fig:fuelfilter:detach]{Демонтаж топливного фильтра}
{\externalfigure[fuelfilter:detach]}

\placefig [margin] [fig:fuelfilter:explosion]{Топливный фильтр}
{\externalfigure[fuelfilter:explosion]}

\startLongsteps [continue]
\item Выньте фильтрующий элемент из нижней части фильтра.
\item Снимите уплотнение (\in{\Ltwo, рис.}[fig:fuelfilter:explosion]) с верхней части фильтра.
\item Тщательно очистите нижнюю и верхнюю часть фильтра.
\item Установите новый фильтрующий элемент в нижнюю часть фильтра.
\item Покройте новое уплотнение (\in{\Ltwo, рис.}[fig:fuelfilter:explosion]) небольшим количеством топлива и вставьте в верхнюю часть.
\item Установите верхнюю часть на нижнюю часть фильтра и равномерно прижмите, чтобы она прилегала по всей окружности.
\item Соедините верхнюю и нижнюю части всеми винтами, затянув их {\em вручную}. Затем затяните все винты крест-накрест предписанным моментом (\TorqueR 5 Нм).
\stopLongsteps

% \subsubsubject{Données techniques}
%
% \hangDescr{Couple de serrage des vis de fixation du couvercle:} \TorqueR 5 Nm.
%% NOTE: redundant [tf]

\startLongsteps [continue]
\item Включите зажигание, чтобы удалить воздух из системы; запустите двигатель и дайте ему проработать 1 - 2 минуты на холостом ходу.
\item Очистите память ошибок в соответствии с описанием в \atpage[sSec:vw:faultMemory:errase].
\stopLongsteps

\testpage [8]

\subsection [sSec:vw:cooling] {Система охлаждения}

\starttextbackground [FC]
\startPictPar
\PMrtfm
\PictPar
\startitemize [1]
\item Использовать\index{Дизельный двигатель+Охлаждение} только охлаждающую жидкость, соответствующую спецификации (см. таблицу \atpage[sec:liqquantities]).
\item Для\index{Хладагент} обеспечения защиты от замерзания и коррозии охлаждающую жидкость можно разбавлять только дистиллированной водой в соответствии с таблицей, приведенной ниже.
\item Запрещается заливать в контур охлаждения воду, так как это нарушает защиту от замерзания и коррозии.
\stopitemize
\stopPictPar
\stoptextbackground


\subsubsection [sSec:vw:coolingLevel] {Уровень охлаждающей жидкости}

\placefig [margin] [fig:coolant:level] {Уровень охлаждающей жидкости}
{\externalfigure[coolant:level]}


\placefig [margin] [fig:refractometer] {Рефрактометр VW T 10007}
{\externalfigure[coolant:refractometer]}

\placefig [margin] [fig:antifreeze] {Контроль плотности антифриза}
{\externalfigure[coolant:antifreeze]}


\startSteps
\item Поднимите бункер и установите защитную подпорку.
\item Определите\index{Уровень+Охлаждающая жидкость} уровень охлаждающей жидкости в расширительном бачке: Он должен находиться выше \aW{минимальной} отметки.
\stopSteps

\start
\define [1] \TableSmallSymb {\externalfigure[#1][height=4ex]}
\define\UC\emptY
\pagereference[page:table:liquids]


\setupTABLE [frame=off,style={\ssx\setupinterlinespace[line=.86\lH]},background=color,
option=stretch,
split=repeat]
\setupTABLE [r] [each] [topframe=on,
framecolor=TableWhite,
% rulethickness=.8pt
]

\setupTABLE [c] [odd] [backgroundcolor=TableMiddle]
\setupTABLE [c] [even] [backgroundcolor=TableLight]
\setupTABLE [r] [first] [topframe=off,style={\bfx\setupinterlinespace[line=.95\lH]},
% backgroundcolor=TableDark
]
\setupTABLE [r] [2][framecolor=black]

\bTABLE

\bTABLEhead
\bTR
\bTC Антифриз до … \eTC
\bTC Доля G12\hairspace ++\eTC
\bTC Объем антифриза \eTC
\bTC Объем дист. воды \eTC
\eTR
\eTABLEhead

\bTABLEbody
\bTR \bTD \textminus 25 °C \eTD
\bTD 40\hairspace\% \eTD
\bTD 3,8 л \eTD
\bTD 4,2 л \eTD
\eTR
\bTR \bTD \textminus 35 °C \eTD
\bTD 50\hairspace\% \eTD
\bTD 4,0 л \eTD
\bTD 4,0 л \eTD
\eTR
\bTR \bTD \textminus 40 °C \eTD
\bTD 60\hairspace\% \eTD
\bTD 4,2 л \eTD
\bTD 3,8 л \eTD
\eTR
\eTABLEbody

\eTABLE
\stop

\adaptlayout [height=+20pt]
\subsubsection [sSec:vw:coolingFreeze] {Уровень охлаждающей жидкости}

Проверьте\index{Плотность антифриза} плотность антифриза подходящим рефрактометром (см. \in{рис.}[fig:refractometer]: VW T 10007).
См. шкалу 1: G12\hairspace ++ (см. \in{рис.}[fig:antifreeze]).

\page [yes]


\subsection [sSec:vw:airFilter] {Подача воздуха}

Получить доступ к воздушному фильтру можно через заднюю дверцу для обслуживания на правой стороне машины (см. \in{рис.}[fig:airFilter]).

\placefig [margin] [fig:airFilter] {Воздушный фильтр двигателя}
{\externalfigure[vw:air:filter]
\noteF
\startLeg
\item Защитная накладка
\item Нижняя часть корпуса
\item Вентиляционное отверстие
\item Датчик давления
\stopLeg}


\subsubsubject{Условия использования}

Уборочные машины часто используются в сильно запыленных условиях. По этой причине воздушный фильтр следует проверять и чистить ежедневно. См. также \about[table:scheduleweekly], \atpage[table:scheduleweekly]. Если необходимо, замените воздушный фильтр.


\subsubsubject{Самодиагностика}

Во всасывающей линии есть датчик давления (\Lfour, \in{рис.}[fig:airFilter]), благодаря которому можно определить вызванные фильтром потери потока\footnote{Падение расхода воздуха из-за снижения воздушной проницаемости фильтра.}.
Если воздушный фильтр засорен, на экране Vpad горит контрольный символ \textSymb{vpadWarningFilter}, и появляется сообщение об ошибке \VpadEr{851}.


\subsubsubject{Ремонт/замена}

\startSteps
\item Потяните защитную накладку \Lone вниз (\in{рис.}[fig:airFilter]).
\item Поверните нижнюю часть корпуса \Ltwo против часовой стрелки и снимите.
\item Выньте фильтрующий элемент и проверьте. При необходимости – замените.
\item Очистите воздушный фильтр изнутри и соберите его в обратной последовательности.
\stopSteps

\page [yes]


\subsection [sSec:vw:belt] {Поликлиновой ремень}

Поликлиновой ремень\index{Дизельный двигатель+Поликлиновой ремень} передает движение маховика коленчатого вала на генератор и компрессор кондиционера (дополнительное оснащение).
Натяжной элемент\index{Поликлиновой ремень} на последнем сегменте (между генератором и коленчатым валом) поддерживает натяжение ремня.


\subsubsection [sSec:belt:change] {Замена поликлинового ремня}

\placefig [margin] [fig:belt:tool] {Стопорный штифт VW T 10060 A}
{\externalfigure[vw:belt:tool]}

\placefig [margin] [fig:belt:overview] {Натяжной элемент}
{\externalfigure[vw:belt:overview]}

\placefig [margin] [fig:belt:tens] {Место установки стопорного штифта}
{\externalfigure[vw:belt:tens]}


\subsubsubject{С компрессором кондиционера}


{\sla Требуемые специальные инструменты:}

Стопорный штифт \aW{VW T 10060 A} для удержания натяжного элемента.

\startSteps
\item Отметьте направление движения ремня.
\item С помощью изогнутого накидного ключа поверните кронштейн натяжного элемента по часовой стрелке (\in {рис.}[fig:belt:overview]).
\item Совместите отверстия (см. стрелки, \in {рис.}[fig:belt:tens]) и зафиксируйте натяжной элемент стопорным штифтом.
\item Снимите ремень.
\stopSteps

Der Монтаж поликлинового ремня производится в обратном порядке.

\starttextbackground [FC]
\startPictPar
\PMrtfm
\PictPar
\startitemize [1]
\item Не перепутайте направление движения ремня.
\item Убедитесь в правильном положении ремня в шкивах.
\item Запустите двигатель и проверьте ход ремня.
\stopitemize
\stopPictPar
\stoptextbackground


\subsubsubject{Без компрессора кондиционера}

{\sla Требуемые материалы:}

Ремонтный комплект, состоящий из инструкции, поликлинового ремня и специального инструмента.\footnote{См. каталог запасных частей, раздел \aW{Детали для техобслуживания}.}

\startSteps
\item Разделите поликлиновой ремень.
\item Далее следуйте инструкции по ремонту.
\stopSteps

\starttextbackground [FC]
\startPictPar
\PMrtfm
\PictPar
\startitemize [1]
\item Убедитесь в правильном положении ремня в шкивах.
\item Запустите двигатель и проверьте ход ремня.
\stopitemize
\stopPictPar
\stoptextbackground


\subsubsection [sSec:belt:tens] {Замена натяжного элемента}

{\sla Только для исполнения с компрессором кондиционера}

\blank [medium]

\placefig [margin] [fig:belt:tens:change] {Замена натяжного элемента}
{\externalfigure[vw:belt:tens:change]
\noteF
\startLeg
\item Крепежный элемент
\item Стопорный винт
\stopLeg

{\bf Момент затяжки}

Стопорный винт:

\TorqueR 20 Нм\:+ ½ оборота (180°).}

\startSteps
\item Снимите поликлиновой ремень в соответствии с описанием (см. \atpage[sSec:belt:change]).
\item Демонтируйте периферийные устройства (в зависимости от оснащения).
\item Выверните стопорный винт (\in{\Ltwo, рис.}[fig:belt:tens:change]).
\stopSteps

Монтаж натяжного элемента производится в обратном порядке.

\starttextbackground [FC]
\startPictPar
\PMrtfm
\PictPar
\startitemize [1]
\item После монтажа обязательно используйте новый стопорный винт.
\item Момент затяжки: См. \in{рис.}[fig:belt:tens:change].
\stopitemize
\stopPictPar
\stoptextbackground

\stopregister[index][reg:main:vw]

\stopsection

\page[yes]


\setups[pagestyle:marginless]


\startsection[title={Гидравлическая система},
reference={sec:hydraulic}]

\starttextbackground [FC]
% \startfiguretext[left,none]{}
% {\externalfigure[toni_melangeur][width=30mm]}

\startSymPar
\externalfigure[toni_melangeur][width=4em]
\SymPar
\textDescrHead{Утилизация рабочих материалов}
Отработанные рабочие материалы и смазочные средства запрещается выбрасывать на природе или сжигать.

Отработанные смазочные средства запрещается сливать в канализацию, сбрасывать в природу или выбрасывать вместе с бытовыми отходами.

Отработанные смазочные средства запрещается смешивать с другими жидкостями, так как существует опасность, что в этой смеси окажутся токсичные вещества или трудноустранимые материалы.
\stopSymPar
\stoptextbackground
\blank [big]

% \starthangaround{\PMgeneric}
% \textDescrHead{Qualification du personnel}
% Toute intervention sur l’installation hydraulique de votre véhicule ne peut être réalisée que par une personne dument qualifiée, ou par un service reconnu par \boschung.
% \stophangaround
% \blank[big]

\startSymList
\PHgeneric
\SymList
\textDescrHead{Чистота} Гидравлическая система чувствительно реагирует на загрязнения, возникающие в гидравлическом масле. В этой связи важно работать в абсолютно чистой среде.
\stopSymList

\startSymList
\PHhot
\SymList
\textDescrHead{Опасность брызг}
Перед работами на гидравлической системе \sdeux\ необходимо сбросить остаточное давление в соответствующих агрегатах. Горячие брызги масла могут стать причиной ожогов.
\stopSymList

\startSymList
\PHhand
\SymList
\textDescrHead{Опасность раздавливания}
До начала работ с гидравлической системой \sdeux\ следует обязательно опустить бункер для грязи или установить на защитные опоры.
\stopSymList

\startSymList
\PImano
\SymList
\textDescrHead{Измерение давления}
Чтобы измерить давление гидравлики, присоедините манометр к одной из муфт \aW{Minimess} системы. Убедитесь, что у манометра подходящий диапазон измерения.
\stopSymList

\page [yes]

\setups[pagestyle:normal]

\subsection{Интервалы обслуживания}

\start

\setupTABLE [frame=off,
style={\ssx\setupinterlinespace[line=.93\lH]},
background=color,
option=stretch,
split=repeat]
\setupTABLE [r] [each] [
topframe=on,
framecolor=white,
backgroundcolor=TableLight,
% rulethickness=.8pt,
]

% \setupTABLE [c] [odd] [backgroundcolor=TableMiddle]
% \setupTABLE [c] [even] [backgroundcolor=TableLight]
\setupTABLE [c] [1][ % width=30mm,
style={\bfx\setupinterlinespace[line=.93\lH]},
]
\setupTABLE [r] [first] [topframe=off,
style={\bfx\setupinterlinespace[line=.93\lH]},
backgroundcolor=TableMiddle,
]
% \setupTABLE [r] [2][style={\ssBfx\setupinterlinespace[line=.93\lH]}]


\bTABLE

\bTABLEhead
\bTR\bTD Вид работы \eTD\bTD Периодичность \eTD\eTR
\eTABLEhead

\bTABLEbody
\bTR\bTD Проверка на течи \eTD\bTD Ежедневно \eTD\eTR
\bTR\bTD Контроль уровня гидравлического масла \eTD\bTD Ежедневно \eTD\eTR
\bTR\bTD Контроль гидравлических линий/шлангов; при необходимости – заменить \eTD\bTD 600 ч / 12 месяцев \eTD\eTR
\bTR\bTD Замена фильтра в обратной магистрали гидросистемы и всасывающего фильтра \eTD\bTD 600 ч / 12 месяцев \eTD\eTR
\bTR\bTD Смазывание сердечников катушек магнитных клапанов магнитной смазкой \eTD\bTD 600 ч / 12 месяцев \eTD\eTR
\bTR\bTD Замена гидравлического масла \eTD\bTD 1200 ч / 24 месяцев \eTD\eTR
\eTABLEbody
\eTABLE
\stop


\subsection[niveau_hydrau]{Уровень наполнения}

\placefig[margin][fig:hydraulic:level]{Уровень гидравлической жидкости}
{\externalfigure[hydraulic:level]
\noteF
\startLeg
\item Оптимальный уровень
\stopLeg}

Прозрачное смотровое стекло\index{Уровень+Гидравлическая жидкость}\index{позволяет +Гидравлическая система} позволяет визуально контролировать уровень гидравлического масла.
Если уровень гидравлического масла понизился, необходимо определить причину, прежде чем снова доливать масло до нужного уровня. Проводите замену в указанные сроки (таблица вверху) и придерживайтесь спецификаций на гидравлическую жидкость (таблица \at{страница}[sec:liqquantities]).


\subsubsection{Долив гидравлической жидкости}

Залейте гидравлическую жидкость, чтобы уровень достиг середины смотрового окошка.
Запустите двигатель и долейте жидкость, если это необходимо, до нужного уровня.


\subsection{Замена гидравлической жидкости}

Количество и спецификация на гидравлическую жидкость приведена в таблице на \at{странице}[sec:liqquantities].

\startSteps
\item Откройте наливное отверстие в гидравлическом баке.
\item Опорожните бак отсасывающим устройством или выньте пробку из сливного отверстия.

Сливное отверстие находится в нижней части бака перед левым задним колесом (\in{рис.}[fig:hydraulic:fluidDrain]).
\item Залейте гидравлическую жидкость, чтобы уровень достиг середины смотрового окошка.
Запустите двигатель и долейте жидкость, если это необходимо, до нужного уровня.
\stopSteps

\placefig[margin][fig:hydraulic:fluidDrain]{Пробка сливного отверстия}
{\externalfigure[hydraulic:fluidDrain]}


\placefig[margin][fig:hydraulic:returnFilter]{Гидравлический фильтр}
{\externalfigure[hydraulic:returnFilter]}

\subsection[filtres:nettoyage]{Фильтр обратной магистрали и всасывающий фильтр}

\startSteps
\item Поднимите бункер и установите защитную подпорку.
\item Снимите крышку фильтра на гидравлическом баке (\in{рис.}[fig:hydraulic:returnFilter]).
\item Замените\index{Масляный фильтр+Гидравлика} фильтрующий элемент новым.
\item Покройте новое уплотнительное кольцо круглого сечения небольшим количеством гидравлической жидкости и установите кольцо.
\item Снова наверните крышку обеими руками (\TorqueR ок. 20 Нм).
\stopSteps

\page [yes]


\subsection[sec:solenoid]{Смазка магнитных клапанов}
\placefig[margin][graissage_bobine]{Смазка магнитных клапанов}
{\externalfigure[graissage_bobine][M]
\noteF
\startLeg
\item Катушка магнитного клапана
\item Сердечник катушки
\stopLeg}

Влага и остатки соли, попадающие в сердечники электромагнитных катушек, ведут к их коррозии. Сердечники катушек следует покрывать медной смазкой один раз в год. Смазка должна быть стойкой к коррозии, воздействию воды и температур до 50 °C:
\startSteps
\item Демонтируйте катушку магнитного клапана (\in{\Lone, рис.}[graissage_bobine]).
\item Смажьте сердечник (\in{\Ltwo, рис.}[graissage_bobine]) предписанной специальной смазкой и смонтируйте катушку.
\stopSteps


\subsection{Замена шлангов}

Защитная резина\index{Шланги+Периодичность замены} и усиливающая ткань шлангов подвержены естественному старению. Кроме того, шланги гидравлической системы нужно обязательно менять в указанные сроки, даже если {\em видимые} повреждения отсутствуют.

Шланги должны быть правильно закреплены на машине фланцами, чтобы трение не привело к преждевременному износу. Для предотвращения повреждений от трения и вибрации расстояние между шлангами и другими компонентами должно быть достаточным.

\stopsection

\page [yes]

\setups [pagestyle:bigmargin]


\startsection[title={Тормозная система},
reference={sec:brake}]

\placefig[margin][fig:brake:rear]{Барабанный тормоз}
{\startcombination [1*2]
{\externalfigure[brake:wheelHub]}{\slx Ступица заднего колеса}
{\externalfigure[brake:drum]}{\slx Механизм и тормозные колодки}
\stopcombination}

При каждом регулярном обслуживании тормозные барабаны \Lfour\ следует демонтировать, чистить тормозной механизм \Lseven\ и осматривать тормозные колодки \Lfive, \Lsix\ (\in{рис.}[fig:brake:rear]).


\subsubject {Демонтаж}

\startSteps
\item Поставьте машину на подходящую подъемную платформу и поднимите колеса.
\item Снимите колеса.
\stopSteps


{\sla Демонтаж тормозов передних колес}

\startSteps [continue]
\item Демонтируйте тормозной барабан \Lfour.
\stopSteps

{\sla Демонтаж тормозов задних колес}

\startSteps [continue]
\item Снимите крышку \Lone\ со ступицы.
\item Выверните винт \Ltwo\ и выньте прокладку.
\item Отверните гайку ступицы \Lthree\ торцовым ключом.
\item Снимите ступицу с тормозного барабана.
\stopSteps


\subsubject {Монтаж}

Соберите тормозной барабан в обратной последовательности. Затяните гайки ступиц задних колес \Lthree\ моментом 190 Нм.

\stopsection

\page [yes]

\setups [pagestyle:normal]


\startsection[title={Проверка и обслуживание шин},
reference={sec:pneumatiques}]

Шины\index{Шины+Техобслуживание} быть всегда в безукоризненном состоянии, с тем чтобы обеспечить выполнение двух основных функций: хорошее сцепление с дорогой и безупречное торможение. Недопустимо высокий износ и неправильное давление, в особенности слишком низкое, являются важными факторами аварий.


\subsection{Работы, важные для безопасности}

\subsubsection{Контроль износа}

Износ шин определяется по индикаторам износа в профильной канавке (\in{рис.}[pneususure]).
Необычное состояние шин и его причины можно определить во время осмотра:

\placefig[margin][pneususure]{Контроль износа}
{\Framed{\externalfigure[pneusUsure][M]}}

\placefig[margin][pneusdomages]{Поврежденные шины}
{\Framed{\externalfigure[pneusDomages][M]}}

\startitemize
\item Износ по сторонам ходовой поверхности: слишком низкое давление.
\item Повышенный износ средней части: слишком высокое давление.
\item Несимметричный износ на сторонах шины: неправильно отрегулирована передняя ось (колея, геометрия оси).
\item Трещины на ходовой поверхности: слишком старые шины, со временем резина стареет и растрескивается (\in{рис.}[pneusdomages]).
\stopitemize

\starttextbackground[CB]
\startPictPar
\PHgeneric
\PictPar
\textDescrHead{Риски, связанные с износом шин}
Изношенная шина не выполняет своих функций, в особенности, в отношении отвода воды и грязи; тормозной путь увеличивается, ходовые качества ухудшаются. Изношенная шина легче скользит, прежде всего, на мокрой дороге. Опасность потери сцепления увеличивается.
\stopPictPar
\stoptextbackground


\subsubsection{Давление в шинах}

Предписанное давление шин указано на заводской табличке колес, на консоли спереди со стороны помощника водителя (см. \atpage [sec:plateWheel]).

Даже \index{Шины+Давление} если шины находятся в хорошем состоянии, со временем, более или менее быстро, они выпускают воздух (чем больше приходится ездить, тем сильнее падает давление). Поэтому давление шин необходимо ежемесячно контролировать при холодных шинах. При проверке разогретых шин к предписанному давлению следует добавить 0,3 бар.

\start
\setupcombinations[M]
\placefig[margin][pneuspression]{Давление в шинах}
{\Framed{\externalfigure[pneusPression][M]}
\noteF
\startLeg
\item Правильное давление
\item Слишком высокое давление
\item Слишком низкое давление
\stopLeg
Предписанное давление шин указано на заводской табличке колес, в кабине со стороны помощника водителя.}
\stop

\starttextbackground[CB]
\startPictPar
\PHgeneric
\PictPar
\textDescrHead{Опасности, связанные с недостаточным давлением в шинах}
При недостаточном давлении шина может порваться. Шина сильнее сжимается, если она недостаточно накачана или машина перегружена. В результате резина нагревается и части шины могут отойти при прохождении поворота.
\stopPictPar
\stoptextbackground

\stopsection

\page [yes]

\setups[pagestyle:marginless]


\startsection[title={Шасси},
reference={main:chassis}]

\subsection{Крепления компонентов, важные для безопасности}

При каждом обслуживании следует проверять посадку крепежных винтов определенных компонентов, связанных с безопасностью, в частности, проверять предписанный момент затяжки. В частности, это относится к шарнирной системе и осям.

\blank [big]

\startfigtext [left] [fig:frontAxle:fixing] {Передняя ось}
{\externalfigure [frontAxle:fixing]}
{\sla Крепления передней оси}
\startLeg
\item Крепление рессорного листа: \TorqueR 150 Нм
\item Крепление тяговой секции: \TorqueR 78 Нм
\stopLeg

{\sla Крепления задней оси}
\startLeg
\item Крепление рессорного листа: \TorqueR 150 Нм
\stopLeg

\stopfigtext

\start

\setupTABLE [frame=off,style={\ssx\setupinterlinespace[line=.93\lH]},background=color,
option=stretch,
split=repeat]

\setupTABLE [r] [each] [topframe=on,
framecolor=white,
% rulethickness=.8pt
]

\setupTABLE [c] [odd] [backgroundcolor=TableMiddle]
\setupTABLE [c] [even] [backgroundcolor=TableLight]
\setupTABLE [c] [1][style={\bfx\setupinterlinespace[line=.93\lH]}]
\setupTABLE [r] [first] [topframe=off,style={\bfx\setupinterlinespace[line=.93\lH]},
]
% \setupTABLE [r] [2][style={\bfx\setupinterlinespace[line=.93\lH]}]


\bTABLE

\bTABLEhead
\bTR [backgroundcolor=TableDark] \bTD [nc=3] Моменты затяжки \eTD\eTR
% \bTR\bTD Position \eTD\bTD Type de vis \eTD\bTD Couple \eTD\eTR
\eTABLEhead

\bTABLEbody
\bTR\bTD Приводные двигатели слева/справа \eTD\bTD M12\:×\:35 8.8 \eTD\bTD 78 Нм \eTD\eTR
%% NOTE @Andrew: das sind Hydraulikmotoren
\bTR\bTD Рабочий насос \eTD\bTD M16\:×\:40 100 \eTD\bTD 330 Нм \eTD\eTR
\bTR\bTD Приводной насос \eTD\bTD M12\:×\:40 100 \eTD\bTD 130 Нм \eTD\eTR
\bTR\bTD Рессорные листы спереди/сзади \eTD\bTD M16\:×\:90/160 8.8 \eTD\bTD 150 Нм \eTD\eTR
% \bTR\bTD Fixation du système oscillant \eTD\bTD M12\:×\:40 8.8 \eTD\bTD 78 Nm \eTD\eTR
\bTR\bTD Крепление бункера \eTD\bTD M10\:×\:30 Verbus Ripp 100 \eTD\bTD 80 Нм \eTD\eTR
\bTR\bTD Гайки колеса \eTD\bTD M14\:×\:1,5 \eTD\bTD 180 Нм \eTD\eTR
\bTR\bTD Крепление передней щетки \eTD\bTD M16\:×\:40 100 \eTD\bTD 180 Нм \eTD\eTR
\eTABLEbody
\eTABLE
\stop


\stopsection

\page [yes]


\setups [pagestyle:marginless]


\startsection[title={План смазки (ручная смазка)},
reference={sec:grasing:plan}]

\starttextbackground [FC]
\startPictPar
\PMgeneric
\PictPar
В точки, указанные в плане смазки (\in{рис.}[fig:greasing:plan]), следует регулярно подавать смазку. Регулярное смазывание абсолютно необходимо для обеспечения {\em снижения трения на длительный срок} и предотвращения воздействия влаги и других субстанций, оказывающих коррозионное действие.
\stopPictPar
\stoptextbackground

\blank [big]

\start

\setupcombinations [width=\textwidth]

\placefig[here][fig:greasing:plan]{План смазки машины}
{\startcombination [3*1]
{\externalfigure[frame:steering:greasing]}{\ssx Управление шарнирным сочленением и качающийся механизм}
{\externalfigure[frame:axles:greasing]}{\ssx Оси}
{\externalfigure[frame:sucMouth:greasing]}{\ssx Патрубок всасывания}
\stopcombination}

\stop

\vfill

\startLeg [columns,three]
\item Подъемные цилиндры управления шарнирным сочленением\crlf {\sl 2 масленки на цилиндр}
\item Подшипник управления шарнирным сочленением\crlf {\sl 2 масленки на левой стороне}
\columnbreak
\item Подшипник качающегося механизма\crlf {\sl 1 масленка перед баком}
\item Листовые рессоры\crlf {\sl 2 масленки на лист}
\columnbreak
\item Патрубок всасывания\crlf {\sl 1 масленка на колесо}
\item Патрубок всасывания\crlf {\sl 1 масленка на тяговом кронштейне}
\stopLeg



\page [yes]


\setups [pagestyle:bigmargin]


\subsubject{Смазывание бункера для грязи}

Смазка наносится в 8 точках бункера (2\:×\:4) один раз в неделю.

\blank [big]


\placefig[here][fig:greasing:container]{Подъемный механизм бункера}
{\externalfigure[container:mechanisme]}


\placelegende [margin,none]{}
{{\sla Условные обозначения:}

\startLeg
\item Левая опора бункера
\item Правая опора бункера
\item Левый гидравлический цилиндр (вверху)
\item Левый гидравлический цилиндр (внизу)

{\em Как правый цилиндр (точка \in[greasing:point;hide]).}
\item Правый гидравлический цилиндр (вверху)
\item [greasing:point;hide] Правый гидравлический цилиндр (внизу)
\stopLeg}

\stopsection

\page [yes]



\startsection[title={Электрическая система},
reference={sec:main:electric}]

\subsection{Центральная электрическая система в шасси}

\startbuffer [fuses:preventive]
\starttextbackground [CB]
\startPictPar
\PHvoltage
\PictPar
\textDescrHead{Правила техники безопасности}
Следуйте правилам техники безопасности в\index{Предохранители+Шасси} данном\index{Реле+Шасси} руководстве: Для замены всегда используйте предохранители с предписанной силой тока; до начала работ с электрической\index{Электрическая система} системой снимите металлические украшения (кольца, браслеты и т.д.).
\stopPictPar
\stoptextbackground
\stopbuffer


\subsubsubject{Предохранители MIDI}

\starttabulate[|l|r|p|]
\HL
\NC\md F 1 \NC 5 A \NC Стоп-сигнал, \aW{+\:15} бортовая диагностика \NC\NR
\NC\md F 2 \NC 5 A \NC \aW{+\:15} Управление двигателем \NC\NR
\NC\md F 3 \NC 7,5 A \NC \aW{+\:30} Управление двигателем и бортовая диагностика \NC\NR
\NC\md F 4 \NC 20 A \NC Топливный насос \NC\NR
\NC\md F 5 \NC 20 A \NC \aW{D\:+} Генератор, \aW{+\:15} реле K 1 \NC\NR
\NC\md F 6 \NC 5 A \NC Управление двигателем \NC\NR
\NC\md F 7 \NC 10 A\NC Система обработки ОГ двигателя \NC\NR
\NC\md F 8 \NC 20 A \NC Электроника двигателя (управление) \NC\NR
\NC\md F 9 \NC 15 A \NC Система обработки ОГ двигателя, Топливный насос, прогрев \NC\NR % электропитание
\NC\md F 10\NC 30 A \NC Управление двигателем \NC\NR
\NC\md F 11\NC 5 A \NC Обратный свет \NC\NR % Рабочий прожектор сзади
%% NOTE @Andrew: Singular
\HL
\stoptabulate

\placefig [margin] [fig:electric:power:rear] {Центральная электрическая система в шасси}
{\externalfigure [electric:power:rear]
\noteF
\startKleg
\sym{K 1} Электронный блок управления двигателем
\sym{K 2} Топливный насос
\sym{K 3} Разблокирование стартера
\sym{K 4} Стоп-сигналы
\sym{K 5} {[}резерв{]}
\sym{K 6} Обратный свет % Рабочий прожектор сзади
\sym{K 7} Предпусковой разогрев
\stopKleg
}


\subsubsubject{Предохранители MAXI}

% \startcolumns [n=2]
\starttabulate[|l|r|p|]
\HL
\NC\md F 15 \NC 50 A \NC Главное питание центральной электросистемы \NC\NR
\HL
\stoptabulate

\page [yes]

\setups[pagestyle:marginless]


\subsection{Центральная электрическая система в кабине}

\startcolumns[rule=on]

\placefig [bottom] [fig:fuse:cab] {Предохранители и реле в кабине}
{\externalfigure [electric:power:front]}

\columnbreak

\subsubsubject{Реле}

\vskip -12pt

\index{Предохранители+Кабина}\index{Реле+Кабина}

\starttabulate[|lB|p|]
\NC K 2\NC Компрессор кондиционера\NC\NR
\NC K 3\NC Компрессор кондиционера\NC\NR
\NC K 4\NC Электрический водяной насос\NC\NR
\NC K 5\NC Проблесковый маячок\NC\NR
\NC K 10 \NC Датчик частоты мигания\NC\NR
\NC K 11 \NC Ближний свет\NC\NR
\NC K 12 \NC Дальний свет {[}Reserve{]} \NC\NR
\NC K 13 \NC Рабочие прожектора\NC\NR
\NC K 14 \NC Прерывистое включение стеклоочистителей\NC\NR
\stoptabulate
\vskip -24pt

\placefig [bottom] [fig:fuse:access] {Щиток отсека электросистемы}
{\externalfigure [electric:power:cabin]}

\stopcolumns

\page [yes]


\subsubsubject{Предохранители MINI}

\startcolumns[rule=on]
% \setuptabulate[frame=on]
%\placetable[here][tab:fuses:cab]{Fusibles dans la cabine}
%{\noteF
\starttabulate[|lB|r|p|]
\NC F 1 \NC 3 A \NC Стояночный свет слева \NC\NR
\NC F 2 \NC 3 A \NC Стояночный свет справа \NC\NR
\NC F 3 \NC 7,5 A \NC Ближний свет слева \NC\NR
\NC F 4 \NC 7,5 A \NC Ближний свет справа \NC\NR
\NC F 5 \NC 7,5 A \NC Дальний свет слева {[}Reserve{]} \NC\NR
\NC F 6 \NC 7,5 A \NC Дальний свет справа {[}Reserve{]} \NC\NR
\NC F 7 \NC 10 A \NC Рабочие прожектора вверху \NC\NR
%% NOTE @Andrew: Plural
\NC F 8 \NC 10 A \NC Рабочие прожектора вверху (резерв) \NC\NR
%% NOTE @Andrew: Plural
\NC F 9 \NC — \NC {[}Frei{]} \NC\NR
\NC F 10 \NC 10 A \NC Стеклоочистители \NC\NR
\NC F 11 \NC 5 A \NC Выключатель освещения и системы проблесковой сигнализации \NC\NR
\NC F 12 \NC 5 A \NC {[}Reserve{]} \NC\NR
\NC F 13 \NC 10 A \NC Обогрев наружного зеркала \NC\NR
\NC F 14 \NC 7,5 A \NC \aW{+\:15} Радио и камера \NC\NR
\NC F 15 \NC 10 A \NC \aW{+\:30} Система проблесковой сигнализации \NC\NR
\NC F 16 \NC 5 A \NC Подсветка рулевой колонки \NC\NR
\NC F 17 \NC 7,5 A \NC \aW{+\:30} Радио и внутреннее освещение \NC\NR
\NC F 18 \NC — \NC {[}Frei{]} \NC\NR
\NC F 19 \NC 20 A \NC \aW{+\:30} RC 12 спереди \NC\NR
\NC F 20 \NC 20 A \NC \aW{+\:30} RC 12 сзади \NC\NR
\NC F 21 \NC 15 A \NC Розетка 12 В \NC\NR
\NC F 22 \NC 5 A \NC Ключ зажигания, многофункциональная консоль, Vpad \NC\NR
\NC F 23 \NC 5 A \NC Аварийный останов, средняя консоль, RC 12 спереди \NC\NR
\NC F 24 \NC 5 A \NC Аварийный останов, средняя консоль, RC 12 сзади \NC\NR
\NC F 25 \NC 2 A \NC \aW{+\:15} RC 12 спереди \NC\NR
\NC F 26 \NC 2 A \NC \aW{+\:15} RC 12 сзади \NC\NR
\NC F 27 \NC 15 A \NC Обогревательный вентилятор \NC\NR
\NC F 28 \NC 10 A \NC Компрессор кондиционера, система централизованной смазки \NC\NR
\NC F 29 \NC 15 A \NC Конденсатор кондиционера\NC\NR
\NC F 30 \NC 5 A \NC Термостат кондиционера \NC\NR
\NC F 31 \NC 5 A \NC \aW{+\:15} Многофункциональная консоль/Vpad \NC\NR
\NC F 32 \NC 15 A \NC Электрический водяной насос, проблесковый маячок \NC\NR
\NC F 33 \NC — \NC {[}Frei{]} \NC\NR
\NC F 34 \NC — \NC {[}Frei{]} \NC\NR
\NC F 35 \NC — \NC {[}Frei{]} \NC\NR
\NC F 36 \NC — \NC {[}Frei{]} \NC\NR
\stoptabulate
\stopcolumns

\page [yes]

\setups [pagestyle:bigmargin]


\subsection[sec:lighting]{Осветительное и сигнальное устройство}


\placefig [here] [fig:lighting] {Осветительное и сигнальное устройство машины}
{\externalfigure [vhc:electric:lighting]}

\placelegende [margin,none]{}{%
\vskip 30pt
{\sla Пояснения:}
\startLongleg\stdfontsemicn
\item Стояночные фонари\hfill 12 В – 5 Вт
\item Фонари ближнего света\hfill H7 12 В – 55 Вт
\item Мигалки\hfill оранжевые 12 В – 21 Вт
\item {\stdfontsemicn Рабочие прожектора}\hfill G886 12 В – 55 Вт
\item Указатели поворота\hfill 12 В – 21 Вт
\item Фонари заднего света/\crlf стоп-сигналы\hfill 12 В – 5/21 Вт
\item Прожектор заднего хода\hfill 12 В – 21 Вт
\item {[}свободно{]}
\item Освещение номерного знака\hfill 12 В – 5 Вт
\item Проблесковый маячок\hfill H1 12 В – 55 Вт
\stopLongleg}

\subsubsubject{Регулировка прожекторов}

\placefig [margin] [fig:lighting:adjustment] {Луч света при 5 м}
{\externalfigure [vhc:lighting:adjustment]
\startitemize
\sym{H\low{1}} Высота нити накала: 100 см
\sym{H\low{2}} Коррекция при 2\hairspace\%: 10 см
\stopitemize}

{\md Необходимые условия:} Бак чистой/регенерационной воды полон, водитель за рулем.
Направление прожекторов настраивается на заводе-изготовителе.
Высоту и угол луча света можно отрегулировать путем наклона пластмассового держателя.

Если при проверке выяснится, что настройку требуется изменить, отверните стопорный винт и скорректируйте угол наклона так, чтобы он соответствовал требованиям законодательства (см. \in{илл.}[fig:lighting:adjustment]). Туго затяните стопорный винт.

\page [yes]
\setups [pagestyle:marginless]


\subsection[sec:battcheck]{Батарея}

\subsubsection{Правила техники безопасности}

\startSymList
\PPfire
\SymList
\textDescrHead{Взрывоопасность}
При\index{Батарея+Правила техники безопасности}\index{Опасность+Взрыв} зарядке батарей образуется взрывоопасный\index{Гремучий газ} гремучий газ. Заряжайте батареи только в хорошо проветриваемых помещениях! Избегайте образования искр! Вблизи батареи запрещается курить и манипулировать огнем и незащищенным освещением.
\stopSymList

\startSymList
\PHvoltage
\SymList
\textDescrHead{Опасность короткого замыкания}
Если\index{Батарея+Техобслуживание} плюсовой зажим батареи приходит в соприкосновение с частями машины, существует\index{Опасность+Пожар}\index{Опасность+Взрыв} опасность короткого замыкания. При этом возможен взрыв смеси газов, который может представлять угрозу для вас и для окружающих.

\startitemize
\item Не кладите на батарею металлические предметы или инструменты.
\item При отсоединении зажимов батареи всегда снимайте сначала зажим с минусом, а затем с плюсом.
\item При присоединении батарей всегда сначала присоединяйте зажим с плюсом, а затем с минусом.
\item Запрещается снимать и присоединять контактные зажимы батарей при работающем двигателе.
\stopitemize
\stopSymList


\startSymList
\PHcorrosive
\SymList
\textDescrHead{Опасность химических ожогов}
Надевайте\index{Опасность+Химический ожог} защитные очки и кислотоупорные перчатки. Электролит батареи является прибл. 27\hairspace\% раствором серной кислоты (H\low{2}SO\low{4}) и может привести к химическим ожогам. Нейтрализуйте\index{Батарея+Опасность}\index{Батарея+Электролит} попавший на кожу электролит раствором двууглекислой соды и промойте чистой водой. При попадании электролита в глаза, промойте их большим количеством холодной воды и немедленно обратитесь к врачу.
\stopSymList

\testpage [18]

\startSymList
\startcombination[1*2]
{\PHcorrosive}{}
{\PHfire}{}
\stopcombination
\SymList
\textDescrHead{Хранение батареи}
Батареи\index{Батарея+Хранение} должны всегда храниться в вертикальном положении. В противном случае возможна утечка электролита батареи и химические ожоги или – при реакции с другими субстанциями – пожар. \par\null\par\null
\stopSymList

\starttextbackground [FC]
\setupparagraphs [PictPar][1][width=2.4em,inner=\hfill]

\startPictPar
\PMproteyes
\PictPar
\textDescrHead{Защитные очки}
При\index{Опасность+Травма глаз} смешивании кислоты с водой возникающие брызги могут попадать в глаза. При попадании кислотных брызг в глаза немедленно промойте их чистой водой и немедленно обратитесь к врачу!
\stopPictPar
\blank [small]

\startPictPar
\PMrtfm
\PictPar
\textDescrHead{Документация}
При обращении с батареями необходимо соблюдать указания по безопасности, меры защиты и методы обращения, описанные в этом руководстве по эксплуатации.
\stopPictPar
\blank [small]

\startPictPar
\PStrash
\PictPar
\textDescrHead{Охрана окружающей среды}
В батареях\index{Защита окружающей среды} содержатся вредные вещества. Не выбрасывайте старые батареи с бытовым мусором. Утилизируйте батареи в соответствии с экологическими нормами. Сдавайте их в специализированную мастерскую или в пункт приема использованных батарей.

Транспортируйте и храните батареи только в вертикальном положении. При транспортировке необходимо предохранять Батареи от опрокидывания. Электролит может выступать из вентиляционных отверстий пробок батареи и попадать в окружающую среду.
\stopPictPar
\stoptextbackground

\page [yes]

\setups[pagestyle:normal]


\subsubsection{Практические советы}

Для максимального срока службы батареи всегда должны быть достаточно хорошо заряжены.

Подзарядка\index{Батарея+Срок службы} батарей во время длительных простоев транспортного средства не только продлевает срок службы батарей, но и поддерживает их постоянную готовность к работе.

\placefig[margin][fig:batterycompartment]{\select{caption}{Отсек батареи (дверца для обслуживания)}{Отсек батареи}}
{\externalfigure[batt:compartment]}


\subsubsection{Уход}

Батарея \sdeux\ представляет собой {\em не требующий обслуживания} свинцовый аккумулятор. Помимо поддержания в заряженном состоянии и очистки другие работы по уходу за батареей не требуются.

\startitemize
\item Следите за тем, чтобы полюса батареи всегда были чистыми и сухими. Слегка смазывайте полюса смазкой, отталкивающей кислоту.
\item Батареи, у которых\index{Батарея+Зарядка} напряжение разомкнутой цепи\index{Батарея+Напряжение разомкнутой цепи батареи} меньше 12,4 В, необходимо подзарядить.
\stopitemize

\placefig[margin][fig:bclean]{Очистка полюсов}
{\externalfigure[batt:clean]
\noteF
Verwenden\index{Батарея+очистить}\index{Очистка+Батареи} Для удаления белого налета, возникшего в результате коррозии, используйте теплую воду. При коррозии одного из полюсов, отсоедините наконечник кабеля и очистите полюс корщеткой. Покройте полюса тонкой пленкой смазки.}


\subsubsection[sec:battery:switch]{Использование разъединителя батареи}

{\sl Не рекомендуется использовать разъединитель батареи регулярно (например, ежедневно)!}

\startSteps
\item Выключите\index{Разъединитель батареи} зажигание и подождите около 1 минуты.
\item Откройте отсек батареи (\inF[fig:batterycompartment]).
\item Нажмите красную кнопку разъединителя, чтобы разомкнуть цепь.
\item Чтобы снова замкнуть цепь, поверните разъединитель батареи на ¼ оборота по часовой стрелке.
\stopSteps



\stopsection

\page [yes]


\setups[pagestyle:marginless]

\section[sec:cleaning]{Очистка транспортного средства}

Перед\startregister[index][vhc:lavage]{Техобслуживание+Очистка} непосредственной очисткой смойте с кузова крупные загрязнения большим количеством воды. При этом необходимо очистить не только боковые поверхности, но также ниши колес и днище транспортного средства.

Тщательная мойка требуется, в первую очередь, зимой – для удаления высококоррозийных\index{Коррозия+Предотвращение} химических реагентов.

\starttextbackground [FC]
\startPictPar
\PHgeneric
\PictPar
\textDescrHead{Избегать повреждений от воды}
Запрещается очищать транспортное средство с помощью {\em водяных пушек} (\eG\ например, противопожарных) или {\em холодными очистителями на углеводородной основе.} При использовании паровых очистителей высокого давления необходимо соблюдать соответствующие предписания.
\stopPictPar
\blank[small]

\startPictPar
\pTwo[monde]
\PictPar
\textDescrHead{Охрана окружающей среды}
Очистка транспортного средства может приводить к серьезному загрязнению окружающей среды.
Очищайте машину только в местах, оборудованных\index{Защита окружающей среды} маслоотделителем. Соблюдайте действующие требования защиты окружающей среды.
\stopPictPar
\blank[small]

\startPictPar
\PMwarranty
\PictPar
\textDescrHead{Проводите очистку надлежащим образом!}
Компания \BosFull{boschung} не несет никакой ответственности за ущерб, возникающий вследствие несоблюдения предписаний по очистке.
\stopPictPar
\stoptextbackground


\subsection{Очистка под высоким давлением}

Для очистки\index{Очистка+Высокое давление} машины под высоким давлением подходит обычный аппарат.

При использовании аппарата для очистки под высоким давлением необходимо принять во внимание следующее:

\startitemize
\item Рабочее давление максимум 50\,бар
\item Плоское сопло с углом разбрызгивания 25°
\item Безопасная дистанция в случае брызг, по крайней мере, 80\,см
\item Температура воды максимум 40\,°C
\item См. также раздел \about[reiMi], \atpage[reiMi].
\stopitemize

\page [yes]

Несоблюдение этих\index{Лак+Повреждения} предписаний может привести к повреждению лака и защитного покрытия\index{Повреждения+Лак}.

См. также руководство по эксплуатации и правила техники безопасности к аппарату очистки под высоким давлением.

\starttextbackground[FC]
\startPictPar\PPspray\PictPar
При очистке аппаратом высокого давления вода может проникать в места, где она может вызывать повреждения. Поэтому запрещается направлять струю воды на чувствительные детали и устройства, например:
\stopPictPar

\startitemize
\item Датчики, электрические соединения и разъемы
\item Стартер, генератор, система впрыска
\item Магнитные клапаны
\item Вентиляционные отверстия
\item Еще не охладившиеся механические компоненты
\item Наклейки с указаниями, предупреждениями и знаками опасности
\item Электронные устройства управления
\stopitemize

\textDescrHead{Промывка двигателя}
Необходимо исключить проникновение воды в отверстия всасывания, подачи и удаления воздуха. Не направляйте струю очистителей под высоким давлением прямо на электрические компоненты и линии. Не направляйте струю на систему впрыска! После мойки необходимо провести консервацию двигателя; при этом ремень должен быть защищен консервирующим составом.

\stoptextbackground

\page [yes]


\starttextbackground [FC]
\setupparagraphs [PictPar][1][width=6em,inner=\hfill]
\startPictPar
\framed[frame=off,offset=none]{\PMproteyes\PMprotears}
\PictPar
\textDescrHead{Остаток воды}
Во время очистки в определенных местах транспортного средства собирается вода (\eG\ например, в нишах блока цилиндров двигателя или коробки передач); ее необходимо удалить с помощью сжатого воздуха. При работе с пневматикой используйте соответствующие средства личной защиты, причем агрегат должен соответствовать действующим нормам техники безопасности (многофункциональное сопло).
\stopPictPar
\stoptextbackground


\subsubsection[reiMi]{Подходящие очистители }

Используйте\index{Очистители} только очистители, обладающие следующими свойствами:

\startitemize
\item отсутствие абразивного эффекта
\item Показатель pH – 6–7
\item без растворителей.
\stopitemize

Для устранения плохо удаляющихся пятен на участках лакового покрытия используйте лишь необходимое количество моющего бензина или спирта, но ни в коем случае другие растворители. Удалите с лакированной поверхности остатки растворителя. Очистка пластмассовых частей бензином может привести к образованию трещин или к изменениям цвета!

Поверхности с\index{Очистка+Наклейки} предупреждающими и предписывающими наклейками необходимо очищать холодной водой и мягкой губкой.

Избегайте проникновения воды в электрические компоненты: Для очистки корпуса мигалок и фонарей используйте не автомобильные щетки, а мягкую тряпку или губку.

\starttextbackground [CB]
\startPictPar
\GHSgeneric\par
\GHSfire
\PictPar
\textDescrHead{Опасность со стороны химикатов}
Средства очистки могут быть источником рисков для здоровья и безопасности (легко воспламеняющиеся материалы). Соблюдайте действующие правила техники безопасности для используемых очистителей; соблюдайте памятки по безопасности и данные технических паспортов используемых очистителей.
\stopPictPar
\stoptextbackground

\stopregister[index][vhc:lavage]

\page [yes]


\setups [pagestyle:bigmargin]

\startsection [title={Настройка всасывающего патрубка},
reference={sec:main:suctionMouth}]


Оптимальное расстояние\index{Всасывающий патрубок+Настройка} между поверхностью дороги и резиновым уплотнением всасывающего патрубка составляет 10\,мм.
Для проверки и настройки расстояния используйте три регулировочные линейки, которые находятся в ящике с инструментами (в кабине со стороны водителя).


\placefig [margin] [fig:suctionMouth] {Настройка всасывающего патрубка}
{\Framed{\externalfigure [suctionMouth:adjust]}}

\placeNote[][service_picto]{}{%
\noteF
\starttextrule{\PHasphyxie\enskip Опасность интоксикации и удушения \enskip}
{\md Указание:} Во время настройки двигатель машины должен работать, чтобы патрубок находился в плавающем положении. В этой связи, для предотвращения интоксикации и удушения следует обязательно использовать систему вытяжки выхлопных газов либо проводить работы только в очень хорошо проветриваемом месте.
\stoptextrule}

\startSteps
\item Поставьте машину на горизонтальную и ровную поверхность в хорошо проветриваемом месте.
\item Активируйте\index{Всасывание}  \aW{рабочий} режим (кнопка на внешней стороне рычага переключения передач).

Включите двигатель на холостом ходу. (Нажмите кнопку~\textSymb{joy_key_engine_decrease} на многофункциональной консоли, чтобы уменьшить частоту вращения двигателя.)
\item Поставьте машину на стояночный тормоз и положите под оба задних колеса противооткатные клинья.
\item Нажмите кнопку~\textSymb{joy_key_suction}, чтобы опустить патрубок всасывания.
\item Разместите три регулировочные линейки~\LAa\ под резиновой кромкой патрубка всасывания так, как показано на иллюстрации.
\item [sucMouth:adjust]Отверните крепежные~\Lone\ и регулировочные винты~\Ltwo\ каждого колеса; все четыре колеса опустятся на землю.
\item Снова затяните винты~\Lone\ и~\Ltwo\ и уберите регулировочные линейки.
\item Поднимите/опустите патрубок всасывания и проверьте положение регулировочными линейками. Если положение не соответствует требуемому, повторите процесс настройки, начиная с пункта~\in[sucMouth:adjust].

\stopSteps


\stopsection
\stopchapter
\stopcomponent


