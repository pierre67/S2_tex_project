\startcomponent c_10_safety_s2_110-ru
\product prd_ba_s2_110-ru

\marking[chapter]{Знаки безопасности}


\chapter{Знаки безопасности}

\setups[pagestyle:marginless]

\section{Новая европейская маркировка опасных и вредных материалов}

{\em Ромбическая табличка с белым фоном и красной каймой.}\par\blank[1*medium]
{\em С 2008 года в ЕС действует так называемое Постановление по классификации, маркировке и упаковке\index{Постановление по классификации, маркировке и упаковке} с новыми знаками предупреждения на опасных материалах и продуктах.}\par\null

\startSymList \GHSgeneric
\SymList
\textDescrHead{Опасность для здоровья}
Предупреждает о\index{Опасность для здоровья} рисках для здоровья, которые не приводят к смертельному исходу или к серьезному ущербу для здоровья. Сюда относятся раздражения кожи или аллергические реакции. Этот символ используется также как предупреждение о других опасностях, таких как воспламеняемость.\par
Взамен:\crlf \HAZOcross\ или \HAZOpoison\ или \PHgeneric
\stopSymList

\startSymList \GHSbody
\SymList
\textDescrHead{Серьезный риск для здоровья; в частности, у детей может также привести к смертельному исходу}
Эти продукты могут иметь серьезные последствия для здоровья. Этот символ предупреждает также об опасности\index{Опасность+Беременность} для беременных, о канцерогенных воздействиях\index{Опасность+Канцерогенные вещества} и других подобных рисках для здоровья. Обращение с такими продуктами требует осторожности.\par
Взамен:\crlf \HAZOcross\ или \HAZOpoison\
\stopSymList

\startSymList \GHSbomb
\SymList
\textDescrHead{Взрывоопасные материалы}
Нестабильные взрывоопасные\index{Опасность+Взрыв} вещества, смеси и продукты со взрывоопасными веществами\index{Взрывоопасные вещества} в случае реакции обладают эффектом сильного расширения, который может вызывать значительные разрушения; при ненадлежащем обращении с ними существует опасность для жизни.\par
Взамен:\crlf \HAZObomb\
\stopSymList


\startSymList \GHSpoison
\SymList
\textDescrHead{Отравление}
Эти продукты\index{Опасность+Отравление} даже в незначительных количествах при попадании на кожу, ингаляции \index{Ядовитые вещества} или проглатывании приводят к тяжелым или даже смертельным отравлениям. Не допускайте прямого контакта.\par
Взамен:\crlf \HAZOpoison\
\stopSymList

\startSymList \GHSfire
\SymList
\textDescrHead{Легко или быстро воспламеняющийся}
Эти продукты\index{Опасность+Огонь} быстро воспламеняются вблизи сильного жара или огня. Аэрозольные баллончики с этим обозначением не должны ни в коем случае распыляться на горячих поверхностях или вблизи открытого пламени.\par
Взамен:\crlf \HAZOfire\ или \HAZOfirebis\
\stopSymList

\startSymList \GHSenvironment
\SymList
\textDescrHead{Угроза для животных и окружающей среды}
Эти продукты\index{Охрана окружающей среды} могут нанести окружающей среде краткосрочный или долговременный ущерб\index{Ядовитые вещества}. Они могут убивать находящиеся в воде живые организмы (\eG\ рыбу) или оказывать на окружающую среду вредное воздействие в течение долгого времени. Не допускать их попадания в канализацию или бытовые отходы!\par
Взамен:\crlf \HAZOenvironment\
\stopSymList

\startSymList \GHScorrosive
\SymList
\textDescrHead{Угроза для кожи и глаз}
Эти продукты\index{Опасность+Повреждение кожи}\index{Опасность+Повреждение глаз} могут уже при кратковременном контакте приводить к воспалению и опухолям на коже или серьезной глазной травме. Защищайте кожу и глаза при работе с такими материалами!\par
Взамен:\crlf \HAZOcross\ или \HAZOcorrosive
\stopSymList

\page [yes]


\section{Предупреждающие знаки}

{\em Черная надпись на желтом фоне}\par\null

\startSymList \PHgeneric
\SymList
\textDescrHead{Предупреждение общего характера}
Указывает\index{Опасность+общая}\index{Предупреждающие знаки} на непосредственную опасность, при которой можете получить травму вы или другие лица.
\crlf\null
\stopSymList

\startSymList \PHpoison
\SymList
\textDescrHead{Предупреждение о токсичных материалах}
Токсичные материалы\index{Опасность+Отравление} при попадании на кожу, ингаляции или проглатывании могут нанести существенный вред здоровью или даже привести к смертельному исходу.
\stopSymList

\startSymList \PHfire
\SymList
\textDescrHead{Предупреждение об огнеопасных материалах}
Следует избегать открытого огня и искрения\index{Опасность+Огонь}. Этот материал легко воспламеняется или поддерживает горение. Не курить!
\stopSymList

\startSymList \PHexplosive
\SymList
\textDescrHead{Предупреждение о взрывоопасных материалах}
Твердые, жидкие или желеобразные материалы или композиции, которые могут привести к взрыву под действием удара, трения, огня, жара и т.\,п..\index{Опасность+Взрыв} Не курить!
\stopSymList

\startSymList \PHcrushing
\SymList
\textDescrHead{Предупреждение об опасности раздавливания}
Указывает на зону\index{Опасность+Раздавливание}, в которой в связи с наличием движущихся механических частей существует опасность раздавливания. Пока устройство включено, держитесь на удалении от этой зоны.
\stopSymList

\startSymList \PHhand
\SymList
\textDescrHead{Предупреждение о травмах рук}
Существует опасность раздавливания\index{Опасность+Раздавливание} рук и других частей тела\index{Опасность+Травмы рук} \eG\ при опрокидывании кабины водителя или загрузочной платформы.
\stopSymList

\startSymList \PHentangle
\SymList
\textDescrHead{Предупреждение о роликах, вращающихся во встречном направлении / об опасности затягивания}
Существует опасность захвата и затягивания конечностей тела\index{Опасность+Затягивание} вращающимися частями. Пока устройство включено, держитесь на удалении от этой зоны.
\stopSymList

\startSymList \PHcorrosive
\SymList
\textDescrHead{Предупреждение о разъедающих материалах}
Обращаться\index{Опасность+Разъедающие материалы} осторожно, надевать соответствующие средства индивидуальной защиты (перчатки, защитные очки, защитную одежду).
\stopSymList

\startSymList \PHhot
\SymList
\textDescrHead{Предупреждение о горячей поверхности}
Не приближайтесь к компонентам или устройствам\index{Опасность+Ожог}
если не обладаете достаточными знаниями; надевайте перчатки.
\stopSymList

\startSymList \PHvoltage
\SymList
\textDescrHead{Предупреждение об опасном для жизни электрическом напряжении}
Не прикасайтесь металлическими предметами\index{Опасность+Электрическое напряжение}.
Опасность травмы или ожога в результате короткого замыкания!
\stopSymList

\startSymList \PHfalling
\SymList
\textDescrHead{Предупреждение об опасности падения}
В этой области следует\index{Опасность+Падение} соблюдать особую осторожность, носить подходящую обувь (с нескользящими подошвами, стойкими к воздействию углеводородов).
\stopSymList

\startSymList \PHbattery
\SymList
\textDescrHead{Предупреждение об опасностях, связанных с аккумуляторами} Указывает на опасности, возникающие при зарядке аккумуляторов (свинцовых)\index{Опасность+Аккумулятор}, в частности, из-за выделения водорода и серной кислоты.
\stopSymList

\startSymList \PHremote
\SymList
\textDescrHead{Предупреждение о возможном автоматическом пуске}
Предупреждает о\index{Опасность+Автоматический пуск} возможности автоматического или дистанционно управляемого пуска одного из устройств.
\stopSymList

% \startSymList \PHquetschgefahr
% \SymList
% \textDescrHead{Risque d’écrasement}
% Risque d’écrasement\index{risque d’écrasement}.
% \stopSymList
% % NOTE: Doppelt! (auch Bilddatei)
%
% % NOTE: Evtl. Folgendes als Ersatz für oben?

% \startSymList\PHhandcrushed
% \SymList
% \textDescrHead{Gefahr von Handquetschungen}
% Es besteht\index{Gefahr+Quetschung} die Gefahr, dass Hände oder Finger
% gequetscht werden. Nähern Sie die Hände nicht an, ohne die Gefahr
% identifiziert und beseitigt zu haben.
% \stopSymList

\startSymList \PHhandfoot
\SymList
\textDescrHead{Предупреждение о движущихся компонентах}
Предупреждает о движущихся компонентах устройств / машины.
\index{Опасность+Движущиеся части}.
\stopSymList

\startSymList \PHnarrowed
\SymList
\textDescrHead{Предупреждение о сужении проезда}
Сужение\index{Опасность+Ширина транспортного средства} проезда.
% Denken Sie an die Breite des Fahrzeugs.
\stopSymList

\page [yes]


\section{Запрещающие знаки}

{\em Круглая табличка с белым фоном, красной каймой и диагональной полосой}
\par\null


\startSymList \PPfire
\SymList
\textDescrHead{Огонь незащищенные источники света и курение запрещены} Запрещается открытый или тлеющий\index{Запрет+Курение, огонь} огонь в любой форме (\eG\ зажженные сигареты, спички, свечи, искрение любого вида).
\stopSymList

\startSymList \PPentry
\SymList
\textDescrHead{Посторонним доступ воспрещен}
Посторонние\index{Запрет+Доступ} лица не должны входить в эту зону и приближаться к ней.
\stopSymList

\startSymList \PPphone
\SymList
\textDescrHead{Запрет мобильной связи}
Необходимо выключить\index{Запрет+Мобильные телефоны} мобильные телефоны и любые другие приборы с электромагнитным излучением. Электромагнитное излучение может привести к неисправности электроники рабочих инструментов.
\stopSymList

\startSymList \PPspray
\SymList
\textDescrHead{Запрет опрыскивания водой}
Не направляйте струю воды или пара\index{Запрет+Струя воды, пар} на чувствительные элементы оборудования (\eG\ датчики, контроллеры, система впрыска и т.д.).
\stopSymList

\startSymList \PPchildren
\SymList
\textDescrHead{Не подпускать детей}
Указывает\index{Запрет+Дети} на особую опасность для детей. В общем действует правило: Дети не должны приближаться к включенной машине, даже во время технического обслуживания.
\stopSymList

\startSymList \PPwater
\SymList
\textDescrHead{Не питьевая вода}
Запрещается пить воду из бака\index{Запрет+Не питьевая вода}. Существует опасность отравления.
\stopSymList

% \page [yes]


\section{Знак охраны окружающей среды}

\startSymList \PSrecycle
\SymList
\textDescrHead{Вторичная переработка}
Конкретные предписания для надлежащей утилизации определенных отходов.
\stopSymList

\startSymList \PSwelt
\SymList
\textDescrHead{Охрана окружающей среды}
Указание на действующие правила охраны окружающей среды.
\stopSymList

\startSymList \PStrash[width=\PictoHeight,height=,]
\SymList
\textDescrHead{Утилизировать отходы в соответствии с предписаниями}
Для определенных отходов, \eG\ свинцовых аккумуляторов, действуют специальные предписания по утилизации.
\stopSymList


\testpage[12]


\section{Предписывающие знаки}


{\em Круглые на синем фоне}\par\null

\startSymList \PMgeneric
\SymList
\textDescrHead{Общий предписывающий знак}
Этот знак должен использоваться только в комбинации с дополнительным знаком, конкретизирующим предписание.
\stopSymList


\startSymList \PMrtfm
\SymList
\textDescrHead{Соблюдать руководство по использованию}
Перед вводом в эксплуатацию необходимо в обязательном порядке\index{Прочитать руководство по использованию} прочитать инструкции по этой теме, а также инструкции для определенного устройства или продукта. Руководство по применению должно храниться в кабине водителя в хорошо доступном месте.
\stopSymList

\startSymList \PMproteyes
\SymList
\textDescrHead{Использовать средства защиты глаз}
Если при выполнении работ существует опасность получения травмы глаз, необходимо надевать средства защиты глаз\index{Средства защиты глаз}.
\stopSymList

\startSymList \PMprothands
\SymList
\textDescrHead{Использовать средства защиты рук}
Если при выполнении работ существует опасность получения травмы рук, необходимо использовать защитные перчатки\index{Использовать защитные перчатки}.
\stopSymList

\startSymList \PMprotears
\SymList
\textDescrHead{Использовать средства защиты слуха}
Следует использовать средства защиты слуха\index{Опасность+Органы слуха} (\eG вблизи работающего вентилятора или вращающейся турбины).
\stopSymList

\startSymList \PMsafetybelt
\SymList
\textDescrHead{Использовать ремень безопасности} Для безопасности пристегивайте\index{Ремень безопасности} ремень безопасности.
\stopSymList

\section{Дополнительные знаки}

% \adaptlayout[height=+5mm]{{{

% \startSymList \SETshoe
% \SymList
% \textDescrHead{Port de chaussures de sécurité obligatoire}
% Le port de chaussures de sécurité est obligatoire\index{chaussures de sécurité}.
% \stopSymList
%
% \startSymList \SETglasses
% \SymList
% \textDescrHead{Port de lunettes des protection obligatoire}
% Le port de lunettes est obligatoire\index{lunette de protection}.
% \stopSymList
%
% \startSymList \SEToreillettes
% \SymList
% \textDescrHead{Port de casque obligatoire}
% Le port d’un casque de protection est \index{casque} obligatoire.
% \stopSymList
%
% \startSymList \SETgloves
% \SymList
% \textDescrHead{Port de gants de protection obligatoire}
% Le port de gants de protection est obligatoire\index{gants}.
% \stopSymList
%
% \startSymList \SETmainecrase
% \SymList
% \textDescrHead{Risque d’écrasement}
% Danger pour les mains\index{écrasement} et les pieds.
% \stopSymList
%
% \startSymList \SETgetriebe
% \SymList
% \textDescrHead{Risque de happement}
% Risque de happement par\index{happement} des pièces en rotation.
% \stopSymList
%
% \startSymList \SETradkeil
% \SymList
% \textDescrHead{Cale de roue}
% Sécuriser le véhicule contre toute mise\index{Cale de roue} en marche involontaire.
% \stopSymList
%}}}

\startSymList \SETfirstaid
\SymList
\textDescrHead{Первая медицинская помощь}
Указывает на место хранения средств первой помощи. Важной частью первой медицинской помощи является быстрое оповещение службы спасения.\index{Первая медицинская помощь}\index{Экстренный вызов} Запишите здесь телефоны служб спасения:
\fillinrules[n=1]{\bf
\framed[align=right,frame=off,offset=none,width=30mm]{Служба спасения}}
\fillinrules[n=1]{\bf
\framed[align=right,frame=off,offset=none,width=30mm]{Полиция}}
\fillinrules[n=1]{\bf
\framed[align=right,frame=off,offset=none,width=30mm]{Пожарная команда}}
\stopSymList

\startSymList \SETbrandschutzzeichen
\SymList
\textDescrHead{Огнетушители}
Определенные устройства оснащены одним или несколькими огнетушителями\index{Огнетушители}. Они обычно требуют специального технического обслуживания; дополнительная информация приведена на устройстве или в инструкции по эксплуатации устройства.
\stopSymList


\page[yes]

\section{Три этапа оказания помощи}
% NOTE [tf]: Shouldn't be in this book, IMO

\starttextbackground [CB]
\textDescrHead{Оградить место аварии и пострадавших лиц}
\startitemize
\item Убедиться в безопасности места, где произошел несчастный случай, и убедиться в отсутствии других рисков.
\stopitemize
\textDescrHead{Определить состояние пострадавших лиц}
\startitemize
\item Убедиться в том, что потерпевшие находятся в сознании и дышат нормально.
При необходимости, очистить дыхательные пути.
\stopitemize
\textDescrHead{Поставить в известность спасателей}
\startitemize При звонке в службу спасения сообщить следующую информацию:\par
\item Номер телефона, по которому вам можно перезвонить.
\item Характер случившегося (болезнь, несчастный случай).
\item Существующие риски (пожар, взрыв, опасность падения и т.п.).
\item Точное описание места происшествия.
\item Количество пострадавших и их состояние.
\item Принятые меры по оказанию первой медицинской помощи.
\item Ответьте на дальнейшие вопросы, которые Вам будут заданы.
\stopitemize
\stoptextbackground

\stopcomponent


