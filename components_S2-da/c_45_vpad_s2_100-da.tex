\startcomponent c_45_vpad_s2_100-da
\product prd_ba_s2_100-da

\startchapter[title={On-board computer (Vpad)},
reference={sec:vpad}]

\setups[pagestyle:marginless]


\startsection[title={Beskrivelse af Vpad},
reference={vpad:description}]

\startfigtext [left] {Vpad SN ved førerpladsen}
{\externalfigure[vpad:inside:view]}
\textDescrHead{Innovativ, intelligent … } \Vpad\ er udviklet til styring af aggregater fra kommunen, hvis teknologi er blevet mere og mere kompliceret, og som stiller et væld af forskellige funktioner til rådighed.
Med \Vpad\ har brugeren et system ved hånden, der ikke begrænser sig til at levere informationer i realtid - visuelt eller akustisk - i alle arbejds- og maskinprocesser.
Det, der især kendetegner \Vpad\ og hvor det sætter nye standarder, er den intuitive brugerguide, brugervenlige ergonomi og kommandologik.

I kraft af dens mange funktioner, kan \Vpad\ anvendes overordentlig fleksibelt, og bliver dermed til mere end blot en elektronisk styreenhed.
\stopfigtext

\textDescrHead{… universel} Kompatibilitet og fleksibilitet har været i fokus under udviklingen af \Vpad\:
Kan som modulopbygget styreenhed tilpasses lokale forhold og udstyrsvarianter individuelt, og ved hjælp af dens talrige elektroniske grænseflader og dataoverførselsmetoder - til WLAN - står alle muligheder åbne.
\Vpad\ arbejder med den mest moderne elektronik med 32-bit-teknologi og styresystem i realtid.
\vfill


\startfigtext[left]{Multifunktionskonsol}
{\externalfigure[console:topview]}
\textDescrHead{… og modulopbygget} Med dens modularitet har \Vpad\ en enorm fordel:
Således kan version SN, der som standard anvendes i \sdeux\ til enhver tid udvides gradvist med yderligere komponenter, som for eksempel et modem eller en konsol (se billedet).
Modulariteten er ikke begrænset til hardwaren, også på softwaresiden kan systemet i høj grad udvides og tilpasses skiftende behov.

Multifunktionskonsollen i \sdeux\ er en avanceret grænseflade mellem brugeren og maskinen. Hele feje-/sugesystemet kan styres med denne konsol.
\stopfigtext

\page [yes]


\subsection[vpad:home]{Hovedskærmbillede}

%% Note: outcommented by PB
% \placefig[left][fig:vpad:engineData]{Accueil mode transport}
% {\scale[sx=1.5,sy=1.5]
% {\setups[VpadFramedFigureHome]
% \VpadScreenConfig{
% \VpadFoot{\VpadPictures{vpadClear}{vpadBeacon}{vpadEngine}{vpadSignal}}}
% \framed{\null}}
% }


\start

\setupcombinations[width=\textwidth]

\placefig [here][fig:vpad:engineData]{Hovedskærmbillede}
{\startcombination [2*1]
{\setups[VpadFramedFigureHome]% \VpadFramedFigureK pour bande noire
\VpadScreenConfig{
\VpadFoot{\VpadPictures{vpadClear}{vpadBeacon}{vpadEngine}{vpadSignal}}}%
\scale[sx=1.5,sy=1.5]{\framed{\null}}}{\aW{Køre}modus}
{\setups[VpadFramedFigureWork]% \VpadFramedFigureK pour bande noire
\VpadScreenConfig{
\VpadFoot{\VpadPictures{vpadClear}{vpadBeacon}{vpadEngine}{vpadSignal}}}%
\scale[sx=1.5,sy=1.5]{\framed{\null}}}{\aW{Arbejds}modus}
\stopcombination}

\stop

\blank [1*big]

Hovedskærmbilledet i \Vpad\ omfatter alle nødvendige elementer til overvågning af samtlige funktioner i \sdeux.

I øverste område findes kontrolvisningerne.

Det midterste område viser bl.\,a. følgende data i realtid:
Hastighed, motorens omdrejningstal og temperatur, brændstofniveau, genbrugsvandstand osv.

Modus \aW{Kørsel} symboliseres med en hare~\textSymb{transport_mode}, og modus \aW{Arbejde} med en skildpadde~\textSymb{working_mode}.

Menulinjen nederst viser de tilgængelige menuer: Tryk i midten af den berøringsfølsomme skærm (touchscreen) for at få vist flere menuer.

\page [yes]

\start % local group for temporary redefinition of \textDescrHead [TF]
\define[1]\textDescrHead{{\bf#1\fourperemspace}}
\startcolumns

\startSymVpad
\externalfigure[vpadTEnginOilPressure][height=1.7\lH]
\SymVpad
\textDescrHead{Motorolietryk}(rød) Motorolietrykket er for lavt. Sluk omgående motoren.

+\:Fejlmelding \# 604
\stopSymVpad

\startSymVpad
\externalfigure[vpadWarningBattery][height=1.7\lH]
\SymVpad
\textDescrHead{Batteriniveau}(rød) Batteriniveauet er for lavt. Underret værkstedet.
\stopSymVpad

\startSymVpad
\externalfigure[vpadWarningEngine1][height=1.7\lH]
\SymVpad
\textDescrHead{Motordiagnose}(gul) Fejl i motorstyringen. Underret værkstedet.
\stopSymVpad

\startSymVpad
\externalfigure[vpadWarningService][height=1.7\lH]
\SymVpad
\textDescrHead{Køretøjet skal på værksted}(gul) Køretøjet skal til regelmæssigt serviceeftersyn.
Se efter i vedligeholdelsesplanen.

+\:Fejlmeldinger \# 650 til \# 653, eller \# 703
\stopSymVpad

\startSymVpad
\externalfigure[vpadTBrakeError][height=1.7\lH]
\SymVpad
\textDescrHead{Bremsesystem}(rød) Fejl i bremsesystemet. Underret værkstedet.

+\:Fejlmelding \# 902
\stopSymVpad


\startSymVpad
\externalfigure[vpadTBrakePark][height=1.7\lH]
\SymVpad
\textDescrHead{Håndbremse}(rød) Køretøjets håndbremse er trukket.

+\:Fejlmelding \# 905
\stopSymVpad

\startSymVpad
\externalfigure[vpadTEngineHeating][height=1.7\lH]
\SymVpad
\textDescrHead{Forvarmning}(gul) Motoren forvarmes.

En blinkende lampe viser, at der er registreret en fejl i hændelsesloggen.
\stopSymVpad

\columnbreak

\startSymVpad
\externalfigure[vpadTFuelReserve][height=1.7\lH]
\SymVpad
\textDescrHead{Brændstofniveau}(gul) Brændstofniveauet er meget lavt (reserve).
\stopSymVpad

\startSymVpad
\externalfigure[vpadTBlink][height=1.7\lH]
\SymVpad
\textDescrHead{Blinklys}(grøn) Blinklys er aktiveret.
\stopSymVpad

\startSymVpad
\externalfigure[vpadTLowBeam][height=1.7\lH]
\SymVpad
\textDescrHead{Parkeringslys}(grøn) Parkeringslyset er tændt.
\stopSymVpad

\startSymVpad
\HL\NC \externalfigure[vpadSyWaterTemp][height=1.7\lH]
\SymVpad
\textDescrHead{Temperatur}(rød) Hydraulikvæskens eller motorens temperatur er for høj. Underret værkstedet.

+\:Fejlmelding \# 700 eller \# 610
\stopSymVpad

\startSymVpad
\externalfigure[vpadWarningFilter][height=1.7\lH]
\SymVpad
\textDescrHead{Filteret er tilstoppet}(rød) Det kombinerede hydraulikfilter eller luftfilteret er tilstoppet.

+\:Fejlmelding \# 702 eller \# 851
\stopSymVpad

\startSymVpad
\externalfigure[vpadTSpray][height=1.7\lH]
\SymVpad
\textDescrHead{Vandpistol}(gul) Højtryksvandpumpen til vandpistolen er aktiveret.

Kontakter \textSymb{temoin_buse} i loftskonsollen.
\stopSymVpad

\startSymVpad
\externalfigure[vpadTClear][height=1.7\lH]
\SymVpad
\textDescrHead{Fejlmelding}(rød) Der er en fejlmelding i loggen i \Vpad. Tryk på tasten~\textSymb{vpadClear} for at få vist alle registrerede beskeder. Underret værkstedet.
\stopSymVpad

\stopcolumns
\stop % local group for temporary redefinition of \textDescrHead

\stopsection

\page [yes]


\section{Vpad-menuer}

\start

\setupTABLE [background=color,
frame=off,
option=stretch,textwidth=\makeupwidth]

\setupTABLE [r] [each] [style=sans, background=color, bottomframe=on, framecolor=TableWhite, rulethickness=1.5pt]
\setupTABLE [r] [first][backgroundcolor=TableDark, style=sansbold]
\setupTABLE [r] [odd][backgroundcolor=TableMiddle]
\setupTABLE [r] [even] [backgroundcolor=TableLight]
\bTABLE [split=repeat]
\bTABLEhead
\bTR\bTD Menu \eTD\bTD Betegnelse\index{Vpad+Display} \eTD\bTD Funktion \eTD\eTR
\eTABLEhead

\bTABLEbody
\bTR\bTD \externalfigure [v:symbole:clear] \eTD\bTD Fejlmelding(er) \eTD\bTD Vis og kvittér fejlmeldinger, der er registreret i Vpad. \eTD\eTR
\bTR\bTD \framed[frame=off]{\externalfigure [v:symbole:beacon]\externalfigure [v:symbole:beacon:black]} \eTD\bTD Rotorblink \eTD\bTD Tænd/sluk rotorblink \eTD\eTR
\bTR\bTD \externalfigure [v:symbole:engine] \eTD\bTD Data i realtid \eTD\bTD Vis driftsdata i realtid for motor og hydraulik\eTD\eTR
\bTR\bTD \externalfigure [v:symbole:oneTwoThree] \eTD\bTD Tæller \eTD\bTD Visning af driftstimetælleren: Dagtæller, sæsontæller, totaltæller\eTD\eTR
\bTR\bTD \externalfigure [v:symbole:serviceInfo] \eTD\bTD Serviceinterval \eTD\bTD Viser datoen og de resterende driftstimer til næste vedligeholdelse, eller til næste store serviceeftersyn \eTD\eTR
\bTR\bTD \externalfigure [v:symbole:trash] \eTD\bTD Tæller \eTD\bTD Nulstil tæller eller serviceinterval \eTD\eTR
\bTR\bTD \externalfigure [v:symbole:sunglasses] \eTD\bTD Skærmmodus \eTD\bTD Skift skærmbelysning mellem \aW{dag} og \aW{nat} \eTD\eTR
\bTR\bTD \externalfigure [v:symbole:color] \eTD\bTD Lysstyrke/kontrast \eTD\bTD Indstilling af skærmens lysstyrke og kontrast \eTD\eTR
\bTR\bTD \externalfigure [v:symbole:select] \eTD\bTD Valg \eTD\bTD Valg af det markerede punkt, eller kvittering af en fejlmelding \eTD\eTR
\bTR\bTD \externalfigure [v:symbole:return] \eTD\bTD Bekræftelse \eTD\bTD Bekræftelse af valget \eTD\eTR
\bTR\bTD \framed[frame=off]{\externalfigure [v:symbole:up]\externalfigure [v:symbole:down]} \eTD\bTD Op/ned, \\Pile \eTD\bTD Flyt markeringen op/ned, eller øg/reducer den valgte værdi \eTD\eTR
\bTR\bTD \externalfigure [v:symbole:rSignal] \eTD\bTD Bakalarm \eTD\bTD Aktiver/deaktiver akustisk bakalarm \eTD\eTR
\eTABLEbody
\eTABLE
\stop


\subsubsubject{Andre visninger på Vpad}

\start % local group for temporary redefinition of \textDescrHead [TF]
\define[1]\textDescrHead{{\bf#1\fourperemspace}}

\startcolumns

\startSymVpad
\externalfigure[sym:vpad:water]
\SymVpad
\textDescrHead{Vandstand ferskvand} Ferskvandets vandstand er for lav (maks. 190\,l; bag førerhuset).
\stopSymVpad

\startSymVpad
\externalfigure[sym:vpad:rwater:yellow]
\SymVpad
\textDescrHead{Vandstand genbrugsvand}(gul) Genbrugsvandets vandstand under varmevekslerens. Hydraulikvæsken afkøles ikke og sugekanalens befugtningssystem forvarmes ikke.
\stopSymVpad

\startSymVpad
\externalfigure[sym:vpad:rwater]
\SymVpad
\textDescrHead{Vandstand genbrugsvand}(rød) Genbrugsvandets vandstand er for lav (maks. 140\,l; under smudsbeholderen).
\stopSymVpad

\stopcolumns
\stop % local group for temporary redefinition of \textDescrHead

\page [yes]

\startsection[title={Indstilling af skærmens lysstyrke},
reference={sec:vpad:brightness}]

Skærmen på \Vpad\ kan anvendes med to forkonfigurerede lysstyrker: Modus \aW{Dag}~– \textSymb{vpadSunglasses}, normal lysstyrke~– og modus \aW{Nat}~– \textSymb{vpadMoon}, reduceret lysstyrke.
Med tasten \textSymb{vpadColor} har De adgang til forskellige parametre.

Sådan ændrer De de forkonfigurerede lysstyrker:

\startSteps
\item Tryk i midten af den berøringsfølsomme skærm (touchscreen) for at rulle gennem menulinjen nederst i skærmbilledet.
\item Tryk på symbolet \textSymb{vpadSunglasses} eller
\textSymb{vpadMoon}for at vælge den modus, De vil ændre.
\item Tryk på \textSymb{vpadColor} for at få vist parametrene.
\item Marker med
pilesymbolerne~\textSymb{vpadUp}\textSymb{vpadDown} den parameter, De vil ændre, og vælg den med~\textSymb{vpadSelect}.
\item Værdien ændres med symbolerne
\textSymb{vpadMinus}\textSymb{vpadPlus}. Forsigtig, reducer ikke lysstyrken så meget (\VpadOp{162} -255), at De ikke længere kan se noget på skærmen!
\stopSteps
\blank [1*big]

\start
\setupcombinations[width=\textwidth]
\startcombination [3*1]
{\setups[VpadFramedFigureHome]% \VpadFramedFigureK pour bande noire
\VpadScreenConfig{
\VpadFoot{\VpadPictures{vpadOneTwoThree}{vpadServiceInfo}{vpadSunglasses}{vpadColor}}}%
\framed{\null}}{Tryk i midten af touchscreenen}
{\setups[VpadFramedFigure]
\VpadScreenConfig{
\VpadFoot{\VpadPictures{vpadReturn}{vpadUp}{vpadDown}{vpadSelect}}}%
\framed{\bTABLE
\bTR\bTD \VpadOp{160} \eTD\eTR
\bTR\bTD [backgroundcolor=black,color=TableWhite] \VpadOp{162}\hfill 15 \eTD\eTR
\bTR\bTD \VpadOp{163}\hfill 180 \eTD\eTR
\bTR\bTD \VpadOp{164}\hfill 55 \eTD\eTR
\bTR\bTD \VpadOp{165}\hfill 3 \eTD\eTR
\eTABLE}}{Vælg med \textSymb{vpadSelect}}
{\setups[VpadFramedFigure]% \VpadFramedFigureK pour bande noire
\VpadScreenConfig{
\VpadFoot{\VpadPictures{vpadReturn}{vpadMinus}{vpadPlus}{vpadNull}}}%
\framed[backgroundscreen=.9]{\bTABLE
\bTR\bTD \VpadOp{160} \eTD\eTR
\bTR\bTD \VpadOp{162}\hfill -80 \eTD\eTR
\bTR\bTD \VpadOp{163}\hfill 180 \eTD\eTR
\bTR\bTD \VpadOp{164}\hfill 55 \eTD\eTR
\bTR\bTD \VpadOp{165}\hfill 3 \eTD\eTR
\eTABLE}}{Værdien ændres med \textSymb{vpadMinus}\textSymb{vpadPlus}}
\stopcombination
\stop
\blank [1*big]

\startSteps [continue]
\item Bekræft værdien med \textSymb{vpadReturn}.
\item Tryk på symbolet \textSymb{vpadReturn}igen for at komme tilbage til hovedskærmbilledet.
\stopSteps

\stopsection

\page [yes]


\startsection[title={Driftstime- og kilometertæller},
reference={vpad:compteurs}]

Softwaren i \Vpad\ har tre forskellige måleperioder~– \aW{Dag}, \aW{Sæson}, \aW{Total}~–, hvor der kan køre forskellige tællere, som \aW{Tilbagelagt strækning}, \aW{Driftstimer} (motor eller børste), \aW{Arbejdstid} (pr. fører).

Sådan aflæser eller nulstiller De tællerne:

\startSteps
\item Tryk i midten af touchscreenen for at rulle gennem menulinjen.
\item Tryk på symbolet \textSymb{vpadOneTwoThree} for at få vist dagtælleren.
\item Med tilbage-/frem-symbolerne~\textSymb{vpadBW}\textSymb{vpadFW} kan De skifte til Total- eller Sæsontælleren.
\item Tryk på \textSymb{vpadTrash}for at nulstille den viste tæller.
\item I et dialogvindue opfordres De til at bekræfte nulstillingen.
\stopSteps
\blank [1*big]

\start
\setupcombinations[width=\textwidth]
\startcombination [3*1]
{\setups[VpadFramedFigure]% \VpadFramedFigureK pour bande noire
\VpadScreenConfig{
\VpadFoot{\VpadPictures{vpadOneTwoThree}{vpadServiceInfo}{vpadSunglasses}{vpadColor}}}%
\framed{\bTABLE
\bTR\bTD \VpadOp{120} \eTD\eTR
\bTR\bTD \VpadOp{123}\hfill 87.4\,h \eTD\eTR
\bTR\bTD \VpadOp{125}\hfill 62.0\,h \eTD\eTR
\bTR\bTD \VpadOp{126}\hfill 240.2\,km \eTD\eTR
\bTR\bTD \VpadOp{124}\hfill 901.9\,km \eTD\eTR
\bTR\bTD \VpadOp{127}\hfill 2,1\,l/h \eTD\eTR
\eTABLE}}{Tryk på symbolet~\textSymb{vpadOneTwoThree}og derefter på~\textSymb{vpadBW} eller~\textSymb{vpadFW}}
{\setups[VpadFramedFigure]
\VpadScreenConfig{
\VpadFoot{\VpadPictures{vpadReturn}{vpadBW}{vpadFW}{vpadTrash}}}%
\framed{\bTABLE
\bTR\bTD \VpadOp{121} \eTD\eTR
\bTR\bTD \VpadOp{123}\hfill 522.0\,h \eTD\eTR
\bTR\bTD \VpadOp{125}\hfill 662.8\,h \eTD\eTR
\bTR\bTD \VpadOp{126}\hfill 1605.5\,km \eTD\eTR
\bTR\bTD \VpadOp{124}\hfill 2608.4\,km \eTD\eTR
\bTR\bTD \VpadOp{127}\hfill 2,0\,l/h \eTD\eTR
\eTABLE}}{Nulstil tælleren med \textSymb{vpadTrash}}
{\setups[VpadFramedFigure]% \VpadFramedFigureK pour bande noire
\VpadScreenConfig{
\VpadFoot{\VpadPictures{vpadReturn}{vpadTrash}{vpadNull}{vpadNull}}}%
\framed{\bTABLE
\bTR\bTD \VpadOp{121} \eTD\eTR
\bTR\bTD \null \eTD\eTR
\bTR\bTD \VpadOp{136} \eTD\eTR
\bTR\bTD \null \eTD\eTR
\bTR\bTD \VpadOp{137} \eTD\eTR
\eTABLE}}{Bekræft med \textSymb{vpadTrash}}
\stopcombination
\stop
\blank [1*big]

\startSteps [continue]
\item Indtast adgangskoden, hvis nødvendigt, og bekræft derefter nulstillingen med symbolet \textSymb{vpadTrash}.
\item Tryk på symbolet \textSymb{vpadReturn} for at komme tilbage til hovedskærmbilledet.
\stopSteps

\stopsection

\page [yes]

\startsection[title={Vedligeholdelsesintervaller},
reference={vpad:maintenance}]

Vedligeholdelsesplanen for \sdeux\ tillader to primære typer vedligeholdelse: den regelmæssige vedligeholdelse og det store serviceeftersyn (på et autoriseret værksted, der er aftalt med Boschung-kundeservice ).

Sådan aflæser eller nulstiller De tællerne:
\startSteps
\item Tryk i midten af touchscreenen for at rulle gennem menulinjen.
\item Tryk på symbolet \textSymb{vpadServiceInfo} for at få vist vedligeholdelsesintervallerne.
\item Skift til det ønskede interval med pilesymbolerne~\textSymb{vpadUp}\textSymb{vpadDown}.
\item Tryk på symbolet~\textSymb{vpadTrash} for at nulstille et interval. Indtast adgangskoden med~\textSymb{vpadPlus}\textSymb{vpadMinus} og bekræft med~\textSymb{vpadSelect}.
\item I et dialogvindue opfordres De til at bekræfte nulstillingen.
\stopSteps
\blank [1*big]

\start
\setupcombinations[width=\textwidth]
\startcombination [3*1]
{\setups[VpadFramedFigure]% \VpadFramedFigureK pour bande noire
\VpadScreenConfig{
\VpadFoot{\VpadPictures{vpadReturn}{vpadNull}{vpadNull}{vpadTrash}}}%
\framed{\bTABLE
\bTR\bTD[nc=2] \VpadOp{190} \eTD\eTR
\bTR\bTD \VpadOp{191}\eTD\bTD \VpadOp{195}\hfill 600\,h \eTD\eTR % [backgroundcolor=black,color=TableWhite]
\bTR\bTD \VpadOp{192}\eTD\bTD \VpadOp{195}\hfill 600\,h \eTD\eTR
\bTR\bTD \VpadOp{193}\eTD\bTD \VpadOp{195}\hfill 2400\,h \eTD\eTR
\eTABLE}}{Tryk på symbolet~\textSymb{vpadTrash}for at nulstille
et interval}
{\setups[VpadFramedFigure]
\VpadScreenConfig{
\VpadFoot{\VpadPictures{vpadReturn}{vpadMinus}{vpadPlus}{vpadSelect}}}%
\framed{\bTABLE
\bTR\bTD \VpadOp{190} \eTD\eTR
\bTR\bTD \hfill 2014-03-31 \eTD\eTR
\bTR\bTD \null \eTD\eTR
\bTR\bTD \null \eTD\eTR
\bTR\bTD \null \eTD\eTR
\bTR\bTD \null \eTD\eTR
\bTR\bTD \VpadOp{002}\hfill 0000 \eTD\eTR
\eTABLE}}{Indtast adgangskoden (numerisk kode)}
{\setups[VpadFramedFigure]% \VpadFramedFigureK pour bande noire
\VpadScreenConfig{
\VpadFoot{\VpadPictures{vpadReturn}{vpadUp}{vpadDown}{vpadSelect}}}%
\framed{\bTABLE
\bTR\bTD \VpadOp{190} \eTD\eTR
\bTR\bTD[backgroundcolor=black,color=TableWhite] \VpadOp{041}\eTD\eTR % [backgroundcolor=black,color=TableWhite]
\bTR\bTD \VpadOp{042} \eTD\eTR
\bTR\bTD \VpadOp{043} \eTD\eTR
\eTABLE}}{Vælg og bekræft med~\textSymb{vpadSelect}}
\stopcombination
\stop
\blank [1*big]

\startSteps [continue]
\item Bekræft nulstillingen med symbolet~\textSymb{vpadSelect}.
\item Tryk på symbolet \textSymb{vpadReturn} for at komme tilbage til hovedskærmbilledet.
\stopSteps

\stopsection

\page [yes]


\startsection[title={Fejlstyring via Vpad},
reference={vpad:error}]


\Vpad\ viser fejl\index{Vpad+Fejlmeldinger}, der er overført af de elektroniske styresystemer og af CAN-bussen.
Hvis der registreres en mindre alvorlig fejl, lyser symbolet~\textSymb{VpadTClear} (rød).
Hvis det drejer sig om en fejl med høj prioritet, lyser symbolet~\textSymb{VpadTClear} og samtidigt høres en akustisk alarm.
For at afslutte alarmen, skal fejlmeldingen kvitteres (bekræftes som \aW{taget til efterretning}).

Sådan læser og kvitterer De fejlmeldinger:

\startSteps
\item Tryk på symbolet~\textSymb{vpadClear} på skærmen på \Vpad.
\item Tryk på symbolet~\textSymb{vpadClear} for at kvittere den valgte melding.
\item Ved siden af den kvitterede melding vises nu et \aW{\#}-symbol, der kendetegner meldingen som \aW{taget til efterretning}, og markeringen hopper til næste melding (hvis relevant).
\item Når alle meldinger er kvitteret, skifter displayet tilbage til hovedskærmbilledet.
\stopSteps
\blank [1*big]

\start
\setupcombinations[width=\textwidth]
\startcombination [3*1]
{\setups[VpadFramedFigure]% \VpadFramedFigureK pour bande noire
\VpadScreenConfig{
\VpadFoot{\VpadPictures{vpadReturn}{vpadUp}{vpadDown}{vpadSelect}}}%
\framed{\bTABLE
\bTR\bTD \VpadEr{000} \eTD\eTR
\bTR\bTD [backgroundcolor=black,color=TableWhite] \VpadEr{851a} \eTD\eTR
\bTR\bTD \VpadEr{902} \eTD\eTR
\eTABLE}}{Visning af meldinger}
{\setups[VpadFramedFigure]
\VpadScreenConfig{
\VpadFoot{\VpadPictures{vpadReturn}{vpadUp}{vpadDown}{vpadSelect}}}%
\framed{\bTABLE
\bTR\bTD \VpadEr{000} \eTD\eTR
\bTR\bTD [backgroundcolor=black,color=TableWhite] \VpadEr{851} \eTD\eTR
\bTR\bTD \VpadEr{902} \eTD\eTR
\eTABLE}}{Kvitter med~\textSymb{vpadClear}}
{\setups[VpadFramedFigureHome]% \VpadFramedFigureK pour bande noire
\VpadScreenConfig{
\VpadFoot{\VpadPictures{vpadClear}{vpadBeacon}{vpadBeam}{vpadEngine}}}%
\framed{\null}}{Tilbage til hovedskærmbilledet}
\stopcombination
\stop
\blank [1*big]

\startSteps [continue]
\item Tryk på symbolet~\textSymb{vpadClear} for at få vist meldingerne igen. Fejlmeldinger slettes først fra \Vpad, når årsagen til problemet er afhjulpet.
\stopSteps


\subsection{De hyppigste fejlmeldinger (med fejlfinding)}

\subsubsubject{\VpadEr{604}} % {\#\ 604 Pression huile moteur basse}

+ \textSymb{vpadTEnginOilPressure}~– Sluk omgående motoren. Kontroller oliestanden og underret værkstedet.


\subsubsubject{\VpadEr{609}} % {\#\ 609 Température eau refroidissement moteur haute}

+ \textSymb{vpadSyWaterTemp}~– Afbryd arbejdet. Lad motoren fortsætte med at køre uden last og hold øje med temperaturudviklingen:

Hvis temperaturen falder, skal kølevæske-, motorolie- og hydraulikvæskestanden og kølerens tilstand kontrolleres.
Hvis væskestandene og køleren er i orden, skal De køre forsigtigt på værksted til yderligere fejldiagnose.

\subsubsubject{\VpadEr{610}} % {\#\ 610 Température eau refroidissement moteur trop haute}

+ \textSymb{vpadSyWaterTemp}~– Afbryd arbejdet. Kontroller kølevæske- og motoroliestanden og underret omgående værkstedet.


\subsubsubject{\VpadEr{650}} % {\#\ 650 Se rendre à un garage}

+ \textSymb{vpadWarningService}~– Underret omgående Deres værksted.
% \VpadEr{651} % {\#\ 651 Moteur en mode urgence}


\subsubsubject{\VpadEr{652}} % {\#\ 652 Inspection véhicule}
% \VpadEr{653} % {\#\ 653 Grand service moteur}

+ \textSymb{vpadWarningService}~– Den næste regelmæssige vedligeholdelse skal udføres. Se vedligeholdelsesplanen og aftal en dato med Deres værksted.


\subsubsubject{\VpadEr{700}} % {\#\ 700 Température d'huile hydraulique}

+ \textSymb{vpadSyWaterTemp}~– Afbryd arbejdet. Lad motoren fortsætte med at køre uden last og hold øje med temperaturudviklingen:

Hvis temperaturen falder, skal kølevæske-, motorolie- og hydraulikvæskestanden og kølerens tilstand kontrolleres.
Hvis væskestandene og køleren er i orden, skal De køre forsigtigt på værksted til yderligere fejldiagnose.


\subsubsubject{\VpadEr{702}} % {\#\ 702 Filtre d'huile hydraulique}

+ \textSymb{vpadWarningFilter}~– Hydraulik-retur- og/eller indsugningsfilteret er tilstoppet. Udskift omgående filterelementet.
% \VpadEr{703} % {\#\ 703 Vidange d'huile hydraulique}


\subsubsubject{\VpadEr{800}} % {\#\ 800 Interrupteur d'urgence actionné}

+ \textSymb{vpadTClear}~– De har trykket på nødstop-kontakten. Slå tændingen fra og start motoren igen for at slette meldingen.


\subsubsubject{\VpadEr{801}} % {\#\ 905 Frein à main actionné}

Smudsbeholderen er løftet, eller ikke sænket helt. Køretøjets hastighed er begrænset til 5\,km/h, så længe smudsbeholderen ikke er sænket.

\subsubsubject{\VpadEr{851}} % {\#\ 851 Filtre à air}

+ \textSymb{vpadWarningFilter}~– Luftfilteret er tilstoppet. Udskift omgående filterelementet.


\subsubsubject{\VpadEr{902}} % {\#\ 902 Pression de freinage}

+ \textSymb{vpadTBrakeError}~– Bremsetrykket er for svagt. Afbryd arbejdet og underret omgående værkstedet.
% \VpadEr{904} % {\#\ 904 Interrupteur de direction d'avancement}


\subsubsubject{\VpadEr{905}} % {\#\ 905 Frein à main actionné}

+ \textSymb{vpadTBrakePark}~– Håndbremsen er ikke løsnet helt. Køretøjets hastighed er begrænset til 5\,km/h, så længe håndbremsen ikke er løsnet.


\stopsection

\stopchapter

\stopcomponent














