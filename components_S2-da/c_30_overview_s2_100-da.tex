\startcomponent c_30_overview_s2_100-da
\product prd_ba_s2_100-da

\chapter{Oversigt over køretøjet}

\setups [pagestyle:marginless]


\placefig [here] [] {Oversigt over køretøjets venstre side}
{\externalfigure [overview:side:left:da]}


\page [yes]


\placefig [here] [] {Oversigt over køretøjets højre side}
{\externalfigure [overview:side:right:da]}

\page [yes]

\setups [pagestyle:normal]


\section{Generelt}

\placefig[margin][p4_vue_01]{\sdeux\ ved overdragelsen}
{%
\startcombination [1*3]
{\externalfigure[overview:vhc:01]}{}
{\externalfigure[overview:vhc:02]}{}
{\externalfigure[overview:vhc:03]}{}
\stopcombination}

Med fejemaskinen \BosFull{sdeux} giver Boschung hele sin erfaring og kompetence fra årtiers kontinuerlige samarbejde med sine trofaste kunder og partnere, videre.
De krav, som kommuner og tjenesteydere stiller, er med henblik på mobilitet og alsidighed, steget enormt i løbet af denne tid. Udviklerne af \sdeux\ har, vejledt af kundernes krav og fremmet af Boschung-kundeservices fremsynede forbedringsforslag, taget udfordringen op.
\sdeux\ har sit udspring i denne sammenfatning af kundeorientering og konsekvent gennemførelse af den erhvervede praktiske erfaring.


\subsection{Innovativ teknologi}

Den kompakte fejemaskine \BosFull{sdeux} er i sin klasse især kendetegnet ved dens lave vægt (2300\,kg), dens høje kapacitet (smudsbeholder i 2,0-m\high{3}-klassen), dens kompakte dimensioner (bredde 1,15\,m) og den særligt ergonomiske arbejdsplads til føreren af køretøjet.

Den smalle konstruktion gør \sdeux\ til en \quotation{universal}-fejemaskine til gader og fortove i byer og landsbyer. Den stærke dieselmotor, sammen med det kompakte hydrostatiske drev (hydraulisk motor med radialstempel på forhjulene), sørger til enhver tid for optimal mobilitet, uafhængigt af stedets beskaffenhed, hvor køretøjet anvendes, eller smudsbeholderens påfyldningsniveau.

Hydraulikpumperne drives af en dieselmotor af typen \aW{VW 2.0 CDI} iht. Euro-V-standard. Den leverer et drejningsmoment på 285\,Nm ved 1750~omdrejninger og en maksimal effekt på 75\,kW ved 3000~omdrejninger. Dermed kan maskinen anvendes effektivt, selv ved et lavt motoromdrejningstal~– og dermed lav støjforurening. \sdeux\ er som standard udstyret med et partikelfilter.


\section{Innovationer i kundens tjeneste}

Knækstyringen på \sdeux\ sørger for en lille venderadius og dermed for maksimal bevægelighed. Specielle materialer, som Domex® og en fuldstændig CAD-baseret udvikling af køretøjet, giver mulighed for en anselig nyttelast på 1200\,kg.

\placefig[margin][overview:cab:frontright]{\sdeux\ klar til indsats}
{\externalfigure[overview:cab:twoleft][width=\Bildwidth]}

Førerhuset med glas hele vejen rundt er udstyret med to komfortable sæder og med trepunkts sikkerhedsseler. \sdeux\ kan efter ønske udstyres med klimaanlæg.

Køretøjet tilpasser sig uden problemer bytrafikken med dens maksimumshastighed på 40\,km/h. Den behagelige for- og bagakselaffjedring sikrer, at køretøjet kører sikkert og behageligt på selv den dårligste strækning.

Føreren har et godt overblik over fejeaggregatet~, der er monteret på to leddelte arme~, og sugemunden, der er placeret foran forakslen. En dobbelt drejelig frontkost fås som ekstraudstyr.

\page [yes]


\subsection{Lydisoleret og komfortabelt førerhus}

Førerhuset\index{Førerhus} i \sdeux\ er udstyret med højrestyring og er konstrueret til to personer. Det er lydisoleret og monteret på vibrationsdæmpende silentblokke.

Dørene og gulvet er forsynet med glas, hvilket giver et omfattende synsfelt. Forruden dækker hele køretøjets forside og giver mulighed for uhindret udsyn over kostenes arbejde.

Førersædet er udstyret med mekanisk eller~– som ekstraudstyr~– pneumatisk affjedring. Fører- og passagersædet er monteret på justerbare glideskinner.


\subsubsubject{Ergonomi}

\startfigtext[right][overview:joy:sideview]{Betjeningskonsol}
{\externalfigure[overview:joy:top]}
Multifunktionskonsollen til venstre for førersædet, gør det nemt at udføre alle elementære funktioner med én hånd. De to koste kan styres uafhængigt af hinanden ved hjælp af to joysticks, tommelfingeren og pegefingeren. Kontakterne til kostene og til frontkosten (ekstraudstyr), til motoromdrejningstallet, fartpiloten osv. findes også på multifunktionskonsollen.
\stopfigtext

Nederst i førerens synsfelt findes der en touchscreen, der viser alle vigtige informationer om maskinens funktioner i realtid, uden at forringe udsynet.

\placefig[margin][overview:vhc:left]{\sdeux\ foran en historisk bygning}
% \placefig[margin][overview:vhc:left]{\sdeux\ sur site historique}
{\externalfigure[overview:vhc:left]}

\page [yes]


\subsubsubject{Førerplads}

Gearvælgere\index{Førerplads} (\quotation{gearskift}) findes til højre på ratstammen; der er to gear til fremadkørsel og et bakgear til rådighed. Uden på gearvælgeren findes der en trykknap til at skifte mellem de to drivmodi \aW{Arbejde} og \aW{Kørsel}. \sdeux\ skal ikke standses for at skifte. (Læs mere i kapitlet \about[sec:using:work], \atpage[sec:using:work].)

\placefig[margin][fig:overview:steeringwheel]{Førerplads}
{\externalfigure[overview:driver:place]}

Når køretøjet sættes i bakgear, tændes bakkameraets skærm og der høres et akustisk advarselssignal (kan deaktiveres via Vpad).

Multifunktionsgrebet på venstre side af ratstammen omfatter vinduesviskerkontakten (to trin og interval) samt lys og akustisk horn.

I kapitlet \about[chap:using] fra \atpage[chap:using] finder De detaljerede oplysninger om disse og andre funktioner i \sdeux.

\page [yes]

\setups[pagestyle:marginless]


\subsection[overview:brushsystem]{Feje- og sugeanordning}

\subsubsubject{Koste}

\startfigtext[left][fig:overview:steeringwheel]{Feje-/sugeanordning}
{\externalfigure[system:brush]}
Kostene\index{Fejning} sidder på justerbare hoveder, der er monteret på leddelte arme. Støvet, der hvirvles op under fejningen bindes ved at sprøjte det med vand: Hver kost er udstyret med en dyse, der henter vandet fra ferskvands- eller genbrugsvandtanken.

En kontakt\index{Sugning} på multifunktionskonsollen aktiverer samtidigt kostene og vandpumpen.\footnote{Om vandpumpen, se kapitel \in[chap:using] \about[chap:using], især \about[sec:using:work], \atpage[sec:using:work].}
Kostenes positioner og deres hældning på tværs og på langs, kan styres direkte med den pågældende joystick på multifunktionskonsollen.
\stopfigtext

Kostene er beskyttet med et mekanisk og hydraulisk antikollosionssystem.


\subsubsubject{Sugemund}

I arbejdsposition (sænket) hviler sugemunden på 4~ruller og dækker fuldstændigt fladen mellem kostene, der er kørt ud fra hinanden. Ved hjælp af dens \quotation{slæbende} position, er den stort set beskyttet mod mekaniske skader ved sammenstød med forhindringer. I bakgear løftes sugemunden automatisk.

En tyk, udskiftelig gummilæbe sørger for at slutte tæt mod vejens overflade. En elektrisk-hydraulisk styrbar lem på forsiden af sugemunden giver mulighed for at samle større affaldsgenstande op.


\subsubsubject{Smudsbeholder}

Aluminiums-smudsbeholderen kan tippes 55° og i en højde på 1,5\,m (afløbshøjde). I smudsbeholderen udmunder sugekanalen nedefra med en åbningsdiameter på 180\,mm.

Indsugningsvakuummet dannes af en kraftig turbine, der er monteret horisontalt i smudsbeholderen. Den har en servicelem til rengøring og visuel kontrol.

I smudsbeholderens lem findes der to indsugningsgitre af rustfrit stål. De kan klappes ud uden værktøj til rengøring. Lemmen kan låses op og åbnes med hånden.

Luftstrømmen kan uden problemer skiftes mellem sugekanal og håndsugeslange (valgfrit udstyr) ved hjælp af en lem, der kan klappes om med hånden.


\subsection{Befugtningsanordning}

\subsubsubject{Ferskvandssystem}

Tanken\index{Fejning+Befugtning} af ABS-støbejern findes i stående position bag førerhuset. Den har en kapacitet\index{Ferskvands+-tank} på 190\,l.

En elektrisk pumpe (10\,l/min) transporterer vandet til sprøjtedyserne over hver kost (inklusiv en valgfri tredje kost).


\subsubsubject{Genbrug af spildevand}

Spildevandet strømmer gennem mikroperforationerne i de indvendige sider i spildevandsbeholderen for derefter at strømme via genbrugslemmen i genbrugsvandtanken nedenunder. Genbrugsvandtanken\index{Genbrugsvand+-tank} har en kapacitet på 140\,l.

En elektrisk dykpumpe (10\,l/min) transporterer vandet til sprøjtedyserne inden i sugemunden og sugekanalen.


\testpage [8]
\subsubsubject{Genbrugsvandtank}

Genbrugsvandtanken er udstyret med en vand-hydraulikvæske-varmeveksler med dobbelt funktion:

\startitemize[width=35mm,style=\md, command={\setupwhitespace[small]}]
\sym{Funktion om sommeren} Vandet leder hydraulikvæskens varme via konvektion til tankens aluminiumsvægge, hvorfra den udstråles til den omgivende luft.

\sym{Funktion om vinteren} Hydraulikvæsken opvarmer vandet i tanken. Dette giver mulighed for, at sugekanalen og sugemunden også kan sprøjtes ved temperaturer lidt under frysepunktet.
\stopitemize


\subsubsubject{Overvågning af vandstande}

\startitemize[width=35mm,style=\md, command={\setupwhitespace[small]}]
\sym{Ferskvand} Når vandstanden er for lav, vises symbolet~\textSymb{vpad_water} på Vpad-skærmen.
\sym{Genbrugsvand} Hvis vandstanden i genbrugstanken er under varmevekslerens (se ovenfor), vises symbolet~\textSymb{vpad_rwater_orange} (gul) på Vpad-skærmen. Når vandstanden er for lav, vises symbolet ~\textSymb{vpad_rwater} (rød).
\stopitemize

\page [yes]
\setups[pagestyle:normal]


\section{Identifikation af køretøjet}

\subsection{Køretøjets typeskilt}

Køretøjets typeskilt\index{Identifikation+Køretøj} findes i førerhuset, over for konsollen under passagersædet (se \inF[fig:identity:location], \atpage[fig:identity:location]).


\subsection{Motorkode og -nummer}

Motorkoden findes på motorens typeskilt (mærkat) på kølekredsens metalledning med knæk foran på motoren (løft smudsbeholderen).

Motornummeret er indgraveret på motoren (\inF[identity:engine:number]). Det består af ni alfanumeriske tegn: De første tre er motorkoden, og de næste seks er motorens serienummer.


\placefig[margin][idvhc]{Køretøjets typeskilt}
{\externalfigure[s2:id:plaque]}

\placefig[margin][identity:engine:code]{Motorens typeskilt}
{\externalfigure[engine:id:code]}

\placefig[margin][identity:engine:number]{Motornummer}
{\externalfigure[engine:id:number]}

\page [yes]


\subsection [sec:plateWheel]{Hjulenes typeskilt}

Fælgenes og dækkenes typeskilt \index{Dæk+Dæktryk}findes i førerhuset\index{Fælge+Dimensioner} under passagersædet.


\subsection{Stelnummer}

Stelnummeret\index{Identifikation+Stelnummer} (chassis-nummeret) er præget ind i højre side af køretøjet på understellet under førerhuset.


\subsection{\symbol[europe][CEsign]-overensstemmelse og -mærkning}

~\symbol[europe][CEsign]-mærket findes i førerhuset over for konsollen under passagersædet.

\sdeux\ opfylder væsentlige sikkerheds- og sundhedskrav i maskindirektiv\index{Certifikat+CE-overensstemmelse}\index{Maskindirektiv} 2006/42/EF\footnote{Europa-Parlamentets og Rådets direktiv 2006/42/EF af 17.~maj 2006}.
% \textrule

\placefig[margin][idpneus]{Dæktryk}
{\externalfigure[identity:tires]}

\placefig[margin][fig:identity:location]{Typeskilte}
{\externalfigure[identity:location]}

\page [yes]
\setups [pagestyle:marginless]


\startsection[title={Tekniske data},
reference={donnees_techniques}]

\subsection [sec:measurement] {Køretøjets dimensioner}

\placefig[here][fig:measurement]{\select{caption}{Bredde~– kost i hvilestilling eller kørt ud~–, køretøjets længde og højde}{Køretøjets dimensioner}}
{\Framed{\externalfigure[s2:measurement]}}

\page [yes]

\placefig[here][fig:measurement]{\select{caption}{Køretøjets højde, når smudsbeholderen er tippet op}{Køretøjets højde}}
{\Framed{\externalfigure[s2:measurement:02]}}

\page [yes]

\starttabulate [|lBw(45mm)|p|l|rw(35mm)|]
\FL
\NC Gruppe\index{Mål} \NC \bf Mål \NC \bf Enhed \NC \bf Værdi \NC\NR
\ML
\NC Køretøjets dimensioner \NC Længde (overalt) \NC \unite{mm} \NC 4588,00 \NC\NR
\NC\NC Længde med 3.\,kost \NC \unite{mm} \NC 5020,00 \NC\NR
\NC\NC Køretøjets bredde \NC \unite{mm} \NC 1150,00 \NC\NR
\NC\NC Køretøjets bredde (overalt) \NC \unite{mm} \NC 1575,00 \NC\NR
\NC\NC Højde uden rotorblink \NC \unite{mm} \NC 1990,00 \NC\NR
\NC\NC Akselafstand \NC \unite{mm} \NC 1740,00 \NC\NR
\NC\NC Hjulafstand \NC \unite{mm} \NC 894,00 \NC\NR
\ML
\NC Fejebredde \NC Standardkoste \NC \unite{mm} \NC 2300,00 \NC\NR
\NC\NC Med 3.\,kost \NC \unite{mm} \NC 2600,00 \NC\NR
\NC\NC Diameter kost \NC \unite{mm} \NC 800,00 \NC\NR
\NC\NC Bredde sugemund \NC \unite{mm} \NC 800,00 \NC\NR
\ML
\NC Lastfordeling \NC Egenvægt\note[weight:empty] foraksel \NC \unite{kg} \NC ca. 1100,00 \NC\NR
\NC\NC Egenvægt\note[weight:empty] bagaksel \NC \unite{kg} \NC ca. 1200,00 \NC\NR
\NC\NC Egenvægt\note[weight:empty] \NC \unite{kg} \NC ca. 2300,00 \NC\NR
\NC\NC Til.totalvægt \NC \unite{kg} \NC 3500,00 \NC\NR
\LL
\stoptabulate


\subsection{Sporradius og fejeradius}

\starttabulate [|lBw(45mm)|p|l|rw(35mm)|]
\FL
\NC Dimension\index{Dimensioner} \NC \bf Mål \NC \bf Enhed \NC \bf Værdi \NC\NR
\ML
\NC Sporradius\index{Sporradius}\index{Mål+Sporradius} \NC Minimal venderadius med koste \NC \unite{mm} \NC 3325,00 \NC\NR
\ML
\NC Fejeradius \NC udvendig \NC \unite{mm} \NC 3425,00 til 3850,00 \NC\NR
\NC\NC indvendig \NC \unite{mm} \NC 2025,00 til 1675,00 \NC\NR
\LL
\stoptabulate

%% TODO en/de/fr: Footnote on preceeding page
\footnotetext[weight:empty]{Standardkonfiguration, med fører (ca. 75\,kg).}

\placefig[here][pict:steerin_sweeping:radius]{Spor-/venderadius og fejeradius}
{\externalfigure[steerin_sweeping:radius]}

\page [yes]


\subsection{Hjul og dæk}

\starttabulate[|lBw(45mm)|p|rw(55mm)|]
\FL
\NC Komponenter \NC \bf Udstyr \NC \bf Værdi \NC\NR
\ML
\NC Dæk \NC Standarddimensioner \NC 205/70 R 15 C \NC\NR
\ML
\NC Fælge \NC Standarddimensioner \NC 6J\;×\;15 H2 ET 60 \NC\NR
\ML
\NC Dæktryk \NC Standard, for/bag \NC 4,5/5,8\,bar \NC\NR
\LL
\stoptabulate


\subsection{Dieselmotor}

\starttabulate [|lBw(45mm)|l|rp|]
\FL
\NC \bf Gruppe\index{Dieselmotor+Identifikation} \NC \bf Parameter \NC \bf Værdi\NC\NR
\ML
\NC Motortype \NC \NC VW CJDA TDI 2.0 – 475 NE \NC\NR
\NC Generelt \NC Takt \NC Firetaktsmotor \NC\NR
\NC\NC Antal cylindere \unite{n} \NC 4 \NC\NR
\NC\NC Boring x slaglængde \unite{mm} \NC 81\;×\;95,5 \NC\NR
\NC\NC Total slagvolumen \unite{cm\high{3}} \NC 1968 \NC\NR
\NC\NC Ventiler pr. cylinder \NC 4 \NC\NR
\NC\NC Ventilstyringens rækkefølge \NC 1-3-4-2 \NC\NR
\NC\NC Laveste tomgangshastighed \unite{min\high{−1}} \NC 830 +50/−25 \NC\NR
\NC Effekt/drejningsmoment \NC Maks. omdrejningstal \unite{min\high{−1}} \NC 3400 \NC\NR
\NC\NC Maks. effekt \unite{kW} ved \unite{min\high{−1}} \NC 75 til 3000 \NC\NR
\NC\NC Maks. drejningsmoment \unite{Nm} ved \unite{min\high{−1}} \NC 285 til 1750 \NC\NR
\NC Specifikt forbrug\index{Dieselmotor+Forbrug} \NC Brændstof \unite{g/kWh} \NC 224 (ved maks. effekt) \NC\NR
\NC\NC Olie \unite{g/kWh} \NC 0,22 \NC\NR
\NC Brændstofsystem \NC Indsprøjtningssystem \NC Direkte indsprøjtning \quote{Common Rail} \NC\NR
\NC\NC Brændstofforsyning \NC Tandhjulspumpe \NC\NR
\NC\NC Opladning \NC Ja \NC\NR
\NC\NC Ladeluftkøling \NC Ja \NC\NR
\NC\NC Ladetryk \unite{mbar} \NC 1300\NC\NR
\NC Smørekreds\index{Dieselmotor+Smøring} \NC Type \NC Forceret smøring med olie-/vand-varmeveksler \NC\NR
\NC\NC Ledningstilførsel \NC Roterende pumpe \NC\NR
\NC\NC Olieforbrug \unite{liter/20\,h} \NC <\:0,1 \NC\NR
\NC Kølekreds\index{Dieselmotor+Køling} \NC Total kapacitet \unite{l} \NC ca. 12 \NC\NR
\NC\NC Kalibreringstryk ekspansionsbeholder \unite{bar} \NC 1,4 \NC\NR
\NC\NC Termostat (åbning) \unite{°C} \NC 87 \NC\NR
\NC\NC Termostat (fuld) \unite{°C} \NC 102 \NC\NR
\NC Udstødningsgas \NC Partikelfilter \NC Ja \NC\NR
\NC\NC Rensning af udstødningsgas \NC Ja \NC\NR
\NC\NC Standard \NC Euro 5 \NC\NR
\LL
\stoptabulate


\subsection{Køreydelser}

\starttabulate[|lBw(45mm)|p|l|rw(35mm)|]
\FL
\NC Køreydelse\index{Køreydelser} \NC \bf Konfiguration \NC \bf Enhed \NC \bf Værdi \NC\NR
\ML
\NC Hastighed \NC \aW{Arbejds}modus \NC \unite{km/h} \NC 0 til 18 (trinløs) \NC\NR
\NC\NC \aW{Køre}modus \NC \unite{km/h} \NC 0 til 40 \NC\NR
\ML
\NC Hastighedsbegrænsning \NC Indstillelig \NC \unite{km/h} \NC 0 til 25 \NC\NR
\LL
\stoptabulate


\subsection{Elektrisk system}

{\starttabulate [|lw(65mm)|p|rw(30mm)|]
\FL
\NC \bf Gruppe \NC \bf Komponenter \NC \bf Værdi \NC\NR
\ML
\NC Batteri \NC Blyakkumulator \NC 12\,V 63\,Ah \NC\NR
\NC Strømforsyning \NC Generator \NC 14,8\,V 90\,A \NC\NR
\NC Startaggregat \NC Effekt \NC 1,8\,kW \NC\NR
\NC Audioudstyr \NC Radiotilslutning\index{Radiotilslutning} og højttaler\index{Højttaler} \NC Standardudstyr \NC\NR
% \NC Sécurité et surveillance \NC Tachygraphe\index{tachygraphe} \NC En option \NC\NR
% \NC\NC Enregistreur de fin de parcours\index{fin de parcours} \NC En option \NC\NR
\NC Lys-/signaludstyr foran \NC Parkeringslys \NC 12\,V 5\,W \NC\NR
\NC\NC Nærlys \NC H7, 12\,V 55\,W \NC\NR
\NC\NC Arbejdslygter \NC G886, 12\,V 55\,W \NC\NR
\NC\NC Blinklys \NC 12\,V 21\,W \NC\NR
\NC Lys-/signaludstyr bag \NC Kombinerede stoplys \NC 12\,V 5/21\,W \NC\NR
\NC\NC Blinklys \NC 12\,V 21\,W \NC\NR
\NC\NC Baklygter \NC 12\,V 21\,W \NC\NR
\NC\NC Nummerpladelys \NC 12\,V 5\,W \NC\NR
\NC Ekstra lys \NC Rotorblink \NC H1, 12\,V 55\,W \NC\NR
\LL
\stoptabulate
}
\stopsection

\stopcomponent


