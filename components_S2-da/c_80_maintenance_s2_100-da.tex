\startcomponent c_80_maintenance_s2_100-da
\product prd_ba_s2_100-da

\startchapter [title={Service og vedligeholdelse},
reference={chap:maintenance}]

\setups[pagestyle:marginless]


\startsection [title={Generelle oplysninger}]


\subsection{Miljøbeskyttelse}

\starttextbackground [FC]
\setupparagraphs [PictPar][1][width=2.45em,inner=\hfill]

\startPictPar
\Penvironment
\PictPar
\Boschung\ implementerer miljøbeskyttelse\index{Miljøbeskyttelse} i praksis. Vi gør noget ved årsagerne og inddrager alle produktionsprocessens og produktets påvirkninger på miljøet i virksomhedens beslutninger.
Målene er sparsom brug af ressourcer og en skånsom behandling af de naturlige livsgrundlag, hvis bevarelse er til mennesker og naturens bedste.
De kan bidrage til miljøbeskyttelse ved at overholde visse regler under brugen af køretøjet. Hertil hører også en passende og
forskriftsmæssig håndtering af stoffer og materialer i forbindelse med
køretøjets vedligeholdelse (\eG\ bortskaffelse af kemikalier og farligt affald).

En motors brændstofforbrug og slitage afhænger af
driftsbetingelserne. Derfor beder vi Dem om at tage hensyn til et par punkter:

\startitemize
\item Lad ikke motoren køre varm i tomgang.
\item Sluk motoren under driftsbetingede ventetider.
\item Kontroller brændstofforbruget regelmæssigt.
\item {\em Lad et kompetent autoriseret værksted udføre vedligeholdelsesarbejdet i overensstemmelse med vedligeholdelsesplanen.}
\stopitemize
\stopSymList
\stoptextbackground

\page [yes]


\subsection{Sikkerhedsforskrifter}

\startSymList
\PHgeneric
\SymList
For\index{Vedligeholdelse+Sikkerhedsforskrifter} at forhindre skader på køretøjet og aggregater og ulykker under vedligeholdelsen, er det strengt nødvendigt, at overholde følgende sikkerhedsforskrifter. Overhold også de generelle sikkerhedsforskrifter (\about[safety:risques], \at{fra side}[safety:risques]).
\stopSymList

\starttextbackground [FC]
\startPictPar
\PMgeneric
\PictPar
\textDescrHead{Forebyggelse af ulykker}
Kontroller\index{Forebyggelse af ulykker} køretøjets tilstand efter hvert vedligeholdelses- eller reparationsarbejde. Kontroller især, at alle sikkerhedsrelevante komponenter og lygter og signaludstyr fungerer fejlfrit, før De kører på offentlige veje.
\stopPictPar
\stoptextbackground
\blank [big]

\start
\setupparagraphs [SymList][1][width=6em,inner=\hfill]
\startSymList\PHcrushing\PHfalling\SymList
\textDescrHead{Stabilisering af køretøjet}
Før enhver form for vedligeholdelsesarbejde påbegyndes, skal køretøjet sikres mod utilsigtede bevægelser: Stil gearvælgeren på \aW{Neutral}, træk håndbremsen og lås køretøjet med hjulkiler.
\stopSymList
\stop

\starttextbackground[CB]
\startPictPar\PHpoison\PictPar
\textDescrHead{Start af motoren}
Hvis\index{Fare+Forgiftning} De skal starte motoren et dårligt ventileret sted, må De kun lade den køre så længe som nødvendigt\index{Fare+Udstødningsgasser} for at forhindre kulilteforgiftning.
\stopPictPar
\startitemize
\item Start kun motoren, når batteriet er tilsluttet korrekt.
\item Frakobl aldrig batteriet mens motoren er i gang.
\item Start ikke motoren med en starthjælper.
Hvis\index{Batteri+Ladeapparat} batteriet skal oplades med en lynoplader, skal det først adskilles fra køretøjet. Læs og overhold de driftsmæssige forskrifter for lynopladeren.
\stopitemize
\stoptextbackground

\page [yes]


\subsubsection{Beskyttelse af de elektroniske komponenter}

\startitemize
\item Før\index{Elektrosvejsning} De påbegynder svejsearbejde, skal batterikablet adskilles fra batteriet og plus- og jordkablet sammenføjes.
\item Tilslut\index{Elektronik} og frakobl kun elektroniske styreanordninger, når de ikke er spændingsførende.
\item En forkert\index{Styreanordning} polaritet i strømforsyningen (\eG\
på grund af forkert tilsluttede batterier), kan ødelægge elektroniske komponenter og apparater.
\item Hvis\index{Omgivelsestemperatur+ekstrem} omgivelsestemperaturen er over 80 °C (\eG\ i et tørrekammer) skal elektroniske komponenter/apparater fjernes.
\stopitemize


\subsubsection{Diagnose og målinger}

\startitemize
\item Anvend kun {\em egnede}
testkabler til måle- og diagnosearbejde (\eG\ redskabets originale kabler).
\item Mobiltelefoner\index{Mobiltelefon} og lignende radioudstyr kan påvirke køretøjets funktioner, diagnoseapparatet og dermed den driftsmæssige sikkerhed.
\stopitemize


\subsubsection{Personalets kvalifikationer}

\starttextbackground[CB]
\startPictPar
\PHgeneric
\PictPar
\textDescrHead{Risiko for ulykker}
Hvis\index{Kvalifikation+Vedligeholdelsespersonale} vedligeholdelsesarbejdet udføres forkert, kan dette påvirke køretøjets funktion og sikkerhed. Dette medfører en øget risiko for ulykker og personskader.

Brug\index{Kvalifikation+Værksted} et kvalificeret, autoriseret værksted, der har den nødvendige viden og det nødvendige værktøj til at udføre vedligeholdelses- og reparationsarbejdet.

Kontakt i tvivlstilfælde \Boschung-kundeservice.
\stopPictPar
\stoptextbackground

\ProductId må kun betjenes, vedligeholdes eller repareres af personale, der er uddannet af \Boschung-kundeservice.

Kompetencerne til betjening, vedligeholdelse og reparation tildeles af \Boschung-kundeservice.

\page [yes]


\subsubsection{Ændringer og ombygninger}

\starttextbackground[CB]
\startPictPar
\PHgeneric
\PictPar
\textDescrHead{Risiko for ulykker}
Samtlige\index{Ændring på køretøjet} ændringer, De selv laver på køretøjet, kan påvirke \ProductId 's funktion og driftsmæssige sikkerhed, og dermed medføre en uundgåelig risiko for ulykker og personskader.
\stopPictPar

\startPictPar
\PMwarranty
\PictPar
For skader, der opstår på grund af\index{Garanti+Betingelser} egne indgreb eller modifikationer på \ProductId eller et aggregat, gives der ingen garanti eller kulance fra \Boschung\ 's side.
\stopPictPar
\stoptextbackground

\stopsection


\startsection [title={Driftsmidler og smøremidler}, reference={sec:liquids}]


\subsection{Korrekt anvendelse}

\starttextbackground[CB]
\startPictPar
\PHpoison
\PictPar
\textDescrHead{Risiko for personskade og forgiftning}
\index{Brændstof} Hudkontakt\index{Smøremiddel}
eller\index{Fare+Forgiftning} indtagelse af driftsmidler og smøremidler kan\index{Brændstof+Sikkerhed} medføre alvorlige kvæstelser eller forgiftninger. Overhold altid lovens forskrifter under håndtering, opbevaring og bortskaffelse af disse stoffer.
\stopPictPar
\stoptextbackground

\starttextbackground [FC]
\startPictPar
\PMproteyes\par
\PMprothands
\PictPar
Bær altid en passende beskyttelsesdragt og åndedrætsværn under håndtering af drifts- og smøremidler. Undgå at indånde dampene.
Undgå enhver kontakt med hud, øjne eller beklædningen. Rengør omgående steder på huden, der har været i kontakt med driftsmidlerne med vand og sæbe. Hvis der kommer driftsmidler i øjnene, skal øjnene skylles med rigelige mængder rent vand og søg evt. øjenlæge. Efter indtagelse af driftsmidler skal der omgående søges læge!
\stopPictPar
\stoptextbackground

\startSymList
\PPchildren
\SymList
Driftsmidler skal opbevares utilgængeligt for børn.
\stopSymList

\startSymList
\PPfire
\SymList
\textDescrHead{Brandfare}
Driftsmidler\index{Fare+Brand} er meget brandfarlige. Derfor er der en øget risiko for brand under håndtering af disse midler. Rygning\index{Rygeforbud} og åben ild er strengt forbudt under håndtering af driftsmidler.
\stopSymList

\starttextbackground [FC]
\startPictPar
\PMgeneric
\PictPar
Der må kun anvendes smøremidler, der er egnet til komponenterne i \ProductId. Anvend derfor kun produkter, der er testet og godkendt af \Boschung\. Disse produkter finder De på listen over driftsmidler \atpage[sec:liqquantities]. Additiver\index{Additiver} til smøremidler er ikke nødvendige. Hvis De tilsætter additiver, kan dette medføre, at garantikrav\index{Garanti+Betingelser} bortfalder.
Kontakt \Boschung's kundeservice for yderligere oplysninger.
\stopPictPar
\stoptextbackground

\starttextbackground [FC]
\startPictPar
\Penvironment
\PictPar
\textDescrHead{Miljøbeskyttelse}
Overhold\index{Smøremidler+Bortskaffelse} reglerne om miljøbeskyttelse ved bortskaffelse af drifts- og\crlf smøremidler\index{Miljøbeskyttelse} eller genstande, der er kontamineret med disse midler (\eG\ filtre, klude)\index{Driftsmidler+Bortskaffelse}.
\stopPictPar
\stoptextbackground

\page [yes]

\setups [pagestyle:normal]


\subsection[sec:liqquantities]{Specifikationer og påfyldningsmængder}

Samtlige\index{Driftsmidler+Påfyldningsmængde}\index{Smøremiddel+Påfyldningsmængde}\index{Påfyldningsmængder+Driftsmidler og smøremidler}\index{Specifikationer+Driftsmidler og smøremidler} påfyldningsmængder i følgende tabel er vejledende værdier. Hver gang et driftsmiddel/smøremiddel er udskiftet, skal den faktiske væskestand kontrolleres og påfyldningsmængden skal evt. øges eller reduceres.
% \blank[big]

\placetable[margin][tab:glyco]{Frostbeskyttelse (\index{Frostbeskyttelse}motor)}
{\noteF\startframedcontent[FrTabulate]
%\starttabulate[|Bp(80pt)|r|r|]
\starttabulate[|Bp|r|r|]
\NC Frostbeskyttelse til {[}°C{]}\NC \bf \textminus 25 \NC \bf \textminus 40 \NC\NR
\NC Destill. vand [Vol.-\%] \NC 60 \NC 40 \NC\NR
\NC Frostbeskyttelsesmiddel \break [Vol.-\%] \NC 40 \NC {\em maks.} 60 \NC\NR
\stoptabulate\stopframedcontent\endgraf
Bemærk: Ved en volumenprocent på mere end 60 \hairspace\percent\ frostbeskyttelsesmiddel {\em reduceres} frostbeskyttelsen og køleeffekten forringes!}

\placefig[margin][fig:hydrgauge]{\select{caption}{Niveauvisning
hydraulikvæske (køretøjets venstre side)}{Niveauvisning
hydraulikvæske}}
{\externalfigure[main:hy:level_temp]
\noteF Hydrauliktankens påfyldningsniveau kan aflæses på skueglasset og skal kontrolleres {\em dagligt}.}

\vskip -8pt
\start
\define [1] \TableSmallSymb {\externalfigure[#1][height=4ex]}
\define\UC\emptY
\pagereference[page:table:liquids]

\setupTABLE [frame=off,style={\ssx\setupinterlinespace[line=.86\lH]},background=color,
option=stretch,
split=repeat]
\setupTABLE [r] [each] [topframe=on,
framecolor=TableWhite,
% rulethickness=.8pt
]

\setupTABLE [c] [odd] [backgroundcolor=TableMiddle]
\setupTABLE [c] [even] [backgroundcolor=TableLight]
\setupTABLE [c] [1][width=30mm]
\setupTABLE [c] [2][width=20mm]
\setupTABLE [c] [4][width=25mm]
\setupTABLE [c] [last] [width=10mm]
\setupTABLE [r] [first] [topframe=off,style={\bfx\setupinterlinespace[line=.95\lH]},
% backgroundcolor=TableDark
]
\setupTABLE [r] [2][framecolor=black]

\bTABLE

\bTABLEhead
\bTR
\bTC Gruppe \eTC
\bTC Kategori \eTC
\bTC Klassificering \eTC
\bTC Produkt\note[Produkt] \eTC
\bTC Mængde \eTC
\eTR
\eTABLEhead

\bTABLEbody
\bTR \bTD Dieselmotor \eTD
\bTD Motorolie\eTD
\bTD \liqC{SAE 5W-30}; \liqC{VW 507.00}\eTD
\bTD Total Quartz INEO Long Life \eTD
\bTD 4,3\,l\eTD
\eTR
\bTR \bTD Hydraulikkreds \eTD
\bTD Hydraulikolie \eTD
\bTD \liqC{ISO VG 46} \eTD
\bTD Total Equiviz ZS 46 (tank ca. 40\,l) \eTD
\bTD ca. 50\,l\eTD
\eTR
\bTR \bTD Hydraulikkreds (valmulighed \aW{Bio})\eTD
\bTD Hydraulikolie \eTD
\bTD \liqC{ISO VG 46} \eTD
\bTD Total Biohydran TMP SE 46\eTD
\bTD ca. 50\,l\eTD
\eTR
\bTR \bTD Magnetventiler: Spolekerner \eTD
\bTD Smøremiddel\eTD
\bTD Kobberfedt \eTD
\bTD \emptY\eTD
\bTD NB:\note[Bedarf] \eTD
\eTR
\bTR \bTD Diverse: Låse, dørmekanisme, bremsepedal \eTD
\bTD Smøremiddel\eTD
\bTD Universal-spray\eTD
\bTD \emptY\eTD
\bTD NB:\note[Bedarf] \eTD
\eTR
\bTR \bTD Centralsmøring \eTD
\bTD Universal-lejefedt\eTD
\bTD \liqC{nlgi} 000 eller 00 (Li-sæbefedt)\eTD
\bTD Total Multis EP 00\eTD
\bTD NB:\note[Bedarf] \eTD
\eTR
\bTR \bTD Kølesystem \eTD
\bTD Frost-/rustbeskyttelsesmiddel\eTD
\bTD TL VW 774 F/G; maks. 60\hairspace\% vol.\eTD
\bTD G12+/G12++ (rosa/violet)\eTD
\bTD ca. 14\,l \eTD
\eTR
\bTR \bTD Højtryksvandpumpe \eTD
\bTD Motorolie\eTD
\bTD \liqC{SAE 10W-40}; \liqC{api cf – acea e6}\eTD
\bTD Total Rubia TIR 8900 \eTD
\bTD 0,29\,l\eTD
\eTR
\bTR \bTD Klimaanlæg \eTD
\bTD Kølemiddel\eTD
\bTD + 20\,ml POE-olie\eTD
\bTD R 134a\eTD
\bTD 700\,g\eTD
\eTR
\bTR \bTD Sprinkleranlæg \eTD
\bTD [nc=2] Vand og sprinklervæske, \aW{S} sommer, \aW{W} vinter; se blandingsforhold \eTD
\bTD Detailhandel \eTD
\bTD NB:\note[Bedarf] \eTD
\eTR
\eTABLEbody

\eTABLE
\stop

\footnotetext[Bedarf]{{\it NB:} efter behov, iht. den pågældende vejledning}
\footnotetext[Produkt]{Produkter anvendt af \Boschung\, eller andre produkter, der svarer til specifikationerne, kan også anvendes.}

\stopsection

\page [yes]

\setups [pagestyle:marginless]


\startsection [title={Vedligeholdelse af dieselmotoren},
reference={sec:workshop:vw}]


\subsection [sSec:vw:diagTool]{On-board-diagnosesystem}

\startregister[index][reg:main:vw]{Vedligeholdelse+Dieselmotor} Motorstyreanordningen (J623) er udstyret med en fejllog.
Hvis der opstår forstyrrelser i de overvågede sensorer eller komponenter, gemmes de sammen med oplysningerne om fejlens type i fejlloggen.

\index{Dieselmotor+Diagnose} Motorstyreanordningen skelner efter analysen af informationen mellem de forskellige fejlklasser og gemmer dem, indtil fejlloggens indhold slettes.

Fejl, der kun opstår {\em sporadisk}, vises med tilføjelsen \aW{SP}. Årsagen til sporadiske fejl kan \eG\ være en dårlig forbindelse, eller en kortvarig ledningsafbrydelse. Hvis en sporadisk fejl ikke længere opstår inden for 50 motorstarter, slettes den fra fejlloggen.

Hvis der er detekteret fejl, der påvirker motorens funktion, lyser kontrolsymbolet \aW{Motordiagnose} \textSymb{vpadWarningEngine1} på Vpad-skærmen.

De gemte fejl kan udlæses sammen med køretøjets diagnose, -måle og informationssystem \aW{VAS 5051/B}.

Når fejlen(e) er afhjulpet, skal fejlloggen slettes.


\subsubsection[sSec:vw:diagTool:connect]{Ibrugtagning af diagnosesystemet}

\starttextbackground [FC]
\startPictPar
\PMgeneric
\PictPar
De finder detaljerede oplysninger om køretøjets diagnosesystem VAS 5051/B i betjeningsvejledningen til systemet.

De kan også anvende andre kompatible diagnosesystemer, \eG\ \aW{DiagRA}.
\stopPictPar
\stoptextbackground

\page [yes]


\subsubsubsubject{Forudsætninger}

\startitemize
\item Sikringerne skal være i orden.
\item Batterispændingen skal være mere end 11,5 V.
\item Alle elektriske forbrugere skal være slukket.
\item Stelforbindelsen skal være i orden.
\stopitemize


\subsubsubsubject{Fremgangsmåde}

\startSteps
\item Sæt stikket til diagnoseledningen VAS 5051B/1 på diagnosetilslutningen.
\item Slå tændingen til, eller start motoren, alt efter funktionen.
\stopSteps

\subsubsubsubject{Valg af driftstype}

\startSteps [continue]
\item Tryk på displayet på knappen \aW{Køretøjs-selvdiagnose}.
\stopSteps


\subsubsubsubject{Valg af køretøjssystem}

\startSteps [continue]
\item Tryk på displayet på knappen \aW{01-motorelektronik}.
\stopSteps

På displayet vises nu identifikationen af styreanordningen og motorstyreanordningens kodning.

Hvis kodningerne ikke stemmer overens, skal styreanordningens kodning kontrolleres.


\subsubsubsubject{Valg af diagnosefunktion}

På displayet vises alle diagnosefunktioner, der kan udføres.

\startSteps [continue]
\item Tryk på knappen for den ønskede funktion på displayet.
\stopSteps



\subsection [sSec:vw:faultMemory]{Fejllog}


\subsubsection{Udlæsning af fejlloggen}

\subsubsubject{Arbejdsforløb}

\startSteps
\item Lad motoren køre i tomgang.
\item Tilslut VAS 5051/B (se \in{afsnit}[sSec:vw:diagTool:connect])
og vælg motorstyreanordningen.
\item Vælg diagnosefunktionen \aW{004-fejllogindhold}.
\item Vælg diagnosefunktionen \aW{004.01-Søg i fejllog}.
\stopSteps

{\sla Kun hvis motoren ikke starter:}

\startitemize [2]
\item Slå tændingen til.
\item Hvis der ikke er gemt en fejl i motorstyreanordningen, vises \aW{0 fejl fundet} på displayet.
\item Hvis der er gemt fejl i motorstyreanordningen, vises de under hinanden på displayet.
\item Afslut diagnosefunktionen.
\item Slå tændingen fra.
\item Afhjælp evt. viste fejl ved hjælp af fejltabellen (se servicedokumentationen) og slet derefter fejlloggen.
\stopitemize

\starttextbackground [FC]
\startPictPar
\PMrtfm
\PictPar
Hvis en fejl ikke kan slettes, kontakt venligst \Boschung's kundeservice.
\stopPictPar
\stoptextbackground


\subsubsubject{Statiske fejl}

Kontakt venligst Boschung's kundeservice, hvis der findes en eller flere statiske fejl i dataloggen for at afhjælpe disse fejl ved hjælp af en \aW{guidet fejlfinding}.


\subsubsubject{Sporadiske fejl}

Hvis der kun er gemt sporadiske fejl eller henvisninger i fejlloggen, og der ikke konstateres funktionsfejl i det elektroniske køretøjssystem, kan fejlloggen slettes:

\startSteps [continue]
\item Tryk på tasten \aW{Næste} \inframed[strut=local]{>}igen for at komme til kontrolplanen.
\item Tryk på tasten \aW{Hop} og derefter \aW{Af slut} for at afslutte den guidede fejlfinding.
\stopSteps

Der søges nu i alle fejllogge igen.

I et vindue bekræftes, at alle sporadiske fejl er slettet. Diagnoseprotokollen sendes automatisk (online).

Køretøjssystemtesten er dermed afsluttet.


\subsubsection[sSec:vw:faultMemory:errase]{Sletning af fejlloggen}

\subsubsubject{Arbejdsforløb}

{\sla Forudsætninger:}

\startitemize [2]
\item Alle fejl skal være afhjulpet og fejlårsagerne fjernet.
\stopitemize

\page [yes]


{\sla Fremgangsmåde:}

\starttextbackground [FC]
\startPictPar
\PMrtfm
\PictPar
Efter fejlafhjælpningen skal der søges igen i fejlloggen og derefter skal den slettes:
\stopPictPar
\stoptextbackground

\startSteps
\item Lad motoren køre i tomgang.
\item Tilslut VAS 5051/B (se \in{afsnit}[sSec:vw:diagTool:connect])
og vælg motorstyreanordningen.
\item Vælg diagnosefunktionen \aW{004-Søg i fejllog}.
\item Vælg diagnosefunktionen \aW{004.10-Slet fejllog}.
\stopSteps

\starttextbackground [FC]
\startPictPar
\PMrtfm
\PictPar
Hvis fejlloggen ikke kan slettes, findes der stadig en fejl, som skal afhjælpes.
\stopPictPar
\stoptextbackground

\startSteps [continue]
\item Afslut diagnosefunktionen.
\item Slå tændingen fra.
\stopSteps


\subsection [sSec:vw:lub] {Smøring af dieselmotoren}

\subsubsection [ssSec:vw:oilLevel] {Kontrol af motoroliestanden}

\starttextbackground [FC]
\startPictPar
\PMrtfm
\PictPar
\index{Motorolie+-stand}Oliestanden må aldrig overskride \aW{Max.}-markeringen. Ellers er der\index{Påfyldningsniveau+Motorolie} risiko for katalysatorskader.
\stopPictPar
\stoptextbackground

\startSteps
\item Sluk motoren og vent mindst 3 minutter, så olien kan løbe tilbage i bundkarret.
\item Træk målepinden ud og tør den af. Skub pinden i igen så langt den kan komme.
\item Træk pinden ud igen og vurder oliestanden:

\startfigtext[right][fig:vw:gauge]{Aflæsning af oliestanden}
{\externalfigure[VW_Oil_Gauge][width=50mm]}
\startitemize [A]
\item Maksimalt påfyldningsniveau; der må ikke fyldes mere olie på.
\item Tilstrækkeligt påfyldningsniveau; der {\em kan} fyldes olie på til markeringen \aW{A}.
\item Utilstrækkeligt påfyldningsniveau; der {\em skal} fyldes olie på, indtil påfyldningsniveauet er ved \aW{B}.
\stopitemize
{\em Hvis påfyldningsniveauet er over markeringen \aW{A} er der risiko for katalysatorskader.}
\stopfigtext
\stopSteps


\subsubsection [ssSec:vw:oilDraining] {Motorolieskift}

\starttextbackground [FC]
\startPictPar
\PMrtfm
\PictPar
Motoroliefilteret i S2 er monteret opretstående. Det betyder, at filteret skal udskiftes {\em før} olieskift. Når filterelementet tages ud, åbnes en ventil, og olien i filterhuset løber automatisk ind i krumtaphuset.
\stopPictPar
\stoptextbackground

\startSteps
\item Stil en egnet\index{Dieselmotor+Olieskift} opsamlingsbeholder under motoren.
\item Skru olieaftapningsskruen\index{Motorolie+-skift} ud og lad olien løbe ud.
\stopSteps

\starttextbackground [FC]
\startPictPar
\PMrtfm
\PictPar
Sørg for, at opsamlingsbeholderen har kapacitet til den totale spildoliemængde.
De finder den nødvendige oliespecifikation og påfyldningsmængde i \in{afsnit}[sec:liqquantities].

Olieaftapningsskruen er forsynet med en tætningsring, der ikke kan fjernes. Olieaftapningsskruen skal derfor altid udskiftes.
\stopPictPar
\stoptextbackground

\startSteps [continue]
\item Skru en ny olieaftapningsskrue med tætningsring i (\TorqueR 30 Nm).
\item Fyld motorolie med en egnet specifikation på (se \in{afsnit}[sec:liqquantities]).
\stopSteps


\subsubsection [ssSec:vw:oilFilter] {Udskiftning af motoroliefilteret}

\starttextbackground [FC]
\startPictPar
\PMrtfm
\PictPar
\startitemize [1]
\item Overhold\index{Dieselmotor+Oliefilter} forskrifterne om bortskaffelse og genbrug.
\item Udskift\index{Oliefilter+Dieselmotor} filteret {\em før} olieskift (se \in{afsnit}[ssSec:vw:oilDraining]).
\item Påfør en smule olie på det nye filters pakning, før det monteres.
\stopitemize
\stopPictPar
\stoptextbackground

\startfigtext[right][fig:vw:oilFilter]{Oliefilter}
{\externalfigure[VW_OilFilter_03][width=50mm]}
\startSteps
\item Skru dæksel \Lone\ på filterhuset af med en egnet skruetrækker.
\item Rengør dækslets og filterhusets tætningsflader.
\item Udskift filterelementet \Lthree.
\item Udskift O-ringene \Ltwo\ og \Lfour.
\item Skru dækslet på filterhuset igen (\TorqueR 25 Nm).
\stopSteps



%\subsubsubject{Données techniques}
%
%
%\hangDescr{Couple de serrage du couvercle:} \TorqueR 25 Nm.
%
%\hangDescr{Huile moteur prescrite:} Selon tableau \atpage[sec:liqquantities].
%% NOTE: Redundant [tf]

\stopfigtext



\subsubsection [ssSec:vw:oilreplenish] {Påfyldning af motorolie}

\starttextbackground [FC]
\startPictPar
\PMrtfm
\PictPar
\startitemize [1]
\item Rengør\index{Motorolie} påfyldningsstudsen med en klud, {\em før} De tager dækslet af.
\item Fyld\index{Dieselmotor+Påfyldning af olie} kun olie på, der svarer til den foreskrevne specifikation.
\item Fyld små mængder på ad gangen.
\item Vent lidt efter hver påfyldning for at undgå overfyldning og for at olien kan løbe i bundkarret til målepindens markering (se \in{afsnit}[ssSec:vw:oilLevel]).
\stopitemize
\stopPictPar
\stoptextbackground

\startfigtext[right][fig:vw:oilFilter]{Påfyldning af olie}
{\externalfigure[s2_bouchonRemplissage][width=50mm]}
\startSteps
\item Træk oliemålepinden ca. 10 cm ud, så luften kan slippe ud under påfyldningen.
\item Åbn påfyldningsåbningen.
\item Fyld olie på iht. forskrifterne ovenfor.
\item Luk påfyldningsåbningen omhyggeligt.
\item Start motoren.
\item Kontroller påfyldningsniveauet. (Se \in{afsnit}[ssSec:vw:oilLevel].)
\stopSteps

\stopfigtext


\subsection [sSec:vw:fuel] {Brændstofforsyningssystem}

\subsubsection [ssSec:vw:fuelFilter] {Udskiftning af brændstoffilteret}

\starttextbackground [FC]
\startPictPar
\PMrtfm
\PictPar
\startitemize [1]
\item Overhold\index{Dieselmotor+Brændstoffilter} lovens forskrifter om bortskaffelse og genbrug af farligt affald.
\item Tag ikke brændstofledningerne af filterets øverste del.
\item Træk ikke i brændstofledningernes fastgørelsespunkter; i modsat fald kan filterets øverste del blive beskadiget.
\stopitemize
\stopPictPar
\stoptextbackground

\startfigtext[right][fig:vw:oilFilter]{Brændstoffilter}
{\externalfigure[s2_fuelFilter_location][width=50mm]}

{\sla Forberedelse:}

\index{Brændstoffilter} Brændstoffilterhuset er fastgjort foran motoren på højre side af chassiset.
Fjern de to fastspændingsskruer med en 10 mm topnøgle og en 10 mm ringnøgle.

\stopfigtext


\page [yes]

\setups [pagestyle:normal]

{\sla Fremgangsmåde:}

\startLongsteps
\item Fjern alle skruer i filterets øverste del. Tag filterets øverste del af.
\stopLongsteps

\starttextbackground [FC]
\startPictPar
\PMrtfm
\PictPar
Løft den øverste del af. Sæt en vinkelskruetrækker i monteringsnoten, (\in{\LAa, fig.}[fig:fuelfilter:detach]), hvis nødvendigt, og løft den øverste del af.
\stopPictPar
\stoptextbackground

\placefig [margin] [fig:fuelfilter:detach]{Udtagning af brændstoffilteret}
{\externalfigure[fuelfilter:detach]}

\placefig [margin] [fig:fuelfilter:explosion]{Brændstoffilter}
{\externalfigure[fuelfilter:explosion]}

\startLongsteps [continue]
\item Træk filterelementet ud af filterets underdel.
\item Tag pakningen (\in{\Ltwo, fig.}[fig:fuelfilter:explosion]) af filterets øverste del.
\item Rengør filterets under- og overdel omhyggeligt.
\item Sæt et nyt filterelement i filterets underdel.
\item Fugt en ny pakning (\in{\Ltwo, fig.}[fig:fuelfilter:explosion]) med en smule brændstof og sæt den i den øverste del.
\item Sæt den øverste del på filterets underdel og tryk den fast, så den øverste del ligger ensartet hele vejen rundt.
\item Skru over- og underdelen {\em håndfast} sammen igen med alle skruer. Spænd derefter alle skruer på kryds med det forskrevne tilspændingsmoment (\TorqueR 5 Nm).
\stopLongsteps

% \subsubsubject{Données techniques}
%
% \hangDescr{Couple de serrage des vis de fixation du couvercle:} \TorqueR 5 Nm.
%% NOTE: redundant [tf]

\startLongsteps [continue]
\item Slå tændingen til for at udlufte systemet; start motoren og lad den køre 1 til 2 minutter i tomgang.
\item Slet fejlloggen, som beskrevet på \atpage[sSec:vw:faultMemory:errase].
\stopLongsteps


\subsection [sSec:vw:cooling] {Kølesystem}

\starttextbackground [FC]
\startPictPar
\PMrtfm
\PictPar
\startitemize [1]
\item Brug\index{Dieselmotor+Køling} kun kølemidler med den foreskrevne specifikation (se tabellen \atpage[sec:liqquantities]).
\item For\index{Kølemiddel} at sikre frost- og korrosionsbeskyttelsen, må kølemidlet kun fortyndes med destilleret vand og i overensstemmelse med tabellen nedenfor.
\item Fyld aldrig kølemiddelkredsen op med vand, fordi dette kan forringe frost- og korrosionsbeskyttelsen.
\stopitemize
\stopPictPar
\stoptextbackground


\subsubsection [sSec:vw:coolingLevel] {Kølemiddelstand}

\placefig [margin] [fig:coolant:level] {Kølemiddelstand}
{\externalfigure[coolant:level]}


\placefig [margin] [fig:refractometer] {Refraktometer VW T 10007}
{\externalfigure[coolant:refractometer]}

\placefig [margin] [fig:antifreeze] {Kontrol af frostsikringspunkt/blandingsforhold}
{\externalfigure[coolant:antifreeze]}


\startSteps
\item Løft smudsbeholderen.
\item Kontroller\index{Påfyldningsniveau+Kølemiddel} kølemidlets påfyldningsniveau i ekspansionsbeholderen: Det skal være over \aW{min}-markeringen.
\stopSteps

\start
\define [1] \TableSmallSymb {\externalfigure[#1][height=4ex]}
\define\UC\emptY
\pagereference[page:table:liquids]


\setupTABLE [frame=off,style={\ssx\setupinterlinespace[line=.86\lH]},background=color,
option=stretch,
split=repeat]
\setupTABLE [r] [each] [topframe=on,
framecolor=TableWhite,
% rulethickness=.8pt
]

\setupTABLE [c] [odd] [backgroundcolor=TableMiddle]
\setupTABLE [c] [even] [backgroundcolor=TableLight]
\setupTABLE [r] [first] [topframe=off,style={\bfx\setupinterlinespace[line=.95\lH]},
% backgroundcolor=TableDark
]
\setupTABLE [r] [2][framecolor=black]

\bTABLE

\bTABLEhead
\bTR
\bTC Frostbeskyttelse til … \eTC
\bTC Andel G12\hairspace ++\eTC
\bTC Vol. frostbeskyttelsesmiddel \eTC
\bTC Vol. destilleret vand \eTC
\eTR
\eTABLEhead

\bTABLEbody
\bTR \bTD \textminus 25 °C \eTD
\bTD 40\hairspace\% \eTD
\bTD 3,8 l \eTD
\bTD 4,2 l \eTD
\eTR
\bTR \bTD \textminus 35 °C \eTD
\bTD 50\hairspace\% \eTD
\bTD 4,0 l \eTD
\bTD 4,0 l \eTD
\eTR
\bTR \bTD \textminus 40 °C \eTD
\bTD 60\hairspace\% \eTD
\bTD 4,2 l \eTD
\bTD 3,8 l \eTD
\eTR
\eTABLEbody

\eTABLE
\stop

\adaptlayout [height=+20pt]
\subsubsection [sSec:vw:coolingFreeze] {Kølemiddelstand}

Kontroller\index{Frostsikringspunkt/blandingsforhold } frostsikringspunktet/blandingsforholdet med et egnet refraktometer (se \in{fig.}[fig:refractometer]: VW T 10007).
Se skala 1: G12\hairspace ++ (se \in{fig.}[fig:antifreeze]).

\page [yes]


\subsection [sSec:vw:airFilter] {Luftforsyning}

Luftfilteret er tilgængeligt gennem den bagerste servicedør på højre side af køretøjet (se \in{fig.}[fig:airFilter]).

\placefig [margin] [fig:airFilter] {Motorens luftfilter}
{\externalfigure[vw:air:filter]
\noteF
\startLeg
\item Sikkerhedslask
\item Husets underdel
\item Ventilationsåbning
\item Tryksensor
\stopLeg}


\subsubsubject{Anvendelsesbetingelser}

En fejemaskine anvendes ofte i meget støvede omgivelser. Derfor er det nødvendigt, at kontrollere og rengøre luftfilteret ugentligt. Se også \about[table:scheduleweekly], \atpage[table:scheduleweekly]. Hvis nødvendigt, skal luftfilteret udskiftes.


\subsubsubject{Autodiagnostic}

Indsugningsledningen er udstyret med en tryksensor (\Lfour, \in{fig.}[fig:airFilter]), som detekterer en forringet luftgennemstrømning\footnote{Forringet luftgennemstrømning på grund af en forringet luftgennemtrængelighed i filteret.} gennem filteret.
Hvis luftfilteret er tilstoppet, lyser kontrolsymbolet \textSymb{vpadWarningFilter} på Vpad-skærmen og fejlmeldingen \VpadEr{851} registreres.


\subsubsubject{Vedligeholdelse/udskiftning}

\startSteps
\item Træk sikkerhedslasken \Lone nedad (\in{fig.}[fig:airFilter]).
\item Drej husets underdel \Ltwo mod uret og tag den af.
\item Tag filterelementet ud og kontroller det. Udskift det, hvis nødvendigt.
\item Rengør filteret inden i og saml luftfilteret i omvendt rækkefølge igen.
\stopSteps

\page [yes]


\subsection [sSec:vw:belt] {Kilerem}

\index{Dieselmotor+Kilerem}Kileremmen overfører krumtapakselsvinghjulets bevægelse til generatoren og klimakompressoren (ekstraudstyr).
Et\index{Kilerem} spændeelement i sidste segment (mellem generatoren og krumtapakslen) holder remmen stram.


\subsubsection [sSec:belt:change] {Udskiftning af kileremmen}

\placefig [margin] [fig:belt:tool] {Dorn VW T 10060 A}
{\externalfigure[vw:belt:tool]}

\placefig [margin] [fig:belt:overview] {Spændeelement}
{\externalfigure[vw:belt:overview]}

\placefig [margin] [fig:belt:tens] {Dornens placering}
{\externalfigure[vw:belt:tens]}


\subsubsubject{Med klimakompressor}


{\sla Nødvendigt specialværktøj:}

Dorn \aW{VW T 10060 A} til at holde spændeelementet.

\startSteps
\item Marker kileremmens omdrejningsretning.
\item Drej spændeelementets arm med uret med en ringnøgle med knæk (\in {fig.}[fig:belt:overview]).
\item Tilpas hullerne (se pilene, \in {fig.}[fig:belt:tens]) og lås spændeelementet med dornen.
\item Tag kileremmen af.
\stopSteps

Kileremmen monteres i omvendt rækkefølge.

\starttextbackground [FC]
\startPictPar
\PMrtfm
\PictPar
\startitemize [1]
\item Vær opmærksom på kileremmens omdrejningsretning.
\item Sørg for, at remmen ligger rigtigt i remskiverne.
\item Start motoren og kontroller remmens omdrejning.
\stopitemize
\stopPictPar
\stoptextbackground


\subsubsubject{Uden klimakompressor}

{\sla Nødvendigt materiale:}

Reparationssæt, der består af en reparationsvejledning, kilerem og specialværktøj.\footnote{Se reservedelskataloget under \aW{Vedligeholdelsesdele}.}

\startSteps
\item Skær kileremmen over.
\item Følg de næste trin i reparationsvejledningen.
\stopSteps

\starttextbackground [FC]
\startPictPar
\PMrtfm
\PictPar
\startitemize [1]
\item Sørg for, at remmen ligger rigtigt i remskiverne.
\item Start motoren og kontroller remmens omdrejning.
\stopitemize
\stopPictPar
\stoptextbackground


\subsubsection [sSec:belt:tens] {Udskiftning af spændeelementet}

{\sla Kun for version med klimakompressor}

\blank [medium]

\placefig [margin] [fig:belt:tens:change] {Udskiftning af spændeelementet}
{\externalfigure[vw:belt:tens:change]
\noteF
\startLeg
\item Spændeelement
\item Låseskrue
\stopLeg

{\bf Tilspændingsmoment}

Låseskrue:

\TorqueR 20 Nm\:+ ½ omdrejning (180°).}

\startSteps
\item Afmonter kileremmen, som beskrevet (se \atpage[sSec:belt:change]).
\item Afmonter de perifere dele (afhængigt af udstyr).
\item Skru låseskruen ud (\in{\Ltwo, fig.}[fig:belt:tens:change]).
\stopSteps

Spændeelementet monteres i omvendt rækkefølge.

\starttextbackground [FC]
\startPictPar
\PMrtfm
\PictPar
\startitemize [1]
\item Brug altid en ny låseskrue efter monteringen.
\item Tilspændingsmoment: Se \in{fig.}[fig:belt:tens:change].
\stopitemize
\stopPictPar
\stoptextbackground

\stopregister[index][reg:main:vw]

\stopsection

\page[yes]


\setups[pagestyle:marginless]


\startsection[title={Hydrauliksystem},
reference={sec:hydraulic}]

\starttextbackground [FC]
% \startfiguretext[left,none]{}
% {\externalfigure[toni_melangeur][width=30mm]}

\startSymPar
\externalfigure[toni_melangeur][width=4em]
\SymPar
\textDescrHead{Genbrug af driftsmidler}
Brugte driftsmidler og smøremidler må hverken bortskaffes i naturen eller forbrændes.

Brugte smøremidler må hverken udledes i kloaksystemet eller i naturen og må ikke bortskaffes sammen med husholdningsaffald.

Brugte smøremidler må ikke blandes med andre væsker på grund af risikoen for, at der dannes giftstoffer eller stoffer, der er vanskelige at bortskaffe.
\stopSymPar
\stoptextbackground
\blank [big]

% \starthangaround{\PMgeneric}
% \textDescrHead{Qualification du personnel}
% Toute intervention sur l’installation hydraulique de votre véhicule ne peut être réalisée que par une personne dument qualifiée, ou par un service reconnu par \boschung.
% \stophangaround
% \blank[big]

\startSymList
\PHgeneric
\SymList
\textDescrHead{Renlighed} Hydrauliksystemet reagerer meget følsomt på urenheder i olie. Derfor er det vigtigt at arbejde i helt rene omgivelser.
\stopSymList

\startSymList
\PHhot
\SymList
\textDescrHead{Risiko for stænk}
Før arbejdet på hydrauliksystemet i \sdeux\ påbegyndes, skal det resterende tryk i den pågældende hydraulikkreds lukkes ud. Stænk fra varm olie kan medføre forbrændinger.
\stopSymList

\startSymList
\PHhand
\SymList
\textDescrHead{Klemningsfare}
Smudsbeholderen skal altid sænkes, eller være sikret mekanisk med sikkerhedsafstivningen, før arbejdet på hydrauliksystemet i \sdeux\ påbegyndes.
\stopSymList

\startSymList
\PImano
\SymList
\textDescrHead{Trykmåling}
Placer et manometer på en af kredsens \aW{Minimess}-tilslutninger for at måle hydrauliktrykket. Kontroller, at manometeret viser et egnet måleområde.
\stopSymList

\page [yes]

\setups[pagestyle:normal]

\subsection{Vedligeholdelsesintervaller}

\start

\setupTABLE [frame=off,
style={\ssx\setupinterlinespace[line=.93\lH]},
background=color,
option=stretch,
split=repeat]
\setupTABLE [r] [each] [
topframe=on,
framecolor=white,
backgroundcolor=TableLight,
% rulethickness=.8pt,
]

% \setupTABLE [c] [odd] [backgroundcolor=TableMiddle]
% \setupTABLE [c] [even] [backgroundcolor=TableLight]
\setupTABLE [c] [1][ % width=30mm,
style={\bfx\setupinterlinespace[line=.93\lH]},
]
\setupTABLE [r] [first] [topframe=off,
style={\bfx\setupinterlinespace[line=.93\lH]},
backgroundcolor=TableMiddle,
]
% \setupTABLE [r] [2][style={\ssBfx\setupinterlinespace[line=.93\lH]}]


\bTABLE

\bTABLEhead
\bTR\bTD Vedligeholdelsesarbejde \eTD\bTD Interval \eTD\eTR
\eTABLEhead

\bTABLEbody
\bTR\bTD Kontroller for lækager \eTD\bTD Dagligt \eTD\eTR
\bTR\bTD Kontroller hydraulikoliestanden \eTD\bTD Dagligt \eTD\eTR
\bTR\bTD Kontroller hydraulikledningernes/-slangernes tilstand og udskift evt. \eTD\bTD 600 h / 12 måneder \eTD\eTR
\bTR\bTD Udskift hydraulikolie-retur- og indsugningsfilter \eTD\bTD 600 h / 12 måneder \eTD\eTR
\bTR\bTD Smør magnetventilernes spolekerner med kobberfedt \eTD\bTD 600 h / 12 måneder \eTD\eTR
\bTR\bTD Udskift hydraulikolien \eTD\bTD 1200 h / 24 måneder \eTD\eTR
\eTABLEbody
\eTABLE
\stop


\subsection[niveau_hydrau]{Påfyldningsniveau}

\placefig[margin][fig:hydraulic:level]{Hydraulikvæskens påfyldningsniveau}
{\externalfigure[hydraulic:level]
\noteF
\startLeg
\item Optimalt påfyldningsniveau
\stopLeg}

Et gennemsigtigt skueglas\index{Påfyldningsniveau+Hydraulikvæske}\index{Vedligeholdelse+Hydrauliksystem}
giver mulighed for at kontrollere hydraulikoliestanden.
Hvis hydraulikoliestanden er faldet, skal årsagen findes,
før der fyldes olie på igen. Overhold de foreskrevne udskiftningsintervaller (tabellen ovenfor)
og specifikationer for hydraulikvæske (tabellen \at{side}[sec:liqquantities]).

\subsubsection{Påfyldning af hydraulikvæske}

Fyld hydraulikvæske på, indtil det mellemste skueglas er helt dækket.
Start motoren og fyld evt. mere olie på, indtil det påkrævede påfyldningsniveau.

\subsection{Udskiftning af hydraulikvæske}

De finder hydraulikvæskens påfyldningsmængde og nødvendige specifikationer i tabellen på \at{side}[sec:liqquantities].

\startSteps
\item Åbn hydrauliktankens påfyldningsåbning.
\item Tøm tanken med en oliesuger, eller fjern aftapningsskruen.

Aftapningsskruen findes nederst på hydrauliktanken, foran venstre baghjul (\in{fig.}[fig:hydraulic:fluidDrain]).
\item Fyld hydraulikvæske på til midten af skueglasset.
Start motoren og fyld evt. mere olie på, indtil det påkrævede påfyldningsniveau.
\stopSteps

\placefig[margin][fig:hydraulic:fluidDrain]{Aftapningsskrue}
{\externalfigure[hydraulic:fluidDrain]}


\placefig[margin][fig:hydraulic:returnFilter]{Hydraulikfilter}
{\externalfigure[hydraulic:returnFilter]}

\subsection[filtres:nettoyage]{Retur- og indsugningsfilter}

\startSteps
\item Løft smudsbeholderen og anbring sikkerhedsafstivningen.
\item Tag filterets dæksel på hydrauliktanken af (\in{fig.}[fig:hydraulic:returnFilter]).
\item Udskift\index{Oliefilter+Hydraulik-} filterelementet med et nyt.
\item Fugt en ny O-ring-pakning med en smule hydraulikvæske og anbring den.
\item Skru dækslet på igen med begge hænder (\TorqueR ca. 20 Nm).
\stopSteps

\page [yes]


\subsection{Smøring af magnetventilerne}

\placefig[margin][graissage_bobine]{Smøring af magnetventilerne}
{\externalfigure[graissage_bobine][M]
\noteF
\startLeg
\item Magnetventilens spole
\item Spolekerne
\stopLeg}

Fugt og saltrester, der trænger ind i de elektromagnetiske spolers kerne medfører, at kernerne korroderer. Spolekernerne skal smøres en gang årligt med kobberfedt. Fedtet skal ære korrosions-, vand- og temperaturbestandigt til 50 °C:
\startSteps
\item Afmonter magnetventilens spole (\in{\Lone, fig.}[graissage_bobine]).
\item Smør kernen (\in{\Ltwo, fig.}[graissage_bobine]) med det foreskrevne specialfedt og monter spolen igen.
\stopSteps


\subsection{Udskiftning af slangerne}

Slangernes beskyttelsesgummi\index{Slanger+Udskiftningsintervaller} og det vævede stof til forstærkning er udsat for naturlig ældning. Derfor er det meget vigtigt, at slangerne i hydrauliksystemet udskiftes i de foreskrevne intervaller, også selvom de ikke har {\em synlige} skader.

Sørg for, at slangerne påflanges korrekt på køretøjet for at udelukke en for tidlig slitage på grund af friktion. De skal have en tilstrækkelig afstand til andre komponenter, så friktions- og vibrationsskader forhindres.

\stopsection

\page [yes]

\setups [pagestyle:bigmargin]


\startsection[title={Bremsesystem},
reference={sec:brake}]

\placefig[margin][fig:brake:rear]{Tromlebremse}
{\startcombination [1*2]
{\externalfigure[brake:wheelHub]}{\slx Baghjulsnav}
{\externalfigure[brake:drum]}{\slx Mekanisme og bremsesæt}
\stopcombination}

Bremsetromlerne \Lfour\ skal afmonteres ved hver regelmæssig vedligeholdelse, bremsemekanismen \Lseven\ skal renses og bremsesættene \Lfive,
\Lsix\ skal kontrolleres visuelt (\in{fig.}[fig:brake:rear]).


\subsubject {Afmontering}

\startSteps
\item Kør køretøjet op på en egnet lift og løft hjulene.
\item Tag hjulene af.
\stopSteps


{\sla Afmontering af forhjulsbremserne}

\startSteps [continue]
\item Afmonter bremsetromlen\Lfour.
\stopSteps

{\sla Afmontering af baghjulsbremserne}

\startSteps [continue]
\item Tag afskærmningen \Lone\ af navet.
\item Fjern skruen \Ltwo\ og tag mellemstykket af.
\item Skru navmøtrikken \Lthree\ af med en topnøgle.
\item Tag navet med bremsetromlen af.
\stopSteps


\subsubject {Genmontering}

Monter bremsetromlerne igen i omvendt rækkefølge. Spænd møtrikkerne i baghjulsnavene \Lthree\ med det foreskrevne tilspændingsmoment på 190 Nm.

\stopsection

\page [yes]

\setups [pagestyle:normal]


\startsection[title={Kontrol og vedligeholdelse af dækkene},
reference={sec:pneumatiques}]

Dækkene\index{Dæk+Vedligeholdelse} skal altid være i en fejlfri tilstand for at de kan opfylde deres to hovedfunktioner: et godt vejgreb og upåklagelige bremseegenskaber. Et utilladeligt højt slid og et forkert dæktryk, især et for lavt dæktryk, er vigtige faktorer, der kan forårsage ulykker.


\subsection{Sikkerhedsrelevante punkter}

\subsubsection{Kontrol af slitage}

Slitagen af dækkene skal kontrolleres ved hjælp af de indikatorer for slitage, der findes i en profilrille (\in{fig.}[pneususure]).
Uregelmæssigheder på dækkene og årsagerne dertil kan konstateres ved hjælp af en visuel kontrol:

\placefig[margin][pneususure]{Kontrol af slitage}
{\Framed{\externalfigure[pneusUsure][M]}}

\placefig[margin][pneusdomages]{Beskadiget dæk}
{\Framed{\externalfigure[pneusDomages][M]}}

\startitemize
\item Slitage på siderne af slidbanen: For lavt dæktryk.
\item Kraftig slitage i midten: For højt dæktryk.
\item Asymmetrisk slitage i begge sider af dækket: Forakslen (spor, akselgeometri) er forkert indstillet.
\item Revner i slidbanen: Dækket er for gammelt. Dækgummi bliver hårdt og revner med tiden (\in{fig.}[pneusdomages]).
\stopitemize

\starttextbackground[CB]
\startPictPar
\PHgeneric
\PictPar
\textDescrHead{Risici ved nedslidte dæk}
Et nedslidt dæk opfylder ikke længere dets funktioner, især under kørsel i vand og mudder. Bremselængden øges og køreegenskaberne forringes. Et nedslidt dæk glider nemmere, især i vådt føre. Risikoen for, at dækket mister vejgrebet, øges.
\stopPictPar
\stoptextbackground


\subsubsection{Dæktryk}

Det foreskrevne dæktryk er noteret på dækkets typeskilt foran på konsollen i passagersiden (se \atpage [sec:plateWheel]).

Selvom\index{Dæk+Dæktryk} dækkenes tilstand er god, mister de med tiden mere eller mindre hurtigt luft (jo mere køretøjet bliver brugt, desto større tryktab). Derfor skal dæktrykket kontrolleres en gang om måneden når dækkene er kolde. Hvis De kontrollerer trykket når dækkene er varme, skal De lægge 0,3 bar til det foreskrevne tryk.

\start
\setupcombinations[M]
\placefig[margin][pneuspression]{Dæktryk}
{\Framed{\externalfigure[pneusPression][M]}
\noteF
\startLeg
\item Korrekt tryk
\item For højt tryk
\item For lavt tryk
\stopLeg
Det foreskrevne dæktryk er angivet på hjulenes typeskilt, i førerhuset i passagersiden.}
\stop

\starttextbackground[CB]
\startPictPar
\PHgeneric
\PictPar
\textDescrHead{Risici på grund af for lavt dæktryk}
Et dæk kan flænges, hvis trykket er for lavt. Dækket trykkes mere sammen, hvis det ikke er pumpet tilstrækkeligt op, eller hvis køretøjet er overlæsset. Gummiet bliver for varmt og dele af dækket kan løsne sig i et sving.
\stopPictPar
\stoptextbackground

\stopsection

\page [yes]

\setups[pagestyle:marginless]


\startsection[title={Chassis},
reference={main:chassis}]

\subsection{Sikkerhedsrelevant fastgørelse af komponenter}

Under hver vedligeholdelse skal man kontrollere, at sikkerhedsrelevante fastspændingsskruer for visse komponenter sidder korrekt, herunder kontrol af det foreskrevne tilspændingsmoment. Dette gælder især for knækstyringssystemer og akslerne.

\blank [big]

\startfigtext [left] [fig:frontAxle:fixing] {Foraksel}
{\externalfigure [frontAxle:fixing]}
{\sla Forakslens fastgørelser}
\startLeg
\item Fjederbladets fastgørelse: \TorqueR 150 Nm
\item Trækenhedernes fastgørelse: \TorqueR 78 Nm
\stopLeg

{\sla Bagakslens fastgørelser}
\startLeg
\item Fjederbladets fastgørelse: \TorqueR 150 Nm
\stopLeg

\stopfigtext

\start

\setupTABLE [frame=off,style={\ssx\setupinterlinespace[line=.93\lH]},background=color,
option=stretch,
split=repeat]

\setupTABLE [r] [each] [topframe=on,
framecolor=white,
% rulethickness=.8pt
]

\setupTABLE [c] [odd] [backgroundcolor=TableMiddle]
\setupTABLE [c] [even] [backgroundcolor=TableLight]
\setupTABLE [c] [1][style={\bfx\setupinterlinespace[line=.93\lH]}]
\setupTABLE [r] [first] [topframe=off,style={\bfx\setupinterlinespace[line=.93\lH]},
]
% \setupTABLE [r] [2][style={\bfx\setupinterlinespace[line=.93\lH]}]


\bTABLE

\bTABLEhead
\bTR [backgroundcolor=TableDark] \bTD [nc=3] Tilspændingsmomenter \eTD\eTR
% \bTR\bTD Position \eTD\bTD Type de vis \eTD\bTD Couple \eTD\eTR
\eTABLEhead

\bTABLEbody
\bTR\bTD Drivmotorer venstre/højre \eTD\bTD M12\:×\:35 8.8 \eTD\bTD 78 Nm \eTD\eTR
%% NOTE @Andrew: das sind Hydraulikmotoren
\bTR\bTD Arbejdspumpe \eTD\bTD M16\:×\:40 100 \eTD\bTD 330 Nm \eTD\eTR
\bTR\bTD Drivpumpe \eTD\bTD M12\:×\:40 100 \eTD\bTD 130 Nm \eTD\eTR
\bTR\bTD Fjederblad for/bag \eTD\bTD M16\:×\:90/160 8.8 \eTD\bTD 150 Nm \eTD\eTR
% \bTR\bTD Fixation du système oscillant \eTD\bTD M12\:×\:40 8.8 \eTD\bTD 78 Nm \eTD\eTR
\bTR\bTD Smudsbeholderens fastgørelse \eTD\bTD M10\:×\:30 Verbus Ripp 100 \eTD\bTD 80 Nm \eTD\eTR
\bTR\bTD Hjulmøtrikker \eTD\bTD M14\:×\:1,5 \eTD\bTD 180 Nm \eTD\eTR
\bTR\bTD Frontkostens fastgørelse \eTD\bTD M16\:×\:40 100 \eTD\bTD 180 Nm \eTD\eTR
\eTABLEbody
\eTABLE
\stop


\stopsection

% \page [yes]
%
%
% \startsection[title={Centralsmøring},
% reference={main:graissageCentral}]
%
%
% \subsection{Beskrivelse af styremodulet}
%
% \sdeux\ kan udstyres med\index{Centralsmøring} en centralsmøring (ekstraudstyr). Centralsmøringen forsyner med regelmæssige mellemrum alle køretøjets smøresteder med smøremiddel.
%
% \startfigtext [left] [vogel_affichage] {Displaymodul}
% {\externalfigure[vogel_base2][W50]}
% \blank
% \startLeg
% \item 7-cifret display: Værdier og driftsmæssig tilstand
% \item \LED: System på pause (standbyposition)
% \item \LED: Pumpe i brug
% \item \LED: Styring af systemet med cyklusafbryder
% \item \LED: Overvågning af systemet med trykafbryder
% \item \LED: Fejlmelding
% \item Piletaster:
% \startLeg [R]
% \item Aktiver display
% \item Vis værdier
% \item Rediger værdier
% \stopLeg
% \item Tast til skift af driftstype. Bekræftelse af værdier
% \item Udløsning af en mellemsmørecyklus
% \stopLeg
% \stopfigtext
%
% Centralsmøringen omfatter smøremiddelpumpen, den gennemsigtige smøremiddelbeholder på venstre side af chassiset og styremodulet i det centrale elektriske anlæg.
% % \blank
% \page [yes]
%
%
% \subsubsubject{Styremodulets display og taster}
%
% \start
%
% \setupTABLE [frame=off,style={\ssx\setupinterlinespace[line=.93\lH]},background=color,
% option=stretch,
% split=repeat]
%
% \setupTABLE [r] [each] [topframe=on,
% framecolor=white,
% % rulethickness=.8pt
% ]
%
% \setupTABLE [c] [odd] [backgroundcolor=TableMiddle]
% \setupTABLE [c] [even] [backgroundcolor=TableLight]
% \setupTABLE [c] [1][width=9mm,style={\bfx\setupinterlinespace[line=.93\lH]}]
% \setupTABLE [r] [first] [topframe=off,style={\bfx\setupinterlinespace[line=.93\lH]},
% ]
% % \setupTABLE [r] [2][style={\bfx\setupinterlinespace[line=.93\lH]}]
%
%
% \bTABLE
% \bTABLEhead
% % \bTR [backgroundcolor=TableDark]
% % \bTD [nc=4] Anzeige und Tasten des Steuermoduls \eTD\eTR
% \bTR\bTD Pos. \eTD
% \bTD \LED \eTD\bTD Displaymodus \eTD
% \bTD Programmeringsmodus \eTD\eTR
% \eTABLEhead
%
% \bTABLEbody
% \bTR\bTD 2 \eTD
% \bTD Driftstilstand {\em Pause}\hskip.5em\null \eTD
% \bTD Anlægget er på standby\hskip.5em\null \eTD % -tilstand
% \bTD Pausetiden kan ændres \eTD\eTR
% \bTR\bTD 3 \eTD
% \bTD Driftstilstand {\em Contact} \eTD
% \bTD Pumpen arbejder \eTD
% \bTD Arbejdstiden kan ændres \eTD\eTR
% \bTR\bTD 4 \eTD
% \bTD Systemkontrol {\em CS} \eTD
% \bTD Med den eksterne cyklusafbryder \eTD
% \bTD Kontrolmodus kan deaktiveres eller ændres \eTD\eTR
% \bTR\bTD 5 \eTD
% \bTD Systemkontrol {\em PS} \eTD
% \bTD Med den eksterne trykafbryder \eTD
% \bTD Kontrolmodus kan deaktiveres eller ændres \eTD\eTR
% \bTR\bTD 6 \eTD
% \bTD Fejl {\em Fault} \eTD
% \bTD [nc=2] Der er opstået en funktionsfejl. Årsagen vises i form af en fejlkode, efter der er trykket på tasten \textSymb{vogel_DK}. Funktionernes udførelse afbrydes. \eTD\eTR
% \bTR\bTD 7 \eTD
% \bTD Piletaster \textSymb{vogelTop} \textSymb{vogelBottom} \eTD
% \bTD [nc=2] \items[symbol=R]{Aktivering af displayet, søgning efter parametre (displaymodus), indstilling af den viste (I) værdi (programmeringsmodus)}
% \eTD\eTR
% \bTR\bTD 8 \eTD
% \bTD Tasten \textSymb{vogelSet} \eTD
% \bTD [nc=2] Skift mellem display- og programmeringsmodus, eller bekræftelse af de indtastede værdier. \eTD\eTR
% \bTR\bTD 9 \eTD
% \bTD Tasten \textSymb{vogel_DK} \eTD
% \bTD [nc=2] Hvis apparatet er på {\em Pause}, udløser en bekræftelse med tasten mellemsmørecyklussen. Fejlmeldingerne bekræftes og slettes. \eTD\eTR
% \eTABLEbody
% \eTABLE
% \stop
% \vfill
%
% \startfigtext [left] [vogel_touches]{Displaymodul}
% {\externalfigure[vogel_base][width=50mm]}
% \textDescrHead{Displaymodus} Tryk kort på en af piletasterne \textSymb{vogelTop} \textSymb{vogelBottom} for at aktivere det 7-cifrede display \textSymb{led_huit}. Ved at trykke på tasten \textSymb{vogelTop} igen, kan de forskellige parametre efterfulgt af deres værdi, vises. Modus {\em Display} kendes på de konstant lysende \LED\char"2060er (\in{2 til 6, fig.}[vogel_affichage]).
% \blank [medium]
% \textDescrHead{Programmeringsmodus} Tryk i mindst 2 sekunder på tasten \textSymb{vogelSet}for at skifte til modus {\em Programmering}: \LED\char"2060erne blinker. Tryk kort på tasten \textSymb{vogelSet} for at ændre\index{Centralsmøring+Programmering} visningen. Derefter kan den ønskede værdi ændres med tasterne \textSymb{vogelTop} \textSymb{vogelBottom}. Bekræft\index{Centralsmøring+Display} med tasten \textSymb{vogelSet}.
% \stopfigtext
%
% \page [yes]
%
%
% \subsection{Undermenuer i {\em displayet}}-modus
%
% \vskip -9pt
%
% \adaptlayout [height=+5mm]
%
% \startcolumns[balance=no]\stdfontsemicn
%
% \startSymVogel
% \externalfigure[vogel_tpa][width=26mm]
% \SymVogel
% \textDescrHead{Pausetid [h]} Tryk på tasten \textSymb{vogelTop} for at få vist de programmerede værdier.
% \stopSymVogel
%
% \startSymVogel
% \externalfigure[vogel_068][width=26mm]
% \SymVogel
% \textDescrHead{Resterende pausetid [h]} Resterende tid til næste smørecyklus.
% \stopSymVogel
%
% \startSymVogel
% \externalfigure[vogel_090][width=26mm]
% \SymVogel
% \textDescrHead{Total pausetid [h]} Total pausetid mellem to cyklusser.
% \stopSymVogel
%
% \startSymVogel
% \externalfigure[vogel_tco][width=26mm]
% \SymVogel
% \textDescrHead{Smøretid [min]} Tryk på \textSymb{vogelTop} for at få vist de programmerede værdier.
% \stopSymVogel
%
% \startSymVogel
% \externalfigure[vogel_tirets][width=26mm]
% \SymVogel
% \textDescrHead{Apparat på standby} Visning er ikke mulig, fordi apparatet er på standby (pause).
% \stopSymVogel
%
% \startSymVogel
% \externalfigure[vogel_026][width=26mm]
% \SymVogel
% \textDescrHead{Smøretid [min]} En smørings varighed.
% \stopSymVogel
%
% \startSymVogel
% \externalfigure[vogel_cop][width=26mm]
% \SymVogel
% \textDescrHead{Systemkontrol} Tryk på \textSymb{vogelTop} for at få vist de programmerede værdier.
% \stopSymVogel
%
% \startSymVogel
% \externalfigure[vogel_off][width=26mm]
% \SymVogel
% \textDescrHead{Kontrolmodus} \hfill PS: Trykafbryder;\crlf
% CS: Cyklusafbryder; OFF: deaktiveret.
% \stopSymVogel
%
% \startSymVogel
% \externalfigure[vogel_0h][width=26mm]
% \SymVogel
% \textDescrHead{Driftstimer} Tryk på \textSymb{vogelTop} for at få vist værdien i to trin.
% \stopSymVogel
%
% \startSymVogel
% \externalfigure[vogel_005][width=26mm]
% \SymVogel
% \textDescrHead{Del 1: 005} Driftstiden vises i to dele; til del 2 med tasten \textSymb{vogelTop}.
% \stopSymVogel
%
% \startSymVogel
% \externalfigure[vogel_338][width=26mm]
% \SymVogel
% \textDescrHead{Del 2: 33,8} Den 2. del af tallet er 33,8; giver i alt en driftstid på 533,8 h.
% \stopSymVogel
%
% \startSymVogel
% \externalfigure[vogel_fh][width=26mm]
% \SymVogel
% \textDescrHead{Fejltid} Tryk på\textSymb{vogelTop} for at få vist værdien i to trin.
% \stopSymVogel
%
% \startSymVogel
% \externalfigure[vogel_000][width=26mm]
% \SymVogel
% \textDescrHead{Del 1: 000} Fejltiden vises i to dele;\crlf til del 2 med \textSymb{vogelTop}.
% \stopSymVogel
%
% \startSymVogel
% \externalfigure[vogel_338][width=26mm]
% \SymVogel
% \textDescrHead{Del 2: 33,8} Den 2. del af tallet er 33,8; giver i alt en fejltid på 33,8 h.
% \stopSymVogel
%
% \stopcolumns
%
% \stopsection


\page [yes]


\setups [pagestyle:marginless]


\startsection[title={Smøreplan til manuel smøring},
reference={sec:grasing:plan}]

\starttextbackground [FC]
\startPictPar
\PMgeneric
\PictPar
De smøresteder, der er angivet i smøreplanen (\in{fig.}[fig:greasing:plan]) skal smøres regelmæssigt. Regelmæssig smøring er nødvendig for at sikre en konstant {\em forringelse af friktionen} og for at holde fugt og andre korroderende substanser ude.
\stopPictPar
\stoptextbackground

\blank [big]

\start

\setupcombinations [width=\textwidth]

\placefig[here][fig:greasing:plan]{Køretøjets smøreplan}
{\startcombination [3*1]
{\externalfigure[frame:steering:greasing]}{\ssx Knækstyring og pendulmekanisme}
{\externalfigure[frame:axles:greasing]}{\ssx Aksler}
{\externalfigure[frame:sucMouth:greasing]}{\ssx Sugemund}
\stopcombination}

\stop

\vfill

\startLeg [columns,three]
\item Knækstyringens løftecylindere\crlf {\sl 2 smørenipler pr. cylinder}
\item Knækstyringens lejer\crlf {\sl 2 smørenipler på venstre side}
\columnbreak
\item Pendulmekanismens leje\crlf {\sl 1 smørenippel foran tanken}
\item Bladfjedre\crlf {\sl 2 smørenipler pr. fjederblad}
\columnbreak
\item Sugemund\crlf {\sl 1 smørenippel pr. hjul}
\item Sugemund\crlf {\sl 1 smørenippel på trækarmen}
\stopLeg



\page [yes]


\setups [pagestyle:bigmargin]


\subsubject{Smøring af smudsbeholderen}

Smudsbeholderen har 6 smøresteder (2\:×\:3), der skal smøres ugentligt.

\blank [big]


\placefig[here][fig:greasing:container]{Beholderens løftemekanisme}
{\externalfigure[container:mechanisme]}


\placelegende [margin,none]{}
{{\sla Billedtekst:}

\startLeg
\item Beholderens venstre leje
\item Beholderens højre leje
\item Venstre hydraulikcylinder (øverst)
\item Venstre hydraulikcylinder (nederst)

{\em Som højre cylinder (punkt \in[greasing:point;hide]).}
\item Højre hydraulikcylinder (øverst)
\item [greasing:point;hide]Højre hydraulikcylinder (nederst)
\stopLeg}

\stopsection

\page [yes]



\startsection[title={Elektrisk system},
reference={sec:main:electric}]

\subsection{Centralt elektrisk anlæg i chassiset}

\startbuffer [fuses:preventive]
\starttextbackground [CB]
\startPictPar
\PHvoltage
\PictPar
\textDescrHead{Sikkerhedsforskrifter}
Overhold sikkerhedsforskrifterne i\index{Sikringer+Chassis}
denne\index{Relæ+Chassis} vejledning: Udskift altid sikringerne med sikringer med det foreskrevne amperetal: tag metalsmykker af, før De arbejder på det elektriske\index{Elektriske anlæg} anlæg (ringe, armbånd osv.).
\stopPictPar
\stoptextbackground
\stopbuffer


\subsubsubject{MIDI-sikringer}

\starttabulate[|l|r|p|]
\HL
\NC\md F 1 \NC 5 A \NC Stoplygte, \aW{+\:15} OBD \NC\NR
\NC\md F 2 \NC 5 A \NC \aW{+\:15} motorstyring \NC\NR
\NC\md F 3 \NC 7,5 A \NC \aW{+\:30} motorstyring og OBD \NC\NR
\NC\md F 4 \NC 20 A \NC Brændstofpumpe \NC\NR
\NC\md F 5 \NC 20 A \NC \aW{D\:+} generator, \aW{+\:15} relæ K 1 \NC\NR
\NC\md F 6 \NC 5 A \NC motorstyring \NC\NR
\NC\md F 7 \NC 10 A\NC Rensning af motorens udstødningsgas \NC\NR
\NC\md F 8 \NC 20 A \NC Motorelektronik (styring) \NC\NR
\NC\md F 9 \NC 15 A \NC Rensning af motorens udstødningsgas, forsyning, forvarmning \NC\NR
\NC\md F 10\NC 30 A \NC Motorstyring \NC\NR
\NC\md F 11\NC 5 A \NC Arbejdslygter bag \NC\NR
%% NOTE @Andrew: Singular
\HL
\stoptabulate

\placefig [margin] [fig:electric:power:rear] {Centralt elektrisk anlæg i chassiset}
{\externalfigure [electric:power:rear]
\noteF
\startKleg
\sym{K 1} Elektronisk motorstyreanordning
\sym{K 2} Brændstofpumpe
\sym{K 3} Frigivelse af startaggregatet
\sym{K 4} Stoplygter
\sym{K 5} {[}Reserve{]}
\sym{K 6} Arbejdslygter bag
%% NOTE @Andrew: Singular
\sym{K 7} Forvarmning
\stopKleg
}


\subsubsubject{MAXI-sikringer}

% \startcolumns [n=2]
\starttabulate[|l|r|p|]
\HL
\NC\md F 15 \NC 50 A \NC Det centrale elektriske anlægs hovedforsyning \NC\NR
\HL
\stoptabulate

\page [yes]

\setups[pagestyle:marginless]


\subsection{Centralt elektrisk anlæg i førerhuset}

\startcolumns[rule=on]

\placefig [bottom] [fig:fuse:cab] {Sikringer og relæ i førerhuset}
{\externalfigure [electric:power:front]}

%\vfill

\subsubsubject{Relæ}

\index{Sikringer+Førerhus}\index{Relæ+Førerhus}

\starttabulate[|lB|p|]
\NC K 2\NC Klimakompressor\NC\NR
\NC K 3\NC Klimakompressor\NC\NR
\NC K 4\NC Elektrisk vandpumpe\NC\NR
\NC K 5\NC Rotorblink\NC\NR
\NC K 10 \NC Blink-frekvensmåler\NC\NR
\NC K 11 \NC Nærlys\NC\NR
\NC K 12 \NC Fjernlys {[}Reserve{]} \NC\NR
\NC K 13 \NC Arbejdslygter\NC\NR
\NC K 14 \NC Vinduesvisker-intervalskift\NC\NR
\stoptabulate

\vskip -24pt

\placefig [bottom] [fig:fuse:access] {Adgangslem til det centrale elektriske anlæg}
{\externalfigure [electric:power:cabin]}

\stopcolumns

\page [yes]


\subsubsubject{MINI-sikringer}

\startcolumns[rule=on]
% \setuptabulate[frame=on]
%\placetable[here][tab:fuses:cab]{Fusibles dans la cabine}
%{\noteF
\starttabulate[|lB|r|p|]
\NC F 1 \NC 3 A \NC Parkeringslys venstre \NC\NR
\NC F 2 \NC 3 A \NC Parkeringslys højre \NC\NR
\NC F 3 \NC 7,5 A \NC Nærlys venstre \NC\NR
\NC F 4 \NC 7,5 A \NC Nærlys højre \NC\NR
\NC F 5 \NC 7,5 A \NC Fjernlys venstre {[}Reserve{]} \NC\NR
\NC F 6 \NC 7,5 A \NC Fjernlys højre {[}Reserve{]} \NC\NR
\NC F 7\NC 10 A \NC Arbejdslygter øverst \NC\NR
%% NOTE @Andrew: Plural
\NC F 8\NC 10 A \NC Arbejdslygter nederst (reserve) \NC\NR
%% NOTE @Andrew: Plural
\NC F 9 \NC — \NC {[}Ledig{]} \NC\NR
\NC F 10 \NC 10 A \NC Vinduesvisker \NC\NR
\NC F 11 \NC 5 A \NC Kontakt for lys og advarselsblink \NC\NR
\NC F 12 \NC 5 A \NC {[}Reserve{]} \NC\NR
\NC F 13 \NC 10 A \NC Opvarmning af sidespejle \NC\NR
\NC F 14 \NC 7,5 A \NC \aW{+\:15} Radio og kamera \NC\NR
\NC F 15 \NC 10 A \NC \aW{+\:30} Advarselsblink \NC\NR
\NC F 16 \NC 5 A \NC Lys ratstamme \NC\NR
\NC F 17 \NC 7,5 A \NC \aW{+\:30} Radio og indvendig belysning \NC\NR
\NC F 18 \NC — \NC {[}Ledig{]} \NC\NR
\NC F 19 \NC 20 A \NC \aW{+\:30} RC 12 forrest \NC\NR
\NC F 20 \NC 20 A \NC \aW{+\:30} RC 12 bagerst \NC\NR
\NC F 21 \NC 15 A \NC 12-V-stikkontakt \NC\NR
\NC F 22 \NC 5 A \NC Tændingsnøgle, multifunktionskonsol, Vpad \NC\NR
\NC F 23 \NC 5 A \NC Nødstop, midterkonsol, RC 12 forrest \NC\NR
\NC F 24 \NC 5 A \NC Nødstop, midterkonsol, RC 12 bagerst \NC\NR
\NC F 25 \NC 2 A \NC \aW{+\:15} RC 12 forrest \NC\NR
\NC F 26 \NC 2 A \NC \aW{+\:15} RC 12 bagerst \NC\NR
\NC F 27 \NC 15 A \NC Varmeblæser \NC\NR
\NC F 28 \NC 10 A \NC Klimakompressor, centralsmøring \NC\NR
\NC F 29 \NC 15 A \NC Klimakondensator \NC\NR
\NC F 30 \NC 5 A \NC Termostat klimaanlæg \NC\NR
\NC F 31 \NC 5 A \NC \aW{+\:15} Multifunktionskonsol/Vpad \NC\NR
\NC F 32 \NC 15 A \NC Elektrisk vandpumpe, rotorblink \NC\NR
\NC F 33 \NC — \NC {[}Ledig{]} \NC\NR
\NC F 34 \NC — \NC {[}Ledig{]} \NC\NR
\NC F 35 \NC — \NC {[}Ledig{]} \NC\NR
\NC F 36 \NC — \NC {[}Ledig{]} \NC\NR
\stoptabulate
\stopcolumns

\page [yes]

\setups [pagestyle:bigmargin]


\subsection[sec:lighting]{Lys og signaludstyr}


\placefig [here] [fig:lighting] {Køretøjets lys og signaludstyr}
{\externalfigure [vhc:electric:lighting]}

\placelegende [margin,none]{}{%
\vskip 30pt
{\sla Billedtekst:}
\startLongleg
\item Parkeringslys\hfill 12 V – 5 W
\item Nærlys\hfill H7 12 V – 55 W
\item Blinklys\hfill orange 12 V – 21 W
\item {\stdfontsemicn Arbejdslygter}\hfill G886 12 V – 55 W
\item Retningsviser\hfill 12 V – 21 W
\item Bag-/stoplygter\hfill 12 V – 5/21 W
\item Baklygter\hfill 12 V – 21 W
\item {[}Ledig{]}
\item Nummerpladelys\hfill 12 V – 5 W
\item Rotorblink\hfill H1 12 V – 55 W
\stopLongleg}

\subsubsubject{Indstilling af lygterne}

\placefig [margin] [fig:lighting:adjustment] {Lysstråle ved 5 m}
{\externalfigure [vhc:lighting:adjustment]
\startitemize
\sym{H\low{1}} Glødetrådens højde: 100 cm
\sym{H\low{2}} Korrektion ved 2\hairspace\%: 10 cm
\stopitemize}

{\md Forudsætninger:} Fersk-/genbrugsvandbeholderen er fuld, føreren sidder ved rattet.

Lygternes retning forudindstilles på fabrikken. Lysstrålens højde og hældning kan indstilles ved at dreje plastholderen.

Hvis det i forbindelse med en kontrol konstateres, at indstillingen skal ændres, skal låseskruen løsnes og hældningen korrigeres således, at lovens forskrifter opfyldes (se \in{fig.}[fig:lighting:adjustment]). Spænd låseskruen fast igen.

\page [yes]
\setups [pagestyle:marginless]


\subsection[sec:battcheck]{Batteri}

\subsubsection{Sikkerhedsforskrifter}

\startSymList
\PPfire
\SymList
\textDescrHead{Eksplosionsfare}
Når\index{Batteri+Sikkerhedsanvisninger}\index{Fare+Eksplosion} batterierne oplades, dannes der eksplosivt\index{Knaldgas} knaldgas. Oplad kun batterierne i rum med god ventilation! Undgå gnistdannelse!
Brand, åben ild og rygning forbudt i nærheden af batteriet.
\stopSymList

\startSymList
\PHvoltage
\SymList
\textDescrHead{Risiko for kortslutning}
Hvis\index{Batteri+Vedligeholdelse} det tilsluttede batteris pluspol rører ved køretøjets dele, er der\index{Fare+Brand}\index{Fare+Eksplosion} risiko for kortslutning.
Dette kan resultere i, at den gasblanding, der siver ud af batteriet, kan eksplodere og De og andre kan komme alvorligt til skade.

\startitemize
\item Læg ikke metalgenstande eller værktøj på batteriet.
\item Når batteriet frakobles, skal minuspolen altid tages af først, og derefter pluspolen.
\item Når batteriet tilkobles, skal pluspolen altid anbringes først, og derefter minuspolen.
\item Når motoren er startet, må batteriets tilslutningsklemmer ikke løsnes eller tages af.
\stopitemize
\stopSymList


\startSymList
\PHcorrosive
\SymList
\textDescrHead{Risiko for ætsning}
Bær\index{Fare+Ætsning} beskyttelsesbriller og syrefaste beskyttelseshandsker. Batterivæsken består af ca. 27\percent svovlsyre (H\low{2}SO\low{4}) og kan derfor medføre ætsninger.
Neutraliser\index{Batteri+Fare}\index{Batteri+-væske} batterivæske, der kommer på huden med en opløsning af tvekulsurt natron og skyl efter med rent vand. Hvis der kommer batterivæske i øjnene, skal øjnene skylles med rigelige mængder koldt vand og søg omgående læge.
\stopSymList

\startSymList
\startcombination[1*2]
{\PHcorrosive}{}
{\PHfire}{}
\stopcombination
\SymList
\textDescrHead{Opbevaring af batterier}
Batterier\index{Batteri+opbevaring} skal altid opbevares opretstående. I modsat fald kan batterivæsken sive ud og medføre ætsninger eller – ved reaktion med andre substanser – medføre brand. \par\null\par\null
\stopSymList

\testpage [16]

\starttextbackground [FC]
\setupparagraphs [PictPar][1][width=2.4em,inner=\hfill]

\startPictPar
\PMproteyes
\PictPar
\textDescrHead{Beskyttelsesbriller}
Når\index{Fare+Øjenskade} vand og syre blandes, kan væsken sprøjte og komme i øjnene. Skyl omgående øjnene med rent vand, hvis syrestænk kommer i øjnene og søg omgående læge!
\stopPictPar
\blank [small]

\startPictPar
\PMrtfm
\PictPar
\textDescrHead{Dokumentation}
Under håndtering af batterier er det meget vigtigt, at sikkerhedsanvisningerne, beskyttelsesforanstaltningerne og fremgangsmåderne i denne betjeningsvejledning overholdes.
\stopPictPar
\blank [small]

\startPictPar
\PStrash
\PictPar
\textDescrHead{Miljøbeskyttelse}
Batterier\index{Miljøbeskyttelse} indeholder skadelige stoffer. Bortskaf aldrig gamle batterier
sammen med husholdningsaffald. Bortskaf batterier miljøvenligt. Aflever dem på et autoriseret værksted, eller et sted, der indsamler gamle batterier.

Fyldte batterier skal altid transporteres og opbevares opretstående. Under transporten skal batterierne sikres mod at vælte. Der kan sive batterivæske ud af proppernes ventilationsåbninger og ud i omgivelserne.
\stopPictPar
\stoptextbackground

\page [yes]

\setups[pagestyle:normal]


\subsubsection{Praktiske råd}

For at sikre, at batteriet har en maksimal levetid, skal det så vidt muligt altid være fuldt opladet.

En\index{Batteri+Levetid} vedligeholdelsesopladning af batteriet, når køretøjet står stille i længere tid, forlænger ikke kun batteriets levetid, men sikrer også, at køretøjet altid starter.

\placefig[margin][fig:batterycompartment]{\select{caption}{Batterirum (servicelem)}{Batterirum}}
{\externalfigure[batt:compartment]}


\subsubsection{Vedligeholdelse}

Batteriet i \sdeux\ er en {\em vedligeholdelsesfri} blyakkumulator. Udover opladning og rengøring, kræver batteriet ingen vedligeholdelse.

\startitemize
\item Sørg for, at batteriets poler altid er rene og tørre. Smør polerne med en smule syreafvisende fedt.
\item Batterier, der\index{Batteri+opladning} har en hvilespænding
på\index{Batteri+Hvilespænding} mindre end 12,4 V, skal oplades.
\stopitemize

\placefig[margin][fig:bclean]{Rengøring af polerne}
{\externalfigure[batt:clean]
\noteF
Anvend\index{Batteri+rengøring}\index{Rengøring+Batterier} varmt vand til at fjerne det hvide pulver, der opstår på grund af korrosion. Frakobl batterikablet, hvis en pol er rusten, og rengør polen med en stålbørste. Påfør et tyndt lag fedt på polerne.}


\subsubsection[sec:battery:switch]{Brug af batteriskillekontakten}

{\sl Det anbefales ikke at aktivere batteriskillekontakten regelmæssigt (for eksempel dagligt)!}

\startSteps
\item Slå\index{Batteriskillekontakt} tændingen fra og vent derefter ca. 1 minut.
\item Åbn batterirummet (\inF[fig:batterycompartment]).
\item Tryk på batteriskillekontaktens røde knap for at afbryde strømkredsen.
\item Drej batteriskillekontakten ¼ omdrejning med uret for at lukke strømkredsen igen.
\stopSteps

% \starttextbackground [FCnb]
% \startPictPar
% \PMgeneric
% \PictPar
% Der Batterietrennschalter ist dafür vorgesehen, die Batterie für bestimmte Wartungs- und Reparaturarbeiten vorübergehend vom Stromkreis zu trennen. Es ist nicht empfehlenswert, den Batterietrennschalter regelmäßig (\eG\ täglich) zu betätigen: Bestimmte elektronische Komponenten sollten ständig unter Spannung stehen, ansonsten kann es zu Fehlermeldungen im Fehlerspeicher kommen.
% \stopPictPar
% \stoptextbackground

\stopsection

\page [yes]


\setups[pagestyle:marginless]

\section[sec:cleaning]{Rengøring af køretøjet}

Skyl\startregister[index][vhc:lavage]{Vedligeholdelse+Rengøring} groft mudder og snavs af karosseriet med rigelige mængder vand, før den egentlige rengøring. Vask ikke kun sidefladerne, men også skærmkasserne og undersiden af køretøjet.

Især om vinteren skal køretøjet vaskes grundigt for at fjerne stærkt korroderende\index{Korrosion+Forebyggelse} rester af vejsalt.

\starttextbackground [FC]
\startPictPar
\PHgeneric
\PictPar
\textDescrHead{Undgå skader på grund af vand}
Rengør aldrig køretøjet med {\em vandkanoner} (\eG\ brandvæsen) eller {\em kolde rengøringsmidler på kulbrintebasis.} Overhold forskrifterne nedenfor, hvis De arbejder med en højtryks-damp-renser.
\stopPictPar
\blank[small]

\startPictPar
\pTwo[monde]
\PictPar
\textDescrHead{Miljøbeskyttelse}
Rengøring af et køretøj kan medføre alvorlige miljøbelastninger.
Rengør kun køretøjet et\index{Miljøbeskyttelse} sted, der er udstyret med en olieudskiller. Overhold de gældende regler om miljøbeskyttelse.
\stopPictPar
\blank[small]

\startPictPar
\PMwarranty
\PictPar
\textDescrHead{Rengør fagligt korrekt!}
For skader, der kan opstå på grund af, at rengøringsforskrifterne ikke overholdes, kan der ikke gøres erstatningsansvar eller garantikrav gældende over for \BosFull{Boschung}.
\stopPictPar
\stoptextbackground


\subsection{Højtryksrensning}

Til højtryksrensning\index{Rengøring+Højtryk} af køretøjet kan der anvendes en almindelig højtryksrenser.

Rengøring med en højtryksrenser kræver, at følgende punkter overholdes:

\startitemize
\item Arbejdstryk maks. 50\,bar
\item Spaltedyse med en sprøjtevinkel på 25°
\item Sprøjteafstand mindst 80\,cm
\item Vandtemperatur maks. 40\,°C
\item Se afsnittet \about[reiMi], \atpage[reiMi].
\stopitemize

Hvis disse\index{Lak+Skader} forskrifter ikke overholdes, kan dette medføre lakskader og skader på korrosionsbeskyttelsen\index{Skader+Lak}.

Læs og overhold også betjeningsvejledningen og sikkerhedsforskrifterne til højtryksrenseren.

\starttextbackground[FC]
\startPictPar\PPspray\PictPar
Under rengøring med en højtryksrenser, er der steder, hvor der kan trænge vand ind og forårsage skader. Ret derfor aldrig vandstrålen mod følsomme dele og redskaber:
\stopPictPar

\startitemize
\item Sensorer, elektriske forbindelser og tilslutninger
\item Startaggregat, generator, indsprøjtningssystem
\item Magnetventiler
\item Ventilationsåbninger
\item Endnu ikke afkølede mekaniske komponenter
\item Mærkater med henvisninger, advarsler og farer
\item Elektroniske styreanordninger
\stopitemize

\textDescrHead{Motorvask}
Vandindtrængning i indsugnings-, ventilations- og udluftningsåbninger
skal under alle omstændigheder undgås. Ret ikke højtryksrenserens stråle direkte mod elektriske komponenter og ledninger. Ret ikke strålen mod indsprøjtningssystemet! Konserver motoren efter motorvask. Beskyt remmen mod konserveringsproduktet.
\stoptextbackground

\starttextbackground [FC]
\setupparagraphs [PictPar][1][width=6em,inner=\hfill]
\startPictPar
\framed[frame=off,offset=none]{\PMproteyes\PMprotears}
\PictPar
\textDescrHead{Resterende vand}
Under rengøringen samler der sig vand bestemte steder på køretøjet (\eG\ i motorblokkens eller drevets fordybninger), som skal fjernes med trykluft. Husk at bære personlige værnemidler, når der arbejdes med trykluft, og at anlægget skal opfylde de gældende sikkerhedsforskrifter (multidyse).
\stopPictPar
\stoptextbackground


\subsubsection[reiMi]{Egnede rengøringsmidler}

Anvend\index{Rengøringsmidler} udelukkende rengøringsmidler med følgende egenskaber:

\startitemize
\item Ingen skureeffekt
\item PH-værdi på 6–7
\item Uden opløsningsmidler
\stopitemize

Til at fjerne hårdnakkede pletter, kan De med forsigtighed anvende rensebenzin eller spiritus på små lakflader, men aldrig andre opløsningsmidler.
Fjern rester af opløsningsmidler fra lakken. Rengøring af plastdele med benzin kan medføre ridser eller misfarvninger!

Rengør flader med\index{Rengøring+Mærkat} mærkater med advarsler eller henvisninger med rent vand og en blød svamp.

Pas på, at der ikke kommer vand ind i elektriske komponenter: Anvend ikke en bilbørste til rengøring af blinklys- og lygtehuset, men en blød klud eller svamp.

\starttextbackground [CB]
\startPictPar
\GHSgeneric\par
\GHSfire
\PictPar
\textDescrHead{Fare på grund af kemikalier}
Rengøringsmidler kan indebære sundheds- og sikkerhedsrisici (let antændelige stoffer). Overhold de gældende sikkerhedsforskrifter for de anvendte rengøringsmidler og de anvendte midlers sikkerhedsdatablade.
\stopPictPar
\stoptextbackground

\stopregister[index][vhc:lavage]


\page [yes]


\setups [pagestyle:bigmargin]

\startsection [title={Indstilling af sugemunden},
reference={sec:main:suctionMouth}]


Den optimale\index{Sugemunden+Indstilling} afstand mellem vejens overflade og gummilæben er 10\,mm.
Brug de tre indstillingslærer, De finder i værktøjskassen (førerhus, førersiden) til at kontrollere eller indstille afstanden med.


\placefig [margin] [fig:suctionMouth] {Indstilling af sugemunden}
{\externalfigure [suctionMouth:adjust]}

\placeNote[][service_picto]{}{%
\noteF
\starttextrule{\PHasphyxie\enskip Risiko for forgiftning og kvælning \enskip}
{\md Bemærk:} Køretøjets motor skal være i gang under indstillingsarbejdet for at kunne holde sugemunden i flydeposition. For at udelukke risikoen for forgiftning eller kvælning er det absolut nødvendigt at anvende et udsugningsanlæg til udstødningsgas, eller at udføre arbejdet et sted med god ventilation.
\stoptextrule}

\startSteps
\item Parker køretøjet et sted med god ventilation på en vandret og jævn overflade.
\item Aktiver\index{Udsugning} \aW{Arbejds}modus (tryk på knappen uden på gearvælgeren).

Lad motoren køre i tomgang. (Tryk på tasten~\textSymb{joy_key_engine_decrease} på multifunktionskonsollen for at reducere motorens omdrejningstal.)
\item Træk håndbremsen og bloker hvert baghjul med en kile.
\item Tryk på tasten~\textSymb{joy_key_suction} for at sænke sugemunden.
\item Placer de tre indstillingslærer~\LAa\ under sugemundens gummilæbe, som vist i illustrationen.
\item [sucMouth:adjust]Løsn fastspændings-~\Lone\ og stilleskruerne~\Ltwo\ på hvert hjul; de fire hjul sænker sig ned på jorden.
\item Bloker skruerne~\Lone\ og~\Ltwo og fjern derefter de tre indstillingslærer.
\item Hæv/sænk sugemunden og kontroller indstillingen med de tre indstillingslærer. Hvis indstillingen stadig ikke er helt korrekt, skal indstillingsproceduren gentages fra punkt~\in[sucMouth:adjust].

\stopSteps


\stopsection
\stopchapter
\stopcomponent


