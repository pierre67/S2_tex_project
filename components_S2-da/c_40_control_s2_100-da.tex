\startcomponent c_40_control_s2_100-da
\product prd_ba_s2_100-da


\startchapter [title={Køretøjets betjeningselementer},
reference={chap:ctrl}]

\setups[pagestyle:marginless]

\placefig[here][fig:ctrl:cab:front]{Betjeningselementer}
{\externalfigure[ctrl:cab:front]}

\startcolumns [n=3]
\startLongleg
\item Ratstamme (\in{§}[sec:steeringColumn])
\item Indstilling ratstamme
\item Køre- og bremsepedal
\item On-board computer \Vpad SN (\inP[sec:vpad])
\item Loftskonsol (\inP[sec:ctrl:aux])
\stopLongleg


\subsubsubject{Ekstraudstyr}

\startLongleg [continue]
\item Bakskærm
\item Radio/MP3
\stopLongleg
\stopcolumns

\startsection [title={Ratstamme},
reference={sec:steeringColumn}]

\subsection{Indstilling af ratstammen}

\textDescrHead{Rattets hældning} Tryk på pedalen \Ltwo og indstil samtidigt ratstammens hældning. Slip pedalen for at låse ratstammens mekanisme igen.

\page[yes]
\setups [pagestyle:normal]


\subsection{Lys og signaludstyr}

\placefig [margin] [fig:column:left] {Multifunktionsgreb og drejekontakt}
{\externalfigure[ctrl:column:left]}

\placefig [margin] [fig:column:right] {Gearvælger}
{\externalfigure[ctrl:column:right]}


\subsubsubject{Drejekontakt}

\startitemize[width=1.7em]
\sym{\textSymb{com_lowlight}} Nærlys (drej \TorqueR).
\startitemize
\sym{1} Parkeringslys
\sym{2} Nærlys
\stopitemize
\stopitemize


\subsubsubject{Multifunktionsgreb}

\startitemize[width=1.7em]
\sym{\textSymb{com_lowlight}} {[}Ikke aktiveret{]}
\sym{\textSymb{com_light}} Overhalingsblink (tryk grebet kortvarigt opad)
\sym{\textSymb{com_blink}} Retningsviserblinklys (greb fremad/bagud)
\sym{\textSymb{com_claxonArrow}} Horn (tryk på knappen yderst på grebet)
\sym{\textSymb{com_wipper}} Vinduesvisker
\startitemize
\sym{J} Intervalskift
\sym{O} Fra
\sym{I} 1. hastighedstrin
\sym{II} 2. hastighedstrin
\stopitemize
\sym{\textSymb{com_washerArrow}} Sprinkleranlæg (tryk på kransen for enden af grebet).
\stopitemize


\subsubsubject{Gearvælger}

Gearvælgerens funktioner er beskrevet detaljeret i kapitlet \about[chap:using], fra \atpage[sec:using:start].

\stopsection

\page [yes]


\startsection [title={Yderligere funktioner},
reference={sec:ctrl:add}]


\subsection[sec:ctrl:aux]{Loftskonsol}

{\sl Loftskonsollen\index{Loftskonsol} findes forrest i førerhusets loft i førersiden.}
\placefig [margin] [fig:console:aux] {Loftskonsol}
{\externalfigure[ctrl:console:aux]}


\placefig [margin] [fig:console:climat] {Varme og klimaanlæg}
{\externalfigure[ctrl:console:climat]}


\startitemize [unpacked][width=1.7em]
\sym{\textBigSymb{temoin_retrochauffant}} Opvarmning af sidespejle
\sym{\textBigSymb{temoin_hazard}} Advarselsblink
\sym{\textBigSymb{temoin_eclairage_L}} Arbejdslygter
\stopitemize


\subsubsubject{Ekstraudstyr}

\startLeg [unpacked][width=1.7em]
\sym{\textBigSymb{temoin_buse}} Højtryksvandpumpe til vandpistol \crlf {\sl se \atpage[sec:using:water:spray]}
\sym{\textBigSymb{temoin_aspiration_manuelle}} Turbine for håndsugeslange \crlf {\sl se \atpage[sec:using:suction:hose]}
\stopLeg


\subsection[sec:ctrl:climat]{Varme og klimaanlæg}

{\sl Denne konsol\index{Varmekonsol} findes på førerhusets bagvæg mellem sæderne.}

\startitemize [unpacked][width=23mm]
\sym{\bf 0\quad I\quad II\quad III} Blæser-drejekontakt
\sym{\externalfigure[tirette_chauffage][height=1em]} Temperatur-skyderegulator
\stopitemize


\subsubsubject{Ekstraudstyr}

\startitemize [unpacked][width=1.7em]
\sym{\textBigSymb{temoin_climat_bk}} Klimaanlæg
\stopitemize

\page [yes]

\setups [pagestyle:bigmargin]


\subsection[sec:ctrl:central]{Midterkonsol}

{\sl Midterkonsollen\index{Midterkonsol} findes mellem sæderne.}

\placefig [margin] [fig:console:central] {Midterkonsol}
{\externalfigure[ctrl:console:central]}


\subsubsubject{Befugtning af kostene}

\startLeg [unpacked][width=1.7em]
\sym{\textBigSymb{temoin_busebalais}} Lavtryksvandpumpe\index{Vandpumpe} til kostenes befugtningssystem\index{Vandpumpe+Befugtning}. (Position 1: \aW{Automatisk}; position 2: \aW{Permanent})
\stopLeg


\subsubsubject{Tipning af smudsbeholderen}

\setupinmargin[right][style=normal]
\inright{%
\startitemize
\sym{\textSymb{mand_readtheoperatingmanual}} Overhold anvisningerne om brug af håndbremsen på \atpage[sec:using:stop].
\stopitemize}

\startLeg [unpacked][width=1.7em]
\sym{\textBigSymb{temoin_kipp2}} Tipning af smudsbeholderen.
For\index{Smudsbeholder+Tipning} at kunne tippe smudsbeholderen, skal håndbremsen være trukket og gearvælgeren skal stå på Neutral.
\stopLeg


\subsubsubject{Nødstop}

\starttextbackground [FC]
\startPictPar
\externalfigure[Emergency_Stop][Pict]
\PictPar
I en nødsituation\index{Nødstop} kan De slukke alle suge- og fejeredskaber ved at trykke på nødstop-kontakten.
\stopPictPar
\stoptextbackground


\subsection[sec:foot:switch]{Fodkontakt}

\placefig [margin] [fig:foot:switch] {Fodkontakt}
{\vskip 60pt
\externalfigure[work:foot:switch]}

Ved hjælp af\index{Fodkontakt} denne kontakt ved foden af ratstammen (\inF[fig:foot:switch]) kan De hurtigt og nemt sænke kostene, når det er nødvendigt (\eG\ på toppen af en skråning, opkørsel på fortov).

\stopsection
\page[yes]
\setups [pagestyle:marginless]


\startsection[title={Multifunktionskonsol},
reference={ctrl:console:middle}]

\startlocalfootnotes

\startfigtext[left]{Multifunktionskonsol}
{\externalfigure[overview:joy:large]}


\subsubsubject{Joysticks}

\textDescrHead{Uden frontkost (eller frontkost deaktiveret):}
Det to joysticks styrer hver især en kost uafhængigt af hinanden: Løft/sænk~(\textSymb{joystick_aa}) eller venstre/højre~(\textSymb{joystick_gd}). Venstre joystick styrer venstre kost og højre joystick højre kost.\footnote{For at kunne ændre sidekostenes position på et køretøj, der er udstyret med en frontkost (ekstraudstyr), skal frontkosten deaktiveres (tasten~\textSymb{joy_key_frontbrush_act}).}

\textDescrHead{Med frontkost:}
Med venstre joystick kan De løfte/sænke frontkosten (\textSymb{joystick_aa}) og bevæge den mod venstre/højre (\textSymb{joystick_gd}). Med højre joystick hælder De kosten på langs~(\textSymb{joystick_aa}) og på tværs~(\textSymb{joystick_gd}).

\placelocalfootnotes %[height=\textheight]
\stopfigtext
\stoplocalfootnotes
\vfill


\subsubsubject{Taster i siden}

\startcolumns

\startPictList
\VPcltr
\PictList
Fartpilot: Øgning af den indstillede hastighed
\stopPictList\vskip -3pt

\startPictList
\VPclbr
\PictList
Fartpilot: Reducering af den indstillede hastighed
\stopPictList\vskip -3pt

\startPictList
\VPcrtr
\PictList
Løft sugemund
\stopPictList

\startPictList
\VPcrbr
\PictList
Sænk sugemund
\stopPictList\vskip -3pt

\startPictList
\VPcrtf
\PictList
Åbn lem for groft snavs (foran på sugemunden)
\stopPictList\vskip -3pt

\startPictList
\VPcrbf
\PictList
Luk lem for groft snavs
\stopPictList

\stopcolumns


\subsubsubject{Symboltaster}

\startcolumns

\startSymVpad
\externalfigure[joy:stop]
\SymVpad
\textDescrHead{Stop} Stands aktiveret redskab:

1\:× tryk: Deaktiver 3.\,kost\crlf
2\:× tryk: Deaktiver alle
\stopSymVpad

\startSymVpad
\externalfigure[joy:tempomat]
\SymVpad
\textDescrHead{Fartpilot} Indstilling og aktivering af fartpiloten til den aktuelle hastighed. Tryk på tasten~\textSymb{joy:tempomat} igen for at deaktivere, eller brems. Sæt farten op/ned med tasterne i siden.
\stopSymVpad

\startSymVpad
\externalfigure[joy:ftbrs:minus]
\SymVpad
\textDescrHead{Kosthastighed} Reducering af sidekostenes eller frontkostens rotationshastighed.
\stopSymVpad

\startSymVpad
\externalfigure[joy:ftbrs:plus]
\SymVpad
\textDescrHead{Kosthastighed} Øgning af sidekostenes eller frontkostens rotationshastighed.
\stopSymVpad

\startSymVpad
\externalfigure[joy:eng:minus]
\SymVpad
\textDescrHead{Motoromdrejningstal} Reducering af dieselmotorens omdrejningstal.
\stopSymVpad

\startSymVpad
\externalfigure[joy:eng:plus]
\SymVpad
\textDescrHead{Motoromdrejningstal} Øgning af dieselmotorens omdrejningstal.
\stopSymVpad
\columnbreak

\startSymVpad
\externalfigure[joy:suc]
\SymVpad
\textDescrHead{Sugning} Aktivering af sugesystemet: Sugemunden sænkes,
turbinen og genbrugs-vandpumpen startes.\note [recyclingwaterpump] \crlf
Tryk på stoptasten~\textSymb{joy:stop} for at deaktivere systemet.
\stopSymVpad

\startSymVpad
\externalfigure[joy:sucbrs]
\SymVpad
\textDescrHead{Fejning/sugning}Aktivering af suge-/fejesystemet: Sugemunden sænkes, sidekostene sænkes og positioneres, turbinen, kostene og genbrugs-vandpumpen startes.\note [recyclingwaterpump] \crlf
Tryk på stoptasten~\textSymb{joy:stop} for at deaktivere systemet.
\stopSymVpad

\footnotetext[recyclingwaterpump]{Ferskvandspumpen startes også, hvis kontakten~\textBigSymb{temoin_busebalais} står på \aW{Automatisk} (se \in [sec:ctrl:central] på \atpage [sec:ctrl:central]).}
\startSymVpad
\externalfigure[joy:ftbrs:act]
\SymVpad
\textDescrHead{Frontkost aktiveret} Aktivering/deaktivering af frontkosten.
%% NOTE @Andrew: Singular
\stopSymVpad

\startSymVpad
\externalfigure[joy:ftbrs:right]
\SymVpad
\textDescrHead{Frontkost venstre} Drejeretning til arbejde med frontkosten på venstre side
(Drejeretning: Med uret).
\stopSymVpad

\startSymVpad
\externalfigure[joy:ftbrs:left]
\SymVpad
\textDescrHead{Frontkost højre} Drejeretning til arbejde med frontkosten på højre side
(Drejeretning: Mod uret).
\stopSymVpad

\stopcolumns

\stopsection

\stopchapter

\stopcomponent










