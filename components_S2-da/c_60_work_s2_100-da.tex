\startcomponent c_60_work_s2_100-da
\product prd_ba_s2_100-da


\startchapter [title={S2 i hverdagen},
reference={chap:using}]

\setups [pagestyle:marginless]


% \placefig[margin][fig:ignition:key]{Clé de contact}
% {\externalfigure [work:ignition:key]}
\startregister[index][chap:using]{Ibrugtagning}

\startsection [title={Ibrugtagning},
reference={sec:using:start}]


\startSteps
\item Kontroller, at de regelmæssige kontroller og vedligeholdelser er udført forskriftsmæssigt.
\item Start motoren med tændingsnøglen: Slå tændingen til, fortsæt med at dreje nøglen med uret og stop, indtil motoren starter (kun muligt, hvis gearvælgeren er på Neutral).
\stopSteps

\start
\setupcombinations [width=\textwidth]

\placefig[here][fig:select:drive]{Gearvælger}
{\startcombination [2*1]
{\externalfigure [work:select:fDrive]}{Gearvælger i stillingen \aW{Fremadkørsel}}
{\externalfigure [work:select:rDrive]}{Gearvælger i stillingen \aW{Bakgear}}
\stopcombination}
\stop


\startSteps [continue]
\item Drej gearvælgerens kontakt for at sætte i et gear i \aW{Køre}modus:
\startitemize [R]
\item Første gear
\item Andet gear (automatisk drift; starter automatisk i første gear) \stopitemize, eller tryk på knappen yderst på gearvælgeren for at aktivere/deaktivere \aW{Arbejds}modus.
\stopSteps

\startbuffer [work:config]
\starttextbackground [FC]
\startPictPar
\PMrtfm
\PictPar
I Arbejdsmodus er kun første gear til rådighed og motoren drejer med 1300\,min\high{\textminus 1}.

De kan styre motorens omdrejningstal med tasterne~\textSymb{joy_key_engine_increase} og~\textSymb{joy_key_engine_decrease} på multifunktionskonsollen.
\stopPictPar
\stoptextbackground
\stopbuffer

\getbuffer [work:config]

\startSteps [continue]
\item Tryk gearvælgeren opad og fremad (fremadkørsel), eller opad og bagud (bakgear). Se billederne ovenfor.
\item Løsn håndbremsen, før De starter med at køre.
\stopSteps

\starttextbackground [FC]
\startPictPar
\PMrtfm
\PictPar
{\md Løsn håndbremsen helt!} Håndbremsestangens position overvåges af en elektronisk sensor: Hvis håndbremsen ikke er helt løsnet, er køretøjets hastighed begrænset til 5\,km/h.
\stopPictPar
\stoptextbackground

\startSteps [continue]
\item Tryk langsomt på speederen for at sætte køretøjet i bevægelse.
\stopSteps


%% NOTE: New text [2014-04-29]:
\subsection [sSec:suctionClap] {Sugekanallem}

Sugesystemet genererer en luftstrøm, enten fra sugemunden eller fra håndsugeslangen (ekstraudstyr) til smudsbeholderen.

En lem, der skal åbnes og lukkes med hånden (\inF[fig:suctionClap], \atpage[fig:suctionClap]) giver mulighed for at skifte luftstrømmen mellem sugemunden og håndsugeslangen.

\placefig [here] [fig:suctionClap] {Sugekanallem}
{\startcombination [2*1]
{\externalfigure [work:suctionClap:open]}{Sugekanal åbnet}
{\externalfigure [work:suctionClap:closed]}{Sugekanal lukket}
\stopcombination}

I normal drift~– arbejde med sugemunden~– skal sugekanalen være åbnet (skiftestangen peger opad).

For at kunne anvende håndsugeslangen, skal sugekanalen være lukket (skiftestangen peger nedad). På denne måde ledes luftstrømmen gennem håndsugeslangen.
%% End new text

\stopsection


\startsection [title={Udtagning af drift},
reference={sec:using:stop}]

\index{Udtagning af drift}

\startSteps
\item Træk håndbremsen (stangen mellem sæderne) og sæt gearvælgeren på \aW{Neutral}.
\item Udfør det påkrævede kontrolarbejde~– daglige og evt. ugentlige kontroller~– som beskrevet på \atpage[table:scheduledaily].
\stopSteps

\getbuffer [prescription:handbrake]

\stopsection

\page [yes]

\startsection [title={Fejning og sugning},
reference={sec:using:work}]

\startSteps
\item Gennemfør\index{Fejning} ibrugtagningen af køretøjet, som beskrevet i \in{§}[sec:using:start], \atpage[sec:using:start].
\item Aktiver\index{Sugning} \aW{Arbejds}modus (knap på gearvælgeren).
\stopSteps

% \getbuffer [work:config]
%% NOTE: outcommented by PB

\startSteps [continue]
\item Tryk på tasten~\textSymb{joy_key_suction_brush} for at starte turbinen og kostene.

{\md Variant:} {\lt Tryk på tasten~\textSymb{joy_key_suction} for kun at arbejde med sugemunden.}

\item Indstil kostenes omdrejningshastighed ved hjælp af tasterne~\textSymb{joy_key_frontbrush_increase}\textSymb{joy_key_frontbrush_decrease} på multifunktionskonsollen.

\item Bring kostene i position med det pågældende joystick således, at de har en optimal arbejdsbredde.
\stopSteps

\vfill

\start
\setupcombinations [width=\textwidth]

\placefig[here][fig:brush:position]{Anbringelse af kostene}
{\startcombination [2*1]
{\externalfigure [work:brushes:enlarge]}{Koste udad/indad}
{\externalfigure [work:brush:left:raise]}{Koste op/ned}
\stopcombination}
\stop

\page [yes]


\subsubsubject{Befugtning af koste og sugekanal}

Tryk på\index{Fejning+Befugtning} kontakten~\textSymb{temoin_busebalais} mellem sæderne:

{\md Position 1:} Vandpumpen kører automatisk, så længe kostene er aktiveret.

{\md Position 2:} Vandpumpen kører permanent. (Nyttigt til \eG\ indstillingsarbejde.)


\subsubsubject{Groft snavs}

\startSteps [continue]
\item Hvis der er risiko for, at større stykker affald (\eG\ PET-flasker) blokerer sugemunden, kan De åbne\index{Lem for groft snavs} lemmen for groft snavs med tasterne i siden på multifunktionskonsollen, eller~– hvis det ikke er nok~– kan De løfte\index{Sugemund+Groft snavs} sugemunden midlertidigt.
\stopSteps

\start
\setupcombinations [width=\textwidth]

\placefig[here][fig:suctionMouth:clap]{Håndtering af groft snavs}
{\startcombination [2*1]
{\externalfigure [work:suction:open]}{Åbn lemmen for groft snavs}
{\externalfigure [work:suction:raise]}{Løft sugemunden midlertidigt}
\stopcombination}
\stop

\stopsection


\startsection [title={Tømning af smudsbeholderen},
reference={sec:using:container}]

\startSteps
\item Kør\index{Smudsbeholder+Tømning} køretøjet til en egnet plads for at tømme det. Sørg for, at de gældende regler om miljøbeskyttelse overholdes.
\item Træk håndbremsen. (Nødvendigt til aktivering af kontakten til at tippe beholderen).
\stopSteps

\getbuffer [prescription:container:gravity]

\startSteps [continue]
\item Lås smudsbeholderens dæksel op og åbn det.
\item Tryk på kontakten~\textSymb{temoin_kipp2} (midterkonsol, mellem sæderne) for at tippe smudsbeholderen op.
\item Vask beholderen inden i med en vandstråle, når beholderen er tømt. De kan anvende den integrerede vandpistol (valgfrit udstyr).
\stopSteps

\start
\setupcombinations [width=\textwidth]
\placefig[here][fig:brush:adjust]{Håndtering af smudsbeholderen}
{\startcombination [3*1]
{\externalfigure [container:cover:unlock]}{Låsning af dækslet}
{\externalfigure [container:safety:unlocked]}{Sikkerhedsstiver}
{\externalfigure [container:safety:locked]}{Sikkerhedsstiver låst}
\stopcombination}
\stop

\startSteps [continue]
\item Kontroller/rengør beholderens, genbrugssystemets og sugekanalens pakninger og pakningernes kontaktflader.
\stopSteps

\getbuffer [prescription:container:tilt]

\startSteps [continue]
\item Tryk på kontakten~\textSymb{temoin_kipp2} for at sænke smudsbeholderen. (Fjern evt. først sikkerhedsstiverne fra hydraulikcylinderne).
\item Lås smudsbeholderens dæksel.
\stopSteps

\stopsection


\startsection [title={Håndsugeslange},
reference={sec:using:suction:hose}]

\sdeux\ kan efter ønske\index{Håndsugeslange} udstyres med en håndsugeslange. Den er spændt fast på smudsbeholderens dæksel og er nem at betjene.

{\sla Forudsætninger:}

Smudsbeholderen er sænket helt; \sdeux\ er i \aW{Arbejds}modus. (Se \in{§}[sec:using:start], \atpage[sec:using:start].)

\startfigtext[left][fig:using:suction:hose]{Håndsugeslange}
{\externalfigure[work:suction:hose]}
\startSteps
\item Tryk på tasten~\textSymb{temoin_aspiration_manuelle} på loftskonsollen for at aktivere sugesystemet.
\item Træk håndbremsen, før De forlader førerhuset.
\item Luk sugekanalen med sugekanallemmen. (Se \in{§}[sSec:suctionClap], \atpage[sSec:suctionClap].)
\item Træk håndsugeslangen ud af dens holder i mundstykket og påbegynd arbejdet.
\item Tryk på tasten~\textSymb{temoin_aspiration_manuelle} igen, når arbejdet er afsluttet, for at slukke sugesystemet.
\stopSteps
\stopfigtext

\stopsection

\page [yes]

\setups[pagestyle:normal]


\startsection [title={Højtryksvandpistol},
reference={sec:using:water:spray}]

\sdeux\ kan efter ønske\index{Vandpistol} udstyres med en højtryksvandpistol. Vandpistolen er fastgjort i servicedøren bagerst til højre og er forbundet med en 10 meter slangerulle~– på den modsatte side af køretøjet.

Sådan anvender de vandpistolen:

{\sla Forudsætninger:}

Der er nok vand i ferskvandstanken; \sdeux\ er i \aW{Arbejds}modus. (Se \in{§}[sec:using:start], \atpage[sec:using:start].)

\placefig[margin][fig:using:water:spray]{Højtryksvandpistol}
{\externalfigure[work:water:spray]}

\startSteps
\item Tryk på tasten~\textSymb{temoin_buse} på loftskonsollen for at aktivere højtryksvandpumpen.
\item Træk håndbremsen, før De forlader førerhuset.
\item Åbn servicedøren bagerst til højre og tag vandpistolen ud.
\item Rul så meget slange af, som nødvendigt og påbegynd arbejdet.
\item Tryk på tasten~\textSymb{temoin_buse} igen, når arbejdet er afsluttet, for at slukke højtryksvandpumpen.
\item Træk kort i slangen for at løsne blokeringen og rulle slangen op.
\item Sæt vandpistolen fast igen i dens holder og luk servicedøren.
\stopSteps

\stopsection
\stopregister[index][chap:using]

\stopchapter
\stopcomponent

