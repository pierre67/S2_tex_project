\startcomponent c_20_prescriptions_s2_100-da
\product prd_ba_s2_100-da


\chapter [safety:risques] {Sikkerhedsforskrifter}

\setups [pagestyle:marginless]


\section{Generelle anvisninger}

\subsubject{Retsgrundlag}

Ulykker kan få alvorlige konsekvenser, både for arbejdsgiveren og for den ansatte. Vi vil hermed endnu engang minde om begge parters forpligtelser:\note[prescription:user:right].

 Arbejdsgiveren er forpligtet til at overholde følgende punkter, før han/hun lader en ansat betjene fejemaskinen:

\startSteps
\item Alle førere af køretøjet skal være uddannet til at føre køretøjet. Vedkommende skal have et uddannelsesbevis.
\item Alle førere af køretøjet skal have en formel køretilladelse (kørekort). Den må kun udstedes, hvis følgende tre betingelser er opfyldt:
\startitemize [2]
\item Den ansatte har bestået en medicinsk egnethedstest udført af bedriftslægen.
\item Den ansatte kender de steder, hvor arbejdet skal udføres, og kender alle sikkerhedsforskrifter, som den ansatte har modtaget fra sin overordnede og som er gældende for det sted, hvor køretøjet skal anvendes.
\item Den ansatte har bestået en egnethedstest, der attesterer den viden, der er nødvendig for at føre køretøjet.
\stopitemize
\stopSteps

Hvis køretøjets maksimumshastighed er over 25 km/h\note[prescription:user:right], skal køretøjet være indregistreret og føreren af køretøjet skal have følgende køretilladelse:
\startitemize
\item Kørekort klasse B\note[prescription:lisence] til køretøjer med en tilladt totalvægt på under 3,5 tons, eller \item kørekort klasse C\note[prescription:lisence] til køretøjer med en tilladt totalvægt på over 3,5 tons.
\stopitemize

Hvis køretøjets maksimumshastighed er 25 km/h, skal føreren af køretøjet som minimum kende de færdselsregler, der er gældende for kørsel på offentlige veje og pladser, selvom et kørekort klasse B\note[prescription:user:right] ikke er påkrævet til at føre køretøjet.

\footnotetext [prescription:user:right] {Arbejdsgiverens og de ansattes forpligtelser kan variere efter land eller region. Læs og overhold de forskrifter, der er gældende for Deres land eller region.}

\footnotetext[prescription:lisence] {Europa-Parlamentets og Rådets direktiv 2006/126/EF af 20. december 2006 om kørekort.}


\subsubject{Anvendelsesbetingelser}

Fejemaskinen \sdeux\ må udelukkende anvendes, hvis den er i en fejlfri, driftsmæssig tilstand. Derudover skal brugeren overholde sikkerhedsanvisningerne og forskrifterne i den foreliggende betjeningsvejledning. Funktionsfejl, der forringer sikkerheden, skal omgående afhjælpes/repareres på et egnet værksted.
\blank [big]

\startSymList
\externalfigure [s2_inspection] [width=4.5em]
\SymList
{\md Daglig vedligeholdelse:}
Inspicer køretøjet hver gang det har været i brug og reparer synlige skader og defekter. Informer omgående et autoriseret værksted i tilfælde af skader eller funktionsfejl på køretøjet. Hvis dette ikke er muligt, skal køretøjet omgående standses og uheldsstedet skal sikres.
\stopSymList


\subsubject{Korrekt anvendelse}

Fejemaskinen \sdeux\ er beregnet til rengørings- og vedligeholdelsesarbejde på gader, veje og pladser. Hvis den anvendes til andre formål, anvendes den ikke korrekt. Som følge deraf, fralægger \boschung\ sig ethvert ansvar for opståede skader. I tilfælde af uhensigtsmæssig anvendelse, er det alene brugeren, der er ansvarlig for følgerne. {\em Korrekt anvendelse omfatter også, at sikkerhedsanvisningerne og vedligeholdelsesplanen i den foreliggende betjeningsvejledning overholdes.}


\section{Kørsel på offentlige veje}

\subsubject{Generelle forskrifter}

Udover betjeningsanvisningerne, skal alle generelle regler, lovens gældende og andre forskrifter og bestemmelser om forebyggelse af ulykker, overholdes.


\subsubject{Passagersæde}

En passager må sidde på det sæde, det såkaldte {\em passagersæde}, der er beregnet til dette formål.


\subsubject{Sikkerhedsseler}

\startSymList
% \externalfigure [prescription:safety:belt]
\PMbelt
\SymList
Føreren af \sdeux\ og passageren skal tage sikkerhedsselerne på i overensstemmelse med de gældende færdselsregler, når de tager plads i køretøjet.
\stopSymList


\subsubject{At se og blive set}

\startSymList
\externalfigure [travaux_deviation] [width=3.5em]
\SymList
Sørg for, at De er let at se, især på stærkt trafikerede veje.

Hvis føreren af køretøjet ikke har et tilstrækkeligt udsyn under en bestemt manøvre eller handling, skal føreren have hjælp fra en anden person, som føreren konstant skal have visuel kontakt med.
\stopSymList


\subsubject{Lys og signaludstyr}

Afhængigt af de gældende færdselsregler, skal køretøjets forlygter og/eller identifikationslys også være tændt om dagen.


\subsubject{Brug af mobiltelefoner}

\startSymList
\PPphone
\SymList
Det er forbudt at bruge mobiltelefoner eller radioudstyr under kørsel på offentlige veje, medmindre køretøjet er udstyret med håndfrit udstyr.

At tale i telefon\index{Sikkerhed+Mobiltelefon} ved rattet~– også med håndfrit udstyr~– medfører, at chaufføren ikke kan koncentrere sig om trafikken.
\stopSymList


\section{Vedligeholdelsesforskrifter}

\subsubject{Vedligeholdelsesanvisninger}

Vedligeholdelsespersonalet skal have læst betjeningsvejledningen til \sdeux, især afsnittene om sikkerhed og vedligeholdelse, før arbejdet påbegyndes.


\subsubject{Påkrævede kvalifikationer}

\startSymList
\externalfigure [mecanicienne] [width=3.5em]
\SymList
Kun personer, der har den nødvendige viden fra en egnet uddannelse, er autoriseret til at udføre vedligeholdelsesarbejde på \sdeux. Dette gælder især for arbejde på motoren, på bremsesystemet, styretøjet og det elektriske og hydrauliske system.
\stopSymList


\testpage [6]
\subsubject{Opsyn}

\startSymList
\externalfigure [mecanicien_hyerarchie] [width=3.5em]
\SymList
Personer under uddannelse~– i praktik eller lære~– må kun arbejde med køretøjet under opsyn af en faglært person. Kontroller ved hjælp af stikprøver, at personalet overholder betjeningsvejledningen og sikkerhedsforskrifterne.
\stopSymList


\subsubject{Svejsearbejde}

\startSymList
\externalfigure [pince_soudure2] [width=3.5em]
\SymList
Før der udføres svejsearbejde på karosseriet eller chassiset, skal batteriet og alle elektroniske styreanordninger altid frakobles.
\stopSymList

\subsubject{Rengøring af køretøjet}

\startSymList
\externalfigure [washer_pressure] [width=3.5em]
\SymList
Læs før rengøring af \sdeux\ afsnittet \about[sec:cleaning] fra \atpage[sec:cleaning], især afsnittet om rengøringsforskrifter.
\stopSymList


\subsubject{Køretøjsdokumentationens tilgængelighed}

\startSymList
\externalfigure [lecteur_1] [width=3.5em]%\PMrtfm
\SymList
Opbevar altid køretøjsdokumentationen let tilgængeligt i køretøjets førerhus.
\stopSymList


\section{Særlige brugsbestemmelser}

\subsubject{Køretøjets højde}

\startSymList
\PPmaxheight
\SymList
Kontroller altid under arbejde/kørsel i ikke-åbne områder (parkeringskældre, under tunneller, strømledninger osv.), at der er tilstrækkelig frihøjde til \sdeux\ (se \in{afsnit}[sec:measurement], \atpage[sec:measurement]).
\stopSymList


\subsubject{Køretøjets stabilitet}

Undgå manøvrer, der kan forringe køretøjets stabilitet. Høj hastighed i sving kan medføre, at \sdeux\ vælter på grund af dens smalle konstruktion og det højtliggende tyngdepunkt, hvis smudsbeholderen er fuld.


\subsubject{Utilsigtet køretøjsbevægelse}

Når De forlader køretøjet, skal De sikre det mod at blive brugt af uvedkommende personer. Aktiver altid håndbremsen, før De forlader køretøjet. Evt. skal hjulene sikres med kiler.

\startbuffer [prescription:handbrake]
\starttextbackground [CB]
\startPictPar
\PPstop
\PictPar
{\md Træk håndbremsen!} I modsat fald, kan køretøjet utilsigtet sætte i bevægelse, selv\index{Håndbremse+Risikopotentiale} på knapt synlige skråninger og forårsage en ulykke med risiko for livsfarlig kvæstelse af tredjemand.

{\lt På grund af det hydrostatiske drivsystem, reduceres trykket i hydraulikkredsen gradvist, når køretøjet står stille, hvilket medfører en forringelse af motorens holdekraft. Derfor er det særlig vigtigt at trække håndbremsen, når køretøjet forlades.}
\stopPictPar
\stoptextbackground

\stopbuffer

\getbuffer [prescription:handbrake]


\testpage [6]
\subsubject{Smudsbeholder}

\startbuffer [prescription:container:gravity]
\starttextbackground [CB]
\startPictPar
\PHgravite
\PictPar
{\md Risiko for ulykker:}
{\lt Når smudsbeholderen tippes op, flyttes tyngdepunktet opad. Dette øger risikoen for, at køretøjet vælter. Sørg derfor for, at køretøjet står på et vandret og fast underlag, når smudsbeholderen tippes.}
\stopPictPar
\stoptextbackground

\stopbuffer

\getbuffer [prescription:container:gravity]


\startbuffer [prescription:container:tilt]
\starttextbackground [CB]
\startPictPar
\PHcrushing
\PictPar
{\md Risiko for ulykker:}
{\lt Udfør aldrig arbejde under smudsbeholderen, før De har anbragt sikkerhedsstivere på køretøjets hydrauliske løftecylindere.}
\stopPictPar
\stoptextbackground

\stopbuffer

\getbuffer [prescription:container:tilt]


\stopcomponent

