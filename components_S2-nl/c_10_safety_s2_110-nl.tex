\startcomponent c_10_safety_s2_110-nl
\product prd_ba_s2_110-nl

\marking[chapter]{Veiligheidssymbolen}


\chapter{Veiligheidssymbolen}

\setups[pagestyle:marginless]

\section{Nieuwe Europese kenmerking van gevaarlijke stoffen}

{\em Ruitvormig met witte achtergrond en rode rand.}\par\blank[1*medium]
{\em Sinds 2008 geldt in de EU de zogenaamde CLP-verordening\index{CLP-verordening} met nieuwe waarschuwingssymbolen voor gevaarlijke stoffen en producten.}\par\null

\startSymList \GHSgeneric
\SymList
\textDescrHead{Gevaar voor de gezondheid}
Waarschuwt voor\index{Gevaar voor de gezondheid} gevaren voor de gezondheid, die niet tot de dood of ernstige schade aan de gezondheid leiden. Hiertoe behoort de irritatie van de huid of een allergische reactie. Het symbool wordt ook gebruikt als waarschuwing voor andere gevaren, zoals ontvlambaarheid.\par
Vervangt:\crlf \HAZOcross\ of \HAZOpoison\ of \PHgeneric
\stopSymList

\startSymList \GHSbody
\SymList
\textDescrHead{Ernstig gevaar voor de gezondheid; kan met name bij kinderen ook de dood tot gevolg hebben}
Producten kunnen ernstige schade aan de gezondheid veroorzaken. Dit symbool waarschuwt ook voor gevaren\index{Gevaar+Zwangerschap} bij zwangerschap, voor kankerverwekkende werkingen\index{Gevaar+Kankerverwekkende stoffen} en gelijkaardig ernstige risico´s voor de gezondheid. Producten moeten met de nodige voorzichtigheid worden gebruikt.\par
Vervangt:\crlf \HAZOcross\ of \HAZOpoison\
\stopSymList

\startSymList \GHSbomb
\SymList
\textDescrHead{Explosieve stoffen}
Instabiele explosieve\index{Gevaar+Explosie} stoffen, mengsels en substanties met explosieve stoffen\index{Explosieve stoffen} hebben bij hun reactie een heftig expanderende werking, die aanzienlijke verwoesting kan aanrichten; bij ondeskundige omgang bestaat levensgevaar.\par
Vervangt:\crlf \HAZObomb\
\stopSymList


\startSymList \GHSpoison
\SymList
\textDescrHead{Vergiftiging}
Producten\index{Gevaar+Vergiftiging} kunnen zelfs in kleine hoeveelheid op de huid, door inademen\index{Giftige stoffen} of door inslikken zware of zelfs dodelijke vergiftigingen tot gevolg hebben. Geen rechtstreeks contact toelaten.\par
Vervangt:\crlf \HAZOpoison\
\stopSymList

\startSymList \GHSfire
\SymList
\textDescrHead{Licht- of zeer licht ontvlambaar}
Producten\index{Gevaar+Brand} ontbranden snel in de buurt van hitte of vlammen. Sprays met deze kenmerking mogen in geen geval op hete oppervlakken of in de buurt van open vlammen worden versproeid.\par
Vervangt:\crlf \HAZOfire\ of \HAZOfirebis\
\stopSymList

\startSymList \GHSenvironment
\SymList
\textDescrHead{Gevaar voor dieren en milieu}
Producten\index{Bescherming van het milieu} kunnen in het milieu op korte en lange termijn schade\index{Giftige stoffen} veroorzaken. Ze kunnen in het water levende organismen (\eG\ vissen) doden of ook op langere termijn schade toebrengen aan het milieu. In geen geval bij het afvalwater of het huisvuil doen!\par
Vervangt:\crlf \HAZOenvironment\
\stopSymList

\startSymList \GHScorrosive
\SymList
\textDescrHead{Gevaar voor huid of ogen}
Producten\index{Gevaar+Huidverwonding}\index{Gevaar+Oogletsel} kunnen al na kort contact delen van de huid beschadigen en de vorming van littekens tot gevolg hebben, of de ogen onherstelbaar beschadigen. Bescherm bij het gebruik huid en ogen!\par
Vervangt:\crlf \HAZOcross\ of \HAZOcorrosive
\stopSymList

\page [yes]


\section{Waarschuwingssymbool}

{\em Zwarte tekst tegen gele achtergrond}\par\null

\startSymList \PHgeneric
\SymList
\textDescrHead{Algemeen waarschuwingssymbool}
Wijst\index{Gevaar+algemeen}\index{Waarschuwingssymbool} op een onmiddellijk dreigend gevaar, waarbij u of andere personen zich zouden kunnen verwonden.
\crlf\null
\stopSymList

\startSymList \PHpoison
\SymList
\textDescrHead{Waarschuwing voor giftige stoffen}
Giftige stoffen\index{Gevaar+Vergiftiging} kunnen door contact met de huid, inademen of inname ernstige acute of chronische schade aan de gezondheid of zelfs de dood veroorzaken.
\stopSymList

\startSymList \PHfire
\SymList
\textDescrHead{Waarschuwing voor brandgevaarlijke stoffen}
Open vlammen en vonkvorming\index{Gevaar+Brand} vermijden. Stof is licht ontvlambaar of kan brandbevorderend werken. Rookverbod!
\stopSymList

\startSymList \PHexplosive
\SymList
\textDescrHead{Waarschuwing voor explosieve stoffen}
Vaste, vloeibare of gelachtige stoffen of bereidingen, die onder inwerking van schokken, wrijving, vuur, hitte e.\,d. kunnen exploderen.\index{Gevaar+Explosie} Rookverbod!
\stopSymList

\startSymList \PHcrushing
\SymList
\textDescrHead{Waarschuwing voor pletgevaar}
Wijst op een bereik\index{Gevaar+Kneuzing}, waarin op grond van zich bewegende mechanische delen pletgevaar bestaat. Blijf uit de buurt van deze plaats, zolang de inrichting is ingeschakeld.
\stopSymList

\startSymList \PHhand
\SymList
\textDescrHead{Waarschuwing voor handletsels}
Er bestaat het gevaar dat\index{Gevaar+Kneuzing} handen of andere lichaamsdelen\index{Gevaar+Handletsels} worden gekneusd \eG\ tijdens het kantelen van bestuurderscabine of laadbrug.
\stopSymList

\startSymList \PHentangle
\SymList
\textDescrHead{Waarschuwing voor contraroterende rollen/voor intrekgevaar}
Er bestaat het gevaar dat ledematen\index{Gevaar+Intrekken} door roterende delen gegrepen en ingetrokken worden. Blijf uit de buurt, zolang de inrichting is ingeschakeld.
\stopSymList

\startSymList \PHcorrosive
\SymList
\textDescrHead{Waarschuwing voor bijtende stoffen}
Voorzichtig\index{Gevaar+Bijtende stoffen} hanteren, adequate persoonlijke beschermende uitrusting dragen (handschoenen, veiligheidsbril, beschermende kleding).
\stopSymList

\startSymList \PHhot
\SymList
\textDescrHead{Waarschuwing voor heet oppervlak}
Nader het constructiedeel of de inrichting\index{Gevaar+Verbranding}
niet zonder voldoende kennis; handschoenen dragen.
\stopSymList

\startSymList \PHvoltage
\SymList
\textDescrHead{Waarschuwing voor gevaarlijke elektrische spanning}
Niet aanraken met metalen voorwerpen\index{Gevaar+Elektrische spanning}.
Verwondings- of verbrandingsgevaar bij kortsluiting!
\stopSymList

\startSymList \PHfalling
\SymList
\textDescrHead{Waarschuwing voor valgevaar}
In dit bereik bijzonder\index{Gevaar+Val} voorzichtig zijn, aangepast schoeisel dragen (met slipvrije zolen, bestendig tegen koolwaterstof).
\stopSymList

\startSymList \PHbattery
\SymList
\textDescrHead{Waarschuwing voor gevaar door accu´s} Wijst op gevaren die ontstaan bij het laden van accu´s (loodaccu´s)\index{Gevaar+Accu}, met name door ontsnappend waterstofgas en de zwavelzuren in de accu.
\stopSymList

\startSymList \PHremote
\SymList
\textDescrHead{Waarschuwing voor automatische aanloop}
Waarschuwt voor\index{Gevaar+Automatische aanloop} mogelijke automatische of op afstand bediende aanloop van een inrichting.
\stopSymList

% \startSymList \PHquetschgefahr
% \SymList
% \textDescrHead{Risque d’écrasement}
% Risque d’écrasement\index{risque d’écrasement}.
% \stopSymList
% % NOTE: Doppelt! (auch Bilddatei)
%
% % NOTE: Evtl. Folgendes als Ersatz für oben?

% \startSymList\PHhandcrushed
% \SymList
% \textDescrHead{Gefahr von Handquetschungen}
% Es besteht\index{Gefahr+Quetschung} die Gefahr, dass Hände oder Finger
% gequetscht werden. Nähern Sie die Hände nicht an, ohne die Gefahr
% identifiziert und beseitigt zu haben.
% \stopSymList

\startSymList \PHhandfoot
\SymList
\textDescrHead{Waarschuwing voor zich bewegende constructiedelen}
Waarschuwt voor machine-/voertuigdelen in beweging.
\index{Gevaar+Delen in beweging}.
\stopSymList

\startSymList \PHnarrowed
\SymList
\textDescrHead{Waarschuwing voor versmalde rijstrook}
Versmalde\index{Gevaar+Voertuigbreedte} rijstrook.
% Denken Sie an die Breite des Fahrzeugs.
\stopSymList

\page [yes]


\section{Verbodsteken}

{\em Rond met witte achtergrond, rode rand en diagonale balk}
\par\null


\startSymList \PPfire
\SymList
\textDescrHead{Vuur, open licht en roken verboden} Open vuur\index{Verbod+Roken, vuur} en gloed in welke vorm dan ook zijn verboden (\eG\ brandende sigaret, lucifer, kaars; ook elke vorm van vonkvorming).
\stopSymList

\startSymList \PPentry
\SymList
\textDescrHead{Verboden toegang voor onbevoegden}
Onbevoegde\index{Verbod+Toegang} personen mogen dit bereik niet betreden of naderen.
\stopSymList

\startSymList \PPphone
\SymList
\textDescrHead{Mobiele telefoon verboden}
Mobiele telefoons\index{Verbod+Mobiele telefoon} en alle apparatuur die elektromagnetische straling uitzendt, moeten zijn uitgeschakeld. De elektromagnetische straling kan tot functiestoringen van de elektronica leiden.
\stopSymList

\startSymList \PPspray
\SymList
\textDescrHead{Met water spuiten verboden}
Richt een water- of stoomstraal\index{Verbod+Waterstraal, stoom} nooit op gevoelige delen en apparatuur (\eG\ sensoren, besturingsapparaten, injectie-installatie enz.).
\stopSymList

\startSymList \PPchildren
\SymList
\textDescrHead{Kinderen uit de buurt houden}
Aanwijzing\index{Verbod+Kinderen} over een bijzonder gevaar voor kinderen. In het algemeen geldt: Kinderen mogen een ingeschakelde machine niet naderen, ook niet bij onderhoudswerkzaamheden.
\stopSymList

\startSymList \PPwater
\SymList
\textDescrHead{Geen drinkwater}
Het water uit de tank\index{Verbod+Geen drinkwater} niet drinken. Er bestaat vergiftigingsgevaar.
\stopSymList

% \page [yes]


\section{Symbool bescherming van het milieu}

\startSymList \PSrecycle
\SymList
\textDescrHead{Recycling}
Specifieke voorschriften voor de verwerking van bepaald afval volgens de voorschriften.
\stopSymList

\startSymList \PSwelt
\SymList
\textDescrHead{Bescherming van het milieu}
Aanwijzing over geldende voorschriften ter bescherming van het milieu.
\stopSymList

\startSymList \PStrash[width=\PictoHeight,height=,]
\SymList
\textDescrHead{Afval volgens de voorschriften verwerken}
Voor bepaald afval, \eG\ loodaccu´s, gelden speciale verwerkingsvoorschriften.
\stopSymList


\testpage[12]


\section{Gebodsteken}


{\em Rond met blauwe achtergrond}\par\null

\startSymList \PMgeneric
\SymList
\textDescrHead{Algemeen gebodsteken}
Dit teken mag alleen worden gebruikt in combinatie met een aanvullende symbool, dat het gebod preciseert.
\stopSymList


\startSymList \PMrtfm
\SymList
\textDescrHead{Gebruiksaanwijzing in acht nemen}
Vóór de inbedrijfstelling moeten absoluut\index{Gebruiksaanwijzing lezen} de aanwijzingen over dit thema, over een bepaald apparaat of product worden gelezen. De gebruiksaanwijzing moet bij de hand worden bewaard in de bestuurderscabine.
\stopSymList

\startSymList \PMproteyes
\SymList
\textDescrHead{Oogbescherming gebruiken}
Bij werkzaamheden waarbij verwondingsgevaar voor de ogen bestaat, moet een oogbescherming\index{Oogbescherming} worden gedragen.
\stopSymList

\startSymList \PMprothands
\SymList
\textDescrHead{Handbescherming gebruiken}
Bij werkzaamheden waarbij verwondingen aan de hand kunnen ontstaan, moeten werkhandschoenen\index{Werkhandschoenen gebruiken} worden gedragen.
\stopSymList

\startSymList \PMprotears
\SymList
\textDescrHead{Gehoorbescherming gebruiken}
Er moet een gehoorbescherming\index{Gevaar+Gehoor} worden gedragen (\eG in de buurt van een lopende ventilator of een lopende turbine).
\stopSymList

\startSymList \PMsafetybelt
\SymList
\textDescrHead{Veiligheidsgordel gebruiken} Doe\index{Veiligheidsgordel} voor uw eigen veiligheid de veiligheidsgordel om.
\stopSymList

\section{Aanvullende symbolen}

% \adaptlayout[height=+5mm]{{{

% \startSymList \SETshoe
% \SymList
% \textDescrHead{Port de chaussures de sécurité obligatoire}
% Le port de chaussures de sécurité est obligatoire\index{chaussures de sécurité}.
% \stopSymList
%
% \startSymList \SETglasses
% \SymList
% \textDescrHead{Port de lunettes des protection obligatoire}
% Le port de lunettes est obligatoire\index{lunette de protection}.
% \stopSymList
%
% \startSymList \SEToreillettes
% \SymList
% \textDescrHead{Port de casque obligatoire}
% Le port d’un casque de protection est \index{casque} obligatoire.
% \stopSymList
%
% \startSymList \SETgloves
% \SymList
% \textDescrHead{Port de gants de protection obligatoire}
% Le port de gants de protection est obligatoire\index{gants}.
% \stopSymList
%
% \startSymList \SETmainecrase
% \SymList
% \textDescrHead{Risque d’écrasement}
% Danger pour les mains\index{écrasement} et les pieds.
% \stopSymList
%
% \startSymList \SETgetriebe
% \SymList
% \textDescrHead{Risque de happement}
% Risque de happement par\index{happement} des pièces en rotation.
% \stopSymList
%
% \startSymList \SETradkeil
% \SymList
% \textDescrHead{Cale de roue}
% Sécuriser le véhicule contre toute mise\index{Cale de roue} en marche involontaire.
% \stopSymList
%}}}

\startSymList \SETfirstaid
\SymList
\textDescrHead{Eerste Hulp}
Toont de opbergplaats van de Eerste Hulp uitrusting. De snelle alarmering van de reddingsdienst is een belangrijk bestanddeel van de Eerste Hulp.\index{Eerste Hulp}\index{Alarmnummer} Noteer hier uw alarmnummers:
\fillinrules[n=1]{\bf
\framed[align=right,frame=off,offset=none,width=30mm]{Reddingsdienst}}
\fillinrules[n=1]{\bf
\framed[align=right,frame=off,offset=none,width=30mm]{Politie}}
\fillinrules[n=1]{\bf
\framed[align=right,frame=off,offset=none,width=30mm]{Brandweer}}
\stopSymList

\startSymList \SETbrandschutzzeichen
\SymList
\textDescrHead{Brandblussers}
Bepaalde apparaten zijn uitgerust met een of meerdere brandblussers\index{Brandblussers}. Deze vergen in de regel een speciaal onderhoud; meer informatie hierover vindt u op het apparaat of in de gebruiksinstructies van het apparaat.
\stopSymList


\page[yes]

\section{De drie stappen van de hulpverlening}
% NOTE [tf]: Shouldn't be in this book, IMO

\starttextbackground [CB]
\textDescrHead{Beveilig de plaats van het ongeval en de getroffen personen}
\startitemize
\item Controleer de veiligheid van de plaats van het ongeval en garandeer dat er geen andere gevaren optreden.
\stopitemize
\textDescrHead{Stel de toestand van de gewonden vast}
\startitemize
\item Controleer of de gewonden bij bewustzijn zijn en normaal ademen.
Maak evt. de luchtwegen vrij.
\stopitemize
\textDescrHead{Breng de reddingsdiensten op de hoogte}
\startitemize In uw alarmoproep moet u de volgende informatie geven:\par
\item Het telefoonnummer waarop u bent te bereiken.
\item De aard van het voorval (ziekte, ongeval).
\item Bestaande risico´s (brand, explosie, instortgevaar enz.).
\item De precieze plaats van het voorval.
\item Het aantal gewonden en hun toestand.
\item Hulpmaatregelen die reeds werden getroffen.
\item Antwoord op verdere vragen die u worden gesteld.
\stopitemize
\stoptextbackground

\stopcomponent


