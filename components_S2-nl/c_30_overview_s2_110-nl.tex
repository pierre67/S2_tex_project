
\startcomponent c_30_overview_s2_110-nl

\startchapter [title={Overzicht van het voertuig}]

\setups [pagestyle:marginless]


\placefig [here] [] {Overzicht linker voertuigkant}
{\externalfigure [overview:side:left:nl]}


\page [yes]


\placefig [here] [] {Overzicht rechter voertuigkant}
{\externalfigure [overview:side:right:nl]}

\page [yes]

\setups [pagestyle:normal]


\startsection [title={Algemeen}]

\placefig[margin][p4_vue_01]{\sdeux\ bij de aflevering}
{%
\startcombination [1*3]
{\externalfigure[overview:vhc:01]}{}
{\externalfigure[overview:vhc:02]}{}
{\externalfigure[overview:vhc:03]}{}
\stopcombination}

Met het veegvoertuig \BosFull{sdeux} geeft Boschung zijn hele ervaring en competentie, verworven in tientallen jaren tijdens continue samenwerking met zijn trouwe klanten en partners, door.
De eisen van de gemeentes en dienstverleners zijn met het oog op mobiliteit en veelzijdigheid in de loop der tijd enorm gegroeid. De ontwikkelaars van de \sdeux\ zijn deze uitdaging aangegaan, geleid door de behoeften van de klanten, en gestimuleerd door de vooruitziende voorstellen voor verbetering van de Boschung-klantendienst.
Uit deze synthese van klantgerichtheid en het consequent in de praktijk brengen van onze ervaring ontstond de \sdeux\ .


\subsection{Innovatieve technologie}

Het compacte veegvoertuig \BosFull{sdeux} onderscheidt zich in zijn klasse met name door zijn lage gewicht (2300\,kg), zijn hoge capaciteit (vuilcontainer van de 2,0m\high{3} klasse), zijn compacte afmetingen (breedte 1,15\,m) en de bijzonder ergonomische werkplaats voor de bestuurder.

Dankzij de smalle bouwwijze wordt de \sdeux\ een \quotation{'overal'}-veegmachine voor wegen en trottoirs in steden en dorpen. Zijn krachtige dieselmotor in combinatie met de compacte hydrostatische aandrijving (hydromotoren met radiale zuigers op de voorwielen) zorgt te allen tijde voor maximale mobiliteit, onafhankelijk van de gesteldheid op de plaats van inzet of de vulgraad van de vuilcontainer.

De hydraulische pompen worden aangedreven door een dieselmotor van het type \aW{VW 2.0 CDI} volgens Euro-V-norm. Deze levert een koppel van 285\,Nm bij 1750~omdraaiingen en een maximaal vermogen van 75\,kW bij 3000~omdraaiingen. Hierdoor kan de machine reeds bij laag motortoerental~– en daardoor met weinig lawaaioverlast~– effectief worden ingezet. De \sdeux\ bezit standaard een partikelfilter.



\startsection [title={Innovaties in dienst van de klant}]

De schemelstuurinrichting van de \sdeux\ zorgt voor een kleine draaicirkel en daardoor voor maximale beweeglijkheid. Speciale materialen zoals Domex® en de volledig op CAD gebaseerde ontwikkeling van het voertuig maken een opmerkelijke nuttige last van 1200\,kg mogelijk.

\placefig[margin][overview:cab:frontright]{\sdeux\ operationeel}
{\externalfigure[overview:cab:twoleft][width=\Bildwidth]}

De over de hele omtrek beglaasde bestuurderscabine bezit twee comfortabele zitplaatsen, uitgerust met driepuntsgordels. De \sdeux\ kan naar keuze worden uitgerust met een airco.

Met zijn maximumsnelheid van 40\,km/h voegt het voertuig zich probleemloos in in het stadsverkeer. Dankzij de comfortabele vering van de voor- en achteras kan ook op het slechtste traject nog veilig en comfortabel worden gereden.

Het veegaggregaat~– gemonteerd op twee scharnierarmen~– bevindt zich volledig in het gezichtsveld van de bediener en de zuigmond is goed zichtbaar vóór de vooras geplaatst. Een dubbel zwenkbare frontbezem is verkrijgbaar als aanvullende uitrusting.

\page [yes]


\subsection{Geluidsgeïsoleerde en comfortabele bestuurderscabine}

De bestuurderscabine\index{Bestuurderscabine} van de \sdeux\ beschikt over rechtsbesturing en is geconcipieerd voor twee personen. Hij is geluidsgeïsoleerd en gemonteerd op trillingsdempende silentblocs.

Deuren en vloer zijn beglaasd, waardoor een omvattend gezichtsveld ontstaat. De voorruit strekt zich uit over de hele voorkant van het voertuig en geeft zo ongehinderd zicht op het werk van de bezems.

De bestuurdersstoel bezit een mechanische of~– in optie~– pneumatische vering. Bestuurders- en passagiersstoel zijn gemonteerd op instelbare glijrails.


\subsubsubject{Ergonomie}

\startfigtext[right][overview:joy:sideview]{Bedieningsconsole}
{\externalfigure[overview:joy:top]}
De multifunctionele console, aan de linkerkant van de bestuurdersstoel, maakt alle elementaire functies met één hand bereikbaar. De beide bezems kunnen onafhankelijk van elkaar door middel van twee joysticks met duim en wijsvinger worden gestuurd. De schakelaars voor de bezems en voor de frontbezem (optie), voor het motortoeretal, de cruisecontrols enz. bevinden zich eveneens op de multifunctionele console.
\stopfigtext

Aan de onderste rand van het gezichtsveld van de bestuurder bevindt zich een touchscreen, dat alle belangrijke informatie over de functies van de machine in real-time toont, zonder het zicht naar buiten te hinderen.

\placefig[margin][overview:vhc:left]{\sdeux\ voor historische muren}
% \placefig[margin][overview:vhc:left]{\sdeux\ sur site historique}
{\externalfigure[overview:vhc:left]}

\page [yes]


\subsubsubject{Cabine}

De\index{Cabine} keuzehendel voor het rijniveau (\quotation{versnelling}) bevindt zich rechts aan de stuurkolom; er staan twee vooruit- en één achteruitversnelling ter beschikking. Buiten aan de keuzehendel voor het rijniveau zit de drukknop om om te schakelen tussen de beide aandrijfmodi \aW{Werk} en \aW{Rit}. De \sdeux\ hoeft om om te schakelen niet te worden gestopt. (Meer daarover in hoofdstuk \about[sec:using:work], \atpage[sec:using:work].)

\placefig[margin][fig:overview:steeringwheel]{Cabine}
{\externalfigure[overview:driver:place]}

Bij achteruitrijden schakelt de monitor van de achteruitrijcamera in en er weerklinkt een akoestisch waarschuwingssignaal (dat kan worden gedeactiveerd via de Vpad).

De multifunctionele hendel aan de linkerkant van de stuurkolom omvat de ruitenwisserschakelaar (twee niveaus en interval) en lichtsignaal en akoestische claxon.

In hoofdstuk \about[chap:using] vanaf \atpage[chap:using] vindt u details over deze en verdere functies van de \sdeux.

\page [yes]

\setups[pagestyle:marginless]


\subsection[overview:brushsystem]{Veeg- en zuiginrichting}

\subsubsubject{Bezems}

\startfigtext[left][fig:overview:steeringwheel]{Veeg-/Zuiginrichting}
{\externalfigure[system:brush]}
De bezems\index{Vegen} zitten op uitrichtbare koppen, die op hun beurt op scharnierarmen zijn gemonteerd. Het bij het vegen opgewervelde stof wordt gebonden door het te besproeien met water: elke bezem is uitgerust met twee mondstukken, dat zijn water krijgt uit de verswater- of uit de recyclingwatertank.

Een schakelaar\index{Zuigen} van de multifunctionele console activeert tegelijkertijd bezem en waterpomp.\footnote{Voor de waterpomp zie hoofdstuk \in[chap:using] \about[chap:using], met name \about[sec:using:work], \atpage[sec:using:work].}
De posities van de bezems en hun dwars- en langsneiging kunnen rechtstreeks via de bijhorende joystick van de multifunctionele console worden gestuurd.
\stopfigtext

De bezems zijn beschermd door een mechanisch en hydraulisch anti-collisie systeem.


\subsubsubject{Zuigmond}

In werkpositie (neergelaten) rust de zuigmond op 4~rollen en bedekt volledig het vlak tussen de uit elkaar bewogen bezems. Door zijn \quotation{gesleepte} positie is hij bij collisies met hindernissen in hoge mate beschermd tegen mechanische beschadigingen. Bij het achteruitrijden wordt de zuigmond automatisch opgetild.

Een dikke, vervangbare rubber lip zorgt voor de dichte afsluiting naar het wegdek toe. Een elektro-hydraulisch bestuurbare klep aan de voorkant van de zuigmond maakt het mogelijk om grovere objecten op te nemen.


\subsubsubject{Vuilcontainer}

De aluminium vuilcontainer kan tot 50° en tot een hoogte van 1,5\,m (uitloophoogte) omhoog worden gekanteld. Daarin mondt van beneden komend het zuigkanaal met een openingsdiameter van 180\,mm uit.

De aanzuigonderdruk wordt gegenereerd door een krachtige turbine, die horizontaal in de vuilcontainer is gemonteerd. Deze bezit een onderhoudsklep voor reiniging en zichtcontrole.

In de afsluitklep van de vuilcontainer bevinden zich twee aanzuigroosters van roestvrij staal. Deze kunnen voor de reiniging zonder gereedschap worden uitgeklapt. De afsluitklep kan met de hand ontgrendeld en geopend worden.

Door middel van een klep die met de hand kan worden omgelegd, kan de luchtstroom ongecompliceerd tussen zuigkanaal en handzuigslang (optionele uitrusting) worden omgeschakeld.


\subsection{Bevochtigingsinrichting}

\subsubsubject{Verswatersysteem}

De\index{Vegen+Bevochtiging} tank van PE-gietwerk bevindt zich in staande positie achter de bestuurderscabine. De inhoud ervan\index{Verswater+-tank} bedraagt 190\,l.

Een elektrische pomp (6,5\,l/min) transporteert het water naar de sproeimondstukken boven elke bezem (inclusief optionele derde bezem).


\subsubsubject{Afvalwaterrecycling}

Het afvalwater loopt door de microperforaties van de binnenwanden van de afvalwatertank om dan via de recyclingklep weg te stromen in de daaronder gelegen recyclingwatertank. De\index{Recyclingwater+-tank} recyclingwatertank bezit een inhoud van 140\,l.

Een hydraulische dompelpomp transporteert het water naar de sproeimondstukken in het inwendige van de zuigmond en het zuigkanaal.


\testpage [8]
\subsubsubject{Recyclingwatertank}

De recyclingwatertank beschikt over een water-hydraulische vloeistof warmtewisselaar met dubbele functie:

\startitemize[width=35mm,style=\md, command={\setupwhitespace[small]}]
\sym{Functie in de zomer} Het water leidt de warmte van de hydraulische vloeistof via convectie naar de aluminium wanden van de tank, vanwaar die wordt afgestraald aan de omgevingslucht.

\sym{Functie in de winter} De hydraulische vloeistof verwarmt het water in de tank. Hierdoor is het mogelijk om het zuigkanaal en de zuigmond ook bij temperaturen iets onder het vriespunt nog te besproeien.
\stopitemize


\subsubsubject{Bewaking van de watervulstanden}

\startitemize[width=35mm,style=\md, command={\setupwhitespace[small]}]
\sym{Vers water} Bij niet toereikende vulstand verschijnt het symbool~\textSymb{vpad_water} op het Vpad beeldscherm.
\sym{Recyclingwater} Als de vulstand van de recyclingtank staat onder de warmtewisselaar (zie hierboven), dan verschijnt het symbool~\textSymb{vpad_rwater_orange} (geel) op het Vpad beeldscherm. Bij niet toereikende vulstand verschijnt het symbool~\textSymb{vpad_rwater} (rood).
\stopitemize

\subsubsubject{Brede banden (optie)}

De bodemdruk\index{Brede banden} komt overeen met de bandenspanning. Met een bandenspanning van 1,8\,bar wordt een bodemdruk van 18\,N/cm² bereikt. Niettemin wordt de draaglast van de band voor de gegarandeerde asbelasting niet meer bereikt. Met 1,8\,bar kan bij 40\,km/h nog maar een asbelasting van 1495\,kg worden gegarandeerd. Als de bandenspanning anders wordt gekozen dan 3.0\,bar, dan ligt de verantwoordelijkheid bij de eigenaar van het voertuig.

\subsubsubject{Overbelastingsindicatie (optie)}

Als het voertuig\index{Overbelastingsindicatie} wordt overladen, dan verschijnt er een melding op de Vpad. De overbelading wordt vastgesteld met een hoeksensor op de achteras. Standaard is de overbelastingsindicatie ingesteld op 3500\,kg, een tolerantieveld van deze waarde moet echter worden vermeden. Deze instelling van 3500\,kg kan door een vakbedrijf worden gewijzigd.
\page [yes]

\stopsection

\setups[pagestyle:normal]


\startsection [title={Identificatie van het voertuig}]

\subsection{Typeplaatje van het voertuig}

Het typeplaatje van het voertuig\index{Identificatie+Voertuig} bevindt zich in de bestuurderscabine, tegenover de console, onder de passagiersstoel (zie \inF[fig:identity:location], \atpage[fig:identity:location]).


\subsection{Motorcode en -nummer}

De motorcode bevindt zich op het typeplaatje van de motor (sticker), op de haaks gebogen metalen leiding van de koelkring, vooraan aan de motor (vuilcontainer optillen).

Het motornummer is in de motor gegraveerd (\inF[identity:engine:number]). Het bestaat uit negen alfanumerieke tekens: de eerste drie zijn de motorcode, de zes volgende het serienummer van de motor.


\placefig[margin][idvhc]{Typeplaatje van het voertuig}
{\externalfigure[s2:id:plaque]}

\placefig[margin][identity:engine:code]{Motortypeplaatje}
{\externalfigure[engine:id:code]}

\placefig[margin][identity:engine:number]{Motornummer}
{\externalfigure[engine:id:number]}

\page [yes]


\subsection [sec:plateWheel]{Wieltypeplaatje}

Het typeplaatje van de velgen en banden\index{Banden+Spanning} bevindt zich in de bestuurderscabine\index{Velgen+Afmetingen} onder de passagiersstoel.


\subsection{Chassisnummer}

Het chassisnummer\index{Identificatie+Chassisnummer} is ingeslagen aan de rechterkant van het voertuig, onder de bestuurderscabine aan het chassis.


\subsection{\symbol[europe][CEsign]-conformiteit en -kenmerking}

Het~\symbol[europe][CEsign]-conformiteitsteken bevindt zich in de bestuurderscabine tegenover de console, onder de passagiersstoel.

De \sdeux\ vervult de fundamentele veiligheids- en gezondheidseisen van de machinerichtlijn\index{Certificaat+CE-conformiteit}\index{Machinerichtlijn} 2006/42/EG\footnote{richtlijn 2006/42/EG van het Europese Parlement en van de Raad van 17~mei 2006}.
% \textrule

\placefig[margin][idpneus]{Bandenspanning}
{\externalfigure[identity:tires]}

\placefig[margin][fig:identity:location]{Typeplaatjes}
{\externalfigure[identity:location]}

\page [yes]

\setups [pagestyle:marginless]


\startsection[title={Technische gegevens},
reference={donnees_techniques}]

\subsection [sec:measurement] {Voertuigmaten}

\placefig[here][fig:measurement]{\select{caption}{Breedte~– bezems in ruststand of uitgeschoven~–, lengte en hoogte van het voertuig}{Voertuigmaten}}
{\Framed{\externalfigure[s2:measurement]}}

\page [yes]

\placefig[here][fig:measurement]{\select{caption}{Hoogte van het voertuig met omhoog gekantelde vuilcontainer}{Hoogte van het voertuig}}
{\Framed{\externalfigure[s2:measurement:02]}}

\page [yes]

\starttabulate [|lBw(45mm)|p|l|rw(35mm)|]
\FL
\NC Groep\index{Maten} \NC \bf Maat \NC \bf Eenheid \NC \bf Waarde \NC\NR
\ML
\NC Voertuigmaten \NC Lengte (over alles) \NC \unite{mm} \NC 4588,00 \NC\NR
\NC\NC Lengte met 3de\,bezem \NC \unite{mm} \NC 5020,00 \NC\NR
\NC\NC Breedte van het voertuig \NC \unite{mm} \NC 1150,00 \NC\NR
\NC\NC Breedte van het voertuig (over alles) \NC \unite{mm} \NC 1575,00 \NC\NR
\NC\NC Hoogte zonder alzijdig zwaailicht \NC \unite{mm} \NC 1990,00 \NC\NR
\NC\NC Wielstand \NC \unite{mm} \NC 1740,00 \NC\NR
\NC\NC Spoorwijdte \NC \unite{mm} \NC 894,00 \NC\NR
\ML
\NC Veegbreedte \NC Standaard bezems \NC \unite{mm} \NC 2300,00 \NC\NR
\NC\NC Met 3de\,bezem \NC \unite{mm} \NC 2600,00 \NC\NR
\NC\NC Diameter bezems \NC \unite{mm} \NC 800,00 \NC\NR
\NC\NC Breedte zuigmond \NC \unite{mm} \NC 800,00 \NC\NR
\ML
\NC Lastverdeling \NC Leeggewicht\note[weight:empty] vooras \NC \unite{kg} \NC ca. 1100,00 \NC\NR
\NC\NC Leeggewicht\note[weight:empty] achteras \NC \unite{kg} \NC ca. 1200,00 \NC\NR
\NC\NC Leeggewicht\note[weight:empty] \NC \unite{kg} \NC ca. 2300,00 \NC\NR
\NC\NC Toel. totaalgewicht \NC \unite{kg} \NC 3500,00 \NC\NR
\LL
\stoptabulate


\subsection{Spoorradius en veegradius}

\starttabulate [|lBw(45mm)|p|l|rw(35mm)|]
\FL
\NC Afmeting\index{Afmetingen} \NC \bf Maat \NC \bf Eenheid \NC \bf Waarde \NC\NR
\ML
\NC Spoorradius\index{Spoorradius}\index{Maat+Spoorradius} \NC Minimale draaicirkel met bezems \NC \unite{mm} \NC 3325,00 \NC\NR
\ML
\NC Veegradius \NC buiten \NC \unite{mm} \NC 3425,00 tot 3850,00 \NC\NR
\NC\NC binnen \NC \unite{mm} \NC 2025,00 tot 1675,00 \NC\NR
\LL
\stoptabulate

%% TODO en/de/fr: Footnote on preceeding page
\footnotetext[weight:empty]{Standaard configuratie, met bestuurder (ca. 75\,kg).}

\placefig[here][pict:steerin_sweeping:radius]{Spoor-/draaicirkel en veegradius}
{\externalfigure[steerin_sweeping:radius]}

\page [yes]


\subsection{Wielen en banden}

{\sla Standaard afmetingen}

\starttabulate[|lBw(45mm)|p|rw(55mm)|]
\FL
\NC Componenten \NC \bf Uitrusting \NC \bf Waarde \NC\NR
\ML
\NC Banden \NC Standaard afmetingen \NC 205/70 R 15 C \NC\NR
\ML
\NC Velgen \NC Standaard afmetingen \NC 6J\;×\;15 H2 ET 60 \NC\NR
\ML
\NC Bandenspanning\index{Bandenspanning}\NC Standaard, voor/achter \NC 4,5/5,8\,bar \NC\NR
\LL
\stoptabulate

{\sla Brede banden}

\starttabulate[|lBw(45mm)|p|rw(55mm)|]
\FL
\NC Componenten \NC \bf Uitrusting \NC \bf Waarde \NC\NR
\ML
\NC Banden\index{Brede banden} \NC Brede banden \NC 275/60 R 15 107H \NC\NR
\ML
\NC Velgen \NC Brede banden \NC 8LB\;×\;15 ET 30 \NC\NR
\ML
\NC Bandenspanning\index{Bandenspanning} \NC Standaard, voor|/|achter \NC 3,0|/|3,0\,bar \NC\NR
\LL
\stoptabulate


\subsection{Dieselmotor}

\starttabulate [|lBw(45mm)|l|rp|]
\FL
\NC \bf Groep\index{Dieselmotor+Identificatie} \NC \bf Parameters \NC \bf Waarde\NC\NR
\ML
\NC Motortype \NC \NC VW CJDA TDI 2.0 – 475 NE \NC\NR
\NC Algemeen \NC Slag \NC Viertakt \NC\NR
\NC\NC Aantal cilinders \unite{n} \NC 4 \NC\NR
\NC\NC Boring x slag \unite{mm} \NC 81\;×\;95,5 \NC\NR
\NC\NC Totale slagvolume \unite{cm\high{3}} \NC 1968 \NC\NR
\NC\NC Kleppen per cilinder \NC 4 \NC\NR
\NC\NC Volgorde van de klepregeling \NC 1-3-4-2 \NC\NR
\NC\NC Laagste stationaire toerental \unite{min\high{−1}} \NC 830 +50/−25 \NC\NR
\NC Vermogen/Koppel \NC Max. toerental \unite{min\high{−1}} \NC 3400 \NC\NR
\NC\NC Max. vermogen \unite{kW} bij \unite{min\high{−1}} \NC 75 tot 3000 \NC\NR
\NC\NC Max. koppel \unite{Nm} bij \unite{min\high{−1}} \NC 285 tot 1750 \NC\NR
\NC Specifiek verbruik\index{Dieselmotor+Verbruik} \NC Brandstof \unite{g/kWh} \NC 224 (bij max. vermogen) \NC\NR
\NC\NC Olie \unite{g/kWh} \NC 0,22 \NC\NR
\NC Brandstofinstallatie \NC Injectiesysteem \NC Directe injectie \quote{Common Rail} \NC\NR
\NC\NC Brandstoftoevoer \NC Tandwielpomp \NC\NR
\NC\NC Oplading \NC Ja \NC\NR
\NC\NC Laadluchtkoeling \NC Ja \NC\NR
\NC\NC Laaddruk \unite{mbar} \NC 1300\NC\NR
\NC Smeerkring\index{Dieselmotor+Smering} \NC Type \NC Geforceerde smering met olie-/waterwisselaar \NC\NR
\NC\NC Leidingvoeding \NC Rotorpomp \NC\NR
\NC\NC Olieverbruik \unite{Liter/20\,h} \NC <\:0,1 \NC\NR
\NC Koelkring\index{Dieselmotor+Koeling} \NC Totale capaciteit \unite{l} \NC ca. 12 \NC\NR
\NC\NC IJkdruk expansievat \unite{bar} \NC 1,4 \NC\NR
\NC\NC Thermostaat (opening) \unite{°C} \NC 87 \NC\NR
\NC\NC Thermostaat (vol) \unite{°C} \NC 102 \NC\NR
\NC Uitlaatgas \NC Partikelfilter \NC Ja \NC\NR
\NC\NC Zuivering uitlaatgas \NC Ja \NC\NR
\NC\NC Norm \NC Euro 5 \NC\NR
\LL
\stoptabulate


\subsection{Rijprestaties}

\starttabulate[|lBw(45mm)|p|l|rw(35mm)|]
\FL
\NC Rijprestatie\index{Rijprestaties} \NC \bf Configuratie \NC \bf Eenheid \NC \bf Waarde \NC\NR
\ML
\NC Snelheid \NC \aW{werk}modus \NC \unite{km/h} \NC 0 tot 18 (traploos) \NC\NR
\NC\NC \aW{Rij}modus \NC \unite{km/h} \NC 0 tot 40 \NC\NR
\ML
\NC Snelheidsbegrenzing \NC Instelbaar \NC \unite{km/h} \NC 0 tot 25 \NC\NR
\LL
\stoptabulate


\subsection{Elektrische installatie}

{\starttabulate [|lw(65mm)|p|rw(30mm)|]
\FL
\NC \bf Groep \NC \bf Component \NC \bf Waarde \NC\NR
\ML
\NC Accu \NC Loodaccu \NC 12\,V 75\,Ah \NC\NR
\NC Stroomtoevoer \NC Dynamo \NC 14,8\,V 140\,A \NC\NR
\NC Starter \NC Vermogen \NC 1,8\,kW \NC\NR
\NC Audio-uitrusting \NC Radio-aansluiting\index{Radio-aansluiting} en luidsprekers\index{Luidsprekers} \NC Standaard uitrusting \NC\NR
% \NC Sécurité et surveillance \NC Tachygraphe\index{tachygraphe} \NC En option \NC\NR
% \NC\NC Enregistreur de fin de parcours\index{fin de parcours} \NC En option \NC\NR
\NC Verlichtings-/Signaalinrichtingen vooraan \NC Parkeerlicht \NC 12\,V 5\,W \NC\NR
\NC\NC Dimlicht \NC H7, 12\,V 55\,W \NC\NR
\NC\NC Schijnwerpers \NC G886, 12\,V 55\,W \NC\NR
\NC\NC Knipperlichten \NC 12\,V 21\,W \NC\NR
\NC Verlichtings-/Signaalinrichtingen achteraan \NC Gecombineerde remlichten \NC 12\,V 5/21\,W \NC\NR
\NC\NC Knipperlichten \NC 12\,V 21\,W \NC\NR
\NC\NC Achteruitrijlichten \NC 12\,V 21\,W \NC\NR
\NC\NC Kentekenverlichting \NC 12\,V 5\,W \NC\NR
\NC Hulpverlichting \NC Alzijdig zwaailicht \NC H1, 12\,V 55\,W \NC\NR
\LL
\stoptabulate
}
\stopsection

\stopchapter

\stopcomponent

