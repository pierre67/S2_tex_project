\startcomponent c_45_vpad_s2_110-nl
\product prd_ba_s2_110-nl

\startchapter[title={Boordcomputer (Vpad)},
reference={sec:vpad}]

\setups[pagestyle:marginless]


\startsection[title={Beschrijving van de Vpad},
reference={vpad:description}]

\startfigtext [left] {De Vpad SN in de cabine}
{\externalfigure[vpad:inside:view]}
\textDescrHead{Innovatief, intelligent … } De \Vpad\ werd geconcipieerd voor de besturing van aggregaten van gemeentes, wier technologie alsmaar complexer is geworden en die een groot aantal zeer uiteenlopende functies ter beschikking stellen.
Met de \Vpad\ heeft de bediener een systeem bij de hand, dat zich niet daartoe beperkt om in real-time informatie~– visueel of akoestisch~– over alle werk- en machineprocessen te leveren.
Wat de \Vpad\ vooral onderscheidt en wat hem zo baanbrekend maakt, is de intuïtieve gebruikersbegeleiding, bedieningsergonomie en commandologica.

Dankzij zijn veelzijdige functies kan de \Vpad\ uitermate flexibel worden gebruikt, en wordt daardoor tot veel meer dan gewoon een elektronische besturingseenheid.
\stopfigtext

\textDescrHead{… universeel} Compatibiliteit en flexibiliteit stonden bij de ontwikkeling van de \Vpad\ voorop:
als modulaire besturingseenheid kan hij individueel worden aangepast aan lokale omstandigheden en uitrustingsvarianten, en dankzij zijn talrijke elektronische interfaces en routes voor gegevensoverdracht~– tot en met WLAN~– staan alle mogelijkheden open.
De \Vpad\ werkt met de modernste elektronica met 32-bit technologie en een real-time besturingssysteem.
\vfill


\startfigtext[left]{Multifunctionele console}
{\externalfigure[console:topview]}
\textDescrHead{… en modulair} Dankzij zijn modulariteit bezit de \Vpad\ een enorm voordeel:
zo kan de in de \sdeux\ standaard ingezette versie~SN te allen tijde stapsgewijs met andere componenten, zoals bijvoorbeeld een modem of een console (zie afbeelding), worden uitgebreid.
De modulariteit is niet beperkt tot de hardware, ook de software van het systeem kan in hoge mate uitgebreid en aan veranderende behoeften aangepast worden.

De multifunctionele console van de \sdeux\ is een hoogontwikkelde interface tussen bediener en machine. Het hele veeg-/zuigsysteem kan via deze console worden bestuurd.
\stopfigtext

\page [yes]


\subsection[vpad:home]{Hoofdbeeldscherm}

%% Note: outcommented by PB
% \placefig[left][fig:vpad:engineData]{Accueil mode transport}
% {\scale[sx=1.5,sy=1.5]
% {\setups[VpadFramedFigureHome]
% \VpadScreenConfig{
% \VpadFoot{\VpadPictures{vpadClear}{vpadBeacon}{vpadEngine}{vpadSignal}}}
% \framed{\null}}
% }


\start

\setupcombinations[width=\textwidth]

\placefig [here][fig:vpad:engineData]{Hoofdbeeldscherm}
{\startcombination [2*1]
{\setups[VpadFramedFigureHome]% \VpadFramedFigureK pour bande noire
\VpadScreenConfig{
\VpadFoot{\VpadPictures{vpadClear}{vpadBeacon}{vpadEngine}{vpadSignal}}}%
\scale[sx=1.5,sy=1.5]{\framed{\null}}}{\aW{Rij}modus}
{\setups[VpadFramedFigureWork]% \VpadFramedFigureK pour bande noire
\VpadScreenConfig{
\VpadFoot{\VpadPictures{vpadClear}{vpadBeacon}{vpadEngine}{vpadSignal}}}%
\scale[sx=1.5,sy=1.5]{\framed{\null}}}{\aW{Werk}modus}
\stopcombination}

\stop

\blank [1*big]

Het hoofdbeeldscherm van de \Vpad\ omvat alle benodigde elementen voor de bewaking van alle functies van de \sdeux.

In het bovenste deel bevinden zich de controle-indicaties.

Het middelste deel toont in real-time o.\,a. de volgende gegevens:
snelheid, toerental en temperatuur van de motor, brandstofpeil, vulstand van het recyclingwater enz.

De modus \aW{Rijden} wordt gesymboliseerd door een haas~\textSymb{transport_mode}, de modus \aW{Werk} door een schildpad~\textSymb{working_mode}.

De menubalk aan de onderste rand toont de beschikbare menu´s: Druk in het midden van het aanrakingsgevoelige beeldscherm (touchscreen) om aanvullende menu´s te tonen.

\page [yes]

\start % local group for temporary redefinition of \textDescrHead [TF]
\define[1]\textDescrHead{{\bf#1\fourperemspace}}
\startcolumns

\startSymVpad
\externalfigure[vpadTEnginOilPressure][height=1.7\lH]
\SymVpad
\textDescrHead{Motoroliedruk}(rood) Motoroliedruk te laag. Zet de motor onmiddellijk af.

+\:Foutmelding \# 604
\stopSymVpad

\startSymVpad
\externalfigure[vpadWarningBattery][height=1.7\lH]
\SymVpad
\textDescrHead{Acculading}(rood) Acculaadstroom te laag. Breng de garage op de hoogte.
\stopSymVpad

\startSymVpad
\externalfigure[vpadWarningEngine1][height=1.7\lH]
\SymVpad
\textDescrHead{Motordiagnose}(geel) Fout in de motorbesturing. Breng de garage op de hoogte.
\stopSymVpad

\startSymVpad
\externalfigure[vpadWarningService][height=1.7\lH]
\SymVpad
\textDescrHead{Garage opzoeken}(geel) Regulier voertuigonderhoud is noodzakelijk,
of is een fout in de motorbesturing (Garage opzoeken). % Zie na in het onderhoudsschema.

+\:Foutmeldingen \# 650 tot \# 653, of \# 703
\stopSymVpad

\startSymVpad
\externalfigure[vpadTBrakeError][height=1.7\lH]
\SymVpad
\textDescrHead{Remsysteem}(rood) Fout in het remsysteem. Breng de garage op de hoogte.

+\:Foutmelding \# 902
\stopSymVpad


\startSymVpad
\externalfigure[vpadTBrakePark][height=1.7\lH]
\SymVpad
\textDescrHead{Vastzetrem}(rood) Vastzetrem van het voertuig is geactiveerd.

+\:Foutmelding \# 905
\stopSymVpad

\startSymVpad
\externalfigure[vpadTEngineHeating][height=1.7\lH]
\SymVpad
\textDescrHead{Voorgloei-installatie}(geel) Motor wordt voorgegloeid.

Een knipperende lamp geeft aan dat er een fout werd geregistreerd in het gebeurtenissengeheugen.
\stopSymVpad

\columnbreak

\startSymVpad
\externalfigure[vpadTFuelReserve][height=1.7\lH]
\SymVpad
\textDescrHead{Brandstofpeil}(geel) Brandstofpeil is zeer laag (reserve).
\stopSymVpad

\startSymVpad
\externalfigure[vpadTBlink][height=1.7\lH]
\SymVpad
\textDescrHead{Waarschuwingsknipperlichten}(groen) Waarschuwingsknipperlichten zijn geactiveerd.
\stopSymVpad

\startSymVpad
\externalfigure[vpadTLowBeam][height=1.7\lH]
\SymVpad
\textDescrHead{Parkeerlicht}(groen) Parkeerlicht is ingeschakeld.
\stopSymVpad

\startSymVpad
\HL\NC \externalfigure[vpadSyWaterTemp][height=1.7\lH]
\SymVpad
\textDescrHead{Temperatuur}(rood) Temperatuur van de hydraulische vloeistof of van de motor te hoog. Breng de garage op de hoogte.

+\:Foutmelding \# 700 of \# 610
\stopSymVpad

\startSymVpad
\externalfigure[vpadWarningFilter][height=1.7\lH]
\SymVpad
\textDescrHead{Filter verstopt}(rood) Het gecombineerde hydraulische filter of het luchtfilter is verstopt.

+\:Foutmelding \# 702 of \# 851
\stopSymVpad

\startSymVpad
\externalfigure[vpadTSpray][height=1.7\lH]
\SymVpad
\textDescrHead{Waterpistool}(geel) Hogedrukwaterpomp voor het waterpistool is geactiveerd.

Schakelaar \textSymb{temoin_buse} in de plafondconsole.
\stopSymVpad

\startSymVpad
\externalfigure[vpadTClear][height=1.7\lH]
\SymVpad
\textDescrHead{Foutmelding}(rood) Er bevindt zich een foutmelding in het geheugen van de \Vpad. Druk op de toets~\textSymb{vpadClear} om alle geregistreerde berichten te tonen. Breng de garage op de hoogte.
\stopSymVpad

\stopcolumns
\stop % local group for temporary redefinition of \textDescrHead

\stopsection

\page [yes]


\section{Vpad menu´s}

\start

\setupTABLE [background=color,
frame=off,
option=stretch,textwidth=\makeupwidth]

\setupTABLE [r] [each] [style=sans, background=color, bottomframe=on, framecolor=TableWhite, rulethickness=1.5pt]
\setupTABLE [r] [first][backgroundcolor=TableDark, style=sansbold]
\setupTABLE [r] [odd][backgroundcolor=TableMiddle]
\setupTABLE [r] [even] [backgroundcolor=TableLight]
\bTABLE [split=repeat]
\bTABLEhead
\bTR\bTD Menu \eTD\bTD Benaming\index{Vpad+Indicatie} \eTD\bTD Functie \eTD\eTR
\eTABLEhead

\bTABLEbody
\bTR\bTD \externalfigure [v:symbole:clear] \eTD\bTD Foutmelding(en) \eTD\bTD De in de Vpad geregistreerde foutmeldingen tonen en bevestigen. \eTD\eTR
\bTR\bTD \framed[frame=off]{\externalfigure [v:symbole:beacon]\externalfigure [v:symbole:beacon:black]} \eTD\bTD Alzijdig zwaailicht \eTD\bTD Alzijdig zwaailicht aan/uit \eTD\eTR
\bTR\bTD \externalfigure [v:symbole:engine] \eTD\bTD Real-time gegevens \eTD\bTD Real-time gegevens van motor en hydrauliek tonen\eTD\eTR
\bTR\bTD \externalfigure [v:symbole:oneTwoThree] \eTD\bTD Tellers \eTD\bTD Weergave van de bedrijfsurentellers: dagteller, seizoensteller, totaalteller\eTD\eTR
\bTR\bTD \externalfigure [v:symbole:serviceInfo] \eTD\bTD Onderhoudsinterval \eTD\bTD Toont de datum en de resterende bedrijfsuren tot aan het volgende onderhoud of tot aan de volgende grote service \eTD\eTR
\bTR\bTD \externalfigure [v:symbole:trash] \eTD\bTD Teller \eTD\bTD Teller terugzetten of service-interval terugzetten \eTD\eTR
\bTR\bTD \externalfigure [v:symbole:sunglasses] \eTD\bTD Beeldschermmodus \eTD\bTD Soort beeldschermverlichting omschakelen tussen \aW{Dag} en \aW{Nacht} \eTD\eTR
\bTR\bTD \externalfigure [v:symbole:color] \eTD\bTD Helderheid/Contrast \eTD\bTD Instellingen voor helderheid en contrast van het beeldscherm \eTD\eTR
\bTR\bTD \externalfigure [v:symbole:select] \eTD\bTD Selectie \eTD\bTD Selecteren van de gemarkeerde invoer of bevestigen van een foutmelding \eTD\eTR
\bTR\bTD \externalfigure [v:symbole:return] \eTD\bTD Bevestiging \eTD\bTD Bevestigen van de selectie \eTD\eTR
\bTR\bTD \framed[frame=off]{\externalfigure [v:symbole:up]\externalfigure [v:symbole:down]} \eTD\bTD Omhoog/Omlaag, \\pijlen \eTD\bTD Markering naar boven/beneden verschuiven of geselecteerde waarde verhogen/verlagen \eTD\eTR
\bTR\bTD \externalfigure [v:symbole:rSignal] \eTD\bTD Waarschuwingstoon bij achteruitrijden \eTD\bTD Akoestisch signaal bij achteruitrijden activeren/deactiveren \eTD\eTR
\eTABLEbody
\eTABLE
\stop


\subsubsubject{Verdere indicaties op de Vpad}

\start % local group for temporary redefinition of \textDescrHead [TF]
\define[1]\textDescrHead{{\bf#1\fourperemspace}}

\startcolumns

\startSymVpad
\externalfigure[sym:vpad:water]
\SymVpad
\textDescrHead{Vulstand vers water} Vulstand van het verse water niet voldoende (max. 190\,l; achter de bestuurderscabine).
\stopSymVpad

\startSymVpad
\externalfigure[sym:vpad:rwater:yellow]
\SymVpad
\textDescrHead{Vulstand recyclingwater}(geel) Vulstand van het recyclingwater onder de warmtewisselaar. De hydraulische vloeistof wordt niet gekoeld en het bevochtigingssysteem van het zuigkanaal niet opgewarmd.
\stopSymVpad

\startSymVpad
\externalfigure[sym:vpad:rwater]
\SymVpad
\textDescrHead{Vulstand recyclingwater}(rood) Vulstand van het recyclingwater niet voldoende (max. 140\,l; onder de vuilcontainer).
\stopSymVpad

\stopcolumns
\stop % local group for temporary redefinition of \textDescrHead

\page [yes]

\startsection[title={Instellen van de helderheid van het beeldscherm},
reference={sec:vpad:brightness}]

Het beeldscherm van de \Vpad\ kan worden ingezet in twee voorgeconfigureerde helderheidsniveaus: Modus \aW{Dag}~– \textSymb{vpadSunglasses}, normale helderheid~– en modus \aW{Nacht}~– \textSymb{vpadMoon}, verminderde helderheid.
Met de toets \textSymb{vpadColor} krijgt u toegang tot verschillende parameters.
Om de voorgeconfigureerde helderheidsniveaus te veranderen gaat u als volgt te werk:

\startSteps
\item Druk op het midden van het aanrakingsgevoelige beeldscherm (touchscreen) om door de menubalk aan de onderste rand van het beeldscherm te scrollen.
\item Druk op het symbool \textSymb{vpadSunglasses} resp.
\textSymb{vpadMoon} om de modus te selecteren die u wilt veranderen.
\item Druk op \textSymb{vpadColor} om de parameters te tonen.
\item Markeer met behulp van de pijlsymbolen~\textSymb{vpadUp}\textSymb{vpadDown} de parameter die u wilt wijzigen, en selecteer hem met~\textSymb{vpadSelect}.
\item Wijzig de waarde met behulp van de symbolen\textSymb{vpadMinus}\textSymb{vpadPlus}. Voorzichtig, verminder de helderheid niet zo sterk (\VpadOp{162} -255), dat u niets meer op het beeldscherm kunt herkennen!
\stopSteps
\blank [1*big]

\start
\setupcombinations[width=\textwidth]
\startcombination [3*1]
{\setups[VpadFramedFigureHome]% \VpadFramedFigureK pour bande noire
\VpadScreenConfig{
\VpadFoot{\VpadPictures{vpadOneTwoThree}{vpadServiceInfo}{vpadSunglasses}{vpadColor}}}%
\framed{\null}}{Druk in het midden op het touchscreen}
{\setups[VpadFramedFigure]
\VpadScreenConfig{
\VpadFoot{\VpadPictures{vpadReturn}{vpadUp}{vpadDown}{vpadSelect}}}%
\framed{\bTABLE
\bTR\bTD \VpadOp{160} \eTD\eTR
\bTR\bTD [backgroundcolor=black,color=TableWhite] \VpadOp{162}\hfill 15 \eTD\eTR
\bTR\bTD \VpadOp{163}\hfill 180 \eTD\eTR
\bTR\bTD \VpadOp{164}\hfill 55 \eTD\eTR
\bTR\bTD \VpadOp{165}\hfill 3 \eTD\eTR
\eTABLE}}{Selecteer met \textSymb{vpadSelect}}
{\setups[VpadFramedFigure]% \VpadFramedFigureK pour bande noire
\VpadScreenConfig{
\VpadFoot{\VpadPictures{vpadReturn}{vpadMinus}{vpadPlus}{vpadNull}}}%
\framed[backgroundscreen=.9]{\bTABLE
\bTR\bTD \VpadOp{160} \eTD\eTR
\bTR\bTD \VpadOp{162}\hfill -80 \eTD\eTR
\bTR\bTD \VpadOp{163}\hfill 180 \eTD\eTR
\bTR\bTD \VpadOp{164}\hfill 55 \eTD\eTR
\bTR\bTD \VpadOp{165}\hfill 3 \eTD\eTR
\eTABLE}}{Waarde wijzigen met \textSymb{vpadMinus}\textSymb{vpadPlus}}
\stopcombination
\stop
\blank [1*big]

\startSteps [continue]
\item Bevestig de waarde met \textSymb{vpadReturn}.
\item Druk nog eens op het symbool \textSymb{vpadReturn} om terug te gaan naar het hoofdbeeldscherm.
\stopSteps

\stopsection

\page [yes]


\startsection[title={Bedrijfsuren- en kilometerteller},
reference={vpad:compteurs}]

De software van de \Vpad\ beschikt over drie verschillende meetperioden~– \aW{Dag}, \aW{Seizoen}, \aW{Totaal}~–, waarin verschillende tellers kunnen lopen, zoals \aW{Afgelegde traject}, \aW{Bedrijfsuren} (motor of bezem), \aW{Werktijd} (per bestuurder).

Om de tellers af te lezen of ze terug te zetten gaat u als volgt te werk:

\startSteps
\item Druk op het midden van het touchscreen om door de menubalk te scrollen.
\item Druk op het symbool \textSymb{vpadOneTwoThree} om de dagteller te tonen.
\item Met behulp van de Terug/Verder symbolen~\textSymb{vpadBW}\textSymb{vpadFW} kunt u omschakelen naar de totaal- of seizoensteller.
\item Druk op \textSymb{vpadTrash} om de getoonde teller terug te zetten.
\item In een dialoogvenster wordt u gevraagd om het terugzetten te bevestigen.
\stopSteps
\blank [1*big]

\start
\setupcombinations[width=\textwidth]
\startcombination [3*1]
{\setups[VpadFramedFigure]% \VpadFramedFigureK pour bande noire
\VpadScreenConfig{
\VpadFoot{\VpadPictures{vpadOneTwoThree}{vpadServiceInfo}{vpadSunglasses}{vpadColor}}}%
\framed{\bTABLE
\bTR\bTD \VpadOp{120} \eTD\eTR
\bTR\bTD \VpadOp{123}\hfill 87.4\,h \eTD\eTR
\bTR\bTD \VpadOp{125}\hfill 62.0\,h \eTD\eTR
\bTR\bTD \VpadOp{126}\hfill 240.2\,km \eTD\eTR
\bTR\bTD \VpadOp{124}\hfill 901.9\,km \eTD\eTR
\bTR\bTD \VpadOp{127}\hfill 2,1\,l/h \eTD\eTR
\eTABLE}}{Druk op het symbool~\textSymb{vpadOneTwoThree}, vervolgens op~\textSymb{vpadBW} of~\textSymb{vpadFW}}
{\setups[VpadFramedFigure]
\VpadScreenConfig{
\VpadFoot{\VpadPictures{vpadReturn}{vpadBW}{vpadFW}{vpadTrash}}}%
\framed{\bTABLE
\bTR\bTD \VpadOp{121} \eTD\eTR
\bTR\bTD \VpadOp{123}\hfill 522.0\,h \eTD\eTR
\bTR\bTD \VpadOp{125}\hfill 662.8\,h \eTD\eTR
\bTR\bTD \VpadOp{126}\hfill 1605.5\,km \eTD\eTR
\bTR\bTD \VpadOp{124}\hfill 2608.4\,km \eTD\eTR
\bTR\bTD \VpadOp{127}\hfill 2,0\,l/h \eTD\eTR
\eTABLE}}{Zet de teller terug met \textSymb{vpadTrash}}
{\setups[VpadFramedFigure]% \VpadFramedFigureK pour bande noire
\VpadScreenConfig{
\VpadFoot{\VpadPictures{vpadReturn}{vpadTrash}{vpadNull}{vpadNull}}}%
\framed{\bTABLE
\bTR\bTD \VpadOp{121} \eTD\eTR
\bTR\bTD \null \eTD\eTR
\bTR\bTD \VpadOp{136} \eTD\eTR
\bTR\bTD \null \eTD\eTR
\bTR\bTD \VpadOp{137} \eTD\eTR
\eTABLE}}{Bevestig met \textSymb{vpadTrash}}
\stopcombination
\stop
\blank [1*big]

\startSteps [continue]
\item Voer indien nodig het wachtwoord in, en bevestig dan het terugzetten met behulp van het symbool \textSymb{vpadTrash}.
\item Druk op het symbool \textSymb{vpadReturn} om terug te gaan naar het hoofdbeeldscherm.
\stopSteps

\stopsection

\page [yes]

\startsection[title={Wartungsintervalle},
reference={vpad:maintenance}]

Het onderhoudsschema van de \sdeux\ kent twee basisvormen van het onderhoud: het reguliere onderhoud en de grote service (door een met de \boschung-klantendienst overeengekomen vakgarage).

Om de tellers af te lezen of terug te zetten gaat u als volgt te werk:
\startSteps
\item Druk op het midden van het touchscreen om door de menubalk te scrollen.
\item Druk op het symbool \textSymb{vpadServiceInfo} om de onderhoudsintervallen te tonen.
\item Ga met behulp van de pijlsymbolen~\textSymb{vpadUp}\textSymb{vpadDown} naar de gewenste interval.
\item Druk op het symbool~\textSymb{vpadTrash} om een interval terug te zetten. Voer het wachtwoord in met behulp van~\textSymb{vpadPlus}\textSymb{vpadMinus} en bevestig met~\textSymb{vpadSelect}).
\item In een dialoogvenster wordt u gevraagd om het terugzetten te bevestigen.
\stopSteps
\blank [1*big]

\start
\setupcombinations[width=\textwidth]
\startcombination [3*1]
{\setups[VpadFramedFigure]% \VpadFramedFigureK pour bande noire
\VpadScreenConfig{
\VpadFoot{\VpadPictures{vpadReturn}{vpadNull}{vpadNull}{vpadTrash}}}%
\framed{\bTABLE
\bTR\bTD[nc=2] \VpadOp{190} \eTD\eTR
\bTR\bTD \VpadOp{191}\eTD\bTD \VpadOp{195}\hfill 600\,h \eTD\eTR % [backgroundcolor=black,color=TableWhite]
\bTR\bTD \VpadOp{192}\eTD\bTD \VpadOp{195}\hfill 600\,h \eTD\eTR
\bTR\bTD \VpadOp{193}\eTD\bTD \VpadOp{195}\hfill 2400\,h \eTD\eTR
\eTABLE}}{Druk op het symbool~\textSymb{vpadTrash} om een interval terug te zetten}
{\setups[VpadFramedFigure]
\VpadScreenConfig{
\VpadFoot{\VpadPictures{vpadReturn}{vpadMinus}{vpadPlus}{vpadSelect}}}%
\framed{\bTABLE
\bTR\bTD \VpadOp{190} \eTD\eTR
\bTR\bTD \hfill 2014-03-31 \eTD\eTR
\bTR\bTD \null \eTD\eTR
\bTR\bTD \null \eTD\eTR
\bTR\bTD \null \eTD\eTR
\bTR\bTD \null \eTD\eTR
\bTR\bTD \VpadOp{002}\hfill 0000 \eTD\eTR
\eTABLE}}{Voer het wachtwoord in (cijfercode)}
{\setups[VpadFramedFigure]% \VpadFramedFigureK pour bande noire
\VpadScreenConfig{
\VpadFoot{\VpadPictures{vpadReturn}{vpadUp}{vpadDown}{vpadSelect}}}%
\framed{\bTABLE
\bTR\bTD \VpadOp{190} \eTD\eTR
\bTR\bTD[backgroundcolor=black,color=TableWhite] \VpadOp{041}\eTD\eTR % [backgroundcolor=black,color=TableWhite]
\bTR\bTD \VpadOp{042} \eTD\eTR
\bTR\bTD \VpadOp{043} \eTD\eTR
\eTABLE}}{Selecteer en bevestig met~\textSymb{vpadSelect}}
\stopcombination
\stop
\blank [1*big]

\startSteps [continue]
\item Bevestig het terugzetten met behulp van het symbool~\textSymb{vpadSelect}.
\item Druk op het symbool \textSymb{vpadReturn} om terug te gaan naar het hoofdbeeldscherm.
\stopSteps

\stopsection

\page [yes]


\startsection[title={Foutmanagement via de Vpad},
reference={vpad:error}]


De \Vpad\ toont fouten\index{Vpad+Foutmeldingen}, die door de elektronische besturingssystemen gediagnostiseerd en door de CAN-bus overgedragen werden.
Wanneer er een minder ernstige fout wordt geregistreerd, dan brandt het symbool~\textSymb{VpadTClear} (rood).
Wanneer het gaat om een fout van hoge prioriteit, dan brandt het symbool~\textSymb{VpadTClear} en weerklinkt er bovendien een alarmtoon.
Om het alarm te beëindigen moet de foutmelding bevestigd (als \aW{ter kennis genomen} bevestigd) worden.

Om foutmeldingen te lezen en te bevestigen gaat u als volgt te werk:

\startSteps
\item Druk op het symbool~\textSymb{vpadClear} op het beeldscherm van de \Vpad.
\item Druk op het symbool~\textSymb{vpadClear} om de geselecteerde melding te bevestigen.
\item Naast de bevestigde melding verschijnt nu een \aW{\#}-symbool, dat de melding bevestigt als \aW{ter kennis genomen}, en de markering springt naar de volgende melding (voor zover voorhanden).
\item Nadat alle meldingen werden bevestigd, keert het display terug naar het hoofdbeeldscherm.
\stopSteps
\blank [1*big]

\start
\setupcombinations[width=\textwidth]
\startcombination [3*1]
{\setups[VpadFramedFigure]% \VpadFramedFigureK pour bande noire
\VpadScreenConfig{
\VpadFoot{\VpadPictures{vpadReturn}{vpadUp}{vpadDown}{vpadSelect}}}%
\framed{\bTABLE
\bTR\bTD \VpadEr{000} \eTD\eTR
\bTR\bTD [backgroundcolor=black,color=TableWhite] \VpadEr{851a} \eTD\eTR
\bTR\bTD \VpadEr{902} \eTD\eTR
\eTABLE}}{Weergave van de meldingen}
{\setups[VpadFramedFigure]
\VpadScreenConfig{
\VpadFoot{\VpadPictures{vpadReturn}{vpadUp}{vpadDown}{vpadSelect}}}%
\framed{\bTABLE
\bTR\bTD \VpadEr{000} \eTD\eTR
\bTR\bTD [backgroundcolor=black,color=TableWhite] \VpadEr{851} \eTD\eTR
\bTR\bTD \VpadEr{902} \eTD\eTR
\eTABLE}}{Bevestig met~\textSymb{vpadClear}}
{\setups[VpadFramedFigureHome]% \VpadFramedFigureK pour bande noire
\VpadScreenConfig{
\VpadFoot{\VpadPictures{vpadClear}{vpadBeacon}{vpadBeam}{vpadEngine}}}%
\framed{\null}}{Terug naar het hoofdbeeldscherm}
\stopcombination
\stop
\blank [1*big]

\startSteps [continue]
\item Om de meldingen opnieuw te tonen drukt u op het symbool~\textSymb{vpadClear}. Foutmeldingen worden pas dan van de \Vpad\ verwijderd, als de oorzaak van het probleem werd geëlimineerd.
\stopSteps


\subsection{De meest frequente foutmeldingen (met storingsoorzaak)}

\subsubsubject{\VpadEr{604}} % {\#\ 604 Pression huile moteur basse}

+ \textSymb{vpadTEnginOilPressure}~– Schakel de motor onmiddellijk uit. Controleer het oliepeil en breng de garage op de hoogte.


\subsubsubject{\VpadEr{609}} % {\#\ 609 Température eau refroidissement moteur haute}

+ \textSymb{vpadSyWaterTemp}~– Onderbreek uw werk. Laat de motor zonder last verder lopen en observeer de temperatuurontwikkeling:

Wanneer de temperatuur daalt, dan controleert u de vulstanden van koelvloeistof, motorolie en hydraulische vloeistof en de toestand van de koeler.
Wanneer de vulstanden en de koeler in orde zijn, dan rijdt u voor de verdere foutdiagnose voorzichtig naar de garage.

\subsubsubject{\VpadEr{610}} % {\#\ 610 Température eau refroidissement moteur trop haute}

+ \textSymb{vpadSyWaterTemp}~– Onderbreek uw werk. Controleer de vulstanden van koelvloeistof en motorolie, en breng onmiddellijk de garage op de hoogte.


\subsubsubject{\VpadEr{650}} % {\#\ 650 Se rendre à un garage}

+ \textSymb{vpadWarningService}~– Breng onmiddellijk uw garage op de hoogte.
% \VpadEr{651} % {\#\ 651 Moteur en mode urgence}


\subsubsubject{\VpadEr{652}} % {\#\ 652 Inspection véhicule}
% \VpadEr{653} % {\#\ 653 Grand service moteur}

+ \textSymb{vpadWarningService}~– Het volgende reguliere onderhoud is noodzakelijk. Raadpleeg het onderhoudsschema en maak een afspraak met uw garage.


\subsubsubject{\VpadEr{700}} % {\#\ 700 Température d'huile hydraulique}

+ \textSymb{vpadSyWaterTemp}~– Onderbreek uw werk. Laat de motor zonder last verder lopen en observeer de temperatuurontwikkeling:

Wanneer de temperatuur daalt, dan controleert u de vulstanden van koelvloeistof, motorolie en hydraulische vloeistof en de toestand van de koeler.
Wanneer de vulstanden en de koeler in orde zijn, dan rijdt u voor de verdere foutdiagnose voorzichtig naar de garage.


\subsubsubject{\VpadEr{702}} % {\#\ 702 Filtre d'huile hydraulique}

+ \textSymb{vpadWarningFilter}~– Het hydraulische retour- en/of aanzuigfilter is verstopt. Vervang onmiddellijk het filterelement.
% \VpadEr{703} % {\#\ 703 Vidange d'huile hydraulique}


\subsubsubject{\VpadEr{800}} % {\#\ 800 Interrupteur d'urgence actionné}

+ \textSymb{vpadTClear}~– U heeft de Noodstop-schakelaar geactiveerd. Schakel de ontsteking uit en start de motor opnieuw om de melding te verwijderen.


\subsubsubject{\VpadEr{801}} % {\#\ 905 Frein à main actionné}

De vuilcontainer is opgetild of niet helemaal neergelaten. De snelheid van het voertuig is begrensd op 5\,km/h, zolang de vuilcontainer niet is neergelaten.

\subsubsubject{\VpadEr{851}} % {\#\ 851 Filtre à air}

+ \textSymb{vpadWarningFilter}~– Het luchtfilter is verstopt. Vervang onmiddellijk het filterelement.


\subsubsubject{\VpadEr{902}} % {\#\ 902 Pression de freinage}

+ \textSymb{vpadTBrakeError}~– De remdruk is niet voldoende. Onderbreek uw werk en breng onmiddellijk de garage op de hoogte.
% \VpadEr{904} % {\#\ 904 Interrupteur de direction d'avancement}


\subsubsubject{\VpadEr{905}} % {\#\ 905 Frein à main actionné}

+ \textSymb{vpadTBrakePark}~– De vastzetrem is niet volledig ontspannen. De snelheid van het voertuig is begrensd op 5\,km/h, zolang de vastzetrem niet is ontspannen.


\stopsection

\stopchapter

\stopcomponent














