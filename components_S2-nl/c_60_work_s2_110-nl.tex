\startcomponent c_60_work_s2_110-nl
\product prd_ba_s2_110-nl

\startchapter [title={De S2 in het alledaagse leven},
reference={chap:using}]

\setups [pagestyle:marginless]


% \placefig[margin][fig:ignition:key]{Clé de contact}
% {\externalfigure [work:ignition:key]}
\startregister[index][chap:using]{Inbedrijfstelling}

\startsection [title={Inbedrijfstelling},
reference={sec:using:start}]


\startSteps
\item Garandeer dat de reguliere controles en onderhoud volgens de voorschriften werden uitgevoerd.
\item Start de motor met de contactsleutel: ontsteking inschakelen, dan sleutel met de klop mee verder draaien en vasthouden, tot de motor start (alleen mogelijk indien de keuzehendel voor het rijniveau op Neutraal staat).
\stopSteps

\start
\setupcombinations [width=\textwidth]

\placefig[here][fig:select:drive]{Keuzehendel voor het rijniveau}
{\startcombination [2*1]
{\externalfigure [work:select:fDrive]}{Keuzehendel in stand \aW{Vooruit}}
{\externalfigure [work:select:rDrive]}{Keuzehendel in stand \aW{Achteruit}}
\stopcombination}
\stop


\startSteps [continue]
\item Draai de schakelaar van de keuzehendel voor het rijniveau om in de \aW{rij}modus in een rijniveau te schakelen:
\startitemize [R]
\item Eerste niveau
\item Tweede niveau (automatisch bedrijf; start automatisch in eerste niveau)
\stopitemize

of druk op de knop buiten aan de hendel om de \aW{werk}modus te activeren/deactiveren.
\stopSteps

\startbuffer [work:config]
\starttextbackground [FC]
\startPictPar
\PMrtfm
\PictPar
In de werkmodus staat alleen het eerste rijniveau ter beschikking en de motor draait met 1300\,min\high{\textminus 1}.

Regel het motortoerental met behulp van de toetsen~\textSymb{joy_key_engine_increase} en~\textSymb{joy_key_engine_decrease} van de multifunctionele console.
\stopPictPar
\stoptextbackground
\stopbuffer

\getbuffer [work:config]

\startSteps [continue]
\item Druk de keuzehendel voor het rijniveau naar boven en naar voor (vooruit) resp. naar boven en naar achter (achteruit). Zie bovenstaande afbeeldingen.
\item Ontspan alvorens te versnellen te vastzetrem.
\stopSteps

\starttextbackground [FC]
\startPictPar
\PMrtfm
\PictPar
{\md Ontspan de vastzetrem volledig!} De positie van de hendel van de vastzetrem wordt bewaakt door een elektronische sensor: wanneer de vastzetrem niet volledig is ontspannen, dan is de rijsnelheid begrensd op 5\,km/h.
\stopPictPar
\stoptextbackground

\startSteps [continue]
\item Druk langzaam op het gaspedaal om het voertuig in beweging te zetten.
\stopSteps


%% NOTE: New text [2014-04-29]:
\subsection [sSec:suctionClap] {Zuigkanaalklep}

Het zuigsysteem genereert een luchtstroom ofwel van de zuigmond of van de handzuigslang (optie) naar de vuilcontainer.

Een met de hand te bedienen klep (\inF[fig:suctionClap], \atpage[fig:suctionClap]) maakt het mogelijk om de luchtstroom om te schakelen tussen zuigmond en handzuigslang.

\placefig [here] [fig:suctionClap] {Zuigkanaalklep}
{\startcombination [2*1]
{\externalfigure [work:suctionClap:open]}{Zuigkanaal geopend}
{\externalfigure [work:suctionClap:closed]}{Zuigkanaal gesloten}
\stopcombination}

In het normale bedrijf~– werken met de zuigmond~– moet het zuigkanaal geopend zijn (omschakelhendel wijst naar boven).

Om de handzuigslang te kunnen inzetten moet het zuigkanaal gesloten zijn (omschakelhendel wijst naar beneden). Op deze manier wordt de luchtstroom door de handzuigslang geleid.
%% End new text

\stopsection


\startsection [title={Buitenbedrijfstelling},
reference={sec:using:stop}]

\index{Buitenbedrijfstelling}

\startSteps
\item Activeer de vastzetrem (hendel tussen de stoelen) en breng de keuzehendel voor het rijniveau in positie \aW{Neutraal}.
\item Voer de vereiste controlewerkzaamheden~– dagelijkse en evt. wekelijkse controles~– uit zoals beschreven op \atpage[table:scheduledaily].
\stopSteps

\getbuffer [prescription:handbrake]

\stopsection


\startsection [title={Vegen en zuigen},
reference={sec:using:work}]

\startSteps
\item Voer de\index{Vegen} inbedrijfstelling van het voertuig uit zoals beschreven in\in{§}[sec:using:start], \atpage[sec:using:start].
\item Activeer\index{Zuigen} de \aW{werk}modus (knop buiten aan de keuzehendel voor het rijniveau).
\stopSteps

% \getbuffer [work:config]
%% NOTE: outcommented by PB

\startSteps [continue]
\item Druk op de toets~\textSymb{joy_key_suction_brush} om turbine en bezems in te schakelen.

{\md Variant:} {\lt Druk op de toets~\textSymb{joy_key_suction} om alleen met de zuigmond te werken.}

\item Stel de draaisnelheid van de bezems in met behulp van de toetsen~\textSymb{joy_key_frontbrush_increase}\textSymb{joy_key_frontbrush_decrease} van de multifunctionele console.

\item Breng de bezems met behulp van de bijhorende joystick zo in positie, dat u de optimale werkbreedte bereikt.
\stopSteps

\vfill

\start
\setupcombinations [width=\textwidth]

\placefig[here][fig:brush:position]{Positioneren van de bezems}
{\startcombination [2*1]
{\externalfigure [work:brushes:enlarge]}{Bezem naar buiten/binnen}
{\externalfigure [work:brush:left:raise]}{Bezem omhoog/omlaag}
\stopcombination}
\stop

\page [yes]


\subsubsubject{Bevochtiging van bezem en zuigkanaal}

Activeer\index{Vegen+Bevochtiging} de schakelaar~\textSymb{temoin_busebalais} tussen de stoelen:

{\md Positie 1:} Waterpomp loopt automatisch, zolang de bezems geactiveerd zijn.

{\md Positie 2:} Waterpomp loopt permanent. (Nuttig \eG\ voor instelwerkzaamheden.)


\subsubsubject{Grof vuil}

\startSteps [continue]
\item Wanneer het gevaar bestaat dat grotere objecten (\eG\ PET-flessen) de zuigmond blokkeren, dan opent\index{Klep voor grof vuil} u de klep voor grof vuil met de zijdelingse toetsen van de multifunctionele console of~– indien dat niet volstaat~– tilt u\index{Zuigmond+Grof vuil} de zuigmond tijdelijk op.
\stopSteps

\start
\setupcombinations [width=\textwidth]

\placefig[here][fig:suctionMouth:clap]{Omgaan met grof vuil}
{\startcombination [2*1]
{\externalfigure [work:suction:open]}{Klep voor grof vuil openen}
{\externalfigure [work:suction:raise]}{Zuigmond tijdelijk optillen}
\stopcombination}
\stop

\stopsection


\startsection [title={Vuilcontainer leegmaken},
reference={sec:using:container}]

\startSteps
\item Rijd\index{Vuilcontainer+Leegmaken} het voertuig naar een geschikte plaats om de vuilcontainer leeg te maken. Let erop dat de geldende voorschriften ter bescherming van het milieu worden nageleefd.
\item Activeer de vastzetrem en breng de keuzehendel voor het rijniveau in positie \aW{Neutraal}. (Vereist voor de vrijgave van de schakelaar om de container te kantelen).
\stopSteps

\adaptlayout [height=+5mm]

\getbuffer [prescription:container:gravity]

\startSteps [continue]
\item Ontgrendel en open de afsluitklep van de vuilcontainer.
\item Activeer de schakelaar~\textSymb{temoin_kipp2} (middenconsole, tussen de stoelen) om de vuilcontainer omhoog te kantelen.
\item Wanneer de container is leeggemaakt, dan wast u het inwendige met een waterstraal. Hiervoor kunt u het geïntegreerde waterpistool (optionele uitrusting) gebruiken.
\stopSteps

\start
\setupcombinations [width=\textwidth]
\placefig[here][fig:brush:adjust]{Hantering van de vuilcontainer}
{\startcombination [3*1]
{\externalfigure [container:cover:unlock]}{Vergrendeling van de afsluitklep}
{\externalfigure [container:safety:unlocked]}{Veiligheidsbalk}
{\externalfigure [container:safety:locked]}{Veiligheidsbalk vergrendeld}
\stopcombination}
\stop

\startSteps [continue]
\item Controleer/reinig de afdichtingen en de draagvlakken van de afdichtingen van de container, van het recyclingsysteem en van het zuigkanaal.
\stopSteps

\getbuffer [prescription:container:tilt]

\startSteps [continue]
\item Activeer de schakelaar~\textSymb{temoin_kipp2} om de vuilcontainer neer te laten. (Verwijder evt. eerst de veiligheidsbalk van de hydraulische cilinders.)
\item Vergrendel de afsluitklep van de vuilcontainer.
\stopSteps

\stopsection


\startsection [title={Handzuigslang},
reference={sec:using:suction:hose}]

De \sdeux\ kan optioneel\index{Handzuigslang} met een handzuigslang worden uitgerust. Deze is gefixeerd op de afsluitklep van de vuilcontainer; de bediening ervan is ongecompliceerd.

{\sla Voorwaarden:}

De vuilcontainer is volledig neergelaten; de \sdeux\ bevindt zich in de \aW{werk}modus. (Zie \in{§}[sec:using:start], \atpage[sec:using:start].)

\startfigtext[left][fig:using:suction:hose]{Handzuigslang}
{\externalfigure[work:suction:hose]}
\startSteps
\item Druk op de toets~\textSymb{temoin_aspiration_manuelle} van de plafondconsole om het zuigsysteem te activeren.
\item Trek de vastzetrem stevig aan, voordat u de bestuurderscabine verlaat.
\item Sluit het zuigkanaal met de klep. (Zie \in{§}[sSec:suctionClap], \atpage[sSec:suctionClap].)
\item Trek de handzuigslang aan het mondstuk uit zijn houder en begin met het werk.
\item Na beëindiging van het werk activeert u opnieuw de toets~\textSymb{temoin_aspiration_manuelle} om het zuigsysteem uit te schakelen.
\stopSteps
\stopfigtext

\stopsection

\page [yes]

\setups[pagestyle:normal]


\startsection [title={Hogedrukwaterpistool},
reference={sec:using:water:spray}]

De \sdeux\ kan optioneel\index{Waterpistool} met een hogedrukwaterpistool worden uitgerust. Het waterpistool is bevestigd in de rechter onderhoudsdeur achteraan en verbonden met een 10 meter lange slanghaspel~ aan de andere kant van het voertuig~.

Ga als volgt te werk om het waterpistool in te zetten:

{\sla Voorwaarden:}

In de verswatertank zit voldoende water; de \sdeux\ bevindt zich in de \aW{werk}modus. (Zie \in{§}[sec:using:start], \atpage[sec:using:start].)

\placefig[margin][fig:using:water:spray]{Hogedrukwaterpistool}
{\externalfigure[work:water:spray]}

\startSteps
\item Druk op de toets~\textSymb{temoin_buse} van de plafondconsole om de hogedrukwaterpomp te activeren.
\item Trek de vastzetrem stevig aan, voordat u de bestuurderscabine verlaat.
\item Open de rechter onderhoudsdeur achteraan en neem het waterpistool eruit.
\item Rol zo veel slang af als nodig en begin met uw werk.
\item Na beëindiging van het werk activeert u opnieuw de toets~\textSymb{temoin_buse} om de hogedrukwaterpomp uit te schakelen.
\item Trek kort aan de slang om de blokkering los te maken en de slang op te rollen.
\item Bevestig het waterpistool weer in zijn houder en sluit de onderhoudsdeur.
\stopSteps

\stopsection

\page [yes]


\setups [pagestyle:marginless]


\startsection [title={Met de derde borstel werken (optie)},
reference={sec:using:frontBrush},
]

\startSteps
\item Neem\index{vegen} het voertuig in gebruik zoals beschreven in  \in{hoofdstuk }[sec:using:start] \atpage[sec:using:start].
\item Activeren van de \index{3e borstel} Zie \aW{werk}modus (knop buiten aan de keuzehendel voor het rijniveau).
\stopSteps

% \getbuffer [work:config]

\startSteps [continue]
\item Zorg ervoor dat de derde borstel op het scherm van de Vpad geactiveerd is (zie \textSymb{vpadFrontBrush} \textSymb{vpadFrontBrushK}, \atpage[vpad:menu]).
\item Druk op de toets ~\textSymb{joy_key_frontbrush_act} om de hydraulica van de derde borstel te activeren.
\item Druk op de toets~\textSymb{joy_key_frontbrush_left} of~\textSymb{joy_key_frontbrush_right} om de derde borstel in de gewenste richting te laten draaien.

\item Stel de draaisnelheid in met behulp van de toetsen~\textSymb{joy_key_frontbrush_increase} en~\textSymb{joy_key_frontbrush_decrease} van de multifunctionele console.

\item Plaats de borstel met behulp van de joystick zoals getoond in onderstaande afbeeldingen.

\stopSteps

{\md Opmerking:} {\lt Om de zijborstel te kunnen plaatsen, moet met de toets ~\textSymb{joy_key_frontbrush_act} de hydraulica van de derde borstel gedeactiveerd worden.}
\vfill

\start
\setupcombinations [width=\textwidth]

\placefig[here][fig:brush:position]{Plaatsen van de derde borstel}
{\startcombination [2*1]
{\externalfigure [work:frontBrush:move]}{Naar boven/beneden; naar links/rechts}
{\externalfigure [work:frontBrush:incline]}{Hellend}
\stopcombination}
\stop

\stopsection

\stopregister[index][chap:using]

\stopchapter
\stopcomponent

