\startcomponent c_40_control_s2_110-nl
\product prd_ba_s2_110-nl

\startchapter [title={Bedieningselementen van het voertuig},
reference={chap:ctrl}]

\setups[pagestyle:marginless]

\placefig[here][fig:ctrl:cab:front]{Bedieningselementen}
{\externalfigure[ctrl:cab:front]}

\startcolumns [n=3]
\startLongleg
\item Stuurkolom (\in{§}[sec:steeringColumn])
\item Instelling stuurkolom
\item Gas- en rempedaal
\item Boordcomputer \Vpad~SN (\inP[sec:vpad])
\item Plafondconsole (\inP[sec:ctrl:aux])
\item Radio/MP3
\stopLongleg


\subsubsubject{Optionele uitrusting}

\startLongleg [continue]
\item Achteruitrijmonitor
\stopLongleg

\stopcolumns

\startsection [title={De stuurkolom},
reference={sec:steeringColumn}]

\subsection{Instellen van de stuurkolom}

\textDescrHead{Neiging van het stuur} Druk op het pedaal~\Ltwo en stel tegelijkertijd de neiging van de stuurkolom in. Laat het pedaal los om het mechanisme van de stuurkolom weer te vergrendelen.

\page[yes]
\setups [pagestyle:normal]


\subsection{Verlichtings- en signaalinrichtingen}

\placefig [margin] [fig:column:left] {Multifunctionele hendel en draaischakelaar}
{\externalfigure[ctrl:column:left]}

\placefig [margin] [fig:column:right] {Keuzehendel voor het rijniveau}
{\externalfigure[ctrl:column:right]}


\subsubsubject{Draaischakelaar}

\startitemize[width=1.7em]
\sym{\textSymb{com_lowlight}} Dimlicht (draaien~\TorqueR).
\startitemize
\sym{1} Parkeerlicht
\sym{2} Dimlicht
\stopitemize
\stopitemize


\subsubsubject{Multifunctionele hendel}

\startitemize[width=1.7em]
\sym{\textSymb{com_lowlight}} {[}Niet bezet{]}
\sym{\textSymb{com_light}} Lichtsignaal (hendel kort naar boven drukken)
\sym{\textSymb{com_blink}} Richtingaanwijzer (hendel naar voor/achter)
\sym{\textSymb{com_claxonArrow}} Claxon (knop buiten aan de hendel indrukken)
\sym{\textSymb{com_wipper}} Ruitenwisser
\startitemize
\sym{J} Intervalschakeling
\sym{O} Uit
\sym{I} 1ste\,snelheidsniveau
\sym{II} 2de\,snelheidsniveau
\stopitemize
\sym{\textSymb{com_washerArrow}} Ruitensproeierinstallatie (op de krans aan het uiteinde van de hendel drukken).
\stopitemize


\subsubsubject{Keuzehendel voor het rijniveau}

De functies van de keuzehendel voor het rijniveau zijn gedetailleerd beschreven in hoofdstuk~\about[chap:using], vanaf~\atpage[sec:using:start].

\stopsection

\page [yes]


\startsection [title={Overige functies},
reference={sec:ctrl:add}]


\subsection[sec:ctrl:aux]{Plafondconsole}

{\sl De\index{Plafondconsole} plafondconsole bevindt zich vooraan aan het plafond van de bestuurderscabine aan de bestuurderskant.}
\placefig [margin] [fig:console:aux] {Plafondconsole}
{\externalfigure[ctrl:console:aux]}


\placefig [margin] [fig:console:climat] {Verwarming en airco}
{\externalfigure[ctrl:console:climat]}


\startitemize [unpacked][width=1.7em]
\sym{\textBigSymb{temoin_retrochauffant}} Buitenspiegelverwarming
\sym{\textBigSymb{temoin_hazard}} Waarschuwingsknipperlichten
\sym{\textBigSymb{temoin_eclairage_L}} Schijnwerpers
\stopitemize


\subsubsubject{Optionele uitrusting}

\startLeg [unpacked][width=1.7em]
\sym{\textBigSymb{temoin_buse}} Hogedrukwaterpomp voor waterpistool \crlf {\sl zie \atpage[sec:using:water:spray]}
\sym{\textBigSymb{temoin_aspiration_manuelle}} Turbine voor handzuigslang \crlf {\sl zie \atpage[sec:using:suction:hose]}
\stopLeg


\subsection[sec:ctrl:climat]{Verwarming en airco}

{\sl Deze console\index{Verwarmingsconsole} bevindt zich aan de achterwand van de bestuurderscabine, tussen de stoelen.}

\startitemize [unpacked][width=23mm]
\sym{\bf 0\quad I\quad II\quad III} Draaischakelaar voor de ventilator
\sym{\externalfigure[tirette_chauffage][height=1em]} Schuifregelaar voor de temperatuur
\stopitemize


\subsubsubject{Optionele uitrusting}

\startitemize [unpacked][width=1.7em]
\sym{\textBigSymb{temoin_climat_bk}} Airco
\stopitemize

\page [yes]

\setups [pagestyle:bigmargin]


\subsection[sec:ctrl:central]{Middenconsole}

{\sl De\index{Middenconsole} middenconsole bevindt zich tussen de stoelen.}

\placefig [margin] [fig:console:central] {Middenconsole}
{\externalfigure[ctrl:console:central]}


\subsubsubject{Bevochtiging van de bezems}

\startLeg [unpacked][width=1.7em]
\sym{\textBigSymb{temoin_busebalais}} Lagedrukwaterpomp\index{Waterpomp} voor het bevochtigingssysteem\index{Waterpomp+Bevochtiging} van de bezems. (Positie~1: \aW{Automatisch}; Positie~2: \aW{Permanent})
\stopLeg


\subsubsubject{Kantelen van de vuilcontainer}

\setupinmargin[right][style=normal]
\inright{%
\startitemize
\sym{\textSymb{mand_readtheoperatingmanual}} Gelieve de aanwijzingen voor het gebruik van de vastzetrem op \atpage[sec:using:stop] in acht te nemen.
\stopitemize}

\startLeg [unpacked][width=1.7em]
\sym{\textBigSymb{temoin_kipp2}} Kantelen van de vuilcontainer. Om\index{Vuilcontainer+Kantelen} de vuilcontainer te kunnen kantelen moet de vastzetrem geactiveerd zijn en de keuzehendel voor het rijniveau op Neutraal staan.
\stopLeg


\subsubsubject{Noodstop}

\starttextbackground [FC]
\startPictPar
\externalfigure[Emergency_Stop][Pict]
\PictPar
In een noodgeval\index{Noodstop} kunt u alle zuig- en veegapparaten, alsmede de rijaandrijving ook, uitschakelen door een druk op de Noodstop-schakelaar.
\stopPictPar
\stoptextbackground


\subsection[sec:foot:switch]{Voetschakelaar}

\placefig [margin] [fig:foot:switch] {Voetschakelaar}
{\vskip 60pt
\externalfigure[work:foot:switch]}

Met behulp\index{Voetschakelaar} van deze schakelaar aan de voet van de stuurkolom (\inF[fig:foot:switch]) kunt u snel en eenvoudig de bezems neerlaten, indien dit vereist is (\eG\ op de top van een helling, rijden op trottoirs).

\stopsection
\page[yes]
\setups [pagestyle:marginless]


\startsection[title={Multifunctionele console},
reference={ctrl:console:middle}]

\startlocalfootnotes

\startfigtext[left]{Multifunctionele console}
{\externalfigure[overview:joy:large]}


\subsubsubject{Joysticks}

\textDescrHead{Zonder frontbezems (of frontbezems gedeactiveerd):}
De joysticks sturen onafhankelijk van elkaar elk een bezem: optillen/neerlaten~(\textSymb{joystick_aa}) of links/rechts~(\textSymb{joystick_gd}). De linker joystick stuurt de linker bezem, de rechter joystick de rechter bezem.\footnote{Om bij een voertuig dat is uitgerust met frontbezems (optie) de positie van de zijbezems te kunnen veranderen moet de frontbezem gedeactiveerd worden (toets~\textSymb{joy_key_frontbrush_act}).}

\textDescrHead{Met frontbezems:}
Met de linker joystick kunt u de frontbezem optillen/neerlaten (\textSymb{joystick_aa}) en naar links/rechts bewegen (\textSymb{joystick_gd}). Met de rechter joystick neigt u de bezem in zijn langs-~(\textSymb{joystick_aa}) en dwarsas~(\textSymb{joystick_gd}).

\placelocalfootnotes %[height=\textheight]
\stopfigtext
\stoplocalfootnotes
\vfill


\subsubsubject{Zijdelingse toetsen}

\startcolumns

\startPictList
\VPcltr
\PictList
Cruisecontrol: Verhogen van de ingestelde snelheid
\stopPictList\vskip -3pt

\startPictList
\VPclbr
\PictList
Cruisecontrol: Verlagen van de ingestelde snelheid
\stopPictList\vskip -3pt

\startPictList
\VPcrtr
\PictList
Zuigmond optillen
\stopPictList

\startPictList
\VPcrbr
\PictList
Zuigmond neerlaten
\stopPictList\vskip -3pt

\startPictList
\VPcrtf
\PictList
Klep voor grof vuil openen (vooraan aan de zuigmond)
\stopPictList\vskip -3pt

\startPictList
\VPcrbf
\PictList
Klep voor grof vuil sluiten
\stopPictList

\stopcolumns


\subsubsubject{Symbooltoetsen}

\startcolumns

\startSymVpad
\externalfigure[joy:stop]
\SymVpad
\textDescrHead{Stop} Geactiveerde apparaat stoppen:

1\:× indrukken: 3de\,bezem deactiveren\crlf
2\:× indrukken: Alles deactiveren
\stopSymVpad

\startSymVpad
\externalfigure[joy:tempomat]
\SymVpad
\textDescrHead{Cruisecontrol} Cruisecontrols instellen op de momentele snelheid en activeren. Om te deactiveren toets~\textSymb{joy:tempomat} opnieuw activeren of remmen. Versnel/Vertraag met de zijdelingse toetsen.
\stopSymVpad

\startSymVpad
\externalfigure[joy:ftbrs:minus]
\SymVpad
\textDescrHead{Bezemsnelheid} Rotatiesnelheid van de zijbezem of van de frontbezem verlagen.
\stopSymVpad

\startSymVpad
\externalfigure[joy:ftbrs:plus]
\SymVpad
\textDescrHead{Bezemsnelheid} Rotatiesnelheid van de zijbezem of van de frontbezem verhogen.
\stopSymVpad

\startSymVpad
\externalfigure[joy:eng:minus]
\SymVpad
\textDescrHead{Motortoerental} Toerental van de dieselmotor verlagen.
\stopSymVpad

\startSymVpad
\externalfigure[joy:eng:plus]
\SymVpad
\textDescrHead{Motortoerental} Toerental van de dieselmotor verhogen.
\stopSymVpad
\columnbreak

\startSymVpad
\externalfigure[joy:suc]
\SymVpad
\textDescrHead{Afzuiging} Zuigsysteem activeren: Zuigmond wordt neergelaten, turbine en recyclingwaterpomp worden ingeschakeld.\note [recyclingwaterpump] \crlf
Stoptoets~\textSymb{joy:stop} indrukken om het systeem te deactiveren.
\stopSymVpad

\startSymVpad
\externalfigure[joy:sucbrs]
\SymVpad
\textDescrHead{Vegen/Zuigen}Zuig-/veegsysteem activeren: Zuigmond wordt neergelaten, zijbezems worden neergelaten en gepositioneerd, turbine, bezems en recyclingwaterpomp worden ingeschakeld.\note [recyclingwaterpump] \crlf
Stoptoets~\textSymb{joy:stop} indrukken om het systeem te deactiveren.
\stopSymVpad

\footnotetext[recyclingwaterpump]{Verswaterpomp wordt eveneens ingeschakeld, indien de schakelaar~\textBigSymb{temoin_busebalais} op \aW{Automatisch} staat (zie \in [sec:ctrl:central] op \atpage [sec:ctrl:central]).}
\startSymVpad
\externalfigure[joy:ftbrs:act]
\SymVpad
\textDescrHead{Frontbezem geactiveerd} Frontbezem activeren/deactiveren.
%% NOTE @Andrew: Singular
\stopSymVpad

\startSymVpad
\externalfigure[joy:ftbrs:right]
\SymVpad
\textDescrHead{Frontbezem naar links} Draairichting voor werkzaamheden met de frontbezem aan de linkerkant (draairichting: met de klok mee).
\stopSymVpad

\startSymVpad
\externalfigure[joy:ftbrs:left]
\SymVpad
\textDescrHead{Frontbezem naar rechts} Draairichting voor werkzaamheden met de frontbezem aan de rechterkant (draairichting: tegen de klok in).
\stopSymVpad

\stopcolumns

\stopsection

\stopchapter

\stopcomponent













