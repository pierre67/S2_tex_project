\startcomponent c_20_prescriptions_s2_110-nl
\product prd_ba_s2_110-nl


\chapter [safety:risques] {Veiligheidsvoorschriften}

\setups [pagestyle:marginless]


\section{Fundamentele aanwijzingen}

\subsubject{Wettelijke grondslagen}

Ongevallen kunnen ernstige gevolgen met zich meebrengen, zowel voor de werkgever als voor personeel. Wij willen u nog eens herinneren aan de plichten van beide partijen:\note[prescription:user:right].

De werkgever is verplicht om voordat hij een werknemer de bediening van de veegmachine toevertrouwt, de volgende punten in acht te nemen:

\startSteps
\item Elke bestuurder moet zijn opgeleid om het voertuig te besturen. Het bewijs van de opleiding moet voorhanden zijn.
\item Elke bestuurder moet in het bezit zijn van een geldig rijbewijs. Dit mag alleen worden uitgereikt, als de volgende drie voorwaarden zijn vervuld:
\startitemize [2]
\item De werknemer heeft een medische geschiktheidstest door de bedrijfsarts doorstaan.
\item De werknemer kent de plaatsen waar hij voor werk wordt ingezet, en is vertrouwd met alle veiligheidsvoorschriften die gelden voor de plaats van inzet van het voertuig, die hem door zijn meerdere werden meegedeeld.
\item De medewerker heeft een geschiktheidstest doorstaan, die de voor het besturen van het voertuig noodzakelijke kennis attesteert.
\stopitemize
\stopSteps

Wanneer de maximumsnelheid van het voertuig meer dan 25\,km/h bedraagt\note[prescription:user:right], dan moet het voertuig van overheidswege zijn toegelaten en de bestuurder moet in het bezit zijn van het volgende rijbewijs:
\startitemize
\item Rijbewijs van de klasse B\note[prescription:lisence] voor voertuigen met een toegelaten totaalgewicht van minder dan 3,5~ton resp.
\item Rijbewijs van de klasse C\note[prescription:lisence] voor voertuigen met een toegelaten totaalgewicht van meer dan 3,5~ton.
\stopitemize

Wanneer de maximumsnelheid van het voertuig 25\,km/h bedraagt, dan moet de bestuurder minstens het op openbare wegen en pleinen geldige verkeersreglement kennen, ook als voor het besturen van het voertuig geen rijbewijs van de klasse B\note[prescription:user:right] vereist is.

\footnotetext [prescription:user:right] {De verplichtingen van werkgever en personeel kunnen al naargelang land of regio variëren. Maak u vertrouwd met de in uw land resp. uw regio geldende voorschriften.}

\footnotetext[prescription:lisence] {Richtlijn 2006/126/EG van het Europese Parlement en van de Raad van 20~december 2006 over het rijbewijs.}


\subsubject{Gebruiksvoorwaarden}

De \sdeux\ mag uitsluitend worden gebruikt, als hij zich in een foutloze operationele toestand bevindt. Daarnaast moet de bediener de veiligheidsinstructies en voorschriften in deze handleiding naleven. Functiestoringen die de veiligheid negatief beïnvloeden, moeten onmiddellijk door een geschikt vakbedrijf geëlimineerd/gerepareerd worden.
\blank [big]

\startSymList
\externalfigure [s2_inspection] [width=4.5em]
\SymList
{\md Dagelijks onderhoud:}
Onderwerp het voertuig na elke inzet aan een inspectie en repareer zichtbare schade en defecten. Informeer in het geval van schade of functiestoringen onmiddellijk de vakgarage. Als dit niet mogelijk is, stop het voertuig dan onmiddellijk en zet de plaats voor de pechverhelping af.
\stopSymList


\subsubject{Doelmatig gebruik}

De \sdeux\ is geconcipieerd voor reinigings- en onderhoudswerkzaamheden op straten, wegen en pleinen. Elk gebruik buiten dit kader geldt als niet-doelmatig. De firma \boschung\ wijst elke verantwoordelijkheid voor schade als gevolg daarvan van de hand. Bij niet-doelmatig gebruik is alleen de bediener verantwoordelijk voor de gevolgen. {\em Tot het doelmatig gebruik behoort eveneens de naleving van veiligheidsinstructies en het onderhoudsschema in deze handleiding.}


\section{Rijden op openbare wegen}

\subsubject{Algemene voorschriften}

Naast de handleidingen moeten alle algemeen geldige regels, de geldende wettelijke en andere voorschriften en bepalingen ter preventie van ongevallen en ter bescherming van het milieu worden nageleefd.


\subsubject{Passagiersplaats}

Een passagier~ mag alleen plaatsnemen op de voor dit doel voorziene stoel, de zgn. {\em passagiersstoel}.


\subsubject{Veiligheidsgordel}

\startSymList
% \externalfigure [prescription:safety:belt]
\PMbelt
\SymList
Bestuurder en passagier van de \sdeux\ moeten conform het geldende verkeeersreglement de veiligheidsgordel omdoen als ze plaatsnemen in het voertuig.
\stopSymList


\subsubject{Zien en gezien worden}

\startSymList
\externalfigure [travaux_deviation] [width=3.5em]
\SymList
Zorg ervoor dat u goeg zichtbaar bent, met name op drukke wegen.

Wanneer de bestuurder bij een bepaalde manoeuvre of een bepaalde werkzaamheid niet genoeg kan zien, dan moet hij de hulp inroepen van een tweede persoon, waarmee hij continu oogcontact heeft.
\stopSymList


\subsubject{Verlichting en signaalmiddelen}

Afhankelijk van het geldende verkeersreglement moeten evt. ook overdag schijnwerpers en/of alzijdige zwaailichten van het voertuig worden ingeschakeld.


\subsubject{Gebruik van mobiele telefoons}

\startSymList
\PPphone
\SymList
Het gebruik van een mobiele telefoon of zendontvanger tijdens de rit op openbare wegen is verboden, tenzij het voertuig is uitgerust met een handsfree set.

Telefoneren\index{Veiligheid+Mobiele telefoon} aan het stuur~– ook met handsfree set~– beïnvloedt in elk geval de concentratie op het wegverkeer.
\stopSymList


\section{Onderhoudsvoorschriften}

\subsubject{Onderhoudsinstructies}

Het onderhoudspersoneel moet vóór het begin van de werkzaamheden de handleiding van de \sdeux, met name de hoofdstukken over veiligheid en onderhoud, hebben gelezen.


\subsubject{Vereiste kwalificaties}

\startSymList
\externalfigure [mecanicienne] [width=3.5em]
\SymList
Alleen personen die in een adequate opleiding de vereiste kennis hebben verworven, zijn bevoegd om onderhoudswerkzaamheden uit te voeren aan de \sdeux\. Dit geldt met name voor werkzaamheden aan de motor, aan het remsysteem, aan de besturing en aan de elektrische en hydraulische installatie.
\stopSymList


\testpage [6]
\subsubject{Toezicht}

\startSymList
\externalfigure [mecanicien_hyerarchie] [width=3.5em]
\SymList
Personen in opleiding~– stage of in de leer~– mogen alleen onder toezicht van een vakman aan het voertuig werken. Controleer steekproefsgewijs of het personeel de handleiding kent en de veiligheidsvoorschriften naleeft.
\stopSymList


\subsubject{Laswerkzaamheden}

\startSymList
\externalfigure [pince_soudure2] [width=3.5em]
\SymList
Vóór de uitvoering van laswerkzaamheden aan carrosserie of chassis moeten de accu en alle elektronische besturingsapparaten geïsoleerd worden.
\stopSymList

\subsubject{Reiniging van het voertuig}

\startSymList
\externalfigure [washer_pressure] [width=3.5em]
\SymList
Lees vóór de reiniging van de \sdeux\ het hoofdstuk \about[sec:cleaning] vanaf \atpage[sec:cleaning], met name het hoofdstuk over de reinigingsvoorschriften.
\stopSymList


\subsubject{Toegankelijkheid van de voertuigdocumenten}

\startSymList
\externalfigure [lecteur_1] [width=3.5em]%\PMrtfm
\SymList
Bewaar bij inzet de voertuigdocumenten altijd gemakkelijk toegankelijk in de bestuurderscabine van het voertuig.
\stopSymList


\section{Bijzondere gebruiksvoorschriften}

\subsubject{Voertuighoogte}

\startSymList
\PPmaxheight
\SymList
Vergewis u er bij werkzaamheden/ritten in niet-open terrein (ondergrondse garages, tunnels, stroomleidingen enz.) altijd van, dat de doorrijhoogte voor de \sdeux\ voldoende is (zie \in{hoofdstuk}[sec:measurement], \atpage[sec:measurement]).
\stopSymList


\subsubject{Stabiliteit van het voertuig}

Vermijd elke manoeuvre, die de stabiliteit van het voertuig negatief zou kunnen beïnvloeden. Bij verhoogde snelheid in bochten zou de \sdeux\ op grond van zijn smalle bouwwijze en het verhoogde zwaartepunt bij volle vuilcontainer kunnen kantelen.


\subsubject{Ongewilde voertuigbeweging}

Wanneer u het voertuig verlaat, beveilig het dan tegen gebruik door onbevoegde personen. Activeer in principe de vastzetrem voordat u het voertuig verlaat; beveilig de wielen evt. met spieën.

\startbuffer [prescription:handbrake]
\starttextbackground [CB]
\startPictPar
\PPstop
\PictPar
{\md Trek de vastzetrem stevig aan!} Anders kan het voertuig zich ongewild in beweging zetten, zelfs\index{Vastzetrem+Potentiële gevaren} op nauwelijks waarneembare hellingen, en een ongeval met het gevaar van dodelijke verwonding van derden veroorzaken.

{\lt Door het hydrostatische aandrijfsysteem wordt bij stilstand de druk in de hydraulische kring stapsgewijs verminderd, waardoor de motor minder krachtig kan stoppen. Om deze reden is het bijzonder belangrijk om de vastzetrem bij het verlaten van het voertuig altijd stevig aan te trekken.}
\stopPictPar
\stoptextbackground

\stopbuffer

\getbuffer [prescription:handbrake]


\testpage [6]
\subsubject{Vuilcontainer}

\startbuffer [prescription:container:gravity]
\starttextbackground [CB]
\startPictPar
\PHgravite
\PictPar
{\md Ongevallenrisico:}
{\lt Bij het omhoog kantelen van de vuilcontainer verplaatst het zwaartepunt zich naar boven. Hierdoor neemt het gevaar toe dat het voertuig kantelt. Let er daarom bij het kantelen van de vuilcontainer op, dat het voertuig op een horizontale en stabiele ondergrond staat.}
\stopPictPar
\stoptextbackground

\stopbuffer

\getbuffer [prescription:container:gravity]


\startbuffer [prescription:container:tilt]
\starttextbackground [CB]
\startPictPar
\PHcrushing
\PictPar
{\md Ongevallenrisico:}
{\lt Voer nooit werkzaamheden uit onder de vuilcontainer, voordat u de veiligheidsbalk heeft aangebracht aan de hydraulische hefcilinders van de vuilcontainer.}
\stopPictPar
\stoptextbackground

\stopbuffer

\getbuffer [prescription:container:tilt]


\stopcomponent

