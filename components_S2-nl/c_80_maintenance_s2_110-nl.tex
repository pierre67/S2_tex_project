\startcomponent c_80_maintenance_s2_110-nl
\product prd_ba_s2_110-nl

\startchapter [title={Onderhoud en instandhouding},
reference={chap:maintenance}]

\setups[pagestyle:marginless]


\startsection [title={Algemene informatie}]


\subsection{Bescherming van het milieu}

\starttextbackground [FC]
\setupparagraphs [PictPar][1][width=2.45em,inner=\hfill]

\startPictPar
\Penvironment
\PictPar
\Boschung\ zet bescherming van het milieu\index{Bescherming van het milieu} om in de praktijk. Wij beginnen bij de oorzaken en houden bij onze beslissingen als ondernemer rekening met alle uitwerkingen van het productieproces en van het product op het milieu. Doelstellingen zijn de spaarzame inzet van grondstoffen en een behoedzame omgang met de natuurlijke bestaansvoorwaarden, wier behoud mens en natuur dient. Door het naleven van bepaalde regels bij de inzet van het voertuig kunt u bijdragen aan de bescherming van het milieu. Daartoe behoort ook de verstandige omgang conform de voorschriften met stoffen en materialen in het kader van het voertuigonderhoud (\eG\ het verwerken van chemicaliën en giftig afval).

Brandstofverbruik en slijtage van een motor hangen af van de operationele voorwaarden. Daarom verzoeken wij u om op enkele punten te letten:

\startitemize
\item Laat de motor niet stationair warmlopen.
\item Zet de motor gedurende eventuele wachttijden tijdens het bedrijf af.
\item Controleer regelmatig het brandstofverbruik.
\item {\em Laat de onderhoudswerkzaamheden uitvoeren conform het onderhoudsschema en door een competente vakgarage.}
\stopitemize
\stopSymList
\stoptextbackground

\page [yes]


\subsection{Veiligheidsvoorschriften}

\startSymList
\PHgeneric
\SymList
Om\index{Onderhoud+Veiligheidsvoorschriften} schade aan voertuig en aggregaten en ongevallen bij het onderhoud te verhinderen is het absoluut vereist om de volgende veiligheidsvoorschriften na te leven. Neem eveneens de algemene veiligheidsvoorschriften (\about[safety:risques], \at{vanaf pagina}[safety:risques]) in acht.
\stopSymList

\starttextbackground [FC]
\startPictPar
\PMgeneric
\PictPar
\textDescrHead{Preventie van ongevallen}
Controleer\index{Preventie van ongevallen} de toestand van het voertuig na elke onderhouds- of reparatiewerkzaamheid. Garandeer met name dat alle veiligheidsrelevante componenten en verlichtings- en signaalinrichtingen foutloos functioneren, voordat u zich op de openbare weg begeeft.
\stopPictPar
\stoptextbackground
\blank [big]

\start
\setupparagraphs [SymList][1][width=6em,inner=\hfill]
\startSymList\PHcrushing\PHfalling\SymList
\textDescrHead{Stabilisering van het voertuig}
Vóór elke onderhoudswerkzaamheid moet het voertuig tegen ongewilde bewegingen worden beveiligd: zet de keuzehendel voor het rijniveau op \aW{Neutraal}, activeer de vastzetrem en beveilig het voertuig met wielspieën.
\stopSymList
\stop

\starttextbackground[CB]
\startPictPar\PHpoison\PictPar
\textDescrHead{Starten van de motor}
Wanneer\index{Gevaar+Vergiftiging} u de motor moet starten op een slecht geventileerde plaats, laat hem dan slechts zo lang als nodig\index{Gevaar+Uitlaatgassen} lopen, om koolmonoxidevergiftigingen te verhinderen.
\stopPictPar
\startitemize
\item Start de motor alleen bij zoals voorgeschreven aangesloten accu.
\item Isoleer de accu nooit bij lopende motor.
\item Start de motor niet met een hulpontsteking.
Wanneer\index{Accu+Lader} de accu met een snellader moet worden geladen, dan moet hij eerst van het voertuig geïsoleerd worden. Neem de bedrijfsvoorschriften van de snellader in acht.
\stopitemize
\stoptextbackground

\page [bigpreference]

\subsubsection{Bescherming van de elektronische componenten}

\startitemize
\item Voordat\index{Elektrisch lassen} u begint met laswerkzaamheden, isoleert u de accukabels van de accu en sluit u plus- en massakabel aaneen.
\item Sluit\index{Elektronica} elektronische besturingsapparaten alleen aan en isoleer deze alleen, als ze niet onder spanning staan.
\item Een verkeerde\index{Besturingsapparaat} polariteit in de stroomtoevoer (\eG\ door verkeerd aangesloten accu´s) kan elektronische componenten en apparaten vernietigen.
\item Bij\index{Omgevingstemperatuur+extreme} omgevingstemperaturen hoger dan 80 °C (\eG\ in een droogkamer) moeten elektronische componenten/apparaten worden verwijderd.
\stopitemize


\subsubsection{Diagnose en metingen}

\startitemize
\item Gebruik voor meet- en diagnosewerkzaamheden alleen {\em geschikte} testkabels (\eG\ de originele kabels van het apparaat).
\item Mobiele telefoons\index{Mobiele telefoon}, en vergelijkbare zendontvangers, kunnen de functies van het voertuig, van de zendontvanger en daarmee de bedrijfsveiligheid negatief beïnvloeden.
\stopitemize



\subsubsection{Kwalificatie van het personeel}

\starttextbackground[CB]
\startPictPar
\PHgeneric
\PictPar
\textDescrHead{Ongevallenrisico}
Bij\index{Kwalificatie+Onderhoudspersoneel} ondeskundige uitvoering van onderhoudswerkzaamheden kunnen functionaliteit en veiligheid van het voertuig negatief worden beïnvloed. Dit brengt een verhoogd ongevallen- en verwondingsrisico met zich mee.

Wend u\index{Kwalificatie+Garage} voor onderhouds- en reparatiewerkzaamheden tot een gekwalificeerde vakgarage, die over de vereiste kennis en gereedschappen beschikt.

Wend u in geval van twijfel tot de \Boschung-klantendienst.
\stopPictPar
\stoptextbackground

% \page [yes]

De \ProductId mag uitsluitend door gekwalificeerd en door de \Boschung-klantendienst opgeleid personeel bediend, onderhouden of gerepareerd worden.

De bevoegdheden voor bediening, instandhouding en reparatie worden verleend door de \Boschung-klantendienst.

%\adaptlayout [height=+5mm]


\subsubsection{Veranderingen en ombouwingen}

\starttextbackground[CB]
\startPictPar
\PHgeneric
\PictPar
\textDescrHead{Ongevallenrisico}
Alle\index{Verandering aan het voertuig} veranderingen die u eigenmachtig uitvoert aan het voertuig, kunnen het goede functioneren en de bedrijfsveiligheid van de \ProductId negatief beïnvloeden en daardoor een niet in te schatten ongevallen- en verwondingsrisico met zich meebrengen.
\stopPictPar

\startPictPar
\PMwarranty
\PictPar
Voor schade als gevolg van\index{Garantie+Voorwaarden} eigenmachtige ingrepen of modificaties aan de \ProductId of een aggregaat verleent \Boschung\ geen enkele garantie of coulance.
\stopPictPar
\stoptextbackground

\stopsection


\startsection [title={Bedrijfsstoffen en smeermiddelen}, reference={sec:liquids}]


\subsection{Juiste omgang}

\starttextbackground[CB]
\startPictPar
\PHpoison
\PictPar
\textDescrHead{Verwondings- en vergiftigingsgevaar}
Het\index{Brandstof} contact met de huid\index{Smeermiddelen} of\index{Gevaar+Vergiftiging} inslikken van bedrijfsstoffen en smeermiddelen kan\index{Brandstof+Veiligheid} aanzienlijke verwondingen of vergiftigingen veroorzaken. Neem bij de hantering, opslag en verwerking van deze stoffen altijd de wettelijke voorschriften in acht.
\stopPictPar
\stoptextbackground

\starttextbackground [FC]
\startPictPar
\PMproteyes\par
\PMprothands
\PictPar
Draag bij de omgang met bedrijfsstoffen en smeermiddelen altijd geschikte beschermende kleding en ademhalingsbescherming. Vermijd het inademen van de dampen.
Vermijd elk contact met huid, ogen of kleding. Reinig plekken van de huid die in aanraking zijn gekomen met bedrijfsstoffen, meteen met water en zeep. Als bedrijfsstoffen in aanraking zijn gekomen met de ogen, spoel deze dan rijkelijk met helder water en raadpleeg evt. een oogarts. Na het inslikken van bedrijfsstoffen moet onmiddellijk een arts worden opgezocht!
\stopPictPar
\stoptextbackground

\startSymList
\PPchildren
\SymList
Bedrijfsstoffen moeten voor kinderen ontoegankelijk worden bewaard.
\stopSymList

\startSymList
\PPfire
\SymList
\textDescrHead{Brandgevaar}
Op grond\index{Gevaar+Brand} van de hoge ontvlambaarheid van bedrijfsstoffen neemt bij de omgang hiermee het brandrisico toe. Roken, vuur\index{Rookverbod} en open licht zijn bij de omgang met bedrijfsstoffen ten strengste verboden.
\stopSymList


%% TODO; en
\starttextbackground [FC]
\startPictPar
\PMgeneric
\PictPar
Er mogen alleen smeermiddelen worden ingezet die geschikt zijn voor de in de \ProductId ingezette componenten. Gebruik daarom alleen door \Boschung\ geteste en vrijgegeven producten. Deze vindt u in de lijst met bedrijfsmiddelen \atpage[sec:liqquantities]. Additieven\index{Additieven} voor smeermiddelen zijn niet vereist. Indien u additieven toevoegt, kan dit tot gevolg hebben dat het recht op garantie\index{Garantie+Voorwaarden} komt te vervallen.
Wend u voor meer informatie tot de \Boschung-klantendienst.
\stopPictPar
\stoptextbackground

\starttextbackground [FC]
\startPictPar
\Penvironment
\PictPar
\textDescrHead{Bescherming van het milieu}
Let\index{Smeermiddelen+Verwerking} bij de verwerking van bedrijfsstoffen en\crlf smeermiddelen\index{Bescherming van het milieu} of voorwerpen die hiermee gecontamineerd zijn (\eG\ filters, doeken), op\index{Bedrijfsstoffen+Verwerking} de naleving van de voorschriften ter bescherming van het milieu.
\stopPictPar
\stoptextbackground

\page [yes]

\setups [pagestyle:normal]


\subsection[sec:liqquantities]{Specificaties en vulhoeveelheden}

Alle\index{Bedrijfsstoffen+Vulhoeveelheid}\index{Smeermiddelen+Vulhoeveelheid}\index{Vulhoeveelheden+Bedrijfsstoffen en smeermiddelen}\index{Specificaties+Bedrijfsstoffen en smeermiddelen} in de volgende tabel aangegeven vulhoeveelheden zijn richtwaarden. Na elke vervanging van bedrijfsstoffen/smeermiddelen moet de daadwerkelijke vulstand gecontroleerd en evt. de vulhoeveelheid verhoogd of verlaagd worden.
% \blank[big]

\placetable[margin][tab:glyco]{Antivries (\index{Antivries}motor)}
{\noteF\startframedcontent[FrTabulate]
%\starttabulate[|Bp(80pt)|r|r|]
\starttabulate[|Bp|r|r|]
\NC Bescherming tegen bevriezing tot {[}°C{]}\NC \bf \textminus 25 \NC \bf \textminus 40 \NC\NR
\NC Gedestill. water [Vol.-\%] \NC 60 \NC 40 \NC\NR
\NC Antivriesmiddel \break [Vol.-\%] \NC 40 \NC {\em max.} 60 \NC\NR
\stoptabulate\stopframedcontent\endgraf
Opgelet: Bij een volumeaandeel van meer dan 60\hairspace\percent\ antivriesmiddel {\em daalt} de bescherming tegen bevriezing en verslechtert de koelcapaciteit!}

\placefig[margin][fig:hydrgauge]{\select{caption}{Niveau-indicatie hydraulische vloeistof (linkerkant van het voertuig)}{Niveau-indicatie hydraulische vloeistof}}
{\externalfigure[main:hy:level_temp]
\noteF De vulstand van de hydraulische tank kan aan het kijkglas afgelezen en moet {\em dagelijks} gecontroleerd worden.}


\vskip -8pt
\start
\define [1] \TableSmallSymb {\externalfigure[#1][height=4ex]}
\define\UC\emptY
\pagereference[page:table:liquids]

\setupTABLE [frame=off,style={\ssx\setupinterlinespace[line=.86\lH]},background=color,
option=stretch,
split=repeat]
\setupTABLE [r] [each] [topframe=on,
framecolor=TableWhite,
% rulethickness=.8pt
]

\setupTABLE [c] [odd] [backgroundcolor=TableMiddle]
\setupTABLE [c] [even] [backgroundcolor=TableLight]
\setupTABLE [c] [1][width=30mm]
\setupTABLE [c] [2][width=20mm]
\setupTABLE [c] [4][width=25mm]
\setupTABLE [c] [last] [width=10mm]
\setupTABLE [r] [first] [topframe=off,style={\bfx\setupinterlinespace[line=.95\lH]},
% backgroundcolor=TableDark
]
\setupTABLE [r] [2][framecolor=black]

\bTABLE

\bTABLEhead
\bTR
\bTC Groep \eTC
\bTC Categorie \eTC
\bTC Classificatie \eTC
\bTC Product\note[Produkt] \eTC
\bTC Hoeveelheid \eTC
\eTR
\eTABLEhead

\bTABLEbody
\bTR \bTD Dieselmotor \eTD
\bTD Motorolie\eTD
\bTD \liqC{SAE 5W-30}; \liqC{VW 507.00}\eTD
\bTD Total Quartz INEO Long Life \eTD
\bTD 4,3 l\eTD
\eTR
\bTR \bTD Hydraulische kring \eTD
\bTD Hydraulische olie \eTD
\bTD \liqC{ISO VG 46} \eTD
\bTD Total Equiviz ZS 46 (tank ca. 40 l) \eTD
\bTD ca. 50 l\eTD
\eTR
\bTR \bTD Hydraulische kring (optie \aW{Bio})\eTD
\bTD Hydraulische olie \eTD
\bTD \liqC{ISO VG 46} \eTD
\bTD Total Biohydran TMP SE 46\eTD
\bTD ca. 50 l\eTD
\eTR
\bTR \bTD Magneetkleppen: spoelkernen \eTD
\bTD Smeermiddel\eTD
\bTD Kopervet \eTD
\bTD \emptY\eTD
\bTD n. b.\note[Bedarf] \eTD
\eTR
\bTR \bTD Verscheidene: Sloten, deurmechaniek, rempedaal \eTD
\bTD Smeermiddel\eTD
\bTD Universele spray\eTD
\bTD \emptY\eTD
\bTD n. b.\note[Bedarf] \eTD
\eTR
\bTR \bTD Centrale smeerinstallatie \eTD
\bTD Universeel lagervet\eTD
\bTD \liqC{nlgi~2}\eTD
\bTD Total Multis EP~2\eTD
\bTD n. b.\note[Bedarf] \eTD
\eTR
\bTR \bTD Koelsysteem \eTD
\bTD Antivries-/Roestwerend middel\eTD
\bTD TL VW 774 F/G; max. 60\hairspace\% vol.\eTD
\bTD G12+/G12++ (roze/violet)\eTD
\bTD ca. 14 l \eTD
\eTR
\bTR \bTD Hogedrukwaterpomp \eTD
\bTD Motorolie\eTD
\bTD \liqC{SAE 10W-40}; \liqC{api cf – acea e6}\eTD
\bTD Total Rubia TIR 8900 \eTD
\bTD 0,28\,l\eTD
\eTR
\bTR \bTD Airco \eTD
\bTD Koelmiddel\eTD
\bTD + 20 ml POE-olie\eTD
\bTD R 134a\eTD
\bTD 700 g\eTD
\eTR
\bTR \bTD Ruitensproeierinstallatie \eTD
\bTD [nc=2] Water en ruitensproeierconcentraat, \aW{S} zomer, \aW{W} winter; mengverhouding in acht nemen \eTD
\bTD Detailhandel \eTD
\bTD n. b.\note[Bedarf] \eTD
\eTR
\eTABLEbody

\eTABLE

\stop
\footnotetext[Bedarf]{{\it n. b.} naar behoefte, overeenkomstig de betreffende handleiding}
\footnotetext[Produkt]{Van \Boschung\ gebruikte producten. Andere producten die overeenkomen met de specificaties, kunnen eveneens worden gebruikt.}

\stopsection

\page [yes]

\setups [pagestyle:marginless]


\startsection [title={Onderhoud van de dieselmotor},
reference={sec:workshop:vw},
]


\subsection [sSec:vw:diagTool]{On-Board diagnosesysteem}

Het\startregister[index][reg:main:vw]{Onderhoud+Dieselmotor} motorbesturingsapparaat (J623) is uitgerust met een foutgeheugen.
Als er storingen optreden in de bewaakte sensoren resp. componenten, dan worden deze met opgave van de aard van de fout opgeslagen in het foutgeheugen.

Het\index{Dieselmotor+Diagnose} motorbesturingsapparaat maakt na evaluatie van de informatie een onderscheid tussen de verschillende foutklassen en slaat deze op, tot de inhoud van het foutgeheugen wordt verwijderd.

Fouten die slechts {\em sporadisch} optreden, worden weergegeven met de toevoeging \aW{SP}. De oorzaak van sporadische fouten kan \eG\ een loszittend contact of een kortstondige stroomonderbreking zijn. Als een sporadische fout binnen 50 motorstarts niet meer optreedt, dan wordt hij uit het foutgeheugen verwijderd.

Als er fouten zijn herkend die het loopgedrag van de motor beïnvloeden, dan licht op het beeldscherm van de Vpad het controlesymbool \aW{Motordiagnose} \textSymb{vpadWarningEngine1} op.

De opgeslagen fouten kunnen met het voertuigdiagnose, -meet en -informatiesysteem \aW{VAS 5051/B} worden uitgelezen.

Nadat de fouten zijn verholpen, moet het foutgeheugen worden gewist.


\subsubsection[sSec:vw:diagTool:connect]{Inbedrijfstelling van het diagnosesysteem}

\starttextbackground [FC]
\startPictPar
\PMgeneric
\PictPar
Gedetailleerde informatie over het voertuigdiagnosesysteem VAS 5051/B vindt u in de handleiding van het systeem.

U kunt ook andere compatibele diagnosesystemen inzetten, \eG\ \aW{DiagRA}.
\stopPictPar
\stoptextbackground

\page [yes]


\subsubsubsubject{Voorwaarden}

\startitemize
\item De zekeringen moeten in orde zijn.
\item De accuspanning moet meer dan 11,5 V bedragen.
\item Alle elektrische verbuikers moeten zijn uitgeschakeld.
\item De aardverbinding moet in orde zijn.
\stopitemize


\subsubsubsubject{Procedure}

\startSteps
\item Steek de stekker van de diagnoseleiding VAS 5051B/1 in de diagnose-aansluiting.
\item Al naargelang functie ofwel ontsteking inschakelen of motor starten.
\stopSteps

\subsubsubsubject{Bedrijfsmodus selecteren}

\startSteps [continue]
\item Druk op het display op de knop \aW{Voertuig-eigendiagnose}.
\stopSteps


\subsubsubsubject{Voertuigsysteem selecteren}

\startSteps [continue]
\item Druk op het display op de knop \aW{01-Motorelektronica}.
\stopSteps

Op het display verschijnt nu de identificatie van de besturingsapparaten en de codering van het motorbesturingsapparaat.

Als de coderingen niet overeenstemmen, dan moet de codering van de besturingsapparaten worden gecontroleerd.


\subsubsubsubject{Diagnosefunctie selecteren}

Op het display krijgt u alle diagnosefuncties te zien die kunnen worden uitgevoerd.

\startSteps [continue]
\item Druk op het display op de knop voor de gewenste functie.
\stopSteps



\subsection [sSec:vw:faultMemory]{Foutgeheugen}


\subsubsection{Foutgeheugen uitlezen}

\subsubsubject{Werkafloop}

\startSteps
\item Laat de motor stationair lopen.
\item Sluit de VAS 5051/B aan (zie \in{paragraaf}[sSec:vw:diagTool:connect]) en selecteer het motorbesturingsapparaat.
\item Kies de diagnosefunctie \aW{004-Inhoud foutgeheugen}.
\item Kies de diagnosefunctie \aW{004.01-Foutgeheugen afvragen}.
\stopSteps

{\sla Alleen indien de motor niet aanslaat:}

\startitemize [2]
\item Schakel de ontsteking in.
\item Als in het motorbesturingsapparaat geen fout is gearchiveerd, dan verschijnt op het display \aW{0 fouten herkend}.
\item Als in het motorbesturingsapparaat fouten zijn gearchiveerd, dan verschijnen ze op het display onder elkaar.
\item Beëindig de diagnosefunctie.
\item Schakel de ontsteking uit.
\item Verhelp evt. getoonde fouten aan de hand van de foutentabel (zie servicedocumentatie) en wis vervolgens het foutgeheugen.
\stopitemize

\starttextbackground [FC]
\startPictPar
\PMrtfm
\PictPar
Als een fout niet kan worden verwijderd, gelieve u dan te wenden tot de \boschung-klantendienst.
\stopPictPar
\stoptextbackground


\subsubsubject{Statische fouten}

Als in het datageheugen een of meerdere statische fouten voorhanden zijn, gelieve u dan te wenden tot onze Boschung-klantendienst om deze fouten met behulp van de \aW{Begeleide foutopsporing} te verhelpen.


\subsubsubject{Sporadische fouten}

Indien in het foutgeheugen uitsluitend sporadische fouten of aanwijzingen zijn opgeslagen, en er geen storingen van het elektronische voertuigsysteem worden vastgesteld, dan kan het foutgeheugen gewist worden:

\startSteps [continue]
\item Druk nog eens op de toets \aW{Verder} \inframed[strut=local]{>} om naar het testschema te gaan.
\item Om de begeleide foutopsporing te beëindigen drukt u op de toets \aW{Sprong} en dan op \aW{Beëindigen}.
\stopSteps

Nu worden nog eens alle foutgeheugens afgevraagd.

In een venster wordt bevestigd, dat alle sporadische fouten werden verwijderd. Het diagnoseprotocol wordt automatisch (online) verstuurd.

De test van het voertuigsysteem is daarmee beëindigd.


\subsubsection[sSec:vw:faultMemory:errase]{Wissen van het foutgeheugen}

\subsubsubject{Werkafloop}

{\sla Voorwaarden:}

\startitemize [2]
\item Alle fouten moeten verholpen en de oorzaken ervan geëlimineerd zijn.
\stopitemize

\page [yes]


{\sla Procedure:}

\starttextbackground [FC]
\startPictPar
\PMrtfm
\PictPar
Na het verhelpen van de fouten moet het geheugen opnieuw afgevraagd en vervolgens gewist worden:
\stopPictPar
\stoptextbackground

\startSteps
\item Laat de motor stationair lopen.
\item Sluit de VAS 5051/B aan (zie \in{paragraaf}[sSec:vw:diagTool:connect]) en selecteer het motorbesturingsapparaat.
\item Kies de diagnosefunctie \aW{004-Foutgeheugen afvragen}.
\item Kies de diagnosefunctie \aW{004.10-Foutgeheugen wissen}.
\stopSteps

\starttextbackground [FC]
\startPictPar
\PMrtfm
\PictPar
Als het foutgeheugen niet kan worden gewist, dan is er nog een fout voorhanden die geëlimineerd moet worden.
\stopPictPar
\stoptextbackground

\startSteps [continue]
\item Beëindig de diagnosefunctie.
\item Schakel de ontsteking uit.
\stopSteps


\subsection [sSec:vw:lub] {Smering van de dieselmotor}

\subsubsection [ssSec:vw:oilLevel] {Motoroliepeil controleren}

\starttextbackground [FC]
\startPictPar
\PMrtfm
\PictPar
Het\index{Motorolie+-peil} oliepeil mag de \aW{Max.}-markering in geen geval overschrijden. Anders bestaat het\index{Vulstand+Motorolie} gevaar van katalysatorschade.
\stopPictPar
\stoptextbackground

\startSteps
\item Motor afzetten en minstens 3 minuten wachten, opdat de olie terug kan stromen in de oliebak.
\item Meetstaaf eruit trekken en schoon afvegen; de staaf weer tot aan de aanslag erin schuiven.
\item Staaf weer eruit trekken en het oliepeil beoordelen:

\startfigtext[right][fig:vw:gauge]{Aflezen van het oliepeil}
{\externalfigure[VW_Oil_Gauge][width=50mm]}
\startitemize [A]
\item Maximale vulstand; er mag geen olie worden bijgevuld.
\item Voldoende vulstand; er {\em kan} olie worden bijgevuld tot aan het bereiken van de markering \aW{A}.
\item Niet voldoende vulstand; er {\em moet} olie worden bijgevuld, tot de vulstand zich in het bereik \aW{B} bevindt.
\stopitemize
{\em Bij een vulstand boven de markering \aW{A} bestaat het gevaar van katalysatorschade.}
\stopfigtext
\stopSteps


\subsubsection [ssSec:vw:oilDraining] {Motorolieverversing}

\starttextbackground [FC]
\startPictPar
\PMrtfm
\PictPar
Het motoroliefilter van de S2 is staand gemonteerd. Dat betekent dat het filter {\em vóór} de olieverversing moet worden vervangen. Door het filterelement eruit te nemen wordt er een klep geopend, en de olie in het filterhuis stroomt automatisch naar het carter.
\stopPictPar
\stoptextbackground

\startSteps
\item Zet een geschikte\index{Dieselmotor+Olieverversing} opvangbak onder de motor.
\item Olieaftapplug eruit schroeven\index{Motorolie+-verversing} en de olie laten weglopen.
\stopSteps

\starttextbackground [FC]
\startPictPar
\PMrtfm
\PictPar
Zorg ervoor dat de hele opvangbak de totale hoeveelheid oude olie kan opvangen.
De vereiste oliespecificatie en vulhoeveelheid vindt u in \in{paragraaf}[sec:liqquantities].

De olieaftapplug is voorzien van een vast aangebrachte afdichtring. De olieaftapplug moet daarom altijd worden vervangen
\stopPictPar
\stoptextbackground

\startSteps [continue]
\item Schroef een nieuwe olieaftapplug met afdichtring erin (\TorqueR 30 Nm).
\item Motorolie met geschikte specificatie erin gieten (zie \in{paragraaf}[sec:liqquantities]).
\stopSteps


\subsubsection [ssSec:vw:oilFilter] {Motoroliefilter vervangen}

\starttextbackground [FC]
\startPictPar
\PMrtfm
\PictPar
\startitemize [1]
\item Neem\index{Dieselmotor+Oliefilter} de voorschriften voor verwerking en recyclage in acht.
\item Vervang\index{Oliefilter+Dieselmotor} het filter {\em vóór} de olieverversing (zie \in{paragraaf}[ssSec:vw:oilDraining]).
\item Olie vóór de montage de afdichting van het nieuwe filter licht in.
\stopitemize
\stopPictPar
\stoptextbackground

\startfigtext[right][fig:vw:oilFilter]{Oliefilter}
{\externalfigure[VW_OilFilter_03][width=50mm]}
\startSteps
\item Deksel \Lone\ van het filterhuis eraf schroeven met een geschikte schroefsleutel.
\item Reinig de afdichtingsvlakken van deksel en filterhuis.
\item Vervang het filterelement \Lthree.
\item Vervang de O-ringen \Ltwo\ en \Lfour.
\item Deksel weer op het filterhuis schroeven (\TorqueR 25 Nm).
\stopSteps



%\subsubsubject{Données techniques}
%
%
%\hangDescr{Couple de serrage du couvercle:} \TorqueR 25 Nm.
%
%\hangDescr{Huile moteur prescrite:} Selon tableau \atpage[sec:liqquantities].
%% NOTE: Redundant [tf]

\stopfigtext



\subsubsection [ssSec:vw:oilreplenish] {Motorolie bijvullen}

\starttextbackground [FC]
\startPictPar
\PMrtfm
\PictPar
\startitemize [1]
\item Maak\index{Motorolie} {\em alvorens} de dop eraf te nemen de vulopening schoon met een doek.
\item Vul\index{Dieselmotor+Olie bijvullen} uitsluitend olie bij die overeenkomt met de voorgeschreven specificatie.
\item Vul stapsgewijs bij met kleine hoeveelheden.
\item Wacht om overvullen te vermijden na elk bijvullen even, opdat de olie tot aan de markering in het carter kan stromen (zie \in{paragraaf}[ssSec:vw:oilLevel]).
\stopitemize
\stopPictPar
\stoptextbackground

\startfigtext[right][fig:vw:oilFilter]{Olie bijvullen}
{\externalfigure[s2_bouchonRemplissage][width=50mm]}
\startSteps
\item Trek de oliemeetstaaf ongeveer 10 cm eruit, zodat de lucht bij het bijvullen kan ontsnappen.
\item Open de vulopening.
\item Vul olie bij met inachtneming van de bovenstaande voorschriften.
\item Sluit de vulopening zorgvuldig.
\item Start de motor.
\item Voer een vulstandcontrole uit. (Zie \in{paragraaf}[ssSec:vw:oilLevel].)
\stopSteps

\stopfigtext


\subsection [sSec:vw:fuel] {Brandstoftoevoersysteem}

\subsubsection [ssSec:vw:fuelFilter] {Brandstoffilter vervangen}

\starttextbackground [FC]
\startPictPar
\PMrtfm
\PictPar
\startitemize [1]
\item Neem\index{Dieselmotor+Brandstoffilter} de wettelijke voorschriften voor verwerking en recyclage van giftig afval in acht.
\item Neem niet de brandstofleidingen van het bovendeel van het filter af.
\item Oefen geen trekkracht uit op de bevestigingspunten van de brandstofleidingen; anders kan het bovendeel van het filter beschadigd raken.
\stopitemize
\stopPictPar
\stoptextbackground

\startfigtext[right][fig:vw:oilFilter]{Brandstoffilter}
{\externalfigure[s2_fuelFilter_location][width=50mm]}

{\sla Voorbereiding:}

Het\index{Brandstoffilter} brandstoffilterhuis is vóór de motor, aan de rechterkant van het chassis bevestigd.
Verwijder de beide bevestigingsschroeven met een 10mm steeksleutel en een 10mm ringsleutel.

\stopfigtext


\page [yes]

\setups [pagestyle:normal]

{\sla Procedure:}

\startLongsteps
\item Verwijder alle schroeven van het bovendeel van het filter. Neem het bovendeel van het filter eraf.
\stopLongsteps

\starttextbackground [FC]
\startPictPar
\PMrtfm
\PictPar
Til het bovendeel eraf. Indien vereist zet u hiervoor een hoekschroevendraaier aan aan de montagegroef (\in{\LAa, afb.}[fig:fuelfilter:detach]) en licht u het bovendeel eruit.
\stopPictPar
\stoptextbackground

\placefig [margin] [fig:fuelfilter:detach]{Neem het brandstoffilters weg}
{\externalfigure[fuelfilter:detach]}

\placefig [margin] [fig:fuelfilter:explosion]{Brandstoffilter}
{\externalfigure[fuelfilter:explosion]}

\startLongsteps [continue]
\item Trek het filterelement uit het onderdeel van het filter.
\item Neem de afdichting (\in{\Ltwo, afb.}[fig:fuelfilter:explosion]) van het bovendeel van het filter af.
\item Reinig onder- en bovendeel van het filter zorgvuldig.
\item Zet een nieuw filterelement in het onderdeel van het filter.
\item Bevochtig een nieuwe afdichting (\in{\Ltwo, afb.}[fig:fuelfilter:explosion]) met wat brandstof en zet hem in het bovendeel.
\item Zet het bovendeel passend op het onderdeel van het filter en druk het gelijkmatig vast, zodat het bovendeel over de hele omtrek gelijkmatig erop ligt.
\item Schroef boven- en onderdeel met alle schroeven {\em handvast} weer aaneen. Draai dan kruiselings alle schroeven aan met het voorgeschreven aandraaimoment (\TorqueR 5 Nm).
\stopLongsteps

% \subsubsubject{Données techniques}
%
% \hangDescr{Couple de serrage des vis de fixation du couvercle:} \TorqueR 5 Nm.
%% NOTE: redundant [tf]

\startLongsteps [continue]
\item Schakel de ontsteking in om het systeem te ontluchten; start de motor en laat hem 1 tot 2 minuten draaien met stationair toerental.
\item Wis het foutgeheugen zoals beschreven op \atpage[sSec:vw:faultMemory:errase].
\stopLongsteps


\subsection [sSec:vw:cooling] {Koelsysteem}

\starttextbackground [FC]
\startPictPar
\PMrtfm
\PictPar
\startitemize [1]
\item Alleen\index{Dieselmotor+Koeling} koelmiddelen met de voorgeschreven specificatie (zie tabel \atpage[sec:liqquantities]) gebruiken.
\item Om\index{Koelmiddel} bescherming tegen bevriezing en corrosie te garanderen mag het koelmiddel uitsluitend met gedestilleerd water en conform onderstaande tabel worden verdund.
\item Vul de koelmiddelkring nooit met water, aangezien bescherming tegen bevriezing en corrosie hierdoor negatief zouden worden beïnvloed.
\stopitemize
\stopPictPar
\stoptextbackground


\subsubsection [sSec:vw:coolingLevel] {Koelmiddelpeil}

\placefig [margin] [fig:coolant:level] {Koelmiddelpeil}
{\externalfigure[coolant:level]}


\placefig [margin] [fig:refractometer] {Refractometer VW T 10007}
{\externalfigure[coolant:refractometer]}

\placefig [margin] [fig:antifreeze] {Controle van de antivriesdichtheid}
{\externalfigure[coolant:antifreeze]}


\startSteps
\item Til de vuilcontainer op en breng de veiligheidssteunen aan.
\item Stel\index{Vulstand+Koelmiddel} de vulstand van het koelmiddel in het expansievat vast: Het moet boven de \aW{min}-markering staan.
\stopSteps

\start
\define [1] \TableSmallSymb {\externalfigure[#1][height=4ex]}
\define\UC\emptY
\pagereference[page:table:liquids]


\setupTABLE [frame=off,style={\ssx\setupinterlinespace[line=.86\lH]},background=color,
option=stretch,
split=repeat]
\setupTABLE [r] [each] [topframe=on,
framecolor=TableWhite,
% rulethickness=.8pt
]

\setupTABLE [c] [odd] [backgroundcolor=TableMiddle]
\setupTABLE [c] [even] [backgroundcolor=TableLight]
\setupTABLE [r] [first] [topframe=off,style={\bfx\setupinterlinespace[line=.95\lH]},
% backgroundcolor=TableDark
]
\setupTABLE [r] [2][framecolor=black]

\bTABLE

\bTABLEhead
\bTR
\bTC Antivries tot … \eTC
\bTC Aandeel G12\hairspace ++\eTC
\bTC Vol. antivriesmiddel \eTC
\bTC Vol. gedestilleerd water \eTC
\eTR
\eTABLEhead

\bTABLEbody
\bTR \bTD \textminus 25\,°C \eTD
\bTD 40\hairspace\% \eTD
\bTD 3,8\,l \eTD
\bTD 4,2\,l \eTD
\eTR
\bTR \bTD \textminus 35\,°C \eTD
\bTD 50\hairspace\% \eTD
\bTD 4,0\,l \eTD
\bTD 4,0\,l \eTD
\eTR
\bTR \bTD \textminus 40\,°C \eTD
\bTD 60\hairspace\% \eTD
\bTD 4,2\,l \eTD
\bTD 3,8\,l \eTD
\eTR
\eTABLEbody

\eTABLE
\stop

\adaptlayout [height=+20pt]
\subsubsection [sSec:vw:coolingFreeze] {Koelmiddelpeil}

Controleer\index{Antivriesdichtheid} de antivriesdichtheid met behulp van een geschikte refractometer (zie \in{afb.}[fig:refractometer]: VW T 10007).
Let op schaal 1: G12\hairspace ++ (zie \in{afb.}[fig:antifreeze]).

\page [yes]


\subsection [sSec:vw:airFilter] {Luchttoevoer}

Het luchtfilter is toegankelijk via de achterste onderhoudsdeur aan de rechterkant van het voertuig (zie \in{afb.}[fig:airFilter]).

\placefig [margin] [fig:airFilter] {Luchtfilter van de motor}
{\externalfigure[vw:air:filter]
\noteF
\startLeg
\item Veiligheidsbeugel
\item Onderdeel van het huis
\item Ontluchtingsopening
\item Druksensor
\stopLeg}


\subsubsubject{Inzetvoorwaarden}

Een veegvoertuig wordt vaak ingezet in omgevingen met veel stofvorming. Om deze reden is het noodzakelijk om het luchtfilter wekelijks te controleren en te reinigen. Zie ook \about[table:scheduleweekly], \atpage[table:scheduleweekly]. Indien vereist moet het luchtfilter worden vervangen.


\subsubsubject{Autodiagnostic}

De aanzuigleiding bezit een druksensor (\Lfour, \in{afb.}[fig:airFilter]), waarmee laadverliezen\footnote{Verlaagde luchtdoorstroming op grond van verlaagde luchtdoorlaatbaarheid van het filter.} door het filter kunnen worden vastgesteld.
Wanneer het luchtfilter verstopt is, dan licht op het Vpad-beeldscherm het controlesymbool \textSymb{vpadWarningFilter} op en wordt de foutmelding \VpadEr{851} geregistreerd.


\subsubsubject{Instandhouden/Vervangen}

\startSteps
\item Trek de veiligheidsbeugel \Lone naar beneden (\in{afb.}[fig:airFilter]).
\item Draai het onderdeel van het huis \Ltwo met de klok mee en neem het eraf.
\item Neem het filterelement weg en controleer het. Vervang het indien vereist.
\item Reinig het inwendige van het filter en zet het luchtfilter in omgekeerde volgorde weer ineen.
\stopSteps

\page [yes]


\subsection [sSec:vw:belt] {Poly-V-riem}

De\index{Dieselmotor+Poly-V-riem} poly-V-riem draagt de beweging van het schijfvliegwiel van de krukas over op de dynamo en de compressor van de airco (optionele uitrusting).
Een\index{Poly-V-riem} spanelement in het laatste segment (tussen dynamo en krukas) houdt de riem onder spanning.


\subsubsection [sSec:belt:change] {Vervangen van de poly-V-riem}

\placefig [margin] [fig:belt:tool] {Afsteekdoorn VW T 10060 A}
{\externalfigure[vw:belt:tool]}

\placefig [margin] [fig:belt:overview] {Spanelement}
{\externalfigure[vw:belt:overview]}

\placefig [margin] [fig:belt:tens] {Aanzetplaats van de afsteekdoorn}
{\externalfigure[vw:belt:tens]}


\subsubsubject{Met airco-compressor}


{\sla Benodigd speciaal gereedschap:}

Afsteekdoorn \aW{VW T 10060 A} om het spanelement vast te houden.

\startSteps
\item Markeer de looprichting van de poly-V-riem.
\item Zwenk met een haaks gebogen ringsleutel de arm van het spanelement met de klok mee (\in {afb.}[fig:belt:overview]).
\item Breng de boringen (zie pijlen, \in {afb.}[fig:belt:tens]) in overlapping en arrêteer het spanelement met de afsteekdoorn.
\item Neem de poly-V-riem eraf.
\stopSteps

Der Inbouw van de poly-V-riem gebeurt in omgekeerde volgorde.

\starttextbackground [FC]
\startPictPar
\PMrtfm
\PictPar
\startitemize [1]
\item Let op de looprichting van de poly-V-riem.
\item Let op de correcte zitting van de riem in de riemschijven.
\item Start de motor en controleer de riemloop.
\stopitemize
\stopPictPar
\stoptextbackground


\subsubsubject{Zonder airco-compressor}

{\sla Benodigd materiaal:}

Reparatiekit, bestaande uit reparatiehandleiding, poly-V-riem en speciaal gereedschap.\footnote{Zie onderdelencatalogus, onder \aW{Onderhoudsdelen}.}

\startSteps
\item Snij de poly-V-riem door.
\item Volg de verdere stappen in de reparatiehandleiding.
\stopSteps

\starttextbackground [FC]
\startPictPar
\PMrtfm
\PictPar
\startitemize [1]
\item Let op de correcte zitting van de riem in de riemschijven.
\item Start de motor en controleer de riemloop.
\stopitemize
\stopPictPar
\stoptextbackground


\subsubsection [sSec:belt:tens] {Vervangen van het spanelement}

{\sla Alleen voor uitvoering met airco-compressor}

\blank [medium]

\placefig [margin] [fig:belt:tens:change] {Vervangen van het spanelement}
{\externalfigure[vw:belt:tens:change]
\noteF
\startLeg
\item Spanelement
\item Borgschroef
\stopLeg

{\bf Aandraaimoment}

Borgschroef:

\TorqueR 20 Nm\:+ ½ omdraaiing (180°).}

\startSteps
\item Demonteer de poly-V-riem zoals beschreven (zie \atpage[sSec:belt:change]).
\item Demonteer perifere delen (al naargelang uitrusting).
\item Schroef de borgschroef eruit (\in{\Ltwo, afb.}[fig:belt:tens:change]).
\stopSteps

De inbouw van het spanelement gebeurt in omgekeerde volgorde.

\starttextbackground [FC]
\startPictPar
\PMrtfm
\PictPar
\startitemize [1]
\item Gebruik na de inbouw absoluut een nieuwe borgschroef.
\item Aandraaimoment: Zie \in{afb.}[fig:belt:tens:change].
\stopitemize
\stopPictPar
\stoptextbackground

\stopregister[index][reg:main:vw]

\stopsection

\page[yes]


\setups[pagestyle:marginless]


\startsection[title={Hydraulische installatie},
reference={sec:hydraulic}]

\starttextbackground [FC]
% \startfiguretext[left,none]{}
% {\externalfigure[toni_melangeur][width=30mm]}

\startSymPar
\externalfigure[toni_melangeur][width=4em]
\SymPar
\textDescrHead{Recycleren van bedrijfsstoffen}
Gebruikte bedrijfsstoffen en smeermiddelen mogen niet in de natuur verwerkt noch verbrand worden.

Gebruikte smeermiddelen mogen niet naar het rioleringsnetwerk noch naar de natuur geleid en niet met het huisvuil verwerkt worden.

Gebruikte smeermiddelen mogen niet met andere vloeistoffen worden gemengd, aangezien het gevaar bestaat dat er giftige of moeilijk te verwerken stoffen ontstaan.
\stopSymPar
\stoptextbackground
\blank [big]

% \starthangaround{\PMgeneric}
% \textDescrHead{Qualification du personnel}
% Toute intervention sur l’installation hydraulique de votre véhicule ne peut être réalisée que par une personne dument qualifiée, ou par un service reconnu par \boschung.
% \stophangaround
% \blank[big]

\startSymList
\PHgeneric
\SymList
\textDescrHead{Zuiverheid} De hydraulische installatie reageert zeer gevoelig op onzuiverheden in de olie. Daarom is het erg belangrijk om in een absoluut zuivere omgeving te werken.
\stopSymList

\startSymList
\PHhot
\SymList
\textDescrHead{Spatgevaar}
Vóór werkzaamheden aan de hydraulische installatie van de \sdeux\ moet de restdruk in de betreffende hydraulische kring worden afgelaten. Hete oliespetters kunnen brandwonden veroorzaken.
\stopSymList

\startSymList
\PHhand
\SymList
\textDescrHead{Pletgevaar}
De vuilcontainer moet absoluut neergelaten of mechanisch met de veiligheidssteunen beveiligd zijn, voordat er werkzaamheden worden uitgevoerd aan de hydraulische installatie van de \sdeux.
\stopSymList

\startSymList
\PImano
\SymList
\textDescrHead{Drukmeting}
Om de hydraulische druk te meten brengt u een manometer aan op een van de \aW{mini meet}aansluitingen van de kring. Let erop dat de manometer een geschikt meetbereik bezit.
\stopSymList

\page [yes]

\setups[pagestyle:normal]

\subsection{Onderhoudsintervallen}

\start

\setupTABLE [frame=off,
style={\ssx\setupinterlinespace[line=.93\lH]},
background=color,
option=stretch,
split=repeat]
\setupTABLE [r] [each] [
topframe=on,
framecolor=white,
backgroundcolor=TableLight,
% rulethickness=.8pt,
]

% \setupTABLE [c] [odd] [backgroundcolor=TableMiddle]
% \setupTABLE [c] [even] [backgroundcolor=TableLight]
\setupTABLE [c] [1][ % width=30mm,
style={\bfx\setupinterlinespace[line=.93\lH]},
]
\setupTABLE [r] [first] [topframe=off,
style={\bfx\setupinterlinespace[line=.93\lH]},
backgroundcolor=TableMiddle,
]
% \setupTABLE [r] [2][style={\ssBfx\setupinterlinespace[line=.93\lH]}]


\bTABLE

\bTABLEhead
\bTR\bTD Onderhoudswerk \eTD\bTD Interval \eTD\eTR
\eTABLEhead

\bTABLEbody
\bTR\bTD Controleren op lekkages \eTD\bTD Dagelijks \eTD\eTR
\bTR\bTD Hydraulische oliepeil controleren \eTD\bTD Dagelijks \eTD\eTR
\bTR\bTD Toestand van de hydraulische leidingen/slangen controleren; evt. vervangen \eTD\bTD 600 h / 12 maanden \eTD\eTR
\bTR\bTD Hydraulische retour- en aanzuigfilter vervangen \eTD\bTD 600 h / 12 maanden \eTD\eTR
\bTR\bTD Spoelkernen van de magneetkleppen smeren met kopervet \eTD\bTD 600 h / 12 maanden \eTD\eTR
\bTR\bTD Hydraulische olie verversen \eTD\bTD 1200 h / 24 maanden \eTD\eTR
\eTABLEbody
\eTABLE
\stop


\subsection[niveau_hydrau]{Vulstand}

\placefig[margin][fig:hydraulic:level]{Vulstand van de hydraulische vloeistof}
{\externalfigure[hydraulic:level]
\noteF
\startLeg
\item Optimale vulstand
\stopLeg}

Een transparant kijkglas\index{Vulstand+Hydraulische vloeistof}\index{Onderhoud+Hydraulische installatie} maakt een controle van het hydraulische oliepeil mogelijk.
Als het hydraulische oliepeil is gedaald, dan moet de oorzaak worden vastgesteld voordat er weer wordt bijgevuld. Houd u aan de voorgeschreven verversingsintervallen (tabel boven) en specificaties voor de hydraulische vloeistof (tabel \at{pagina}[sec:liqquantities]).


\subsubsection{Hydraulische vloeistof bijvullen}

Vul hydraulische vloeistof bij,
tot het midden kijkglas volledig gedekt.
Start de motor en vul evt. wat bij, tot de vereiste vulstand is bereikt.


\subsection{Hydraulische vloeistof vervangen}

Vulhoeveelheid en vereiste specificaties van de hydraulische vloeistof vindt u in de tabel op \at{pagina}[sec:liqquantities].

\startSteps
\item Open de bijvulopening van de hydraulische tank.
\item Maak de tank leeg met behulp van een olie-afzuiger of verwijder de aftapplug.

De aftapplug bevindt zich onder aan de hydraulische tank, vóór het linker achterwiel (\in{afb.}[fig:hydraulic:fluidDrain]).
\item Vul hydraulische vloeistof bij, tot het midden kijkglas volledig gedekt.
Start de motor en vul evt. wat bij, tot de vereiste vulstand is bereikt.
\stopSteps

\placefig[margin][fig:hydraulic:fluidDrain]{Aftapplug}
{\externalfigure[hydraulic:fluidDrain]}


\placefig[margin][fig:hydraulic:returnFilter]{Hydraulisch filter}
{\externalfigure[hydraulic:returnFilter]}

\subsection[filtres:nettoyage]{Retour- en aanzuigfilter}

\startSteps
\item Til de vuilcontainer op en breng de veiligheidssteunen aan.
\item Neem het deksel van het filter op de hydraulische tank eraf (\in{afb.}[fig:hydraulic:returnFilter]).
\item Vervang\index{Oliefilter+Hydraulisch} het filterelement door een nieuw.
\item Bevochtig een nieuwe O-ringafdichting met wat hydraulische vloeistof en breng hem aan.
\item Schroef het deksel met twee handen weer erop (\TorqueR ca. 20 Nm).
\stopSteps

\page [yes]


\subsection[sec:solenoid]{Smeren van de magneetkleppen}
\placefig[margin][graissage_bobine]{Smeren van de magneetkleppen}
{\externalfigure[graissage_bobine][M]
\noteF
\startLeg
\item Spoel van de magneetklep
\item Spoelkern
\stopLeg}

Vocht en zoutresten die terechtkomen in de kern van de elektromagnetische spoelen, leiden tot corroderen van de kernen. De spoelkernen moeten eenmaal per jaar met kopervet worden gesmeerd. Het vet moet corrosie-, water- en temperatuurbestendig tot 50 °C zijn:
\startSteps
\item Demonteer de spoel van de magneetklep (\in{\Lone, afb.}[graissage_bobine]).
\item Smeer de kern (\in{\Ltwo, afb.}[graissage_bobine]) met het voorgeschreven speciale vet en monteer de spoel weer.
\stopSteps


\subsection{Vervangen van de slangen}

Het beschermende rubber\index{Slangen+Vervangingsintervallen} en het versterkende weefsel van de slangen zijn onderhevig aan een natuurlijke veroudering. Daarom moeten de slangen van de hydraulische installatie absoluut binnen de voorgeschreven intervallen worden vervangen, ook al is er geen {\em zichtbare} schade herkenbaar.

Let erop dat de slangen correct aan het voertuig worden gekoppeld, om een voortijdige slijtage door wrijving uit te sluiten. Ze moeten voldoende afstand tot andere componenten bezitten, zodat schade door wrijving en trillingen wordt verhinderd.

\stopsection

\page [yes]

\setups [pagestyle:bigmargin]


\startsection[title={Remsysteem},
reference={sec:brake}]

\placefig[margin][fig:brake:rear]{Trommelrem}
{\startcombination [1*2]
{\externalfigure[brake:wheelHub]}{\slx Naaf van het achterwiel}
{\externalfigure[brake:drum]}{\slx Mechanisme en remsets}
\stopcombination}

De remtrommels \Lfour\ moeten bij elk regulier onderhoud gedemonteerd, het remmechanisme \Lseven\ moet gereinigd en de remsets \Lfive, \Lsix\ moeten aan een zichtcontrole onderworpen worden (\in{afb.}[fig:brake:rear]).


\subsubject {Demontage}

\startSteps
\item Rijd het voertuig op een geschikte hefbrug en til de wielen op.
\item Neem de wielen eraf.
\stopSteps


{\sla Demontage van de remmen van de voorwielen}

\startSteps [continue]
\item Demonteer de remtrommels\Lfour.
\stopSteps

{\sla Demontage van de remmen van de achterwielen}

\startSteps [continue]
\item Neem de afdekking \Lone\ van de naaf af.
\item Verwijder de schroef \Ltwo\ en neem het tussenstuk eraf.
\item Schroef de naafmoer \Lthree\ met een steeksleutel eraf.
\item Neem de naaf met de remtrommel eraf.
\stopSteps


\subsubject {Nieuwe montage}

Monteer de remtrommels weer in omgekeerde volgorde. Draai de moeren van de naven van de achterwielen \Lthree\ aan met het voorgeschreven aandraaimoment van 190 Nm.

\stopsection

\page [yes]

\setups [pagestyle:normal]


\startsection[title={Controle en onderhoud van de banden},
reference={sec:pneumatiques}]

De banden\index{Banden+Onderhoud} moeten altijd in foutloze toestand zijn, opdat ze hun beide hoofdfuncties kunnen vervullen: goede grip en foutloos remgedrag. Ontoelaatbaar hoge slijtage en verkeerde spanning, met name te lage druk, zijn belangrijke factoren bij ongevallen.


\subsection{Veiligheidsrelevante punten}

\subsubsection{Controle op slijtage}

De bandenslijtage moet worden gecontroleerd aan de hand van de slijtage-indicatoren in een profielgroef (\in{afb.}[pneususure]).
Afwijkingen aan de band en de oorzaken daarvan kunnen met een zichtcontrole worden vastgesteld:

\placefig[margin][pneususure]{Controle op slijtage}
{\Framed{\externalfigure[pneusUsure][M]}}

\placefig[margin][pneusdomages]{Beschadigde banden}
{\Framed{\externalfigure[pneusDomages][M]}}

\startitemize
\item Slijtage aan de zijkanten van het loopvlak: spanning te laag.
\item Versterkte slijtage in het midden: spanning te hoog.
\item Asymmetrische slijtage aan de zijkanten van de band: vooras (spoor, asgeometrie) verkeerd ingesteld.
\item Scheuren in het loopvlak: banden te oud; het rubber van de band wordt mettertijd harder en gebarsten (\in{afb.}[pneusdomages]).
\stopitemize

\starttextbackground[CB]
\startPictPar
\PHgeneric
\PictPar
\textDescrHead{Risico´s door versleten banden}
Een versleten band vervult zijn functies niet meer, met name de afvoer van water en modder; de remweg wordt langer en het rijgedrag verslechtert. Een versleten band slipt sneller, vooral bij natheid. Het gevaar dat de band zijn grip verliest, neemt toe.
\stopPictPar
\stoptextbackground


\subsubsection{Bandenspanning}

De voorgeschreven bandenspanning staat vermeld op het typeplaatje, vooraan aan de console aan de passagierskant (zie \atpage [sec:plateWheel]).

Zelfs\index{Banden+Spanning} wanneer de banden in goede staat zijn, verliezen ze mettertijd meer of minder snel lucht (hoe meer er met het voertuig wordt gereden, des te hoger het drukverlies). Daarom moet de bandenspanning maandelijks, bij koude banden, gecontroleerd worden. Wanneer u de spanning controleert bij warme banden, dan moet u 0,3 bar optellen bij de voorgeschreven druk.

\start
\setupcombinations[M]
\placefig[margin][pneuspression]{Bandenspanning}
{\Framed{\externalfigure[pneusPression][M]}
\noteF
\startLeg
\item Correcte spanning
\item Te hoge spanning
\item Te lage spanning
\stopLeg
De voorgeschreven bandenspanning staat vermeld op het typeplaatje van de wielen, in de bestuurderscabine aan de passagierskant.}
\stop

\starttextbackground[CB]
\startPictPar
\PHgeneric
\PictPar
\textDescrHead{Gevaren door te lage bandenspanning}
Een band kan scheuren, als de spanning te laag is. De band wordt meer samengedrukt, als hij niet voldoende opgepompt of als het voertuig overbeladen is. Het rubber wordt daardoor heet en delen van een band kunnen in een kromme loskomen.
\stopPictPar
\stoptextbackground

\stopsection

\page [yes]

\setups[pagestyle:marginless]


\startsection[title={Chassis},
reference={main:chassis}]

\subsection{Veiligheidsrelevante bevestigingen van componenten}

Bij elk onderhoud moet de correcte zitting van veiligheidsrelevante bevestigingsschroeven van bepaalde componenten gecontroleerd worden, inclusief de controle van het voorgeschreven aandraaimoment. Dit geldt met name voor de schemelstuurinrichting en de assen.

\blank [big]

\startfigtext [left] [fig:frontAxle:fixing] {Vooras}
{\externalfigure [frontAxle:fixing]}
{\sla Bevestigingen van de vooras}
\startLeg
\item Bevestiging van het veerblad: \TorqueR 150 Nm
\item Bevestiging van de trekeenheden: \TorqueR 78 Nm
\stopLeg

{\sla Bevestigingen van de achteras}
\startLeg
\item Bevestiging van het veerblad: \TorqueR 150 Nm
\stopLeg

\stopfigtext

\start

\setupTABLE [frame=off,style={\ssx\setupinterlinespace[line=.93\lH]},background=color,
option=stretch,
split=repeat]

\setupTABLE [r] [each] [topframe=on,
framecolor=white,
% rulethickness=.8pt
]

\setupTABLE [c] [odd] [backgroundcolor=TableMiddle]
\setupTABLE [c] [even] [backgroundcolor=TableLight]
\setupTABLE [c] [1][style={\bfx\setupinterlinespace[line=.93\lH]}]
\setupTABLE [r] [first] [topframe=off,style={\bfx\setupinterlinespace[line=.93\lH]},
]
% \setupTABLE [r] [2][style={\bfx\setupinterlinespace[line=.93\lH]}]


\bTABLE

\bTABLEhead
\bTR [backgroundcolor=TableDark] \bTD [nc=3] Aandraaimomenten \eTD\eTR
% \bTR\bTD Position \eTD\bTD Type de vis \eTD\bTD Couple \eTD\eTR
\eTABLEhead

\bTABLEbody
\bTR\bTD Aandrijfmotoren links/rechts \eTD\bTD M12\:×\:35 8.8 \eTD\bTD 78 Nm \eTD\eTR
%% NOTE @Andrew: das sind Hydraulikmotoren
\bTR\bTD Werkpomp \eTD\bTD M16\:×\:40 100 \eTD\bTD 330 Nm \eTD\eTR
\bTR\bTD Aandrijfpomp \eTD\bTD M12\:×\:40 100 \eTD\bTD 130 Nm \eTD\eTR
\bTR\bTD Veerbladen voor/achter \eTD\bTD M16\:×\:90/160 8.8 \eTD\bTD 150 Nm \eTD\eTR
% \bTR\bTD Fixation du système oscillant \eTD\bTD M12\:×\:40 8.8 \eTD\bTD 78 Nm \eTD\eTR
\bTR\bTD Bevestiging van de vuilcontainer \eTD\bTD M10\:×\:30 Verbus Ripp 100 \eTD\bTD 80 Nm \eTD\eTR
\bTR\bTD Wielmoeren \eTD\bTD M14\:×\:1,5 \eTD\bTD 180 Nm \eTD\eTR
\bTR\bTD Bevestiging van de frontbezem \eTD\bTD M16\:×\:40 100 \eTD\bTD 180 Nm \eTD\eTR
\eTABLEbody
\eTABLE
\stop


\stopsection

\page [yes]

\startmode [main:centralLub]

\startsection[title={Centrale smeerinstallatie},
reference={main:graissageCentral}]


\subsection{Beschrijving van de besturingsmodule}

De \sdeux\ kan met\index{Centrale smeerinstallatie} een centrale smeerinstallatie worden uitgerust (optie). De centrale smeerinstallatie voedt elk smeerpunt van het voertuig in regelmatige intervallen met smeermiddel.

\startfigtext [left] [vogel_affichage] {Indicatiemodule}
{\externalfigure[vogel_base2][W50]}
\blank
\startLeg
\item 7 tekens tellend display: waarden en operationele toestand
\item \LED: Systeem in pauze (standby bedrijf)
\item \LED: Pomp in bedrijf
\item \LED: Besturing van het systeem door cyclusschakelaar
\item \LED: Bewaking van het systeem door drukschakelaar
\item \LED: Foutmelding
\item Toetsen voor beeldverschuiving:
\startLeg [R]
\item Display activeren
\item Waarden weergeven
\item Waarden wijzigen
\stopLeg
\item Toets om de bedrijfsmodus te wisselen; bevestiging van de waarden
\item Triggeren van een tussentijdse smeercyclus
\stopLeg
\stopfigtext

De centrale smeerinstallatie omvat de smeermiddelpomp, het doorzichtige smeermiddelreservoir aan de linkerkant van het chassis en de besturingsmodule in de centrale elektronica.
% \blank
\page [yes]


\subsubsubject{Display en toetsen van de besturingsmodule}

\start

\setupTABLE [frame=off,style={\ssx\setupinterlinespace[line=.93\lH]},background=color,
option=stretch,
split=repeat]

\setupTABLE [r] [each] [topframe=on,
framecolor=white,
% rulethickness=.8pt
]

\setupTABLE [c] [odd] [backgroundcolor=TableMiddle]
\setupTABLE [c] [even] [backgroundcolor=TableLight]
\setupTABLE [c] [1][width=9mm,style={\bfx\setupinterlinespace[line=.93\lH]}]
\setupTABLE [r] [first] [topframe=off,style={\bfx\setupinterlinespace[line=.93\lH]},
]
% \setupTABLE [r] [2][style={\bfx\setupinterlinespace[line=.93\lH]}]


\bTABLE
\bTABLEhead
% \bTR [backgroundcolor=TableDark]
% \bTD [nc=4] Display en toetsen van de besturingsmodule \eTD\eTR
\bTR\bTD Pos. \eTD
\bTD \LED \eTD\bTD Weergavemodus \eTD
\bTD Programmeermodus \eTD\eTR
\eTABLEhead

\bTABLEbody
\bTR\bTD 2 \eTD
\bTD Operationele toestand {\em Pauze}\hskip.5em\null \eTD
\bTD De installatie bevindt zich in standby\hskip.5em\null \eTD % bedrijf
\bTD De pauzetijd kan worden gewijzigd \eTD\eTR
\bTR\bTD 3 \eTD
\bTD Operationele toestand {\em Contact} \eTD
\bTD De pomp werkt \eTD
\bTD De werktijd kan worden gewijzigd \eTD\eTR
\bTR\bTD 4 \eTD
\bTD Systeemcontrole {\em CS} \eTD
\bTD Met de externe cyclusschakelaar \eTD
\bTD De controlemodus kan gedeactiveerd of gewijzigd worden \eTD\eTR
\bTR\bTD 5 \eTD
\bTD Systeemcontrole {\em PS} \eTD
\bTD Met de externe drukschakelaar \eTD
\bTD De controlemodus kan gedeactiveerd of gewijzigd worden \eTD\eTR
\bTR\bTD 6 \eTD
\bTD Storing {\em Fault} \eTD
\bTD [nc=2] Er is sprake van een functiestoring. De oorzaak wordt getoond in de vorm van een foutcode, nadat de toets \textSymb{vogel_DK}werd ingedrukt. De uitvoering van de functies wordt onderbroken. \eTD\eTR
\bTR\bTD 7 \eTD
\bTD Pijltoetsen \textSymb{vogelTop} \textSymb{vogelBottom} \eTD
\bTD [nc=2] \items[symbol=R]{Activeren van het display, Afvragen van de parameters (weergavemodus), Instellen van de getoonde (I) waarde (programmeermodus)}
\eTD\eTR
\bTR\bTD 8 \eTD
\bTD Toets \textSymb{vogelSet} \eTD
\bTD [nc=2] Omschakelen tussen de weergave- en programmeermodus of de ingevoerde waarden bevestigen. \eTD\eTR
\bTR\bTD 9 \eTD
\bTD Toets \textSymb{vogel_DK} \eTD
\bTD [nc=2] Indien het apparaat zich in de toestand {\em Pauze} bevindt, dan triggert een activering van de toets de tussentijdse smeercyclus. De foutmeldingen worden bevestigd en verwijderd. \eTD\eTR
\eTABLEbody
\eTABLE
\stop
\vfill

\startfigtext [left] [vogel_touches]{Indicatiemodule}
{\externalfigure[vogel_base][width=50mm]}
\textDescrHead{Weergavemodus} Kort op een van de pijltoetsen \textSymb{vogelTop} \textSymb{vogelBottom}
drukken om het 7 tekens tellende display \textSymb{led_huit} te activeren.
Door opnieuw op de toets \textSymb{vogelTop} te drukken kunnen de verschillende parameters
gevolgd door hun waarde worden weergegeven.
De modus {\em Weergave} is herkenbaar aan de continu brandende \LED\char"2060s (\in{2 tot 6, afb.}[vogel_affichage]).

\blank [medium]

\textDescrHead{Programmeermodus} Om de waarden te wijzigen drukt u gedurende minstens
2 seconden op de toets \textSymb{vogelSet} om om te schakelen naar de modus {\em Programmering}:
De \LED\char"2060s knipperen. Kort op de toets \textSymb{vogelSet} drukken om
de\index{Centrale smeerinstallatie+Programmering} weergave te wijzigen,
dan de gewenste waarde wijzigen met de toetsen \textSymb{vogelTop} \textSymb{vogelBottom}.
Bevestigen met\index{Centrale smeerinstallatie+Weergave} de toets \textSymb{vogelSet}.
\stopfigtext

\page [yes]


\subsection{Submenu´s in de modus {\em Weergave}}

\vskip -9pt

\adaptlayout [height=+5mm]

\startcolumns[balance=no]\stdfontsemicn

\startSymVogel
\externalfigure[vogel_tpa][width=26mm]
\SymVogel
\textDescrHead{Pauzetijd [h]} Druk op de toets \textSymb{vogelTop} om de geprogrammeerde waarden weer te geven.
\stopSymVogel

\startSymVogel
\externalfigure[vogel_068][width=26mm]
\SymVogel
\textDescrHead{Resterende pauzetijd [h]} Nog resterende tijd tot de volgende smeercyclus.
\stopSymVogel

\startSymVogel
\externalfigure[vogel_090][width=26mm]
\SymVogel
\textDescrHead{Totale pauzetijd [h]} Totale pauzetijd tussen twee cycli.
\stopSymVogel

\startSymVogel
\externalfigure[vogel_tco][width=26mm]
\SymVogel
\textDescrHead{Smeertijd [min]} Druk op \textSymb{vogelTop} om de geprogrammeerde waarden weer te geven.
\stopSymVogel

\startSymVogel
\externalfigure[vogel_tirets][width=26mm]
\SymVogel
\textDescrHead{Apparaat in standby} Weergave niet mogelijk, aangezien het apparaat zich in standby (pauze) bevindt.
\stopSymVogel

\startSymVogel
\externalfigure[vogel_026][width=26mm]
\SymVogel
\textDescrHead{Smeertijd [min]} Duur van een smeerproces.
\stopSymVogel

\startSymVogel
\externalfigure[vogel_cop][width=26mm]
\SymVogel
\textDescrHead{Systeemcontrole} Druk op \textSymb{vogelTop} om de geprogrammeerde waarden weer te geven.
\stopSymVogel

\startSymVogel
\externalfigure[vogel_off][width=26mm]
\SymVogel
\textDescrHead{Controlemodus} \hfill PS: Drukschakelaar;\crlf
CS: Cyclusschakelaar; OFF: Gedeactiveerd.
\stopSymVogel

\startSymVogel
\externalfigure[vogel_0h][width=26mm]
\SymVogel
\textDescrHead{Bedrijfsuren} Druk op \textSymb{vogelTop} om de waarde weer te geven in twee stappen.
\stopSymVogel

\startSymVogel
\externalfigure[vogel_005][width=26mm]
\SymVogel
\textDescrHead{Deel 1: 005} De bedrijfstijd wordt weergegeven in twee delen; naar deel 2 met de toets \textSymb{vogelTop}.
\stopSymVogel

\startSymVogel
\externalfigure[vogel_338][width=26mm]
\SymVogel
\textDescrHead{Deel 2: 33,8} Het 2de deel van het getal is 33,8; resulteert samen in een bedrijfstijd van 533,8 h.
\stopSymVogel

\startSymVogel
\externalfigure[vogel_fh][width=26mm]
\SymVogel
\textDescrHead{Fouttijd} Druk op \textSymb{vogelTop} om de waarde weer te geven in twee stappen.
\stopSymVogel

\startSymVogel
\externalfigure[vogel_000][width=26mm]
\SymVogel
\textDescrHead{Deel 1: 000} De fouttijd wordt weergegeven in twee delen;\crlf
naar deel 2 met \textSymb{vogelTop}.
\stopSymVogel

\startSymVogel
\externalfigure[vogel_338][width=26mm]
\SymVogel
\textDescrHead{Deel 2: 33,8} Het 2de deel van het getal is 33,8; resulteert samen in een fouttijd van 33,8 h.
\stopSymVogel

\stopcolumns

\stopsection

\page [yes]

\stopmode % Graissage central

\setups [pagestyle:marginless]


\startsection[title={Smeerschema voor handmatig smeren},
reference={sec:grasing:plan}]

\starttextbackground [FC]
\startPictPar
\PMgeneric
\PictPar
De in het smeerschema (\in{afb.}[fig:greasing:plan]) aangegeven smeerpunten moeten regelmatig worden gesmeerd. Een regelmatige smering is absoluut noodzakelijk om een permanente {\em verlaging van de wrijving} te garanderen en om vocht en andere corrosieve substanties weg te houden.
\stopPictPar
\stoptextbackground

\blank [big]

\start

\setupcombinations [width=\textwidth]

\placefig[here][fig:greasing:plan]{Smeerschema van het voertuig}
{\startcombination [3*1]
{\externalfigure[frame:steering:greasing]}{\ssx Schemelstuurinrichting en pendelmechanisme}
{\externalfigure[frame:axles:greasing]}{\ssx Assen}
{\externalfigure[frame:sucMouth:greasing]}{\ssx Zuigmond}
\stopcombination}

\stop

\vfill

\startLeg [columns,three]
\item Slagcilinder van de schemelstuurinrichting\crlf {\sl 2 smeernippels per cilinder}
\item Lager van de schemelstuurinrichting\crlf {\sl 2 smeernippels aan de linkerkant}
\columnbreak
\item Lager van het pendelmechanisme\crlf {\sl 1 smeernippel vóór de tank}
\item Bladveren\crlf {\sl 2 smeernippels per veerblad}
\columnbreak
\item Zuigmond\crlf {\sl 1 smeernippel per wiel}
\item Zuigmond\crlf {\sl 1 smeernippel op de trekarm}
\stopLeg



\page [yes]


\setups [pagestyle:bigmargin]


\subsubject{Smeren van de vuilcontainer}

De vuilcontainer bezit 8 smeerpunten (2\:×\:4), die wekelijks gesmeerd moeten worden.

\blank [big]


\placefig[here][fig:greasing:container]{Hefmechanisme van de container}
{\externalfigure[container:mechanisme]}


\placelegende [margin,none]{}
{{\sla Legende:}

\startLeg
\item Linker lager van de container (2\:×)
\item Rechter lager van de container (2\:×)
\item Linker hydraulische cilinder (boven)
\item Linker hydraulische cilinder (beneden)

{\em Zoals rechter cilinder (punt \in[greasing:point;hide]).}
\item Rechter hydraulische cilinder (boven)
\item [greasing:point;hide]Rechter hydraulische cilinder (beneden)
\stopLeg}

\stopsection

\page [yes]



\startsection[title={Elektrische installatie},
reference={sec:main:electric}]

\subsection{Centrale elektronica in het chassis}

\startbuffer [fuses:preventive]
\starttextbackground [CB]
\startPictPar
\PHvoltage
\PictPar
\textDescrHead{Veiligheidsvoorschriften}
Neem de veiligheidsvoorschriften in\index{Zekeringen+Chassis} deze\index{Relais+Chassis} handleiding in acht: Vervang zekeringen altijd alleen door zekeringen met het voorgeschreven amperage; doe metalen sieraden af, voordat u werkt aan de elektrische\index{Elektrische installatie} installatie (ringen, armbanden enz.).
\stopPictPar
\stoptextbackground
\stopbuffer


\subsubsubject{MIDI-zekeringen}

\starttabulate[|l|r|p|]
\HL
\NC\md F\,1 \NC 5\,A \NC Remlicht, \aW{+\:15} OBD \NC\NR
\NC\md F\,2 \NC 5\,A \NC \aW{+\:15} Motorbesturing \NC\NR
\NC\md F\,3 \NC 7,5\,A \NC \aW{+\:30} Motorbesturing en OBD \NC\NR
\NC\md F\,4 \NC 20\,A \NC Brandstofpomp \NC\NR
\NC\md F\,5 \NC 20\,A \NC \aW{D\:+} dynamo, \aW{+\:15} relais K\,1 \NC\NR
\NC\md F\,6 \NC 5\,A \NC Motorbesturing \NC\NR
\NC\md F\,7 \NC 10\,A\NC Zuivering uitlaatgassen motor \NC\NR
\NC\md F\,8 \NC 20\,A \NC Motorelektronica (besturing) \NC\NR
\NC\md F\,9 \NC 15\,A \NC Zuivering uitlaatgassen motor, brandstofpomp, voorgloeien \NC\NR
\NC\md F\,10\NC 30\,A \NC Motorbesturing \NC\NR
\NC\md F\,11\NC 5\,A \NC Achteruitrijlicht \NC\NR
%% NOTE @Andrew: Singular
\HL
\stoptabulate

\placefig [margin] [fig:electric:power:rear] {Centrale elektronica in het chassis}
{\externalfigure [electric:power:rear]
\noteF
\startKleg
\sym{K\,1} Elektronisch motorbesturingsapparaat
\sym{K\,2} Brandstofpomp
\sym{K\,3} Vrijgave van de starter
\sym{K\,4} Remlichten
\sym{K\,5} {[}Reserve{]}
\sym{K\,6} Achteruitrijlicht
\sym{K\,7} Voorgloei-installatie
\stopKleg
}


\subsubsubject{MAXI-zekeringen}

% \startcolumns [n=2]
\starttabulate[|l|r|p|]
\HL
\NC\md F 15 \NC 50 A \NC Hoofdvoeding van de centrale elektronica \NC\NR
\HL
\stoptabulate

\page [yes]

\setups[pagestyle:marginless]


\subsection{Centrale elektronica in de bestuurderscabine}

\startcolumns[rule=on]

\placefig [bottom] [fig:fuse:cab] {Zekeringen en relais in de bestuurderscabine}
{\externalfigure [electric:power:front]}

\columnbreak

\subsubsubject{Relais}

\index{Zekeringen+Bestuurderscabine}\index{Relais+Bestuurderscabine}

\starttabulate[|lB|p|]
\NC K\,2\NC Airco-compressor\NC\NR
\NC K\,3\NC Airco-compressor\NC\NR
\NC K\,4\NC Elektrische waterpomp\NC\NR
\NC K\,5\NC Alzijdig zwaailicht\NC\NR
\NC K\,10 \NC Knipperfrequentiesensor\NC\NR
\NC K\,11 \NC Dimlicht\NC\NR
\NC K\,12 \NC Groot licht {[}Reserve{]} \NC\NR
\NC K\,13 \NC Schijnwerpers\NC\NR
\NC K\,14 \NC Intervalschakeling ruitenwissers\NC\NR
\stoptabulate
\vskip -24pt

\placefig [bottom] [fig:fuse:access] {Toegangsklep naar de centrale elektronica}
{\externalfigure [electric:power:cabin]}

\stopcolumns

\page [yes]


\subsubsubject{MINI-zekeringen}

\startcolumns[rule=on]
% \setuptabulate[frame=on]
%\placetable[here][tab:fuses:cab]{Zekeringen in de bestuurderscabine}
%{\noteF
\starttabulate[|lB|r|p|]
\NC F 1 \NC 3\,A \NC Parkeerlicht links \NC\NR
\NC F 2 \NC 3\,A \NC Parkeerlicht rechts \NC\NR
\NC F\,3 \NC 7,5\,A \NC Dimlicht links \NC\NR
\NC F\,4 \NC 7,5\,A \NC Dimlicht rechts \NC\NR
\NC F\,5 \NC 7,5\,A \NC Groot licht links {[}Reserve{]} \NC\NR
\NC F\,6 \NC 7,5\,A \NC Groot licht rechts {[}Reserve{]} \NC\NR
\NC F\,7 \NC 10\,A \NC Schijnwerper boven \NC\NR
\NC F\,8 \NC 10\,A \NC Schijnwerper beneden (reserve) \NC\NR
\NC F\,9 \NC 10\,A \NC Frontbezem \NC\NR
\NC F\,10 \NC 10\,A \NC Ruitenwisser \NC\NR
\NC F\,11 \NC 5\,A \NC Schakelaar verlichting en waarschuwingsknipperlichten \NC\NR
\NC F\,12 \NC 5\,A \NC {[}Reserve{]} \NC\NR
\NC F\,13 \NC 10\,A \NC Buitenspiegelverwarming \NC\NR
\NC F\,14 \NC 7,5\,A \NC \aW{+\:15} Radio en camera \NC\NR
\NC F\,15 \NC 10\,A \NC \aW{+\:30} Waarschuwingsknipperlichten \NC\NR
\NC F\,16 \NC 5\,A \NC Lichtschakelaar stuurkolom \NC\NR
\NC F\,17 \NC 7,5\,A \NC \aW{+\:30} Radio en binnenverlichting \NC\NR
\NC F\,18 \NC — \NC {[}Vrij{]} \NC\NR
\NC F\,19 \NC 20\,A \NC \aW{+\:30} RC 12 voor \NC\NR
\NC F\,20 \NC 20\,A \NC \aW{+\:30} RC 12 achter \NC\NR
\NC F\,21 \NC 15\,A \NC 12V contactdoos \NC\NR
\NC F\,22 \NC 5\,A \NC Contactsleutel, multifunctionele console, Vpad \NC\NR
\NC F\,23 \NC 5\,A \NC Noodstop, middenconsole, RC 12 voor \NC\NR
\NC F\,24 \NC 5\,A \NC Noodstop, middenconsole, RC 12 achter \NC\NR
\NC F\,25 \NC 2\,A \NC \aW{+\:15} RC 12 voor \NC\NR
\NC F\,26 \NC 2\,A \NC \aW{+\:15} RC 12 achter \NC\NR
\NC F\,27 \NC 25\,A \NC Verwarmingsventilator \NC\NR
\NC F\,28 \NC 10\,A \NC Airco-compressor, centrale smeerinstallatie \NC\NR
\NC F\,29 \NC 25\,A \NC Airco-condensator \NC\NR
\NC F\,30 \NC 5\,A \NC Thermostaat airco \NC\NR
\NC F\,31 \NC 5\,A \NC \aW{+\:15} Multifunctionele console/Vpad \NC\NR
\NC F\,32 \NC 15\,A \NC Elektrische waterpomp, alzijdig zwaailicht \NC\NR
\NC F\,33 \NC — \NC {[}Vrij{]} \NC\NR
\NC F\,34 \NC — \NC {[}Vrij{]} \NC\NR
\NC F\,35 \NC — \NC {[}Vrij{]} \NC\NR
\NC F\,36 \NC — \NC {[}Vrij{]} \NC\NR
\stoptabulate
\stopcolumns

\page [yes]

\setups [pagestyle:bigmargin]


\subsection[sec:lighting]{Verlichtings- en signaalinrichting}


\placefig [here] [fig:lighting] {Verlichtings- en signaalinrichting van het voertuig}
{\externalfigure [vhc:electric:lighting]}

\placelegende [margin,none]{}{%
\vskip 30pt
{\sla Legende:}
\startLongleg
\item Parkeerlichten\hfill 12\,V–5\,W
\item Dimlichten\hfill H7~12\,V–55\,W
\item Knipperlichten\hfill oranje 12\,V–21\,W
\item {\stdfontsemicn Schijnwerpers}\hfill G886~12\,V–55\,W
\item Richtingaanwijzers\hfill 12\,V–21\,W
\item Achteruitrij-/Remlichten\hfill 12\,V–5/21\,W
\item Achteruitrij-schijnwerpers\hfill 12\,V–21\,W
\item {[}Vrij{]}
\item Kentekenverlichting\hfill 12\,V–5\,W
\item Alzijdig zwaailicht\hfill H1~12\,V–55\,W
\stopLongleg}

\subsubsubject{Instellen van de schijnwerpers}

\placefig [margin] [fig:lighting:adjustment] {Lichtstraal bij 5\,m}
{\externalfigure [vhc:lighting:adjustment]
\startitemize
\sym{H\low{1}} Hoogte van de gloeidraad: 100\,cm
\sym{H\low{2}} Correctie bij 2\hairspace\%: 10\,cm
\stopitemize}

{\md Voorwaarden:} Vers-/Recyclingwaterreservoir vol, bestuurder aan het stuur.

De uitrichting van de schijnwerpers wordt in de fabriek ingesteld. Hoogte en neiging van de lichtstraal kunnen worden ingesteld door de kunststof houder te zwenken.

Indien in het kader van een controle blijkt dat de instelling moet worden veranderd, draai dan de borgschroeven los en corrigeer de neiging zo, dat die overeenkomt met de wettelijke voorschriften (zie \in{afb.}[fig:lighting:adjustment]). Draai de borgschroeven weer vast.

\page [yes]
\setups [pagestyle:marginless]


\subsection[sec:battcheck]{Accu}

\subsubsection{Veiligheidsvoorschriften}

\startSymList
\PPfire
\SymList
\textDescrHead{Explosiegevaar}
Bij het\index{Accu+Veiligheidsinstructies}\index{Gevaar+Explosie} laden van accu´s vormt zich explosief\index{Knalgas} knalgas. Laad accu´s alleen in goed geventileerde ruimtes! Vermijd vonkvorming! Hanteer in de buurt van de accu nooit open vuur, open licht en rook niet.
\stopSymList

\startSymList
\PHvoltage
\SymList
\textDescrHead{Gevaar van kortsluiting}
Indien\index{Accu+Onderhoud} de pluspool van de aangesloten accu in aanraking komt met delen van het voertuig, dan bestaat\index{Gevaar+Brand}\index{Gevaar+Explosie} het gevaar van kortsluiting. Daardoor kan het uit de accu ontsnappende gasmengsel exploderen, en u en anderen zouden ernstig verwond kunnen raken.

\startitemize
\item Leg geen metalen voorwerpen of gereedschap op de accu.
\item Bij het isoleren van de accu altijd eerst de min-, dan de pluspool eraf nemen.
\item Bij het aansluiten van de accu altijd eerst de plus-, dan de minpool aansluiten.
\item Bij lopende motor de aansluitklemmen van de accu niet losmaken of eraf nemen.
\stopitemize
\stopSymList


\startSymList
\PHcorrosive
\SymList
\textDescrHead{Gevaar van brandwonden}
Draag \index{Gevaar+Brandwonden} een veiligheidsbril en zuurbestendige werkhandschoenen. Accuvloeistof bevat ongeveer 27%zwavelzuur (H\low{2}SO\low{4}) en kan daardoor brandwonden veroorzaken. Neutraliseer\index{Accu+Gevaar}\index{Accu+-vloeistof} accuvloeistof die in aanraking is gekomen met de huid, met een oplossing van dubbelkoolzure natron en spoel na met zuiver water. Indien accuvloeistof in aanraking is gekomen met de ogen, spoel deze dan met veel koud water en raadpleeg onmiddellijk een arts.
\stopSymList

\startSymList
\startcombination[1*2]
{\PHcorrosive}{}
{\PHfire}{}
\stopcombination
\SymList
\textDescrHead{Opslag van accu´s}
Accu´s\index{Accu+opslaan} altijd rechtop opslaan. Anders zou accuvloeistof kunnen ontsnappen en brandwonden of – bij reactie met andere substanties – branden kunnen veroorzaken. \par\null\par\null
\stopSymList

\testpage [16]

\starttextbackground [FC]
\setupparagraphs [PictPar][1][width=2.4em,inner=\hfill]

\startPictPar
\PMproteyes
\PictPar
\textDescrHead{Veiligheidsbril}
Bij het\index{Gevaar+Oogletsel} mengen van water en zuur kan de vloeistof in de ogen spetteren. Zuurspetters in het oog meteen wegspoelen met zuiver water en onmiddellijk een arts raadplegen!
\stopPictPar
\blank [small]

\startPictPar
\PMrtfm
\PictPar
\textDescrHead{Documentatie}
Bij de omgang met accu´s moeten de veiligheidsinstructies, beschermende maatregelen en procedures absoluut in acht worden genomen.
\stopPictPar
\blank [small]

\startPictPar
\PStrash
\PictPar
\textDescrHead{Bescherming van het milieu}
Accu´s\index{Bescherming van het milieu} bevatten schadelijke stoffen. Verwerk oude accu´s nooit met het huisvuil. Verwerk oude accu´s milieuvriendelijk. Geef ze af in een vakgarage of bij een inzamelpunt voor oude accu´s.

Gevulde accu´s altijd rechtop transporteren en opslaan. Bij het transport moeten accu´s tegen omkantelen worden beveiligd. Uit de ontluchtingsopeningen van de afsluitstop kan accuvloeistof lopen en in het milieu terechtkomen.
\stopPictPar
\stoptextbackground

\page [yes]

\setups[pagestyle:normal]


\subsubsection{Praktische raadgevingen}

Voor een maximale levensduur moet de accu indien mogelijk altijd vol zijn geladen.

Een\index{Accu+Levensduur} behoud van lading van de accu tijdens langere uitschakeltijden van het voertuig verlengt niet alleen de levensduur van de accu, maar garandeert ook een voortdurende gereedheid om het voertuig te starten.

\placefig[margin][fig:batterycompartment]{\select{caption}{Accuvak (onderhoudsklep)}{Accuvak}}
{\externalfigure[batt:compartment]}


\subsubsection{Instandhouding}

De accu van de \sdeux\ is een {\em onderhoudsvrije} loodaccu. Behalve het behouden van de geladen toestand en de reiniging vergt de accu geen onderhoudsmaatregelen.

\startitemize
\item Zorg ervoor dat de polen van de accu altijd schoon en droog zijn. Smeer de polen licht in met wat zuurafstotend vet.
\item Accu´s die\index{Accu+laden} een rustspanning van\index{Accu+Rustspanning} minder dan 12,4 V bezitten, naladen.
\stopitemize

\placefig[margin][fig:bclean]{Reinigen van de polen}
{\externalfigure[batt:clean]
\noteF
Verwenden\index{Accu+reinigen}\index{Reiniging+Accu´s} Gebruik warm water om het door corrosie gevormde witte poeder te verwijderen. Indien een pool verroest is, isoleer dan de accukabel en maak de pool schoon met een draadborstel. Voorzie de pool van een dunne vetfilm.}


\subsubsection[sec:battery:switch]{Gebruik van de accu-isolatieschakelaar}

{\sl Het valt niet aan te bevelen om de accu-isolatieschakelaar regelmatig (bijvoorbeeld dagelijks) te activeren!}

\startSteps
\item Schakel\index{Accu-isolatieschakelaar} de ontsteking uit en wacht vervolgens ongeveer 1 minuut.
\item Open het accuvak (\inF[fig:batterycompartment]).
\item Druk op de rode knop van de accu-isolatieschakelaar om de stroomkring te onderbreken.
\item Om de stroomkring weer te sluiten draait u de accu-isolatieschakelaar ¼ omdraaiing met de klok mee.
\stopSteps

% \starttextbackground [FCnb]
% \startPictPar
% \PMgeneric
% \PictPar
% Der Batterietrennschalter ist dafür vorgesehen, die Batterie für bestimmte Wartungs- und Reparaturarbeiten vorübergehend vom Stromkreis zu trennen. Es ist nicht empfehlenswert, den Batterietrennschalter regelmäßig (\eG\ täglich) zu betätigen: Bestimmte elektronische Komponenten sollten ständig unter Spannung stehen, ansonsten kann es zu Fehlermeldungen im Fehlerspeicher kommen.
% \stopPictPar
% \stoptextbackground

\stopsection

\page [yes]


\setups[pagestyle:marginless]

\section[sec:cleaning]{Reiniging van het voertuig}

Spoel\startregister[index][vhc:lavage]{Onderhoud+Reiniging} vóór de eigenlijke reiniging grof slijk en modder met veel water af van de carrosserie. Was daarbij niet alleen de zijvlakken, maar ook de wielhuizen en de onderkant van het voertuig.

Met name in de winter moet het voertuig grondig worden gewassen on het te ontdoen van uiterst corrosieve\index{Corrosie+Preventie} resten van strooizout.

\starttextbackground [FC]
\startPictPar
\PHgeneric
\PictPar
\textDescrHead{Schade door water verhinderen}
Reinig het voertuig nooit met behulp van {\em waterpistolen} (\eG\ van de brandweer) of {\em koudreinigers op basis van koolwaterstof.} Indien u werkt met een hogedruk-stoomreiniger, neem dan de onderstaande voorschriften hiervoor in acht.
\stopPictPar
\blank[small]

\startPictPar
\pTwo[monde]
\PictPar
\textDescrHead{Bescherming van het milieu}
Het reinigen van een voertuig kan ernstige schade aan het milieu veroorzaken.
Reinig het voertuig alleen op een met een\index{Bescherming van het milieu} olieafscheider uitgeruste standplaats. Neem de geldende voorschriften ter bescherming van het milieu in acht.
\stopPictPar
\blank[small]

\startPictPar
\PMwarranty
\PictPar
\textDescrHead{Deskundig reinigen!}
Voor schade als gevolg van de niet-inachtneming van de reinigingsvoorschriften kunnen tegenover \BosFull{boschung} geen aansprakelijkheids- of garantie-eisen worden ingediend.
\stopPictPar
\stoptextbackground


\subsection{Hogedrukreiniging}

Voor de hogedrukreiniging\index{Reiniging+Hoge druk} van het voertuig is een in de handel verkrijgbare hogedrukreiniger geschikt.

Bij de hogedrukreiniging moeten de volgende punten in acht worden genomen:

\startitemize
\item Werkdruk maximaal 50\,bar
\item Plat straalmondstuk met een spuithoek van 25°
\item Spuitafstand minstens 80\,cm
\item Watertemperatuur maximaal 40\,°C
\item Neem de paragraaf \about[reiMi], in acht \atpage[reiMi].
\stopitemize

Bij niet-inachtneming van deze\index{Lak+Schade} voorschriften kan schade aan lak en corrosiebescherming\index{Schade+Lak} ontstaan.

Neem ook de handleiding en veiligheidsvoorschriften van de hogedrukreiniger in acht.

\starttextbackground[FC]
\startPictPar\PPspray\PictPar
Bij de inzet van hogedrukreinigers kan er water binnendringen op plaatsen, waar die schade kan veroorzaken. Richt de waterstraal daarom nooit op gevoelige delen en uitrustingen:
\stopPictPar

\startitemize
\item Sensoren, elektrische verbindingen en aansluitingen
\item Starter, dynamo, injectiesysteem
\item Magneetkleppen
\item Ventilatieopeningen
\item Nog niet afgekoelde mechanische componenten
\item Informatie-, waarschuwings- en gevaarstickers
\item Elektronische besturingsapparaten
\stopitemize

\textDescrHead{Wassen van de motor}
Binnendringen van water aanzuig-, be- en ontluchtingsopeningen absoluut vermijden. Bij hogedrukreinigers de straal niet direct op elektrische componenten en leidingen richten. De straal niet richten op het injectiesysteem! Motor na het wassen conserveren; daarbij de riem beschermen tegen het conserveringsproduct.
\stoptextbackground

\starttextbackground [FC]
\setupparagraphs [PictPar][1][width=6em,inner=\hfill]
\startPictPar
\framed[frame=off,offset=none]{\PMproteyes\PMprotears}
\PictPar
\textDescrHead{Restwater}
Tijdens de reiniging verzamelt zich op bepaalde plaatsen van het voertuig water (\eG\ in de uithollingen van het motorblok of van de transmissie); dit moet met behulp van perslucht worden verwijderd. Houd er rekening mee dat bij de omgang met perslucht een adequate beschermende uitrusting moet worden gedragen, en dat de installatie moet voldoen aan de geldende veiligheidsvoorschriften (multi-mondstuk).
\stopPictPar
\stoptextbackground


\subsubsection[reiMi]{Geschikte reinigingsmiddelen}

Gebruik\index{Reinigingsmiddel} uitsluitend reinigingsmiddelen met de volgende eigenschappen:

\startitemize
\item Geen schurende werking
\item pH-waarde van 6–7
\item Zonder oplosmiddelen
\stopitemize

Om hardnekkige vlekken te verwijderen gebruikt u op kleine gelakte vlakken weloverwogen wasbenzine of spiritus, in geen geval andere oplosmiddelen. Verwijder resten van oplosmiddel van de lak. Het reinigen van kunststof delen met benzine kan scheuren of verkleuringen veroorzaken!

Reinig vlakken met\index{Reiniging+Stickers} waarschuwings- of informatiestickers met helder water en een zachte spons.

Vermijd dat er water binnendringt in elektrische componenten: gebruik geen autoborstel om de knipperlichten en het huis van de verlichting te reinigen, maar een zachte doek of spons.

\starttextbackground [CB]
\startPictPar
\GHSgeneric\par
\GHSfire
\PictPar
\textDescrHead{Gevaar door chemicaliën}
Van reinigingsmiddelen kunnen gezondheids- en veiligheidsrisico´s (licht ontvlambare stoffen) uitgaan. Neem de voor het gebruikte reinigingsmiddel geldende veiligheidsvoorschriften in acht; neem de gevaren- en informatiebladen van de gebruikte middelen in acht.
\stopPictPar
\stoptextbackground

\stopregister[index][vhc:lavage]

\page [yes]


\setups [pagestyle:bigmargin]

\startsection [title={Instellen van de zuigmond},
reference={sec:main:suctionMouth}]


De optimale afstand\index{Zuigmond+Instellen} tussen wegdek en rubber lip van de zuigmond bedraagt 10\,mm.
Om de afstand te controleren resp. in te stellen gebruikt u de drie instelsjablonen, die u vindt in de gereedschapskist (bestuurderscabine, bestuurderskant).


\placefig [margin] [fig:suctionMouth] {Instellen van de zuigmond}
{\Framed{\externalfigure [suctionMouth:adjust]}}

\placeNote[][service_picto]{}{%
\noteF
\starttextrule{\PHasphyxie\enskip Vergiftigings- en verstikkingsgevaar \enskip}
{\md Aanwijzing:} Tijdens de instelwerkzaamheden moet de motor lopen, om de zuigmond in zweefstand te kunnen houden. Om het gevaar van een vergiftiging of verstikking uit te sluiten moet daarom absoluut een afzuiginstallatie voor uitlaatgassen worden ingezet, resp. de werkzaamheden mogen alleen worden uitgevoerd op een zeer goed geventileerde plaats.
\stoptextrule}

\startSteps
\item Parkeer het voertuig op een zeer goed geventileerde plaats op een horizontale en vlakke ondergrond.
\item Activeer\index{Afzuiging} de \aW{werk}modus (knop buiten aan de keuzehendel voor het rijniveau indrukken).

Laat de motor lopen met stationair toerental. (Druk op de toets~\textSymb{joy_key_engine_decrease} op de multifunctionele console om het motortoerental te verlagen.)
\item Trek de vastzetrem aan en beveilig de achterwielen met elk een spie.
\item Druk op de toets~\textSymb{joy_key_suction} om de zuigmond neer te laten.
\item Plaats daarbij de drie instelsjablonen~\LAa\ onder de rubber lip van de zuigmond, zoals voorgesteld in de afbeelding.
\item [sucMouth:adjust]Draai de bevestigings-~\Lone\ en instelschroeven~\Ltwo\ van elke wiel los; de vier wielen zakken naar de bodem.
\item Draai de schroeven~\Lone\ en~\Ltwo\ weer aan, en verwijder dan de drie instelsjablonen.
\item Til de zuigmond op en laat hem neer, en controleer de instelling met de instelsjablonen. Indien de instelling nog niet helemaal klopt, herhaal de Instelprocedure dan vanaf punt~\in[sucMouth:adjust].

\stopSteps


\stopsection
\stopchapter
\stopcomponent


