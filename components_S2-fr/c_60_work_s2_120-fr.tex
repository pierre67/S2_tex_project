\startcomponent c_60_work_s2_120-fr


\startchapter [title={Votre Urban-Sweeper au quotidien},
							reference={chap:using}]

\setups [pagestyle:marginless]


% \placefig[margin][fig:ignition:key]{Clé de contact}
% {\externalfigure [work:ignition:key]}
\startregister[index][chap:using]{mise en service}

\startsection [title={Mise en service},
							reference={sec:using:start}]


\startSteps
\item Assurez||vous que le contrôle technique requis ait été effectué selon la procédure correcte.
\item Démarrez le moteur diesel au moyen de la clé de contact~— mettre le contact,
maintenir la clé dans le sens horaire jusqu’au démarrage du moteur, puis relâcher~–
placée sur le côté droit de la colonne de direction (possible uniquement si le levier sélecteur de marche est
placé en position neutre).
\stopSteps

\start
\setupcombinations [width=\textwidth]

\placefig[here][fig:select:drive]{Levier sélecteur de marche}
{\startcombination [2*1]
{\externalfigure [work:select:fDrive]}{Sélecteur en position de marche avant}
{\externalfigure [work:select:rDrive]}{Sélecteur en position de marche arrière}
\stopcombination}
\stop


\startSteps [continue]
\item Pivotez le sélecteur pour sélectionnez un rapport en mode \quote{transport}:
\startitemize [R]
\item Premier rapport
\item Deuxième rapport (démarre automatiquement sur le premier rapport)
\stopitemize

ou pressez sur le bouton à l’extrémité du levier pour activer|/|désactiver le mode \quote{travail}.
\stopSteps

\startbuffer [work:config]
\starttextbackground [FC]
\startPictPar
\PMrtfm
\PictPar
En mode travail, le premier rapport est automatiquement enclenché et le moteur diesel
tourne à 1300 min\high{\textminus 1}.

Utilisez les boutons \textSymb{joy_key_engine_increase} et \textSymb{joy_key_engine_decrease}
sur la console de commande pour modifier le régime du moteur.
\stopPictPar
\stoptextbackground
\stopbuffer

\getbuffer [work:config]


\startSteps [continue]
\item Soulevez le sélecteur de marche et positionnez-le en marche avant (vers l’avant) ou marche arrière (voir illustrations ci-dessus).
\item Libérez complètement le frein à main avant d’accélérer.
\stopSteps

\starttextbackground [FC]
\startPictPar
\PMrtfm
\PictPar
{\md Libérez complètement le frein à main!} La position du levier de frein à main est surveillée par un capteur électronique.
Si le levier de frein à main n’est pas complètement abaissé, la vitesse du véhicule est limitée à 5\,km/h.
\stopPictPar
\stoptextbackground

\startSteps [continue]
\item Pressez progressivement sur la pédale de l’accélérateur pour mettre la machine en mouvement.
\stopSteps



\subsection [sSec:suctionClap] {Clapet du canal d’aspiration}

Le système d’aspiration génère un flux d’air aspiré depuis la bouche d’aspiration vers la
cuve à déchets, via le canal d’aspiration ou l’embouchure du tuyau d’aspiration à main (disponible en option).

Un clapet à commande manuelle (voir \in{fig.}[fig:suctionClap]) permet de choisir l’un ou l’autre selon
le service souhaité.

\placefig [here] [fig:suctionClap] {Clapet du canal d’aspiration}
{\startcombination [2*1]
{\externalfigure [work:suctionClap:open]}{Canal d’aspiration ouvert}
{\externalfigure [work:suctionClap:closed]}{Canal d’aspiration fermé}
\stopcombination}


En service normal~– aspiration par la bouche d’aspiration~– le canal d’aspiration doit être ouvert
(levier de commande placé vers le haut).

Pour utiliser le tuyau d’aspiration à main, le canal d’aspiration doit être fermé
(levier de commande placé vers le bas). Ainsi, l’air aspiré passera par l’embouchure du tuyau
d’aspiration à main.




\stopsection



\startsection [title={Mise hors service},
							reference={sec:using:stop}]

\index{mise hors service}

\startSteps
\item Actionnez le frein à main~— le levier de commande du frein de stationnement est placé entre les deux sièges~— et placez le sélecteur de marche en position \quote{neutre}.
\item Effectuez le contrôle technique requis~– quotidien et hebdomadaire si nécessaire~– selon la procédure \atpage[table:scheduledaily].
\stopSteps

\getbuffer [prescription:handbrake]


\stopsection



\startsection [title={Balayage et aspiration des déchets},
							reference={sec:using:work},
							]

\startSteps
\item Procédez\index{balayage} à la mise en service du véhicule selon la procédure décrite \in{§}[sec:using:start], \atpage[sec:using:start].
\item Activez\index{aspiration} le mode \quote{travail} (pressez sur le bouton à l’extrémité du sélecteur de marche).
\stopSteps

% \getbuffer [work:config]

\startSteps [continue]
\item Pressez la touche \textSymb{joy_key_suction_brush} pour actionner la turbine et la rotation des balais.

{\md Variante:} {\lt pressez la touche \textSymb{joy_key_suction} pour ne travailler qu’avec la bouche d’aspiration.}

\item Ajustez la vitesse de rotation des balais au moyen des touches \textSymb{joy_key_frontbrush_increase}
et \textSymb{joy_key_frontbrush_decrease} de la console de médiane.

\item Positionnez les balais au moyen de leur joystick respectif pour adapter la largeur de travail efficace en fonction
de la chaussée.

\stopSteps

\vfill

\start
\setupcombinations [width=\textwidth]

\placefig[here][fig:brush:position]{Positionnement des balais}
{\startcombination [2*1]
{\externalfigure [work:brushes:enlarge]}{Ajustez l’écartement d’un|/|des balais}
{\externalfigure [work:brush:left:raise]}{Adaptez la hauteur d’un|/|des balais}
\stopcombination}
\stop

\page [yes]


\subsubsubject{Humectage des balais et du canal d’aspiration}

Pressez\index{balayage+humectage} sur l’interrupteur \textSymb{temoin_busebalais} 3 positions (0|/|arrêt-1-2) placé entre les deux sièges.

{\md Position 1:} Lorsque les balais sont activés, la pompe à eau alimente automatiquement le système d’humectage des balais.

{\md Position 2:} La pompe à eau est enclenchée en permanence (par ex. pour effectuer des réglages).




\subsubsubject{Faire passer les gros déchets}

\startSteps [continue]
\item Si un objet de grande taille~– une bouteille PET de grande taille ou autres~–
vient\index{bouche d’aspiration+gros déchets} obstruer la bouche d’aspiration,
utilisez les touches latérales de la console multifonction pour ouvrir
le clapet pour gros déchets\index{clapet pour gros déchets} ou~– si cela n’est pas suffisant~–
soulever momentanément la bouche d’aspiration.
\stopSteps

\start
\setupcombinations [width=\textwidth]

\placefig[here][fig:suctionMouth:clap]{Solutions pour les gros déchets}
{\startcombination [2*1]
{\externalfigure [work:suction:open]}{Ouvrez le clapet pour gros déchets}
{\externalfigure [work:suction:raise]}{Soulevez momentanément la bouche d’aspiration}
\stopcombination}
\stop

\stopsection

\startsection [title={Vidange de la cuve à déchets},
							reference={sec:using:container},
							]


\startSteps
\item Placez la balayeuse\index{cuve à déchets+vidange} sur un emplacement prévu à cet effet,
en conformité avec la législation en vigueur,
notamment en matière de protection de l’environnement.
\item Tirez le frein à main et placez le sélecteur de marche en position \quote{neutre}
(sans quoi la commande n’est pas actionnable).
\stopSteps

\getbuffer [prescription:container:gravity]


\startSteps [continue]
\item Déverrouillez|/|ouvrez le couvercle de la cuve à déchets.
\item Pressez sur l’interrupteur \textSymb{temoin_kipp2}~– placé entre les deux sièges~– pour basculer la cuve.
\item Lorsque la cuve est vide, lavez l’intérieur avec un jet d’eau sous pression
(utilisez le pistolet d’eau intégré, disponible en option).
\stopSteps

\start
\setupcombinations [width=\textwidth]

\placefig[here][fig:brush:adjust]{Manipulation de la cuve à déchets}
{\startcombination [3*1]
{\externalfigure [container:cover:unlock]}{Verrouillage du couvercle}
{\externalfigure [container:safety:unlocked]}{Sécurité anti||écrasement}
{\externalfigure [container:safety:locked]}{Sécurité verrouillée}
\stopcombination}
\stop

\startSteps [continue]
\item Vérifiez|/|nettoyez les joints et plan||joints de la cuve et du système de recyclage et le canal d’aspiration.
\stopSteps

\getbuffer [prescription:container:tilt]


\startSteps [continue]
\item Pressez sur le dos de l’interrupteur \textSymb{temoin_kipp2} pour abaisser la cuve à déchets
(retirez les sécurités anti||écrasement sur les vérins si nécessaire).
\item Verrouillez le couvercle de la cuve à déchets.
\stopSteps

\stopsection


\startsection [title={Tuyau d’aspiration à main},
							reference={sec:using:suction:hose},
							]

La balayeuse \sdeux peut être équipée~– en option~– d’un tuyau d’aspiration\index{tuyau d’aspiration à main} à main. Il est placé sur le couvercle de la cuve, son utilisation est simple.

{\sla Prérequis:}

La cuve à déchets doit être complètement abaissée, la machine est en mode \quote{travail} (voir \in{§}[sec:using:start], \atpage[sec:using:start]).

\startfigtext[left][fig:using:suction:hose]{Tuyau d’aspiration à main}
{\externalfigure[work:suction:hose]}
\startSteps
\item Pressez la touche \textSymb{temoin_aspiration_manuelle}~– sur la console de plafond~– pour activer le système d’aspiration.
\item Tirez fermement le frein à main (avant de quitter le poste de conduite).
\item Fermez le clapet du canal d’aspiration (voir \in{§}[sSec:suctionClap], \atpage[sSec:suctionClap]).
\item Retirez l’embout du tuyau d’aspiration de son emplacement et commencez votre mission.
\item Lorsque votre mission est terminée, pressez la touche \textSymb{temoin_aspiration_manuelle} pour
désactiver le système d’aspiration et ouvrez à nouveau le clapet du canal d’aspiration.
\stopSteps
\stopfigtext

\stopsection

\page [yes]

\setups[pagestyle:normal]


\startsection [title={Pistolet d’eau haute pression},
							reference={sec:using:water:spray},
							]

La balayeuse \sdeux peut être équipée~– en option~– d’un pistolet d’eau\index{pistolet d’eau} haute pression.
Le pistolet est placé derrière le portillon arrière droit. Il est raccordé à un dérouleur de tuyau (10\,m)
placé sur le côté opposé de la machine.
Pour utiliser ce dispositif, procédez comme suit:

{\sla Prérequis:}

Le réservoir d’eau claire contient suffisamment d’eau, la machine est en mode \quote{travail} (voir \in{§}[sec:using:start], \atpage[sec:using:start]).

\placefig[margin][fig:using:water:spray]{Pistolet d’eau haute pression}
{\externalfigure[work:water:spray]}

\startSteps
\item Pressez la touche \textSymb{temoin_buse}~– sur la console de plafond~– pour activer la pompe à eau haute pression.
\item Tirez fermement le frein à main (avant de quitter le poste de conduite).
\item Ouvrez le portillon latéral arrière droit et saisissez le pistolet d’eau.
\item Tirez sur le tuyau pour le faire dérouler à votre convenance et commencez votre mission.
\item Lorsque votre mission est terminée, pressez à nouveau la touche \textSymb{temoin_buse}
pour désactiver la pompe à eau haute pression.
\item Exercez une brève traction sur le tuyau pour lâcher le cran de blocage et enrouler le tuyau.
\item Remettez le pistolet dans son logement et fermez le portillon latéral.
\stopSteps

\stopsection

\page [yes]

\setups [pagestyle:marginless]


\startsection [title={Travailler avec le troisième balai (option)},
							reference={sec:using:frontBrush},
							]

\startSteps
\item Procédez\index{balayage} à la mise en service du véhicule selon la procédure décrite \in{§}[sec:using:start], \atpage[sec:using:start].
\item Activez\index{3\high{e} balai} le mode \quote{travail}
(pressez sur le bouton à l’extrémité du sélecteur de marche).
\stopSteps

% \getbuffer [work:config]

\startSteps [continue]
\item Vérifiez que le 3\high{e} balai soit activé à l'écran du Vpad
(voir \textSymb{vpadFrontBrush} \textSymb{vpadFrontBrushK},\atpage[vpad:menu]).
\item Pressez la touche \textSymb{joy_key_frontbrush_act} pour actionner le système hydraulique du 3\high{e} balai.
\item Pressez la touche \textSymb{joy_key_frontbrush_left} ou \textSymb{joy_key_frontbrush_right} pour actionner la rotation du 3\high{e} balai, selon le sens de rotation souhaité.

\item Ajustez la vitesse de rotation du balai au moyen des touches \textSymb{joy_key_frontbrush_increase}
et \textSymb{joy_key_frontbrush_decrease} de la console de médiane.

\item Positionnez le balai au moyen des joysticks selon l'illustration ci||dessous.

\stopSteps

{\md Remarque:} {\lt pour déplacer les balais latéraux, pressez la touche \textSymb{joy_key_frontbrush_act} pour désactiver le système hydraulique du 3\high{e} balai.}
\vfill

\start
\setupcombinations [width=\textwidth]

\placefig[here][fig:brush:position]{Positionnement du troisième balai}
{\startcombination [2*1]
{\externalfigure [work:frontBrush:move]}{Déplacez le balai haut|/|bas; gauche|/|droite}
{\externalfigure [work:frontBrush:incline]}{Inclinez le balai sur l'axe transversal|/|longitudinal}
\stopcombination}
\stop



\stopsection


\stopregister[index][chap:using]










\stopchapter
\stopcomponent



