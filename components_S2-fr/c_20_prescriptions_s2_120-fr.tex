\startcomponent c_20_prescriptions_s2_120-fr


\chapter[safety:risques]{Prescriptions de sécurité}

\setups [pagestyle:marginless]


\section{Instructions fondamentales}


\subsubject{Bases légales}

Les conséquences en cas d’accident peuvent être lourdes pour les employeurs et les salariés. La société \boschung
rappelle les obligations de l’employeur et de l’employé\note[prescription:user:right].

L’employeur doit, avant de confier la balayeuse à un salarié, respecter les points suivants:

\startSteps
\item Chaque conducteur est formé à la conduite de la balayeuse et la preuve est conservée.
\item Chaque conducteur possède une autorisation de conduite formelle, laquelle ne peut-être délivrée que si les 3 conditions suivantes sont respectées:
\startitemize [2]
\item le salarié a satisfait à une visite médicale d’aptitude (réalisé par le médecin du travail);
\item le salarié connaît les lieux et instructions de sécurité à respecter sur le site d’utilisation de
la balayeuse (communiqués sous la responsabilité de sa hiérarchie);
\item le salarié a satisfait à un examen d’évaluation d’aptitude, attestant de son savoir||faire pour conduire
la balayeuse.
\stopitemize
\stopSteps

Lorsqu’une balayeuse peut rouler à plus de 25\,km/heure\note[prescription:user:right], elle doit obligatoirement être immatriculée et son conducteur doit être en possession:
\startitemize
\item du permis B\note[prescription:lisence] si la balayeuse a
un PTAC (poids total autorisé en charge) de moins de 3,5 tonnes;
\item du permis C\note[prescription:lisence] si le PTAC de la balayeuse est supérieur à 3,5 tonnes.
\stopitemize

Lorsqu’une balayeuse ne peut rouler à plus de 25\,km/heure, son conducteur doit au minimum connaître
le Code de la route pour circuler sur la voie publique,
même s’il n’est pas obligatoire d’être titulaire du permis B\note[prescription:user:right].


\footnotetext [prescription:user:right] {Les obligations de l’employeur et de l’employé
peuvent différer selon le pays ou la région où vous travaillez. Renseignez||vous sur la législation en vigueur dans
votre pays et votre région.}

\footnotetext[prescription:lisence] {Directive 2006/126/CE du Parlement européen
et du Conseil du 20 décembre 2006 relative au permis de conduire.}

\page [yes]


\subsubject{Condition d’utilisation}

La \sdeux\ ne peut être utilisée que si elle se trouve dans un parfait état de marche.
Par ailleurs, l’opérateur doit respecter les consignes de sécurité et les prescriptions
contenues dans le manuel d’utilisation. Les dysfonctionnements susceptibles d’entrainer
des dangers pour la sécurité doivent impérativement être corrigés|/|réparés par le service compétent.

\blank [big]


\startSymList
\externalfigure [s2_inspection] [width=4.5em]
\SymList
{\md Inspection quotidienne:}
Inspecter la machine après chaque opération de balayage afin de repérer les éventuels
dommages et défauts visibles.
Informer immédiatement le service compétent en cas de dommages ou de défaut de comportement du véhicule.
Le cas échéant, arrêter la machine et sécuriser l’emplacement immédiatement.
\stopSymList


\subsubject{Utilisation conforme}

La \sdeux\ a été conçue pour effectuer des tâches relatives à
l’entretien et le nettoyage de la chaussée. Toute utilisation en dehors de
ce cadre est considérée comme non conforme. La société \boschung décline par conséquent toute
responsabilité pour les dommages qui en résulteraient. Seul l’utilisateur en
assume les conséquences. (L’utilisation conforme comprend également le respect
des consignes de sécurité et du plan de maintenance contenu dans le présent manuel d’utilisation).



\section{Déplacement sur la voie publique}

\subsubject{Règles de prévention}

Respectez toutes les règles universellement reconnues, les lois et autres règlements en vigueur
en matière de prévention des accidents et de protection de l’environnement.


\subsubject{Place du passager}

Une passagère ou un passager peut prendre place dans le siège de l’aide-chauffeur prévu à cet effet.

\page [yes]


\subsubject{Ceintures de sécurité}

\startSymList
\externalfigure [prescription:safety:belt]
\SymList
Le conducteur et le passager de la \sdeux\ doivent toujours boucler leur ceinture de sécurité une fois assis dans le véhicule, conformément aux lois en vigueur dans le pays.
\stopSymList


\subsubject{Voir et être vu}

\startSymList
\externalfigure [travaux_deviation] [width=3.5em]
\SymList
Signalez clairement votre présence, notamment sur les routes à grand trafic.\par

Si le conducteur n’a pas une vue suffisamment claire, lors d’une manœuvre ou une intervention particulière,
il doit être assisté par un auxiliaire qui garde un contact visuel avec lui.
\stopSymList

\subsubject{Éclairage et signalisation}

Allumez les feux et|/|ou le gyrophare également
en journée selon les règles de sécurité routière en vigueur dans le pays.


\subsubject{Utilisation d’un téléphone mobile}

\startSymList
\PPphone
\SymList
L’utilisation d’un téléphone mobile ou d’une radio en déplacement sur la voie publique est interdite,
à moins que le véhicule soit équipé d’une installation {\em mains libres}.\par

Téléphoner au volant (même avec mains libres) diminue\index{sécurité+téléphone mobile} la concentration.
\stopSymList

\page [yes]


\section{Prescriptions de maintenance}

\subsubject{Instructions de maintenance}

Le personnel de maintenance doit, avant le début des travaux, avoir lu le manuel d’utilisation
et plus particulièrement les chapitres concernant la sécurité et la maintenance de la \sdeux.


\subsubject{Qualifications requises}

\startSymList
\externalfigure [mecanicienne] [width=3.5em]
\SymList
Seules les personnes ayant acquis les connaissances nécessaires par une formation adéquate sont habilitées à effectuer des travaux de maintenance sur la \sdeux. Ceci est particulièrement valable pour les travaux sur le moteur, le système de freinage, la direction, l’installation électrique et hydraulique.
\stopSymList

\subsubject{Surveillance}

\startSymList
\externalfigure [mecanicien_hyerarchie] [width=3.5em]
\SymList
Ne laissez jamais travailler des personnes en formation~— stage ou apprentissage~— sur la machine sans
la surveillance d’une personne compétente. Vérifier ponctuellement que le personnel manipule
la machine dans le respect des règles de sécurité et du manuel d’utilisation.
\stopSymList


\subsubject{Travaux de soudure}

\startSymList
\externalfigure [pince_soudure2] [width=3.5em]
\SymList
Avant d’effectuer des travaux de soudure sur la carrosserie ou sur le châssis, il faut impérativement débrancher la batterie et déconnecter tous les boitiers électroniques de commande.
\stopSymList

\subsubject{Lavage de la machine}

\startSymList
\externalfigure [washer_pressure] [width=3.5em]
\SymList
Avant de laver votre \sdeux, lisez la section \about[sec:cleaning] \at{page}[sec:cleaning], notamment le paragraphe qui concerne les prescriptions de lavage.
\stopSymList


\subsubject{Accessibilité du manuel}

\startSymList
\externalfigure [lecteur_1] [width=3.5em]%\PMrtfm
\SymList
Les instructions d’utilisation doivent toujours être facilement accessibles sur le lieu d’utilisation du véhicule,
en principe à l’intérieur de la cabine.
\stopSymList


\section{Prescriptions d’utilisation particulières}

\subsubject{Hauteur d’obstacles}

\startSymList
\PPmaxheight
\SymList
Lorsque vous pénétrez dans un passage couvert ou dans un souterrain, vous devez toujours vérifier que
la hauteur est suffisante pour le passage de la \sdeux (voir \in{section}[sec:measurement] \atpage[sec:measurement]).
\stopSymList



\subsubject{Stabilité du véhicule}

Évitez toute manœuvre et tout style de conduite pouvant réduire la stabilité du véhicule.
En cas de vitesse trop élevée dans un virage, la \sdeux, de par sa construction étroite
et un centre de gravité déplacé vers le haut lorsque la cuve est chargée, peut se renverser facilement.

\subsubject{Déplacement intempestif}

Lorsque vous quittez le véhicule, sécurisez-le contre l’utilisation par des personnes non autorisées! Activez systématiquement le frein de stationnement avant de quitter le véhicule, placez une cale de roue si nécessaire.

\startbuffer [prescription:handbrake]
\starttextbackground [CB]
\startPictPar
\PPstop
\PictPar
{\md Tirez le frein à main fermement!} Sans quoi, votre balayeuse peut se mettre en mouvement, même sur une légère pente ou un faux plat, et provoquer un accident
avec risque de blesser mortellement de tierces\index{frein à main+risque à connaitre} personnes.

{\lt Le système de transmission hydrostatique du véhicule à l’arrêt libère peu à peu la pression résiduelle dans le circuit, réduisant ainsi la force de retenue
des moteurs. C’est pourquoi le levier de frein à main doit être tiré fermement chaque fois que l’on quitte le véhicule.}
\stopPictPar
\stoptextbackground

\stopbuffer

\getbuffer [prescription:handbrake]


\subsubject{À propos de la cuve à déchets}

\startbuffer [prescription:container:gravity]
\starttextbackground [CB]
\startPictPar
\PHgravite
\PictPar
{\md Risque d’accident:}
{\lt Lors de la manœuvre de basculement de la cuve, le centre de gravité est déplacé vers le haut, augmentant ainsi le risque
de renversement du véhicule. Le véhicule doit être placé sur un sol stable et horizontal.}
\stopPictPar
\stoptextbackground

\stopbuffer

\getbuffer [prescription:container:gravity]


\startbuffer [prescription:container:tilt]
\starttextbackground [CB]
\startPictPar
\PHcrushing
\PictPar
{\md Risque d’accident:}
{\lt N’intervenez en aucun cas au||dessous de la cuve avant d’avoir placé les sécurités anti||écrasement sur les vérins hydrauliques de la cuve.}
\stopPictPar
\stoptextbackground

\stopbuffer

\getbuffer [prescription:container:tilt]


\stopcomponent














