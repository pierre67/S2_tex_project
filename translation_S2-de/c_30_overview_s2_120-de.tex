\startcomponent c_30_overview_s2_120-de
\product prd_ba_s2_120-de

\chapter{Überblick über das Fahrzeug}

\setups [pagestyle:marginless]


\placefig [here] [] {Überblick linke Fahrzeugseite}
{\externalfigure [overview:side:left:de]}


\page [yes]


\placefig [here] [] {Überblick rechte Fahrzeugseite}
{\externalfigure [overview:side:right:de]}

\page [yes]

\setups [pagestyle:normal]


\section{Allgemeines}

\placefig[margin][p4_vue_01]{\sdeux\ bei der Überführung}
{%
\startcombination [1*3]
{\externalfigure[overview:vhc:01]}{}
{\externalfigure[overview:vhc:02]}{}
{\externalfigure[overview:vhc:03]}{}
\stopcombination}

Mit dem Kehrfahrzeug \BosFull{sdeux} gibt Boschung seine ganze Erfahrung und Kompetenz, erworben in der Jahrzehnte währenden kontinuierlichen Zusammenarbeit mit seinen treuen Kunden und Partnern, weiter.
Die Anforderungen der Kommunen und Dienstleister sind im Hinblick auf Mobilität und Vielseitigkeit im Laufe dieser Zeit enorm gewachsen. Die Entwickler der \sdeux\ haben sich dieser Herausforderung gestellt, geleitet von den Bedürfnissen der Kunden, und gefordert durch die vorausschauenden Verbesserungsvorschläge des Boschung||Kundendienstes.
Aus dieser Synthese von Kundenorientierung und konsequenter Umsetzung der erworbenen Praxiserfahrung ging die \sdeux\ hervor.


\subsection{Innovative Technologie}

Das Kompaktkehrfahrzeug \BosFull{sdeux} zeichnet sich in seiner Klasse besonders aus durch sein geringes Gewicht (2300\,kg), seine hohe Kapazität (Schmutzbehälter der 2,0-m\high{3}-Klasse), seine kompakten Abmessungen (Breite 1,15\,m) und den besonders ergonomischen Arbeitsplatz für den Fahrzeugführer.

Durch die schmale Bauweise wird die \sdeux\ zur \quotation{Überall}||Kehrmaschine für Straßen und Bürgersteige in Städten und Dörfern. Ihr kraftvoller Dieselmotor in Verbindung mit dem kompakten hydrostatischen Antrieb (Radialkolben||Hydromotoren auf die Vorderräder) sorgt jederzeit für höchste Mobilität, unabhängig von der Beschaffenheit des Einsatzorts oder dem Füllgrad des Schmutzbehälters.

Die Hyraulikpumpen werden von einem Dieselmotor des Typs \aW{VW 2.0 CDI} nach Euro-V||Norm angetrieben. Er liefert ein Drehmoment von 285\,Nm bei 1750~Umdrehungen und eine maximale Leistung von 75\,kW bei 3000~Umdrehungen. Hierdurch kann die Maschine bereits bei niedriger Motordrehzahl~– und damit geringer Lärmbelastung~– effektiv eingesetzt werden. Die \sdeux\ verfügt serienmäßig über einen Partikelfilter.


\section{Innovationen im Dienst des Kunden}

Die Knicklenkung der \sdeux\ sorgt für einen geringen Wendekreis und damit für maximale Beweglichkeit. Spezielle Materialien wie Domex® und die vollständig CAD||basierte Entwicklung des Fahrzeugs ermöglichten eine beachtliche Nutzlast von 1200\,kg.

\placefig[margin][overview:cab:frontright]{\sdeux\ einsatzbereit}
{\externalfigure[overview:cab:twoleft][width=\Bildwidth]}

Das rundumverglaste Führerhaus verfügt über zwei komfortable Sitzplätze, ausgestattet mit Drei||Punkt||Sicherheitsgurten. Die \sdeux\ kann wahlweise mit einer Klimaanlage ausgestattet werden.

Mit seiner Höchstgeschwindigkeit von 40\,km/h fügt sich das Fahrzeug problemlos in den Stadtverkehr ein. Durch die komfortablen Vorder- und Hinterachsfederungen ist auch die schlechteste Strecke noch sicher und komfortabel befahrbar.

Das Kehraggregat~– auf zwei Gelenkarmen montiert~– befindet sich vollständig im Blickfeld des Bedieners und der Saugmund ist gut einsehbar vor der Vorderachse positioniert. Ein doppelt schwenkbarer Frontbesen steht als Zusatzausrüstung zur Verfügung.

\page [yes]


\subsection{Schallgedämpftes und komfortables Führerhaus}

Das Führerhaus\index{Führerhaus} der \sdeux\ verfügt über Rechtslenkung und ist für zwei Personen konzipiert. Es ist schallisoliert und auf vibrationsdämpfenden Silentblocs montiert.

Türen und Boden sind verglast, wodurch sich ein umfassendes Sichtfeld ergibt. Die Windschutzscheibe erstreckt sich über die komplette Fahrzeugfront und ermöglicht so die ungehinderte Sicht auf die Arbeit der Besen.

Der Fahrersitz verfügt über eine mechanische oder~– in Option~– pneumatische Federung. Fahrer- und Beifahrersitz sind auf einstellbaren Gleitschienen montiert.


\subsubsubject{Ergonomie}

\startfigtext[right][overview:joy:sideview]{Bedienkonsole}
{\externalfigure[overview:joy:top]}
Die Multifunktionskonsole, zur Linken des Fahrersitzes, macht alle elementaren Funktionen mit einer Hand erreichbar. Die beiden Besen lassen sich unabhängig voneinander mittels zweier Joysticks mit Daumen und Zeigefinger steuern. Die Schalter für die Besen und für den Frontbesen (Option), für die Motordrehzahl, den Tempomaten etc. befinden sich ebenfalls auf der Multifunktionskonsole.
\stopfigtext

Am unteren Rand des Blickfelds des Fahrzeugführers befindet sich ein Touchscreen, der alle wichtigen Informationen zu den Funktionen der Maschine in Echtzeit anzeigt, ohne die Sicht nach außen zu beeinträchtigen.

\placefig[margin][overview:vhc:left]{\sdeux\ vor historischen Gemäuern}
% \placefig[margin][overview:vhc:left]{\sdeux\ sur site historique}
{\externalfigure[overview:vhc:left]}

\page [yes]


\subsubsubject{Führerstand}

Der\index{Führerstand} Fahrstufenwahlhebel (\quotation{Gangschaltung}) befindet sich rechts an der Lenksäule; es stehen zwei Vorwärts- und eine Rückwärtsfahrstufe zur Verfügung. Außen am Fahrstufenwahlhebel befindet sich der Druckknopf zum Umschalten zwischen den beiden Antriebsmodi \aW{Arbeit} und \aW{Fahrt}. Die \sdeux\ muss zum Umschalten nicht angehalten werden. (Mehr dazu im Kapitel \about[sec:using:work], \atpage[sec:using:work].)

\placefig[margin][fig:overview:steeringwheel]{Führerstand}
{\externalfigure[overview:driver:place]}

Bei Rückwärtsfahrten schaltet sich der Monitor der Rückfahrkamera ein und es ertönt ein akustisches Warnsignal (deaktivierbar über das Vpad).

Der Multifunktionshebel an der linken Seite der Lenksäule umfasst den Scheibenwischerschalter (zwei Stufen und Intervall) sowie Licht- und akustische Hupe.

Im Kapitel \about[chap:using] ab \atpage[chap:using] finden Sie Details zu diesen und zu weiteren Funktionen der \sdeux.

\page [yes]

\setups[pagestyle:marginless]


\subsection[overview:brushsystem]{Kehr- und Saugvorrichtung}

\subsubsubject{Besen}

\startfigtext[left][fig:overview:steeringwheel]{Kehr-|/|Saugvorrichtung}
{\externalfigure[system:brush]}
Die Besen\index{Kehren} sitzen auf ausrichtbaren Köpfen, die wiederum auf Gelenkarmen montiert sind. Der beim Kehren aufgewirbelte Staub wird durch Besprühen mit Wasser gebunden: Jeder Besen ist mit zwei Düsen ausgestattet, die ihr Wasser aus dem Frischwasser- oder aus dem Recyclingwassertank bezieht.

Ein Schalter\index{Saugen} der Multifunktionskonsole aktiviert gleichzeitig Besen und Wasserpumpe.\footnote{Zur Wasserpumpe siehe Kapitel \in[chap:using] \about[chap:using], insbesondere \about[sec:using:work], \atpage[sec:using:work].}
Die Positionen der Besen sowie ihre Quer- und Längsneigung lassen sich direkt über den entsprechenden Joystick der Multifunktionskonsole steuern.
\stopfigtext

Die Besen sind durch ein mechanisches und hydraulisches Antikollosionssystem geschützt.


\subsubsubject{Saugmund}

In Arbeitsposition (gesenkt) ruht der Saugmund auf 4~Rollen und bedeckt vollständig die Fläche zwischen den auseinandergefahrenen Besen. Durch seine \quotation{geschleppte} Position ist er bei Kollisionen mit Hindernissen vor mechanischen Beschädigungen weitgehend geschützt. In Rückwärtsfahrt wird der Saugmund automatisch angehoben.

Eine dicke, ersetzbare Gummilippe sorgt für den dichten Abschluss zur Straßenoberfläche hin. Eine elektro||hydraulisch steuerbare Klappe an der Vorderseite des Saugmunds erlaubt die Aufnahme gröberer Schmutzobjekte.


\subsubsubject{Schmutzbehälter}

Der Aluminium||Schmutzbehälter lässt sich bis 50° und auf eine Höhe von 1,5\,m (Ablaufhöhe) hochkippen. In ihn mündet von unten kommend der Saugkanal mit einem Öffnungsdurchmesser von 180\,mm.

Der Ansaugunterdruck wird von einer Hochleistungsturbine erzeugt, die horizontal im Schmutzbehälter montiert ist. Sie verfügt über eine Wartungsklappe für Reinigung und Sichtkontrolle.

In der Verschlussklappe des Schmutzbehälters befinden sich zwei Ansauggitter aus Edelstahl. Sie lassen sich zur Reinigung ohne Werkzeug ausklappen. Die Verschlussklappe lässt sich von Hand entriegeln und öffnen.

Mittels einer Klappe, die sich von Hand umlegen lässt, kann der Luftstrom unkompliziert zwischen Saugkanal und Handsaugschlauch (optionale Ausrüstung) umgeschaltet werden.


\subsection{Befeuchtungsvorrichtung}

\subsubsubject{Frischwassersystem}

Der\index{Kehren+Befeuchtung} Tank aus PE||Guss befindet sich in stehender Position hinter dem Führerhaus. Sein Fassungsvermögen\index{Frischwasser+-Tank} beträgt 190\,l.

Eine elektrische Pumpe (6,5\,l/min) befördert das Wasser zu den Sprühdüsen über jedem der Besen (einschließlich optionalem dritten Besen).


\subsubsubject{Schmutzwasserrecycling}

Das Schmutzwasser läuft durch die Mikroperforationen der Innenwände des Schmutzwasserbehälters, um dann über die Recyclingklappe in den darunterliegenden Recyclingwassertank abzufließen. Der\index{Recyclingwasser+-Tank} Recyclingwassertank fasst 140\,l.

Eine hydraulische Tauchpumpe befördert das Wasser zu den Sprühdüsen im Inneren des Saugmunds und Saugkanals.


\testpage [8]
\subsubsubject{Recyclingwassertank}

Der Recyclingwassertank verfügt über einen Wasser||Hydraulikflüssigkeits||Wärmetauscher mit doppelter Funktion:

\startitemize[width=35mm,style=\md, command={\setupwhitespace[small]}]
\sym{Funktion im Sommer} Das Wasser leitet die Wärme der Hydraulikflüssigkeit über Konvektion zu den Aluminiumwänden des Tanks, von wo sie an die Umgebungsluft abgestrahlt wird.

\sym{Funktion im Winter} Die Hydraulikflüssigkeit erwärmt das Wasser im Tank. Hierdurch ist es möglich,
den Saugkanal sowie den Saugmund auch bei Temperaturen etwas unter dem Gefrierpunkt noch zu besprühen.
\stopitemize


\subsubsubject{Überwachung der Wasser||Füllstände}

\startitemize[width=35mm,style=\md, command={\setupwhitespace[small]}]
\sym{Frischwasser} Bei nicht ausreichendem Füllstand erscheint das Symbol~\textSymb{vpad_water} auf dem Vpad||Bildschirm.
\sym{Recyclingwasser} Wenn sich der Füllstand des Recyclinktanks unterhalb des Wärmetauschers befindet (siehe oben), erscheint des Symbol~\textSymb{vpad_rwater_orange} (gelb) auf dem Vpad||Bildschirm. Bei nicht ausreichendem Füllstand erscheint das Symbol~\textSymb{vpad_rwater} (rot).
\stopitemize


\subsubsubject{Breitbereifung (Option)}

Der Bodendruck\index{Breitbereifung} entspricht dem Reifenfüllduck. Mit einem Reifendruck von 1,8\,bar wird ein Bodendruck von 18\,N/cm²  erreicht. Jedoch wird die Traglast des Reifens für die garantierte Achslast nicht mehr erreicht. Mit 1,8\,bar kann bei 40\,km/h nur noch eine Achslast von 1495\,kg gewährleistet werden. Wird der Reifendruck anders als 3.0\,bar gewählt, liegt die Verantwortung beim Fahrzeughalter.

\subsubsubject{Überlastanzeige (Option)}

Wird das Fahrzeug\index{Überlastanzeige} überladen, erscheint eine Meldung auf dem Vpad. Die Überladung wird mit einem Winkelsensor auf der Hinterachse ermittelt. Standard mässig ist die Überlastanzeige auf 3500\,kg eingestellt, ein Toleranzfeld dieses Wertes ist jedoch zu vermeiden. Diese Einstellung von 3500\,kg kann durch ein Fachbetrieb geändert werden.

\page [yes]
\setups[pagestyle:normal]


\section{Identifizierung des Fahrzeugs}

\subsection{Fahrzeugtypenschild}

Das Fahrzeugtypenschild\index{Identifizierung+Fahrzeug} befindet sich im
Führerhaus, gegenüber der Konsole, unter dem Beifahrersitz (siehe \inF[fig:identity:location], \atpage[fig:identity:location]).


\subsection{Motorcode und -nummer}

Der Motorcode befindet sich auf dem Typenschild des Motors (Aufkleber), auf der gekröpften Metallleitung des Kühlkreises, vorne am Motor (Schmutzbehälter anheben).

Die Motornummer ist auf dem Motor eingraviert (\inF[identity:engine:number]). Sie besteht aus neun alphanumerischen Zeichen: Die ersten drei sind der Motorcode, die sechs folgenden die Seriennummer des Motors.


\placefig[margin][idvhc]{Fahrzeugtypenschild}
{\externalfigure[s2:id:plaque]}

\placefig[margin][identity:engine:code]{Motortypenschild}
{\externalfigure[engine:id:code]}

\placefig[margin][identity:engine:number]{Motornummer}
{\externalfigure[engine:id:number]}

\page [yes]


\subsection [sec:plateWheel]{Rädertypenschild}

Das Typenschild der Felgen und Reifen\index{Reifen+Fülldruck} befindet sich im Führerhaus\index{Felgen+Dimensionen}, unterhalb des Beifahrersitzes.


\subsection{Fahrgestellnummer}

Die Fahrgestellnummer\index{Identifizierung+Fahrgestellnummer} (Chassis||Nummer) ist an der rechten Fahrzeugseite, unter dem Führerhaus, am Fahrgestell eingeschlagen.


\subsection{\symbol[europe][CEsign]-Konformität und -Kennzeichnung}

Das~\symbol[europe][CEsign]-Konformitätszeichen befindet sich im Führerhaus, gegenüber der Konsole, unter dem Beifahrersitz.

Die \sdeux\ erfüllt die grundlegenden Sicherheits- und Gesundheitsanforderungen der Maschinenrichtlinie\index{Zertifikat+CE-Konformität}\index{Maschinenrichtlinie} 2006/42/EG\footnote{Richtlinie 2006/42/EG des Europäischen Parlaments und des Rates vom 17.~Mai 2006}.
% \textrule

\placefig[margin][idpneus]{Reifenfülldruck}
{\externalfigure[identity:tires]}

\placefig[margin][fig:identity:location]{Typenschilder}
{\externalfigure[identity:location]}

\page [yes]
\setups [pagestyle:marginless]


\startsection[title={Technische Daten},
							reference={donnees_techniques}]

\subsection [sec:measurement] {Fahrzeugmaße}

\placefig[here][fig:measurement]{\select{caption}{Breite~– Besen in Ruhestellung oder ausgefahren~–, Länge und Höhe des Fahrzeugs}{Fahrzeugmaße}}
{\Framed{\externalfigure[s2:measurement]}}

\page [yes]

\placefig[here][fig:measurement]{\select{caption}{Höhe des Fahrzeugs mit hochgekipptem Schmutzbehälter}{Höhe des Fahrzeugs}}
{\Framed{\externalfigure[s2:measurement:02]}}

\page [yes]

\starttabulate [|lBw(45mm)|p|l|rw(35mm)|]
\FL
\NC Gruppe\index{Maße} \NC \bf Maß \NC \bf Einheit \NC \bf Wert \NC\NR
\ML
\NC Fahrzeugmaße \NC Länge (über alles) \NC \unite{mm} \NC 4588,00 \NC\NR
\NC\NC Länge mit 3.\,Besen \NC \unite{mm} \NC 5020,00 \NC\NR
\NC\NC Breite des Fahrzeugs \NC \unite{mm} \NC 1150,00 \NC\NR
\NC\NC Breite des Fahrzeugs (über alles) \NC \unite{mm} \NC 1575,00 \NC\NR
\NC\NC Höhe ohne Rundumkennleuchte \NC \unite{mm} \NC 1990,00 \NC\NR
\NC\NC Radstand \NC \unite{mm} \NC 1740,00 \NC\NR
\NC\NC Spurweite \NC \unite{mm} \NC 894,00 \NC\NR
\ML
\NC Kehrbreite \NC Standardbesen \NC \unite{mm} \NC 2300,00 \NC\NR
\NC\NC Mit 3.\,Besen \NC \unite{mm} \NC 2600,00 \NC\NR
\NC\NC Durchmesser Besen \NC \unite{mm} \NC 800,00 \NC\NR
\NC\NC Breite Saugmund \NC \unite{mm} \NC 800,00 \NC\NR
\ML
\NC Lastverteilung \NC Leergewicht\note[weight:empty] Vorderachse \NC \unite{kg} \NC ca. 1100,00 \NC\NR
\NC\NC Leergewicht\note[weight:empty] Hinterachse \NC \unite{kg} \NC ca. 1200,00 \NC\NR
\NC\NC Leergewicht\note[weight:empty] \NC \unite{kg} \NC ca. 2300,00 \NC\NR
\NC\NC Zul. Gesamtgewicht \NC \unite{kg} \NC 3500,00 \NC\NR
\LL
\stoptabulate


\subsection{Spurradius und Kehrradius}

\starttabulate [|lBw(45mm)|p|l|rw(35mm)|]
\FL
\NC Dimension\index{Dimensionen} \NC \bf Maß \NC \bf Einheit \NC \bf Wert \NC\NR
\ML
\NC Spurradius\index{Spurradius}\index{Maß+Spurradius} \NC Minimaler Wenderadius mit Besen \NC \unite{mm}	\NC 3325,00 \NC\NR
\ML
\NC Kehrradius \NC außen \NC \unite{mm} \NC 3425,00 bis 3850,00 \NC\NR
\NC\NC innen \NC \unite{mm} \NC 2025,00 bis 1675,00 \NC\NR
\LL
\stoptabulate

%% TODO en/de/fr: Footnote on preceeding page
\footnotetext[weight:empty]{Standardkonfiguration, mit Fahrer (ca. 75\,kg).}

\placefig[here][pict:steerin_sweeping:radius]{Spur-|/|Wendekreis und Kehrradius}
{\externalfigure[steerin_sweeping:radius]}

\page [yes]


\subsection{Räder und Reifen}

{\sla Standarddimensionen}

\starttabulate[|lBw(45mm)|p|rw(55mm)|]
\FL
\NC Komponente \NC \bf Ausstattung \NC \bf Wert \NC\NR
\ML
\NC Reifen \NC Standarddimensionen \NC 205/70 R 15 C \NC\NR
\ML
\NC Felgen \NC Standarddimensionen \NC 6J\;×\;15 H2 ET 60 \NC\NR
\ML
\NC Reifenfülldruck\index{Reifenfülldruck} \NC Standard, vorne|/|hinten \NC 4,5|/|5,8\,bar \NC\NR
\LL
\stoptabulate

{\sla Breitbereifung}

\starttabulate[|lBw(45mm)|p|rw(55mm)|]
\FL
\NC Komponente \NC \bf Ausstattung \NC \bf Wert \NC\NR
\ML
\NC Reifen\index{Breitbereifung} \NC Breitbereifung \NC 275/60 R 15 107H \NC\NR
\ML
\NC Felgen \NC Breitbereifung \NC 8LB\;×\;15 ET 30 \NC\NR
\ML
\NC Reifenfülldruck\index{Reifenfülldruck} \NC Standard, vorne|/|hinten \NC 3,0|/|3,0\,bar \NC\NR
\LL
\stoptabulate


\subsection{Dieselmotor}

\starttabulate [|lBw(45mm)|l|rp|]
\FL
\NC \bf Gruppe\index{Dieselmotor+Identifizierung} \NC \bf Parameter \NC \bf Wert\NC\NR
\ML
\NC Motortyp \NC \NC VW CJDA TDI 2.0 – 475 NE \NC\NR
\NC Allgemeines \NC 	Taktung \NC Viertakter \NC\NR
\NC\NC Anzahl Zylinder \unite{n} \NC 4 \NC\NR
\NC\NC Bohrung x Hub \unite{mm} \NC 81\;×\;95,5 \NC\NR
\NC\NC Gesamthubraum \unite{cm\high{3}} \NC 1968 \NC\NR
\NC\NC Ventile pro Zylinder \NC 4 \NC\NR
\NC\NC Reihenfolge der Ventilsteuerung \NC 1-3-4-2 \NC\NR
\NC\NC Niedrigste Leerlaufdrehzahl \unite{min\high{−1}} \NC 830 +50/−25 \NC\NR
\NC Leistung|/|Drehmoment \NC Max. Drehzahl \unite{min\high{−1}} \NC 3400 \NC\NR
\NC\NC Max. Leistung \unite{kW} bei \unite{min\high{−1}} \NC 75 bei 3000 \NC\NR
\NC\NC Max. Drehmoment \unite{Nm} bei \unite{min\high{−1}} \NC 285 bei 1750 \NC\NR
\NC Spezifischer Verbrauch\index{Dieselmotor+Verbrauch} \NC Kraftstoff \unite{g/kWh} \NC 224 (bei max. Leistung) \NC\NR
\NC\NC Öl \unite{g/kWh} \NC 0,22 \NC\NR
\NC Kraftstoffanlage \NC Einspritzsystem \NC Direkteinspritzung \quote{Common Rail} \NC\NR
\NC\NC Kraftstoffversorgung \NC Zahnradpumpe \NC\NR
\NC\NC Aufladung \NC Ja \NC\NR
\NC\NC Ladeluftkühlung \NC Ja \NC\NR
\NC\NC Ladedruck \unite{mbar} \NC 1300\NC\NR
\NC Schmierkreis\index{Dieselmotor+Schmierung} \NC Typ \NC Forcierte Schmierung mit Öl-|/|Wassertauscher \NC\NR
\NC\NC Leitungsspeisung \NC Rotorpumpe \NC\NR
\NC\NC Ölverbrauch \unite{Liter/20\,h} \NC <\:0,1 \NC\NR
\NC Kühlkreis\index{Dieselmotor+Kühlung} \NC Gesamtkapazität \unite{l} \NC ca. 12 \NC\NR
\NC\NC Eichdruck Ausdehnungsgefäß \unite{bar} \NC 1,4 \NC\NR
\NC\NC Thermostat (Öffnung) \unite{°C} \NC 87 \NC\NR
\NC\NC Thermostat (voll) \unite{°C} \NC 102 \NC\NR
\NC Abgas \NC Partikelfilter \NC Ja \NC\NR
\NC\NC Abgasaufbereitung \NC Ja \NC\NR
\NC\NC Norm \NC Euro 5 \NC\NR
\LL
\stoptabulate


\subsection{Fahrleistungen}

\starttabulate[|lBw(45mm)|p|l|rw(35mm)|]
\FL
\NC Fahrleistung\index{Fahrleistungen} \NC \bf Konfiguration \NC \bf Einheit \NC \bf Wert \NC\NR
\ML
\NC Geschwindigkeit \NC \aW{Arbeits}modus \NC \unite{km/h} \NC 0 bis 18 (stufenlos) \NC\NR
\NC\NC \aW{Fahr}modus \NC \unite{km/h} \NC 0 bis 40 \NC\NR
\ML
\NC Geschwindigkeitsbegrenzung \NC Einstellbar \NC \unite{km/h} \NC 0 bis 25 \NC\NR
\LL
\stoptabulate


\subsection{Elektrische Anlage}

{\starttabulate [|lw(65mm)|p|rw(30mm)|]
\FL
\NC \bf Gruppe \NC \bf Komponente \NC \bf Wert \NC\NR
\ML
\NC Batterie \NC Bleiakku \NC 12\,V 75\,Ah \NC\NR
\NC Stromversorgung \NC Lichtmaschine \NC 14,8\,V 140\,A \NC\NR
\NC Anlasser \NC Leistung \NC 1,8\,kW \NC\NR
\NC Audioausstattung \NC Radioanschluss\index{Radioanschluss} und Lautsprecher\index{Lautsprecher} \NC Serienausstattung \NC\NR
% \NC Sécurité et surveillance \NC Tachygraphe\index{tachygraphe} \NC En option \NC\NR
% \NC\NC Enregistreur de fin de parcours\index{fin de parcours} \NC En option \NC\NR
\NC Beleuchtungs-/Signaleinrichtungen vorne \NC Standlicht \NC 12\,V 5\,W \NC\NR
\NC\NC Abblendlicht \NC H7, 12\,V 55\,W \NC\NR
\NC\NC Arbeitsscheinwerfer \NC G886, 12\,V 55\,W \NC\NR
\NC\NC Blinklichter \NC 12\,V 21\,W \NC\NR
\NC Beleuchtungs-/Signaleinrichtungen hinten \NC Kombinierte Bremslichter \NC 12\,V 5/21\,W \NC\NR
\NC\NC Blinklichter \NC 12\,V 21\,W \NC\NR
\NC\NC Rückfahrlichter \NC 12\,V 21\,W \NC\NR
\NC\NC Kennzeichenbeleuchtung \NC 12\,V 5\,W \NC\NR
\NC Zusatzbeleuchtung \NC Rundumkennleuchte \NC H1, 12\,V 55\,W \NC\NR
\LL
\stoptabulate
}
\stopsection

\stopcomponent
