\startcomponent c_10_safety_s2_120-de
\product prd_ba_s2_120-de

\marking[chapter]{Sicherheitszeichen}


\chapter{Sicherheitszeichen}

\setups[pagestyle:marginless]

\section{Neue europäische Gefahrstoffkennzeichnung}

{\em Rautenförmig mit weißem Hintergrund und rotem
Rand.}\par\blank[1*medium]
{\em Seit 2008 gilt in der EU die sogenannte
CLP-Verordnung\index{CLP-Verordnung} mit neuen Warnkennzeichen für gefährliche
Stoffe und Produkte.}\par\null

\startSymList \GHSgeneric
\SymList
	\textDescrHead{Gesundheitsgefährdung}
	Warnt vor\index{Gesundheitsgefährdung} Gesundheitsgefährdungen, die nicht zum
	Tod oder einem schweren Gesundheitsschaden führen. Hierzu gehört die Reizung
	der Haut oder die Auslösung einer Allergie. Das Symbol wird auch als Warnung
	vor anderen Gefahren, wie Entzündbarkeit genutzt.\par
	Ersetzt:\crlf \HAZOcross\ oder \HAZOpoison\ oder \PHgeneric
\stopSymList

\startSymList \GHSbody
\SymList
	\textDescrHead{Schwere Gesundheitsgefärdung; kann insbesondere bei Kindern auch
	zum Tod führen}
	Produkte können schwere Gesundheitsschäden verursachen. Dieses Symbol warnt
	auch vor Gefährdungen\index{Gefahr+Schwangerschaft} der Schwangerschaft,
	vor krebserregenden Wirkungen\index{Gefahr+Krebserregende Stoffe} und ähnlich
	schweren Gesundheitsrisiken. Produkte sind mit Vorsicht zu benutzen.\par
	Ersetzt:\crlf \HAZOcross\ oder \HAZOpoison\
\stopSymList

\startSymList \GHSbomb
\SymList
	\textDescrHead{Explosivstoffe}
	Instabile explosive\index{Gefahr+Explosion} Stoffe,
	Gemische und Erzeugnisse mit Explosivstoffen\index{Explosivstoffe} haben bei
	ihrer Reaktion eine heftige expandierende Wirkung, die erhebliche Zerstörungen
	anrichten kann; bei unsachgemäßem Umgang besteht Lebensgefahr.\par
	Ersetzt:\crlf \HAZObomb\
\stopSymList


\startSymList \GHSpoison
\SymList
	\textDescrHead{Vergiftung}
	Produkte\index{Gefahr+Vergiftung} können selbst in kleinen Menge auf der Haut,
	durch Einatmen\index{Giftstoffe} oder durch Verschlucken zu schweren oder gar
	tödlichen Vergiftungen führen. Keinen direkten Kontakt zulassen.\par
	Ersetzt:\crlf \HAZOpoison\
\stopSymList

\startSymList \GHSfire
\SymList
	\textDescrHead{Leicht- oder hochentzündlich}
	Produkte\index{Gefahr+Feuer} entzünden sich schnell in der Nähe von
	Hitze oder Flammen. Sprays mit dieser Kennzeichnung dürfen keineswegs auf heiße
	Oberflächen oder in der Nähe offener Flammen versprüht werden.\par
	Ersetzt:\crlf \HAZOfire\ oder \HAZOfirebis\
\stopSymList

\startSymList \GHSenvironment
\SymList
	\textDescrHead{Gefährdung für Tiere und Umwelt}
	Produkte\index{Umweltschutz} können in der Umwelt
	kurz- oder langfristig Schäden\index{Giftstoffe} verursachen. Sie können im
	Wasser lebende Organismen (\eG\ Fische) töten oder auch längerfristig in der
	Umwelt schädlich wirken. Keinesfalls ins Abwasser oder den Hausmüll geben!\par
	Ersetzt:\crlf \HAZOenvironment\
\stopSymList

\startSymList \GHScorrosive
\SymList
	\textDescrHead{Gefährdung von Haut oder Augen}
	Produkte\index{Gefahr+Hautverletzung}\index{Gefahr+Augenverletzung} können bereits nach
	kurzem Kontakt Hautflächen schädigen und zu Narbenbildung führen oder die Augen
	dauerhaft schädigen. Schützen Sie beim Gebrauch Haut und Augen!\par
	Ersetzt:\crlf \HAZOcross\ oder \HAZOcorrosive
\stopSymList

\page [yes]


\section{Warnzeichen}

{\em Schwarze Schrift auf gelbem Hintergrund}\par\null

\startSymList \PHgeneric
\SymList
	\textDescrHead{Allgemeines Warnzeichen}
	Weist\index{Gefahr+allgemein}\index{Warnzeichen} auf eine unmittelbar
	drohende Gefahr hin, bei der Sie sich oder andere Personen verletzen könnten.
	\crlf\null
\stopSymList

\startSymList \PHpoison
\SymList
	\textDescrHead{Warnung vor giftigen Stoffen}
	Giftige Stoffe\index{Gefahr+Vergiftung} können
	durch Hautkontakt, Einatmen oder Einnahme akute oder chronische
	Gesundheitsschäden erheblichen Ausmaßes oder sogar den Tod verursachen.
\stopSymList

\startSymList \PHfire
\SymList
	\textDescrHead{Warnung vor feuergefährlichen Stoffen}
	Offene Flammen und Funkenbildung\index{Gefahr+Feuer} vermeiden. Stoff
	ist leicht entzündlich oder kann brandbeschleunigend wirken. Rauchverbot!
\stopSymList

\startSymList \PHexplosive
\SymList
	\textDescrHead{Warnung vor explosionsgefährlichen Stoffen}
	Feste, flüssige oder gelartige Stoffe oder Zubereitungen, die unter Einwirkung
	von Stoß, Reibung, Feuer, Hitze u.\,Ä. explodieren
	können.\index{Gefahr+Explosion} Rauchverbot!
\stopSymList

\startSymList \PHcrushing
\SymList
	\textDescrHead{Warnung vor Quetschgefahr}
	Weist auf einen Bereich\index{Gefahr+Quetschung} hin, in dem aufgrund sich
	bewegender mechanischer Teile Quetschgefahr besteht. Halten Sie sich von diesem
	Bereich fern, solange die Vorrichtung eingeschaltet ist.
\stopSymList

\startSymList \PHhand
\SymList
	\textDescrHead{Warnung vor Handverletzungen}
	Es besteht die Gefahr, dass\index{Gefahr+Quetschung} Hände oder andere
	Körperteile\index{Gefahr+Handverletzungen} gequetscht werden, \eG\ während des
	Kippens von Führerhaus oder Ladebrücke.
\stopSymList

\startSymList \PHentangle
\SymList
	\textDescrHead{Warnung vor gegenläufigen Rollen / vor Einzugsgefahr}
	Es besteht die Gefahr, dass Gliedmaßen\index{Gefahr+Einzug} von rotierenden
	Teilen erfasst und eingezogen werden. Halten Sie sich fern, solange die
	Vorrichtung eingeschaltet ist.
\stopSymList

\startSymList \PHcorrosive
\SymList
	\textDescrHead{Warnung vor ätzenden Stoffen}
	Vorsichtig\index{Gefahr+ätzende Stoffe} handhaben, angemessene
	Personenschutzausrüstung tragen (Handschuhe, Schutzbrille, Schutzkleidung).
\stopSymList

\startSymList \PHhot
\SymList
	\textDescrHead{Warnung vor heißer Oberfläche}
	Nähern Sie sich dem Bauteil oder der Vorrichtung\index{Gefahr+Verbrennung}
	nicht ohne ausreichende Kenntnisse; Handschuhe tragen.
\stopSymList

\startSymList \PHvoltage
\SymList
	\textDescrHead{Warnung vor gefährlicher elektrischer Spannung}
	Nicht mit Metallgegenständen\index{Gefahr+elektrische Spannung} berühren.
	Verletzungs- oder Verbrennungsgefahr bei Kurzschluss!
\stopSymList

\startSymList \PHfalling
\SymList
	\textDescrHead{Warnung vor Absturzgefahr}
	In diesem Bereich besonders\index{Gefahr+Absturz} vorsichtig sein,
	angemessenes Schuhwerk tragen (mit rutschfesten Sohlen, kohlenwasserstofffest).
\stopSymList

\startSymList \PHbattery
\SymList
	\textDescrHead{Warnung vor Gefahr durch Batterien} Weist auf Gefahren hin, die
	beim Laden von Batterien (Bleiakkus)\index{Gefahr+Batterie} entstehen,
	insbesondere durch austretendes Wasserstoffgas und die in den Batterien
	enthaltene Schwefelsäure.
\stopSymList

\startSymList \PHremote
\SymList
	\textDescrHead{Warnung vor automatischem Anlauf}
	Warnt vor\index{Gefahr+automatischer Anlauf} möglichem automatischen oder
	ferngesteuerten Anlauf einer Vorrichtung.
\stopSymList

% \startSymList \PHquetschgefahr
% \SymList
% \textDescrHead{Risque d’écrasement}
% Risque d’écrasement\index{risque d’écrasement}.
% \stopSymList
% % NOTE: Doppelt! (auch Bilddatei)
%
% % NOTE: Evtl. Folgendes als Ersatz für oben?

% \startSymList\PHhandcrushed
% \SymList
	% \textDescrHead{Gefahr von Handquetschungen}
	% Es besteht\index{Gefahr+Quetschung} die Gefahr, dass Hände oder Finger
	% gequetscht werden. Nähern Sie die Hände nicht an, ohne die Gefahr
	% identifiziert und beseitigt zu haben.
% \stopSymList

\startSymList \PHhandfoot
\SymList
\textDescrHead{Warnung vor sich bewegenden Bauteilen}
Warnt vor Maschinen-|/|Fahrzeugteilen in Bewegung.
\index{Gefahr+Teile in Bewegung}.
\stopSymList

\startSymList \PHnarrowed
\SymList
	\textDescrHead{Warnung vor verengter Fahrbahn}
	Verengte\index{Gefahr+Fahrzeugbreite} Fahrbahn.
	% Denken Sie an die Breite des Fahrzeugs.
\stopSymList

\page [yes]


\section{Verbotszeichen}

{\em Rund mit weißem Hintergrund, rotem Rand und Diagonalbalken}
\par\null


\startSymList \PPfire
\SymList
	\textDescrHead{Feuer, offenes Licht und Rauchen verboten} Offenes
	Feuer\index{Verbot+Rauchen, Feuer} und Glut in jeder Form sind untersagt (\eG\
	brennende Zigarette, Streichholz, Kerze; auch Funkenbildung jeder Art).
\stopSymList

\startSymList \PPentry
\SymList
	\textDescrHead{Zutritt für Unbefugte verboten}
	Unbefugte\index{Verbot+Zutritt} Personen dürfen diesen
	Bereich nicht betreten oder sich ihm nähern.
\stopSymList

\startSymList \PPphone
\SymList
	\textDescrHead{Mobilfunk verboten}
	Mobiltelefone\index{Verbot+Mobilfunk} und jedwede Geräte, die
	elektromagnetische Strahlung emittieren, müssen ausgeschaltet sein. Die
	elektromagnetische Strahlung kann zu Funktionsstörungen der Geräteelektronik
	führen.
\stopSymList

\startSymList \PPspray
\SymList
	\textDescrHead{Mit Wasser spritzen verboten}
	Richten Sie einen Wasser- oder Dampfstrahl\index{Verbot+Wasserstrahl, Dampf} nie auf
	empfindliche Teile und Gerätschaften (\eG\ Sensoren, Steuergeräte,
	Einspritzanlage usw.).
\stopSymList

\startSymList \PPchildren
\SymList
	\textDescrHead{Kinder fern halten}
	Hinweis\index{Verbot+Kinder} auf eine besondere Gefahr für Kinder. Allgemein
	gilt: Kinder dürfen sich einer eingeschalteten Maschine nicht nähern, auch
	nicht bei Wartungsarbeiten.
\stopSymList

\startSymList \PPwater
\SymList
	\textDescrHead{Kein Trinkwasser}
	Das Wasser aus dem Tank\index{Verbot+Kein Trinkwasser} nicht trinken. Es
	besteht Vergiftungsgefahr.
\stopSymList

% \page [yes]


\section{Umweltschutzzeichen}

\startSymList \PSrecycle
\SymList
	\textDescrHead{Recycling}
	Spezifische Vorschriften zur ordnungsgemäßen Entsorgung bestimmter Abfälle.
\stopSymList

\startSymList \PSwelt
\SymList
	\textDescrHead{Umweltschutz}
	Hinweis auf geltende Umweltschutzbestimmungen.
\stopSymList

\startSymList \PStrash[width=\PictoHeight,height=,]
\SymList
	\textDescrHead{Abfälle vorschriftsmäßig entsorgen}
	Für bestimmte Abfälle, \eG\ Bleiakkus, gelten spezielle
	Entsorgungsvorschriften.
\stopSymList


\testpage[12]


\section{Gebotszeichen}


{\em Rund mit blauem Hintergrund}\par\null

\startSymList \PMgeneric
\SymList
	\textDescrHead{Allgemeines Gebotszeichen}
	Dieses Zeichen darf nur in Verbindung mit einem Zusatzzeichen, welches das
	Gebot präzisiert, verwendet werden.
\stopSymList


\startSymList \PMrtfm
\SymList
	\textDescrHead{Gebrauchsanweisung beachten}
	Vor der Inbetriebnahme müssen unbedingt\index{Gebrauchsanweisung lesen} die
	Anweisungen zu diesem Thema, zu einem bestimmten Gerät oder Produkt gelesen
	werden. Die Gebrauchsanweisung muss griffbereit im Führerhaus aufbewahrt
	werden.
\stopSymList

\startSymList \PMproteyes
\SymList
	\textDescrHead{Augenschutz benutzen}
	Bei Arbeiten, bei denen Verletzungsgefahr für die Augen besteht, muss
	Augenschutz\index{Augenschutz} getragen werden.
\stopSymList

\startSymList \PMprothands
\SymList
	\textDescrHead{Handschutz benutzen}
	Bei Arbeiten, bei denen Verletzungen im Handbereich entstehen könnten, müssen
	Schutzhandschuhe\index{Handschutz benutzen} getragen werden.
\stopSymList

\startSymList \PMprotears
\SymList
	\textDescrHead{Gehörschutz benutzen}
	Es muss Gehörschutz\index{Gefahr+Gehör} getragen werden (\eG in der Nähe eines
	laufenden Ventilators oder einer laufenden Turbine).
\stopSymList

\startSymList \PMsafetybelt
\SymList
	\textDescrHead{Sicherheitsgurt benutzen} Legen\index{Sicherheitsgurt} Sie für
	Ihre Sicherheit den Sicherheitsgurt an.
\stopSymList

\section{Zusatzzeichen}

% \adaptlayout[height=+5mm]                                                 {{{

% \startSymList \SETshoe
% \SymList
% \textDescrHead{Port de chaussures de sécurité obligatoire}
% Le port de chaussures de sécurité est obligatoire\index{chaussures de sécurité}.
% \stopSymList
%
% \startSymList \SETglasses
% \SymList
% \textDescrHead{Port de lunettes des protection obligatoire}
% Le port de lunettes est obligatoire\index{lunette de protection}.
% \stopSymList
%
% \startSymList \SEToreillettes
% \SymList
% \textDescrHead{Port de casque obligatoire}
% Le port d’un casque de protection est \index{casque} obligatoire.
% \stopSymList
%
% \startSymList \SETgloves
% \SymList
% \textDescrHead{Port de gants de protection obligatoire}
% Le port de gants de protection est obligatoire\index{gants}.
% \stopSymList
%
% \startSymList \SETmainecrase
% \SymList
% \textDescrHead{Risque d’écrasement}
% Danger pour les mains\index{écrasement} et les pieds.
% \stopSymList
%
% \startSymList \SETgetriebe
% \SymList
% \textDescrHead{Risque de happement}
% Risque de happement par\index{happement} des pièces en rotation.
% \stopSymList
%
% \startSymList \SETradkeil
% \SymList
% \textDescrHead{Cale de roue}
% Sécuriser le véhicule contre toute mise\index{Cale de roue} en marche involontaire.
% \stopSymList
%}}}

\startSymList \SETfirstaid
\SymList
\textDescrHead{Erste Hilfe}
	Zeigt den Aufbewahrungsort der Erste||Hilfe||Ausrüstung. Die rasche Alarmierung
	des Rettungsdienstes ist ein wichtiger Bestandteil der Ersten Hilfe.\index{Erste
	Hilfe}\index{Notruf} Tragen Sie hier Ihre Notrufnummern ein:
	\fillinrules[n=1]{\bf
	\framed[align=right,frame=off,offset=none,width=30mm]{Rettungsdienst}}
\fillinrules[n=1]{\bf
\framed[align=right,frame=off,offset=none,width=30mm]{Polizei}}
\fillinrules[n=1]{\bf
\framed[align=right,frame=off,offset=none,width=30mm]{Feuerwehr}}
\stopSymList

\startSymList \SETbrandschutzzeichen
\SymList
\textDescrHead{Feuerlöscher}
	Bestimmte Geräte sind mit einem oder mehreren Feuerlöschern\index{Feuerlöscher}
	ausgestattet. Diese bedürfen in der Regel einer speziellen Wartung; weitere
	Informationen hierzu finden Sie auf dem Gerät oder in den Gebrauchshinweisen
	des Geräts.
\stopSymList


\page[yes]

\section{Die drei Schritte der Hilfeleistung}
% NOTE [tf]: Shouldn't be in this book, IMO

\starttextbackground [CB]
\textDescrHead{Sichern Sie den Unfallort und die betroffenen Personen}
\startitemize
\item  Überprüfen Sie die Sicherheit des Unfallorts und stellen Sie sicher, dass keine weiteren Gefahren auftreten.
\stopitemize
\textDescrHead{Ermessen Sie den Zustand der Verletzten}
\startitemize
\item  Überprüfen Sie, ob die Verletzten bei Bewusstsein sind und normal atmen.
Befreien Sie ggf. den Atemweg.
\stopitemize
\textDescrHead{Verständigen Sie die Rettungskräfte}
\startitemize Ihr Notruf muss folgende Informationen beinhalten:\par
	\item Die Telefonnummer, unter der Sie zu erreichen sind.
	\item Die Art des Vorfalls (Krankheit, Unfall).
	\item Bestehende Risiken (Brand, Explosion, Einsturzgefahr etc.).
	\item Der genaue Ort des Vorfalls.
	\item Die Zahl der Verletzten und ihre Zustände.
	\item Hilfemaßnahmen, die bereits getroffen wurden.
	\item Antworten Sie auf weitere Fragen, die Ihnen gestellt werden.
\stopitemize
\stoptextbackground

\stopcomponent

