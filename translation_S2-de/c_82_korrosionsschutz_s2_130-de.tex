\startcomponent c_08i_rust_maintenance-de

\setups[pagestyle:marginless]


\section[sec:anticorrosion]{Korrosionsprävention}

Korrosionsprävention\startregister[index][vhc:corrosion]{Wartung+Korrosionsprävention}
bei Winterdienst- und Reinigungsfahrzeugen ist ein Muss und gleichzeitig eine
Herausforderung. Bei der gegenwärtigen Tendenz hin zu immer aggressiveren
Taumitteln gilt dies in verstärktem Maße. Eine wirkungsvolle
Korrosionsprävention beginnt im Herstellungsprozess des Fahrzeugs und endet
erst mit der Außerdienststellung des Fahrzeugs.

Korrosionsprävention lässt sich grob in die folgenden Phasen und Komponenten
unterteilen:

\startSteps
	\item Korrosionsschutzmaßnahmen bei der Produktion des Fahrzeugs
	\item Erhaltung|/|Erneuerung des Korrosionsschutzes
	\item Fachgerechte Fahrzeugreinigung
\stopSteps


\subsection {Korrosionsschutzmaßnahmen bei der Produktion des Fahrzeugs}

Durch sorgfältige Oberflächenbearbeitung und -beschichtung schafft
\boschung\ die bestmöglichen Voraussetzungen für eine lange
Korrosionsfreiheit aller metallenen Bauteile.


\subsubsection{Oberflächenbearbeitung}

Jedes korrosionsanfällige Bauteil wird vor der eigentlichen
Oberflächenbehandlung einer gründlichen Oberflächenreinigung unterzogen. Je
nach Bauteil geschieht dies entweder durch Beizentfettung, Sandstrahlen,
Zinkphosphatierung oder andere geeignete mechanische oder chemische Verfahren.


\subsubsection {Oberflächenbeschichtung}

Aufbau und Beschaffenheit der Fahrzeuglackierung bilden die Grundlage eines dauerhaften Korrosionsschutzes.
Jedes Bauteil des Fahrzeugs ist~– soweit technisch möglich~–
durch zwei Beschichtungen mit einer Gesamtdicke von mindestens 100\,µm geschützt.


\subsubsubsubject{Grundierung}

Die Grundierung bildet die erste Schicht der Fahrzeuglackierung und besteht
aus Zweikomponenten||Epoxidharz bzw. Zweikomponenten||Polyurethan mit einer
Schichtdicke von mindestens 50\,µm. Unmittelbar vor dem Auftrag der
Grundierung werden alle Oberflächen mithilfe von Entfettungsflüssigkeit
gründlich entfettet.


\subsubsubsubject{Lackierung}

Nach dem vollständigen Abtrocknen der Grundierung wird die
Zweikomponenten||Polyurethan||Deckschicht~– ebenfalls mit einer Mindestdicke
von 50\,µm~– aufgetragen.  Die grundierte Oberfläche wird zuvor entfettet.


\subsubsection{Prüfung der Beschichtung}

Nach Abschluss der Lackierung wird diese visuell auf Krater, Blasen und
Einschlüsse geprüft. Die Haftfestigkeit wird mittels standardisierter
Verfahren überprüft.


\subsubsection{Korrosionsschutzwachs (Option)}

Das Fahrzeug ist mit einem umfassenden Oberflächenkorrosionsschutz versiegelt.
Zusätzlich werden bei der Montage besonders sensible und bei der Reinigung
nur schwer zugängliche Stellen punktuell mit einem speziellen
Korrosionsschutzwachs behandelt. Dieses verfügt über haftende und kriechende
Eigenschaften, sodass auch äußerst schwer zugängliche Stellen maximalen Schutz
erfahren.


\subsection{Erhaltung/Erneuerung des Korrosionsschutzes}

Die Korrosionsprävention endet nicht mit der Auslieferung des Fahrzeugs. Um
die Wirksamkeit des Korrosionsschutzes eines Fahrzeugs über einen längeren
Zeitraum zu gewährleisten, ist regelmäßiges, fachkundiges und aktives Handeln
nötig.


\subsubsection{Entfernen von korrosiven Substanzen}

Größere Anhaftungen oder Ansammlungen von korrosiven Substanzen sind sofort vom Fahrzeug zu entfernen.
Typische Situation ist das Befüllen des Fahrzeugs mit Streu- oder Taumitteln: Oft ist es unvermeidbar,
dass beim Befüllen etwas \quotation{daneben geht}. Diese verschütteten Taustoffe müssen unmittelbar
nach Beendigung des Befüllvorgangs vom Fahrzeug entfernt werden. Es reicht {\em nicht},
die verschütteten Taustoffe im Rahmen der nächsten, regulären Reinigung zu entfernen.
Die Gefahr ist zu groß, dass sich die Substanzen in der Zwischenzeit bis in unzugängliche
Bereiche des Fahrzeugs verteilen, wo sie später nur noch unter erhöhtem Zeitaufwand entfernt werden könnten.


\subsubsection{Kontrolle und Ausbessern des Korrosionsschutzes}

Es ist unvermeidlich, dass im laufenden Betrieb Lackierung und
Korrosionsschutzwachs des Fahrzeugs Schaden nehmen. Dies geschieht durch
Kontakt mit harten Objekten, wie der Fahrbahnoberfläche, hartem (harschem)
Schnee, durch umhergeschleuderten Kehrstaub oder auch bei unbeabsichtigten
Kollisionen oder durch Schleifabnutzung beweglicher Teile.  Hieraus ergibt
sich die Notwendigkeit einer regelmäßigen Sichtkontrolle von Lackschicht und
Korrosionsschutzwachs. Werden Beschädigungen gefunden, sind diese durch
fachgerechte Nachlackierung bzw. Ausbesserung des Korrosionsschutzes
unverzüglich zu beheben. Dies gilt auch für kleine Lackschäden.

Alle Sichtkontrollen sind bei gereinigtem Fahrzeug durchzuführen.


\subsubsection{Saisonale Überprüfung des Korrosionsschutzes}

Am Ende der Saison ist der Zustand des Korrosionsschutzes von einem
qualifizierten Fachbetrieb gründlich zu überprüfen und ggf. nachzubessern. Bei
ganzjährig genutzten Fahrzeugen muss dies 2-mal pro Jahr erfolgen.

Diese Kontrolle muss sich auf den Zustand der Lackierung erstrecken, auf den
Zustand der punktuellen Versiegelungen mittels Korrosionsschutzwachs sowie auf
den Zustand der umfassenden Flächenversiegelung.  Die Kontrolle ist bei
gereinigtem Fahrzeug durchzuführen.

In Abhängigkeit von den Einsatzbedingungen kann es nach mehrjährigem Betrieb
auch notwendig sein, anstelle von Nachbesserungen eine Erneuerung
durchzuführen.

\stopregister[vhc:corrosion]


\stopcomponent