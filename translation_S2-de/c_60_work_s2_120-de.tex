\startcomponent c_60_work_s2_120-de
\product prd_ba_s2_120-de


\startchapter [title={Die S2 im Alltag},
							reference={chap:using}]

\setups [pagestyle:marginless]


% \placefig[margin][fig:ignition:key]{Clé de contact}
% {\externalfigure [work:ignition:key]}
\startregister[index][chap:using]{Inbetriebnahme}

\startsection [title={Inbetriebnahme},
							reference={sec:using:start}]


\startSteps
\item Stellen Sie sicher, dass die regulären Kontrollen und Wartungen vorschriftsmäßig durchgeführt wurden.
\item Starten Sie mithilfe des Zündschlüssels den Motor: Zündung einschalten, dann Schlüssel im Uhrzeigersinn weiterdrehen und halten, bis der Motor startet (nur möglich wenn Fahrstufenwahlhebel auf Neutral).
\stopSteps

\start
\setupcombinations [width=\textwidth]

\placefig[here][fig:select:drive]{Fahrstufenwahlhebel}
{\startcombination [2*1]
{\externalfigure [work:select:fDrive]}{Wahlhebel in Stellung \aW{Vorwärtsfahrt}}
{\externalfigure [work:select:rDrive]}{Wahlhebel in Stellung \aW{Rückwärtsfahrt}}
\stopcombination}
\stop


\startSteps [continue]
\item Drehen Sie den Schalter des Fahrstufenwahlhebels, um im \aW{Fahr}modus eine Fahrstufe einzulegen:
\startitemize [R]
\item Erste Stufe
\item Zweite Stufe (Automatikbetrieb; startet automatisch in erster Stufe)
\stopitemize

oder drücken Sie auf den Knopf außen am Hebel, um den \aW{Arbeits}modus zu aktivieren|/|deaktivieren.
\stopSteps

\startbuffer [work:config]
\starttextbackground [FC]
\startPictPar
\PMrtfm
\PictPar
Im Arbeitsmodus steht nur die erste Fahrstufe zur Verfügung und der Motor dreht mit 1300\,min\high{\textminus 1}.

Steuern Sie die Motordrehzahl mithilfe der Tasten~\textSymb{joy_key_engine_increase} und~\textSymb{joy_key_engine_decrease} der Multifunktionskonsole.
\stopPictPar
\stoptextbackground
\stopbuffer

\getbuffer [work:config]

\startSteps [continue]
\item Drücken Sie den Fahrstufenwahlhebel nach oben und nach vorne (Vorwärtsfahrt) bzw. nach oben und nach hinten (Rückwärtsfahrt). Siehe Abbildungen oben.
\item Lösen Sie vor dem Beschleunigen die Feststellbremse.
\stopSteps

\starttextbackground [FC]
\startPictPar
\PMrtfm
\PictPar
{\md Lösen Sie die Feststellbremse vollständig!} Die Position des Feststellbremshebels wird von einem elektronischen Sensor überwacht: Wenn die Feststellbremse nicht vollständig gelöst ist, ist die Fahrgeschwindigkeit auf 5\,km/h begrenzt.
\stopPictPar
\stoptextbackground

\startSteps [continue]
\item Drücken Sie langsam das Fahrpedal, um das Fahrzeug in Bewegung zu versetzen.
\stopSteps


%% NOTE: New text [2014-04-29]:
\subsection [sSec:suctionClap] {Saugkanalklappe}

Das Saugsystem erzeugt einen Luftstrom entweder vom Saugmund oder vom Handsaugschlauch (Option) zum Schmutzbehälter.

Eine von Hand zu betätigende Klappe (\inF[fig:suctionClap], \atpage[fig:suctionClap]) erlaubt es, den Luftstrom zwischen Saugmund und Handsaugschlauch umzuschalten.

\placefig [here] [fig:suctionClap] {Saugkanalklappe}
{\startcombination [2*1]
{\externalfigure [work:suctionClap:open]}{Saugkanal geöffnet}
{\externalfigure [work:suctionClap:closed]}{Saugkanal geschlossen}
\stopcombination}

Im Normalbetrieb~– Arbeiten mit dem Saugmund~– muss der Saugkanal geöffnet sein (Umschalthebel zeigt nach oben).

Um den Handsaugschlauch einsetzen zu können, muss der Saugkanal geschlossen sein (Umschalthebel zeigt nach unten). Auf diese Weise wird der Luftstrom  durch den Handsaugschlauch geleitet.
%% End new text

\stopsection


\startsection [title={Außerbetriebnahme},
							reference={sec:using:stop}]

\index{Außerbetriebnahme}

\startSteps
\item Aktivieren Sie die Feststellbremse (Hebel zwischen den Sitzen) und legen Sie den Fahrstufenwahlhebel auf \aW{Neutral}.
\item Führen Sie die erforderlichen Kontrollarbeiten~– tägliche und ggf. wöchentliche Kontrollen~– durch wie auf \atpage[table:scheduledaily] beschrieben.
\stopSteps

\getbuffer [prescription:handbrake]

\stopsection


\startsection [title={Kehren und Saugen},
							reference={sec:using:work}]

\startSteps
\item Führen Sie die\index{Kehren} Inbetriebnahme des Fahrzeugs durch wie in  \in{§}[sec:using:start], \atpage[sec:using:start] beschrieben.
\item Aktivieren Sie\index{Saugen} den \aW{Arbeits}modus (Knopf außen am Fahrstufenwahlhebel).
\stopSteps

% \getbuffer [work:config]
%% NOTE: outcommented by PB

\startSteps [continue]
\item Drücken Sie die Taste~\textSymb{joy_key_suction_brush}, um Turbine und Besen einzuschalten.

{\md Variante:} {\lt Drücken Sie die Taste~\textSymb{joy_key_suction}, um nur mit dem Saugmund zu arbeiten.}

\item Stellen Sie Umdrehungsgeschwindigkeit der Besen mithilfe der Tasten~\textSymb{joy_key_frontbrush_increase}\textSymb{joy_key_frontbrush_decrease} der Multifunktionskonsole ein.

\item Bringen Sie die Besen mithilfe des jeweiligen Joysticks so in Position, dass Sie die optimale Arbeitsbreite erreichen.
\stopSteps

\vfill

\start
\setupcombinations [width=\textwidth]

\placefig[here][fig:brush:position]{Positionieren der Besen}
{\startcombination [2*1]
{\externalfigure [work:brushes:enlarge]}{Besen nach außen|/|innen}
{\externalfigure [work:brush:left:raise]}{Besen auf|/|ab}
\stopcombination}
\stop

\page [yes]


\subsubsubject{Befeuchtung von Besen und Saugkanal}

Betätigen\index{Kehren+Befeuchtung} Sie den Schalter~\textSymb{temoin_busebalais} zwischen den Sitzen:

{\md Position 1:} Wasserpumpe läuft automatisch, solange die Besen aktiviert sind.

{\md Position 2:} Wasserpumpe läuft permanent. (Nützlich \eG\ für Einstellarbeiten.)


\subsubsubject{Grobschmutz}

\startSteps [continue]
\item Wenn die Gefahr besteht, dass größere Schmutzobjekte (\eG\ PET||Flaschen) den Saugmund blockieren, öffnen\index{Grobschmutzklappe} Sie die Grobschmutzklappe mittels der seitlichen Tasten der Multifunktionskonsole oder~– wenn das nicht ausreicht~– heben Sie\index{Saugmund+Grobschmutz} den Saugmund vorübergehend an.
\stopSteps

\start
\setupcombinations [width=\textwidth]

\placefig[here][fig:suctionMouth:clap]{Umgehen mit Grobschmutz}
{\startcombination [2*1]
{\externalfigure [work:suction:open]}{Grobschmutzklappe öffnen}
{\externalfigure [work:suction:raise]}{Saugmund vorübergehend anheben}
\stopcombination}
\stop

\stopsection


\startsection [title={Entleeren des Schmutzbehälters},
							reference={sec:using:container}]

\startSteps
\item Fahren\index{Schmutzbehälter+Entleeren} Sie das Fahrzeug an einen zum Entleeren geeigneten Platz. Achten Sie darauf, dass die geltenden Umweltschutzbestimmungen eingehalten werden.
\item Aktivieren Sie die Feststellbremse und legen Sie den Fahrstufenwahlhebel auf \aW{Neutral}. (Erforderlich zur Freigabe des Behälter||Kippen||Schalters).
\stopSteps

\getbuffer [prescription:container:gravity]

\startSteps [continue]
\item Entriegeln und öffnen Sie die Verschlussklappe des Schmutzbehälters.
\item Betätigen Sie den Schalter~\textSymb{temoin_kipp2} (Mittelkonsole, zwischen den Sitzen), um den Schmutzbehälter aufzukippen.
\item Wenn der Behälter entleert ist, waschen Sie das Innere mit einem Wasserstrahl. Sie können hierzu die integrierte Wasserpistole (optionale Ausstattung) verwenden.
\stopSteps

\start
\setupcombinations [width=\textwidth]
\placefig[here][fig:brush:adjust]{Handhabung des Schmutzbehälters}
{\startcombination [3*1]
{\externalfigure [container:cover:unlock]}{Verriegelung der Verschlussklappe}
{\externalfigure [container:safety:unlocked]}{Sicherungsstrebe}
{\externalfigure [container:safety:locked]}{Sicherungsstrebe verriegelt}
\stopcombination}
\stop

\startSteps [continue]
\item Überprüfen|/|reinigen Sie die Dichtungen und die Auflageflächen der Dichtungen des Behälters, des Recyclingsystems und des Saugkanals.
\stopSteps

\getbuffer [prescription:container:tilt]

\startSteps [continue]
\item Betätigen Sie den Schalter~\textSymb{temoin_kipp2}, um den Schmutzbehälter abzulassen. (Entfernen Sie ggf. vorher die Sicherungsstreben von den Hydraulikzylindern.)
\item Verriegeln Sie die Verschlussklappe des Schmutzbehälters.
\stopSteps

\stopsection


\startsection [title={Handsaugschlauch},
							reference={sec:using:suction:hose}]

Die \sdeux\ kann optional\index{Handsaugschlauch} mit einem Handsaugschlauch ausgerüstet werden. Dieser ist auf der Verschlussklappe des Schmutzbehälters fixiert; seine Bedienung ist unkompliziert.

{\sla Voraussetzungen:}

Der Schmutzbehälter ist vollständig abgesenkt; die \sdeux\ befindet sich im \aW{Arbeits}modus. (Siehe \in{§}[sec:using:start], \atpage[sec:using:start].)

\startfigtext[left][fig:using:suction:hose]{Handsaugschlauch}
{\externalfigure[work:suction:hose]}
\startSteps
\item Drücken Sie die Taste~\textSymb{temoin_aspiration_manuelle} der Deckenkonsole, um das Saugsystem zu aktivieren.
\item Ziehen Sie die Feststellbremse fest an, bevor Sie das Führerhaus verlassen.
\item Schließen Sie den Saugkanal mit der Saugkanalklappe. (Siehe \in{§}[sSec:suctionClap], \atpage[sSec:suctionClap].)
\item Ziehen Sie den Handsaugschlauch am Mundstück aus seiner Halterung und beginnen Sie die Arbeit.
\item Nach Beendigung der Arbeit betätigen Sie erneut die Taste~\textSymb{temoin_aspiration_manuelle}, um das Saugsystem auszuschalten.
\stopSteps
\stopfigtext

\stopsection

\page [yes]

\setups[pagestyle:normal]


\startsection [title={Hochdruckwasserpistole},
							reference={sec:using:water:spray}]

Die \sdeux\ kann optional\index{Wasserpistole} mit einer Hochdruckwasserpistole ausgerüstet werden. Die Wasserpistole ist in der hinteren rechten Wartungstür befestigt und mit einer 10-Meter||Schlauchhaspel~– auf der gegenüberliegenden Fahrzeugseite~– verbunden.

Gehen Sie so vor, um die Wasserpistole einzusetzen:

{\sla Voraussetzungen:}

Im Frischwassertank befindet sich ausreichend Wasser; die \sdeux\ befindet sich im \aW{Arbeits}modus. (Siehe \in{§}[sec:using:start], \atpage[sec:using:start].)

\placefig[margin][fig:using:water:spray]{Hochdruckwasserpistole}
{\externalfigure[work:water:spray]}

\startSteps
\item Drücken Sie die Taste~\textSymb{temoin_buse} der Deckenkonsole, um die Hochdruckwasserpumpe zu aktivieren.
\item Ziehen Sie die Feststellbremse fest an, bevor Sie das Führerhaus verlassen.
\item Öffnen Sie die hintere rechte Wartungstür und nehmen Sie die Wasserpistole heraus.
\item Rollen Sie so viel Schlauch ab wie nötig und beginnen Sie Ihre Arbeit.
\item Nach Beendigung der Arbeit betätigen Sie erneut die Taste~\textSymb{temoin_buse}, um die Hochdruckwasserpumpe auszuschalten.
\item Ziehen Sie kurz am Schlauch, um die Blockierung zu lösen und den Schlauch aufzurollen.
\item Befestigen Sie die Wasserpistole wieder in ihrer Halterung und schließen Sie die Wartungstür.
\stopSteps

\stopsection
\stopregister[index][chap:using]

\stopchapter
\stopcomponent
