\startcomponent c_30_overview_s2_120-sv
\product prd_ba_s2_120-sv

\chapter{Översikt över fordonet}

\setups [pagestyle:marginless]


\placefig [here] [] {Översikt vänster fordonssida}
{\externalfigure [overview:side:left:sv]}


\page [yes]


\placefig [here] [] {Översikt höger fordonssida}
{\externalfigure [overview:side:right:sv]}

\page [yes]

\setups [pagestyle:normal]


\section{Allmänt}

\placefig[margin][p4_vue_01]{\sdeux\ överlämnas}
{%
\startcombination [1*3]
{\externalfigure[overview:vhc:01]}{}
{\externalfigure[overview:vhc:02]}{}
{\externalfigure[overview:vhc:03]}{}
\stopcombination}

Tillsammans med gatsopningsfordonet \BosFull{sdeux} överlämnar Boschung all den erfarenhet och kompetens som vi har samlat på oss under flera decennier genom samarbetet med våra trogna kunder och partner.
Under denna tid har kommunernas och tjänsteleverantörernas krav på fordonens manöverbarhet och mångsidighet ökat enormt. Utvecklarna av \sdeux\ har tagit sig an denna utmaning för att uppfylla kundernas behov, och samtidigt tagit hänsyn till Boschungs förutseende förslag till förbättringar.
\sdeux\ är resultaten av vårt företags kundorientering i kombination med den konsekventa omsättningen av vår praktiska erfarenhet.


\subsection{Innovativ teknik}

Det kompakta gatsopningsfordonet \BosFull{sdeux} utmärker sig i sin klass tack vare låg vikt (2 300\, kg), hög kapacitet (smutsbehållare i 2,0-m\high{3}-klassen), kompakt utförande (bredd 1,15\,m) och en särskilt ergonomisk arbetsplats för fordonets förare.

Den slimmade designen gör \sdeux\ till en riktig allroundmaskin för gator och trottoarer i tätorter. Den kraftfulla dieselmotorn i kombination med den kompakta hydrostatiska framdrivningen (hydrauliska radialkolvmotorer på framhjulen) säkerställer högsta mobilitet i alla situationer, oberoende av hur arbetsplatsen ser ut eller fyllnadsnivån i smutsbehållaren.

Hydraulpumparna drivs av en dieselmotor av typen \aW{VW 2.0 CDI} som uppfyller avgasnormen Euro 5. Den har ett vridmoment på 285\,Nm vid 1750~varv och en högsta effekt på 75\,kW vid 3000~varv. Det innebär att maskinen kan arbeta effektivt redan vid låga varvtal~, vilket i sin tur minskar arbetsbullret~. \sdeux\ är som standard utrustat med ett partikelfilter.


\section{Innovationer för kunden}

Ramstyrningen på \sdeux\ ger en liten vändradie och därmed högsta manöverbarhet. Tack vare specialmaterial som Domex® och en fullständigt CAD-baserad design har man kunnat öka fordonets nyttolast till hela 1 200\,kg.

\placefig[margin][overview:cab:frontright]{\sdeux\ i driftklart skick}
{\externalfigure[overview:cab:twoleft][width=\Bildwidth]}

Den inglasade förarhytten har två bekväma sittplatser som är utrustade med trepunktsbälten. \sdeux\ kan som tillval även utrustas med klimatanläggning.

Med en maxhastighet på 40\,km/h kan fordonet smidigt ta sig fram i stadstrafiken. Den bekväma fjädringen på fram- och bakaxel gör att fordonet är bekvämt och säkert att köra även på mycket ojämna underlag.

Sopaggregatet~ som är monterat på två ledade armar~ befinner sig helt inom förarens synfält. Sugmunstycket är monterat på en synlig position framför framaxeln. En frontkvast som kan vridas i dubbla riktningar kan köpas som tillval.

\page [yes]


\subsection{Vibrationsdämpad och bekväm förarhytt}

Förarhytten i S2 är utrustad\index{Förarhytt}  med högerstyrning och utformad för två personer. Den är ljudisolerad och har monterats på vibrationsdämpande silentblock.

Dörrar och golv är inglasade, vilket ger föraren ett stort synfält. Vindrutan sträcker sig över hela fordonets front och ger obehindrad sikt över de arbetande kvastarna.

Förarstolen är utrustad med mekanisk eller~– som tillval~– pneumatisk fjädring. Förar- och passagerarstolen är monterade på inställbara glidskenor.


\subsubsubject{Ergonomi}

\startfigtext[right][overview:joy:sideview]{Manöverkonsol}
{\externalfigure[overview:joy:top]}
Multifunktionskonsolen befinner sig till vänster om förarstolen så att alla viktiga funktioner kan nås med en hand. De båda kvastarna styrs oberoende av varandra med hjälp av två spakar som manövreras med tummen och pekfingret. Kontakterna för kvastar och frontkvast (tillval), motorvarvtal, farthållare etc. är också placerade på multifunktionskonsolen.
\stopfigtext

Längst ned i förarens synfält finns en pekskärm som visar alla viktiga uppgifter om maskinens funktioner i realtid utan att hindra sikten utåt.

\placefig[margin][overview:vhc:left]{\sdeux\ framför historiska murar}
% \placefig[margin][overview:vhc:left]{\sdeux\ sur site historique}
{\externalfigure[overview:vhc:left]}

\page [yes]


\subsubsubject{Förarplats}

Växelspaken\index{Förarplats} (\quotation{manuell växellåda}) befinner sig på höger sida av rattstången. Det finns två växlar för framåtkörning och en backväxel. På växelspaken finns en knapp för att växla mellan driftlägena \aW{arbete} och \aW{körning}. \sdeux\ behöver inte stannas när man ska växla mellan lägena. (Mer information finns i kapitel \about[sec:using:work], \atpage[sec:using:work].)

\placefig[margin][fig:overview:steeringwheel]{Förarplats}
{\externalfigure[overview:driver:place]}

När fordonet backar slås backkamerans skärm på och det ljuder en akustisk varningssignal (kan aktiveras via Vpad).

På multifunktionsspaken på vänster sida av rattstången finns knappen för vindrutetorkaren (två steg samt intervalltorkning) samt ljustutan och signalhornet.

I kapitel \about[chap:using] fr.o.m. \atpage[chap:using] hittar du närmare information om denna och fler funktioner hos \sdeux.

\page [yes]

\setups[pagestyle:marginless]


\subsection[overview:brushsystem]{Sop- och sugenhet}

\subsubsubject{Kvast}

\startfigtext[left][fig:overview:steeringwheel]{Sop-/sugenhet}
{\externalfigure[system:brush]}
Kvastarna\index{Sopning} sitter på justerbara huvuden som i sin tur är monterade på ledade armar. Damm som virvlas upp under sopning sprutas med vatten så att det lägger sig. Varje kvast är utrustad med två munstycken som tar upp vatten från renvattentanken eller tanken för återvinningsvatten.

En kontakt\index{Uppsugning} på multifunktionskonsolen slår på kvastar och vattenpump samtidigt.\footnote{För mer information om vattenpumpen, se \in{kapitel}[chap:using] \about[chap:using], särskilt \about[sec:using:work], \atpage[sec:using:work].}
Kvastarnas positioner samt lutningen i tvär- och längdled kan styras direkt med respektive spak på multifunktionskonsolen.
\stopfigtext

Kvastarna stöds av ett mekaniskt och hydrauliskt antikollisionssystem.


\subsubsubject{Sugmunstycke}

När sugmunstycket befinner sig i arbetsposition (nedsänkt) vilar den på 4~rullar och täcker helt den yta som uppstår när kvastarna går isär. Tack vare den \quotation{nedsänkta} positionen är sugmunstycket till stor del skyddat mot mekaniska skador vid en kollision. När fordonet backar lyfts munstycket upp automatiskt.

En kraftig, utbytbar gummilist säkerställer att munstycket ligger tätt mot underlaget. Grövre smuts sugs in av en elektrohydraulisk lucka på framsidan av munstycket.


\subsubsubject{Smutsbehållare}

Smutsbehållaren är tillverkad i aluminium och kan tippas upp till 50° och en höjd på 1,5\,m (tömningshöjd). En sugkanal leds underifrån och mynnar ut i smutsbehållaren. Den har en öppningsdiameter på 180\,mm.

Insugsvakuumet genereras av en högeffektsturbin som är monterad vågrätt i smutsbehållaren. På turbinen sitter en underhållslucka för rengöring och underhåll.

I smutsbehållarens lock sitter två insugsgaller i rostfritt stål. De kan öppnas utan verktyg för rengöring. Locket kan låsas upp och öppnas för hand.

Med hjälp av en lucka som kan manövreras för hand kan luftflödet enkelt kopplas om mellan sugkanal och handsugslang (tillval).


\subsection{Befuktningssystem}

\subsubsubject{Renvattensystem}

Tanken\index{Sopning+Befuktning} är tillverkad i gjutplast och befinner sig i stående läge bakom förarhytten. Tanken\index{Renvatten+tank} rymmer 190\, l.

En elpump (6,5\, l/min) pumpar vatten till sprutmunstyckena ovanför varje kvast (inkl. den valfria tredje kvasten).


\subsubsubject{Återvinning av smutsvatten}

Smutsvattnet leds först genom mikrohålen i smutsvattenbehållarens innerväggar för att sedan rinna ut i den underliggande tanken för återvinningsvatten genom återvinningsluckan. Tanken\index{Tank för+ återvinningsvatten} för återvinningsvatten rymmer 140\, l.

En dränkbar hydraulpump pumpar vattnet till sprutmunstyckena på insidan av sugmunstycket och sugkanalen.


\testpage [8]
\subsubsubject{Tank för återvinningsvatten}

Tanken för återvinningsvatten är utrustad med en värmeväxlare med vatten/hydraulvätska och har dubbla funktioner.

\startitemize[width=41mm,style=\md, command={\setupwhitespace[small]}]
\sym{Funktion under sommaren} Vattnet leder värmen från hydraulvätskan genom konvektion till tankens aluminiumväggar, där den avges till omgivningen.

\sym{Funktion under vintern} Hydraulvätskan värmer upp vattnet i tanken. Detta gör det möjligt att spruta sugkanalen och sugmunstycket med vatten även vid temperaturer som understiger fryspunkten.
\stopitemize


\subsubsubject{Övervakning av vattennivåerna}

\startitemize[width=41mm,style=\md, command={\setupwhitespace[small]}]
\sym{Renvatten} Om nivån är för låg visas symbolen~\textSymb{vpad_water} på Vpad-bildskärmen.
\sym{Återvinningsvatten} Om nivån i tanken för återvinningsvatten ligger under värmeväxlaren (se ovan), visas symbolen~\textSymb{vpad_rwater_orange} (gul) på Vpad-bildskärmen. Om nivån är för låg visas symbolen ~\textSymb{vpad_rwater} (röd).
\stopitemize


\subsubsubject{Breddäck (tillval)}

Marktrycket\index{Breddäck} motsvarar däcktrycket. Med ett däcktryck på 1,8\,bar uppnås ett marktryck på 18\,N/cm². Däcken uppnår dock inte längre bärkapaciteten för den garanterade axellasten. Med 1,8\,bar går det vid en hastighet på 40\,km/h endast att garantera en axellast på 1 495\,kg. Om ett annat däcktryck än 3,0\, bar väljs, ligger ansvaret hos fordonets operatör.

\subsubsubject{Indikering av överbelastning (tillval)}

Om fordonet\index{Indikering av överbelastning} överbelastas visas ett meddelande på Vpad. Överbelastningen fastställs med en vinkelsensor på bakaxeln. Som standard är indikeringen av överbelastning inställd på 3500\,kg. Ett toleransområde för detta värde bör undvikas. Inställningen på 3500\,kg kan ändras av en fackverkstad.

\page [yes]
\setups[pagestyle:normal]


\section{Identifiering av fordonet}

\subsection{Fordonets typskylt}

Fordonets typskylt\index{Identifiering+fordon} befinner sig i
förarhytten, under passagerarstolen mittemot konsolen (se \inF[fig:identity:location], \atpage[fig:identity:location]).


\subsection{Motorns kod och nummer}

Motorkoden finns på motorns typskylt (etikett), på den böjda metalledningen på kylkretsen framtill på motorn (lyft smutsbehållaren).

Motornumret är graverat på motorn (\inF[identity:engine:number]). Det består av nio alfanumeriska tecken. De första tre tecknen är motorkoden, de sista sex är motorns serienummer.


\placefig[margin][idvhc]{Fordonets typskylt}
{\externalfigure[s2:id:plaque]}

\placefig[margin][identity:engine:code]{Motorns typskylt}
{\externalfigure[engine:id:code]}

\placefig[margin][identity:engine:number]{Motornummer}
{\externalfigure[engine:id:number]}

\page [yes]


\subsection [sec:plateWheel]{Hjulens typskylt}

Fälgarnas och däckens typskylt\index{Däck+Däcktryck} sitter i förarhytten\index{Fälgar+Mått} under passagerarstolen.


\subsection{Chassinummer}

Chassinumret\index{Identifiering+Chassinummer} är instämplat på chassit på fordonets högra sida under förarhytten.


\subsection{\symbol[europe][CEsign]--märkning }

~\symbol[europe][CEsign]--märkningen hittar du i förarhytten under förarstolen mittemot konsolen.

\sdeux\ uppfyller de grundläggande säkerhets- och hälsokraven i maskindirektivet\index{Intyg+CE-överensstämmelse}\index{Maskindirektivet} 2006/42/EG\footnote{Europaparlamentets och rådets direktiv 2006/42/EG av den 17 ~maj 2006}.
% \textrule

\placefig[margin][idpneus]{Däcktryck}
{\externalfigure[identity:tires]}

\placefig[margin][fig:identity:location]{Typskyltar}
{\externalfigure[identity:location]}

\page [yes]
\setups [pagestyle:marginless]


\startsection[title={Tekniska specifikationer},
reference={donnees_techniques}]

\subsection [sec:measurement] {Fordonsmått}

\placefig[here][fig:measurement]{\select{caption}{Bredd~– kvast i viloläge eller utkört läge~– Längd och höjd på fordonet}{Fordonsmått}}
{\Framed{\externalfigure[s2:measurement]}}

\page [yes]

\placefig[here][fig:measurement]{\select{caption}{Höjd på fordonet med upplyft smutsbehållare}{Höjd på fordonet}}
{\Framed{\externalfigure[s2:measurement:02]}}

\page [yes]

\starttabulate [|lBw(45mm)|p|l|rw(35mm)|]
\FL
\NC Grupp\index{Mått} \NC \bf Mått \NC \bf Enhet \NC \bf Värde \NC\NR
\ML
\NC Fordonsmått \NC Längd (överallt) \NC \unite{mm} \NC 4588,00 \NC\NR
\NC\NC Längd med tredje \,kvast \NC \unite{mm} \NC 5020,00 \NC\NR
\NC\NC Fordonets bredd \NC \unite{mm} \NC 1150,00 \NC\NR
\NC\NC Fordonets bredd (överallt) \NC \unite{mm} \NC 1575,00 \NC\NR
\NC\NC Höjd utan roterande varningsljus \NC \unite{mm} \NC 1990,00 \NC\NR
\NC\NC Axelavstånd \NC \unite{mm} \NC 1740,00 \NC\NR
\NC\NC Spårvidd \NC \unite{mm} \NC 894,00 \NC\NR
\ML
\NC Bredd sopningsområde \NC Standardkvast \NC \unite{mm} \NC 2300,00 \NC\NR
\NC\NC Med tredje \,kvast \NC \unite{mm} \NC 2600,00 \NC\NR
\NC\NC Diameter kvast \NC \unite{mm} \NC 800,00 \NC\NR
\NC\NC Bredd sugmunstycke \NC \unite{mm} \NC 800,00 \NC\NR
\ML
\NC Lastfördelning \NC Tomvikt\note[weight:empty] Framaxel \NC \unite{kg} \NC ca 1100,00 \NC\NR
\NC\NC Tomvikt\note[weight:empty] Bakaxel\NC \unite{kg} \NC ca 1200,00 \NC\NR
\NC\NC Tomvikt\note[weight:empty] \NC \unite{kg} \NC ca 2300,00 \NC\NR
\NC\NC Tillåten totalvikt \NC \unite{kg} \NC 3500,00 \NC\NR
\LL
\stoptabulate


\subsection{Spårradie och sopningsradie}

\starttabulate [|lBw(45mm)|p|l|rw(35mm)|]
\FL
\NC Dimension\index{Dimensioner} \NC \bf Mått \NC \bf Enhet \NC \bf Värde \NC\NR
\ML
\NC Spårradie\index{Spårradie}\index{Mått+Spårradie} \NC Minsta vändradie med kvastar \NC \unite{mm} \NC 3325,00 \NC\NR
\ML
\NC Sopningsradie \NC utsida \NC \unite{mm} \NC 3425,00 till 3850,00 \NC\NR
\NC\NC insida \NC \unite{mm} \NC 2025,00 till 1675,00 \NC\NR
\LL
\stoptabulate

%% TODO en/de/fr: Footnote on preceeding page
\footnotetext[weight:empty]{Standardkonfiguration, med förare (ca 75\,kg).}

\placefig[here][pict:steerin_sweeping:radius]{Spår-/vändradie och sopningsradie}
{\externalfigure[steerin_sweeping:radius]}

\page [yes]


\subsection{Hjul och däck}

{\sla Standarddimensioner}

\starttabulate[|lBw(45mm)|p|rw(55mm)|]
\FL
\NC Komponenter \NC \bf Utrustning \NC \bf Värde \NC\NR
\ML
\NC Däck \NC Standarddimensioner \NC 205/70 R 15 C \NC\NR
\ML
\NC Fälgar \NC Standarddimensioner \NC 6J\;×\;15 H2 ET 60 \NC\NR
\ML
\NC Däcktryck\index{Däcktryck} \NC Standard, fram/bak\NC 4,5/5,8\,bar \NC\NR
\LL
\stoptabulate

{\sla Breddäck}

\starttabulate[|lBw(45mm)|p|rw(55mm)|]
\FL
\NC Komponenter \NC \bf Utrustning \NC \bf Värde \NC\NR
\ML
\NC Däck\index{Breddäck} \NC Breddäck\NC 275/60 R 15 107H \NC\NR
\ML
\NC Fälgar \NC Breddäck \NC 8LB\;×\;15 ET 30 \NC\NR
\ML
\NC Däcktryck\index{Däcktryck} \NC Standard, fram/bak\NC 3,0/3,0\,bar \NC\NR
\LL
\stoptabulate


\subsection{Dieselmotor}

\starttabulate [|lBw(45mm)|l|rp|]
\FL
\NC \bf Grupp\index{Dieselmotor+Identifiering} \NC \bf Parameter \NC \bf Värde\NC\NR
\ML
\NC Motortyp \NC \NC VW CJDA TDI 2.0 – 475 NE \NC\NR
\NC Allmänt \NC Takt \NC Fyrtakts \NC\NR
\NC\NC Antal cylindrar \unite{n} \NC 4 \NC\NR
\NC\NC Borrhål x slag \unite{mm} \NC 81\;×\;95,5 \NC\NR
\NC\NC Total slagvolym \unite{cm\high{3}} \NC 1968 \NC\NR
\NC\NC Ventiler per cylinder \NC 4 \NC\NR
\NC\NC Ordningsföljd för ventilstyrningen \NC 1-3-4-2 \NC\NR
\NC\NC Lägsta tomgångsvarvtal \unite{min\high{−1}} \NC 830 +50/−25 \NC\NR
\NC Effekt/vridmoment \NC Max. varvtal\unite{min\high{−1}} \NC 3400 \NC\NR
\NC\NC Max. effekt\unite{kW} vid\unite{min\high{−1}} \NC 75 vid 3000 \NC\NR
\NC\NC Max. vridmoment \unite{Nm} vid\unite{min\high{−1}} \NC 285 vid 1750 \NC\NR
\NC Specifik bränsleförbrukning\index{Dieselmotor+Förbrukning} \NC Bränsle \unite{g/kWh} \NC 224 (vid maxeffekt) \NC\NR
\NC\NC Olja \unite{g/kWh} \NC 0,22 \NC\NR
\NC Bränslesystem \NC Insprutningssystem\NC Direktinsprutning \quote{Common Rail} \NC\NR
\NC\NC Bränslematning \NC Kugghjulspump\NC\NR
\NC\NC Uppladdning \NC Ja \NC\NR
\NC\NC Laddluftskylning \NC Ja \NC\NR
\NC\NC Laddtryck \unite{mbar} \NC 1300\NC\NR
\NC Smörjkrets \index{Dieselmotor+Smörjning} \NC Typ \NC Forcerad smörjning med olje-/vattenväxlare \NC\NR
\NC\NC Ledningsmatning \NC Rotorpump \NC\NR
\NC\NC Oljeförbrukning \unite{l/20\,h} \NC <\:0,1 \NC\NR
\NC Kylkrets\index{Dieselmotor+Kylning} \NC Total kapacitet \unite{l} \NC ca 12 \NC\NR
\NC\NC Kalibreringstryck expansionskärl \unite{bar} \NC 1,4 \NC\NR
\NC\NC Termostat (öppning) \unite{°C} \NC 87 \NC\NR
\NC\NC Termostat (full) \unite{°C} \NC 102 \NC\NR
\NC Avgas \NC Partikelfilter \NC Ja \NC\NR
\NC\NC Avgasbehandling \NC Ja \NC\NR
\NC\NC Avgasnorm \NC Euro 5 \NC\NR
\LL
\stoptabulate


\subsection{Körprestanda}

\starttabulate[|lBw(45mm)|p|l|rw(35mm)|]
\FL
\NC Körprestanda\index{Körprestanda} \NC \bf Konfiguration \NC \bf Enhet\NC \bf Värde \NC\NR
\ML
\NC Hastighet \NC \aW{Arbets}läge\NC \unite{km/h} \NC 0 till 18 (steglöst) \NC\NR
\NC\NC \aW{Kör}läge \NC \unite{km/h} \NC 0 till 40 \NC\NR
\ML
\NC Hastighetsbegränsning \NC Inställbar\NC \unite{km/h} \NC 0 till 25 \NC\NR
\LL
\stoptabulate


\subsection{Elsystem}

{\starttabulate [|lw(65mm)|p|rw(30mm)|]
\FL
\NC \bf Grupp \NC \bf Komponenter \NC \bf Värde\NC\NR
\ML
\NC Batteri \NC Blybatteri \NC 12\,V 75\,Ah \NC\NR
\NC Strömförsörjning \NC Generator\NC 14,8\,V 140\,A \NC\NR
\NC Startmotor \NC Effekt \NC 1,8\,kW \NC\NR
\NC Ljudutrustning \NC Radioanslutning \index{Radioanslutning} och högtalare\index{Högtalare} \NC Standardutrustning \NC\NR
% \NC Sécurité et surveillance \NC Tachygraphe\index{tachygraphe} \NC En option \NC\NR
% \NC\NC Enregistreur de fin de parcours\index{fin de parcours} \NC En option \NC\NR
\NC Belysnings-/signalanordningar fram \NC Positionsljus \NC 12\,V 5\,W \NC\NR
\NC\NC Halvljus \NC H7, 12\,V 55\,W \NC\NR
\NC\NC Arbetsstrålkastare \NC G886, 12\,V 55\,W \NC\NR
\NC\NC Blinkers \NC 12\,V 21\,W \NC\NR
\NC Belysnings-/signalanordningar bak \NC Kombinerade bromsljus \NC 12\,V 5/21\,W \NC\NR
\NC\NC Blinkers \NC 12\,V 21\,W \NC\NR
\NC\NC Backljus \NC 12\,V 21\,W \NC\NR
\NC\NC Registreringsskyltsbelysning \NC 12\,V 5\,W \NC\NR
\NC Extrabelysning \NC Roterande varningsljus\NC H1, 12\,V 55\,W \NC\NR
\LL
\stoptabulate
}
\stopsection

\stopcomponent

