\startcomponent c_40_control_s2_120-sv
\product prd_ba_s2_120-sv


\startchapter [title={Manöverdon på fordonet},
reference={chap:ctrl}]

\setups[pagestyle:marginless]

\placefig[here][fig:ctrl:cab:front]{Manöverdon}
{\externalfigure[ctrl:cab:front]}

\startcolumns [n=3]
\startLongleg
 \item Rattstång(\in{§}[sec:steeringColumn])
 \item Inställning rattstång
 \item Gas- och bromspedal
 \item Färddator \Vpad SN (\inP[sec:vpad])
 \item Takkonsol (\inP[sec:ctrl:aux])
 \item Radio/MP3
\stopLongleg


\subsubsubject{Extrautrustning}

\startLongleg [continue]
 \item Backövervakning
\stopLongleg
\stopcolumns

\startsection [title={Rattstång},
reference={sec:steeringColumn}]

\subsection{Ställa in rattstången}

\textDescrHead{Rattstångens lutning} Tryck på pedalen \Ltwo och ställ samtidigt in lutningen på rattstången. Släpp pedalen för att låsa rattstångens mekanism igen.

\page[yes]
\setups [pagestyle:normal]


\subsection{Belysnings- och signalanordningar}

\placefig [margin] [fig:column:left] {Multifunktionsspak och vridomkopplare}
{\externalfigure[ctrl:column:left]}

\placefig [margin] [fig:column:right] {Växelspak}
{\externalfigure[ctrl:column:right]}


\subsubsubject{Vridomkopplare}

\startitemize[width=1.7em]
\sym{\textSymb{com_lowlight}} Halvljus (vrid \TorqueR).
\startitemize
\sym{1} Positionsljus
\sym{2} Halvljus
\stopitemize
\stopitemize


\subsubsubject{Multifunktionsspak}

\startitemize[width=1.7em]
\sym{\textSymb{com_lowlight}} {[}Nicht belegt{]}
\sym{\textSymb{com_light}} Ljustuta (tryck upp spaken kort
\sym{\textSymb{com_blink}} Körriktningsvisare (spak framåt/bakåt)
\sym{\textSymb{com_claxonArrow}} Signalhorn (tryck på knappen på utsidan av spaken)
\sym{\textSymb{com_wipper}} Vindrutetorkare
\startitemize
\sym{J} Intervalltorkning
\sym{O} Från
\sym{I} Första hastighetssteg
\sym{II} Andra hastighetssteg
\stopitemize
\sym{\textSymb{com_washerArrow}} Spolarsystem (tryck på ringen på spakens ände).
\stopitemize


\subsubsubject{Växelspak}

Växelspakens funktioner beskrivs närmare i kapitel \about[chap:using], fr.o.m.\atpage[sec:using:start].

\stopsection

\page [yes]


\startsection [title={Fler funktioner},
reference={sec:ctrl:add}]


\subsection[sec:ctrl:aux]{Takkonsol}

{\sl Takkonsolen\index{Takkonsol} befinner sig framtill i förarhyttens tak på förarsidan.}
\placefig [margin] [fig:console:aux] {Takkonsol}
{\externalfigure[ctrl:console:aux]}


\placefig [margin] [fig:console:climat] {Uppvärmning och klimatanläggning}
{\externalfigure[ctrl:console:climat]}


\startitemize [unpacked][width=1.7em]
\sym{\textBigSymb{temoin_retrochauffant}} Uppvärmning av ytterspeglar
\sym{\textBigSymb{temoin_hazard}} Varningsblinkers
\sym{\textBigSymb{temoin_eclairage_L}} Arbetsstrålkastare
\stopitemize


\subsubsubject{Extrautrustning}

\startLeg [unpacked][width=1.7em]
\sym{\textBigSymb{temoin_buse}} Högtrycksvattenpump för vattenspruta \crlf {\sl se \atpage[sec:using:water:spray]}
\sym{\textBigSymb{temoin_aspiration_manuelle}} Turbin för handsugslang \crlf {\sl se \atpage[sec:using:suction:hose]}
\stopLeg


\subsection[sec:ctrl:climat]{Uppvärmning och klimatanläggning}

{\sl Den här konsolen\index{Uppvärmningskonsol} befinner sig på förarhyttens bakvägg, mellan stolarna.}

\startitemize [unpacked][width=23mm]
\sym{\bf 0\quad I\quad II\quad III} Vridomkopplare för fläkt
\sym{\externalfigure[tirette_chauffage][height=1em]} Temperaturreglage
\stopitemize


\subsubsubject{Extrautrustning}

\startitemize [unpacked][width=1.7em]
\sym{\textBigSymb{temoin_climat_bk}} Klimatanläggning
\stopitemize

\page [yes]

\setups [pagestyle:bigmargin]


\subsection[sec:ctrl:central]{Mittkonsol}

{\sl Mittkonsolen\index{Mittkonsol} befinner sig mellan stolarna.}

\placefig [margin] [fig:console:central] {Mittkonsol}
{\externalfigure[ctrl:console:central]}


\subsubsubject{Fuktning av kvastarna}

\startLeg [unpacked][width=1.7em]
\sym{\textBigSymb{temoin_busebalais}} Lågtrycksvattenpump\index{Vattenpump} för kvastarnas befuktningssystem\index{Vattenpump+Befuktning}. (Position 1: \aW{automatiskt}; Position 2: \aW{permanent})
\stopLeg


\subsubsubject{Tippa smutsbehållaren}

\setupinmargin[right][style=normal]
\inright{%
\startitemize
\sym{\textSymb{mand_readtheoperatingmanual}} Beakta anvisningarna för användning av handbromsen på \atpage[sec:using:stop].
\stopitemize}

\startLeg [unpacked][width=1.7em]
\sym{\textBigSymb{temoin_kipp2}} Tippa smutsbehållaren.
För\index{Smutsbehållare+Tippa} att kunna tippa smutsbehållaren måste du dra åt handbromsen
och ställa växelspaken i neutralläge.
\stopLeg


\subsubsubject{Nödstopp}

\starttextbackground [FC]
\startPictPar
\externalfigure[Emergency_Stop][Pict]
\PictPar
I ett nödfall\index{Nödstopp} kan du slå från alla sug- och sopenheter samt motorn genom att trycka på nödstoppsknappen.
\stopPictPar
\stoptextbackground


\subsection[sec:foot:switch]{Fotkontakt}

\placefig [margin] [fig:foot:switch] {Fotkontakt}
{\vskip 60pt
\externalfigure[work:foot:switch]}

Med\index{Fotkontakt} fotkontakten på rattstången (\inF[fig:foot:switch]) kan du snabbt och enkelt sänka kvastarna om det skulle behövas (\eG\ längst upp på en uppförsbacke, vid uppkörning på en trottoar).

\stopsection
\page[yes]
\setups [pagestyle:marginless]


\startsection[title={Multifunktionskonsol},
reference={ctrl:console:middle}]

\startlocalfootnotes

\startfigtext[left]{Multifunktionskonsol}
{\externalfigure[overview:joy:large]}


\subsubsubject{Spakar}

\textDescrHead{Utan frontkvast (eller frontkvast avaktiverad):}
Spakarna styr var sin kvast, oberoende av varandra: Lyft/sänk~(\textSymb{joystick_aa}) eller vänster/höger~(\textSymb{joystick_gd}). Vänster spak styr vänster kvast, höger spak styr höger kvast.\footnote{För att kunna ändra sidokvastarnas position på ett fordon som är utrustat med frontkvast (tillval) måste man avaktivera frontkvasten (knapp~\textSymb{joy_key_frontbrush_act}).}

\textDescrHead{Med frontkvast:}
Med vänster spak kan du höja/sänka frontkvasten (\textSymb{joystick_aa}) eller röra den åt vänster/höger (\textSymb{joystick_gd}). Med höger spak kan du tippa kvasten runt sin egen längs-~(\textSymb{joystick_aa}) och tväraxel~(\textSymb{joystick_gd}).

\placelocalfootnotes %[height=\textheight]
\stopfigtext
\stoplocalfootnotes
\vfill


\subsubsubject{Sidoknappar}

\startcolumns

\startPictList
\VPcltr
\PictList
Farthållare: Öka den inställda hastigheten
\stopPictList\vskip -3pt

\startPictList
\VPclbr
\PictList
Farthållare: Sänk den inställda hastigheten
\stopPictList\vskip -3pt

\startPictList
\VPcrtr
\PictList
Lyft sugmunstycket
\stopPictList

\startPictList
\VPcrbr
\PictList
Sänk sugmunstycket
\stopPictList\vskip -3pt

\startPictList
\VPcrtf
\PictList
Öppna luckan för grov smuts (framtill på sugmunstycket)
\stopPictList\vskip -3pt

\startPictList
\VPcrbf
\PictList
Stäng luckan för grovsmuts
\stopPictList

\stopcolumns


\subsubsubject{Symbolknappar}

\startcolumns

\startSymVpad
\externalfigure[joy:stop]
\SymVpad
\textDescrHead{Stopp} Stoppa den aktiverade enheten:

1\:× tryck: Avaktivera\,den tredje kvasten\crlf
2\:× tryck: Avaktivera alla
\stopSymVpad

\startSymVpad
\externalfigure[joy:tempomat]
\SymVpad
\textDescrHead{Farthållare} Ställ in farthållaren på den aktuella hastigheten och aktivera. Tryck på knappen~\textSymb{joy:tempomat} igen eller bromsa för att avaktivera farthållaren. Accelerera/sakta ned med sidoknapparna.
\stopSymVpad

\startSymVpad
\externalfigure[joy:ftbrs:minus]
\SymVpad
\textDescrHead{Kvasthastighet} Minska rotationshastigheten på sido- eller frontkvasten.
\stopSymVpad

\startSymVpad
\externalfigure[joy:ftbrs:plus]
\SymVpad
\textDescrHead{Kvasthastighet} Öka rotationshastigheten på sido- eller frontkvasten.
\stopSymVpad

\startSymVpad
\externalfigure[joy:eng:minus]
\SymVpad
\textDescrHead{Motorvarvtal} Minska dieselmotorns varvtal.
\stopSymVpad

\startSymVpad
\externalfigure[joy:eng:plus]
\SymVpad
\textDescrHead{Motorvarvtal} Öka dieselmotorns varvtal.
\stopSymVpad
\columnbreak

\startSymVpad
\externalfigure[joy:suc]
\SymVpad
\textDescrHead{Uppsugning} Aktivera sugsystemet: sugmunstycket sänks, turbinen och pumpen för återvinningsvatten slås på.\note [recyclingwaterpump] \crlf
Tryck på stoppknappen~\textSymb{joy:stop} för att avaktivera systemet.
\stopSymVpad

\startSymVpad
\externalfigure[joy:sucbrs]
\SymVpad
\textDescrHead{Sopning/uppsugning}Aktivera sug-/sopningssystemet: Sugmunstycket sänks, sidokvastarna sänks och positioneras, turbin, kvastar och pumpen för återvinningsvatten slås på.\note [recyclingwaterpump] \crlf
Tryck på Stopptaste~\textSymb{joy:stop} för att avaktivera systemet.
\stopSymVpad

\footnotetext[recyclingwaterpump]{Även renvattenpumpen slås på när brytaren~\textBigSymb{temoin_busebalais} står på\aW{Automatiskt} (se \in [sec:ctrl:central] på \atpage [sec:ctrl:central]).}
\startSymVpad
\externalfigure[joy:ftbrs:act]
\SymVpad
\textDescrHead{Frontkvast aktiverad} Aktivera/avaktivera frontkvast.
%% NOTE @Andrew: Singular
\stopSymVpad

\startSymVpad
\externalfigure[joy:ftbrs:right]
\SymVpad
\textDescrHead{Frontkvast vänster} Rotationsriktning för arbete med frontkvasten på vänster sida
(Rotationsriktning: medurs).
\stopSymVpad

\startSymVpad
\externalfigure[joy:ftbrs:left]
\SymVpad
\textDescrHead{Frontkvast höger} Rotationsriktning för arbete med frontkvasten på höger sida
(Rotationsriktning: moturs).
\stopSymVpad

\stopcolumns

\stopsection

\stopchapter

\stopcomponent












