\startcomponent c_80_maintenance_s2_120-sv
\product prd_ba_s2_120-sv

\startchapter [title={Wartung und Instandhaltung},
							reference={chap:maintenance}]

\setups[pagestyle:marginless]


\startsection [title={Allgemeine Hinweise}]


\subsection{Umweltschutz}

\starttextbackground [FC]
\setupparagraphs [PictPar][1][width=2.45em,inner=\hfill]

\startPictPar
\Penvironment
\PictPar
	\Boschung\ setzt Umweltschutz\index{Umweltschutz} in die Praxis um. Wir setzen
	an den Ursachen an und beziehen alle Auswirkungen des Produktionsprozesses und
	des Produkts auf die Umwelt in die unternehmerischen Entscheidungen mit ein.
	Ziele sind der sparsame Einsatz von Ressourcen und ein schonender Umgang mit
	den natürlichen Lebensgrundlagen, deren Erhaltung Mensch und Natur dient.
	Durch das Einhalten bestimmter Regeln beim Betrieb des Fahrzeugs können Sie
	zum Umweltschutz beitragen. Dazu gehört auch der angemessene und
	vorschriftsmäßige Umgang mit Stoffen und Materialien im Rahmen der
	Fahrzeugwartung (\eG\ das Entsorgen von Chemikalien und Sondermüll).

	Kraftstoffverbrauch und Verschleiß eines Motors hängen von den
	Betriebsbedingungen ab. Deshalb bitte wir Sie, auf einige Punkte zu achten:

\startitemize
	\item Lassen Sie den Motor nicht im Leerlauf warmlaufen.
	\item Stellen Sie den Motor während betriebsbedingter Wartezeiten ab.
	\item Kontrollieren Sie regelmäßig den Kraftstoffverbrauch.
	\item {\em Lassen Sie die Wartungsarbeiten gemäß Wartungsplan und von einer
	kompetenten Fachwerkstatt ausführen.}
\stopitemize
\stopSymList
\stoptextbackground

\page [yes]


\subsection{Sicherheitsvorschriften}

\startSymList
\PHgeneric
\SymList
Um\index{Wartung+Sicherheitsvorschriften} Schäden an Fahrzeug und Aggregaten
sowie Unfälle bei der Wartung zu verhindern, ist es absolut erforderlich, die
folgenden Sicherheitsvorschriften einzuhalten. Beachten Sie ebenfalls die
allgemeinen Sicherheitsvorschriften (\about[safety:risques], \at{ab
Seite}[safety:risques]).
\stopSymList

\starttextbackground [FC]
\startPictPar
\PMgeneric
\PictPar
\textDescrHead{Unfallverhütung}
	Kontrollieren\index{Unfallverhütung} Sie den Zustand des Fahrzeugs nach
	jeder Wartungs- oder Reparaturtätigkeit. Stellen Sie insbesondere sicher,
	dass alle sicherheitsrelevanten Komponenten sowie Beleuchtungs- und
	Signaleinrichtungen einwandfrei funktionieren, bevor Sie auf öffentliche
	Verkehrswege fahren.
\stopPictPar
\stoptextbackground
\blank [big]

\start
\setupparagraphs [SymList][1][width=6em,inner=\hfill]
\startSymList\PHcrushing\PHfalling\SymList
\textDescrHead{Stabilisierung des Fahrzeugs}
	Vor jeglicher Wartungstätigkeit ist das Fahrzeug gegen ungewollte
	Fahrzeugbewegungen zu sichern: Stellen Sie den Fahrstufenwählhebel auf
	\aW{Neutral}, aktivieren Sie die Feststellbremse und sichern Sie das Fahrzeug
	mit Radkeilen.
\stopSymList
\stop

\starttextbackground[CB]
\startPictPar\PHpoison\PictPar
\textDescrHead{Starten des Motors}
	Wenn\index{Gefahr+Vergiftung} Sie den Motor an einem schlecht belüfteten Ort
	starten müssen, lassen Sie ihn nur solange wie nötig\index{Gefahr+Abgase}
	laufen, um Kohlenmonoxidvergiftungen zu verhindern.
	\stopPictPar
\startitemize
	\item Starten Sie den Motor nur bei ordnungsgemäß angeschlossener Batterie.
	\item Klemmen Sie die Batterie nie bei laufendem Motor ab.
	\item Starten Sie den Motor nicht unter Zuhilfenahme eines Starthilfegeräts.
	Wenn\index{Batterie+Ladegerät} die Batterie mit einem Schnellladegerät
	geladen werden soll, ist sie vorher vom Fahrzeug zu trennen. Beachten Sie
	die Betriebsvorschriften des Schnellladegeräts.
\stopitemize
\stoptextbackground

\page [bigpreference]


\subsubsection{Schutz der elektronischen Komponenten}

\startitemize
	\item Bevor\index{Elektroschweißen} Sie mit Schweißarbeiten beginnen,
	trennen Sie die Batteriekabel von der Batterie und schließen Sie Plus-
	und Massekabel zusammen.
	\item Schließen\index{Elektronik} Sie elektronische Steuergeräte nur an und
	trennen Sie sie nur, wenn sie nicht unter Spannung stehen.
	\item Eine falsche\index{Steuergerät} Polarität in der Stromversorgung (\eG\
	durch falsch angeschlossene Batterien) kann elektronische Bauteile und
	Geräte zerstören.
	\item Bei\index{Umgebungstemperatur+extreme} Umgebungstemperaturen von über
	80\,°C (\eG\ in einer Trockenkammer) sind elektronische Bauteile|/|Geräte zu
	entfernen.
\stopitemize


\subsubsection{Diagnose und Messungen}

\startitemize
	\item Verwenden Sie für Mess- und Diagnosearbeiten nur {\em geeignete}
	Prüfkabel (\eG\ die Originalkabel des Geräts).
	\item Mobiltelefone\index{Mobiltelefon}, und vergleichbare Funk||Geräte,
	können die Funktionen des Fahrzeugs, des Diagnosegeräts und damit die
	Betriebssicherheit beeinträchtigen.
\stopitemize


\subsubsection{Qualifizierung des Personals}

\starttextbackground[CB]
\startPictPar
\PHgeneric
\PictPar
\textDescrHead{Unfallgefahr}
Bei\index{Qualifikation+Wartungspersonal} unsachgemäßer Ausführung von
Wartungsarbeiten  können Funktionsfähigkeit und Sicherheit des Fahrzeugs
beeinträchtigt werden. Dies zieht ein erhöhtes Unfall- und Verletzungsrisiko
nach sich.

Wenden Sie\index{Qualifikation+Werkstatt} sich für Wartungs- und
Reparaturarbeiten an eine qualifizierte Fachwerkstatt, welche über
die erforderlichen Kenntnisse und Werkzeuge verfügt.

Wenden Sie sich im Zweifelsfall an den \Boschung||Kundendienst.
\stopPictPar
\stoptextbackground

Das \ProductId darf ausschließlich von qualifiziertem und von durch den
\Boschung||Kundendienst ausgebildetem Personal bedient, gewartet oder
repariert werden.

Die Kompetenzen für Bedienung, Instandhaltung und Reparatur werden vom
\Boschung||Kundendienst vergeben.


\subsubsection{Änderungen und Umbauten}

\starttextbackground[CB]
\startPictPar
\PHgeneric
\PictPar
\textDescrHead{Unfallgefahr}
Sämtliche\index{Änderung am Fahrzeug} Änderungen, die Sie am Fahrzeug
eigenmächtig vornehmen, können Funktionsfähigkeit und Betriebssicherheit des
\ProductId beeinträchtigen und damit ein nicht abwägbares Unfall- und
Verletzungsrisiko nach sich ziehen.
\stopPictPar

\startPictPar
\PMwarranty
\PictPar
Für Schäden, die durch\index{Garantie+Bedingungen} eigenmächtige Eingriffe
oder Modifikationen am \ProductId oder einem Aggregat entstehen, übernimmt
\Boschung\ keinerlei Garantie oder Kulanz.
\stopPictPar
\stoptextbackground

\stopsection


\startsection [title={Betriebsstoffe und Schmiermittel}, reference={sec:liquids}]


\subsection{Richtiger Umgang}

\starttextbackground[CB]
\startPictPar
\PHpoison
\PictPar
\textDescrHead{Verletzungs- und Vergiftungsgefahr}
Durch\index{Kraftstoff} Hautkontakt\index{Schmiermittel}
oder\index{Gefahr+Vergiftung} Verschlucken von Betriebsstoffen und
Schmiermitteln kann\index{Kraftstoff+Sicherheit} es zu erheblichen Verletzungen
oder Vergiftungen kommen. Beachten Sie bei der Handhabung, Lagerung und Entsorgung dieser Stoffe stets
die gesetzlichen Vorschriften.
\stopPictPar
\stoptextbackground

\starttextbackground [FC]
\startPictPar
\PMproteyes\par
\PMprothands
\PictPar
Tragen Sie beim Umgang mit Betriebs- und Schmierstoffen stets angemessene
Schutzkleidung und Atemschutz. Vermeiden Sie das Einatmen der Dämpfe.
Vermeiden Sie jeglichen Kontakt mit Haut, Augen oder Kleidung. Reinigen Sie
Hautstellen, die in Kontakt mit Betriebsstoffen gekommen sind, sofort mit
Wasser und Seife.  Sollten Betriebsstoffe in die Augen gelangt sein, spülen
Sie diese mit reichlich klarem Wasser und suchen Sie ggf. einen Augenarzt
auf. Nach dem Verschlucken von Betriebsstoffen ist unverzüglich ein Arzt
aufzusuchen!
\stopPictPar
\stoptextbackground

\startSymList
\PPchildren
\SymList
Betriebsstoffe sind für Kinder unzugänglich aufzubewahren.
\stopSymList

\startSymList
\PPfire
\SymList
\textDescrHead{Brandgefahr}
	Aufgrund\index{Gefahr+Feuer} der hohen Entflammbarkeit von Betriebsstoffen
	erhöht sich beim Umgang mit diesen das Brandrisiko.  Rauchen,
	Feuer\index{Rauchverbot} und offenes Licht sind beim Umgang mit
	Betriebsstoffen strengstens verboten.
\stopSymList

\starttextbackground [FC]
\startPictPar
\PMgeneric
\PictPar
Es dürfen nur Schmierstoffe eingesetzt werden, die für die im \ProductId
eingesetzten Komponenten geeignet sind. Verwenden Sie deshalb nur von
\Boschung\ geprüfte und freigegebene Produkte. Diese finden Sie in der
Betriebsstoffliste \atpage[sec:liqquantities]. Additive\index{Additive} für
Schmiermittel sind nicht erforderlich. Sollten Sie Additive zusetzen, kann
dies zum Erlöschen der Garantieansprüche\index{Garantie+Bedingungen} führen.
Für weitere Informationen wenden Sie sich an den \Boschung||Kundendienst.
\stopPictPar
\stoptextbackground

\starttextbackground [FC]
\startPictPar
\Penvironment
\PictPar
\textDescrHead{Umweltschutz}
Achten\index{Schmiermittel+Entsorgung} Sie bei der Entsorgung von Betriebs-
und\crlf Schmierstoffen\index{Umweltschutz} oder Gegenständen, die mit diesen
kontaminiert sind (\eG\ Filter, Lappen),
auf\index{Betriebsstoffe+Entsorgung} die Einhaltung der Umweltschutzbestimmungen.
\stopPictPar
\stoptextbackground

\page [yes]

\setups [pagestyle:normal]


\subsection[sec:liqquantities]{Spezifikationen und Füllmengen}

Sämtliche\index{Betriebsstoffe+Füllmenge}\index{Schmiermittel+Füllmenge}\index{Füllmengen+Betriebsstoffe
und Schmiermittel}\index{Spezifikationen+Betriebsstoffe und
Schmiermittel} in der folgenden Tabelle angegebenen Füllmengen sind
Richtwerte. Nach jedem Betriebsstoff-|/|Schmiermittelwechsel muss der
tatsächliche Füllstand kontrolliert werden und ggf. muss die Füllmenge erhöht
oder verringert werden.
% \blank[big]

\placetable[margin][tab:glyco]{Frostschutz (\index{Frostschutz}Motor)}
{\noteF\startframedcontent[FrTabulate]
%\starttabulate[|Bp(80pt)|r|r|]
\starttabulate[|Bp|r|r|]
\NC Gefrierschutz bis {[}°C{]}\NC \bf \textminus 25 \NC \bf \textminus 40 \NC\NR
\NC Destill. Wasser [Vol.||\%] \NC 60 \NC 40 \NC\NR
\NC Frostschutzmittel \break [Vol.||\%] \NC 40 \NC {\em max.} 60 \NC\NR
\stoptabulate\stopframedcontent\endgraf
Achtung: Bei einem Volumenanteil von mehr als 60\hairspace\percent\
Frostschutzmittel {\em sinkt} der Gefrierschutz und die Kühlleistung verschlechtert sich!}

\placefig[margin][fig:hydrgauge]{\select{caption}{Niveauanzeige
Hydraulikflüssigkeit (linke Fahrzeugseite)}{Niveauanzeige
Hydraulikflüssigkeit}}
{\externalfigure[main:hy:level_temp]
\noteF Der Füllstand des Hydrauliktanks kann am Schauglas abgelesen werden und
ist {\em täglich} zu überprüfen.}

\vskip -8pt
\start
\define [1] \TableSmallSymb {\externalfigure[#1][height=4ex]}
\define\UC\emptY
\pagereference[page:table:liquids]

\setupTABLE	[frame=off,style={\ssx\setupinterlinespace[line=.86\lH]},background=color,
			option=stretch,
			split=repeat]
\setupTABLE	[r]	[each]	[topframe=on,
						framecolor=TableWhite,
						% rulethickness=.8pt
						]

\setupTABLE	[c]	[odd]	[backgroundcolor=TableMiddle]
\setupTABLE	[c]	[even]	[backgroundcolor=TableLight]
\setupTABLE	[c]	[1]		[width=30mm]
\setupTABLE	[c]	[2]		[width=20mm]
\setupTABLE	[c]	[4]		[width=25mm]
\setupTABLE	[c]	[last]	[width=10mm]
\setupTABLE	[r] [first]	[topframe=off,style={\bfx\setupinterlinespace[line=.95\lH]},
						% backgroundcolor=TableDark
						]
\setupTABLE	[r]	[2]		[framecolor=black]

\bTABLE

\bTABLEhead
	\bTR
	\bTC Gruppe \eTC
	\bTC Kategorie \eTC
	\bTC Klassifizierung \eTC
	\bTC Produkt\note[Produkt] \eTC
	\bTC Menge \eTC
	\eTR
\eTABLEhead

\bTABLEbody
 \bTR \bTD	Dieselmotor \eTD
  \bTD Motoröl\eTD
  \bTD \liqC{SAE 5W-30}; \liqC{VW\,507.00}\eTD
  \bTD Total Quartz INEO Long Life \eTD
  \bTD	4,3\,l\eTD
  \eTR
 \bTR \bTD	Hydraulikkreis \eTD
  \bTD ATF-Öl \eTD
  \bTD \liqC{dexron~iii} \eTD
  \bTD  Total Equiviz ZS 46 (Tank ca. 40\,l) \eTD
  \bTD ca. 50\,l\eTD
  \eTR
 \bTR \bTD Hydraulikkreis (Option~\aW{Bio})\eTD
  \bTD ATF-Öl \eTD
  \bTD \liqC{dexron~iii} \eTD
  \bTD  Total Biohydran TMP SE 46 (Tank ca. 45\,l) \eTD
  \bTD ca. 60\,l\eTD
  \eTR
 \bTR \bTD	Magnetventile: Spulenkerne \eTD
  \bTD Schmiermittel\eTD
  \bTD Kupferfett \eTD
  \bTD \emptY\eTD
  \bTD n.\,B.\note[Bedarf] \eTD
  \eTR
 \bTR \bTD	Verschiedenes: Schlösser, Türmechanik, Bremspedal \eTD
  \bTD Schmiermittel\eTD
  \bTD Universal||Spray\eTD
  \bTD \emptY\eTD
  \bTD n.\,B.\note[Bedarf] \eTD
  \eTR
 \bTR \bTD	Zentralschmieranlage \eTD
  \bTD Universal||Lagerfett\eTD
  \bTD \liqC{nlgi}~2 \eTD
  \bTD Total Multis EP~2\eTD
  \bTD n.\,B.\note[Bedarf] \eTD
  \eTR
 \bTR \bTD	Kühlsystem \eTD
  \bTD Frost-|/|Rostschutzmittel\eTD
  \bTD TL VW 774 F/G; max. 60\hairspace\% vol.\eTD
  \bTD G12+|/|G12++ (rosa|/|violett)\eTD
  \bTD ca. 14\,l \eTD
  \eTR
 \bTR \bTD	Hochdruckwasserpumpe \eTD
  \bTD Motoröl\eTD
  \bTD \liqC{SAE 10W-40}; \liqC{api cf~– acea e6}\eTD
  \bTD Total Rubia TIR 8900 \eTD
  \bTD 0,29\,l \eTD
  \eTR
 \bTR \bTD	Klimaanlage \eTD
  \bTD Kältemittel\eTD
  \bTD + 20\,ml POE||Öl\eTD
  \bTD R\,134a\eTD
  \bTD	700\,g\eTD
  \eTR
 \bTR \bTD	Scheibenwaschanlage \eTD
	\bTD [nc=2] Wasser und Scheibenwaschkonzentrat, \aW{S}~Sommer, \aW{W}~Winter; Mischverhältnis beachten \eTD
	\bTD Einzelhandel \eTD
  \bTD n.\,B.\note[Bedarf] \eTD
  \eTR
\eTABLEbody

\eTABLE
\stop

\footnotetext[Bedarf]{{\it n.\,B.} nach Bedarf, entsprechend der jeweiligen
Anleitung}
\footnotetext[Produkt]{Von \Boschung\ verwendete Produkte. Andere Produkte, die den Spezifikationen entsprechen, können ebenfalls verwendet werden.}

\stopsection

\page [yes]

\setups [pagestyle:marginless]


\startsection [title={Wartung des Dieselmotors},
							reference={sec:workshop:vw}]


\subsection [sSec:vw:diagTool]{On||Board||Diagnosesystem}

Das\startregister[index][reg:main:vw]{Wartung+Dieselmotor} Motorsteuergerät (J623) ist mit einem Fehlerspeicher ausgestattet.
Treten Störungen in den überwachten Sensoren bzw. Bauteilen auf, werden diese mit Angabe der Fehlerart im Fehlerspeicher gespeichert.

Das\index{Dieselmotor+Diagnose} Motorsteuergerät unterscheidet nach Auswertung der Information zwischen den unterschiedlichen Fehlerklassen und speichert diese bis zum Löschen des Fehlerspeicherinhaltes.

Fehler, die nur {\em sporadisch} auftreten, werden mit dem Zusatz \aW{SP} angezeigt. Die Ursache für sporadische Fehler kann \eG\ ein Wackelkontakt oder eine kurzzeitige Leitungsunterbrechung sein. Tritt ein sporadischer Fehler innerhalb von 50 Motorstarts nicht mehr auf, wird er aus dem Fehlerspeicher gelöscht.

Sind Fehler erkannt worden, die das Laufverhalten des Motors beeinflussen, leuchtet auf dem Bildschirm des Vpad das Kontrollsymbol \aW{Motordiagnose}~\textSymb{vpadWarningEngine1} auf.

Die gespeicherten Fehler können mit dem Fahrzeug||Diagnose, -Mess und -Informationssystem \aW{VAS\,5051/B} ausgelesen werden.

Nachdem der oder die Fehler behoben sind, muss der Fehlerspeicher gelöscht werden.


\subsubsection[sSec:vw:diagTool:connect]{Inbetriebnahme des Diagnosesystems}

\starttextbackground [FC]
\startPictPar
\PMgeneric
\PictPar
Detaillierte Informationen zum Fahrzeugdiagnosesystem VAS\,5051/B finden Sie in der Bedienungsanleitung des Systems.

Sie können auch andere kompatible Diagnosesysteme einsetzen, \eG\ \aW{DiagRA}.
\stopPictPar
\stoptextbackground

\page [yes]


\subsubsubsubject{Voraussetzungen}

\startitemize
\item Die Sicherungen müssen in Ordnung sein.
\item Die Batteriespannung muss mehr als 11,5\,V betragen.
\item Alle elektrischen Verbraucher müssen ausgeschaltet sein.
\item Der Masseanschluss muss in Ordnung sein.
\stopitemize


\subsubsubsubject{Vorgehensweise}

\startSteps
\item Stecken Sie den Stecker der Diagnoseleitung VAS\,5051B/1 auf den Diagnoseanschluss.
\item Je nach Funktion, entweder Zündung einschalten oder Motor starten.
\stopSteps

\subsubsubsubject{Betriebsart auswählen}

\startSteps [continue]
\item Drücken Sie auf dem Display auf die Schaltfläche \aW{Fahrzeug||Eigendiagnose}.
\stopSteps


\subsubsubsubject{Fahrzeugsystem auswählen}

\startSteps [continue]
\item Drücken Sie auf dem Display auf die Schaltfläche \aW{01-Motorelektronik}.
\stopSteps

Auf dem Display wird nun die Steuergeräte||Identifikation und die Kodierung des Motorsteuergerätes angezeigt.

Stimmen die Kodierungen nicht überein, muss die Steuergeräte||Kodierung überprüft werden.


\subsubsubsubject{Diagnosefunktion auswählen}

Auf dem Display werden Ihnen alle ausführbaren Diagnosefunktionen angezeigt.

\startSteps [continue]
\item Drücken Sie auf dem Display die Schaltfläche für die gewünschte Funktion.
\stopSteps



\subsection [sSec:vw:faultMemory]{Fehlerspeicher}


\subsubsection{Fehlerspeicher auslesen}

\subsubsubject{Arbeitsablauf}

\startSteps
\item Lassen Sie den Motor im Leerlauf laufen.
\item Schließen Sie das VAS\,5051/B an (siehe \in{Abschnitt}[sSec:vw:diagTool:connect])
und wählen Sie das Motorsteuergerät an.
\item Wählen Sie die Diagnosefunktion \aW{004-Fehlerspeicherinhalt}.
\item Wählen Sie die Diagnosefunktion \aW{004.01-Fehlerspeicher abfragen}.
\stopSteps

{\sla Nur wenn Motor nicht anspringt:}

\startitemize [2]
\item Schalten Sie die Zündung ein.
\item Ist kein Fehler im Motorsteuergerät abgelegt, erscheint auf dem Display \aW{0~Fehler erkannt}.
\item Sind Fehler im Motorsteuergerät abgelegt, werden sie im Display untereinander angezeigt.
\item Beenden Sie die Diagnosefunktion.
\item Schalten Sie die Zündung aus.
\item Beheben Sie ggf. angezeigte Fehler anhand der Fehlertabelle (siehe Servicedokumentation) und löschen Sie anschließend den Fehlerspeicher.
\stopitemize

\starttextbackground [FC]
\startPictPar
\PMrtfm
\PictPar
Wenn sich ein Fehler nicht löschen lässt, wenden Sie sich bitte an den \boschung||Kundendienst.
\stopPictPar
\stoptextbackground


\subsubsubject{Statische Fehler}

Sind im Datenspeicher ein oder mehrere statische Fehler vorhanden, wenden Sie sich bitte an den Boschung||Kundendienst, um diese Fehler mithilfe der \aW{Geführten Fehlersuche} zu beheben.


\subsubsubject{Sporadische Fehler}

Falls im Fehlerspeicher ausschließlich sporadische Fehler oder Hinweise abgespeichert sind, und keine Fehlfunktionen des elektronischen Fahrzeugsystem festgestellt werden, kann der Fehlerspeicher gelöscht werden:

\startSteps [continue]
\item Drücken Sie nochmals die Taste \aW{Weiter}~\inframed[strut=local]{>}, um in den Prüfplan zu gelangen.
\item Um die geführte Fehlersuche zu beenden, drücken Sie die Taste \aW{Sprung} und dann \aW{Beenden}.
\stopSteps

Es werden nun nochmals alle Fehlerspeicher abgefragt.

In einem Fenster wird bestätigt, dass alle sporadischen Fehler gelöscht wurden.
% Das Diagnoseprotokoll wird automatisch (online) verschickt.

Der Fahrzeugsystemtest ist damit beendet.


\subsubsection[sSec:vw:faultMemory:errase]{Löschen des Fehlerspeichers}

\subsubsubject{Arbeitsablauf}

{\sla Voraussetzungen:}

\startitemize [2]
\item Alle Fehler müssen behoben und die Fehlerursachen beseitigt sein.
\stopitemize

\page [yes]


{\sla Vorgehensweise:}

\starttextbackground [FC]
\startPictPar
\PMrtfm
\PictPar
Nach der Fehlerbehebung muss der Fehlerspeicher erneut abgefragt und anschließend gelöscht werden:
\stopPictPar
\stoptextbackground

\startSteps
\item Lassen Sie den Motor im Leerlauf laufen.
\item Schließen Sie das VAS\,5051/B an (siehe \in{Abschnitt}[sSec:vw:diagTool:connect])
und wählen Sie das Motorsteuergerät an.
\item Wählen Sie die Diagnosefunktion \aW{004-Fehlerspeicher abfragen}.
\item Wählen Sie die Diagnosefunktion \aW{004.10-Fehlerspeicher löschen}.
\stopSteps

\starttextbackground [FC]
\startPictPar
\PMrtfm
\PictPar
Wenn sich der Fehlerspeicher nicht löschen lässt, ist noch ein Fehler vorhanden und muss beseitigt werden.
\stopPictPar
\stoptextbackground

\startSteps [continue]
\item Beenden Sie die Diagnosefunktion.
\item Schalten Sie die Zündung aus.
\stopSteps


\subsection [sSec:vw:lub] {Schmierung des Dieselmotors}

\subsubsection [ssSec:vw:oilLevel] {Motorölstand überprüfen}

\starttextbackground [FC]
\startPictPar
\PMrtfm
\PictPar
Der\index{Motoröl+-stand} Ölstand darf die \aW{Max.}||Markierung in keinem Fall überschreiten. Ansonsten besteht die\index{Füllstand+Motoröl} Gefahr von Katalysatorschäden.
\stopPictPar
\stoptextbackground

\startSteps
\item Motor abstellen und mindestens 3~Minuten warten, damit das Öl in die Ölwanne zurückfließen kann.
\item Messstab herausziehen und sauber abwischen; den Stab wieder bis zum Anschlag hineinschieben.
\item Stab wieder herausziehen und den Ölstand beurteilen:

\startfigtext[right][fig:vw:gauge]{Ablesen des Ölstands}
{\externalfigure[VW_Oil_Gauge][width=50mm]}
\startitemize [A]
\item Maximaler Füllstand; es darf kein Öl nachgefüllt werden.
\item Ausreichender Füllstand; es {\em kann} Öl bis zum Erreichen der Markierung~\aW{A} nachgefüllt werden.
\item Nicht ausreichender Füllstand; es {\em muss} Öl nachgefüllt werden, bis sich der Füllstand im Bereich \aW{B} befindet.
\stopitemize
{\em Bei einem Füllstand über der Markierung~\aW{A} besteht die Gefahr von Katalysatorschäden.}
\stopfigtext
\stopSteps


\subsubsection [ssSec:vw:oilDraining] {Motorölwechsel}

\starttextbackground [FC]
\startPictPar
\PMrtfm
\PictPar
Der Motorölfilter der S2 ist stehend montiert. Das bedeutet, dass der Filter {\em vor} dem Ölwechsel gewechselt werden muss. Durch das Herausnehmen des Filterelements wird ein Ventil geöffnet, und das Öl im Filtergehäuse fließt automatisch ins Kurbelgehäuse.
\stopPictPar
\stoptextbackground

\startSteps
\item Stellen Sie einen geeigneten\index{Dieselmotor+Ölwechsel} Auffangbehälter unter den Motor.
\item Ölablassschraube herausschrauben\index{Motoröl+-wechsel} und das Öl ablaufen lassen.
\stopSteps

\starttextbackground [FC]
\startPictPar
\PMrtfm
\PictPar
Achten Sie darauf, dass der Auffangbehälter die gesamte Altölmenge aufnehmen kann.
Die erforderliche Ölspezifikation und Füllmenge finden Sie in \in{Abschnitt}[sec:liqquantities].

Die Ölablassschraube ist mit einem unverlierbaren Dichtring versehen. Die Ölablassschraube muss deshalb immer ersetzt werden
\stopPictPar
\stoptextbackground

\startSteps [continue]
\item Schrauben Sie eine neue Ölablassschraube mit Dichtring ein (\TorqueR~30\,Nm).
\item Motoröl geeigneter Spezifikation einfüllen (siehe \in{Abschnitt}[sec:liqquantities]).
\stopSteps


\subsubsection [ssSec:vw:oilFilter] {Motorölfilter ersetzen}

\starttextbackground [FC]
\startPictPar
\PMrtfm
\PictPar
\startitemize [1]
\item Beachten\index{Dieselmotor+Ölfilter} Sie die Vorschriften zu Entsorgung und Recycling.
\item Wechseln\index{Ölfilter+Dieselmotor} Sie den Filter {\em vor} dem Ölwechsel (siehe \in{Abschnitt}[ssSec:vw:oilDraining]).
\item Ölen Sie vor der Montage die Dichtung des neuen Filters leicht ein.
\stopitemize
\stopPictPar
\stoptextbackground

\startfigtext[right][fig:vw:oilFilter]{Ölfilter}
{\externalfigure[VW_OilFilter_03][width=50mm]}
\startSteps
\item Deckel~\Lone\ des Filtergehäuses mit einem geeigneten Schraubenschlüssel abschrauben.
\item Reinigen Sie die Dichtflächen von Deckel und Filtergehäuses.
\item Ersetzen Sie das Filterelement \Lthree.
\item Ersetzen Sie die O-Ringe \Ltwo\ und \Lfour.
\item Deckel wieder auf das Filtergehäuse schrauben (\TorqueR~25\,Nm).
\stopSteps



%\subsubsubject{Données techniques}
%
%
%\hangDescr{Couple de serrage du couvercle:} \TorqueR~25\,Nm.
%
%\hangDescr{Huile moteur prescrite:} Selon tableau \atpage[sec:liqquantities].
%% NOTE: Redundant [tf]

\stopfigtext



\subsubsection [ssSec:vw:oilreplenish] {Motoröl nachfüllen}

\starttextbackground [FC]
\startPictPar
\PMrtfm
\PictPar
\startitemize [1]
\item Säubern\index{Motoröl} Sie {\em vor} dem Abnehmen des Deckels den Einfüllstutzen mit einem Lappen.
\item Füllen\index{Dieselmotor+Öl nachfüllen} Sie ausschließlich Öl nach, das der vorgeschriebenen Spezifikation entspricht.
\item Füllen Sie schrittweise in kleinen Mengen nach.
\item Um ein Überfüllen zu vermeiden, warten Sie nach jedem Nachfüllen etwas, damit das Öl in die Motorölwanne bis zur Markierung des Messstabs fließen kann (siehe \in{Abschnitt}[ssSec:vw:oilLevel]).
\stopitemize
\stopPictPar
\stoptextbackground

\startfigtext[right][fig:vw:oilFilter]{Öl nachfüllen}
{\externalfigure[s2_bouchonRemplissage][width=50mm]}
\startSteps
\item Ziehen Sie den Ölmessstab etwa 10~cm heraus, sodass beim Nachfüllen die Luft entweichen kann.
\item Öffnen Sie die Einfüllöffnung.
\item Füllen Sie Öl unter Beachtung der obigen Vorschriften nach.
\item Verschließen Sie sorgfältig die Einfüllöffnung.
\item Starten Sie den Motor.
\item Führen Sie eine Füllstandskontrolle durch. (Siehe \in{Abschnitt}[ssSec:vw:oilLevel].)
\stopSteps

\stopfigtext


\subsection [sSec:vw:fuel] {Kraftstoffversorgungssystem}

\subsubsection [ssSec:vw:fuelFilter] {Kraftstofffilter ersetzen}

\starttextbackground [FC]
\startPictPar
\PMrtfm
\PictPar
\startitemize [1]
\item Beachten\index{Dieselmotor+Kraftstofffilter} Sie die gesetzlichen Vorschriften zu Entsorgung und Recycling von Sondermüll.
\item Nehmen Sie nicht die Kraftstoffleitungen vom Oberteil des Filters ab.
\item Üben Sie keinen Zug auf die Befestigungspunkte der Kraftstoffleitungen aus; anderfalls kann es zu Beschädigungen des Oberteils des Filters kommen.
\stopitemize
\stopPictPar
\stoptextbackground

\startfigtext[right][fig:vw:oilFilter]{Kraftstofffilter}
{\externalfigure[s2_fuelFilter_location][width=50mm]}

{\sla Vorbereitung:}

Das\index{Kraftstofffilter} Kraftstofffiltergehäuse ist vor dem Motor, an der rechten Chassisseite befestigt.
Entfernen Sie die beiden Befestigungsschrauben mithilfe eines 10-mm-Steckschlüssels und eines 10-mm-Ringschlüssels.

\stopfigtext


\page [yes]

\setups [pagestyle:normal]

{\sla Vorgehensweise:}

\startLongsteps
\item Entfernen Sie alle Schrauben des Filter||Oberteils. Nehmen Sie das Filter||Oberteil ab.
\stopLongsteps

\starttextbackground [FC]
\startPictPar
\PMrtfm
\PictPar
Heben Sie das Oberteil ab. Falls erforderlich, setzen Sie hierzu einen Winkelschraubendreher an der Montagenut (\in{\LAa, Abb.}[fig:fuelfilter:detach]) an und hebeln Sie das Oberteil aus.
\stopPictPar
\stoptextbackground

\placefig [margin] [fig:fuelfilter:detach]{Entnehmen des Kraftstofffilters}
{\externalfigure[fuelfilter:detach]}

\placefig [margin] [fig:fuelfilter:explosion]{Kraftstofffilter}
{\externalfigure[fuelfilter:explosion]}

\startLongsteps [continue]
\item Ziehen Sie das Filterelement aus dem Unterteil des Filters.
\item Nehmen Sie die Dichtung (\in{\Ltwo, Abb.}[fig:fuelfilter:explosion]) vom Oberteil des Filters ab.
\item Reinigen Sie Unter- und Oberteil des Filters sorgfältig.
\item Setzen Sie ein neues Filterelement in das Unterteil des Filters ein.
\item Benetzen Sie eine neue Dichtung (\in{\Ltwo, Abb.}[fig:fuelfilter:explosion]) mit etwas Kraftstoff und setzen Sie sie in das Oberteil ein.
\item Setzen Sie das Oberteil passend auf das Unterteil des Filters und drücken Sie es gleichmäßig fest, sodass das Oberteil rundum gleichmäßig aufliegt.
\item Schrauben Sie Ober- und Unterteil mit allen Schrauben {\em handfest} wieder zusammen. Ziehen Sie dann über Kreuz alle Schrauben mit dem vorgeschriebenen Anzugsdrehmoment (\TorqueR~5\,Nm) an.
\stopLongsteps

% \subsubsubject{Données techniques}
%
% \hangDescr{Couple de serrage des vis de fixation du couvercle:} \TorqueR 5\,Nm.
%% NOTE: redundant [tf]

\startLongsteps [continue]
\item Schalten Sie die Zündung ein, um das System zu entlüften; starten Sie den Motor und lassen Sie ihn 1~bis 2~Minuten mit Leerlaufdrehzahl laufen.
\item Löschen Sie den Fehlerspeicher wie auf \atpage[sSec:vw:faultMemory:errase] beschrieben.
\stopLongsteps


\subsection [sSec:vw:cooling] {Kühlsystem}

\starttextbackground [FC]
\startPictPar
\PMrtfm
\PictPar
\startitemize [1]
\item Nur\index{Dieselmotor+Kühlung} Kühlmittel der vorgeschriebenen Spezifikation (siehe Tabelle \atpage[sec:liqquantities]) verwenden.
\item Um\index{Kühlmittel} Frost- und Korrosionsschutz sicherzustellen, darf das Kühlmittel ausschließlich mit destiliertem Wasser und gemäß untenstehender Tabelle verdünnt werden.
\item Füllen Sie den Kühlmittelkreis nie mit Wasser auf, da hierdurch Frost- und Korrosionsschutz beeinträchtigt würden.
\stopitemize
\stopPictPar
\stoptextbackground


\subsubsection [sSec:vw:coolingLevel] {Kühlmittelstand}

\placefig [margin] [fig:coolant:level] {Kühlmittelstand}
{\externalfigure[coolant:level]}


\placefig [margin] [fig:refractometer] {Refraktometer VW~T\,10007}
{\externalfigure[coolant:refractometer]}

\placefig [margin] [fig:antifreeze] {Kontrolle der Frostschutzdichte}
{\externalfigure[coolant:antifreeze]}


\startSteps
\item Heben Sie den Schmutzbehälter und bringen sie die Sicherheitsstütze an.
\item Stellen\index{Füllstand+Kühlmittel} Sie den Füllstand des Kühlmittels im Ausdehnungsgefäß fest: Er muss sich oberhalb der \aW{min}||Markierung befinden.
\stopSteps

\start
\define [1] \TableSmallSymb {\externalfigure[#1][height=4ex]}
\define\UC\emptY
\pagereference[page:table:liquids]


\setupTABLE	[frame=off,style={\ssx\setupinterlinespace[line=.86\lH]},background=color,
			option=stretch,
			split=repeat]
\setupTABLE	[r]	[each]	[topframe=on,
						framecolor=TableWhite,
						% rulethickness=.8pt
						]

\setupTABLE	[c]	[odd]	[backgroundcolor=TableMiddle]
\setupTABLE	[c]	[even]	[backgroundcolor=TableLight]
\setupTABLE	[r] [first]	[topframe=off,style={\bfx\setupinterlinespace[line=.95\lH]},
						% backgroundcolor=TableDark
						]
\setupTABLE	[r]	[2]		[framecolor=black]

\bTABLE

\bTABLEhead
 \bTR
 \bTC Frostschutz bis … \eTC
 \bTC Anteil G12\hairspace ++\eTC
 \bTC Vol. Frostschutzmittel \eTC
 \bTC Vol. destilliertes Wasser \eTC
 \eTR
\eTABLEhead

\bTABLEbody
 \bTR \bTD \textminus 25\,°C \eTD
  \bTD 40\hairspace\% \eTD
  \bTD 3,8\,l \eTD
  \bTD 4,2\,l \eTD
  \eTR
 \bTR \bTD \textminus 35\,°C \eTD
  \bTD 50\hairspace\% \eTD
  \bTD 4,0\,l \eTD
  \bTD 4,0\,l \eTD
  \eTR
 \bTR \bTD \textminus 40\,°C \eTD
  \bTD 60\hairspace\%  \eTD
  \bTD 4,2\,l \eTD
  \bTD 3,8\,l \eTD
  \eTR
\eTABLEbody

\eTABLE
\stop

\adaptlayout [height=+20pt]
\subsubsection [sSec:vw:coolingFreeze] {Kühlmittelstand}

Überprüfen\index{Frostschutzdichte} Sie die Frostschutzdichte mithilfe eines geeigneten Refraktometers  (siehe \in{Abb.}[fig:refractometer]: VW T\,10007).
Achten Sie auf Skala~1: G12\hairspace ++ (siehe \in{Abb.}[fig:antifreeze]).

\page [yes]


\subsection [sSec:vw:airFilter] {Luftversorgung}

Der Luftfilter ist über die hintere Wartungstür auf der rechten Fahrzeugseite zugänglich (siehe \in{Abb.}[fig:airFilter]).

\placefig [margin] [fig:airFilter] {Luftfilter des Motors}
{\externalfigure[vw:air:filter]
\noteF
\startLeg
\item Sicherheitslasche
\item Unterteil des Gehäuses
\item Entlüftungsöffnung
\item Drucksensor
\stopLeg}


\subsubsubject{Einsatzbedingungen}

Ein Kehrfahrzeug wird häufig in Umgebungen mit starkem Staubanfall eingesetzt. Aus diesem Grund ist es notwendig, den Luftfilter wöchentlich zu überprüfen und zu reinigen. Siehe auch \about[table:scheduleweekly], \atpage[table:scheduleweekly]. Wenn erforderlich, muss der Luftfilter ersetzt werden.


\subsubsubject{Autodiagnostic}

Die Ansaugleitung verfügt über einen Drucksensor (\Lfour, \in{Abb.}[fig:airFilter]), über den sich Ladeverluste\footnote{Verringerter Luftdurchfluss aufgrund verringerter Luftdurchlässigkeit des Filters.} durch den Filter ermitteln lassen.
Wenn der Luftfilter zugesetzt ist, leuchtet auf dem Vpad||Bildschirm das Kontrollsymbol~\textSymb{vpadWarningFilter} auf und die Fehlermeldung \VpadEr{851} wird registriert.


\subsubsubject{Instandhalten/Ersetzen}

\startSteps
\item Ziehen Sie die Sicherheitslasche~\Lone nach unten (\in{Abb.}[fig:airFilter]).
\item Drehen Sie das Gehäuseunterteil~\Ltwo im Gegenuhrzeigersinn und nehmen Sie es ab.
\item Entnehmen Sie das Filterelement und überprüfen Sie es. Ersetzen Sie es, falls erforderlich.
\item Reinigen Sie das Filter||Innere und bauen Sie den Luftfilter in umgekehrter Reihenfolge wieder zusammen.
\stopSteps

\page [yes]


\subsection [sSec:vw:belt] {Keilrippenriemen}

Der\index{Dieselmotor+Keilrippenriemen} Keilrippenriemen überträgt die Bewegung der Kurbelwellenschwungscheibe auf die Lichtmaschine und den Klimakompressor (optionale Ausstattung).
Ein\index{Keilrippenriemen} Spannelement im letzten Segment (zwischen Lichtmaschine und Kurbelwelle) hält den Riemen unter Spannung.


\subsubsection [sSec:belt:change] {Ersetzen des Keilrippenriemens}

\placefig [margin] [fig:belt:tool] {Absteckdorn VW T\,10060\,A}
{\externalfigure[vw:belt:tool]}

\placefig [margin] [fig:belt:overview] {Spannelement}
{\externalfigure[vw:belt:overview]}

\placefig [margin] [fig:belt:tens] {Ansatzstelle des Absteckdorns}
{\externalfigure[vw:belt:tens]}


\subsubsubject{Mit Klimakompressor}


{\sla Benötigtes Spezialwerkzeug:}

Absteckdorn \aW{VW T\,10060\,A} zum Halten des Spannelements.

\startSteps
\item Markieren Sie die Laufrichtung des Keilrippenriemens.
\item Schwenken Sie mit einem gekröpftem Ringschlüssel den Arm des Spannelements im Uhrzeigersinn (\in {Abb.}[fig:belt:overview]).
\item Bringen Sie die Bohrungen (siehe Pfeile, \in {Abb.}[fig:belt:tens]) zur Deckung und arretieren Sie das Spannelement mit dem Absteckdorn.
\item Nehmen Sie den Keilrippenriemen ab.
\stopSteps

Der Einbau des Keilrippenriemens erfolgt in umgekehrter Reihenfolge.

\starttextbackground [FC]
\startPictPar
\PMrtfm
\PictPar
\startitemize [1]
\item Achten Sie auf die Laufrichtung des Keilrippenriemens.
\item Achten Sie auf den korrekten Sitz des Riemens in den Riemenscheiben.
\item Starten Sie den Motor und kontrollieren Sie den Riemenlauf.
\stopitemize
\stopPictPar
\stoptextbackground


\subsubsubject{Ohne Klimakompressor}

{\sla Benötigtes Material:}

Reparaturkit, bestehend aus Reparaturanleitung, Keilrippenriemen und Spezialwerkzeug.\footnote{Siehe Ersatzteilkatalog, unter \aW{Wartungsteile}.}

\startSteps
\item Durchtrennen Sie den Keilrippenriemen.
\item Folgen Sie den weiteren Schritten in der Reparaturanleitung.
\stopSteps

\starttextbackground [FC]
\startPictPar
\PMrtfm
\PictPar
\startitemize [1]
\item Achten Sie auf den korrekten Sitz des Riemens in den Riemenscheiben.
\item Starten Sie den Motor und kontrollieren Sie den Riemenlauf.
\stopitemize
\stopPictPar
\stoptextbackground


\subsubsection [sSec:belt:tens] {Ersetzen des Spannelements}

{\sla Nur für Ausführung mit Klimakompressor}

\blank [medium]

\placefig [margin] [fig:belt:tens:change] {Ersetzen des Spannelements}
{\externalfigure[vw:belt:tens:change]
\noteF
\startLeg
\item Spannelement
\item Sicherungsschraube
\stopLeg

{\bf Anzugsdrehmoment}

Sicherungsschraube:

\TorqueR~20\,Nm\:+\,½~Umdrehung (180°).}

\startSteps
\item Bauen Sie den Keilrippenriemen wie beschrieben aus (siehe \atpage[sSec:belt:change]).
\item Bauen Sie periphere Teile aus (je nach Ausstattung).
\item Schrauben Sie die Sicherungsschraube heraus (\in{\Ltwo, Abb.}[fig:belt:tens:change]).
\stopSteps

Der Einbau des Spannelements erfolgt in umgekehrter Reihenfolge.

\starttextbackground [FC]
\startPictPar
\PMrtfm
\PictPar
\startitemize [1]
\item Verwenden Sie nach dem Einbau unbedingt eine neue Sicherungsschraube.
\item Anzugsdrehmoment: Siehe \in{Abb.}[fig:belt:tens:change].
\stopitemize
\stopPictPar
\stoptextbackground

\stopregister[index][reg:main:vw]

\stopsection

\page[yes]


\setups[pagestyle:marginless]


\startsection[title={Hydraulikanlage},
							reference={sec:hydraulic}]

\starttextbackground [FC]
% \startfiguretext[left,none]{}
% {\externalfigure[toni_melangeur][width=30mm]}

\startSymPar
\externalfigure[toni_melangeur][width=4em]
\SymPar
\textDescrHead{Recyceln von Betriebsstoffen}
Gebrauchte Betriebsstoffe und Schmiermittel dürfen weder in der Natur entsorgt noch verbrannt
werden.

Gebrauchte Schmiermittel dürfen weder in das Abwassernetz noch in die Natur
geleitet werden und dürfen nicht im Hausmüll entsorgt werden.

Gebrauchte Schmiermittel dürfen nicht mit anderen Flüssigkeiten gemischt werden,
da die Gefahr besteht, dass Giftstoffe oder schwer entsorgbare Stoffe entstehen.
\stopSymPar
\stoptextbackground
\blank [big]

% \starthangaround{\PMgeneric}
% \textDescrHead{Qualification du personnel}
% Toute intervention sur l’installation hydraulique de votre véhicule ne peut être réalisée que par une personne dument qualifiée, ou par un service reconnu par \boschung.
% \stophangaround
% \blank[big]

\startSymList
\PHgeneric
\SymList
\textDescrHead{Sauberkeit} Die Hydraulikanlage reagiert sehr empfindlich auf
Unreinheiten im Öl. Daher ist es wichtig, in einer absolut sauberen Umgebung zu
arbeiten.
\stopSymList

\startSymList
\PHhot
\SymList
\textDescrHead{Spritzgefahr}
Vor Arbeiten an der Hydraulikanlage der \sdeux\ muss der Restdruck im
jeweiligen Hydraulikkreis abgelassen werden. Heiße Ölspritzer können zu
Verbrennungen führen.
\stopSymList

\startSymList
\PHhand
\SymList
\textDescrHead{Quetschgefahr}
Der Schmutzbehälter muss unbedingt abgesenkt oder mechanisch mittels der
Sicherheitsstütze gesichert sein, bevor Arbeiten an der Hydraulikanlage
der \sdeux\ vorgenommen werden.
\stopSymList

\startSymList
\PImano
\SymList
\textDescrHead{Druckmessung}
Um den Hydraulikdruck zu messen, bringen Sie ein Manometer auf einem der
\aW{Minimess}||Anschlüsse des Kreises an. Achten Sie darauf, dass das Manometer
einen geeigneten Messbereich aufweist.
\stopSymList

\page [yes]

\setups[pagestyle:normal]

\subsection{Wartungsintervalle}

\start

	\setupTABLE	[frame=off,
				style={\ssx\setupinterlinespace[line=.93\lH]},
				background=color,
				option=stretch,
				split=repeat]
	\setupTABLE	[r]	[each]	[
							topframe=on,
							framecolor=white,
							backgroundcolor=TableLight,
							% rulethickness=.8pt,
							]

	% \setupTABLE	[c]	[odd]	[backgroundcolor=TableMiddle]
	% \setupTABLE	[c]	[even]	[backgroundcolor=TableLight]
	\setupTABLE	[c]	[1]		[ % width=30mm,
							style={\bfx\setupinterlinespace[line=.93\lH]},
							]
	\setupTABLE	[r] [first]	[topframe=off,
							style={\bfx\setupinterlinespace[line=.93\lH]},
							backgroundcolor=TableMiddle,
							]
	% \setupTABLE	[r]	[2]		[style={\ssBfx\setupinterlinespace[line=.93\lH]}]


\bTABLE

\bTABLEhead
\bTR\bTD Wartungsarbeit \eTD\bTD Intervall \eTD\eTR
\eTABLEhead

\bTABLEbody
\bTR\bTD Auf Leckagen prüfen \eTD\bTD Täglich \eTD\eTR
\bTR\bTD Hydraulikölstand kontrollieren \eTD\bTD Täglich \eTD\eTR
\bTR\bTD Zustand der Hydraulikleitungen|/|-schläuche prüfen; ggf. ersetzen \eTD\bTD 600\,h / 12~Monate \eTD\eTR
\bTR\bTD Hydrauliköl||Rücklauf- und Ansaugfilter wechseln \eTD\bTD 600\,h / 12~Monate \eTD\eTR
\bTR\bTD Spulenkerne der Magnetventile mit Kupferfett schmieren \eTD\bTD 600\,h / 12~Monate \eTD\eTR
\bTR\bTD Hydrauliköl wechseln \eTD\bTD 1200\,h / 24~Monate \eTD\eTR
\eTABLEbody
\eTABLE
\stop


\subsection[niveau_hydrau]{Füllstand}

\placefig[margin][fig:hydraulic:level]{Füllstand der Hydraulikflüssigkeit}
{\externalfigure[hydraulic:level]
\noteF
\startLeg
\item Optimaler Füllstand
\stopLeg}

Ein transparentes
Schauglas\index{Füllstand+Hydraulikflüssigkeit}\index{Wartung+Hydraulikanlage}
ermöglicht eine Prüfung des Hydraulikölstands.
Wenn der Hydraulikölstand gesunken ist, muss die Ursache
ermittelt werden, bevor wieder aufgefüllt wird. Halten Sie sich an die
vorgeschriebenen Wechselintervalle (Tabelle oben) und Spezifikationen für die
Hydraulikflüssigkeit (Tabelle \at{Seite}[sec:liqquantities]).


\subsubsection{Hydraulikflüssigkeit nachfüllen}

Füllen Sie Hydraulikflüssigkeit nach, bis das mittlere Schauglas voll gedeckt ist.
Starten Sie den Motor und füllen Sie ggf. etwas nach, bis der erforderliche
Füllstand erreicht ist.


\subsection{Hydraulikflüssigkeit wechseln}

Füllmenge und erforderliche Spezifikationen der Hydraulikflüssigkeit finden Sie in der
Tabelle auf \at{Seite}[sec:liqquantities].

\startSteps
\item Öffnen Sie die Nachfüllöffnung des Hydrauliktanks.
\item Leeren Sie den Tank mithilfe eines Ölabsaugers oder entfernen Sie die Ablassschraube.

Die Ablassschraube befindet sich unten am Hydrauliktank, vor dem linken Hinterrad (\in{Abb.}[fig:hydraulic:fluidDrain]).
\item Füllen Sie Hydraulikflüssigkeit nach, bis das mittlere Schauglas voll gedeckt ist.
Starten Sie den Motor und füllen Sie ggf. etwas nach, bis der erforderliche Füllstand erreicht ist.
\stopSteps

\placefig[margin][fig:hydraulic:fluidDrain]{Ablassschraube}
{\externalfigure[hydraulic:fluidDrain]}


\placefig[margin][fig:hydraulic:returnFilter]{Hydraulikfilter}
{\externalfigure[hydraulic:returnFilter]}

\subsection[filtres:nettoyage]{Rücklauf- und Ansaugfilter}

\startSteps
\item Heben Sie den Schmutzbehälter und bringen Sie die Sicherheitsstütze an.
\item Nehmen Sie den Deckel des Filters auf dem Hydrauliktank ab (\in{Abb.}[fig:hydraulic:returnFilter]).
\item Ersetzen\index{Ölfilter+Hydraulik-} Sie das Filterelement durch ein neues.
\item Benetzen Sie eine neue O-Ring||Dichtung mit etwas Hydraulikflüssigkeit und bringen Sie ihn an.
\item Schrauben Sie den Deckel mit zwei Händen wieder auf (\TorqueR~ca.~20\,Nm).
\stopSteps

\page [yes]


\subsection[sec:solenoid]{Schmieren der Magnetventile}

\placefig[margin][graissage_bobine]{Schmieren der Magnetventile}
{\externalfigure[graissage_bobine][M]
\noteF
\startLeg
\item Spule des Magnetventils
\item Spulenkern
\stopLeg}

Feuchtigkeit und Salzrückstände, die in den Kern der elektromagnetischen Spulen
gelangen, führen zur Korrodierung der Kerne. Die Spulenkerne müssen ein Mal pro
Jahr mit Kupferfett geschmiert werden. Das Fett muss korrosions-, wasser- und
temperaturbeständig bis 50\,°C sein:
\startSteps
\item Bauen Sie die Spule des Magnetventils aus (\in{\Lone, Abb.}[graissage_bobine]).
\item Schmieren Sie den Kern (\in{\Ltwo, Abb.}[graissage_bobine]) mit dem
vorgeschriebenen Spezialfett und bauen Sie die Spule wieder ein.
\stopSteps


\subsection{Auswechseln der Schläuche}

Der Schutzgummi\index{Schläuche+Wechselintervalle} und das Verstärkungsgewebe
der Schläuche unterliegen einer natürlichen Alterung. Deswegen müssen die
Schläuche der Hydraulikanlage unbedingt in den vorgeschriebenen Intervallen
gewechselt werden, auch wenn keine {\em sichtbaren} Schäden vorhanden sind.

Achten Sie darauf, dass die Schläuche korrekt am Fahrzeug angeflanscht werden,
um einen vorzeitigen Verschleiß durch Reibung auszuschließen. Sie müssen
ausreichend Abstand zu anderen Bauteilen aufweisen, sodass Reibungs- und
Vibrationsschäden verhindert werden.

\stopsection

\page [yes]

\setups [pagestyle:bigmargin]


\startsection[title={Bremssystem},
							reference={sec:brake}]

\placefig[margin][fig:brake:rear]{Trommelbremse}
{\startcombination [1*2]
{\externalfigure[brake:wheelHub]}{\slx Hinterradnabe}
{\externalfigure[brake:drum]}{\slx Mechanismus und Bremsgarnituren}
\stopcombination}

Die Bremstrommeln~\Lfour\ müssen bei jeder regulären Wartung ausgebaut werden,
der Bremsmechanismus~\Lseven\ muss gereinigt und die Bremsgarnituren~\Lfive,
\Lsix\ einer Sichtkontrolle unterzogen werden (\in{Abb.}[fig:brake:rear]).


\subsubject {Ausbau}

\startSteps
\item Fahren Sie das Fahrzeug auf eine geeignete Hebebühne und heben Sie die Räder an.
\item Nehmen Sie die Räder ab.
\stopSteps


{\sla Ausbau der Vorderradbremsen}

\startSteps [continue]
\item Bauen Sie die Bremstrommel~\Lfour\ aus.
\stopSteps

{\sla Ausbau der Hinterradbremsen}

\startSteps [continue]
\item Nehmen Sie die Abdeckung~\Lone\ von der Nabe ab.
\item Entfernen Sie die Schraube~\Ltwo\ und nehmen Sie das Zwischenstück ab.
\item Schrauben Sie die Nabenmutter~\Lthree\ mit einem Steckschlüssel ab.
\item Nehmen Sie die Nabe mit der Bremstrommel ab.
\stopSteps


\subsubject {Wiedereinbau}

Bauen Sie die Bremstrommeln in umgekehrter Reihenfolge wieder ein. Ziehen Sie
die Muttern der Hinterradnaben~\Lthree\ mit dem vorgeschriebenen
Anzugsdrehmoment von 190\,Nm an.

\stopsection

\page [yes]

\setups [pagestyle:normal]


\startsection[title={Kontrolle und Wartung der Reifen},
							reference={sec:pneumatiques}]

Die Reifen\index{Reifen+Wartung} müssen stets in einwandfreien Zustand
sein, damit sie ihre beiden Hauptfunktionen erfüllen können: gute Haftung und
einwandfreies Bremsverhalten. Unzulässig hohe Abnutzung und falscher Fülldruck,
insbesondere zu niedriger Druck, sind wichtige Unfallfaktoren.


\subsection{Sicherheitsrelevante Punkte}

\subsubsection{Verschleißkontrolle}

Der Reifenverschleiß muss anhand der Verschleißindikatoren, die sich in einer
Profilrille befinden (\in{Abb.}[pneususure]), kontrolliert werden.
Auffälligkeiten am Reifen und deren Ursachen lassen sich mittels Sichtkontrolle feststellen:

\placefig[margin][pneususure]{Verschleißkontrolle}
{\Framed{\externalfigure[pneusUsure][M]}}

\placefig[margin][pneusdomages]{Beschädigte Reifen}
{\Framed{\externalfigure[pneusDomages][M]}}

\startitemize
\item Abnutzung an den Seiten der Lauffläche: Fülldruck zu niedrig.
\item Verstärkte Abnutzung in der Mitte: Fülldruck zu hoch.
\item Asymmetrische Abnutzung an den Seiten des Reifens: Vorderachse (Spur,
Achsgeometrie) falsch eingestellt.
\item Risse in der Lauffläche: Reifen zu alt; der Reifengummi wird mit der Zeit
härter und rissig (\in{Abb.}[pneusdomages]).
\stopitemize

\starttextbackground[CB]
\startPictPar
\PHgeneric
\PictPar
\textDescrHead{Risiken durch abgefahrene Reifen}
Ein abgefahrener Reifen erfüllt seine Funktionen nicht mehr, insbesondere in
Bezug auf Wasser- und Schlammableitung; der Bremsweg verlängert sich und das
Fahrverhalten verschlechtert sich. Ein abgefahrener Reifen rutscht leichter, vor
allem bei Nässe. Die Gefahr, dass der Reifen die Haftung verliert, steigt.
\stopPictPar
\stoptextbackground


\subsubsection{Reifenfülldruck}

Der vorgeschriebene Reifenfülldruck ist auf dem Rädertypenschild, vorne an der
Konsole auf der Beifahrerseite, notiert (siehe \atpage [sec:plateWheel]).

Selbst\index{Reifen+Fülldruck} wenn sich die Reifen in einem guten Zustand
befinden, verlieren sie im Laufe der Zeit mehr oder weniger schnell Luft (je
mehr das Fahrzeug gefahren wird, desto höher ist der Druckverlust). Daher muss
der Reifendruck monatlich, bei kalten Reifen, überprüft werden. Wenn Sie den
Druck bei warmen Reifen prüfen, müssen Sie 0,3\,bar zum vorgeschriebenen Druck
hinzufügen.

\start
\setupcombinations[M]
\placefig[margin][pneuspression]{Reifenfülldruck}
{\Framed{\externalfigure[pneusPression][M]}
\noteF
\startLeg
\item Korrekter Druck
\item Zu hoher Druck
\item Zu niedriger Druck
\stopLeg
Der vorgeschriebene Reifendruck ist auf dem Typenschild der Räder, im Führerhaus
auf der Beifahrerseite, angegeben.}
\stop

\starttextbackground[CB]
\startPictPar
\PHgeneric
\PictPar
\textDescrHead{Gefahren durch zu geringen Reifenfülldruck}
Ein Reifen kann zerreißen, wenn der Druck zu niedrig ist. Der Reifen wird mehr
zusammengedrückt, wenn er nicht ausreichend aufgepumpt ist oder wenn das
Fahrzeug überladen ist. Der Gummi wird dadurch heiß und und Teile des Reifens
können sich in einer Kurve ablösen.
\stopPictPar
\stoptextbackground

\stopsection

\page [yes]

\setups[pagestyle:marginless]


\startsection[title={Chassis},
							reference={main:chassis}]

\subsection{Sicherheitsrelevante Befestigungen von Komponenten}

Bei jeder Wartung muss der korrekte Sitz sicherheitsrelevanter Befestigungsschrauben bestimmter Komponenten überprüft werden, einschließlich Überprüfung des vorgeschriebenen Anzugsdrehmoments. Dies gilt insbesondere für das Knicklenksystem und die Achsen.

\blank [big]

\startfigtext [left] [fig:frontAxle:fixing] {Vorderachse}
{\externalfigure [frontAxle:fixing]}
{\sla Befestigungen der Vorderachse}
\startLeg
\item Befestigung des Federblatts: \TorqueR~150\,Nm
\item Befestigung der Zugeinheiten: \TorqueR~78\,Nm
\stopLeg

{\sla Befestigungen der Hinterachse}
\startLeg
\item Befestigung des Federblatts: \TorqueR~150\,Nm
\stopLeg

\stopfigtext

\start

\setupTABLE	[frame=off,style={\ssx\setupinterlinespace[line=.93\lH]},background=color,
			option=stretch,
			split=repeat]

\setupTABLE	[r]	[each]	[topframe=on,
						framecolor=white,
						% rulethickness=.8pt
						]

\setupTABLE	[c]	[odd]	[backgroundcolor=TableMiddle]
\setupTABLE	[c]	[even]	[backgroundcolor=TableLight]
\setupTABLE	[c]	[1]		[style={\bfx\setupinterlinespace[line=.93\lH]}]
\setupTABLE	[r] [first]	[topframe=off,style={\bfx\setupinterlinespace[line=.93\lH]},
						]
% \setupTABLE	[r]	[2]		[style={\bfx\setupinterlinespace[line=.93\lH]}]


\bTABLE

\bTABLEhead
\bTR [backgroundcolor=TableDark] \bTD [nc=3] Anzugsdrehmomente \eTD\eTR
% \bTR\bTD Position \eTD\bTD Type de vis \eTD\bTD Couple \eTD\eTR
\eTABLEhead

\bTABLEbody
\bTR\bTD Antriebsmotoren links|/|rechts \eTD\bTD M12\:×\:35~8.8 \eTD\bTD 78\,Nm \eTD\eTR
%% NOTE @Andrew: das sind Hydraulikmotoren
\bTR\bTD Arbeitspumpe \eTD\bTD M16\:×\:40~100 \eTD\bTD 330\,Nm \eTD\eTR
\bTR\bTD Antriebspumpe \eTD\bTD M12\:×\:40~100 \eTD\bTD 130\,Nm \eTD\eTR
\bTR\bTD Federblätter vorne|/|hinten \eTD\bTD M16\:×\:90|/|160~8.8 \eTD\bTD 150\,Nm \eTD\eTR
% \bTR\bTD Fixation du système oscillant \eTD\bTD M12\:×\:40~8.8 \eTD\bTD 78\,Nm \eTD\eTR
\bTR\bTD Befestigung des Schmutzbehälters \eTD\bTD M10\:×\:30 Verbus Ripp~100 \eTD\bTD 80\,Nm \eTD\eTR
\bTR\bTD Radmuttern \eTD\bTD M14\:×\:1,5 \eTD\bTD 180\,Nm \eTD\eTR
\bTR\bTD Befestigung des Frontbesens \eTD\bTD M16\:×\:40~100 \eTD\bTD 180\,Nm \eTD\eTR
\eTABLEbody
\eTABLE
\stop


\stopsection

\page [yes]


\startsection[title={Zentralschmieranlage},
							reference={main:graissageCentral}]


\subsection{Beschreibung des Steuermoduls}

Die \sdeux\ kann mit\index{Zentralschmieranlage} einer Zentralschmieranlage
ausgestattet werden (Option). Die Zentralschmieranlage versorgt jeden
Schmierpunkt des Fahrzeugs in regelmäßigen Abständen mit Schmiermittel.

\startfigtext [left] [vogel_affichage] {Anzeigemodul}
{\externalfigure[vogel_base2][W50]}
\blank
\startLeg
	\item 7-stelliges Display: Werte und Betriebszustand
	\item \LED: System in Pause (Standby||Betrieb)
	\item \LED: Pumpe in Betrieb
	\item \LED: Steuerung des Systems durch Zyklusschalter
	\item \LED: Überwachung des Systems durch Druckschalter
	\item \LED: Fehlermeldung
	\item Bildlauftasten:
	\startLeg [R]
	\item Display aktivieren
	\item Werte anzeigen
	\item Werte ändern
	\stopLeg
	\item Taste zum Wechseln der Betriebsart; Bestätigung der Werte
	\item Auslösen eines Zwischenschmierzyklus
\stopLeg
\stopfigtext

Die Zentralschmieranlage umfasst die Schmiermittelpumpe, den durchsichtigen
Schmiermittelbehälter an der linken Seite des Chassis und das Steuermodul in der
Zentralelektrik.
% \blank
\page [yes]


\subsubsubject{Anzeige und Tasten des Steuermoduls}

\start

\setupTABLE	[frame=off,style={\ssx\setupinterlinespace[line=.93\lH]},background=color,
			option=stretch,
			split=repeat]

\setupTABLE	[r]	[each]	[topframe=on,
						framecolor=white,
						% rulethickness=.8pt
						]

\setupTABLE	[c]	[odd]	[backgroundcolor=TableMiddle]
\setupTABLE	[c]	[even]	[backgroundcolor=TableLight]
\setupTABLE	[c]	[1]		[width=9mm,style={\bfx\setupinterlinespace[line=.93\lH]}]
\setupTABLE	[r] [first]	[topframe=off,style={\bfx\setupinterlinespace[line=.93\lH]},
						]
% \setupTABLE	[r]	[2]		[style={\bfx\setupinterlinespace[line=.93\lH]}]


\bTABLE
\bTABLEhead
% \bTR [backgroundcolor=TableDark]
% \bTD [nc=4] Anzeige und Tasten des Steuermoduls \eTD\eTR
\bTR\bTD Pos. \eTD
\bTD \LED \eTD\bTD Anzeigemodus \eTD
\bTD Programmiermodus \eTD\eTR
\eTABLEhead

\bTABLEbody
	\bTR\bTD 2 \eTD
	\bTD Betriebszustand {\em Pause}\hskip .5em\null \eTD
	\bTD Die Anlage befindet sich in Standby\hskip .5em\null \eTD % ||Betrieb
	\bTD Die Pausenzeit kann geändert werden \eTD\eTR
	\bTR\bTD 3 \eTD
	\bTD Betriebszustand {\em Contact} \eTD
	\bTD Die Pumpe arbeitet \eTD
	\bTD Die Arbeitszeit kann geändert werden \eTD\eTR
	\bTR\bTD 4 \eTD
	\bTD Systemkontrolle {\em CS} \eTD
	\bTD Mit dem externen Zyklusschalter \eTD
	\bTD Der Kontrollmodus kann deaktiviert oder geändert werden \eTD\eTR
	\bTR\bTD 5 \eTD
	\bTD Systemkontrolle {\em PS} \eTD
	\bTD Mit dem externen Druckschalter \eTD
	\bTD Der Kontrollmodus kann deaktiviert oder geändert werden \eTD\eTR
	\bTR\bTD 6 \eTD
	\bTD Störung {\em Fault} \eTD
	\bTD [nc=2] Es liegt eine Funktionsstörung vor. Die Ursache wird in Form
	eines Fehlercodes angezeigt, nachdem die Taste~\textSymb{vogel_DK}
	gedrückt wurde. Die Ausführung der Funktionen wird unterbrochen. \eTD\eTR
	\bTR\bTD 7 \eTD
	\bTD Pfeiltasten \textSymb{vogelTop}~\textSymb{vogelBottom} \eTD
	\bTD [nc=2] \items[symbol=R]{Aktivieren des Displays,Abfrage der Parameter
	(Anzeigemodus),Einstellen des angezeigten (I) Wertes (Programmiermodus)}
	\eTD\eTR
	\bTR\bTD 8 \eTD
	\bTD Taste \textSymb{vogelSet} \eTD
	\bTD [nc=2] Zwischen dem Anzeige- und Programmiermodus umschalten oder die
	eingegebenen Werte bestätigen. \eTD\eTR
	\bTR\bTD 9 \eTD
	\bTD Taste \textSymb{vogel_DK} \eTD
	\bTD [nc=2] Wenn sich das Gerät im Zustand {\em Pause} befindet, löst eine
	Betätigung der Taste den Zwischenschmierzyklus aus. Die Fehlermeldungen
	werden bestätigt und gelöscht. \eTD\eTR
\eTABLEbody
\eTABLE
\stop
\vfill

\startfigtext [left] [vogel_touches]{Anzeigemodul}
{\externalfigure[vogel_base][width=50mm]}
\textDescrHead{Anzeigemodus} Kurz eine der
Pfeiltasten~\textSymb{vogelTop}~\textSymb{vogelBottom} drücken, um das
7-stellige Display~\textSymb{led_huit} zu aktivieren. Durch erneuten Druck auf
die Taste~\textSymb{vogelTop} können die verschiedenen Parameter gefolgt von
ihrem Wert angezeigt werden. Der Modus {\em Anzeige} ist an den ständig
leuchtenden \LED\char"2060s erkennbar (\in{2~bis 6, Abb.}[vogel_affichage]).
\blank [medium]
\textDescrHead{Programmiermodus} Zum Ändern der Werte drücken Sie für
mindenstens 2~Sekunden
die Taste~\textSymb{vogelSet}, um in den Modus {\em Programmierung}
umzuschalten: Die \LED\char"2060s blinken. Kurz die Taste~\textSymb{vogelSet} drücken,
um die\index{Zentralschmieranlage+Programmierung} Anzeige zu ändern, dann den
gewünschten Wert mit den Tasten~\textSymb{vogelTop}~\textSymb{vogelBottom}
ändern. Mit\index{Zentralschmieranlage+Anzeige} der Taste~\textSymb{vogelSet}
bestätigen.
\stopfigtext

\page [yes]


\subsection{Untermenüs im Modus {\em Anzeige}}

\vskip -9pt

\adaptlayout [height=+5mm]

\startcolumns[balance=no]\stdfontsemicn

\startSymVogel
\externalfigure[vogel_tpa][width=26mm]
\SymVogel
\textDescrHead{Pausenzeit [h]} Drücken Sie die Taste~\textSymb{vogelTop}, um die
programmierten Werte anzuzeigen.
\stopSymVogel

\startSymVogel
\externalfigure[vogel_068][width=26mm]
\SymVogel
\textDescrHead{Verbleibende Pausenzeit [h]} Noch verbleibende Zeit bis zum
nächsten Schmierzyklus.
\stopSymVogel

\startSymVogel
\externalfigure[vogel_090][width=26mm]
\SymVogel
\textDescrHead{Gesamtpausenzeit [h]} Gesamtpausenzeit zwischen zwei Zyklen.
\stopSymVogel

\startSymVogel
\externalfigure[vogel_tco][width=26mm]
\SymVogel
\textDescrHead{Schmierzeit [min]} Drücken Sie~\textSymb{vogelTop}, um die
programmierten Werte anzuzeigen.
\stopSymVogel

\startSymVogel
\externalfigure[vogel_tirets][width=26mm]
\SymVogel
\textDescrHead{Gerät in Standby} Anzeige nicht möglich, da Gerät in
Standby (Pause).
\stopSymVogel

\startSymVogel
\externalfigure[vogel_026][width=26mm]
\SymVogel
\textDescrHead{Schmierzeit [min]} Dauer eines Schmiervorgangs.
\stopSymVogel

\startSymVogel
\externalfigure[vogel_cop][width=26mm]
\SymVogel
\textDescrHead{Systemkontrolle} Drücken Sie~\textSymb{vogelTop}, um die
programmierten Werte anzuzeigen.
\stopSymVogel

\startSymVogel
\externalfigure[vogel_off][width=26mm]
\SymVogel
\textDescrHead{Kontrollmodus} \hfill PS:~Druckschalter;\crlf
CS:~Zyklusschalter; OFF:~deaktiviert.
\stopSymVogel

\startSymVogel
\externalfigure[vogel_0h][width=26mm]
\SymVogel
\textDescrHead{Betriebsstunden} Drücken Sie~\textSymb{vogelTop}, um den Wert
in zwei Schritten anzuzeigen.
\stopSymVogel

\startSymVogel
\externalfigure[vogel_005][width=26mm]
\SymVogel
\textDescrHead{Teil 1: 005} Die Betriebszeit wird in zwei Teilen angezeigt; zu
Teil~2 mit der Taste~\textSymb{vogelTop}.
\stopSymVogel

\startSymVogel
\externalfigure[vogel_338][width=26mm]
\SymVogel
\textDescrHead{Teil 2: 33,8} Der 2.~Teil der Zahl ist 33,8; ergibt zusammen
eine Betriebszeit von 533,8\,h.
\stopSymVogel

\startSymVogel
\externalfigure[vogel_fh][width=26mm]
\SymVogel
\textDescrHead{Fehlerzeit} Drücken Sie~\textSymb{vogelTop}, um den Wert in
zwei Schritten anzuzeigen.
\stopSymVogel

\startSymVogel
\externalfigure[vogel_000][width=26mm]
\SymVogel
\textDescrHead{Teil 1: 000} Die Fehlerzeit wird in zwei Teilen angezeigt;\crlf
zu Teil~2 mit~\textSymb{vogelTop}.
\stopSymVogel

\startSymVogel
\externalfigure[vogel_338][width=26mm]
\SymVogel
\textDescrHead{Teil 2: 33,8} Der 2.~Teil der Zahl ist 33,8; ergibt zusammen
eine Fehlerzeit von 33,8\,h.
\stopSymVogel

\stopcolumns

\stopsection


\page [yes]


\setups [pagestyle:marginless]


\startsection[title={Schmierplan für manuelles Schmieren},
							reference={sec:grasing:plan}]

\starttextbackground [FC]
\startPictPar
\PMgeneric
\PictPar
Die im Schmierplan (\in{Abb.}[fig:greasing:plan]) angegebenen Schmierstellen
müssen regelmäßig geschmiert werden. Eine regelmäßige Schmierung ist
unerlässlich, um eine dauerhafte {\em Verringerung der Reibung} zu gewährleisten
und um Feuchtigkeit und andere korrosive Substanzen fernzuhalten.
\stopPictPar
\stoptextbackground

\blank [big]

\start

\setupcombinations [width=\textwidth]

\placefig[here][fig:greasing:plan]{Schmierplan des Fahrzeugs}
{\startcombination [3*1]
{\externalfigure[frame:steering:greasing]}{\ssx Knicklenkung und Pendelmechanismus}
{\externalfigure[frame:axles:greasing]}{\ssx Achsen}
{\externalfigure[frame:sucMouth:greasing]}{\ssx Saugmund}
\stopcombination}

\stop

\vfill

\startLeg [columns,three]
\item Hubzylinder der Knicklenkung\crlf {\sl 2 Schmiernippel pro Zylinder}
\item Lager der Knicklenkung\crlf {\sl 2 Schmiernippel auf der linken Seite}
\columnbreak
\item Lager des Pendelmechanismus\crlf {\sl 1 Schmiernippel vor dem Tank}
\item Blattfedern\crlf {\sl 2 Schmiernippel pro Federblatt}
\columnbreak
\item Saugmund\crlf {\sl 1 Schmiernippel pro Rad}
\item Saugmund\crlf {\sl 1 Schmiernippel auf dem Zugarm}
\stopLeg



\page [yes]


\setups [pagestyle:bigmargin]


\subsubject{Schmieren des Schmutzbehälters}

Der Schmutzbehälter verfügt über 6~Schmierpunkte (2\:×\:4), die wöchentlich geschmiert werden müssen.

\blank [big]


\placefig[here][fig:greasing:container]{Hebemechanismus des Behälters}
{\externalfigure[container:mechanisme]}


\placelegende [margin,none]{}
{{\sla Legende:}

\startLeg
\item Linkes Lager des Behälters (2\:×)
\item Rechtes Lager des Behälters (2\:×)
\item Linker Hydraulikzylinder (oben)
\item Linker Hydraulikzylinder (unten)

{\em Wie rechter Zylinder (Punkt \in[greasing:point;hide]).}
\item Rechter Hydraulikzylinder (oben)
\item [greasing:point;hide] Rechter Hydraulikzylinder (unten)
\stopLeg}

\stopsection

\page [yes]



\startsection[title={Elektrische Anlage},
							reference={sec:main:electric}]

\subsection{Zentralelektrik im Chassis}

\startbuffer [fuses:preventive]
\starttextbackground [CB]
\startPictPar
\PHvoltage
\PictPar
\textDescrHead{Sicherheitsvorschriften}
Beachten Sie die Sicherheitsvorschriften in\index{Sicherungen+Chassis}
dieser\index{Relais+Chassis} Anleitung: Ersetzen Sie Sicherungen stets nur mit
Sicherungen der vorgeschriebenen Amperezahl; nehmen Sie Metallschmuck ab, bevor
Sie an der elektrischen\index{Elektrische Anlage} Anlage arbeiten (Ringe,
Armreife etc.).
\stopPictPar
\stoptextbackground
\stopbuffer


\subsubsubject{MIDI-Sicherungen}

\starttabulate[|l|r|p|]
\HL
\NC\md F\,1 \NC 5\,A  \NC Bremsleuchte, \aW{+\:15} OBD \NC\NR
\NC\md F\,2 \NC 5\,A  \NC \aW{+\:15} Motorsteuerung \NC\NR
\NC\md F\,3 \NC 7,5\,A \NC \aW{+\:30} Motorsteuerung und OBD \NC\NR
\NC\md F\,4 \NC 20\,A \NC Kraftstoffpumpe \NC\NR
\NC\md F\,5 \NC 20\,A \NC \aW{D\:+} Lichtmaschine, \aW{+\:15} Relais K\,1 \NC\NR
\NC\md F\,6 \NC 5\,A \NC Motorsteuerung \NC\NR
\NC\md F\,7 \NC 10\,A\NC Motor||Abgasaufbereitung \NC\NR
\NC\md F\,8 \NC 20\,A \NC Motorelektronik (Steuerung) \NC\NR
\NC\md F\,9 \NC 15\,A \NC Motor||Abgasaufbereitung, Kraftstoffpumpe, Vorglühen \NC\NR
\NC\md F\,10\NC 30\,A \NC Motorsteuerung \NC\NR
\NC\md F\,11\NC 5\,A \NC Rückfahrleuchte \NC\NR
%% NOTE @Andrew: Singular
\HL
\stoptabulate

\placefig [margin] [fig:electric:power:rear] {Zentralelektrik im Chassis}
{\externalfigure [electric:power:rear]
\noteF
\startKleg
\sym{K\,1} Elektronisches Motorsteuergerät
\sym{K\,2} Kraftstoffpumpe
\sym{K\,3} Freigabe des Anlassers
\sym{K\,4} Bremsleuchten
\sym{K\,5} {[}Reserve{]}
\sym{K\,6} Rückfahrleuchte
\sym{K\,7} Vorglühanlage
\stopKleg
}


\subsubsubject{MAXI-Sicherungen}

% \startcolumns [n=2]
\starttabulate[|l|r|p|]
\HL
\NC\md F\,15 \NC 50\,A \NC Hauptversorgung der Zentralelektrik \NC\NR
\HL
\stoptabulate

\page [yes]

\setups[pagestyle:marginless]


\subsection{Zentralelektrik im Führerhaus}

\startcolumns[rule=on]

\placefig [bottom] [fig:fuse:cab] {Sicherungen und Relais im Führerhaus}
{\externalfigure [electric:power:front]}

\columnbreak

\subsubsubject{Relais}

\index{Sicherungen+Führerhaus}\index{Relais+Führerhaus}

\starttabulate[|lB|p|]
\NC K\,2 	\NC Klimakompressor\NC\NR
\NC K\,3 	\NC Klimakompressor\NC\NR
\NC K\,4 	\NC Elektrische Wasserpumpe\NC\NR
\NC K\,5 	\NC Rundumkennleuchte\NC\NR
\NC K\,10 \NC Blinkfrequenzgeber\NC\NR
\NC K\,11 \NC Abblendlicht\NC\NR
\NC K\,12 \NC Fernlicht {[}Reserve{]} \NC\NR
\NC K\,13 \NC Arbeitsscheinwerfer\NC\NR
\NC K\,14 \NC Scheibenwischer||Intervallschaltung\NC\NR
\stoptabulate

\vskip -24pt

\placefig [bottom] [fig:fuse:access] {Zugangsklappe zur Zentralelektrik}
{\externalfigure [electric:power:cabin]}

\stopcolumns

\page [yes]


\subsubsubject{MINI-Sicherungen}

\startcolumns[rule=on]
% \setuptabulate[frame=on]
%\placetable[here][tab:fuses:cab]{Fusibles dans la cabine}
%{\noteF
\starttabulate[|lB|r|p|]
\NC F\,1  \NC 3\,A \NC Standlicht links \NC\NR
\NC F\,2  \NC 3\,A \NC Standlicht rechts \NC\NR
\NC F\,3  \NC 7,5\,A \NC Abblendlicht links \NC\NR
\NC F\,4  \NC 7,5\,A \NC Abblendlicht rechts \NC\NR
\NC F\,5  \NC 7,5\,A \NC Fernlicht links {[}Reserve{]} \NC\NR
\NC F\,6  \NC 7,5\,A \NC Fernlicht rechts {[}Reserve{]} \NC\NR
\NC F\,7  \NC 10\,A \NC Arbeitsscheinwerfer oben \NC\NR
%% NOTE @Andrew: Plural
\NC F\,8  \NC 10\,A \NC Arbeitsscheinwerfer unten (Reserve) \NC\NR
%% NOTE @Andrew: Plural
\NC F\,9  \NC 10\,A \NC Frontbesen \NC\NR
\NC F\,10 \NC 10\,A \NC Scheibenwischer \NC\NR
\NC F\,11 \NC 5\,A \NC Schalter Beleuchtung und Warnblinkanlage \NC\NR
\NC F\,12 \NC 5\,A \NC {[}Reserve{]} \NC\NR
\NC F\,13 \NC 10\,A \NC Außenspiegelheizung \NC\NR
\NC F\,14 \NC 7,5\,A \NC \aW{+\:15} Radio und Kamera \NC\NR
\NC F\,15 \NC 10\,A \NC \aW{+\:30} Warnblinkanlage \NC\NR
\NC F\,16 \NC 5\,A \NC Beleuchtung Lenksäule \NC\NR
\NC F\,17 \NC 7,5\,A \NC \aW{+\:30} Radio, Lichtschalter und Innenbeleuchtung \NC\NR
\NC F\,18 \NC — \NC {[}Frei{]} \NC\NR
\NC F\,19 \NC 20\,A \NC \aW{+\:30} RC\,12 vorne \NC\NR
\NC F\,20 \NC 20\,A \NC \aW{+\:30} RC\,12 hinten \NC\NR
\NC F\,21 \NC 15\,A \NC 12-V-Steckdose \NC\NR
\NC F\,22 \NC 5\,A \NC Zündschlüssel, Multifunktionskonsole, Vpad \NC\NR
\NC F\,23 \NC 5\,A \NC Not||Halt, Mittelkonsole, RC\,12 vorne \NC\NR
\NC F\,24 \NC 5\,A \NC Not||Halt, Mittelkonsole, RC\,12 hinten \NC\NR
\NC F\,25 \NC 2\,A \NC \aW{+\:15} RC\,12 vorne \NC\NR
\NC F\,26 \NC 2\,A \NC \aW{+\:15} RC\,12 hinten \NC\NR
\NC F\,27 \NC 25\,A \NC Heizungsgebläse \NC\NR
\NC F\,28 \NC 10\,A \NC Klimakompressor, Zentralschmieranlage \NC\NR
\NC F\,29 \NC 25\,A \NC Klimakondensator \NC\NR
\NC F\,30 \NC 5\,A \NC Thermostat Klimaanlage \NC\NR
\NC F\,31 \NC 5\,A \NC \aW{+\:15} Multifunktionskonsole|/|Vpad \NC\NR
\NC F\,32 \NC 15\,A \NC Elektrische Wasserpumpe, Rundumkennleuchte \NC\NR
\NC F\,33 \NC — \NC {[}Frei{]} \NC\NR
\NC F\,34 \NC — \NC {[}Frei{]} \NC\NR
\NC F\,35 \NC — \NC {[}Frei{]} \NC\NR
\NC F\,36 \NC — \NC {[}Frei{]} \NC\NR
\stoptabulate
\stopcolumns

\page [yes]

\setups [pagestyle:bigmargin]


\subsection[sec:lighting]{Beleuchtungs- und Signaleinrichtung}


\placefig [here] [fig:lighting] {Beleuchtungs- und Signaleinrichtung des Fahrzeugs}
{\externalfigure [vhc:electric:lighting]}

\placelegende [margin,none]{}{%
\vskip 30pt
{\sla Legende:}
\startLongleg
\item Standlichter\hfill 12\,V – 5\,W
\item Abblendlichter\hfill H7 12\,V – 55\,W
\item Blinker\hfill orange 12\,V – 21\,W
\item {\stdfontsemicn Arbeitsscheinwerfer}\hfill G886 12\,V – 55\,W
\item Fahrtrichtungsanzeiger\hfill 12\,V – 21\,W
\item Rück-|/|Bremsleuchten\hfill 12\,V – 5|/|21\,W
\item Rückfahrscheinwerfer\hfill 12\,V – 21\,W
\item {[}Frei{]}
\item Kennzeichenbeleuchtung\hfill 12\,V – 5\,W
\item Rundumkennleuchte\hfill H1 12\,V – 55\,W
\stopLongleg}

\subsubsubject{Einstellen der Scheinwerfer}

\placefig [margin] [fig:lighting:adjustment] {Lichtstrahl bei 5\,m}
{\externalfigure [vhc:lighting:adjustment]
\startitemize
\sym{H\low{1}} Höhe des Leuchtfadens: 100\,cm
\sym{H\low{2}} Korrektur bei 2\hairspace\%: 10\,cm
\stopitemize}

{\md Voraussetzungen:} Frisch-|/|Recyclingwasserbehälter voll, Fahrer am Steuer.

Die Ausrichtung der Scheinwerfer wird im Werk voreingestellt. Höhe und Neigung des Lichtstrahls können durch Schwenken der Kunststoffhalterung eingestellt werden.

Wenn sich im Rahmen einer Überprüfung herausstellt, dass die Einstellung
verändert werden muss, lösen Sie die Sicherungsschraube und korrigieren
Sie die Neigung so, dass sie den gesetzlichen Vorschriften entspricht (siehe
\in{Abb.}[fig:lighting:adjustment]). Ziehen Sie die Sicherungsschraube wieder fest.

\page [yes]
\setups [pagestyle:marginless]


\subsection[sec:battcheck]{Batterie}

\subsubsection{Sicherheitsvorschriften}

\startSymList
\PPfire
\SymList
\textDescrHead{Explosionsgefahr}
Beim\index{Batterie+Sicherheitshinweise}\index{Gefahr+Explosion} Laden von
Batterien bildet sich explosives\index{Knallgas} Knallgas. Laden Sie
Batterien nur in gut belüfteten Räumen! Vermeiden Sie Funkenbildung!
Hantieren Sie in der Nähe der Batterie nicht mit Feuer, offenem Licht und
rauchen Sie nicht.
\stopSymList

\startSymList
\PHvoltage
\SymList
\textDescrHead{Kurzschlussgefahr}
Wenn\index{Batterie+Wartung} die Plusklemme der angeschlossenen Batterie mit
Fahrzeugteilen in Berührung kommt,
besteht\index{Gefahr+Feuer}\index{Gefahr+Explosion} Kurzschlussgefahr.
Dadurch kann das aus der Batterie austretende Gasgemisch explodieren, Sie und
andere könnten schwer verletzt werden.

\startitemize
\item Legen Sie keine Metallgegenstände oder Werkzeug auf die Batterie.
\item Beim Abklemmen der Batterie immer zuerst die Minus-, dann die
Plusklemme abnehmen.
\item Beim Anklemmen der Batterie immer zuerst die Plus-, dann die
Minusklemme anbringen.
\item Bei laufendem Motor die Anschlussklemmen der Batterie nicht lockern
oder abnehmen.
\stopitemize
\stopSymList


\startSymList
\PHcorrosive
\SymList
\textDescrHead{Verätzungsgefahr}
Tragen Sie\index{Gefahr+Verätzung} eine Schutzbrille und säurefeste
Schutzhandschuhe. Batterieflüssigkeit ist etwa 27\percent||ige
Schwefelsäure (H\low{2}SO\low{4}) und kann daher zu Verätzungen führen.
Neutralisieren Sie\index{Batterie+Gefahr}\index{Batterie+-flüssigkeit}
Batterieflüssigkeit, die auf die Haut gelangt ist, mit einer Lösung aus
doppeltkohlensaurem Natron und spülen Sie mit sauberem Wasser nach.  Falls
Batterieflüssigkeit in die Augen gelangt ist, spülen Sie diese mit viel kaltem
Wasser und suchen Sie sofort einen Arzt auf.
\stopSymList

\startSymList
\startcombination[1*2]
 {\PHcorrosive}{}
 {\PHfire}{}
 \stopcombination
\SymList
\textDescrHead{Lagerung von Batterien}
Batterien\index{Batterie+lagern} immer aufrecht lagern. Andernfalls könnte
Batterieflüssigkeit austreten und zu Verätzungen oder~– bei Reaktion mit
anderen Substanzen~– zu Bränden führen. \par\null\par\null
\stopSymList

\testpage [16]

\starttextbackground [FC]
\setupparagraphs [PictPar][1][width=2.4em,inner=\hfill]

\startPictPar
\PMproteyes
\PictPar
\textDescrHead{Schutzbrille}
Beim\index{Gefahr+Augenverletzung} Mischen von Wasser und Säure kann die
Flüssigkeit in die Augen spritzen. Säurespritzer im Auge sofort mit klarem
Wasser ausspülen und unverzüglich einen Arzt aufsuchen!
\stopPictPar
\blank [small]

\startPictPar
\PMrtfm
\PictPar
\textDescrHead{Dokumentation}
Beim Umgang mit Batterien sind die in dieser Betriebsanleitung enthaltenen
Sicherheitshinweise, Schutzmaßnahmen und Vorgehensweisen unbedingt zu
beachten.
\stopPictPar
\blank [small]

\startPictPar
\PStrash
\PictPar
\textDescrHead{Umweltschutz}
Batterien\index{Umweltschutz} enthalten Schadstoffe. Entsorgen Sie Altbatterien
nie mit dem Hausmüll. Entsorgen Sie Batterien umweltgerecht. Geben Sie sie in
einer Fachwerkstatt oder bei einer Rücknahmestelle für Altbatterien ab.

Gefüllte Batterien immer aufrecht transportieren und lagern. Beim Transport sind
Batterien gegen Umkippen zu sichern. Aus den Entlüftungsöffnungen der
Verschlussstopfen kann Batterieflüssigkeit austreten und in die Umwelt gelangen.
\stopPictPar
\stoptextbackground

\page [yes]

\setups[pagestyle:normal]


\subsubsection{Praktische Ratschläge}

Für eine maximale Lebensdauer muss die Batterie möglichst immer voll geladen sein.

Eine\index{Batterie+Lebensdauer} Erhaltungsladung der Batterie während
längerer Abstellzeiten des Fahrzeugs verlängert nicht nur die Lebensdauer der
Batterie, sondern gewährt auch ständige Startbereitschaft.

\placefig[margin][fig:batterycompartment]{\select{caption}{Batteriefach (Wartungsklappe)}{Batteriefach}}
{\externalfigure[batt:compartment]}


\subsubsection{Instandhaltung}

Bei der Batterie der \sdeux\ handelt es sich um einen {\em wartungsfreien} Bleiakku. Außer dem Erhalten des geladenen Zustands und der Reinigung verlangt die Batterie keine Instandhaltungsmaßnahmen.

\startitemize
\item Achten Sie darauf, dass die Pole der Batterie stets sauber und trocken sind. Schmieren Sie die Pole leicht mit etwas säureabweisendem Fett ein.
\item Batterien, die\index{Batterie+laden} eine Ruhespannung
von\index{Batterie+Ruhespannung} weniger als 12,4\,V aufweisen, nachladen.
\stopitemize

\placefig[margin][fig:bclean]{Reinigen der Pole}
{\externalfigure[batt:clean]
\noteF
Verwenden\index{Batterie+reinigen}\index{Reinigung+Batterien} Sie warmes
Wasser, um das durch Korrosion entstandene weiße Pulver zu entfernen. Falls
ein Pol verrostet ist, klemmen Sie das Batteriekabel ab und säubern Sie den
Pol mit einer Drahtbürste. Versehen Sie die Pole mit einem dünnen Fettfilm.}


\subsubsection[sec:battery:switch]{Gebrauch des Batterietrennschalters}

{\sl Es ist nicht empfehlenswert, den Batterietrennschalter regelmäßig (beispielsweise täglich) zu betätigen!}

\startSteps
\item Schalten\index{Batterietrennschalter} Sie die Zündung aus und warten Sie anschließend etwa 1~Minute.
\item Öffnen Sie das Batteriefach (\inF[fig:batterycompartment]).
\item Drücken Sie auf den roten Knauf des Batterietrennschalters, um den Stromkreis zu unterbrechen.
\item Um den Stromkreis wieder zu schließen, drehen Sie den Batterietrennschalter ¼~Umdrehung im Uhrzeigersinn.
\stopSteps

% \starttextbackground [FCnb]
% \startPictPar
% \PMgeneric
% \PictPar
% Der Batterietrennschalter ist dafür vorgesehen, die Batterie für bestimmte Wartungs- und Reparaturarbeiten vorübergehend vom Stromkreis zu trennen. Es ist nicht empfehlenswert, den Batterietrennschalter regelmäßig (\eG\ täglich) zu betätigen: Bestimmte elektronische Komponenten sollten ständig unter Spannung stehen, ansonsten kann es zu Fehlermeldungen im Fehlerspeicher kommen.
% \stopPictPar
% \stoptextbackground

\stopsection

\page [yes]


\setups[pagestyle:marginless]

\section[sec:cleaning]{Fahrzeugreinigung}

Spülen\startregister[index][vhc:lavage]{Wartung+Reinigung} Sie vor der
eigentlichen Reinigung groben Schlamm und Dreck mit reichlich Wasser von der
Karosserie. Waschen Sie dabei nicht nur die Seitenflächen, sondern auch die
Radgehäuse und die Unterseite des Fahrzeugs.

Besonders im Winter muss das Fahrzeug gründlich gewaschen werden, um es von den
hochkorrosiven\index{Korrosion+Vorbeugung} Streusalzrückständen zu befreien.

\starttextbackground [FC]
\startPictPar
\PHgeneric
\PictPar
\textDescrHead{Schäden durch Wasser verhindern}
Reinigen Sie das Fahrzeug nie mithilfe von {\em Wasserkanonen} (\eG\ der
Feuerwehr) oder {\em Kaltreinigern auf Kohlenwasserstoffbasis.} Wenn Sie mit
einem Hochdruck||Dampfreiniger arbeiten, beachten Sie die diesbezüglichen
Vorschriften weiter unten.
\stopPictPar
\blank[small]

\startPictPar
\pTwo[monde]
\PictPar
\textDescrHead{Umweltschutz}
Das Reinigen eines Fahrzeugs kann zu schweren Umweltbelastungen führen.
Reinigen Sie das Fahrzeug nur an einem mit einem\index{Umweltschutz}
Ölabscheider ausgerüsteten Standort. Beachten Sie die geltenden
Umweltschutzbestimmungen.
\stopPictPar
\blank[small]

\startPictPar
\PMwarranty
\PictPar
\textDescrHead{Fachgerecht reinigen!}
Für Schäden, die durch Nichtbeachten der Reinigungsvorschriften entstehen,
können der \BosFull{boschung} gegenüber keinerlei Haftungs- oder
Garantieansprüche geltend gemacht werden.
\stopPictPar
\stoptextbackground


\subsection{Hochdruckreinigung}

Zum Hochdruck||Reinigen\index{Reinigung+Hochdruck} des Fahrzeugs eignet
sich ein handelsübliches Hochdruckreinigungsgerät.

Bei der Hochdruckreinigung sind folgende Punkte zu beachten:

\startitemize
	\item Arbeitsdruck maximal 50\,bar
	\item Flachstrahldüse mit einem Spritzwinkel von 25°
	\item Spritzabstand mindestens 80\,cm
	\item Wassertemperatur maximal 40\,°C
	\item Beachten Sie den Abschnitt \about[reiMi], \atpage[reiMi].
\stopitemize

Bei Nichtbeachtung dieser\index{Lack+Schäden} Vorschriften kann es zu Schäden
an Lack und Korrosionsschutz\index{Schäden+Lack} kommen.

Beachten Sie auch Bedienungsanleitung und Sicherheitsvorschriften des
Hochdruckreinigungsgeräts.

\starttextbackground[FC]
\startPictPar\PPspray\PictPar
Beim Hochdruckreinigen kann Wasser an Stellen eindringen, wo es Schaden
verursachen kann. Richten Sie deshalb den Wasserstrahl nie auf empfindliche
Teile und Gerätschaften:
\stopPictPar

\startitemize
	\item Sensoren, elektrische Verbindungen und Anschlüsse
	\item Anlasser, Lichtmaschine, Einspritzanlage
	\item Magnetventile
	\item Lüftungsöffnungen
	\item Noch nicht abgekühlte mechanische Komponenten
	\item Hinweis-, Warn|| und Gefahrenaufkleber
	\item Elektronische Steuergeräte
\stopitemize

\textDescrHead{Motorwäsche}
Wassereintritt in Ansaug-, Be- und Entlüftungsöffnungen unbedingt
vermeiden. Bei Hochdruckreinigern den Strahl nicht direkt auf elektrische
Bauteile und Leitungen richten. Den Strahl nicht auf die Einspritzanlage
richten! Motor nach der Motorwäsche konservieren; dabei den Riemen vor dem
Konservierungsprodukt schützen.
\stoptextbackground

\starttextbackground [FC]
\setupparagraphs [PictPar][1][width=6em,inner=\hfill]
\startPictPar
\framed[frame=off,offset=none]{\PMproteyes\PMprotears}
\PictPar
\textDescrHead{Restwasser}
	Während der Reinigung sammelt sich an bestimmten Stellen des Fahrzeugs
	Wasser (\eG\ in den Kuhlen des Motorblocks oder des Getriebes); dieses ist
	mithilfe von Druckluft zu entfernen. Beachten Sie, dass beim Umgang mit
	Druckluft entsprechende Schutzausrüstung zu tragen ist, und die Anlage den
	geltenden Sicherheitsvorschriften entsprechen muss (Multidüse).
\stopPictPar
\stoptextbackground


\subsubsection[reiMi]{Geeignete Reinigungmittel}

Verwenden\index{Reinigungsmittel} Sie ausschließlich Reinigungmittel, die
folgende Eigenschaften aufweisen:

\startitemize
	\item Keine Scheuerwirkung
	\item PH-Wert von 6–7
	\item Lösungsmittelfrei
\stopitemize

Zur Beseitigung von hartnäckigen Flecken verwenden Sie auf kleinen Lackflächen
mit Bedacht Waschbenzin oder Spiritus, keinesfalls andere Lösungsmittel.
Entfernen Sie Lösungsmittelrückstände vom Lack. Das Reinigen von
Kunststoffteilen mit Benzin kann zu Rissen oder Verfärbungen führen!

Reinigen Sie Flächen mit\index{Reinigung+Aufkleber} Warn- oder
Hinweisaufklebern mit klarem Wasser unter Verwendung eines weichen Schwamms.

Vermeiden Sie Wassereintritt in elektrische Komponenten: Verwenden Sie keine
Autobürste zur Reinigung der Blinker- und Leuchtengehäuse, sondern einen
weichen Lappen oder Schwamm.

\starttextbackground [CB]
\startPictPar
\GHSgeneric\par
\GHSfire
\PictPar
\textDescrHead{Gefahr durch Chemikalien}
Von Reinigungsmitteln können Gesundheits- und Sicherheitsrisiken (leicht
entflammbare Stoffe) ausgehen. Beachten Sie die für die verwendeten
Reinigungsmittel geltenden Sicherheitsvorschriften; beachten Sie die Gefahren-
und Datenblätter der verwendeten Mittel.
\stopPictPar
\stoptextbackground

\stopregister[index][vhc:lavage]


\page [yes]


\setups [pagestyle:bigmargin]

\startsection	[title={Einstellen des Saugmunds},
				 reference={sec:main:suctionMouth}]


Der optimale Abstand\index{Saugmund+Einstellen} zwischen Straßenoberfläche und Kunststoffschiene des Saugmunds beträgt 8\,mm.
Um den Abstand zu kontrollieren bzw. einzustellen, benutzen Sie die drei Einstelllehren, die Sie im Werkzeugkasten (Führerhaus, Fahrerseite) finden.


\placefig [margin] [fig:suctionMouth] {Einstellen des Saugmunds}
{\Framed{\externalfigure [suctionMouth:adjust]}}

\placeNote[][service_picto]{}{%
\noteF
\starttextrule{\PHasphyxie\enskip Vergiftungs- und Erstickungsgefahr \enskip}
{\md Hinweis:} Während der Einstellarbeiten muss der Fahrzeugmotor laufen, um den Saugmund in Schwimmstellung halten zu können. Um die Gefahr einer Vergiftung oder Erstickung auszuschließen, muss deshalb unbedingt eine Abgasabsauganlage verwendet werden, bzw. die Arbeiten dürfen ausschließlich an einem sehr gut belüfteten Ort durchgeführt werden.
\stoptextrule}

\startSteps
\item Stellen Sie das Fahrzeug an einem gut belüfteten Ort auf einer waagerechten und ebenen Fläche ab.
\item Aktivieren Sie\index{Absaugung} den \aW{Arbeits}modus (Knopf außen am Fahrstufenwahlhebel drücken).

Lassen Sie den Motor mit Leerlaufdrehzahl laufen. (Drücken Sie die Taste~\textSymb{joy_key_engine_decrease} auf der Multifunktionskonsole, um die Motordrehzahl zu verringern.)
\item Ziehen Sie die Feststellbremse an und sichern Sie die Hinterräder mit jeweils einem Keil.
\item Drücken Sie die Taste~\textSymb{joy_key_suction}, um den Saugmund zu senken.
\item Platzieren Sie die drei Einstelllehren~\LAa\ unter der Kunststoffschiene des Saugmunds, so wie in der Abbildung dargestellt.
\item [sucMouth:adjust]Lösen Sie die Befestigungs-~\Lone\ und Einstellschrauben~\Ltwo\ eines jeden Rads; die vier Räder senken sich auf den Boden ab.
\item Ziehen Sie die Schrauben~\Lone\ und~\Ltwo\ wieder an, und entfernen Sie dann die drei Einstelllehren.
\item Heben|/|senken Sie den Saugmund und überprüfen Sie die Einstellung mit den Einstelllehren. Falls die Einstellung noch nicht ganz stimmt, wiederholen Sie die Einstellprozedur ab Punkt~\in[sucMouth:adjust].

\stopSteps


\stopsection
\stopchapter
\stopcomponent

