\startcomponent c_80_maintenance_s2_120-sv
\product prd_ba_s2_120-sv

\startchapter [title={Underhåll och reparation},
reference={chap:maintenance}]

\setups[pagestyle:marginless]


\startsection [title={Allmän information}]


\subsection{Miljöskydd}

\starttextbackground [FC]
\setupparagraphs [PictPar][1][width=2.45em,inner=\hfill]

\startPictPar
\Penvironment
\PictPar
\Boschung\ omsätter miljöskydd\index{Miljöskydd} till praktisk handling. Vi fokuserar på orsakerna och tar hänsyn till alla konsekvenser som produktionsprocessen eller produkten har på miljön i företagets beslut.
Vi strävar efter en omsorgsfull hantering av resurser och den naturliga livsmiljön, som är en viktig förutsättning för både människa och natur.
Genom att följa vissa regler när du arbetar med fordonet kan även du bidra till att skydda miljön. Till dessa hör bland annat lämplig och korrekt hantering av medel och material i samband med underhållsarbeten på fordonet (t.ex. hantering av kemikalier som farligt avfall).

Motorns bränsleförbrukning och slitage beror på driftsvillkoren. Beakta därför följande punkter:

\startitemize
\item Låt inte motorn varmköras i tomgång.
\item Slå från motorn vid väntetider under arbetet.
\item Kontrollera bränsleförbrukningen regelbundet.
\item {\em Underhållsarbetena ska genomföras i enlighet med underhållsschemat av en behörig fackverkstad.}
\stopitemize
\stopSymList
\stoptextbackground

\page [yes]


\subsection{Säkerhetsanvisningar}

\startSymList
\PHgeneric
\SymList
För att\index{Underhåll+Säkerhetsanvisningar} undvika skador på fordonet och dess aggregat under underhållsarbeten är det viktigt att du beaktar följande säkerhetsanvisningar. Beakta även de allmänna säkerhetsanvisningarna (\about[safety:risques], \at{fr.o.m.
sidan}[safety:risques]).
\stopSymList

\starttextbackground [FC]
\startPictPar
\PMgeneric
\PictPar
\textDescrHead{Olycksförebyggande}
Kontrollera\index{Olycksförebyggande} fordonets skick efter varje underhålls- och reparationsarbete. Se framför allt till att alla säkerhetsrelevanta komponenter samt belysnings- och signalanordningar fungerar som de ska innan fordonet framförs på allmänna vägar.
\stopPictPar
\stoptextbackground
\blank [big]

\start
\setupparagraphs [SymList][1][width=6em,inner=\hfill]
\startSymList\PHcrushing\PHfalling\SymList
\textDescrHead{Stabilisering av fordonet}
Före alla typer av underhållsarbeten ska fordonet säkras så att det inte kan sättas i rörelse av misstag. Ställ växelspaken \aW{i neutralläge}, dra åt handbromsen och säkra fordonet med hjulkilar.
\stopSymList
\stop

\starttextbackground[CB]
\startPictPar\PHpoison\PictPar
\textDescrHead{Starta motorn}
Om\index{Fara+Förgiftning} du måste starta motorn på ett plats med dålig ventilation ska du inte låta den vara igång längre än nödvändigt\index{Fara+avgaser} för att undvika risken för kolmonoxidförgiftning.
\stopPictPar
\startitemize
\item Starta motorn endast när batteriet är korrekt anslutet.
\item Koppla aldrig från batteriet när motorn är igång.
\item Använd inte någon typ av starthjälp för att starta fordonet.
Om\index{Batteri+laddare} batteriet ska laddas med en snabbladdare måste det först kopplas från fordonet. Beakta snabbladdarens bruksanvisning.
\stopitemize
\stoptextbackground

\page [bigpreference]


\subsubsection{Skydd av elkomponenter}

\startitemize
\item Innan\index{Elsvetsning} du genomför elsvetsarbeten ska du koppla batterikablarna från batteriet och ansluta plus- och jordkabeln till varandra.
\item Elektriska\index{Elektronik} styrenheter får endast anslutas eller kopplas från om de inte är spänningssatta.
\item Felaktig\index{Styrenhet} polaritet i strömförsörjningen (\eG\
på grund av felaktigt anslutna batterier) kan leda till att elektroniska komponenter och enheter förstörs.
\item Om\index{Omgivningstemperaturer +extrema} omgivningstemperaturen överstiger 80 °C (t.ex.i en torkkammare) ska elektroniska komponenter/enheter tas bort.
\stopitemize


\subsubsection{Diagnos och mätningar}

\startitemize
\item För mät- och diagnosarbeten ska du endast använda {\em lämpliga}
provkablar (t.ex.enhetens originalkablar).
\item Mobiltelefoner\index{Mobiltelefon}och andra liknande trådlösa enheter kan orsaka störningar i fordonets och diagnosenhetens funktioner och därmed påverka driftsäkerheten negativt.
\stopitemize


\subsubsection{Personalens kvalifikation}

\starttextbackground[CB]
\startPictPar
\PHgeneric
\PictPar
\textDescrHead{Olycksrisk}
Om\index{Kvalifikation+Underhållspersonal} underhållsarbeten genomförs på ett felaktigt sätt kan fordonets funktionsduglighet och säkerhet sättas på spel. Det i sin tur leder till ökad risk för olyckor och skador.

Underhålls- och reparationsarbeten\index{Kvalifikation+Verkstad} ska genomföras av en auktoriserad fackverkstad som har lämpliga kunskaper och verktyg för denna typ av arbeten.

Om du är tveksam, kontakta \Boschung-Kundendienst.
\stopPictPar
\stoptextbackground

 \ProductId får endast användas, underhållas och repareras av kvalificerad personal som utbildats av \Boschung-kundtjänst.

De behörigheter som krävs för användning, underhåll och reparation tilldelas av \Boschung-kundtjänst.


\subsubsection{Ändringar och ombyggnader}

\starttextbackground[CB]
\startPictPar
\PHgeneric
\PictPar
\textDescrHead{Olycksrisk}
Alla\index{Ändringar på fordonet} ändringar som genomförs på fordonet utan tillverkarens godkännande kan påverka funktions- och driftssäkerheten på \ProductId negativt. Detta kan leda till en ej förutsägbar risk för olyckor och skador.
\stopPictPar

\startPictPar
\PMwarranty
\PictPar
För skador som uppstår\index{Garanti+Villkor} till följd av egenmäktiga ändringar eller modifikationer på\ProductId eller ett aggregat erbjuder \Boschung\ ingen garanti eller goodwill.
\stopPictPar
\stoptextbackground

\stopsection


\startsection [title={Driv- och smörjmedel}, reference={sec:liquids}]


\subsection{Korrekt hantering}

\starttextbackground[CB]
\startPictPar
\PHpoison
\PictPar
\textDescrHead{Skade- och förgiftningsrisk}
Hudkontakt\index{Bränsle}\index{Smörjmedel}
eller\index{Fara+Förgiftning} förtäring av drivmedel\index{Bränsle+Säkerhet} kan leda till allvarliga skador eller förgiftning. Beakta alltid gällande bestämmelser för hantering, lagring och bortskaffande av dessa ämnen.
\stopPictPar
\stoptextbackground

\starttextbackground [FC]
\startPictPar
\PMproteyes\par
\PMprothands
\PictPar
Använd alltid lämplig personlig skyddsutrustning och andningsskydd vid hantering av driv- och smörjmedel. Undvik att andas in ångor.
Undvik all kontakt med hud, ögon och kläder. Alla hudpartier som har kommit i kontakt med drivmedel ska omedelbart sköljas med vatten. Om du får drivmedel i ögonen ska du spola dem med rikligt med rent vatten och vid behov kontakta en ögonläkare. Kontakta omedelbart läkare vid förtäring av drivmedel!
\stopPictPar
\stoptextbackground

\startSymList
\PPchildren
\SymList
Drivmedel ska förvaras oåtkomligt för barn.
\stopSymList

\startSymList
\PPfire
\SymList
\textDescrHead{Brandfara}
Drivmedel\index{Fara+Brand} är mycket lättantändliga, vilket ökar brandrisken när man hanterar dessa ämnen. Brand, rökning\index{Rökning förbjuden} och öppen låga är strängt förbjudet vid hantering av drivmedel.
\stopSymList

\starttextbackground [FC]
\startPictPar
\PMgeneric
\PictPar
Använd endast smörjmedel som lämpar sig för de komponenter som används i \ProductId. Använd därför endast produkter som kontrollerats och godkänts av \Boschung\. Du hittar dem i listan över drivmedel \atpage[sec:liqquantities]. Tillsatser\index{Tillsatser} för smörjmedel krävs ej. Om du tillsätter tillsatser kan det leda till att vår garanti upphör att gälla\index{Garanti+villkor}.
För mer information, kontakta \Boschung-kundtjänsten.
\stopPictPar
\stoptextbackground

\starttextbackground [FC]
\startPictPar
\Penvironment
\PictPar
\textDescrHead{Miljöskydd}
Smörj-\index{Smörjmedel+Bortskaffande} och drivmedel\crlf eller\index{Miljöskydd} föremål som förorenats med dessa medel (\eG\ filter, trasor) ska\index{Drivmedel+Bortskaffande} bortskaffas i enlighet med gällande miljöregler.
\stopPictPar
\stoptextbackground

\page [yes]

\setups [pagestyle:normal]


\subsection[sec:liqquantities]{Specifikationer och fyllmängder}

Samtliga\index{Drivmedel+Fyllmängd}\index{Smörjmedel+Fyllmängd}\index{Fyllmängder+Drivmedel
och smörjmedel}\index{Specifikationer+Driv- och smörjmedel} fyllmängder i denna tabell är riktvärden. Varje gång du fyller på med driv- eller smörjmedel ska du kontrollera den faktiska fyllnivån och vid behov öka eller minska mängden.
% \blank[big]

\placetable[margin][tab:glyco]{Frostskydd (\index{Frostskydd}motor)}
{\noteF\startframedcontent[FrTabulate]
%\starttabulate[|Bp(80pt)|r|r|]
\starttabulate[|Bp|r|r|]
\NC Frostkydd ned till {[}°C{]}\NC \bf \textminus 25 \NC \bf \textminus 40 \NC\NR
\NC Destill. vatten [Vol.-\%] \NC 60 \NC 40 \NC\NR
\NC Frostskyddmedel \break [Vol.-\%] \NC 40 \NC {\em max.} 60 \NC\NR
\stoptabulate\stopframedcontent\endgraf
Achtung: Vid en volymandel på mer än 60\hairspace\percent\
Frostschutzmittel {\em sjunker} frostskyddet och kyleffekten försämras!}

\placefig[margin][fig:hydrgauge]{\select{caption}{Nivåindikering
Hydraulvätska (vänster fordonssida)}{Nivåindikering
hydraulvätska}}
{\externalfigure[main:hy:level_temp]
\noteF Nivån i hydraultanken kan avläsas på synglaset och ska kontrolleras {\em varje} dag.}

\vskip -8pt
\start
\define [1] \TableSmallSymb {\externalfigure[#1][height=4ex]}
\define\UC\emptY
\pagereference[page:table:liquids]

\setupTABLE [frame=off,style={\ssx\setupinterlinespace[line=.86\lH]},background=color,
option=stretch,
split=repeat]
\setupTABLE [r] [each] [topframe=on,
framecolor=TableWhite,
% rulethickness=.8pt
]

\setupTABLE [c] [odd] [backgroundcolor=TableMiddle]
\setupTABLE [c] [even] [backgroundcolor=TableLight]
\setupTABLE [c] [1][width=30mm]
\setupTABLE [c] [2][width=20mm]
\setupTABLE [c] [4][width=25mm]
\setupTABLE [c] [last] [width=10mm]
\setupTABLE [r] [first] [topframe=off,style={\bfx\setupinterlinespace[line=.95\lH]},
% backgroundcolor=TableDark
]
\setupTABLE [r] [2][framecolor=black]

\bTABLE

\bTABLEhead
\bTR
\bTC Grupp \eTC
\bTC Kategori \eTC
\bTC Klassning \eTC
\bTC Produkt\note[Produkt] \eTC
\bTC Mängd \eTC
\eTR
\eTABLEhead

\bTABLEbody
\bTR \bTD Dieselmotor \eTD
\bTD Motorolja\eTD
\bTD \liqC{SAE 5W-30}; \liqC{VW 507.00}\eTD
\bTD Total Quartz INEO Long Life \eTD
\bTD 4,3 l\eTD
\eTR
\bTR \bTD Hydraulkrets \eTD
\bTD ATF-olja \eTD
\bTD \liqC{dexron iii} \eTD
\bTD Total Equiviz ZS 46 (tank ca 40 l) \eTD
\bTD ca 50 l\eTD
\eTR
\bTR \bTD Hydraulkrets (tillval \aW{Bio})\eTD
\bTD ATF-olja \eTD
\bTD \liqC{dexron iii} \eTD
\bTD Total Biohydran TMP SE 46\eTD
\bTD ca 50 l\eTD
\eTR
\bTR \bTD Magnetventiler: Spolkärna \eTD
\bTD Smörjmedel\eTD
\bTD Kopparfett \eTD
\bTD \emptY\eTD
\bTD E. b.\note[Bedarf] \eTD
\eTR
\bTR \bTD Diverse: lås, dörrmekanik, bromspedal \eTD
\bTD Smörjmedel\eTD
\bTD Universalspray\eTD
\bTD \emptY\eTD
\bTD E. b.\note[Bedarf] \eTD
\eTR
\bTR \bTD Centralsmörjsystem \eTD
\bTD Universallagerfett\eTD
\bTD \liqC{nlgi} 2 \eTD
\bTD Total Multis EP 2\eTD
\bTD E. b.\note[Bedarf] \eTD
\eTR
\bTR \bTD Kylsystem \eTD
\bTD Frost-/rostskyddsmedel\eTD
\bTD TL VW 774 F/G; max 60\hairspace\% vol.\eTD
\bTD G12+/G12++ (rosa/violett)\eTD
\bTD ca 14 l \eTD
\eTR
\bTR \bTD Högtrycksvattenpump \eTD
\bTD Motorolja\eTD
\bTD \liqC{SAE 10W-40}; \liqC{api cf – acea e6}\eTD
\bTD Total Rubia TIR 8900 \eTD
\bTD 0,29 l \eTD
\eTR
\bTR \bTD Klimatanläggning \eTD
\bTD Kylmedel\eTD
\bTD + 20 ml POE-Öl\eTD
\bTD R 134a\eTD
\bTD 700 g\eTD
\eTR
\bTR \bTD Vindrutespolare \eTD
\bTD [nc=2] Vatten och spolarvätskekoncentrat, \aW{S} Sommar, \aW{W} Vinter; Beakta blandningsförhållandet\eTD
\bTD Detaljhandel \eTD
\bTD E. b.\note[Bedarf] \eTD
\eTR
\eTABLEbody

\eTABLE
\stop

\footnotetext[Bedarf]{{\it E. b.} Efter behov, i enlighet med respektive anvisning}
\footnotetext[Produkt]{Produkter som används av\Boschung\. Andra produkter som uppfyller specifikationerna kan också användas.}

\stopsection

\page [yes]

\setups [pagestyle:marginless]


\startsection [title={Underhåll av dieselmotorn},
reference={sec:workshop:vw}]


\subsection [sSec:vw:diagTool]{Fordonsdiagnossystem}

\startregister[index][reg:main:vw]{Underhåll+Dieselmotor} Motorns styrenhet (J623) är utrustad med ett felminne.
Om det uppstår fel i de övervakade sensorerna resp. komponenterna sparas felen i felminnet tillsammans med information om feltypen.

Efter utvärdering av informationen\index{Dieselmotor+Diagnos} skiljer motorstyrenheten mellan de olika felklasserna och sparar dem fram tills det att felminnets innehåll raderas.

Fel som endast förekommer {\em sporadiskt} visas med tillägget \aW{SP}. Orsaken till de sporadiska felen kan vara en \eG\ glappkontakt eller ett tillfälligt kabelbrott. Om ett sporadiskt fel inte längre förekommer inom 50 motorstarter, raderas felet ur felminnet.

Om systemet identifierar fel som påverkar motorns driftsegenskaper lyser kontrollsymbolen \aW{motordiagnos} \textSymb{vpadWarningEngine1} på bildskärmen till Vpad.

Sparade fel kan avläsas med fordonsdiagnos-, mät- och informationssystemet \aW{VAS 5051/B}.

När felen har åtgärdats måste felminnet raderas.


\subsubsection[sSec:vw:diagTool:connect]{Starta diagnossystemet}

\starttextbackground [FC]
\startPictPar
\PMgeneric
\PictPar
Närmare information om fordonsdiagnossystemet VAS 5051/B hittar du i systemets bruksanvisning.

Du kan även använda andra kompatibla diagnossystem \eG\ \aW{DiagRA}.
\stopPictPar
\stoptextbackground

\page [yes]


\subsubsubsubject{Förutsättningar}

\startitemize
\item Säkringarna måste vara felfria.
\item Batterispänningen måste vara mer än 11,5 V.
\item Alla elektriska förbrukare måste vara frånslagna.
\item Jordanslutningen måste vara felfri.
\stopitemize


\subsubsubsubject{Tillvägagångssätt}

\startSteps
\item Anslut kontakten på diagnosledningen VAS 5051B/1 till diagnosuttaget.
\item Beroende på funktion ska du slå på tändningen eller starta motorn.
\stopSteps

\subsubsubsubject{Välja driftläge}

\startSteps [continue]
\item Tryck på knappen \aW{fordonssjälvdiagnos}på displayen.
\stopSteps


\subsubsubsubject{Välja fordonssystem}

\startSteps [continue]
\item Tryck på knappen \aW{01-Motorelektronik}på displayen.
\stopSteps

På displayen visas nu styrenhetens ID och motorstyrenhetens kodning.

Om koderna inte stämmer överens måste styrenhetens kod kontrolleras.


\subsubsubsubject{Välja diagnosfunktion}

På displayen visas alla tillgängliga diagnosfunktioner.

\startSteps [continue]
\item Tryck på knappen för önskad funktion på displayen.
\stopSteps



\subsection [sSec:vw:faultMemory]{Felminne}


\subsubsection{Avläsa felminnet}

\subsubsubject{Arbetsprocess}

\startSteps
\item Låt motorn arbeta på tomgång.
\item Anslut VAS 5051/B (se \in{avsnitt}[sSec:vw:diagTool:connect])
och välj motorstyrenheten.
\item Välj diagnosfunktionen \aW{004-Felminnesinnehåll}.
\item Välj diagnosfunktionen \aW{004.01-Avfråga felminne}.
\stopSteps

{\sla Endast om motorn inte startar:}

\startitemize [2]
\item Slå till tändningen.
\item Om det inte finns några fel i motorstyrenheten visas \aW{0 Fel identifierade} på displayen.
\item Om det finns fel i motorstyrenheten visas felen på rad på displayen.
\item Avsluta diagnosfunktionen.
\item Slå från tändningen.
\item Åtgärda eventuella indikerade fel med hjälp av feltabellen (se servicedokumentationen) och radera därefter felminnet.
\stopitemize

\starttextbackground [FC]
\startPictPar
\PMrtfm
\PictPar
Omett fel inte går att radera ska du kontakta \boschung-Kundendienst.
\stopPictPar
\stoptextbackground


\subsubsubject{Statiska fel}

Om det finns ett eller flera statiska fel i minnet ska du kontakta Boschungs kundtjänst som kan hjälpa dig åtgärda dessa fel med hjälp av den \aW{guidade felsökningen}.


\subsubsubject{Sporadiska fel}

Om endast sporadiska fel eller anvisningar är sparade i felminnet och det inte går att fastställa några funktionsfel på det elektroniska fordonssystemet, kan felminnet raderas.

\startSteps [continue]
\item Tryck på knappen \aW{Fortsätt} \inframed[strut=local]{>} för att komma till kontrollschemat.
\item För att avsluta den guidade felsökningen ska du trycka på knappen \aW{Hoppa} och därefter\aW{Avsluta}.
\stopSteps

Alla felminnen kontrolleras ännu en gång.

I ett fönster bekräftas att alla sporadiska fel har raderats.
% Das Diagnoseprotokoll wird automatisch (online) verschickt.

Fordonssystemtestet är därmed avslutat.


\subsubsection[sSec:vw:faultMemory:errase]{Radering av felminnet}

\subsubsubject{Arbetsprocess}

{\sla Förutsättningar:}

\startitemize [2]
\item Alla fel och felorsaker måste vara åtgärdade.
\stopitemize

\page [yes]


{\sla Tillvägagångssätt:}

\starttextbackground [FC]
\startPictPar
\PMrtfm
\PictPar
När felen har åtgärdats måste felminnet kontrolleras igen och därefter raderas:
\stopPictPar
\stoptextbackground

\startSteps
\item Låt motorn arbeta på tomgång.
\item Anslut VAS 5051/B (se \in{avsnitt}[sSec:vw:diagTool:connect])
och välj motorstyrenheten.
\item Välj diagnosfunktionen \aW{004-Avfråga felminne}.
\item Välj diagnosfunktionen \aW{004.10-Radera felminne}.
\stopSteps

\starttextbackground [FC]
\startPictPar
\PMrtfm
\PictPar
Om felminnet inte kan raderas finns det fortfarande fel som måste åtgärdas.
\stopPictPar
\stoptextbackground

\startSteps [continue]
\item Avsluta diagnosfunktionen.
\item Slå från tändningen.
\stopSteps


\subsection [sSec:vw:lub] {Smörjning av dieselmotorn}

\subsubsection [ssSec:vw:oilLevel] {Kontrollera nivån på motorolja}

\starttextbackground [FC]
\startPictPar
\PMrtfm
\PictPar
Oljenivån\index{Motorolje+-nivå} får aldrig överskrida \aW{Max.}-markeringen. I annat fall finns risk för\index{Nivå+Motorolja} skador på katalysatorn.
\stopPictPar
\stoptextbackground

\startSteps
\item Stäng av motorn och vänta i minst 3 minuter så att oljan kan rinna tillbaka till oljetråget.
\item Dra ut mätstickan och torka av den. Skjut in stickan helt igen.
\item Dra ut stickan och kontrollera oljenivån:

\startfigtext[right][fig:vw:gauge]{Läsa av oljenivån}
{\externalfigure[VW_Oil_Gauge][width=50mm]}
\startitemize [A]
\item Maxnivå – ingen mer olja får fyllas på.
\item Tillräcklig nivå – olja {\em kan} fyllas på upp till markeringen \aW{A}.
\item Ej tillräcklig nivå – olja {\em måste} fyllas på upp till området \aW{B}.
\stopitemize
{\em Om oljenivån ligger över markeringen \aW{A} finns risk för katalysatorskador.}
\stopfigtext
\stopSteps


\subsubsection [ssSec:vw:oilDraining] {Byte av motorolja}

\starttextbackground [FC]
\startPictPar
\PMrtfm
\PictPar
Motoroljefiltret i S2 är monterat i stående läge. Det innebär att filtret måste bytas ut {\em innan} ny olja fylls på. När filterelementet tas ut öppnas en ventil och oljan i filterhuset rinner automatiskt till vevhuset.
\stopPictPar
\stoptextbackground

\startSteps
\item Placera en lämplig\index{Dieselmotor+Oljebyte} uppsamlingsbehållare under motorn.
\item Skruva ur oljepluggen\index{Motorolje+-byte} och låt oljan rinna ut.
\stopSteps

\starttextbackground [FC]
\startPictPar
\PMrtfm
\PictPar
Se till att uppsamlingsbehållaren får plats med hela mängden gammal olja.
Oljespecifikationer och fyllmängder finns i \in{avsnitt}[sec:liqquantities].

Oljepluggen har en fast monterad tätring. Oljepluggen måste därför alltid bytas ut.
\stopPictPar
\stoptextbackground

\startSteps [continue]
\item Skruva ur den nya oljepluggen med tätring (\TorqueR 30 Nm).
\item Fyll på motorolja med rätt specifikationer (se \in{avsnitt}[sec:liqquantities]).
\stopSteps


\subsubsection [ssSec:vw:oilFilter] {Byte av motoroljefilter}

\starttextbackground [FC]
\startPictPar
\PMrtfm
\PictPar
\startitemize [1]
\item Beakta\index{Dieselmotor+Oljefilter} anvisningarna för avfallshantering och återvinning.
\item Byt ut\index{Oljefilter+Dieselmotor} filtret{\em före} oljebyte (se\in{avsnitt}[ssSec:vw:oilDraining]).
\item Olja in det nya filtrets tätning lätt före montering.
\stopitemize
\stopPictPar
\stoptextbackground

\startfigtext[right][fig:vw:oilFilter]{Oljefilter}
{\externalfigure[VW_OilFilter_03][width=50mm]}
\startSteps
\item Skruva av\Lone\ filterhusets lock med en lämplig skruvmejsel.
\item Rengör lockets och filterhusets tätytor.
\item Byt ut filterelementet \Lthree.
\item Byt ut O-ringarna \Ltwo\ och\Lfour.
\item skruva tillbaka locket på filterhuset (\TorqueR 25 Nm).
\stopSteps



%\subsubsubject{Données techniques}
%
%
%\hangDescr{Couple de serrage du couvercle:} \TorqueR 25 Nm.
%
%\hangDescr{Huile moteur prescrite:} Selon tableau \atpage[sec:liqquantities].
%% NOTE: Redundant [tf]

\stopfigtext



\subsubsection [ssSec:vw:oilreplenish] {Fylla på motorolja}

\starttextbackground [FC]
\startPictPar
\PMrtfm
\PictPar
\startitemize [1]
\item Innan du\index{Motorolja} tar{\em av} locket ska du rengöra påfyllningsstutsen med en trasa.
\item Fyll\index{Dieselmotor+Fylla på olja} endast på med olja som uppfyller specifikationerna.
\item Fyll på oljan stegvis i små mängder.
\item För att undvika överfyllning ska du efter varje påfyllning vänta så att oljan i motorns oljetråg kan rinna till mätstickans markering (se\in{avsnitt}[ssSec:vw:oilLevel]).
\stopitemize
\stopPictPar
\stoptextbackground

\startfigtext[right][fig:vw:oilFilter]{Fylla på olja}
{\externalfigure[s2_bouchonRemplissage][width=50mm]}
\startSteps
\item Dra ut oljestickan ca 10 cm så att luft kan tränga ut under påfyllning.
\item Öppna påfyllningsöppningen.
\item Fyll på med olja under beaktande av anvisningarna ovan.
\item Stäng påfyllningsöppningen noggrant.
\item Starta motorn.
\item Kontrollera fyllnivån. (Se \in{avsnitt}[ssSec:vw:oilLevel].)
\stopSteps

\stopfigtext


\subsection [sSec:vw:fuel] {Bränslematningssystem}

\subsubsection [ssSec:vw:fuelFilter] {Byte av bränslefilter}

\starttextbackground [FC]
\startPictPar
\PMrtfm
\PictPar
\startitemize [1]
\item Beakta\index{Dieselmotor+Bränslefilter} gällande bestämmelser för avfallshantering och återvinning av specialavfall.
\item Ta inte bort bränsleledningarna från filtrets ovandel.
\item Utöva inget tryck på bränsleledningarnas fästpunkter. Det kan leda till skador på filtrets ovandel.
\stopitemize
\stopPictPar
\stoptextbackground

\startfigtext[right][fig:vw:oilFilter]{Bränslefilter}
{\externalfigure[s2_fuelFilter_location][width=50mm]}

{\sla Förberedelser:}

Bränslefiltrets\index{Bränslefilter} hus är monterat framför motorn, på den högra sidan av chassit.
Ta bort de båda fästskruvarna med hjälp av en hylsnyckel (10 mm) och en ringnyckel (10 mm).

\stopfigtext


\page [yes]

\setups [pagestyle:normal]

{\sla Tillvägagångssätt:}

\startLongsteps
\item Ta bort alla skruvar på filtrets ovandel. Avlägsna filtrets ovandel.
\stopLongsteps

\starttextbackground [FC]
\startPictPar
\PMrtfm
\PictPar
Lyft upp ovandelen. Om nödvändigt kan du sätta en vinkelskruvmejsel mot monteringsspåret (\in{\LAa, fig.}[fig:fuelfilter:detach]) och lirka upp ovandelen.
\stopPictPar
\stoptextbackground

\placefig [margin] [fig:fuelfilter:detach]{Ta ut bränslefiltret}
{\externalfigure[fuelfilter:detach]}

\placefig [margin] [fig:fuelfilter:explosion]{Bränslefilter}
{\externalfigure[fuelfilter:explosion]}

\startLongsteps [continue]
\item Dra ut filterelementet ur filtrets underdel.
\item Ta bort packningen (\in{\Ltwo, fig.}[fig:fuelfilter:explosion]) från filtrets ovandel.
\item Rengör filtrets ovan- och underdel noggrant.
\item Sätt i ett nytt filterelement i filtrets underdel.
\item Fukta en ny packning (\in{\Ltwo, fig.}[fig:fuelfilter:explosion]) med lite bränsle och sätt i den i ovandelen.
\item Sätt fast ovandelen på filtrets underdel och tryck ihop jämnt, så att ovandelen ligger tätt runtom.
\item Skruva ihop ovan- och underdel {\em handfast}. Dra sedan åt alla skruvar korsvis med föreskrivet åtdragningsmoment (\TorqueR 5 Nm).
\stopLongsteps

% \subsubsubject{Données techniques}
%
% \hangDescr{Couple de serrage des vis de fixation du couvercle:} \TorqueR 5 Nm.
%% NOTE: redundant [tf]

\startLongsteps [continue]
\item Slå till tändningen för att avlufta systemet. Starta motorn och låt den arbeta på tomgång 1–2 minuter.
\item Radera felminnet på det sätt som beskrivs i \atpage[sSec:vw:faultMemory:errase].
\stopLongsteps


\subsection [sSec:vw:cooling] {Kylsystem}

\starttextbackground [FC]
\startPictPar
\PMrtfm
\PictPar
\startitemize [1]
\item Använd\index{Dieselmotor+Kylning} endast kylvätskor med de föreskrivna specifikationerna (se tabell \atpage[sec:liqquantities]).
\item För\index{Kylvätska} att frost- och korrosionsskyddet ska kunna säkerställas får kylvätskan endast tunnas ut med destillerat vatten enligt tabellen nedan.
\item Fyll aldrig kylvätskekretsen med vatten eftersom det kan påverka frost- och korrosionsskyddet negativt.
\stopitemize
\stopPictPar
\stoptextbackground


\subsubsection [sSec:vw:coolingLevel] {Kylvätskenivå}

\placefig [margin] [fig:coolant:level] {Kylvätskenivå}
{\externalfigure[coolant:level]}


\placefig [margin] [fig:refractometer] {Refraktometer VW T 10007}
{\externalfigure[coolant:refractometer]}

\placefig [margin] [fig:antifreeze] {Kontroll av frostskyddsdensitet}
{\externalfigure[coolant:antifreeze]}


\startSteps
\item Lyft smutsbehållaren och sätt upp säkerhetsstöttan.
\item Kontrollera\index{Nivå+Kylvätska} nivån på kylvätska i expansionskärlet: Den måste ligga ovanför \aW{min}-markeringen.
\stopSteps

\start
\define [1] \TableSmallSymb {\externalfigure[#1][height=4ex]}
\define\UC\emptY
\pagereference[page:table:liquids]


\setupTABLE [frame=off,style={\ssx\setupinterlinespace[line=.86\lH]},background=color,
option=stretch,
split=repeat]
\setupTABLE [r] [each] [topframe=on,
framecolor=TableWhite,
% rulethickness=.8pt
]

\setupTABLE [c] [odd] [backgroundcolor=TableMiddle]
\setupTABLE [c] [even] [backgroundcolor=TableLight]
\setupTABLE [r] [first] [topframe=off,style={\bfx\setupinterlinespace[line=.95\lH]},
% backgroundcolor=TableDark
]
\setupTABLE [r] [2][framecolor=black]

\bTABLE

\bTABLEhead
\bTR
\bTC Frostskydd upp till… \eTC
\bTC Andel G12\hairspace ++\eTC
\bTC Vol. frostskyddsmedel\eTC
\bTC Vol. destillerat vatten \eTC
\eTR
\eTABLEhead

\bTABLEbody
\bTR \bTD \textminus 25 °C \eTD
\bTD 40\hairspace\% \eTD
\bTD 3,8 l \eTD
\bTD 4,2 l \eTD
\eTR
\bTR \bTD \textminus 35 °C \eTD
\bTD 50\hairspace\% \eTD
\bTD 4,0 l \eTD
\bTD 4,0 l \eTD
\eTR
\bTR \bTD \textminus 40 °C \eTD
\bTD 60\hairspace\% \eTD
\bTD 4,2 l \eTD
\bTD 3,8 l \eTD
\eTR
\eTABLEbody

\eTABLE
\stop

\adaptlayout [height=+20pt]
\subsubsection [sSec:vw:coolingFreeze] {Kylvätskenivå}

Kontrollera\index{Frostskyddsdensitet} frostskyddets densitet med hjälp av en lämplig refraktometer (se \in{fig.}[fig:refractometer]: VW T 10007).
Beakta skalan 1: G12\hairspace ++ (se \in{fig.}[fig:antifreeze]).

\page [yes]


\subsection [sSec:vw:airFilter] {Luftförsörjning}

Luftfiltret kan kommas åt genom den bakre underhållsluckan på höger fordonssida (se \in{fig.}[fig:airFilter]).

\placefig [margin] [fig:airFilter] {Motorns luftfilter}
{\externalfigure[vw:air:filter]
\noteF
\startLeg
\item Säkerhetsrem
\item Underdel på huset
\item Ventilationsöppning
\item Trycksensor
\stopLeg}


\subsubsubject{Driftsvillkor}

Gatsopningsfordon används ofta i omgivningar där det finns mycket damm. Därför är det nödvändigt att kontrollera och rengöra luftfiltret varje vecka. Se även\about[table:scheduleweekly], \atpage[table:scheduleweekly]. Om nödvändigt måste filtret bytas ut.


\subsubsubject{Självdiagnos}

Insugsledningen är utrustad med en trycksensor (\Lfour, \in{fig.}[fig:airFilter]) med hjälp av vilken laddförluster\footnote{Minskat luftflöde på grund av reducerad luftgenomsläpplighet på filtret.} kan fastställas genom filtret.
Om luftfiltret är tilltäppt lyser kontrollsymbolen \textSymb{vpadWarningFilter} på Vpad-bildskärmen och felmeddelandet\VpadEr{851} registreras.


\subsubsubject{Underhåll/byte}

\startSteps
\item Dra säkerhetsremmen \Lone nedåt (\in{fig.}[fig:airFilter]).
\item Vrid underdelen på huset \Ltwo medurs och ta av den.
\item Ta ut filterelementet och kontrollera det. Byt ut det om nödvändigt.
\item Rengör filtrets insida och montera ihop luftfiltret i omvänd ordning.
\stopSteps

\page [yes]


\subsection [sSec:vw:belt] {Ribbrem}

Ribbremmen\index{Dieselmotor+Ribbrem} överför rörelsen från vevaxelns svänghjul till generatorn och klimatanläggningens kompressor (extrautrustning).
Ett\index{Ribbrem} spännelement i det sista segmentet (mellan generator och vevaxel) håller remmen spänd.


\subsubsection [sSec:belt:change] {Byte av ribbrem}

\placefig [margin] [fig:belt:tool] {Fästdorn VW T 10060 A}
{\externalfigure[vw:belt:tool]}

\placefig [margin] [fig:belt:overview] {Spännelement}
{\externalfigure[vw:belt:overview]}

\placefig [margin] [fig:belt:tens] {Punkt för anbringande av fästdornet}
{\externalfigure[vw:belt:tens]}


\subsubsubject{Med kompressor för klimatanläggning}


{\sla Nödvändiga specialverktyg:}

Fästdorn\aW{VW T 10060 A} för att hålla spännelementet.

\startSteps
\item Markera ribbremmens arbetsriktning.
\item Vrid spännelementets arm medurs med hjälp av en böjd ringnyckel (\in {fig.}[fig:belt:overview]).
\item För hålen över varandra (se pilar, \in {fig.}[fig:belt:tens]) och säkra spännelementet med fästdornet.
\item Ta av ribbremmen.
\stopSteps

Ribbremmen monteras i omvänd ordning.

\starttextbackground [FC]
\startPictPar
\PMrtfm
\PictPar
\startitemize [1]
\item Beakta ribbremmens arbetsriktning.
\item Kontrollera att ribbremmen sitter korrekt i remskivorna.
\item Starta motorn och kontrollera remmens arbetsriktning.
\stopitemize
\stopPictPar
\stoptextbackground


\subsubsubject{Utan kompressor för klimatanläggning}

{\sla Nödvändigt material:}

Reparationssats bestående av reparationsanvisning, ribbrem och specialverktyg.\footnote{Se reservdelskatalog under \aW{Underhållsdelar}.}

\startSteps
\item Kapa ribbremmen.
\item Följ de övriga arbetsstegen i reparationsanvisningen.
\stopSteps

\starttextbackground [FC]
\startPictPar
\PMrtfm
\PictPar
\startitemize [1]
\item Kontrollera att ribbremmen sitter korrekt i remskivorna.
\item Starta motorn och kontrollera remmens arbetsriktning.
\stopitemize
\stopPictPar
\stoptextbackground


\subsubsection [sSec:belt:tens] {Byte av spännelementet}

{\sla Endast för utförande med kompressor för klimatanläggning}

\blank [medium]

\placefig [margin] [fig:belt:tens:change] {Byte av spännelementet}
{\externalfigure[vw:belt:tens:change]
\noteF
\startLeg
\item Spännelement
\item Låsskruv
\stopLeg

{\bf Åtdragningsmoment}

Låsskruv:

\TorqueR 20 Nm\:+ ½ varv (180°).}

\startSteps
\item Demontera ribbremmen enligt anvisningarna (se \atpage[sSec:belt:change]).
\item Demontera kringutrustningen (beroende på utrustning).
\item Skruva ur låsskruven (\in{\Ltwo, fig.}[fig:belt:tens:change]).
\stopSteps

Spännelementet monteras i omvänd ordning.

\starttextbackground [FC]
\startPictPar
\PMrtfm
\PictPar
\startitemize [1]
\item Använd alltid en ny låsskruv efter montering.
\item Åtdragningsmoment: Se \in{fig.}[fig:belt:tens:change].
\stopitemize
\stopPictPar
\stoptextbackground

\stopregister[index][reg:main:vw]

\stopsection

\page[yes]


\setups[pagestyle:marginless]


\startsection[title={Hydraulsystem},
reference={sec:hydraulic}]

\starttextbackground [FC]
% \startfiguretext[left,none]{}
% {\externalfigure[toni_melangeur][width=30mm]}

\startSymPar
\externalfigure[toni_melangeur][width=4em]
\SymPar
\textDescrHead{Återvinning av drivmedel}
Förbrukade driv- och smörjmedel får inte hamna i naturen eller förbrännas.

Förbrukade smörjmedel får inte slängas i avloppssystemet, naturen eller hushållssoporna.

Förbrukade smörjmedel får inte blandas med andra vätskor eftersom det kan leda till att det bildas giftiga ämnen eller ämnen som är svåra att bortskaffa.
\stopSymPar
\stoptextbackground
\blank [big]

% \starthangaround{\PMgeneric}
% \textDescrHead{Qualification du personnel}
% Toute intervention sur l’installation hydraulique de votre véhicule ne peut être réalisée que par une personne dument qualifiée, ou par un service reconnu par \boschung.
% \stophangaround
% \blank[big]

\startSymList
\PHgeneric
\SymList
\textDescrHead{Hygien} Hydraulsystemet är mycket känsligt mot smuts i oljan. Därför är det mycket viktigt att omgivningen är helt ren.
\stopSymList

\startSymList
\PHhot
\SymList
\textDescrHead{Risk för stänk}
Innan man genomför arbeten på hydraulsystemet \sdeux\ måste resttrycket i respektive hydraulkrets släppas ut. Stänk av varm olja kan orsaka brännskador.
\stopSymList

\startSymList
\PHhand
\SymList
\textDescrHead{Klämrisk}
Smutsbehållaren ska alltid sänkas eller säkras mekaniskt med hjälp av säkerhetsstöttan innan man genomför arbeten på hydraulsystemet på \sdeux.
\stopSymList

\startSymList
\PImano
\SymList
\textDescrHead{Tryckmätning}
För att mäta hydraultrycket ska du fästa en manometer på en av kretsens \aW{Minimess}-anslutningar. Se till att manometern har ett lämpligt mätintervall.
\stopSymList

\page [yes]

\setups[pagestyle:normal]

\subsection{Underhållsintervaller}

\start

\setupTABLE [frame=off,
style={\ssx\setupinterlinespace[line=.93\lH]},
background=color,
option=stretch,
split=repeat]
\setupTABLE [r] [each] [
topframe=on,
framecolor=white,
backgroundcolor=TableLight,
% rulethickness=.8pt,
]

% \setupTABLE [c] [odd] [backgroundcolor=TableMiddle]
% \setupTABLE [c] [even] [backgroundcolor=TableLight]
\setupTABLE [c] [1][ % width=30mm,
style={\bfx\setupinterlinespace[line=.93\lH]},
]
\setupTABLE [r] [first] [topframe=off,
style={\bfx\setupinterlinespace[line=.93\lH]},
backgroundcolor=TableMiddle,
]
% \setupTABLE [r] [2][style={\ssBfx\setupinterlinespace[line=.93\lH]}]


\bTABLE

\bTABLEhead
\bTR\bTD Underhållsarbete\eTD\bTD Intervall \eTD\eTR
\eTABLEhead

\bTABLEbody
\bTR\bTD Titta efter läckage \eTD\bTD Varje dag \eTD\eTR
\bTR\bTD Kontrollera nivån på hydrauloljan\eTD\bTD Varje dag \eTD\eTR
\bTR\bTD Kontrollera hydraulledningarnas/hydraulslangarnas skick, byt ut vid behov \eTD\bTD 600 h/12 månader \eTD\eTR
\bTR\bTD Byt ut returledningsfiltret för hydraulolja och insugsfiltret \eTD\bTD 600 h/12 månader\eTD\eTR
\bTR\bTD Smörj magnetventilernas spolkärna med kopparfett\eTD\bTD 600 h/12 månader\eTD\eTR
\bTR\bTD Byt ut hydrauloljan \eTD\bTD 1200 h/24 månader\eTD\eTR
\eTABLEbody
\eTABLE
\stop


\subsection[niveau_hydrau]{Nivå}

\placefig[margin][fig:hydraulic:level]{Hydraulvätskans nivå}
{\externalfigure[hydraulic:level]
\noteF
\startLeg
\item Optimal nivå
\stopLeg}

Nivån på hydrauloljan
kan kontrolleras\index{Nivå+Hydraulvätska}\index{Underhåll+Hydraulsystem}
genom ett transparent synglas.
Om nivån på hydrauloljan har sjunkit måste man först fastställa orsaken innan man fyller på med ny olja. Beakta de föreskrivna intervallerna för byte av hydraulvätska (tabell ovan) samt specifikationerna för vätskan \at{sida}[sec:liqquantities]).


\subsubsection{Fylla på hydraulvätska}

Fyll på med hydraulvätska tills det mellersta synglaset täcks helt.
Starta motorn och fyll på med mer vätska om nödvändigt tills den nödvändiga nivån har nåtts.


\subsection{Byta hydraulvätska}

Fyllmängder och nödvändiga specifikationer för hydraulvätskan hittar du i tabellen på\at{sidan}[sec:liqquantities].

\startSteps
\item Öppna påfyllningsöppningen på hydraultanken.
\item Töm tanken med hjälp av en oljesug eller ta bort pluggen.

Pluggen befinner sig på hydraultankens undersida, framför vänster bakhjul (\in{fig.}[fig:hydraulic:fluidDrain]).
\item Fyll på med hydraulvätska tills det mellersta synglaset täcks helt.
Starta motorn och fyll på med mer vätska om nödvändigt till den nödvändiga nivån har nåtts.
\stopSteps

\placefig[margin][fig:hydraulic:fluidDrain]{Plugg}
{\externalfigure[hydraulic:fluidDrain]}


\placefig[margin][fig:hydraulic:returnFilter]{Hydraulfilter}
{\externalfigure[hydraulic:returnFilter]}

\subsection[filtres:nettoyage]{Returlednings- och insugsfilter}

\startSteps
\item Lyft upp smutsbehållaren och sätt upp säkerhetsstöttan.
\item Ta av locket från filtret på hydraultanken (\in{fig.}[fig:hydraulic:returnFilter]).
\item Byt ut\index{Oljefilter+Hydraul-} filterelementet mot ett nytt.
\item Fukta en ny O-ring med lite hydraulvätska och sätt i den.
\item Skruva tillbaka locket med båda händerna (\TorqueR ca 20 Nm).
\stopSteps

\page [yes]


\subsection[sec:solenoid]{Smörjning av magnetventilerna}

\placefig[margin][graissage_bobine]{Smörjning av magnetventilerna}
{\externalfigure[graissage_bobine][M]
\noteF
\startLeg
\item Magnetventilens spole
\item Spolkärna
\stopLeg}

Fukt och saltavlagringar som hamnar i de elektromagnetiska spolarnas kärna kan leda till korrosion. Spolkärnorna ska smörjas med kopparfett en gång per år. Fettet måste vara korrosions-, vatten- och temperaturbeständigt upp till 50 °C:
\startSteps
\item Demontera magnetventilens spole (\in{\Lone, fig.}[graissage_bobine]).
\item Smörj kärnan (\in{\Ltwo, fig.}[graissage_bobine]) med det föreskrivna specialfettet och montera in spolen igen.
\stopSteps


\subsection{Byte av slangar}

Slangarnas gummihölje\index{Slangar+Bytesintervaller} och förstärkningsväven åldras naturligt. Slangarna i hydraulsystemet måste därför bytas ut i de föreskrivna intervallerna, även om de inte har {\em några} synliga skador.

Se till att slangarna flänsas fast korrekt på fordonet för att undvika förtida slitage på grund av friktion. De måste ha ett tillräckligt avstånd till andra komponenter så att friktions- och vibrationsskador undviks.

\stopsection

\page [yes]

\setups [pagestyle:bigmargin]


\startsection[title={Bromssystem},
reference={sec:brake}]

\placefig[margin][fig:brake:rear]{Trumbroms}
{\startcombination [1*2]
{\externalfigure[brake:wheelHub]}{\slx Bakhjulnav}
{\externalfigure[brake:drum]}{\slx Mekanism och bromsgarnitur}
\stopcombination}

Bromstrumman ska\Lfour\ demonteras vid varje normalt underhållsarbete, bromsmekanismen\Lseven\ ska rengöras och bromsgarnituren\Lfive, \Lsix\ kontrolleras visuellt (\in{fig.}[fig:brake:rear]).


\subsubject {Demontering}

\startSteps
\item Kör fordonet på en lämplig lyftplattform och lyft upp hjulen.
\item Ta av hjulen.
\stopSteps


{\sla Demontering av framhjulsbromsarna}

\startSteps [continue]
\item Demontera bromstrumman \Lfour.
\stopSteps

{\sla Demontering av bakhjulsbromsarna}

\startSteps [continue]
\item Ta av kåpan \Lone\ från navet.
\item Skruva loss skruven \Ltwo\ och avlägsna mellanstycket
\item Skruva av hjulnavets mutter \Lthree\ med en hylsnyckel.
\item Ta av hjulnavet med bromstrumman.
\stopSteps


\subsubject {Återmontering}

Montera tillbaka bromstrumman i omvänd ordningsföljd. Dra åt muttrarna på bakhjulens nav \Lthree\ med det föreskrivna åtdragningsmomentet 190 Nm.

\stopsection

\page [yes]

\setups [pagestyle:normal]


\startsection[title={Kontroll och underhåll av däck},
reference={sec:pneumatiques}]

Däcken måste\index{Däck+Underhåll} alltid befinna sig i ett felfritt skick
för att kunna uppfylla sina båda huvudfunktioner, nämligen att erbjuda bra väggrepp och optimala bromsegenskaper. För stor nedslitning och felaktigt däcktryck, särskilt för lågt tryck, är ofta orsak till olyckor med fordonet.


\subsection{Säkerhetsrelevanta punkter}

\subsubsection{Slitagekontroll}

Däckens slitage kan kontrolleras med hjälp av slitageindikeringarna i slitbanespåren (\in{fig.}[pneususure]).
Avvikelser på däcket och deras orsaker kan fastställas genom en visuell kontroll.

\placefig[margin][pneususure]{Slitagekontroll}
{\Framed{\externalfigure[pneusUsure][M]}}

\placefig[margin][pneusdomages]{Skadade däck}
{\Framed{\externalfigure[pneusDomages][M]}}

\startitemize
\item Slitage på slitbanans sidor: För lågt däcktryck.
\item Förstärkt slitage i mitten: För högt däcktryck.
\item Asymmetriskt slitage på däckens sidor: Framaxel (spår, hjulinställning) felaktigt inställd.
\item Sprickor på slitbanan: däck för gammalt; däckets gummi blir hårdare med tiden och spricker (\in{fig.}[pneusdomages]).
\stopitemize

\starttextbackground[CB]
\startPictPar
\PHgeneric
\PictPar
\textDescrHead{Risker till följd av nednötta däck}
Ett nednött däck uppfyller inte längre sin funktion, i synnerhet inte vad gäller att leda bort vatten och slam. Det leder till längre bromssträcka och försämrade köregenskaper. Om däcket är nednött glider det lättare, framför allt på våt vägbana. Risken för att däcket förlorar sin vidhäftningsförmåga ökar därmed.
\stopPictPar
\stoptextbackground


\subsubsection{Däcktryck}

Det föreskrivna däcktrycket finns angivet på däckens typskylt, framtill på konsolen på passagerarsidan (se \atpage [sec:plateWheel]).

Även\index{Däck+Däcktryck} om däcken är i gott skick tappar de med tiden en större eller mindre mängd luft (ju oftare fordonet körs desto högre är tryckförlusten). Därför ska däcktrycket kontrolleras varje månad medan däcken är kalla. Om du kontrollerar trycket medan däcken är varma ska du lägga till 0,3 bar till det föreskrivna trycket.

\start
\setupcombinations[M]
\placefig[margin][pneuspression]{Däcktryck}
{\Framed{\externalfigure[pneusPression][M]}
\noteF
\startLeg
\item Korrekt tryck
\item För högt tryck
\item För lågt tryck
\stopLeg
Det föreskrivna däcktrycket finns angivet på hjulens typskylt på passagerarsidan i förarhytten.}
\stop

\starttextbackground[CB]
\startPictPar
\PHgeneric
\PictPar
\textDescrHead{Risker vid för lågt däcktryck}
Om däcktrycket är för lågt kan däcket gå sönder. Däcket pressas ihop mer än normalt om det inte har pumpats upp ordentligt eller om fordonet är överbelastat. Detta värmer upp gummit och delar av slitbanan kan lossa när fordonet kör i en kurva.
\stopPictPar
\stoptextbackground

\stopsection

\page [yes]

\setups[pagestyle:marginless]


\startsection[title={Chassi},
reference={main:chassis}]

\subsection{Säkerhetsrelevanta fästelement för komponenter}

Varje gång underhåll genomförs ska man kontrollera att fästskruvar på säkerhetsrelevanta komponenter är ordentligt åtdragna med föreskrivet åtdragningsmoment. Detta gäller särskilt för ramstyrningssystemet och axlarna.

\blank [big]

\startfigtext [left] [fig:frontAxle:fixing] {Framaxel}
{\externalfigure [frontAxle:fixing]}
{\sla Fästelement på framaxeln}
\startLeg
\item Fästelement för fjäderblad: \TorqueR 150 Nm
\item Fästelement för dragenheter: \TorqueR 78 Nm
\stopLeg

{\sla Fästelement för bakaxlar}
\startLeg
\item Fästelement för fjäderblad: \TorqueR 150 Nm
\stopLeg

\stopfigtext

\start

\setupTABLE [frame=off,style={\ssx\setupinterlinespace[line=.93\lH]},background=color,
option=stretch,
split=repeat]

\setupTABLE [r] [each] [topframe=on,
framecolor=white,
% rulethickness=.8pt
]

\setupTABLE [c] [odd] [backgroundcolor=TableMiddle]
\setupTABLE [c] [even] [backgroundcolor=TableLight]
\setupTABLE [c] [1][style={\bfx\setupinterlinespace[line=.93\lH]}]
\setupTABLE [r] [first] [topframe=off,style={\bfx\setupinterlinespace[line=.93\lH]},
]
% \setupTABLE [r] [2][style={\bfx\setupinterlinespace[line=.93\lH]}]


\bTABLE

\bTABLEhead
\bTR [backgroundcolor=TableDark] \bTD [nc=3] Åtdragningsmoment\eTD\eTR
% \bTR\bTD Position \eTD\bTD Type de vis \eTD\bTD Couple \eTD\eTR
\eTABLEhead

\bTABLEbody
\bTR\bTD Drivmotorer vänster/höger\eTD\bTD M12\:×\:35 8,8 \eTD\bTD 78 Nm \eTD\eTR
%% NOTE @Andrew: das sind Hydraulikmotoren
\bTR\bTD Arbetspump \eTD\bTD M16\:×\:40 100 \eTD\bTD 330 Nm \eTD\eTR
\bTR\bTD Drivpump\eTD\bTD M12\:×\:40 100 \eTD\bTD 130 Nm \eTD\eTR
\bTR\bTD Fjäderblad fram/bak\eTD\bTD M16\:×\:90/160 8,8 \eTD\bTD 150 Nm \eTD\eTR
% \bTR\bTD Fixation du système oscillant \eTD\bTD M12\:×\:40 8,8 \eTD\bTD 78 Nm \eTD\eTR
\bTR\bTD Fästelement för smutsbehållare \eTD\bTD M10\:×\:30 Verbus Ripp 100 \eTD\bTD 80 Nm \eTD\eTR
\bTR\bTD Hjulmuttrar\eTD\bTD M14\:×\:1,5 \eTD\bTD 180 Nm \eTD\eTR
\bTR\bTD Fästelement för frontkvast \eTD\bTD M16\:×\:40 100 \eTD\bTD 180 Nm \eTD\eTR
\eTABLEbody
\eTABLE
\stop


\stopsection

\page [yes]


\startsection[title={Centralsmörjsystem},
reference={main:graissageCentral}]


\subsection{Beskrivning av styrmodulen}

\sdeux\ kan utrustas med\index{Centralsmörjsystem} ett centralsmörjsystem (tillval). Systemet försörjer varje smörjpunkt
på fordonet med smörjmedel med jämna mellanrum.

\startfigtext [left] [vogel_affichage] {Indikeringsmodul}
{\externalfigure[vogel_base2][W50]}
\blank
\startLeg
\item 7-ställig display: Värden och driftstatus
\item \LED: System i viloläge (standby)
\item \LED: Pump i drift
\item \LED: Styrning av systemet med cykelbrytare
\item \LED: Övervakning av systemet med tryckbrytare
\item \LED: Felmeddelande
\item Bläddringsknappar:
\startLeg [R]
\item Aktivera display
\item Visa värden
\item Ändra värden
\stopLeg
\item Knapp för att ändra driftläge; bekräfta värden
\item Aktivera en mellansmörjningscykel
\stopLeg
\stopfigtext

Centralsmörjsystemet omfattar smörjmedelspumpen, den genomskinliga smörjmedelsbehållaren på vänster sida av chassit och styrmodulen i centralelektroniken.
% \blank
\page [yes]


\subsubsubject{Visning av knappar på styrmodulen}

\start

\setupTABLE [frame=off,style={\ssx\setupinterlinespace[line=.93\lH]},background=color,
option=stretch,
split=repeat]

\setupTABLE [r] [each] [topframe=on,
framecolor=white,
% rulethickness=.8pt
]

\setupTABLE [c] [odd] [backgroundcolor=TableMiddle]
\setupTABLE [c] [even] [backgroundcolor=TableLight]
\setupTABLE [c] [1][width=9mm,style={\bfx\setupinterlinespace[line=.93\lH]}]
\setupTABLE [r] [first] [topframe=off,style={\bfx\setupinterlinespace[line=.93\lH]},
]
% \setupTABLE [r] [2][style={\bfx\setupinterlinespace[line=.93\lH]}]


\bTABLE
\bTABLEhead
% \bTR [backgroundcolor=TableDark]
% \bTD [nc=4] Anzeige und Tasten des Steuermoduls \eTD\eTR
\bTR\bTD Pos. \eTD
\bTD \LED \eTD\bTD Visningsläge \eTD
\bTD Programmeringsläge \eTD\eTR
\eTABLEhead

\bTABLEbody
\bTR\bTD 2 \eTD
\bTD Driftläge{\em Paus}\hskip.5em\null \eTD
\bTD Systemet befinner sig i standby\hskip.5em\null \eTD % -Betrieb
\bTD Paustiden kan ändras\eTD\eTR
\bTR\bTD 3 \eTD
\bTD Driftläge{\em Contact} \eTD
\bTD Pumpen arbetar \eTD
\bTD Arbetstiden kan ändras\eTD\eTR
\bTR\bTD 4 \eTD
\bTD Systemkontroll {\em CS} \eTD
\bTD Med den externa cykelbrytaren\eTD
\bTD Kontrolläget kan avaktiveras eller ändras \eTD\eTR
\bTR\bTD 5 \eTD
\bTD Systemkontroll {\em PS} \eTD
\bTD Med den externa tryckbrytaren\eTD
\bTD Kontrolläget kan avaktiveras eller ändras \eTD\eTR
\bTR\bTD 6 \eTD
\bTD Fel{\em Fault} \eTD
\bTD [nc=2] Det har uppstått ett funktionsfel. Orsaken visas i form av en felkod när man trycker på knappen \textSymb{vogel_DK}. Funktionerna avbryts ej. \eTD\eTR
\bTR\bTD 7 \eTD
\bTD Pilknappar\textSymb{vogelTop} \textSymb{vogelBottom} \eTD
\bTD [nc=2] \items[symbol=R]{Aktivering av displayen, kontroll av parametrar (visningsläge), inställning av det visade (I) värdet (programmeringsläge)}
\eTD\eTR
\bTR\bTD 8 \eTD
\bTD Knapp\textSymb{vogelSet} \eTD
\bTD [nc=2] Växla mellan visnings- och programmeringsläge eller bekräfta inmatade värden. \eTD\eTR
\bTR\bTD 9 \eTD
\bTD Knapp\textSymb{vogel_DK} \eTD
\bTD [nc=2] Om enheten befinner sig i läget{\em Paus} aktiveras mellansmörjningscykeln när man trycker på knappen. Felmeddelandena kvitteras och raderas. \eTD\eTR
\eTABLEbody
\eTABLE
\stop
\vfill

\startfigtext [left] [vogel_touches]{Indikeringsmodul}
{\externalfigure[vogel_base][width=50mm]}
\textDescrHead{Visningsläge} Tryck kort på en av pilknapparna \textSymb{vogelTop} \textSymb{vogelBottom} för att aktivera den 7-ställiga displayen \textSymb{led_huit}. Tryck på knappen \textSymb{vogelTop} igen för att visa de olika parametrarna följt av respektive värden. Läget {\em Indikering} kan kännas igen genom att lysdiodernalyser kontinuerligt (\in{2 till 6, fig.}[vogel_affichage]).
\blank [medium]
\textDescrHead{Programmeringsläge} För att ändra värdena ska du trycka minst 2 sekunder på knappen\textSymb{vogelSet}, för att växla till läget {\em programmering}: Lysdiodernablinkar. Tryck på knappen\textSymb{vogelSet} för att ändra\index{Centralsmörjsystem+Programmering} indikeringen, ändra sedan önskat värde med knappen \textSymb{vogelTop} \textSymb{vogelBottom}. Kvittera\index{Centralsmörjsystem+Indikering} med knappen\textSymb{vogelSet}.
\stopfigtext

\page [yes]


\subsection{Undermenyer i läget {\em Indikering}}

\vskip -9pt

\adaptlayout [height=+5mm]

\startcolumns[balance=no]\stdfontsemicn

\startSymVogel
\externalfigure[vogel_tpa][width=26mm]
\SymVogel
\textDescrHead{Paustid [h]} Tryck på knappen \textSymb{vogelTop}, för att visa de programmerade värdena.
\stopSymVogel

\startSymVogel
\externalfigure[vogel_068][width=26mm]
\SymVogel
\textDescrHead{Återstående paustid [h]} Återstående tid till nästa smörjcykel.
\stopSymVogel

\startSymVogel
\externalfigure[vogel_090][width=26mm]
\SymVogel
\textDescrHead{Total paustid [h]} Total paustid mellan två cykler.
\stopSymVogel

\startSymVogel
\externalfigure[vogel_tco][width=26mm]
\SymVogel
\textDescrHead{Smörjtid [min]} Tryck på \textSymb{vogelTop} för att visa de programmerade värdena.
\stopSymVogel

\startSymVogel
\externalfigure[vogel_tirets][width=26mm]
\SymVogel
\textDescrHead{Enhet i standby} Indikering ej möjligt, eftersom enheten befinner sig i standby (paus).
\stopSymVogel

\startSymVogel
\externalfigure[vogel_026][width=26mm]
\SymVogel
\textDescrHead{Smörjtid [min]} Tiden för en smörjomgång.
\stopSymVogel

\startSymVogel
\externalfigure[vogel_cop][width=26mm]
\SymVogel
\textDescrHead{Systemkontroll} Tryck på \textSymb{vogelTop} för att visa de programmerade värdena.
\stopSymVogel

\startSymVogel
\externalfigure[vogel_off][width=26mm]
\SymVogel
\textDescrHead{Kontrolläge} \hfill PS: Tryckbrytare;\crlf
CS: cykelbrytare; OFF: avaktivering.
\stopSymVogel

\startSymVogel
\externalfigure[vogel_0h][width=26mm]
\SymVogel
\textDescrHead{Driftstimmar} Tryck på \textSymb{vogelTop}, för att visa värdet i två steg.
\stopSymVogel

\startSymVogel
\externalfigure[vogel_005][width=26mm]
\SymVogel
\textDescrHead{Del 1: 005} Driftstiden visas i två delar; gå till del 2 med knappen \textSymb{vogelTop}.
\stopSymVogel

\startSymVogel
\externalfigure[vogel_338][width=26mm]
\SymVogel
\textDescrHead{Del 2: 33,8} Den andra delen av talet är 33,8; detta ger sammanlagt en driftstid på 533,8 h.
\stopSymVogel

\startSymVogel
\externalfigure[vogel_fh][width=26mm]
\SymVogel
\textDescrHead{Feltid} Tryck på \textSymb{vogelTop}, för att visa värdet i två steg.
\stopSymVogel

\startSymVogel
\externalfigure[vogel_000][width=26mm]
\SymVogel
\textDescrHead{Del 1: 000} Feltiden visas i två delar;\crlf
gå till del 2 med \textSymb{vogelTop}.
\stopSymVogel

\startSymVogel
\externalfigure[vogel_338][width=26mm]
\SymVogel
\textDescrHead{Del 2: 33,8} Den andra delen av talet är 33,8; detta ger sammanlagt en feltid på 33,8 h.
\stopSymVogel

\stopcolumns

\stopsection


\page [yes]


\setups [pagestyle:marginless]


\startsection[title={Smörjschema för manuell smörjning},
reference={sec:grasing:plan}]

\starttextbackground [FC]
\startPictPar
\PMgeneric
\PictPar
De smörjställen som anges i smörjschemat (\in{fig.}[fig:greasing:plan]) ska smörjas med jämna mellanrum. Regelbunden smörjning är en viktig förutsättning för att {\em minska slitaget} på ett varaktigt sätt och skydda mot fukt och andra korrosiva substanser.
\stopPictPar
\stoptextbackground

\blank [big]

\start

\setupcombinations [width=\textwidth]

\placefig[here][fig:greasing:plan]{Fordonets smörjschema}
{\startcombination [3*1]
{\externalfigure[frame:steering:greasing]}{\ssx Ramstyrning och pendelmekanism}
{\externalfigure[frame:axles:greasing]}{\ssx Axlar}
{\externalfigure[frame:sucMouth:greasing]}{\ssx Sugmunstycke}
\stopcombination}

\stop

\vfill

\startLeg [columns,three]
\item Ramstyrningens slagcylindrar\crlf {\sl 2 smörjnipplar per cylinder}
\item Ramstyrningens lager\crlf {\sl 2 smörjnipplar på vänster sida}
\columnbreak
\item Pendelmekanismens lager\crlf {\sl 1 smörjnippel framför tanken}
\item Bladfjädrar\crlf {\sl 2 smörjnipplar per fjäderblad}
\columnbreak
\item Sugmunstycke\crlf {\sl 1 smörjnippel per hjul}
\item Sugmunstycke\crlf {\sl 1 smörjnippel på dragarmen}
\stopLeg



\page [yes]


\setups [pagestyle:bigmargin]


\subsubject{Smörjning av smutsbehållaren}

Smutsbehållaren har 6 smörjpunkter (2\:×\:4), som ska smörjas varje vecka.

\blank [big]


\placefig[here][fig:greasing:container]{Behållarens lyftmekanism}
{\externalfigure[container:mechanisme]}


\placelegende [margin,none]{}
{{\sla Förklaringar:}

\startLeg
\item Behållarens vänstra lager (2\:×)
\item Behållarens högra lager (2\:×)
\item Vänster hydraulcylinder (upptill)
\item Vänster hydraulcylinder (nedtill)

{\em Som höger cylinder (punkt \i[greasing:point;hide]).}
\item Höger hydraulcylinder (upptill)
\item [greasing:point;hide]Höger hydraulcylinder (nedtill)
\stopLeg}

\stopsection

\page [yes]



\startsection[title={Elsystem},
reference={sec:main:electric}]

\subsection{Centralelektronik i chassit}

\startbuffer [fuses:preventive]
\starttextbackground [CB]
\startPictPar
\PHvoltage
\PictPar
\textDescrHead{Säkerhetsanvisningar}
Beakta säkerhetsanvisningarna i \index{Säkringar+Chassi}
den här\index{Relä+Chassi} anvisningen: Säkringarna får endast bytas ut mot säkringar med samma amperetal. Ta av dig metallsmycken innan du genomför arbeten på det \index{Elsystem} elektriska systemet (ringar, armband etc.).
\stopPictPar
\stoptextbackground
\stopbuffer


\subsubsubject{MIDI-säkringar}

\starttabulate[|l|r|p|]
\HL
\NC\md F 1 \NC 5 A \NC Bromsljus, \aW{+\:15} OBD \NC\NR
\NC\md F 2 \NC 5 A \NC \aW{+\:15} Motorstyrning \NC\NR
\NC\md F 3 \NC 7,5 A \NC \aW{+\:30} Motorstyrning och OBD \NC\NR
\NC\md F 4 \NC 20 A \NC Bränslepump \NC\NR
\NC\md F 5 \NC 20 A \NC \aW{D\:+} Generator, \aW{+\:15} Relä K 1 \NC\NR
\NC\md F 6 \NC 5 A \NC Motorstyrning \NC\NR
\NC\md F 7 \NC 10 A\NC Motoravgasbehandling \NC\NR
\NC\md F 8 \NC 20 A \NC Motorelektronik (styrning) \NC\NR
\NC\md F 9 \NC 15 A \NC Motoravgasbehandling, bränslepump, förglödning\NC\NR
\NC\md F 10\NC 30 A \NC Motorstyrning \NC\NR
\NC\md F 11\NC 5 A \NC Backljus\NC\NR
%% NOTE @Andrew: Singular
\HL
\stoptabulate

\placefig [margin] [fig:electric:power:rear] {Centralelektronik i chassit}
{\externalfigure [electric:power:rear]
\noteF
\startKleg
\sym{K 1} Elektronisk motorstyrenhet
\sym{K 2} Bränslepump
\sym{K 3} Aktivering av startmotorn
\sym{K 4} Bromsljus
\sym{K 5} {[}Reserv{]}
\sym{K 6} Backljus
\sym{K 7} Förglödningssystem
\stopKleg
}


\subsubsubject{MAXI-säkringar}

% \startcolumns [n=2]
\starttabulate[|l|r|p|]
\HL
\NC\md F 15 \NC 50 A \NC Huvudförsörjning för centralelektroniken \NC\NR
\HL
\stoptabulate

\page [yes]

\setups[pagestyle:marginless]


\subsection{Centralelektronik i förarhytten}

\startcolumns[rule=on]

\placefig [bottom] [fig:fuse:cab] {Säkringar och reläer i förarhytten}
{\externalfigure [electric:power:front]}

\columnbreak

\subsubsubject{Reläer}

\index{Säkringar+Förarhytt}\index{Relä+Förarhytt}

\starttabulate[|lB|p|]
\NC K 2\NC Kompressor för klimatsystem\NC\NR
\NC K 3\NC Kompressor för klimatsystem\NC\NR
\NC K 4\NC Elektrisk vattenpump\NC\NR
\NC K 5\NC Roterande varningsljus\NC\NR
\NC K 10 \NC Blinkfrekvensgivare\NC\NR
\NC K 11 \NC Halvljus\NC\NR
\NC K 12 \NC Helljus {[}Reserve{]} \NC\NR
\NC K 13 \NC Arbetsstrålkastare\NC\NR
\NC K 14 \NC Intervalltorkning\NC\NR
\stoptabulate

\vskip -24pt

\placefig [bottom] [fig:fuse:access] {Åtkomstlucka till centralelektronkiken}
{\externalfigure [electric:power:cabin]}

\stopcolumns

\page [yes]


\subsubsubject{MINI-säkringar}

\startcolumns[rule=on]
% \setuptabulate[frame=on]
%\placetable[here][tab:fuses:cab]{Fusibles dans la cabine}
%{\noteF
\starttabulate[|lB|r|p|]
\NC F 1 \NC 3 A \NC Positionsljus vänster\NC\NR
\NC F 2 \NC 3 A \NC Positionsljus höger\NC\NR
\NC F 3 \NC 7,5 A \NC Halvljus vänster\NC\NR
\NC F 4 \NC 7,5 A \NC Halvljus höger\NC\NR
\NC F 5 \NC 7,5 A \NC Helljus vänster{[}Reserve{]} \NC\NR
\NC F 6 \NC 7,5 A \NC Helljus höger{[}Reserve{]} \NC\NR
\NC F 7 \NC 10 A \NC Arbetsstrålkastare upptill \NC\NR
%% NOTE @Andrew: Plural
\NC F 8 \NC 10 A \NC Arbetsstrålkastare nedtill (reserv)\NC\NR
%% NOTE @Andrew: Plural
\NC F 9 \NC 10 A \NC Frontkvast \NC\NR
\NC F 10 \NC 10 A \NC Vindrutetorkare \NC\NR
\NC F 11 \NC 5 A \NC Kontakt belysning och varningsblinkers \NC\NR
\NC F 12 \NC 5 A \NC {[}Reserve{]} \NC\NR
\NC F 13 \NC 10 A \NC Uppvärmning av ytterspeglar\NC\NR
\NC F 14 \NC 7,5 A \NC \aW{+\:15} Radio och kamera \NC\NR
\NC F 15 \NC 10 A \NC \aW{+\:30} Varningsblinkers\NC\NR
\NC F 16 \NC 5 A \NC Belysning rattstång\NC\NR
\NC F 17 \NC 7,5 A \NC \aW{+\:30} Radio, ljusomkopplare och innerbelysning \NC\NR
\NC F 18 \NC — \NC {[}Frei{]} \NC\NR
\NC F 19 \NC 20 A \NC \aW{+\:30} RC 12 fram\NC\NR
\NC F 20 \NC 20 A \NC \aW{+\:30} RC 12 bak \NC\NR
\NC F 21 \NC 15 A \NC 12-V-uttag\NC\NR
\NC F 22 \NC 5 A \NC Tändningsnyckel, multifunktionskonsol, Vpad \NC\NR
\NC F 23 \NC 5 A \NC Nödstopp, mittkonsol, RC 12 fram \NC\NR
\NC F 24 \NC 5 A \NC Nödstopp, mittkonsol, RC 12 bak \NC\NR
\NC F 25 \NC 2 A \NC \aW{+\:15} RC 12 fram\NC\NR
\NC F 26 \NC 2 A \NC \aW{+\:15} RC 12 bak \NC\NR
\NC F 27 \NC 25 A \NC Värmefläkt\NC\NR
\NC F 28 \NC 10 A \NC Kompressor för klimatanläggning, centralsmörjsystem\NC\NR
\NC F 29 \NC 25 A \NC Kondensator för klimatanläggning\NC\NR
\NC F 30 \NC 5 A \NC Termostat för klimatanläggning\NC\NR
\NC F 31 \NC 5 A \NC \aW{+\:15} Multifunktionskonsol/Vpad \NC\NR
\NC F 32 \NC 15 A \NC Elektrisk vattenpump, roterande varningsljus\NC\NR
\NC F 33 \NC — \NC {[}Frei{]} \NC\NR
\NC F 34 \NC — \NC {[}Frei{]} \NC\NR
\NC F 35 \NC — \NC {[}Frei{]} \NC\NR
\NC F 36 \NC — \NC {[}Frei{]} \NC\NR
\stoptabulate
\stopcolumns

\page [yes]

\setups [pagestyle:bigmargin]


\subsection[sec:lighting]{Belysnings- och signalanordning}


\placefig [here] [fig:lighting] {Belysnings- och signalanordning på fordonet}
{\externalfigure [vhc:electric:lighting]}

\placelegende [margin,none]{}{%
\vskip 30pt
{\sla Förklaringar:}
\startLongleg
\item Positionsljus \hfill 12 V–5 W
\item Halvljus\hfill H7 12 V – 55 W
\item Blinkers\hfill orange 12 V – 21 W
\item {\stdfontsemicn Arbetsstrålkastare}\hfill G886 12 V – 55 W
\item Körriktningsvisare \hfill 12 V – 21 W
\item Back-/bromsljus\hfill 12 V – 5/21 W
\item Backstrålkastare\hfill 12 V – 21 W
\item {[}Frei{]}
\item Registreringsskyltsbelysning\hfill 12 V – 5 W
\item Roterande varningsljus\hfill H1 12 V – 55 W
\stopLongleg}

\subsubsubject{Inställning av strålkastarna}

\placefig [margin] [fig:lighting:adjustment] {Ljusstråle vid 5 m}
{\externalfigure [vhc:lighting:adjustment]
\startitemize
\sym{H\low{1}} Höjd på glödtråden: 100 cm
\sym{H\low{2}} Korrektur vid 2\hairspace\%: 10 cm
\stopitemize}

{\md Förutsättningar:} Behållare för ren-/återvinningsvatten full, förare vid ratten.

Strålkastarna ställs in från fabrik. Ljusstrålens höjd och lutning kan ställas in genom att man vrider på plasthållaren.

Om du vid en kontroll upptäcker att inställningen måste ändras, ska du lossa på fästskruven och korrigera lutningen så att den överensstämmer med gällande krav (se \in{fig.}[fig:lighting:adjustment]). Dra åt fästskruven igen.

\page [yes]
\setups [pagestyle:marginless]


\subsection[sec:battcheck]{Batteri}

\subsubsection{Säkerhetsanvisningar}

\startSymList
\PPfire
\SymList
\textDescrHead{Explosionsrisk}
När\index{Batteri+Säkerhetsanvisningar}\index{Fara+Explosion} batterier laddas bildas explosiv\index{Knallgas} knallgas. Batterier får endast laddas i lokaler med god ventilation! Undvik gnistbildning!
Undvik eld, rökning och öppen låga i närheten av batterier.
\stopSymList

\startSymList
\PHvoltage
\SymList
\textDescrHead{Risk för kortslutning}
Om\index{Batteri+Underhåll} pluspolen på det anslutna batteriet kommer i kontakt med fordonsdelar uppstår risk för\index{Fara+Eld}\index{Fara+Explosion} kortslutning.
Detta kan leda till att den gas som tränger ut ur batterier exploderar och orsakar allvarliga personskador.

\startitemize
\item Lägg aldrig metallföremål eller verktyg på batteriet.
\item Börja alltid med minuspolen och därefter pluspolen när du kopplar från batteriet.
\item Börja alltid med pluspolen och därefter minuspolen när du kopplar till batteriet.
\item Batteriets anslutningsklämmor får inte lossas eller tas loss när motorn är igång.
\stopitemize
\stopSymList


\startSymList
\PHcorrosive
\SymList
\textDescrHead{Risk för frätskador}
Använd\index{Fara+Frätskador} skyddsglasögon och syrabeständiga skyddshandskar. Batterivätska består till ca 27 procent
av svavelsyra(H\low{2}SO\low{4}) och kan därför orsaka frätskador.
Neutralisera\index{Batteri+Fara}\index{Batteri+-vätska} batterivätska som kommit i kontakt med huden med en lösning av dubbelkolsyrat natriumvätekarbonat och spola med rent vatten. Om du får batterivätska i ögonen ska du spola dem med rikligt med kallt vatten och kontakta en ögonläkare omedelbart.
\stopSymList

\startSymList
\startcombination[1*2]
{\PHcorrosive}{}
{\PHfire}{}
\stopcombination
\SymList
\textDescrHead{Lagring av batterier}
Batterier\index{Batteri+lagring} ska alltid lagras i upprätt läge. Annars kan batterivätska läcka och leda till frätskador eller, om den kommer i kontakt med andra substanser, till brand. \par\null\par\null
\stopSymList

\testpage [16]

\starttextbackground [FC]
\setupparagraphs [PictPar][1][width=2.4em,inner=\hfill]

\startPictPar
\PMproteyes
\PictPar
\textDescrHead{Skyddsglasögon}
När\index{Fara+Ögonskador} vatten och syra blandas kan vätska stänka i ögonen. Skölj ögonen med rent vatten och uppsök genast läkare!
\stopPictPar
\blank [small]

\startPictPar
\PMrtfm
\PictPar
\textDescrHead{Dokumentation}
Beakta alltid anvisningarna, säkerhetsåtgärderna och arbetsstegen i denna bruksanvisning vid hantering av batterier.
\stopPictPar
\blank [small]

\startPictPar
\PStrash
\PictPar
\textDescrHead{Miljöskydd}
Batterier\index{Miljöskydd} innehåller skadliga ämnen. Batterier får inte slängas i hushållsavfallet utan ska avfallshanteras korrekt. Lämna in batterierna till en fackverkstad eller på en insamlingsplats för batterier.

Uppladdade batterier ska alltid transporteras och lagras i upprätt läge. Säkra batterierna så att de inte kan välta under transport. Batterivätska kan tränga ut ur pluggens ventilationsöppningar och hamna i naturen.
\stopPictPar
\stoptextbackground

\page [yes]

\setups[pagestyle:normal]


\subsubsection{Praktiska råd}

Håll alltid batteriet fullständigt uppladdat. Det förlänger batteriets livslängd.

Genom\index{Batteri+Livslängd} att hålla batteriet uppladdat när fordonet inte ska användas under en längre period, förlänger man inte bara batteriet livslängd utan säkerställer även att fordonet alltid är driftsklart.

\placefig[margin][fig:batterycompartment]{\select{caption}{Batterifack (underhållslucka)}{Batterifack}}
{\externalfigure[batt:compartment]}


\subsubsection{Service}

Batteriet\sdeux\ är{\em ett underhållsfritt} blybatteri. Batteriet kräver inget underhåll utöver laddning och rengöring.

\startitemize
\item Se till att batteriets poler alltid är rena och torra. Smörj polerna lätt med syraavvisande fett.
\item Om batteriernas\index{Batteri+ladda} vilospänning
understiger\index{Batteri+Vilospänning} 12,4 V ska de laddas.
\stopitemize

\placefig[margin][fig:bclean]{Rengöring av polerna}
{\externalfigure[batt:clean]
\noteF
varmt\index{AnvändBatteri+rengöring}\index{Rengöring+batterier} vatten för att avlägsna det vita pulver
som uppstår till följd av korrosion. Om en pol är rostig ska du koppla loss batterikabeln och rengöra
polen med en stålborste. Applicera ett tunt fettlager på polerna.}


\subsubsection[sec:battery:switch]{Användning av batterifrånskiljaren}

{\sl Batterifrånskiljaren bör inte användas för ofta (t.ex. varje dag)!}

\startSteps
\item Slå från\index{Batterifrånskiljare} tändningen och vänta i ca 1 minut.
\item Öppna batterifacket (\inF[fig:batterycompartment]).
\item Tryck på den röda knappen på batterifrånskiljaren för att avbryta strömkretsen.
\item För att sluta strömkretsen igen ska du vrida batterifrånskiljaren ¼ varv medurs.
\stopSteps

% \starttextbackground [FCnb]
% \startPictPar
% \PMgeneric
% \PictPar
% Der Batterietrennschalter ist dafür vorgesehen, die Batterie für bestimmte Wartungs- und Reparaturarbeiten vorübergehend vom Stromkreis zu trennen. Es ist nicht empfehlenswert, den Batterietrennschalter regelmäßig (\eG\ täglich) zu betätigen: Bestimmte elektronische Komponenten sollten ständig unter Spannung stehen, ansonsten kann es zu Fehlermeldungen im Fehlerspeicher kommen.
% \stopPictPar
% \stoptextbackground

\stopsection

\page [yes]


\setups[pagestyle:marginless]

\section[sec:cleaning]{Rengöring av fordonet}

Börja med\startregister[index][vhc:lavage]{Underhåll+Rengöring} att spola bort slam och grovt smuts från karossen med rikligt med vatten. Spola inte bara sidoytorna utan även hjulhusen och fordonets undersida.

Särskilt under vintern är det viktigt att spola fordonet noggrant\index{Korrosion+Förebyggande} för att avlägsna rester av högkorrosivt vägsalt.

\starttextbackground [FC]
\startPictPar
\PHgeneric
\PictPar
\textDescrHead{Förebygga vattenskador}
Rengör aldrig fordonet med hjälp av {\em vattenkanoner} (t.ex.sådana som används inom brandkåren) eller{\em kallavfettning på kolvätebasis.} Om du arbetar med högtrycksånga ska du beakta anvisningarna nedan.
\stopPictPar
\blank[small]

\startPictPar
\pTwo[monde]
\PictPar
\textDescrHead{Miljöskydd}
När fordonet rengörs kan miljön runt arbetsplatsen förorenas.
Fordonet får endast rengöras på platser\index{Miljöskydd} med oljeavskiljare. Beakta gällande miljöbestämmelser.
\stopPictPar
\blank[small]

\startPictPar
\PMwarranty
\PictPar
\textDescrHead{Rengöring ska ske korrekt!}
\BosFull{boschung} ansvarar inte för skador som uppstår till följd av att anvisningarna för rengöring inte har följts.
\stopPictPar
\stoptextbackground


\subsection{Rengöring med högtryck}

Fordonet\index{Rengöring+Högtryck} kan rengöras med en vanlig högtryckstvätt.

Beakta följande punkter vid rengöring med högtryckstvätt:

\startitemize
\item Arbetstryck max 50\,bar
\item Flatstrålemunstycke med en högsta sprutvinkel på 25°
\item Sprutavstånd minst 80\,cm
\item Vattentemperatur max 40\,°C
\item Beakta avsnitt \about[reiMi], \atpage[reiMi].
\stopitemize

Om dessa anvisningar\index{Lack+Skador} inte beaktas kan det leda till skador på lacken och korrosionsskyddet\index{Skador+Lack}.

Beakta även bruksanvisningen och säkerhetsanvisningarna till högtryckstvätten.

\starttextbackground[FC]
\startPictPar\PPspray\PictPar
Vid rengöring med högtryckstvätt kan det tränga in vatten vid ställen på fordonet där det kan orsaka skador. Strålen ska därför aldrig riktas på känsliga delar eller enheter såsom följande:
\stopPictPar

\startitemize
\item Sensorer, elektriska kopplingar och anslutningar
\item Startmotor, generator, insprutningssystem
\item Magnetventiler
\item Ventilationsöppningar
\item Varma mekaniska komponenter
\item Informations-, varnings- och säkerhetsskyltar
\item Elektroniska styrenheter
\stopitemize

\textDescrHead{Motortvätt}
Se till att det inte tränger in vatten i insugs-, ventilations- och avluftningsöppningar. Vid rengöring med högtryckstvätt ska du inte rikta strålen direkt på elutrustning eller ledningar. Rikta inte strålen mot insprutningssystemet! Konservera motorn efter motortvätt. Skydda remmen mot konserveringsmedlet.
\stoptextbackground

\starttextbackground [FC]
\setupparagraphs [PictPar][1][width=6em,inner=\hfill]
\startPictPar
\framed[frame=off,offset=none]{\PMproteyes\PMprotears}
\PictPar
\textDescrHead{Restvatten}
När fordonet rengörs ansamlas vatten på vissa ställen (t.ex.i hålrummen på motorblocket eller växellådan). Avlägsna vattnet med tryckluft. Använd lämplig skyddsutrustning när du arbetar med tryckluft och beakta gällande säkerhetsanvisningar för systemet (multimunstycke).
\stopPictPar
\stoptextbackground


\subsubsection[reiMi]{Lämpliga rengöringsmedel}

Använd\index{Rengöringsmedel} endast rengöringsmedel som uppfyller följande specifikationer:

\startitemize
\item Ej frätande
\item PH-värde på 6–7
\item Utan lösningsmedel
\stopitemize

För att avlägsna envisa fläckar ska du använda sparsamt med rengöringsbensin eller sprit på en liten yta av lacken. Använd aldrig andra lösningsmedel.
Avlägsna rester av rengöringsmedel från lacket. Plastdelar kan spricka eller missfärgas om de rengörs med bensin.

Rengör ytor\index{Rengöring+Skyltar} där varnings- eller informationsskyltar har anbringats med rent vatten och en mjuk svamp.

Undvik att vatten hamnar i elektriska komponenter: Blinkers- och lamphus ska inte rengöras med fordonsborste utan med en mjuk trasa eller svamp.

\starttextbackground [CB]
\startPictPar
\GHSgeneric\par
\GHSfire
\PictPar
\textDescrHead{Fara på grund av kemikalier}
Rengöringsmedel kan medföra hälso- och säkerhetsrisker (brandfarliga material). Beakta de säkerhetsanvisningar och säkerhetsdatablad som gäller för respektive rengöringsmedel.
\stopPictPar
\stoptextbackground

\stopregister[index][vhc:lavage]


\page [yes]


\setups [pagestyle:bigmargin]

\startsection [title={Inställning av sugmunstycket},
reference={sec:main:suctionMouth}]


Det optimala avståndet\index{Sugmunstycke+Inställning} mellan underlaget och sugmunstyckets plastskena är 8\,mm.
Använd de tre inställningsmåtten som finns i verktygslådan (förarhytt, förarsida) för att kontrollera resp. ställa in avståndet.


\placefig [margin] [fig:suctionMouth] {Inställning av sugmunstycket}
{\Framed{\externalfigure [suctionMouth:adjust]}}

\placeNote[][service_picto]{}{%
\noteF
\starttextrule{\PHasphyxie\enskip Förgiftnings- och kvävningsrisk \enskip}
{\md Observera:} Under inställningsarbeten måste fordonets motor vara igång för att sugmunstycket ska hållas i flytande läge. För att undvika risken för förgiftning eller kvävning ska ett utsugssystem för avgaser användas. I annat fall får arbetena endast utföras på en plats med mycket god ventilation.
\stoptextrule}

\startSteps
\item Ställ fordonet på en vågrät och jämn yta på en plats med mycket god ventilation.
\item Aktivera \index{Utsug} \aW{arbets}läget (knapp på utsidan av växelspaken).

Låt motorn arbeta på tomgång. (Tryck på knappen~\textSymb{joy_key_engine_decrease} på multifunktionskonsolen för att sänka motorvarvtalet.)
\item Dra åt handbromsen och säkra bakhjulen med var sin kil.
\item Tryck på knappen ~\textSymb{joy_key_suction} för att sänka sugmunstycket.
\item Placera de tre inställningsmåtten~\LAa\ under sugmunstyckets plastskena så som visas på bilden.
\item [sucMouth:adjust]Lossa på fäst-~\Lone\ och inställningsskruvarna~\Ltwo\ på varje hjul; de fyra hjulen sänks till marken.
\item Dra åt skruvarna~\Lone\ och~\Ltwo\ igen och ta bort de tre inställningsmåtten.
\item Lyft/sänk sugmunstycket och kontrollera inställningen med inställningsmåtten. Om inställningen inte är helt korrekt ska du upprepa proceduren från och med punkt~\in[sucMouth:adjust].

\stopSteps


\stopsection
\stopchapter
\stopcomponent


