\startcomponent c_60_work_s2_120-sv
\product prd_ba_s2_120-sv


\startchapter [title={Det dagliga arbetet med S2},
reference={chap:using}]

\setups [pagestyle:marginless]


% \placefig[margin][fig:ignition:key]{Clé de contact}
% {\externalfigure [work:ignition:key]}
\startregister[index][chap:using]{Inbetriebnahme}

\startsection [title={Driftsättning},
reference={sec:using:start}]


\startSteps
\item Kontrollera att de regelbundna kontrollerna och underhållsarbetena har genomförts som de ska.
\item Starta motorn med hjälp av tändningsnyckeln: Slå på tändningen, fortsätt sedan att vrida nyckeln medurs och håll kvar den där tills motorns startar (fungerar endast när växelspaken befinner sig i neutralläget).
\stopSteps

\start
\setupcombinations [width=\textwidth]

\placefig[here][fig:select:drive]{Växelspak}
{\startcombination [2*1]
{\externalfigure [work:select:fDrive]}{Växelspak i läget\aW{framåtkörning}}
{\externalfigure [work:select:rDrive]}{Växelspak i läget\aW{backning}}
\stopcombination}
\stop


\startSteps [continue]
\item Vrid växelspakens kontakt för att lägga i en växel i\aW{kör}läget:
\startitemize [R]
\item Första steget
\item Andra steget (automatdrift; startar automatiskt i det första steget)
\stopitemize

eller tryck på knappen på spakens utsida för att aktivera/avaktivera \aW{arbets}läget.
\stopSteps

\startbuffer [work:config]
\starttextbackground [FC]
\startPictPar
\PMrtfm
\PictPar
I arbetsläget kan endast det första steget användas och motorn arbetar med 1300\,min\high{\textminus 1}.

Reglera motorvarvtalet med hjälp av knapparna ~\textSymb{joy_key_engine_increase} och~\textSymb{joy_key_engine_decrease} på multifunktionskonsolen.
\stopPictPar
\stoptextbackground
\stopbuffer

\getbuffer [work:config]

\startSteps [continue]
\item Tryck växelspaken uppåt och framåt (framåtkörning) resp. uppåt och bakåt(backning). Se figurer ovan.
\item Lossa handbromsen innan du accelererar.
\stopSteps

\starttextbackground [FC]
\startPictPar
\PMrtfm
\PictPar
{\md Lossa handbromsen helt!} Handbromsens position övervakas av en elektronisk sensor. När handbromsen inte är helt lossad är fordonets hastighet begränsad till 5\,km/h.
\stopPictPar
\stoptextbackground

\startSteps [continue]
\item Tryck långsamt ned gaspedalen för att sätta fordonet i rörelse.
\stopSteps


%% NOTE: New text [2014-04-29]:
\subsection [sSec:suctionClap] {Lucka till sugkanal}

Sugsystemet genererar ett luftflöde antingen från sugmunstycket eller från handsugslangen (tillval) till smutsbehållaren.

Med en lucka som manövreras för hand (\inF[fig:suctionClap], \atpage[fig:suctionClap]) kan man koppla om luftflödet mellan sugmunstycket och handsugslangen.

\placefig [here] [fig:suctionClap] {Lucka till sugkanal}
{\startcombination [2*1]
{\externalfigure [work:suctionClap:open]}{Sugkanal öppen}
{\externalfigure [work:suctionClap:closed]}{Sugkanal stängd}
\stopcombination}

I normal drift~– arbete med sugmunstycket~– måste sugkanalen vara öppen (omkopplingsspaken pekar uppåt).

För att du ska kunna sätta i handsugslangen måste sugkanalen vara stängd (omkopplingsspaken pekar nedåt). På så sätt leds luftflödet genom handsugslangen.
%% End new text

\stopsection


\startsection [title={Ta maskinen ur drift},
reference={sec:using:stop}]

\index{Ta maskinen ur drift}

\startSteps
\item Aktivera handbromsen(spak mellan stolarna) och ställ växelspaken i \aW{neutralläge}.
\item Genomför de nödvändiga kontrollarbetena~– de dagliga och ev. de veckovisa kontrollerna~– på det sätt som beskrivs på \atpage[table:scheduledaily].
\stopSteps

\getbuffer [prescription:handbrake]

\stopsection


\startsection [title={Sopnings- och uppsugningsarbeten},
reference={sec:using:work}]

\startSteps
\item Ta fordonet i drift\index{Sopning} på det sätt som beskrivs i\in{§}[sec:using:start], \atpage[sec:using:start].
\item Aktivera \index{Uppsugning}\aW{arbets}läget (knapp på utsidan av växelspaken).
\stopSteps

% \getbuffer [work:config]
%% NOTE: outcommented by PB

\startSteps [continue]
\item Tryck på knappen~\textSymb{joy_key_suction_brush} för att slå på turbin och kvastar.

{\md Alternativ:} {\lt Tryck på knappen ~\textSymb{joy_key_suction} för att arbeta endast med sugmunstycket.}

\item Ställ in rotationshastigheten på kvastarna med knapparna~\textSymb{joy_key_frontbrush_increase}\textSymb{joy_key_frontbrush_decrease} på multifunktionskonsolen.

\item Med hjälp av respektive spak ska du föra kvastarna till en position där den optimala arbetsbredden nås.
\stopSteps

\vfill

\start
\setupcombinations [width=\textwidth]

\placefig[here][fig:brush:position]{Positionering av kvastarna}
{\startcombination [2*1]
{\externalfigure [work:brushes:enlarge]}{Kvastar inåt/utåt}
{\externalfigure [work:brush:left:raise]}{Kvastar upp/ned}
\stopcombination}
\stop

\page [yes]


\subsubsubject{Befuktning av kvastar och sugkanal}

Aktivera\index{Sopning+Befuktning} kontakten~\textSymb{temoin_busebalais} mellan stolarna.

{\md Position 1:} Vattenpumpen arbetar automatiskt så länge kvastarna är aktiva.

{\md Position 2:} Vattenpumpen arbetar permanent (praktiskt vid t.ex.inställningsarbeten.)


\subsubsubject{Grovsmuts}

\startSteps [continue]
\item Om det finns risk för att större föremål (t.ex.PET-flaskor) blockerar sugmunstycket ska du öppna luckan för grovsmuts\index{Lucka för grovsmuts} med hjälp av knapparna på sidan av multifunktionskonsolen eller ~– om detta inte är tillräckligt~– lyfta upp\index{Sugmunstycke+Grovsmuts} sugmunstycket tillfälligt.
\stopSteps

\start
\setupcombinations [width=\textwidth]

\placefig[here][fig:suctionMouth:clap]{Hantering av grovsmuts}
{\startcombination [2*1]
{\externalfigure [work:suction:open]}{Öppna luckan för grovsmuts}
{\externalfigure [work:suction:raise]}{Lyft sugmunstycket tillfälligt}
\stopcombination}
\stop

\stopsection


\startsection [title={Tömma smutsbehållaren},
reference={sec:using:container}]

\startSteps
\item Kör\index{Smutsbehållare+Tömning} fordonet till en plats som lämpar sig för att tömma smutsbehållaren. Beakta gällande miljöskyddsbestämmelser.
\item Dra åt handbromsen och ställ växelspaken i \aW{neutralläge}. (Nödvändigt för att spärra upp kontakten för tippning av behållaren).
\stopSteps

\getbuffer [prescription:container:gravity]

\startSteps [continue]
\item Lås upp och öppna locket till smutsbehållaren.
\item Aktivera kontakten~\textSymb{temoin_kipp2} (mittkonsol mellan stolarna) för att tippa upp smutsbehållaren.
\item När behållaren är tom ska du tvätta insidan med vattenstråle. Du kan använda den integrerade vattensprutan (tillval).
\stopSteps

\start
\setupcombinations [width=\textwidth]
\placefig[here][fig:brush:adjust]{Hantering av smutsbehållaren}
{\startcombination [3*1]
{\externalfigure [container:cover:unlock]}{Lås till smutsbehållarens lock}
{\externalfigure [container:safety:unlocked]}{Säkerhetsstötta}
{\externalfigure [container:safety:locked]}{Säkerhetsstötta låst}
\stopcombination}
\stop

\startSteps [continue]
\item Kontrollera/rengör packningarna och kontaktytorna på packningarna, behållaren, återvinningssystemet och sugkanalen.
\stopSteps

\getbuffer [prescription:container:tilt]

\startSteps [continue]
\item Aktivera kontakten~\textSymb{temoin_kipp2} för att tömma smutsbehållaren. (Ta först bort säkerhetsstöttorna från hydraulcylindrarna, om tillämpligt.)
\item Lås locket till smutsbehållaren.
\stopSteps

\stopsection


\startsection [title={Handsugslang},
reference={sec:using:suction:hose}]

\sdeux\ kan utrustas\index{Handsugslang} med en handsugslang som tillval. Slangen är fäst på smutsbehållarens lock och enkel att använda.

{\sla Förutsättningar:}

Smutsbehållaren är helt nedsänkt; \sdeux\ befinner sig i \aW{arbets}läget. (Se \in{§}[sec:using:start], \atpage[sec:using:start].)

\startfigtext[left][fig:using:suction:hose]{Handsugslang}
{\externalfigure[work:suction:hose]}
\startSteps
\item Tryck på knappen~\textSymb{temoin_aspiration_manuelle} på takkonsolen för att aktivera sugsystemet.
\item Dra åt handbromsen ordentligt innan du lämnar förarhytten.
\item Stäng sugkanalen med dess lucka. (Se \in{§}[sSec:suctionClap], \atpage[sSec:suctionClap].)
\item Dra handsugslangen ur sin hållare från munstycket och börja arbetet.
\item När du är klar ska du trycka på knappen~\textSymb{temoin_aspiration_manuelle} igen för att slå från sugsystemet.
\stopSteps
\stopfigtext

\stopsection

\page [yes]

\setups[pagestyle:normal]


\startsection [title={Högtrycksvattenspruta },
reference={sec:using:water:spray}]

\sdeux\ kan utrustas\index{Vattenspruta} med en högtrycksvattenspruta som tillval. Vattensprutan är installerad på den bakre högra underhållsdörren och ansluten till en 10 m lång slangvinda~på den motsatta fordonssidan~.

Vattensprutan används på följande sätt:

{\sla Förutsättningar:}

Det finns tillräckligt med vatten i renvattentanken; \sdeux\ befinner sig i \aW{arbets}läget. (Se \in{§}[sec:using:start], \atpage[sec:using:start].)

\placefig[margin][fig:using:water:spray]{Högtrycksvattenspruta }
{\externalfigure[work:water:spray]}

\startSteps
\item Tryck på knappen~\textSymb{temoin_buse} på takkonsolen för att aktivera högtrycksvattenpumpen.
\item Dra åt handbromsen ordentligt innan du lämnar förarhytten.
\item Öppna den bakre högra underhållsdörren och ta ut vattensprutan.
\item Rulla upp så mycket av slangen som du behöver och börja arbetet.
\item När du är klar ska du trycka på knappen~\textSymb{temoin_buse} igen för att slå från högtrycksvattenpumpen.
\item Dra kort i slangen för att låsa upp spärren och rulla upp slangen.
\item Sätt tillbaka vattensprutan i hållaren och stäng underhållsdörren.
\stopSteps

\stopsection

\page [yes]

\setups [pagestyle:marginless]


\startsection [title={Arbeta med den tredje kvasten (tillval)},
reference={sec:using:frontBrush},
]

\startSteps
\item Ta\index{Sopa} fordonet i drift så som beskrivs i \in{avsnitt}[sec:using:start] \atpage[sec:using:start] .
\item Aktivera\index{Tredje kvast} \aW{arbetsläget}(knapp på utsidan av växelspaken).
\stopSteps

% \getbuffer [work:config]

\startSteps [continue]
\item Kontrollera att den tredje kvasten är aktiverad på Vpad-bildskärmen
(se \textSymb{vpadFrontBrush} \textSymb{vpadFrontBrushK}, \atpage[vpad:menu]).
\item Tryck på knappen~\textSymb{joy_key_frontbrush_act} för att aktivera den tredje kvastens hydraulsystem.
\item Tryck på knappen ~\textSymb{joy_key_frontbrush_left} eller~\textSymb{joy_key_frontbrush_right} för att den tredje kvasten ska rotera i önskad riktning.

\item Reglera rotationshastigheten med hjälp av knapparna~\textSymb{joy_key_frontbrush_increase} och ~\textSymb{joy_key_frontbrush_decrease} på multifunktionskonsolen.

\item Positionera kvastarna så som visas på figurerna nedan med hjälp av spakarna.

\stopSteps

{\md Observera:} {\lt För att du ska kunna positionera sidokvastarna måste du aktivera den tredje kvastens hydraulsystem med hjälp av knappen ~\textSymb{joy_key_frontbrush_act}.}
\vfill

\start
\setupcombinations [width=\textwidth]

\placefig[here][fig:brush:position]{Positionera den tredje kvasten}
{\startcombination [2*1]
{\externalfigure [work:frontBrush:move]}{Uppåt/nedåt; åt vänster/höger}
{\externalfigure [work:frontBrush:incline]}{Tvär-/längslutning}
\stopcombination}
\stop

\stopsection

\stopregister[index][chap:using]

\stopchapter
\stopcomponent

