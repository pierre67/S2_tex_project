\startcomponent c_45_vpad_s2_120-sv
\product prd_ba_s2_120-sv

\startchapter[title={Färddator (Vpad)},
reference={sec:vpad}]

\setups[pagestyle:marginless]


\startsection[title={Beskrivning av Vpad},
reference={vpad:description}]

\startfigtext [left] {Vpad SN på förarplatsen}
{\externalfigure[vpad:inside:view]}
\textDescrHead{Innovativ, intelligent … } \Vpad\ har utvecklats för styrning av de allt komplexare maskiner som används inom det kommunala området och som utrustas med allt fler funktioner.
Med \Vpad\ har föraren ett system till hands som inte bara är begränsat till att leverera visuell och akustisk information om samtliga arbets- och maskinprocesser i realtid.
\Vpad\ sätter helt nya standarder och kännetecknas främst av intuitiv användning, praktisk ergonomi och kommandologik.

Tack vare de många funktionerna har \Vpad\ ett mycket brett användningsområde och är därmed mycket mer än bara en elektronisk styrenhet.
\stopfigtext

\textDescrHead{… universell} Kompatibilitet och flexibilitet var ledorden under utvecklingen av \Vpad\.
Styrenheten har en modulär utformning som gör att den enkelt kan anpassas till respektive användningsvillkor och utrustningsvarianter. Den har även flera gränssnitt och alternativ för dataöverföring - och till och med möjlighet till trådlös internetuppkoppling.
\Vpad\ arbetar med avancerad 32-bitsteknik och operativsystem i realtid.
\vfill


\startfigtext[left]{Multifunktionskonsol}
{\externalfigure[console:topview]}
\textDescrHead{… och modulbyggd} Den modulbyggda utformningen av \Vpad\ erbjuder en enorm fördel:
Versionen SN som används som standard i \sdeux\ kan nämligen successivt utökas med fler komponenter, t.ex. ett modem eller en konsol (se figur).
Den modulära utformningen begränsar sig inte bara till maskinvaran, utan även systemets programvara kan i stor omfattning anpassas till ändrade villkor.

Multifunktionskonsolen till \sdeux\ är ett högavancerat gränssnitt mellan användare och maskin. Med konsolen kan man styra hela sop-/sugsystemet.
\stopfigtext

\page [yes]


\subsection[vpad:home]{Huvudbildskärm}

%% Note: outcommented by PB
% \placefig[left][fig:vpad:engineData]{Accueil mode transport}
% {\scale[sx=1.5,sy=1.5]
% {\setups[VpadFramedFigureHome]
% \VpadScreenConfig{
% \VpadFoot{\VpadPictures{vpadClear}{vpadBeacon}{vpadEngine}{vpadSignal}}}
% \framed{\null}}
% }


\start

\setupcombinations[width=\textwidth]

\placefig [here][fig:vpad:engineData]{Huvudbildskärm}
{\startcombination [2*1]
{\setups[VpadFramedFigureHome]% \VpadFramedFigureK pour bande noire
\VpadScreenConfig{
\VpadFoot{\VpadPictures{vpadClear}{vpadBeacon}{vpadEngine}{vpadSignal}}}%
\scale[sx=1.5,sy=1.5]{\framed{\null}}}{\aW{Kör}läge}
{\setups[VpadFramedFigureWork]% \VpadFramedFigureK pour bande noire
\VpadScreenConfig{
\VpadFoot{\VpadPictures{vpadClear}{vpadBeacon}{vpadEngine}{vpadSignal}}}%
\scale[sx=1.5,sy=1.5]{\framed{\null}}}{\aW{Arbets}läge}
\stopcombination}

\stop

\blank [1*big]

Huvudbildskärmen på \Vpad\ omfattar alla nödvändiga element för övervakning av samtliga funktioner på \sdeux.

I det övre området finns alla kontrollindikeringar.

I det mellersta området visas bland annat följande uppgifter i\, realtid:
motorns hastighet, varvtal och temperatur, bränslenivå, nivån på återvinningsvatten etc.

Läget \aW{Körning} indikeras med harsymbolen ~\textSymb{transport_mode}, läget \aW{Arbete} med en sköldpadda~\textSymb{working_mode}.

På menyraden längst ned visas de tillgängliga menyerna: Tryck på mitten av pekskärmen för att visa fler menyer.

\page [yes]

\defineparagraphs[SymVpad][n=2,distance=4mm,rule=off,before={\page[preference]},
							after={\nobreak\hrule\blank [2*medium]\page[preference]}]
\setupparagraphs [SymVpad][1][width=4em,inner=\hfill]


\subsection{Kontrollindikeringar på Vpad-bildskärmen}


% \start % local group for temporary redefinition of \textDescrHead [TF]
% \define[1]\textDescrHead{{\bf#1\fourperemspace}}


\startSymVpad
\externalfigure[vpadTEnginOilPressure][height=1.7\lH]
\SymVpad
\textDescrHead{Motoroljetryck} (röd) Motoroljetryck för lågt. Stäng genast av motorn.

+\:Fehlermeldung \# 604
\stopSymVpad

\startSymVpad
\externalfigure[vpadWarningBattery][height=1.7\lH]
\SymVpad
\textDescrHead{Batteriladdning} (röd) Batteriets laddström för låg. Kontakta en verkstad.
\stopSymVpad

\startSymVpad
\externalfigure[vpadWarningEngine1][height=1.7\lH]
\SymVpad
\textDescrHead{Motordiagnos} (gul) Fel i motorstyrningen. Kontakta en verkstad.
\stopSymVpad

\startSymVpad
\externalfigure[vpadWarningService][height=1.7\lH]
\SymVpad
\textDescrHead{Kontakta en verkstad} (gul) Dags för normalt fordonsunderhåll
(se \about [sec:schedule] \atpage [sec:schedule])
eller ett motorfel har registrerats (kontakta en fackverkstad).

+\:Fehlermeldungen \# 650 till \# 653, eller \# 703
\stopSymVpad


\startSymVpad
\externalfigure[vpadTDPF][height=1.7\lH]
\SymVpad
\textDescrHead{Partikelfilter} (gul) Regenerering av partikelfiltret startar när driftvillkoren är lämpliga.

{\md Observera:} {\lt Stäng om möjligt {\em inte} av motorn när denna indikering lyser!}
\stopSymVpad


\startSymVpad
\externalfigure[vpadTBrakeError][height=1.7\lH]
\SymVpad
\textDescrHead{Bromssystem} (röd) Fel i bromssystemet. Kontakta en verkstad.

+\:Fehlermeldung \# 902
\stopSymVpad


\startSymVpad
\externalfigure[vpadTBrakePark][height=1.7\lH]
\SymVpad
\textDescrHead{Handbroms} (röd) Fordonets handbroms är åtdragen.

+\:Fehlermeldung \# 905
\stopSymVpad

\startSymVpad
\externalfigure[vpadTEngineHeating][height=1.7\lH]
\SymVpad
\textDescrHead{Förglödningssystem} (gul) Motorns förglödningssystem är igång.

En blinkade lampa visar att felet har registrerats i felminnet.
\stopSymVpad


\startSymVpad
\externalfigure[vpadTFuelReserve][height=1.7\lH]
\SymVpad
\textDescrHead{Bränslenivå} (gul) Bränslenivån är mycket låg (reserv).
\stopSymVpad

\startSymVpad
\externalfigure[vpadTBlink][height=1.7\lH]
\SymVpad
\textDescrHead{Varningsblinkers} (grön) Varningsblinkersen är aktiverade.
\stopSymVpad

\startSymVpad
\externalfigure[vpadTLowBeam][height=1.7\lH]
\SymVpad
\textDescrHead{Positionsljus} (grön) Positionsljuset är påslaget.
\stopSymVpad

\startSymVpad
\HL\NC \externalfigure[vpadSyWaterTemp][height=1.7\lH]
\SymVpad
\textDescrHead{Temperatur} (röd) Motorns eller hydraulvätskans temperatur är för hög. Kontakta en verkstad.

+\:Fehlermeldung \# 700 eller \# 610
\stopSymVpad

\startSymVpad
\externalfigure[vpadWarningFilter][height=1.7\lH]
\SymVpad
\textDescrHead{Filter tilltäppt} (röd) Det kombinerade hydraulfiltret eller luftfiltret är tilltäppt.

+\:Fehlermeldung \# 702 eller \# 851
\stopSymVpad

\startSymVpad
\externalfigure[vpadTSpray][height=1.7\lH]
\SymVpad
\textDescrHead{Vattenspruta} (gul) Vattensprutans högtryckspump är aktiverad.

Kontakter \textSymb{temoin_buse} i takkonsolen.
\stopSymVpad

\startSymVpad
\externalfigure[vpadTClear][height=1.7\lH]
\SymVpad
\textDescrHead{Felmeddelande} (röd) Det finns ett felmeddelande i minnet på \Vpad. Tryck på knappen~\textSymb{vpadClear} för att visa alla sparade meddelanden. Kontakta en verkstad.
\stopSymVpad


% \stop % local group for temporary redefinition of \textDescrHead

\stopsection

\page [yes]


\startsection [title={Menyerna på Vpad},
reference={vpad:menu}]

\start

\setupTABLE [background=color,
frame=off,
option=stretch,textwidth=\makeupwidth]

\setupTABLE [r] [each] [style=sans, background=color, bottomframe=on, framecolor=TableWhite, rulethickness=1.5pt]
\setupTABLE [r] [first][backgroundcolor=TableDark, style=sansbold]
\setupTABLE [r] [odd][backgroundcolor=TableMiddle]
\setupTABLE [r] [even] [backgroundcolor=TableLight]
\bTABLE [split=repeat]
\bTABLEhead
\bTR\bTD Meny \eTD\bTD Beteckning\index{Vpad+Indikering} \eTD\bTD Funktion \eTD\eTR
\eTABLEhead

\bTABLEbody
\bTR\bTD \externalfigure [v:symbole:clear] \eTD\bTD Felmeddelande(n) \eTD\bTD Visa och kvittera de felmeddelanden som finns sparade i Vpad. \eTD\eTR
\bTR\bTD \framed[frame=off]{\externalfigure [v:symbole:beacon]\externalfigure [v:symbole:beacon:black]} \eTD\bTD Roterande varningsljus\eTD\bTD Roterande varningsljus till/från \eTD\eTR
\bTR\bTD \externalfigure [v:symbole:engine] \eTD\bTD Uppgifter i realtid\eTD\bTD Driftsdata i realtid från motor och hydraulsystem\eTD\eTR
\bTR\bTD \externalfigure [v:symbole:oneTwoThree] \eTD\bTD Mätare \eTD\bTD Visning av driftstidsmätare: dagsmätare, säsongsmätare, totalmätare\eTD\eTR
\bTR\bTD \externalfigure [v:symbole:serviceInfo] \eTD\bTD Underhållsintervall \eTD\bTD Visar datumet samt den tid som återstår fram till nästa underhåll eller till nästa omfattande service \eTD\eTR
\bTR\bTD \externalfigure [v:symbole:trash] \eTD\bTD Mätare\eTD\bTD Återställ mätare eller serviceintervall \eTD\eTR
\bTR\bTD \externalfigure [v:symbole:sunglasses] \eTD\bTD Bildskärmsläge \eTD\bTD Typ av bildskärmsbelysning \aW{Dag} och\aW{natt} \eTD\eTR
\bTR\bTD \externalfigure [v:symbole:color] \eTD\bTD Ljusstyrka/kontrast \eTD\bTD Inställningar för bildskärmens ljusstyrka och kontrast \eTD\eTR
\bTR\bTD \externalfigure [v:symbole:select] \eTD\bTD Välj\eTD\bTD Välj den markerade posten eller kvittera ett felmeddelande \eTD\eTR
\bTR\bTD \externalfigure [v:symbole:return] \eTD\bTD Bekräfta \eTD\bTD Bekräfta ditt val\eTD\eTR
\bTR\bTD \framed[frame=off]{\externalfigure [v:symbole:up]\externalfigure [v:symbole:down]} \eTD\bTD Pil upp/ned, Flytta markeringen upp eller ned eller öka/minska valt värde \eTD\eTR
\bTR\bTD \externalfigure [v:symbole:rSignal] \eTD\bTD Backsignal \eTD\bTD Aktivera/avaktivera akustisk varningssignal vid backning\eTD\eTR
\bTR\bTD \externalfigure [v:symbole:power] \eTD\bTD Slå från bildskärmen \eTD\bTD Håll intryckt i ca 5 sek. för att slå från bildskärmen på Vpad. \eTD\eTR
\bTR\bTD \framed[frame=off]{\externalfigure [v:symbole:frontBrush]\externalfigure [v:symbole:frontBrush:black]}
\eTD\bTD Tredje kvast\index{Tredje kvast} (tillval) \eTD\bTD Aktivera den tredje kvasten.
Den tredje kvasten kan nu aktiveras på det sätt som beskrivs på sidan \at[sec:using:frontBrush]. \eTD\eTR
\eTABLEbody
\eTABLE
\stop


\subsection{Fler symboler på Vpad-bildskärmen}


\subsubsubject{Renvatten- och återvinningsvattenreservoar}


% \start % local group for temporary redefinition of \textDescrHead [TF]
% \define[1]\textDescrHead{{\bf#1\fourperemspace}}


\startSymVpad
\externalfigure[sym:vpad:water]
\SymVpad
\textDescrHead{Nivå renvatten} Renvattnets nivå är för låg (max. 190\,l; bakom förarhytten).
\stopSymVpad

\startSymVpad
\externalfigure[sym:vpad:rwater:yellow]
\SymVpad
\textDescrHead{Nivå återvinningsvatten} (gul) Återvinningsvattnets nivå ligger under värmeväxlaren. Hydraulvätskan kyls inte ned och befuktningssystemet i sugkanalen värms inte upp.
\stopSymVpad

\startSymVpad
\externalfigure[sym:vpad:rwater]
\SymVpad
\textDescrHead{Nivå återvinningsvatten} (röd) Återvinningsvattnets nivå är för låg (max. 140\,l; under smutsbehållaren).
\stopSymVpad


\subsubsubject{Sugsystem} % nouveau

{\em Den här symbolen visas endast när kvastarna är avaktiverade.}

\startSymVpad
\externalfigure[sym:vpad:sucker]
\SymVpad
\textDescrHead{Sugmunstycke} Sugsystem\index{Sugmunstycke} aktiverat:
Sugmunstycket är nedsänkt och turbinen är aktiverad.
\stopSymVpad


\subsubsubject{Sidokvast} % nouveau

{\em Den här symbolen visas endast när den tredje kvasten inte är aktiverad.}

\startSymVpad
\externalfigure[sym:vpad:sideBrush:83]
\SymVpad
\textDescrHead{Sidokvast} Kvast\index{Sopning}\index{Sidokvast} aktiverad. Rotationshastigheten (i \% av den högsta rotationshastigheten [V\low{max}]) visas under symbolen, den aktuella avlastningen för respektive kvast visas ovanför symbolen (\type{ } = flytande läge, 14 = högsta avlastning).

{\md Avlastning:} {\lt Ju lägre avlastning desto högre är kvastarnas tryck mot marken.}
\stopSymVpad


\startSymVpad
\externalfigure[sym:vpad:sideBrush:float:60]
\SymVpad
\textDescrHead{Flytande läge}(grön i nederkanten)
För att slå från avlastningen ska du trycka spaken framåt i ca 2 sekunder. Kvasten ligger nu mot marken med hela sin egenvikt. Kvastarnas rotationshastighet ligger på 60\hairspace\% av V\low{max} (exempel).
\stopSymVpad

\startSymVpad
\externalfigure[sym:vpad:sideBrush]
\SymVpad
\textDescrHead{Sidokvastar} Kvastarna är aktiverade. De står stilla och är i upplyft läge.
\stopSymVpad


\subsubsubject{Tredje kvast (tillval)} % nouveau

\startSymVpad
\externalfigure[sym:vpad:frontBrush]
\SymVpad
\textDescrHead{Tredje kvast} Den tredje kvasten\index{Tredje kvast} är aktiverad. Rotationshastigheten (i \% av den högsta rotationshastigheten [V\low{max}]) visas under symbolen.
\stopSymVpad


\startSymVpad
\externalfigure[sym:vpad:frontBrush:left]
\SymVpad
\textDescrHead{Flytande läge}(grön i nederkanten)
För att slå från avlastningen ska du trycka spaken framåt i ca 2 sekunder. Kvasten ligger nu mot marken med hela sin egenvikt. Kvastarnas rotationshastighet ligger på 70\hairspace\% av V\low{max} (exempel).

{\md Rotationsriktning:} {\lt I ovankanten visas rotationsriktningen (svart pil på gul bakgrund).}
\stopSymVpad

% \stop % local group for temporary redefinition of \textDescrHead

\stopsection


\page [yes]

\startsection[title={Inställning av bildskärmens ljusstyrka},
reference={sec:vpad:brightness}]

Bildskärmen till \Vpad\ kan arbeta i två förkonfigurerade
ljusstyrkor: Läge \aW{dag}~– \textSymb{vpadSunglasses}, normal ljusstyrka~– och läge \aW{natt}~– \textSymb{vpadMoon}, reducerad ljusstyrka.
Med knappen \textSymb{vpadColor} kan du komma åt olika parametrar.

Du kan ändra de förkonfigurerade ljusstyrkestegen på följande sätt:

\startSteps
\item Tryck mitt på pekskärmen för att bläddra genom menyraden längst ned på bildskärmen.
\item Tryck på symbolen \textSymb{vpadSunglasses} resp. \textSymb{vpadMoon} för att gå till det läge som du vill ändra.
\item Tryck på \textSymb{vpadColor} för att visa parametrarna.
\item Med hjälp av
pilsymbolerna~\textSymb{vpadUp}\textSymb{vpadDown} ska du markera den parameter som du vill ändra och välja den med ~\textSymb{vpadSelect}.
\item Ändra värdet med hjälp av symbolerna
\textSymb{vpadMinus}\textSymb{vpadPlus}. Se till att du inte minskar ljudstyrkan så mycket (\VpadOp{162} -255) att du inte längre kan urskilja något på bildskärmen!
\stopSteps
\blank [1*big]

\start
\setupcombinations[width=\textwidth]
\startcombination [3*1]
{\setups[VpadFramedFigureHome]% \VpadFramedFigureK pour bande noire
\VpadScreenConfig{
\VpadFoot{\VpadPictures{vpadGPS}{vpadTachygraph}{vpadSunglasses}{vpadColor}}}%
\framed{\null}}{Tryck mitt på pekskärmen}
{\setups[VpadFramedFigure]
\VpadScreenConfig{
\VpadFoot{\VpadPictures{vpadReturn}{vpadUp}{vpadDown}{vpadSelect}}}%
\framed{\bTABLE
\bTR\bTD \VpadOp{160} \eTD\eTR
\bTR\bTD [backgroundcolor=black,color=TableWhite] \VpadOp{162}\hfill 15 \eTD\eTR
\bTR\bTD \VpadOp{163}\hfill 180 \eTD\eTR
\bTR\bTD \VpadOp{164}\hfill 55 \eTD\eTR
\bTR\bTD \VpadOp{165}\hfill 3 \eTD\eTR
\eTABLE}}{Välj med \textSymb{vpadSelect}}
{\setups[VpadFramedFigure]% \VpadFramedFigureK pour bande noire
\VpadScreenConfig{
\VpadFoot{\VpadPictures{vpadReturn}{vpadMinus}{vpadPlus}{vpadNull}}}%
\framed[backgroundscreen=.9]{\bTABLE
\bTR\bTD \VpadOp{160} \eTD\eTR
\bTR\bTD \VpadOp{162}\hfill -80 \eTD\eTR
\bTR\bTD \VpadOp{163}\hfill 180 \eTD\eTR
\bTR\bTD \VpadOp{164}\hfill 55 \eTD\eTR
\bTR\bTD \VpadOp{165}\hfill 3 \eTD\eTR
\eTABLE}}{Ändra värdet med\textSymb{vpadMinus}\textSymb{vpadPlus}}
\stopcombination
\stop
\blank [1*big]

\startSteps [continue]
\item Bekräfta värdet med \textSymb{vpadReturn}.
\item Tryck på symbolen \textSymb{vpadReturn}, igen för att gå tillbaka till huvudbildskärmen.
\stopSteps

\stopsection

\page [yes]


\startsection[title={Drifttids- och kilometermätare},
reference={vpad:compteurs}]

Programvaran i \Vpad\ omfattar tre olika mätperioder~– \aW{dag},
\aW{säsong}, \aW{total}~– där olika mätare kan arbeta såsom \aW{tillryggalagd sträcka}, \aW{driftstimmar} (motor eller borste), \aW{arbetstid} (per förare).

Mätarna kan avläsas eller återställas på följande sätt:

\startSteps
\item Tryck mitt på pekskärmen för att bläddra genom menyraden.
\item Tryck på symbolen \textSymb{vpadOneTwoThree} för att visa dagsmätaren.
\item Med hjälp av symbolerna fram/tillbaka~\textSymb{vpadBW}\textSymb{vpadFW}
kan du växla till total- eller säsongsmätaren.
\item Tryck på\textSymb{vpadTrash} för att återställa den visade mätaren.
\item I ett dialogfönster uppmanas du att bekräfta återställningen.
\stopSteps
\blank [1*big]

\start
\setupcombinations[width=\textwidth]
\startcombination [3*1]
{\setups[VpadFramedFigure]% \VpadFramedFigureK pour bande noire
\VpadScreenConfig{
\VpadFoot{\VpadPictures{vpadOneTwoThree}{vpadTachygraph}{vpadSunglasses}{vpadColor}}}%
\framed{\bTABLE
\bTR\bTD \VpadOp{120} \eTD\eTR
\bTR\bTD \VpadOp{123}\hfill 87,4\,h \eTD\eTR
\bTR\bTD \VpadOp{125}\hfill 62,0\,h \eTD\eTR
\bTR\bTD \VpadOp{126}\hfill 240,2\,km \eTD\eTR
\bTR\bTD \VpadOp{124}\hfill 901,9\,km \eTD\eTR
\bTR\bTD \VpadOp{127}\hfill 2,1\,l/h \eTD\eTR
\eTABLE}}{Tryck på symbolen~\textSymb{vpadOneTwoThree}, och därefter~\textSymb{vpadBW} eller~\textSymb{vpadFW}}
{\setups[VpadFramedFigure]
\VpadScreenConfig{
\VpadFoot{\VpadPictures{vpadReturn}{vpadBW}{vpadFW}{vpadTrash}}}%
\framed{\bTABLE
\bTR\bTD \VpadOp{121} \eTD\eTR
\bTR\bTD \VpadOp{123}\hfill 522,0\,h \eTD\eTR
\bTR\bTD \VpadOp{125}\hfill 662,8\,h \eTD\eTR
\bTR\bTD \VpadOp{126}\hfill 1605,5\,km \eTD\eTR
\bTR\bTD \VpadOp{124}\hfill 2608,4\,km \eTD\eTR
\bTR\bTD \VpadOp{127}\hfill 2,0\,l/h \eTD\eTR
\eTABLE}}{Återställ mätaren med \textSymb{vpadTrash}}
{\setups[VpadFramedFigure]% \VpadFramedFigureK pour bande noire
\VpadScreenConfig{
\VpadFoot{\VpadPictures{vpadReturn}{vpadTrash}{vpadNull}{vpadNull}}}%
\framed{\bTABLE
\bTR\bTD \VpadOp{121} \eTD\eTR
\bTR\bTD \null \eTD\eTR
\bTR\bTD \VpadOp{136} \eTD\eTR
\bTR\bTD \null \eTD\eTR
\bTR\bTD \VpadOp{137} \eTD\eTR
\eTABLE}}{Bekräfta med \textSymb{vpadTrash}}
\stopcombination
\stop
\blank [1*big]

\startSteps [continue]
\item Ange lösenordet (om tillämpligt) och bekräfta därefter återställningen med symbolen \textSymb{vpadTrash}.
\item Tryck på symbolen \textSymb{vpadReturn} för att gå tillbaka till huvudbildskärmen.
\stopSteps

\stopsection

\page [yes]

\startsection[title={Underhållsintervaller},
reference={vpad:maintenance}]

I underhållsschemat till \sdeux\ skiljer man mellan två olika grundtyper av underhåll: normalt underhåll och omfattande service (genomförs av en fackverkstad som utsetts efter samråd med \boschung-Kundendienst ).

Mätarna kan avläsas eller återställas på följande sätt:
\startSteps
\item Tryck mitt på pekskärmen för att bläddra genom menyraden.
\item Tryck på symbolen \textSymb{vpadServiceInfo} för att visa underhållsintervallerna.
\item Växla till önskat intervall med hjälp av pilsymbolerna ~\textSymb{vpadUp}\textSymb{vpadDown}.
\item Tryck på symbolen ~\textSymb{vpadTrash} för att återställa ett intervall. Ange lösenordet med hjälp
av~\textSymb{vpadPlus}\textSymb{vpadMinus} och bekräfta med~\textSymb{vpadSelect}.
\item I ett dialogfönster uppmanas du att bekräfta återställningen.
\stopSteps
\blank [1*big]

\start
\setupcombinations[width=\textwidth]
\startcombination [3*1]
{\setups[VpadFramedFigure]% \VpadFramedFigureK pour bande noire
\VpadScreenConfig{
\VpadFoot{\VpadPictures{vpadReturn}{vpadNull}{vpadNull}{vpadTrash}}}%
\framed{\bTABLE
\bTR\bTD[nc=2] \VpadOp{190} \eTD\eTR
\bTR\bTD \VpadOp{191}\eTD\bTD \VpadOp{195}\hfill 600\,h \eTD\eTR % [backgroundcolor=black,color=TableWhite]
\bTR\bTD \VpadOp{192}\eTD\bTD \VpadOp{195}\hfill 600\,h \eTD\eTR
\bTR\bTD \VpadOp{193}\eTD\bTD \VpadOp{195}\hfill 2400\,h \eTD\eTR
\eTABLE}}{Tryck på symbolen ~\textSymb{vpadTrash} för att återställa ett intervall
.}
{\setups[VpadFramedFigure]
\VpadScreenConfig{
\VpadFoot{\VpadPictures{vpadReturn}{vpadMinus}{vpadPlus}{vpadSelect}}}%
\framed{\bTABLE
\bTR\bTD \VpadOp{190} \eTD\eTR
\bTR\bTD \hfill 2014-03-31 \eTD\eTR
\bTR\bTD \null \eTD\eTR
\bTR\bTD \null \eTD\eTR
\bTR\bTD \null \eTD\eTR
\bTR\bTD \null \eTD\eTR
\bTR\bTD \VpadOp{002}\hfill 0000 \eTD\eTR
\eTABLE}}{Ange lösenordet (sifferkod)}
{\setups[VpadFramedFigure]% \VpadFramedFigureK pour bande noire
\VpadScreenConfig{
\VpadFoot{\VpadPictures{vpadReturn}{vpadUp}{vpadDown}{vpadSelect}}}%
\framed{\bTABLE
\bTR\bTD \VpadOp{190} \eTD\eTR
\bTR\bTD[backgroundcolor=black,color=TableWhite] \VpadOp{041}\eTD\eTR % [backgroundcolor=black,color=TableWhite]
\bTR\bTD \VpadOp{042} \eTD\eTR
\bTR\bTD \VpadOp{043} \eTD\eTR
\eTABLE}}{Välj och bekräfta med~\textSymb{vpadSelect}}
\stopcombination
\stop
\blank [1*big]

\startSteps [continue]
\item Bekräfta återställningen med hjälp av symbolen~\textSymb{vpadSelect}.
\item Tryck på symbolen \textSymb{vpadReturn} för att gå tillbaka till huvudbildskärmen.
\stopSteps

\stopsection

\page [yes]


\startsection[title={Felhantering via Vpad},
reference={vpad:error}]


På \Vpad\ visas fel\index{Vpad+Felmeddelanden} som diagnosticeras av de elektroniska styrsystemen och överförs av vomCAN-bussen.
Om ett fel är mindre allvarligt lyser symbolen~\textSymb{VpadTClear} (röd).
Om det handlar om ett fel med hög prioritet lyser symbolen ~\textSymb{VpadTClear} och det ljuder en varningssignal.
För att stänga av larmet ska du kvittera felmeddelandet (bekräftas som \aW{noterat}).

Felmeddelanden kan läsas och kvitteras på följande sätt:

\startSteps
\item Tryck på symbolen ~\textSymb{vpadClear} på bildskärmen \Vpad.
\item Tryck på symbolen~\textSymb{vpadClear} för att kvittera det valda meddelandet.
\item Intill det kvitterade meddelandet visas nu en \aW{\#}-symbol som markerar meddelandet som \aW{noterat} och markeringen hoppar till nästa meddelande (om tillämpligt).
\item När meddelandet har kvitterats växlar visningen till huvudbildskärmen.
\stopSteps
\blank [1*big]

\start
\setupcombinations[width=\textwidth]
\startcombination [3*1]
{\setups[VpadFramedFigure]% \VpadFramedFigureK pour bande noire
\VpadScreenConfig{
\VpadFoot{\VpadPictures{vpadReturn}{vpadUp}{vpadDown}{vpadSelect}}}%
\framed{\bTABLE
\bTR\bTD \VpadEr{000} \eTD\eTR
\bTR\bTD [backgroundcolor=black,color=TableWhite] \VpadEr{851a} \eTD\eTR
\bTR\bTD \VpadEr{902} \eTD\eTR
\eTABLE}}{Visning av meddelanden}
{\setups[VpadFramedFigure]
\VpadScreenConfig{
\VpadFoot{\VpadPictures{vpadReturn}{vpadUp}{vpadDown}{vpadSelect}}}%
\framed{\bTABLE
\bTR\bTD \VpadEr{000} \eTD\eTR
\bTR\bTD [backgroundcolor=black,color=TableWhite] \VpadEr{851} \eTD\eTR
\bTR\bTD \VpadEr{902} \eTD\eTR
\eTABLE}}{Kvittera med~\textSymb{vpadClear}}
{\setups[VpadFramedFigureHome]% \VpadFramedFigureK pour bande noire
\VpadScreenConfig{
\VpadFoot{\VpadPictures{vpadClear}{vpadBeacon}{vpadBeam}{vpadEngine}}}%
\framed{\null}}{Tillbaka till huvudbildskärmen}
\stopcombination
\stop
\blank [1*big]

\startSteps [continue]
\item För att visa meddelandena igen ska du trycka på symbolen~\textSymb{vpadClear}. Felmeddelandena raderas från \Vpad\ först när orsaken har åtgärdats.
\stopSteps


\subsection{De vanligaste felmeddelandena (med felsökning)}

\subsubsubject{\VpadEr{604}} % {\#\ 604 Pression huile moteur basse}

+ \textSymb{vpadTEnginOilPressure}~– stäng genast av motorn. Kontrollera oljenivån, kontakta en fackverkstad.


\subsubsubject{\VpadEr{609}} % {\#\ 609 Température eau refroidissement moteur haute}

+ \textSymb{vpadSyWaterTemp}~– stoppa arbetet. Låt motorn vara igång utan belastning och kontrollera hur temperaturen utvecklas.

Om temperaturen sjunker ska du kontrollera nivån på kylvätska, motorolja och hydraulvätska samt kylarens skick.
Om nivåerna är korrekta och kylaren fungerar som den ska, ska du köra försiktigt till din verkstad för att genomföra en feldiagnos.

\subsubsubject{\VpadEr{610}} % {\#\ 610 Température eau refroidissement moteur trop haute}

+ \textSymb{vpadSyWaterTemp}~– stoppa arbetet. Kontrollera nivån på kylvätska och motorolja, kontakta en verkstad omedelbart.


\subsubsubject{\VpadEr{650}} % {\#\ 650 Se rendre à un garage}

+ \textSymb{vpadWarningService}~– kontakta din verkstad omedelbart.
% \VpadEr{651} % {\#\ 651 Moteur en mode urgence}


\subsubsubject{\VpadEr{652}} % {\#\ 652 Inspection véhicule}
% \VpadEr{653} % {\#\ 653 Grand service moteur}

+ \textSymb{vpadWarningService}~– det är dags för nästa normala underhåll. Titta i underhållsschemat och boka en tid hos din verkstad.


\subsubsubject{\VpadEr{700}} % {\#\ 700 Température d'huile hydraulique}

+ \textSymb{vpadSyWaterTemp}~– stoppa arbetet. Låt motorn vara igång utan belastning och kontrollera hur temperaturen utvecklas.

Om temperaturen sjunker ska du kontrollera nivån på kylvätska, motorolja och hydraulvätska samt kylarens skick.
Om nivåerna är korrekta och kylaren fungerar som den ska, ska du köra försiktigt till din verkstad för att genomföra en feldiagnos.


\subsubsubject{\VpadEr{702}} % {\#\ 702 Filtre d'huile hydraulique}

+ \textSymb{vpadWarningFilter}~– returledningsfiltret för hydraulolja och/eller insugsfiltret är tilltäppt. Byt genast ut filterelementet.
% \VpadEr{703} % {\#\ 703 Vidange d'huile hydraulique}


\subsubsubject{\VpadEr{800}} % {\#\ 800 Interrupteur d'urgence actionné}

+ \textSymb{vpadTClear}~– du har tryckt på nödstoppsknappen. Slå från tändningen och starta om motorn för att radera meddelandet.


\subsubsubject{\VpadEr{801}} % {\#\ 905 Frein à main actionné}

Smutsbehållaren befinner sig i upplyft eller ej helt nedsänkt läge. Fordonets hastighet är begränsad till 5\,km/h när smutsbehållaren inte är nedsänkt.

\subsubsubject{\VpadEr{851}} % {\#\ 851 Filtre à air}

+ \textSymb{vpadWarningFilter}~– luftfiltret är tilltäppt. Byt genast ut filterelementet.


\subsubsubject{\VpadEr{902}} % {\#\ 902 Pression de freinage}

+ \textSymb{vpadTBrakeError}~– för lågt bromstryck. Stoppa arbetet och kontakta en fackverkstad omedelbart.
% \VpadEr{904} % {\#\ 904 Interrupteur de direction d'avancement}


\subsubsubject{\VpadEr{905}} % {\#\ 905 Frein à main actionné}

+ \textSymb{vpadTBrakePark}~– handbromsen har inte lossats helt. Fordonets hastighet är begränsad till 5\,km/h när handbromsen inte är lossad.


\stopsection

\stopchapter

\stopcomponent














