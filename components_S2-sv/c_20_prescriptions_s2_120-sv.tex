\startcomponent c_20_prescriptions_s2_120-sv
\product prd_ba_s2_120-sv


\chapter [safety:risques] {Säkerhetsanvisningar}

\setups [pagestyle:marginless]


\section{Grundläggande anvisningar}

\subsubject{Rättsgrunder}

Olyckor kan ha allvarliga konsekvenser för både arbetsgivare och anställda. I det följande beskrivs de båda parternas skyldigheter.\note[prescription:user:right].

Arbetsgivaren är skyldig att beakta följande punkter innan en anställd får i uppdrag att arbeta med maskinen:

\startSteps
\item Alla förare måste ha lämplig utbildning för att kunna framföra fordonet. Utbildningen måste dokumenteras med ett intyg.
\item Alla förare måste ha ett formellt förartillstånd. Detta får utfärdas endast om följande villkor är uppfyllda:
\startitemize [2]
\item Den anställde har genomgått ett medicinskt lämplighetstest hos företagsläkaren.
\item Den anställde är förtrogen med platsen där arbetena ska utföras och känner till alla säkerhetsföreskrifter som gäller för den plats där fordonet ska användas.
\item Den anställde ska ha klarat ett lämplighetstest, som intygar lämpligas kunskaper för framförandet av fordonet.
\stopitemize
\stopSteps

Om fordonets maxhastighet överstiger 25 km/h\note[prescription:user:right] måste fordonet vara officiellt godkänt och fordonsföraren måste inneha följande körtillstånd:
\startitemize
\item Körkort klass B\note[prescription:lisence] för fordon med en tillåten totalvikt på mindre än 3,5 ton. \item Körkort klass C\note[prescription:lisence] för fordon med en tillåten totalvikt på mer än 3,5 ton.
\stopitemize

Om fordonets maxhastighet är 25 km/h måste fordonsföraren som minimikrav känna till de trafikregler som gäller på allmänna vägar, även om B-körkort inte är nödvändigt för framförande\note[prescription:user:right] av fordonet.

\footnotetext [prescription:user:right] {Arbetsgivarens och personalens skyldigheter kan skilja sig åt mellan olika länder. Ta reda på vilka bestämmelser som gäller för ditt land eller din region.}

\footnotetext[prescription:lisence] {Europaparlamentets och rådets direktiv 2006/126/EG av den 20 december 2006 om körkort.}


\subsubject{Villkor för användning}

\sdeux\ får endast användas om den befinner sig i felfritt skick. Föraren måste dessutom beakta säkerhetsanvisningarna och föreskrifterna i den här bruksanvisningen. Funktionsfel som påverkar säkerheten negativt måste omedelbart åtgärdas/repareras av en fackverkstad.
\blank [big]

\startSymList
\externalfigure [s2_inspection] [width=4.5em]
\SymList
{\md Dagligt underhåll:}
Inspektera fordonet efter varje användning och reparera synliga skador och defekter. Kontakta omedelbart en fackverkstad om du upptäcker skador eller defekter på fordonet. Om detta inte är möjligt ska du omedelbart stanna fordonet och spärra av platsen där det gick sönder.
\stopSymList


\subsubject{Avsedd användning}

\sdeux\ används för rengörings- och underhållsarbeten på vägar, gator och torg. Fordonet får inte användas för andra ändamål. Företaget \boschung\ ansvarar inte för skador som uppstår till följd av att fordonet använts för ej avsedda ändamål. Endast användaren ansvarar för denna typ av skador. {\em För att kunna använda fordonet på avsett sätt ska du även beakta säkerhetsanvisningarna och underhållsschemat i den här bruksanvisningen.}


\section{Körning på allmänna vägar}

\subsubject{Allmänna bestämmelser}

Utöver denna bruksanvisning ska du även beakta alla gällande regler, rättsliga föreskrifter och andra bestämmelser för olycksförebyggande och miljöskydd.


\subsubject{Passagerarplats}

På {\em passagerarstolen} finns plats för en passagerare.


\subsubject{Säkerhetsbälte}

\startSymList
% \externalfigure [prescription:safety:belt]
\PMbelt
\SymList
Föraren och passageraren i \sdeux\ måste ta på sig säkerhetsbältet enligt gällande trafikregler när de tar plats i fordonet.
\stopSymList


\subsubject{Synlighet}

\startSymList
\externalfigure [travaux_deviation] [width=3.5em]
\SymList
Se till att du alltid är väl synlig, särskilt vid körning på vältrafikerade gator.

Om fordonsföraren inte har tillräcklig sikt vid en körmanöver eller en viss arbetsuppgift måste ytterligare en person som har ögonkontakt med föraren hjälpa till.
\stopSymList


\subsubject{Belysning och signalanordningar}

Beroende på gällande trafikregler kan det vara nödvändigt att slå på strålkastarna och/eller det roterande varningsljuset även på dagen.


\subsubject{Användning av mobiltelefoner}

\startSymList
\PPphone
\SymList
Det är inte tillåtet att använda mobiltelefon vid körning på allmänna vägar, såvida inte fordonet är utrustat med en handsfreeanordning.

Att prata i telefon\index{Säkerhet+Mobiltelefon} vid ratten~– med eller utan handsfreeanordning~– avleder förarens uppmärksamhet från trafiken.
\stopSymList


\section{Underhållsanvisningar}

\subsubject{Underhållsanvisningar}

Innan underhållsarbeten påbörjas ska underhållspersonalen läsa igenom bruksanvisningen till\sdeux, särskilt avsnitten om säkerhet och underhåll.


\subsubject{Nödvändiga kvalifikationer}

\startSymList
\externalfigure [mecanicienne] [width=3.5em]
\SymList
Endast personer som genomgått lämplig utbildning får genomföra underhållsarbeten på \sdeux\. Detta gäller i synnerhet för arbeten på motorn, bromssystemet, styrningen samt el- och hydraulsystemet.
\stopSymList


\testpage [6]
\subsubject{Uppsikt}

\startSymList
\externalfigure [mecanicien_hyerarchie] [width=3.5em]
\SymList
Personer som inte är färdigutbildade~(praktikanter, lärlingar)~får endast arbeta på fordonet om de hålls under uppsikt av utbildad fackpersonal. Genomför stickprover för att kontrollera att personalen följer bruksanvisningen och säkerhetsanvisningarna.
\stopSymList


\subsubject{Svetsarbeten}

\startSymList
\externalfigure [pince_soudure2] [width=3.5em]
\SymList
Innan du påbörjar svetsarbeten på karossen eller chassit ska du alltid koppla från batteriet och alla elektroniska styrenheter.
\stopSymList

\subsubject{Rengöring av fordonet}

\startSymList
\externalfigure [washer_pressure] [width=3.5em]
\SymList
Innan du rengör \sdeux\ ska du läsa igenom avsnitt \about[sec:cleaning] fr.o.m. \atpage[sec:cleaning], särskilt rengöringsanvisningarna.
\stopSymList


\subsubject{Fordonsdokumentationens tillgänglighet}

\startSymList
\externalfigure [lecteur_1] [width=3.5em]%\PMrtfm
\SymList
Fordonsdokumentationen ska alltid förvaras på en lättåtkomlig plats i förarhytten under arbetena.
\stopSymList


\section{Särskilda anvisningar för användning}

\subsubject{Fordonshöjd}

\startSymList
\PPmaxheight
\SymList
Vid arbeten/körning i ej öppna områden (garage, undergångar, strömledningar etc.) är det viktigt att se till att genomgångshöjden är tillräcklig för \sdeux\ (se \in{avsnitt}[sec:measurement], \atpage[sec:measurement]).
\stopSymList


\subsubject{Fordonets stabilitet}

Undvik alla manövrar som kan påverka fordonets stabilitet negativt. Den smala konstruktionen och den höga tyngdpunkten på \sdeux\ kan göra att fordonet välter vid körning i kurvor i hög hastighet om smutsbehållaren är full.


\subsubject{Ofrivilliga fordonsrörelser}

När du lämnar fordonet ska du säkra det så att det inte kan användas av obehöriga. Dra alltid åt handbromsen innan du lämnar fordonet. Vid behov kan du säkra hjulen med kilar.

\startbuffer [prescription:handbrake]
\starttextbackground [CB]
\startPictPar
\PPstop
\PictPar
{\md Dra åt handbromsen ordentligt!} Annars kan fordonet sättas i rörelse\index{Handbroms+Riskpotential} även vid knappt märkbara lutningar. Detta kan leda till olyckor och orsaka allvarliga skador på andra personer.

{\lt Det hydrostatiska framdrivningssystemet reducerar stegvis trycket i hydraulkretsen när fordonet står stilla, vilket leder till att motorns hållkraft reduceras. Därför är det särskilt viktigt att alltid dra åt handbromsen ordentligt när man lämnar fordonet.}
\stopPictPar
\stoptextbackground

\stopbuffer

\getbuffer [prescription:handbrake]


\testpage [6]
\subsubject{Smutsbehållare}

\startbuffer [prescription:container:gravity]
\starttextbackground [CB]
\startPictPar
\PHgravite
\PictPar
{\md Olycksrisk:}
{\lt När smutsbehållaren tippas upp förflyttas tyngpunkten uppåt. Detta ökar risken för att fordonet välter. Se till att fordonet står på ett fast och jämnt underlag när smutsbehållaren tippas.}
\stopPictPar
\stoptextbackground

\stopbuffer

\getbuffer [prescription:container:gravity]


\startbuffer [prescription:container:tilt]
\starttextbackground [CB]
\startPictPar
\PHcrushing
\PictPar
{\md Olycksrisk:}
{\lt Fäst alltid säkerhetstöttorna på smutsbehållarens hydrauliska lyftcylindrar innan du genomför arbeten under behållaren.}
\stopPictPar
\stoptextbackground

\stopbuffer

\getbuffer [prescription:container:tilt]


\stopcomponent

