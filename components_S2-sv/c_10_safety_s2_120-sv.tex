\startcomponent c_10_safety_s2_120-sv
\product prd_ba_s2_120-sv

\marking[chapter]{Säkerhetsymboler}


\chapter{Säkerhetssymboler}

\setups[pagestyle:marginless]

\section{Ny europeisk märkning av farliga ämnen}

{\em Fyrkantig med vit bakgrund och röd
kant.}\par\blank[1*medium]
{\em Sedan 2008 gäller den så kallade CLP-förordningen\index{CLP-fördordning} inom EU, som föreskriver ny varningsmärkning för farliga ämnen och produkter.}\par\null

\startSymList \GHSgeneric
\SymList
\textDescrHead{Hälsorisk}
Varnar för\index{Hälsorisk} hälsorisker som inte leder till livshotande eller allvarliga skador. Till dessa hör bland annat hudirritation eller framkallande av allergi. Symbolen används även som varning för andra faror, till exempel antändlighet.\par
Ersätter:\crlf \HAZOcross\ eller \HAZOpoison\ eller\PHgeneric
\stopSymList

\startSymList \GHSbody
\SymList
\textDescrHead{Allvarlig hälsorisk; kan leda till döden, särskilt hos barn}
Produkter med denna märkning kan orsaka allvarliga hälsoskador. Denna symbol varnar även för \index{Fara+graviditet} fosterskador, cancerframkallade effekter\index{Fara+cancerframkallande ämnen} och liknande allvarliga hälsorisker. Produkterna ska användas försiktigt.\par
Ersätter:\crlf \HAZOcross\ eller \HAZOpoison\
\stopSymList

\startSymList \GHSbomb
\SymList
\textDescrHead{Explosiva ämnen}
Instabila explosiva\index{Fara+explosion} ämnen,
blandningar och produkter som innehåller explosiva ämnen\index{Explosiva ämnen} expanderar kraftigt vid en explosion och kan orsaka allvarliga skador. Felaktig hantering kan leda till livsfara.\par
Ersätter:\crlf \HAZObomb\
\stopSymList


\startSymList \GHSpoison
\SymList
\textDescrHead{Förgiftning}
Produkter med denna märkning\index{Fara+Förgiftning} kan till och med i små mängder på huden, genom inandning\index{Giftiga ämnen} eller förtäring orsaka allvarlig eller livshotande förgiftning. Undvik direkt kontakt.\par
Ersätter:\crlf \HAZOpoison\
\stopSymList

\startSymList \GHSfire
\SymList
\textDescrHead{Brandfarligt eller synnerligen brandfarligt}
Produkter med denna märkning\index{Fara+brand} antänds snabbt i närheten
av värme eller flamma. Sprejer med denna märkning får aldrig sprutas på varma ytor eller i närheten av öppen låga.\par
Ersätter:\crlf \HAZOfire\ eller \HAZOfirebis\
\stopSymList

\startSymList \GHSenvironment
\SymList
\textDescrHead{Fara för djur och miljö}
Produkter med denna märkning\index{Miljöskydd} kan orsaka
kort- och långfristiga\index{Giftiga ämnen} miljöskador. De kan döda vattenlevande organismer (t.ex. fiskar) och även ha långvariga skadliga effekter på miljön. Dessa produkter får inte hamna i avloppet eller slängas i hushållssoporna!\par
Ersätter:\crlf \HAZOenvironment\
\stopSymList

\startSymList \GHScorrosive
\SymList
\textDescrHead{Fara för hud och ögon}
Produkter med denna märkning\index{Fara+hudskada}\index{Fara+ögonskador} kan orsaka hudskador eller ärrbildning redan efter kortvarig kontakt med huden, eller orsaka kroniska ögonskador. Skydda hud och ögon vid andvändning!\par
Ersätter:\crlf \HAZOcross\ eller \HAZOcorrosive
\stopSymList

\page [yes]


\section{Varningssymboler}

{\em Svart text på gul bakgrund}\par\null

\startSymList \PHgeneric
\SymList
\textDescrHead{Allmän varningssymbol}
Uppmärksammar\index{Fara +allmän}\index{varningssymbol} på en omedelbar fara som kan leda till personskador.
\crlf\null
\stopSymList

\startSymList \PHpoison
\SymList
\textDescrHead{Varning för giftiga ämnen}
Giftiga ämnen\index{Fara+förgiftning} kan leda till allvarliga akuta eller kroniska hälsoskador eller till och med till dödsfall om de kommer i kontakt med huden, andas in eller förtärs.
\stopSymList

\startSymList \PHfire
\SymList
\textDescrHead{Varning för brandfarliga ämnen}
Undvik öppen låga eller gnistbildning\index{Fara+brand}. Materialet är lättantändligt och kan leda till brandspridning. Rök ej!
\stopSymList

\startSymList \PHexplosive
\SymList
\textDescrHead{Varning för explosiva ämnen}
Fasta, flytande eller geléaktiga ämnen eller beredningar som kan explodera om de utsätts för stötar, friktion, brand, värme eller dylikt.\index{Fara+Explosion} Rök ej!
\stopSymList

\startSymList \PHcrushing
\SymList
\textDescrHead{Klämrisk}
Varnar för områden\index{Fara+klämskador} där det finns risk för klämskador på rörliga maskindelar. Håll ett säkert avstånd från sådana områden när maskinen är igång.
\stopSymList

\startSymList \PHhand
\SymList
\textDescrHead{Varning för handskador}
Risk för att\index{Fara+klämskador} händer eller andra kroppsdelar\index{Fara+handskador} hamnar i kläm \eG\ när förarhytten eller lastrampen tippas.
\stopSymList

\startSymList \PHentangle
\SymList
\textDescrHead{Varning för rullar som rör sig i varandras motsatta riktning/risk för indragning}
Risk för att kroppsdelar\index{Fara+indragning} fastnar eller dras in i roterande delar. Håll ett säkert avstånd från dessa delar när de arbetar.
\stopSymList

\startSymList \PHcorrosive
\SymList
\textDescrHead{Varning för frätande ämnen}
Hantera dessa ämnen försiktigt\index{Fara+frätande ämnen} och bär lämplig personlig skyddsutrustning (handskar, skyddsglasögon, skyddskläder).
\stopSymList

\startSymList \PHhot
\SymList
\textDescrHead{Varning för varm yta}
Närma dig inte maskindelen eller anordningen\index{Fara+brännskador} om du inte har tillräckliga kunskaper. Bär handskar.
\stopSymList

\startSymList \PHvoltage
\SymList
\textDescrHead{Varning för farlig elektrisk spänning}
Undvik kontakt\index{Fara+elektrisk spänning} med metallföremål.
Risk för person- och brännskador vid kortslutning!
\stopSymList

\startSymList \PHfalling
\SymList
\textDescrHead{Fallrisk}
Var särskilt försiktig när du rör\index{Fara+Fall} dig i detta område och bär lämpliga skor (med halkfri sula, beständiga mot kolväte).
\stopSymList

\startSymList \PHbattery
\SymList
\textDescrHead{Varning för faror som batterier medför} Uppmärksammar på faror som uppstår vid laddning av batterier (blybatterier)\index{Fara+Batteri}, i synnerhet läckande vätgas eller svavelsyra som finns i batterierna.
\stopSymList

\startSymList \PHremote
\SymList
\textDescrHead{Varning för automatisk start}
Varnar\index{Fara+automatisk start} för att en anordning kan starta automatiskt eller fjärrstyrt.
\stopSymList

% \startSymList \PHquetschgefahr
% \SymList
% \textDescrHead{Risque d’écrasement}
% Risque d’écrasement\index{risque d’écrasement}.
% \stopSymList
% % NOTE: Doppelt! (auch Bilddatei)
%
% % NOTE: Evtl. Folgendes als Ersatz für oben?

% \startSymList\PHhandcrushed
% \SymList
% \textDescrHead{Gefahr von Handquetschungen}
% Es besteht\index{Gefahr+Quetschung} die Gefahr, dass Hände oder Finger
% gequetscht werden. Nähern Sie die Hände nicht an, ohne die Gefahr
% identifiziert und beseitigt zu haben.
% \stopSymList

\startSymList \PHhandfoot
\SymList
\textDescrHead{Varning för rörliga komponenter}
Varnar för maskin-/fordonsdelar som är i rörelse.
\index{Fara+Rörliga delar}.
\stopSymList

\startSymList \PHnarrowed
\SymList
\textDescrHead{Varning för avsmalnande körbara}
Avsmalnande\index{Fara+fordonsbredd} körbana.
% Denken Sie an die Breite des Fahrzeugs.
\stopSymList

\page [yes]


\section{Förbudssymboler}

{\em Rund med vit bakgrund, röd kant och tvärsgående streck}
\par\null


\startSymList \PPfire
\SymList
\textDescrHead{Eld, öppen låda och rökning är förbjudet} Öppen
låga\index{Förbud+rökning, eld} och glöd i alla former är förbjudet (\eG\
brinnande cigaretter, tändstickor, levande ljus, alla typer av gnistbildning).
\stopSymList

\startSymList \PPentry
\SymList
\textDescrHead{Tillträde förbjudet för obehöriga}
Obehöriga\index{Tillträde+förbjudet} personer får inte beträda eller komma i närheten av detta område.
\stopSymList

\startSymList \PPphone
\SymList
\textDescrHead{Mobil kommunikation förbjudet}
Mobiltelefoner\index{Förbud+mobil kommunikation} och alla typer av enheter som avger elektromagnetisk strålning måste vara frånslagna. Elektromagnetisk strålning kan orsaka störningar i enheternas elektroniska system.
\stopSymList

\startSymList \PPspray
\SymList
\textDescrHead{Undvik vattenstänk}
Rikta aldrig en vatten- eller ångstråle \index{Förbud+vattenstråle, ånga} mot känsliga delar eller anordningar (t.ex. sensorer, styrenheter, insprutningssystem och liknande).
\stopSymList

\startSymList \PPchildren
\SymList
\textDescrHead{Håll barn borta}
Uppmärksammar\index{Förbud+barn} på en särskild fara för barn. Tumregel: Barn får inte uppehålla sig i närheten av maskinen när den är igång eller när underhållsarbeten genomförs på den.
\stopSymList

\startSymList \PPwater
\SymList
\textDescrHead{Ej dricksvatten}
Vattnet i tanken\index{Förbud+Ej dricksvatten} är inte drickbart. Risk för förgiftning.
\stopSymList

% \page [yes]


\section{Miljödskyddssymboler}

\startSymList \PSrecycle
\SymList
\textDescrHead{Återvinning}
Särskilda bestämmelser för korrekt återvinning av visst avfall.
\stopSymList

\startSymList \PSwelt
\SymList
\textDescrHead{Miljöskydd}
Hänvisning till gällande miljöskyddsregler.
\stopSymList

\startSymList \PStrash[width=\PictoHeight,height=,]
\SymList
\textDescrHead{Bortskaffa avfall korrekt}
För vissa typer av avfall och \eG\ blybatterier gäller särskilda regler
för avfallshantering.
\stopSymList


\testpage[12]


\section{Påbudssymboler}


{\em Rund med blå bakgrund}\par\null

\startSymList \PMgeneric
\SymList
\textDescrHead{Allmän påbudssymbol}
Denna symbol får endast användas tillsammans med en tilläggssymbol som preciserar påbudet.
\stopSymList


\startSymList \PMrtfm
\SymList
\textDescrHead{Beakta bruksanvisningen}
Läs alltid igenom\index{Läs igenom bruksanvisningen} anvisningarna om detta ämne, en specifik enhet eller produkt. Bruksanvisningen ska alltid förvaras på en lättåtkomlig plats i förarhytten.
\stopSymList

\startSymList \PMproteyes
\SymList
\textDescrHead{Använd ögonskydd}
Bär alltid ögonskydd när du genomför arbeten där det finns risk för ögonskador\index{Ögonskydd}.
\stopSymList

\startSymList \PMprothands
\SymList
\textDescrHead{Använd handskydd}
Bär alltid skyddshandskar när du genomför arbeten där det finns risk för handskador\index{Använd handskar}.
\stopSymList

\startSymList \PMprotears
\SymList
\textDescrHead{Använd hörselskydd}
Bär hörselskydd\index{Fara+Hörsel} (i närheten av t.ex. fläktar eller turbiner som är igång).
\stopSymList

\startSymList \PMsafetybelt
\SymList
\textDescrHead{Använd säkerhetsbälte} Spänn fast\index{Säkerhetsbälte} säkerhetsbältet för din egen säkerhet.
\stopSymList

\section{Tilläggssymboler}

% \adaptlayout[height=+5mm]{{{

% \startSymList \SETshoe
% \SymList
% \textDescrHead{Port de chaussures de sécurité obligatoire}
% Le port de chaussures de sécurité est obligatoire\index{chaussures de sécurité}.
% \stopSymList
%
% \startSymList \SETglasses
% \SymList
% \textDescrHead{Port de lunettes des protection obligatoire}
% Le port de lunettes est obligatoire\index{lunette de protection}.
% \stopSymList
%
% \startSymList \SEToreillettes
% \SymList
% \textDescrHead{Port de casque obligatoire}
% Le port d’un casque de protection est \index{casque} obligatoire.
% \stopSymList
%
% \startSymList \SETgloves
% \SymList
% \textDescrHead{Port de gants de protection obligatoire}
% Le port de gants de protection est obligatoire\index{gants}.
% \stopSymList
%
% \startSymList \SETmainecrase
% \SymList
% \textDescrHead{Risque d’écrasement}
% Danger pour les mains\index{écrasement} et les pieds.
% \stopSymList
%
% \startSymList \SETgetriebe
% \SymList
% \textDescrHead{Risque de happement}
% Risque de happement par\index{happement} des pièces en rotation.
% \stopSymList
%
% \startSymList \SETradkeil
% \SymList
% \textDescrHead{Cale de roue}
% Sécuriser le véhicule contre toute mise\index{Cale de roue} en marche involontaire.
% \stopSymList
%}}}

\startSymList \SETfirstaid
\SymList
\textDescrHead{Första hjälpen}
Visar var första-hjälpen-utrustning finns. Vid ett nödfall är det mycket viktigt att räddningstjänst kontaktas snabbt.\index{Första hjälpen}\index{Nödnummer} Ange dina nödnummer här:
\fillinrules[n=1]{\bf
\framed[align=right,frame=off,offset=none,width=30mm]{Räddningstjänst}}
\fillinrules[n=1]{\bf
\framed[align=right,frame=off,offset=none,width=30mm]{Polis}}
\fillinrules[n=1]{\bf
\framed[align=right,frame=off,offset=none,width=30mm]{Brandkår}}
\stopSymList

\startSymList \SETbrandschutzzeichen
\SymList
\textDescrHead{Brandsläckare}
Vissa enheter är utrustade med en eller flera brandsläckare\index{Brandsläckare}. Dessa behöver i regel särskilt underhåll. Mer information hittar du på produkten eller i produktens bruksanvisning.
\stopSymList


\page[yes]

\section{Tre steg för första hjälpen}
% NOTE [tf]: Shouldn't be in this book, IMO

\starttextbackground [CB]
\textDescrHead{Säkra olycksplatsen och de drabbade personerna}
\startitemize
Kontrollera säkerheten på olycksplatsen och se till att det inte kan uppstå fler faror.
\stopitemize
\textDescrHead{Bedöm de skadades tillstånd}
\startitemize
Kontrollera om de skadade personerna är i medvetande och andas normalt.
Öppna luftvägarna om nödvändigt.
\stopitemize
\textDescrHead{Kontakta räddningspersonal}
\startitemize Uppge följande information vid ditt nödsamtal:\par
\item Det telefonnummer som du kan nås på.
\item Typen av incident (sjukdom, olycka).
\item Befintliga risker (brand, explosion, fallrisk, etc.).
\item Den exakta platsen för incidenten.
\item Antalet skadade och deras tillstånd.
\item Hjälpåtgärder som redan vidtagits.
\item Svara på ytterligare frågor som ställs till dig.
\stopitemize
\stoptextbackground

\stopcomponent


