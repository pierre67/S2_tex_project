
\startcomponent c_20_prescriptions_s2_095-es


\chapter [safety:risques] {Disposiciones de seguridad}

\setups [pagestyle:marginless]


\section{Indicaciones básicas}

\subsubject{Bases legales}

Los accidentes pueden tener graves consecuencias, tanto para la empresa como para el empleado. Vamos a recordar los deberes de ambas partes:\note[prescription:user:right].

La empresa está obligada a tener en cuenta los siguientes puntos antes de familiarizar al empleado en el manejo de la barredora:

\startSteps
\item Todos los conductores del vehículo deberán haber sido formados para conducir el vehículo. El certificado de la formación deberá estar disponible.
\item Todos los conductores del vehículo deberán disponer de un permiso de conducir formal. Este solo podrá ser emitido cuando se cumplan las tres condiciones siguientes:
\startitemize [2]
\item El empleado ha aprobado un chequeo médico realizado por el médico de la empresa.
\item El empleado conoce las particularidades del lugar de trabajo y está familiarizado con todas las disposiciones de seguridad del lugar de trabajo del vehículo, las cuales le han sido transmitidas por su superior.
\item El empleado ha aprobado un test de capacitación que certifica los conocimientos necesarios para conducir el vehículo.
\stopitemize
\stopSteps

Si la velocidad máxima del velocidad máxima es superior a los 25\,km/h\note[prescription:user:right], el vehículo deberá estar matriculado y el conductor del vehículo deberá tener los siguientes permiso de conducir:
\startitemize
\item Permiso de conducir de la clase B\note[prescription:lisence] para vehículos con un peso total autorizado inferior a 3,5~toneladas o
\item permiso de conducir de la clase C\note[prescription:lisence] para vehículos con un peso total autorizado superior a 3,5~toneladas.
\stopitemize

Cuando la velocidad máxima del vehículo es de 25\,km/h, el conductor del vehículo deberá conocer las normas de circulación vigentes para circular por vías públicas, incluso cuando para conducir el vehículo no sea necesario un permiso de conducir de la clase B\note[prescription:user:right].

\footnotetext [prescription:user:right] {Las obligaciones de la empresa y del empleado pueden variar de un país a otro. Familiarícese con las disposiciones vigentes en su país o región.}

\footnotetext[prescription:lisence] {Directiva 2006/126/CE del Parlamento Europea y del Consejo del 20 de~diciembre de 2006 sobre el permiso de conducción.}


\subsubject{Condiciones de uso}

La \sdeux\ únicamente puede utilizarse en un estado operativo perfecto. Además, el operario deberá cumplir las indicaciones de seguridad y las disposiciones contenidas en el presente manual de servicio. Las anomalías en el funcionamiento que perjudiquen la seguridad deberán ser eliminadas/reparadas inmediatamente por un taller técnico adecuado.
\blank [big]

\startSymList
\externalfigure [s2_inspection] [width=4.5em]
\SymList
{\md Mantenimiento diario:}
Después de utilizar el vehículo realice siempre una inspección y repare los daños y defectos visibles. En caso de daños o anomalías en el funcionamiento del vehículo informe inmediatamente al taller técnico. Si esto no fuera posible, detenga inmediatamente el vehículo y asegure el lugar de la avería.
\stopSymList


\subsubject{Uso previsto}

La \sdeux\ ha sido diseñada para realizar trabajo de limpieza y mantenimiento en calles, caminos y plazas. Cualquier otro uso diferente será considerado un uso no previsto. En consecuencia, la empresa \boschung\ declinará cualquier responsabilidad por los daños derivados de ello. Las consecuencias de un uso no previsto son responsabilidad única del operario. {\em El uso previsto comprende también el cumplimiento de las indicaciones de seguridad y del plan de mantenimiento contenidos ambos en el presente manual de servicio.}


\section{Conducción en vías públicas}

\subsubject{Disposiciones generales}

Además de las instrucciones de servicio deben respetarse todas las normas generales vigentes, las disposiciones legales u otras aplicables y las disposiciones sobre prevención de accidentes y protección medioambiental.


\subsubject{Asiento de acompañante}

En el asiento previsto~para tal fin, el denominado {\em asiento del acompañante}, podrá viajar un acompañante.


\subsubject{Cinturón de seguridad}

\startSymList
% \externalfigure [prescription:safety:belt]
\PMbelt
\SymList
El conductor y el acompañante de la \sdeux\ deberán utilizar el cinturón de seguridad, conforme a las normas de circulación aplicables, cuando tomen asiento en el vehículo.
\stopSymList


\subsubject{Ver y ser visto}

\startSymList
\externalfigure [travaux_deviation] [width=3.5em]
\SymList
Asegúrese de que es visto, especialmente en carreteras con mucho tráfico.

Si el conductor del vehículo no tiene visibilidad suficiente al realizar una maniobra o un trabajo determinado, deberá solicitar la ayuda de otra persona con la que mantiene contacto visual.
\stopSymList


\subsubject{Iluminación y señalización}

Dependiendo de las normas de circulación vigentes, deberán encenderse también durante el día las luces y/o las luces giratorias de aviso del vehículo.


\subsubject{Uso de teléfonos móviles}

\startSymList
\PPphone
\SymList
Está prohibido utilizar un teléfono móvil o un aparato de radiotransmisión durante el desplazamiento en vías públicas, a no ser que el vehículo esté equipado con un dispositivo de manos libres.

Hablar por teléfono\index{seguridad+teléfono móvil} durante la conducción, también con un dispositivo de manos libres, reduce en cualquier caso la concentración en el tráfico.
\stopSymList


\section{Disposiciones de mantenimiento}

\subsubject{Instrucciones de mantenimiento}

El personal de mantenimiento deberá haber leído el manual de servicio de la \sdeux antes de iniciar los trabajos, especialmente los apartados sobre seguridad y mantenimiento.


\subsubject{Cualificaciones necesarias}

\startSymList
\externalfigure [mecanicienne] [width=3.5em]
\SymList
Únicamente las personas que hayan obtenido los conocimientos necesarios en una formación adecuada están autorizadas a realizar trabajos de mantenimiento en la \sdeux. Esto se aplica especialmente a los trabajos en el motor, en el sistema de frenos, en la dirección y en la instalación eléctrica e hidráulica.
\stopSymList


\testpage [6]
\subsubject{Supervisión}

\startSymList
\externalfigure [mecanicien_hyerarchie] [width=3.5em]
\SymList
Las personas que se estén formando, en prácticas o aprendices, solo podrán realizar trabajos en el vehículo bajo la supervisión de un técnico. Compruebe de forma aleatoria si el personal cumple el manual de servicio y las disposiciones de seguridad.
\stopSymList


\subsubject{Trabajos de soldadura}

\startSymList
\externalfigure [pince_soudure2] [width=3.5em]
\SymList
Antes de realizar trabajos de soldadura en la carrocería o el chasis, la batería y todas las unidades de control electrónicas deberán ser desconectados.
\stopSymList

\subsubject{Limpiar el vehículo}

\startSymList
\externalfigure [washer_pressure] [width=3.5em]
\SymList
Antes de la limpieza de la \sdeux\ lea el apartado \about[sec:cleaning] a partir de la \atpage[sec:cleaning], especialmente el apartado sobre las normas de limpieza.
\stopSymList


\subsubject{Acceso a la documentación del vehículo}

\startSymList
\externalfigure [lecteur_1] [width=3.5em]%\PMrtfm
\SymList
Durante el uso conserve siempre la documentación del vehículo fácilmente accesible en la cabina del vehículo.
\stopSymList


\section{Disposiciones de uso especiales}

\subsubject{Altura del vehículo}

\startSymList
\PPmaxheight
\SymList
Durante los trabajos/desplazamientos en terrenos no públicos (aparcamientos subterráneos, pasadizos, cables de electricidad) asegúrese siempre de que la altura de paso es suficiente para la \sdeux\ (ver el \in{apartado}[sec:measurement], \atpage[sec:measurement]).
\stopSymList


\subsubject{Estabilidad del vehículo}

Evite cualquier maniobra que pueda perjudicar la estabilidad del vehículo. A una velocidad superior en curvas la \sdeux\ podría volcar debido a su estructura estrecha y a la altura de su centro de gravedad cuando el depósito de material barrido está lleno.


\subsubject{Movimiento accidental del vehículo}

Cuando abandone el vehículo, asegúrelo para evitar que personas no autorizadas puedan utilizarlo. Básicamente, ponga el freno de estacionamiento antes de abandonar el vehículo; asegure las ruedas con cuñas en caso necesario.

\startbuffer [prescription:handbrake]
\starttextbackground [CB]
\startPictPar
\PPstop
\PictPar
{\md ¡Ponga el freno de estacionamiento!} De lo contrario, el vehículo podría moverse accidentalmente, incluso\index{freno de estacionamiento+potencial de riesgo} en inclinaciones apenas perceptibles, y provocar un accidente con riesgo de producir lesiones mortales a terceros.

{\lt Mediante el sistema de accionamiento hidrostático se reduce gradualmente la presión en el circuito hidráulico cuando el vehículo está parado, lo que produce una reducción de la fuerza de retención del motor. Por este motivo es especialmente importante poner siempre el freno de estacionamiento al abandonar el vehículo.}
\stopPictPar
\stoptextbackground

\stopbuffer

\getbuffer [prescription:handbrake]


\testpage [6]
\subsubject{Depósito de material barrido}

\startbuffer [prescription:container:gravity]
\starttextbackground [CB]
\startPictPar
\PHgravite
\PictPar
{\md Riesgo de accidentes:}
{\lt Al levantar el depósito de material barrido el centro de gravedad se mueve hacia arriba. Esto aumenta el riesgo de volcar del vehículo. Por este motivo al volcar el depósito de material barrido preste atención a que el vehículo se encuentre sobre una superficie horizontal y rígida.}
\stopPictPar
\stoptextbackground

\stopbuffer

\getbuffer [prescription:container:gravity]


\startbuffer [prescription:container:tilt]
\starttextbackground [CB]
\startPictPar
\PHcrushing
\PictPar
{\md Riesgo de accidentes:}
{\lt No realice nunca trabajos debajo del depósito de material barrido antes de haber colocado puntales de seguridad en los cilindros hidráulicos de elevación del depósito de material barrido.}
\stopPictPar
\stoptextbackground

\stopbuffer

\getbuffer [prescription:container:tilt]


\stopcomponent

