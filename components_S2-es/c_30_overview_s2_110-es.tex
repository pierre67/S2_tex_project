
\startcomponent c_30_overview_s2_095-es

\startchapter [title={Vista general del vehículo}]

\setups [pagestyle:marginless]


\placefig [here] [] {Vista general de lado izquierdo del vehículo}
{\externalfigure [overview:side:left:es]}


\page [yes]


\placefig [here] [] {Vista general de lado derecho del vehículo}
{\externalfigure [overview:side:right:es]}

\page [yes]

\setups [pagestyle:normal]


\startsection [title={Aspectos generales}]

\placefig[margin][p4_vue_01]{\sdeux\ durante el traslado}
{%
\startcombination [1*3]
{\externalfigure[overview:vhc:01]}{}
{\externalfigure[overview:vhc:02]}{}
{\externalfigure[overview:vhc:03]}{}
\stopcombination}

Con la barredora \BosFull{sdeux} Boschung comparte con sus fieles clientes y socios toda la experiencia y competencia adquiridas durante una década de colaboración continua.
Los requisitos de municipios y empresas de servicio relativos a la movilidad y a la versatilidad han crecido enormemente durante este tiempo. Los desarrolladores de la \sdeux\ se han enfrentado a este reto basado en las necesidades de los clientes y exigido por las previsiones de propuestas de mejora del Servicio de Atención al Cliente de Boschung.
La \sdeux\ es el resultado de esta síntesis entre las necesidades del cliente y la aplicación consecuente de la experiencia obtenida en la práctica.


\subsection{Tecnología innovadora}

El vehículo compacto \BosFull{sdeux} destaca en su clase especialmente por su reducido peso (2300\,kg), su gran capacidad (depósito de material barrido de la categoría 2,0\,m\high{3}), unas dimensiones compactas (ancho 1,15\,m) y el puesto de trabajo especialmente ergonómico para el conductor del vehículo.

La \sdeux\ se convierte en la barredora \quotation{universal} para calles y aceras en ciudades y pueblos. Su potente motor diésel, en combinación con el accionamiento hidrostático compacto (motores hidráulicos de pistón radial en las ruedas delanteras) garantiza en todo momento la máxima movilidad, independientemente de las propiedades del lugar de utilización o del nivel de llenado del depósito de material barrido.

Las bombas hidráulicas están accionadas por un motor diésel tipo \aW{VW 2.0 CDI} conforme a la normativa Euro V. Suministra un par de giro de 285\,Nm a 1750~revoluciones y una potencia máxima de 75\,kW a 3000~revoluciones. Gracias a ello se puede emplear la máquina de forma efectiva incluso a una velocidad menor del motor, lo que conlleva una contaminación acústica menor~. La \sdeux\ viene equipada de serie con un filtro de partículas.

\stopsection


\startsection [title={Innovaciones al servicio del cliente}]

La dirección articulada de la \sdeux\ garantiza un radio de giro menor y, por tanto, máxima movilidad. Los materiales especiales como Domex® y el desarrollo completo basado en CAD del vehículo permiten una respetable carga útil de 1200\,kg.

\placefig[margin][overview:cab:frontright]{\sdeux\ lista para trabajar}
{\externalfigure[overview:cab:twoleft][width=\Bildwidth]}

La cabina con ventanas de cristal en todos los lados dispone de dos cómodos asientos, equipados con cinturones de seguridad de tres puntos. La \sdeux\ puede equiparse también de forma opcional con un equipo de aire acondicionado.

Con una velocidad máxima de 40\,km/h el vehículo se integra sin problemas en el tráfico. Gracias a la cómoda suspensión del eje delantero y el trasero, incluso los peores trayectos se pueden conducir con comodidad.

El equipo de barrido, montado en dos brazos articulados, está completo dentro del campo de visión del operario y la boca de aspiración se puede ver bien posicionada en el eje delantero. Un cepillo delantero de virado doble está disponible como equipamiento adicional.

\page [yes]


\subsection{Cabina insonorizada y confortable}

La cabina\index{cabina} de la \sdeux\ dispone de dirección a la derecha y ha sido diseñada para dos personas. Está insonorizada y montada sobre silentblocks que absorben vibraciones.

Las puertas y ventanas son de cristal, lo que ofrece un campo visual amplio. El parabrisas comprende toda la parte frontal del vehículo y permite una visión libre del trabajo de los cepillos.

El asiento del conductor tiene suspensión mecánica u, opcionalmente, neumática. Los asientos del conductor y del acompañante están montados sobre guías de deslizamiento regulables.


\subsubsubject{Ergonomía}

\startfigtext[right][overview:joy:sideview]{Panel de mando}
{\externalfigure[overview:joy:top]}
La consola multifuncional, a la izquierda del asiento del conductor, hace accesibles todas las funciones elementales con una mano. Los dos cepillos pueden controlarse individualmente mediante dos joysticks utilizando el pulgar y el dedo índice. Los interruptores para los cepillos y para el cepillo delantero (opción), para la velocidad del motor, el control de velocidad, etc. se encuentran también en la consola multifuncional.
\stopfigtext

En el borde inferior del campo visual de la cabina hay una pantalla táctil que muestra todas las funciones importantes de la máquina en tiempo real, sin estorbar al campo de visión hacia el exterior.

\placefig[margin][overview:vhc:left]{\sdeux\ delante de una muralla histórica}
% \placefig[margin][overview:vhc:left]{\sdeux\ sur site historique}
{\externalfigure[overview:vhc:left]}

\page [yes]


\subsubsubject{Cabina}

La\index{cabina} palanca de marchas (\quotation{cambio de marchas}) está a la derecha de la columna de dirección. Hay dos niveles de marcha, hacia atrás y hacia delante. En el exterior, en la palanca de marchas están el botón para cambiar los dos modos de trabajo \aW{Trabajo} y \aW{Marcha}. La \sdeux\ no necesita detenerse para la conmutación. (Más información al respecto en el capítulo \about[sec:using:work], \atpage[sec:using:work].)

\placefig[margin][fig:overview:steeringwheel]{Cabina}
{\externalfigure[overview:driver:place]}

En los desplazamientos hacia atrás se enciende la pantalla de la cámara de marcha atrás y suena una señal acústica (se desactiva en el Vpad).

La palanca multifuncional a la izquierda de la columna de dirección comprende el interruptor del limpiaparabrisas (dos niveles e intervalo), así como las luces y la bocina.

En el capítulo \about[chap:using] a partir de la \atpage[chap:using] encontrará más detalles sobre estas y otras funciones de la \sdeux.

\page [yes]

\setups[pagestyle:marginless]


\subsection[overview:brushsystem]{Dispositivo de barrido y de aspiración}

\subsubsubject{Cepillos}

\startfigtext[left][fig:overview:steeringwheel]{Dispositivo de barrido y de aspiración}
{\externalfigure[system:brush]}
Los cepillos\index{barrer} están en cabezales orientables que, a su vez, están montados sobre brazos articulados. El polvo levantado al barrer se aglutina rociando agua: Todos los cepillos están equipados con una tobera que extrae el agua del depósito de agua limpia o del agua reciclada.

Un interruptor\index{aspirar} de la consola multifuncional activa simultáneamente el cepillo y la bomba de agua.\footnote{Para la bomba de agua ver el capítulo \in[chap:using] \about[chap:using], especialmente \about[sec:using:work], \atpage[sec:using:work].}
Las posiciones de los cepillos, así como su inclinación transversal y longitudinal, pueden controlarse directamente mediante el joystick correspondiente de la consola multifuncional.
\stopfigtext

Los cepillos están protegidos con un sistema anticolisión mecánico e hidráulico.


\subsubsubject{Boca de aspiración}

En la posición de trabajo (abajo) la boca de aspiración reposa sobre 4~rodillos y cubre completamente la superficie entre los cepillos separados. Gracias a su posición \quotation{remolcada} está completamente protegida frente a daños mecánicos en caso de colisiones con obstáculos. Durante la marcha atrás, la boca de aspiración se eleva automáticamente.

Un labio de goma grueso y reemplazable se encarga del cierre hermético frente a la superficie de la carretera. Una tapa con control electrohidráulico en la parte delantera de la boca de aspiración permite la absorción de suciedades más gruesas.


\subsubsubject{Depósito de material barrido}

El depósito de aluminio para suciedad puede volcarse hasta 55° y a una altura de 1,5\,m (altura de caída). El canal de aspiración desemboca en el depósito desde abajo con un diámetro de apertura de 180\,mm.

Una turbina de gran potencia crea el vacío de aspiración. Esta turbina está montada en horizontal en el depósito de material barrido. Dispone de una tapa de mantenimiento para la limpieza y el control visual.

En la tapa de cierre del depósito de material barrido hay dos rejillas de aspiración de acero inoxidable. Estas pueden desplegarse sin herramientas para la limpieza. La tapa de cierre puede desbloquearse y abrirse a mano.

Mediante una válvula que puede plegarse con la mano la corriente de aire puede conmutarse sin problemas entre el canal de aspiración y la manguera manual de aspiración (equipamiento opcional).


\subsection{Dispositivo de humedecimiento}

\subsubsubject{Sistema de agua fresca}

El\index{barrer+humedecimiento} depósito de ABS Guss está de pie detrás de la cabina. Su capacidad\index{agua limpia+depósito} es de 190\,l.

Una bomba eléctrica (10\,l/min) bombea el agua hasta las toberas de rociado a través de cada uno de los cepillos (incluido el tercer cepillo opcional).


\subsubsubject{Reciclado de agua sucia}

El agua sucia pasa por las microperforaciones de las paredes interiores del depósito de agua sucia para salir después a través de la válvula de reciclado al depósito de agua reciclada. El\index{agua reciclada+depósito} depósito de agua reciclada tiene una capacidad de 140\,l.

Una bomba de inmersión eléctrica (10\,l/min) bombea el agua a las toberas de rociado en el interior de la boca de aspiración y del canal de aspiración.


\testpage [8]
\subsubsubject{Depósito de agua reciclada}

El depósito de agua reciclada dispone de un intercambiador de calor del líquido hidráulico-agua con función doble:

\startitemize[width=45mm,style=\md, command={\setupwhitespace[small]}]
\sym{Funcionamiento en verano} El agua dirige el calor del líquido hidráulico mediante convección a las paredes de aluminio del depósito desde donde se emite al aire exterior.

\sym{Funcionamiento en invierno} El líquido hidráulico calienta el agua en el depósito. Esto permite rociar el canal de aspiración, así como la boca de aspiración, incluso a temperaturas algo por debajo del punto de congelación.
\stopitemize


\subsubsubject{Control de los niveles de llenado de agua}

\startitemize[width=45mm,style=\md, command={\setupwhitespace[small]}]
\sym{Agua limpia} Cuando el nivel de agua no es suficiente, aparece el símbolo~\textSymb{vpad_water} en la pantalla del Vpad.
\sym{Agua reciclada} Cuando el nivel de llenado del depósito de agua reciclada está por debajo del intercambiador de calor (ver arriba), aparece el símbolo~\textSymb{vpad_rwater_orange} (amarillo) en la pantalla del Vpad. Si el nivel de llenado no es suficiente, aparece el símbolo~\textSymb{vpad_rwater} (rojo).
\stopitemize

\stopsection

\page [yes]

\setups[pagestyle:normal]


\startsection [title={Identificación del vehículo}]

\subsection{Placa de características del vehículo}

La placa de características del vehículo\index{identificación+vehículo} está en la cabina, enfrente de la consola, debajo del asiento del acompañante (ver \inF[fig:identity:location], \atpage[fig:identity:location]).


\subsection{Código y número de motor}

El código del motor está en la placa de características del motor (adhesivo), en el conducto de metal doblado del circuito de refrigeración, delante en el motor (elevar el depósito de material barrido).

El número de motor está grabado en el motor (\inF[identity:engine:number]). El número se compone de nueve caracteres alfanuméricos: Los tres primeros son el código del motor, los seis siguientes el número de serie del motor.


\placefig[margin][idvhc]{Placa de características del vehículo}
{\externalfigure[s2:id:plaque]}

\placefig[margin][identity:engine:code]{Placa de características del motor}
{\externalfigure[engine:id:code]}

\placefig[margin][identity:engine:number]{Número de motor}
{\externalfigure[engine:id:number]}

\page [yes]


\subsection [sec:plateWheel]{Placa de características de las ruedas}

La placa de características de las llantas y de los neumáticos\index{neumáticos+presión de llenado} está en la cabina\index{llantas+dimensiones}, debajo del asiento del acompañante.


\subsection{Número de chasis}

El número de chasis\index{identificación+número de chasis} está en el lado derecho del vehículo, debajo de la cabina, grabado en el chasis.


\subsection {Conformidad y marca \symbol[europe][CEsign]}

La marca de conformidad~\symbol[europe][CEsign] está en la cabina, enfrente de la consola, debajo del asiento del acompañante.

La \sdeux\ cumple los requisitos básicos de seguridad y salud de la Directiva de Máquinas\index{certificado+Conformidad CE}\index{Directiva de Máquinas} 2006/42/CE\footnote{Directiva 2006/42/CE del Parlamento Europeo y del Consejo del 17 de ~mayo de 2006}.
% \textrule

\placefig[margin][idpneus]{Presión de llenado de los neumáticos}
{\externalfigure[identity:tires]}

\placefig[margin][fig:identity:location]{Placas de características}
{\externalfigure[identity:location]}

\stopsection

\page [yes]

\setups [pagestyle:marginless]


\startsection[title={Datos técnicos},
reference={donnees_techniques}]

\subsection [sec:measurement] {Dimensiones del vehículo}

\placefig[here][fig:measurement]{\select{caption}{Ancho, cepillo en posición de reposo o desplegado, largo y alto del vehículo}{Dimensiones del vehículo}}
{\Framed{\externalfigure[s2:measurement]}}

\page [yes]

\placefig[here][fig:measurement]{\select{caption}{Altura del vehículo con depósito de suciedad elevado}{Altura del vehículo}}
{\Framed{\externalfigure[s2:measurement:02]}}

\page [yes]

\starttabulate [|lBw(45mm)|p|l|rw(35mm)|]
\FL
\NC Grupo\index{dimensiones} \NC \bf Medida \NC \bf Unidad\NC \bf Valor \NC\NR
\ML
\NC Dimensión del vehículo \NC Largo (en todas las partes) \NC \unite{mm} \NC 4588,00 \NC\NR
\NC\NC Largo con 3er\,cepillo\NC \unite{mm} \NC 5020,00 \NC\NR
\NC\NC Ancho del vehículo \NC \unite{mm} \NC 1150,00 \NC\NR
\NC\NC Ancho del vehículo (en todas las partes) \NC \unite{mm} \NC 1575,00 \NC\NR
\NC\NC Alto sin luz giratoria de aviso \NC \unite{mm} \NC 1990,00 \NC\NR
\NC\NC Distancia entre ejes \NC \unite{mm} \NC 1740,00 \NC\NR
\NC\NC Distancia de vía \NC \unite{mm} \NC 894,00 \NC\NR
\ML
\NC Ancho de barrido \NC Cepillo estándar \NC \unite{mm} \NC 2300,00 \NC\NR
\NC\NC Con 3er\,cepillo \NC \unite{mm} \NC 2600,00 \NC\NR
\NC\NC Diámetro de cepillo \NC \unite{mm} \NC 800,00 \NC\NR
\NC\NC Ancho de boca de aspiración \NC \unite{mm} \NC 800,00 \NC\NR
\ML
\NC Distribución de carga \NC Tara\note[weight:empty] Eje delantero \NC \unite{kg} \NC aprox. 1100,00 \NC\NR
\NC\NC Tara\note[weight:empty] Eje trasero\NC \unite{kg} \NC aprox. 1200,00 \NC\NR
\NC\NC Tara\note[weight:empty] \NC \unite{kg} \NC aprox. 2300,00 \NC\NR
\NC\NC Peso total permitido \NC \unite{kg} \NC 3500,00 \NC\NR
\LL
\stoptabulate


\subsection{Radio de vía y de barrido}

\starttabulate [|lBw(45mm)|p|l|rw(35mm)|]
\FL
\NC Dimensión\index{dimensiones} \NC \bf Medida \NC \bf Unidad \NC \bf Valor \NC\NR
\ML
\NC Radio de vía\index{radio de vía}\index{dimensiones+radio de vía} \NC Radio de viraje mínimo con cepillo \NC \unite{mm} \NC 3325,00 \NC\NR
\ML
\NC Radio de barrido \NC exterior\NC \unite{mm} \NC 3425,00 hasta 3850,00 \NC\NR
\NC\NC interior \NC \unite{mm} \NC 2025,00 hasta 1675,00 \NC\NR
\LL
\stoptabulate

%% TODO en/de/fr: Footnote on preceeding page
\footnotetext[weight:empty]{Configuración estándar, con conductor (aprox. 75\,kg).}

\placefig[here][pict:steerin_sweeping:radius]{Radio de vía/giro y radio de barrido}
{\externalfigure[steerin_sweeping:radius]}

\page [yes]


\subsection{Ruedas y neumáticos}

\starttabulate[|lBw(45mm)|p|rw(55mm)|]
\FL
\NC Componentes \NC \bf Equipamiento \NC \bf Valor \NC\NR
\ML
\NC Neumáticos \NC Dimensiones estándar \NC 205/70 R 15 C \NC\NR
\ML
\NC Llantas \NC Dimensiones estándar \NC 6J\;×\;15 H2 ET 60 \NC\NR
\ML
\NC Presión de llenado \NC Estándar, delanteros/traseros \NC 4,5/5,8\,bar \NC\NR
\LL
\stoptabulate


\subsection{Motor diésel}

\starttabulate [|lBw(45mm)|l|rp|]
\FL
\NC \bf Grupo\index{motor diésel+identificación} \NC \bf Parámetros \NC \bf Valor\NC\NR
\ML
\NC Tipo del motor \NC \NC VW CJDA TDI 2.0 – 475 NE \NC\NR
\NC Aspectos generales \NC Ciclo de funcionamiento \NC Motor diesel de cuatro tiempos \NC\NR
\NC\NC Número de cilindros \unite{n} \NC 4 \NC\NR
\NC\NC Perforación x Carrera \unite{mm} \NC 81\;×\;95,5 \NC\NR
\NC\NC Cilindrada total \unite{cm\high{3}} \NC 1968 \NC\NR
\NC\NC Válvulas por cilindro \NC 4 \NC\NR
\NC\NC Orden del control de válvulas \NC 1-3-4-2 \NC\NR
\NC\NC Marcha en ralentí mínima \unite{min\high{−1}} \NC 830 +50/−25 \NC\NR
\NC Potencia/par de giro \NC Velocidad máx. \unite{min\high{−1}} \NC 3400 \NC\NR
\NC\NC Potencia máx. \unite{kW} a \unite{min\high{−1}} \NC 75 hasta 3000 \NC\NR
\NC\NC Par de giro máx. \unite{Nm} a \unite{min\high{−1}} \NC 285 hasta 1750 \NC\NR
\NC Consumo específico\index{motor diésel+consumo} \NC Combustible \unite{g/kWh} \NC 224 (a potencia máx.) \NC\NR
\NC\NC Aceite \unite{g/kWh} \NC 0,22 \NC\NR
\NC Equipo de combustible \NC Sistema de inyección \NC Inyección directa \quote{Common-rail} \NC\NR
\NC\NC Suministro de combustible \NC Bomba de rueda dentada \NC\NR
\NC\NC Carga \NC Sí \NC\NR
\NC\NC Intercooler \NC Sí \NC\NR
\NC\NC Presión de carga \unite{mbar} \NC 1300\NC\NR
\NC Circuito de lubricación \index{motor diésel+lubricación} \NC Tipo \NC Lubricación forzada con intercambiador de aceite/agua \NC\NR
\NC\NC Alimentación de conductos \NC Bomba de rotor \NC\NR
\NC\NC Consumo de aceite \unite{litros/20\,h} \NC <\:0,1 \NC\NR
\NC Circuito de refrigeración\index{motor diésel+refrigeración} \NC Capacidad total \unite{l} \NC aprox. 12 \NC\NR
\NC\NC Presión de calibración en el recipiente de dilatación \unite{bar} \NC 1,4 \NC\NR
\NC\NC Termostato (apertura) \unite{°C} \NC 87 \NC\NR
\NC\NC Termostato (lleno) \unite{°C} \NC 102 \NC\NR
\NC Gases de escape \NC Filtro de partículas \NC Sí \NC\NR
\NC\NC Tratamiento de gases de escape \NC Sí \NC\NR
\NC\NC Norma \NC Euro 5 \NC\NR
\LL
\stoptabulate


\subsection{Rendimiento}

\starttabulate[|lBw(45mm)|p|l|rw(35mm)|]
\FL
\NC Rendimiento\index{rendimiento} \NC \bf Configuración \NC \bf Unidad \NC \bf Valor \NC\NR
\ML
\NC Velocidad \NC Modo \aW{de trabajo}\NC \unite{km/h} \NC 0 hasta 18 (continua) \NC\NR
\NC\NC Modo \aW{de marcha}\NC \unite{km/h} \NC 0 hasta 40 \NC\NR
\ML
\NC Límite de velocidad \NC Regulable \NC \unite{km/h} \NC 0 hasta 25 \NC\NR
\LL
\stoptabulate


\subsection{Equipo eléctrico}

{\starttabulate [|lw(65mm)|p|rw(30mm)|]
\FL
\NC \bf Grupo \NC \bf Componentes \NC \bf Valor \NC\NR
\ML
\NC Batería \NC Batería de plomo \NC 12\,V 63\,Ah \NC\NR
\NC Suministro de corriente \NC Alternador \NC 14,8\,V 90\,A \NC\NR
\NC Motor de arranque \NC Potencia \NC 1,8\,kW \NC\NR
\NC Equipamiento de audio \NC Conexión de radio\index{conexión de radio} y altavoces\index{altavoces} \NC Equipamiento de serie \NC\NR
\NC Iluminación y señalización delanteros \NC Luces de posición \NC 12\,V 5\,W \NC\NR
\NC\NC Luces de cruce \NC H7, 12\,V 55\,W \NC\NR
\NC\NC Focos de trabajo \NC G886, 12\,V 55\,W \NC\NR
\NC\NC Intermitentes \NC 12\,V 21\,W \NC\NR
\NC Iluminación y señalización traseros \NC Luces de freno combinadas \NC 12\,V 5/21\,W \NC\NR
\NC\NC Intermitentes \NC 12\,V 21\,W \NC\NR
\NC\NC Luces de marcha atrás \NC 12\,V 21\,W \NC\NR
\NC\NC Iluminación de la matrícula \NC 12\,V 5\,W \NC\NR
\NC Iluminación adicional \NC Luz giratoria de aviso \NC H1, 12\,V 55\,W \NC\NR
\LL
\stoptabulate
}
\stopsection

\stopchapter

\stopcomponent

