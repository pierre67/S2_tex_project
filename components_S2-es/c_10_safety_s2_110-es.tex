
\startcomponent c_10_safety_s2_095-es

\marking[chapter]{Símbolos de seguridad}


\chapter{Símbolos de seguridad}

\setups[pagestyle:marginless]

\section{Nuevo reglamento europeo sobre etiquetado de sustancias peligrosas}

{\em Señal en forma de rombo con fondo blanco y borde rojo.}\par\blank[1*medium]
{\em Desde el año 2008 está vigente en la UE el reglamento denominado CLP\index{reglamento CLP} con nuevas etiquetas de advertencia para sustancias y productos peligrosos.}\par\null

\startSymList \GHSgeneric
\SymList
\textDescrHead{Riesgo para la salud}
Advierte de \index{riesgo para la salud} riesgos para la salud que no derivan en muerte o en daños graves para la salud. Aquí se incluyen la irritación cutánea o el desencadenamiento de una alergia. El símbolo se emplea también como advertencia para otros riesgos, como la inflamación.\par
Sustituye:\crlf \HAZOcross\ o \HAZOpoison\ o \PHgeneric
\stopSymList

\startSymList \GHSbody
\SymList
\textDescrHead{Riesgos graves para la salud; puede ser causa de muerte especialmente en los niños}
Los productos pueden provocar daños graves para la salud. Este símbolo advierte también de riesgos\index{peligro+embarazo} durante el embarazo, de efectos cancerígenos\index{peligro+sustancias cancerígenas} y de otros riesgos graves para la salud. Los productos deben emplearse con precaución.\par
Sustituye:\crlf \HAZOcross\ o \HAZOpoison\
\stopSymList

\startSymList \GHSbomb
\SymList
\textDescrHead{Sustancias explosivas}
Las sustancias\index{peligro+explosión}, las mezclas y los productos explosivos\index{sustancias explosivas} inestables con sustancias explosivas tienen un fuerte efecto expansivo al reaccionar que puede causar destrucciones considerables. Al manipularlos de forma inadecuada existe peligro de muerte.\par
Sustituye:\crlf \HAZObomb\
\stopSymList


\startSymList \GHSpoison
\SymList
\textDescrHead{Intoxicación}
Los productos\index{peligro+intoxicación} pueden ser tóxicos o incluso mortales si se produce contacto con la piel, inhalación\index{materias tóxicas} o ingestión, aunque sea en pequeñas cantidades. No permitir un contacto directo.\par
Sustituye:\crlf \HAZOpoison\
\stopSymList

\startSymList \GHSfire
\SymList
\textDescrHead{Productos poco o muy inflamables}
Los productos\index{peligro+fuego} se inflaman rápidamente cerca del calor o de llamas. Los aerosoles con esta etiqueta no deben colocarse bajo ninguna circunstancia sobre superficies calientes o pulverizarse cerca de llamas abiertas.\par
Sustituye:\crlf \HAZOfire\ o \HAZOfirebis\
\stopSymList

\startSymList \GHSenvironment
\SymList
\textDescrHead{Peligro para animales y medio ambiente}
Los productos\index{protección medioambiental} pueden causar daños\index{sustancias tóxicas} al medio ambiente, ya sea a corto o a largo plazo. Pueden matar los organismos vivos en el agua (\eG\ peces) o, a largo plazo, tener efectos perjudiciales para el medio ambiente. ¡No desechar bajo ninguna circunstancia por la canalización ni con la basura doméstica!\par
Sustituye:\crlf \HAZOenvironment\
\stopSymList

\startSymList \GHScorrosive
\SymList
\textDescrHead{Peligro para la piel o los ojos}
Los productos\index{peligro+daños en la piel}\index{peligro+daños en los ojos} pueden dañar zonas cutáneas y crear cicatrices inmediatamente tras un contacto breve o dañar los ojos de forma permanente. ¡Al utilizarlos, proteja la piel y los ojos!\par
Sustituye:\crlf \HAZOcross\ o \HAZOcorrosive
\stopSymList

\page [yes]


\section{Señales de advertencia}

{\em Letra negra sobre fondo amarillo}\par\null

\startSymList \PHgeneric
\SymList
\textDescrHead{Señal de advertencia general}
Advierte\index{peligro+general}\index{señal de advertencia} de un peligro directo inminente en el que usted u otras personas pueden resultar dañadas.
\crlf\null
\stopSymList

\startSymList \PHpoison
\SymList
\textDescrHead{Advertencia de materias tóxicas}
Las materias tóxicas\index{peligro+intoxicación} pueden producir riesgos graves o crónicos para la salud considerables o incluso ser mortales si se produce contacto con la piel, inhalación o ingestión.
\stopSymList

\startSymList \PHfire
\SymList
\textDescrHead{Advertencia de materias inflamables}
Evitar las llamas y la formación de\index{peligro+fuego} chispas. La sustancia es fácilmente inflamable o puede tener efectos aceleradores del fuego. ¡Prohibido fumar!
\stopSymList

\startSymList \PHexplosive
\SymList
\textDescrHead{Advertencia de materias potencialmente explosivas}
Las materias y preparaciones sólidas, líquidas o similares que pueden explotar por impacto, fricción, fuego, calor y\,similares.\index{peligro+explosión} ¡Prohibido fumar!
\stopSymList

\startSymList \PHcrushing
\SymList
\textDescrHead{Advertencia de peligro de aplastamiento}
Advierte de una área\index{peligro+aplastamiento} en la que existe el peligro de aplastamiento por piezas mecánicas en movimiento. Manténgase alejado de esta área mientras el dispositivo esté conectado.
\stopSymList

\startSymList \PHhand
\SymList
\textDescrHead{Advertencia de lesiones en las manos}
Existe peligro de\index{peligro+aplastamiento} aplastar manos u otras partes del cuerpo\index{peligro+lesiones en manos} al \eG\ volcar la cabina o el puente de carga.
\stopSymList

\startSymList \PHentangle
\SymList
\textDescrHead{Advertencia de rodillos en marcha opuesta/peligro de tracción}
Existe peligro de que las extremidades\index{peligro+tracción} queden atrapadas por piezas en rotación que ejercerán una tracción. Manténgase alejado mientras el dispositivo esté conectado.
\stopSymList

\startSymList \PHcorrosive
\SymList
\textDescrHead{Advertencia de materias corrosivas}
Manipular con precaución\index{peligro+materias corrosivas}, llevar equipo de protección personal (guantes, gafas protectoras, ropa protectora).
\stopSymList

\startSymList \PHhot
\SymList
\textDescrHead{Advertencia de superficie caliente}
No se acerque al componente o al dispositivo\index{peligro+quemaduras}
sin los conocimientos suficientes. Llevar guantes protectores.
\stopSymList

\startSymList \PHvoltage
\SymList
\textDescrHead{Riesgo eléctrico}
No tocar los objetos metálicos\index{peligro+tensión eléctrica}.
¡Peligro de lesiones o quemaduras por cortocircuito!
\stopSymList

\startSymList \PHfalling
\SymList
\textDescrHead{Advertencia de caída a distinto nivel}
Prestar especial atención\index{peligro+caída} en esta área, llevar calzado adecuado (con suelas antideslizantes, resistentes al hidrocarburo).
\stopSymList

\startSymList \PHbattery
\SymList
\textDescrHead{Advertencia de peligro por baterías} Advierte de los riesgos producidos al cargar las baterías (baterías de plomo)\index{peligro+batería}, especialmente por escape de gas de hidrógeno y del ácido sulfúrico contenido en las baterías.
\stopSymList

\startSymList \PHremote
\SymList
\textDescrHead{Advertencia de marcha automática}
Advierte de\index{peligro+marcha automática} la posible marcha automática o por control remoto de un dispositivo.
\stopSymList

% \startSymList \PHquetschgefahr
% \SymList
% \textDescrHead{Risque d’écrasement}
% Risque d’écrasement\index{risque d’écrasement}.
% \stopSymList
% % NOTE: Doppelt! (auch Bilddatei)
%
% % NOTE: Evtl. Folgendes als Ersatz für oben?

% \startSymList\PHhandcrushed
% \SymList
% \textDescrHead{Gefahr von Handquetschungen}
% Es besteht\index{Gefahr+Quetschung} die Gefahr, dass Hände oder Finger
% gequetscht werden. Nähern Sie die Hände nicht an, ohne die Gefahr
% identifiziert und beseitigt zu haben.
% \stopSymList

\startSymList \PHhandfoot
\SymList
\textDescrHead{Advertencia de componentes en movimiento}
Advierte de piezas de la máquina/el vehículo en movimiento.
\index{peligro+piezas en movimiento}.
\stopSymList

\startSymList \PHnarrowed
\SymList
\textDescrHead{Advertencia de vía estrecha}
Vía\index{peligro+ancho del vehículo} estrecha.
% Denken Sie an die Breite des Fahrzeugs.
\stopSymList

\page [yes]


\section{Señales de prohibición}

{\em Señal redonda con fondo blanco, borde rojo y barra diagonal}
\par\null


\startSymList \PPfire
\SymList
\textDescrHead{Prohibido fumar y encender fuego} Está prohibido encender fuego\index{prohibición+fumar, fuego} y brasas de cualquier forma (\eG\ cigarrillo encendido, cerilla, vela; también cualquier otro tipo de formación de chispas).
\stopSymList

\startSymList \PPentry
\SymList
\textDescrHead{Entrada prohibida a personas no autorizadas}
Está prohibido\index{prohibición+acceso} el acceso a esta área a las personas no autorizadas.
\stopSymList

\startSymList \PPphone
\SymList
\textDescrHead{Prohibido utilizar teléfonos móviles}
Los teléfonos móviles\index{prohibición+telefonía móvil} y cualquier tipo de aparatos que emitan radiación electromagnética deberán estar apagados. La radiación electromagnética puede producir anomalías en el funcionamiento de la electrónica del aparato.
\stopSymList

\startSymList \PPspray
\SymList
\textDescrHead{Prohibido salpicar con agua}
No dirija nunca un chorro de agua o de vapor\index{prohibición+chorro de agua, vapor} hacia piezas sensibles o aperos (\eG\ sensores, unidades de control, equipo de inyección, etc.).
\stopSymList

\startSymList \PPchildren
\SymList
\textDescrHead{Mantener a los niños alejados}
Indicación \index{prohibición+niños} de un peligro especial para niños. Por regla general se aplica: Los niños no pueden acercarse a una máquina conectada, tampoco durante las tareas de mantenimiento.
\stopSymList

\startSymList \PPwater
\SymList
\textDescrHead{Agua no potable}
No beber el agua\index{prohibición+agua no potable} del depósito. Existe peligro de intoxicación.
\stopSymList

% \page [yes]


\section{Señales de protección medioambiental}

\startSymList \PSrecycle
\SymList
\textDescrHead{Reciclado}
Disposiciones específicas sobre la correcta eliminación de determinados desechos.
\stopSymList

\startSymList \PSwelt
\SymList
\textDescrHead{Protección medioambiental}
Indicación sobre las disposiciones medioambientales vigentes.
\stopSymList

\startSymList \PStrash[width=\PictoHeight,height=,]
\SymList
\textDescrHead{Eliminar los desechos según las disposiciones}
Para determinados desechos, \eG\ baterías de plomo, se aplican disposiciones de eliminación de desechos especiales.
\stopSymList


\testpage[12]


\section{Señales de obligación}


{\em Redonda con fondo azul}\par\null

\startSymList \PMgeneric
\SymList
\textDescrHead{Señal de obligación general}
Este símbolo únicamente puede emplearse en combinación con una señal adicional que especifique la obligación.
\stopSymList


\startSymList \PMrtfm
\SymList
\textDescrHead{Es obligatorio leer el manual de uso}
Antes de la puesta en servicio\index{leer indicaciones de uso} es imprescindible haber leído las indicaciones sobre este tema para un aparato o producto concreto. El manual de uso deberá estar a mano en la cabina.
\stopSymList

\startSymList \PMproteyes
\SymList
\textDescrHead{Protección obligatoria de la vista}
Cuando existe peligro de lesiones en los ojos durante el trabajo deberá utilizarse protección para los ojos\index{protección para los ojos}.
\stopSymList

\startSymList \PMprothands
\SymList
\textDescrHead{Protección obligatoria de las manos}
Cuando existe peligro de lesiones en las manos durante el trabajo deberán utilizarse guantes protectores\index{utilizar guantes protectores}.
\stopSymList

\startSymList \PMprotears
\SymList
\textDescrHead{Protección obligatoria del oído}
Debe llevarse protección para los oídos\index{peligro+oídos} (\eG en las inmediaciones de un ventilador o de una turbina en marcha).
\stopSymList

\startSymList \PMsafetybelt
\SymList
\textDescrHead{Es obligatorio utilizar cinturón de seguridad} Póngase\index{cinturón de seguridad} el cinturón de seguridad para su seguridad.
\stopSymList

\section{Símbolos adicionales}

% \adaptlayout[height=+5mm]{{{

% \startSymList \SETshoe
% \SymList
% \textDescrHead{Port de chaussures de sécurité obligatoire}
% Le port de chaussures de sécurité est obligatoire\index{chaussures de sécurité}.
% \stopSymList
%
% \startSymList \SETglasses
% \SymList
% \textDescrHead{Port de lunettes des protection obligatoire}
% Le port de lunettes est obligatoire\index{lunette de protection}.
% \stopSymList
%
% \startSymList \SEToreillettes
% \SymList
% \textDescrHead{Port de casque obligatoire}
% Le port d’un casque de protection est \index{casque} obligatoire.
% \stopSymList
%
% \startSymList \SETgloves
% \SymList
% \textDescrHead{Port de gants de protection obligatoire}
% Le port de gants de protection est obligatoire\index{gants}.
% \stopSymList
%
% \startSymList \SETmainecrase
% \SymList
% \textDescrHead{Risque d’écrasement}
% Danger pour les mains\index{écrasement} et les pieds.
% \stopSymList
%
% \startSymList \SETgetriebe
% \SymList
% \textDescrHead{Risque de happement}
% Risque de happement par\index{happement} des pièces en rotation.
% \stopSymList
%
% \startSymList \SETradkeil
% \SymList
% \textDescrHead{Cale de roue}
% Sécuriser le véhicule contre toute mise\index{cale de roue} en marche involontaire.
% \stopSymList
%}}}

\startSymList \SETfirstaid
\SymList
\textDescrHead{Primeros auxilios}
Señala el lugar donde se encuentra el equipo de primeros auxilios. Avisar con rapidez al servicio de salvamento es parte integrante importante de los primeros auxilios.\index{primeros auxilios}\index{llamada de emergencia} Anote aquí sus números de emergencia:
\fillinrules[n=1]{\bf
\framed[align=right,frame=off,offset=none,width=30mm]{Servicio de salvamento}}
\fillinrules[n=1]{\bf
\framed[align=right,frame=off,offset=none,width=30mm]{Policía}}
\fillinrules[n=1]{\bf
\framed[align=right,frame=off,offset=none,width=30mm]{Bomberos}}
\stopSymList

\startSymList \SETbrandschutzzeichen
\SymList
\textDescrHead{Extintor}
Algunos aparatos están equipados con uno o con varios extintores\index{extintores}. Por norma general, estos necesitan un mantenimiento especial. En el aparato o en las indicaciones de uso encontrará más información al respecto.
\stopSymList


\page[yes]

\section{Los tres pasos de los primeros auxilios}
% NOTE [tf]: Shouldn't be in this book, IMO

\starttextbackground [CB]
\textDescrHead{Asegure el lugar del accidente y a las personas afectadas.}
\startitemize
\item Evalúe la seguridad en el lugar del accidente y asegúrese de que no pueden producirse más peligros.
\stopitemize
\textDescrHead{Evalúe el estado de los heridos.}
\startitemize
\item Compruebe si los heridos están conscientes y respiran con normalidad.
En caso necesario, libere las vías respiratorias.
\stopitemize
\textDescrHead{Avise al equipo de salvamento.}
\startitemize Su llamada de emergencia deberá tener las siguientes informaciones:\par
\item El número de teléfono en el que se le puede contactar.
\item El tipo de suceso (enfermedad, accidente).
\item Riesgos existentes (incendio, explosión, peligro de derrumbamiento).
\item El lugar exacto del suceso.
\item El número de heridos y su estado.
\item Medidas de primeros auxilios que ya han sido realizadas.
\item Responda a otras preguntas que se le harán.
\stopitemize
\stoptextbackground

\stopcomponent


