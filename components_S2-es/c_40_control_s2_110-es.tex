
\startcomponent c_40_control_s2_095-es


\startchapter [title={Elementos de mando del vehículo},
reference={chap:ctrl}]

\setups[pagestyle:marginless]

\placefig[here][fig:ctrl:cab:front]{Elementos de mando}
{\externalfigure[ctrl:cab:front]}

\startcolumns [n=3]
\startLongleg
\item Columna de dirección (\in{§}[sec:steeringColumn])
\item Ajuste de la columna de dirección
\item Pedal de desplazamiento y de freno
\item Ordenador de a bordo \Vpad~SN (\inP[sec:vpad])
\item Consola de techo (\inP[sec:ctrl:aux])
\stopLongleg


\subsubsubject{Equipamiento opcional}

\startLongleg [continue]
\item Pantalla de marcha atrás
\item Radio/MP3
\stopLongleg
\stopcolumns

\startsection [title={Columna de dirección},
reference={sec:steeringColumn}]

\subsection{Ajustar la columna de dirección}

\textDescrHead{Inclinación del volante} Pise el pedal~\Ltwo y ajuste simultáneamente la inclinación de la columna de dirección. Suelte el pedal para volver a bloquear el mecanismo de la columna de dirección.

\page[yes]
\setups [pagestyle:normal]


\subsection{Dispositivo de iluminación y señalización}

\placefig [margin] [fig:column:left] {Palanca multifuncional e interruptor giratorio}
{\externalfigure[ctrl:column:left]}

\placefig [margin] [fig:column:right] {Palanca de marchas}
{\externalfigure[ctrl:column:right]}


\subsubsubject{Interruptor giratorio}

\startitemize[width=1.7em]
\sym{\textSymb{com_lowlight}} Luces de cruce (girar).
\startitemize
\sym{1} Luces de posición
\sym{2} Luces de cruce
\stopitemize
\stopitemize


\subsubsubject{Palanca multifuncional}

\startitemize[width=1.7em]
\sym{\textSymb{com_lowlight}} {[}Sin ocupar{]}
\sym{\textSymb{com_light}} Avisador luminoso (tirar brevemente de la palanca hacia arriba)
\sym{\textSymb{com_blink}} Indicador del sentido de marcha (palanca hacia delante/atrás)
\sym{\textSymb{com_claxonArrow}} Bocina (presionar el botón externo en la palanca)
\sym{\textSymb{com_wipper}} Limpiaparabrisas
\startitemize
\sym{J} Conexión de intervalos
\sym{O} Desconectado
\sym{I} Velocidad\, 1
\sym{II} Velocidad\, 2
\stopitemize
\sym{\textSymb{com_washerArrow}} Sistema limpiaparabrisas (presionar la corona en el extremo de la palanca).
\stopitemize


\subsubsubject{Palanca de marchas}

Las funciones de la palanca de marchas están descritas con detalle en el capítulo~\about[chap:using], a partir de la~\atpage[sec:using:start].

\stopsection

\page [yes]


\startsection [title={Otras funciones},
reference={sec:ctrl:add}]


\subsection[sec:ctrl:aux]{Consola de techo}

{\sl La\index{consola de techo} consola de techo se encuentra delante en el techo de la cabina, del lado del conductor.}
\placefig [margin] [fig:console:aux] {Consola de techo}
{\externalfigure[ctrl:console:aux]}


\placefig [margin] [fig:console:climat] {Calefacción y aire acondicionado}
{\externalfigure[ctrl:console:climat]}


\startitemize [unpacked][width=1.7em]
\sym{\textBigSymb{temoin_retrochauffant}} Calefacción espejos exteriores
\sym{\textBigSymb{temoin_hazard}} Dispositivo de luces de aviso
\sym{\textBigSymb{temoin_eclairage_L}} Focos de trabajo
\stopitemize


\subsubsubject{Equipamiento opcional}

\startLeg [unpacked][width=1.7em]
\sym{\textBigSymb{temoin_buse}} Bomba de agua de alta presión para pistola de agua \crlf {\sl ver \atpage[sec:using:water:spray]}
\sym{\textBigSymb{temoin_aspiration_manuelle}} Turbina para manguera manual de aspiración\crlf {\sl ver \atpage[sec:using:suction:hose]}
\stopLeg


\subsection[sec:ctrl:climat]{Calefacción y aire acondicionado}

{\sl Esta consola\index{consola de calefacción} está en la pared posterior de la cabina, entre los asientos.}

\startitemize [unpacked][width=23mm]
\sym{\bf 0\quad I\quad II\quad III} Interruptor de giro del ventilador
\sym{\externalfigure[tirette_chauffage][height=1em]} Regulador de temperatura
\stopitemize


\subsubsubject{Equipamiento opcional}

\startitemize [unpacked][width=1.7em]
\sym{\textBigSymb{temoin_climat_bk}} Aire acondicionado
\stopitemize

\page [yes]

\setups [pagestyle:bigmargin]


\subsection[sec:ctrl:central]{Consola central}

{\sl La\index{consola central} consola central está entre los asientos.}

\placefig [margin] [fig:console:central] {Consola central}
{\externalfigure[ctrl:console:central]}


\subsubsubject{Humedecer los cepillos}

\startLeg [unpacked][width=1.7em]
\sym{\textBigSymb{temoin_busebalais}} Bomba de agua de baja presión\index{bomba de agua} para el sistema de humedad\index{bomba de agua+humedecer} de los cepillos. (Posición~1: \aW{automática}; Posición~2: \aW{fija})
\stopLeg


\subsubsubject{Volcado del depósito de material barrido}

\setupinmargin[right][style=normal]
\inright{%
\startitemize
\sym{\textSymb{mand_readtheoperatingmanual}} Tenga en cuenta las instrucciones de uso del freno de estacionamiento en la \atpage[sec:using:stop].
\stopitemize}

\startLeg [unpacked][width=1.7em]
\sym{\textBigSymb{temoin_kipp2}} Volcado del depósito de material barrido. Para\index{depósito de material barrido+volcar} poder volcar el depósito de material barrido, el freno de estacionamiento deberá estar puesto y la palanca de marchas en posición neutra.
\stopLeg


\subsubsubject{Parada de emergencia}

\starttextbackground [FC]
\startPictPar
\externalfigure[Emergency_Stop][Pict]
\PictPar
En caso de emergencia\index{parada de emergencia} puede desconectar todos los equipos de aspiración y barrido presionando el interruptor de parada de emergencia.
\stopPictPar
\stoptextbackground


\subsection[sec:foot:switch]{Pedal}

\placefig [margin] [fig:foot:switch] {Pedal}
{\vskip 60pt
\externalfigure[work:foot:switch]}

Mediante\index{pedal} este interruptor en la base de la columna de dirección (\inF[fig:foot:switch]) puede bajar rápida y fácilmente los cepillos cuando sea necesario (\eG\ en la cumbre de una inclinación, al subir a la acera).

\stopsection
\page[yes]
\setups [pagestyle:marginless]


\startsection[title={Consola multifuncional},
reference={ctrl:console:middle}]

\startlocalfootnotes

\startfigtext[left]{Consola multifuncional}
{\externalfigure[overview:joy:large]}


\subsubsubject{Joysticks}

\textDescrHead{Sin cepillo delantero (o cepillo delantero desactivado):}
Los joysticks controlan independientemente un cepillo cada uno: Subir/bajar~(\textSymb{joystick_aa}) o izquierda/derecha~(\textSymb{joystick_gd}). El joystick izquierdo controla el cepillo izquierdo, el joystick derecho el cepillo derecho.\footnote{Para poder modificar la posición del cepillo lateral en un vehículo equipado con cepillo delantero (opción), el cepillo delantero deberá estar desactivado (tecla~\textSymb{joy_key_frontbrush_act}).}

\textDescrHead{Con cepillo delantero:}
Con el joystick izquierdo puede subir/bajar el cepillo delantero (\textSymb{joystick_aa}) y moverlo a derecha/izquierda (\textSymb{joystick_gd}). Con el joystick derecho inclina el cepillo sobre su eje longitudinal~(\textSymb{joystick_aa}) y transversal~(\textSymb{joystick_gd}).

\placelocalfootnotes %[height=\textheight]
\stopfigtext
\stoplocalfootnotes
\vfill


\subsubsubject{Teclas laterales}

\startcolumns

\startPictList
\VPcltr
\PictList
Limitador automático de velocidad: Incrementar la velocidad ajustada
\stopPictList\vskip -3pt

\startPictList
\VPclbr
\PictList
Limitador automático de velocidad: Reducir la velocidad ajustada
\stopPictList\vskip -3pt

\startPictList
\VPcrtr
\PictList
Subir boca de aspiración
\stopPictList

\startPictList
\VPcrbr
\PictList
Bajar boca de aspiración
\stopPictList\vskip -3pt

\startPictList
\VPcrtf
\PictList
Abrir tapa de suciedad gruesa (delante en la boca de aspiración)
\stopPictList\vskip -3pt

\startPictList
\VPcrbf
\PictList
Cerrar tapa de suciedad gruesa
\stopPictList

\stopcolumns


\subsubsubject{Teclas de símbolos}

\startcolumns

\startSymVpad
\externalfigure[joy:stop]
\SymVpad
\textDescrHead{Stop} Para el aparato activo:

Pulsar1 vez: Desactivar\,tercer cepillo\crlf
Pulsar2 veces: Desactivar todo
\stopSymVpad

\startSymVpad
\externalfigure[joy:tempomat]
\SymVpad
\textDescrHead{Limitador automático de velocidad} Ajustar y activar el limitador automático de velocidad a la velocidad actual. Para desactivar volver a pulsar la tecla~\textSymb{joy:tempomat} o frenar. Acelera/ralentice con las teclas laterales.
\stopSymVpad

\startSymVpad
\externalfigure[joy:ftbrs:minus]
\SymVpad
\textDescrHead{Velocidad de los cepillos} Reducir la velocidad de rotación del cepillo lateral o del cepillo delantero.
\stopSymVpad

\startSymVpad
\externalfigure[joy:ftbrs:plus]
\SymVpad
\textDescrHead{Velocidad de los cepillos} Incrementar la velocidad de rotación del cepillo lateral o del cepillo delantero.
\stopSymVpad

\startSymVpad
\externalfigure[joy:eng:minus]
\SymVpad
\textDescrHead{Velocidad del motor} Reducir la velocidad del motor diésel.
\stopSymVpad

\startSymVpad
\externalfigure[joy:eng:plus]
\SymVpad
\textDescrHead{Velocidad del motor} Incrementar la velocidad del motor diésel.
\stopSymVpad
\columnbreak

\startSymVpad
\externalfigure[joy:suc]
\SymVpad
\textDescrHead{Aspiración} Activar el sistema de aspiración: Se baja la boca de aspiración, la turbina y la bomba de agua reciclada se conectan.\note [recyclingwaterpump]
Pulsar la tecla de parada~\textSymb{joy:stop} para desactivar el sistema.
\stopSymVpad

\startSymVpad
\externalfigure[joy:sucbrs]
\SymVpad
\textDescrHead{Barrer/Aspirar}Activar el sistema de aspiración/barrido: Se baja la boca de aspiración, se bajan y posicionan los cepillos laterales, se conectan la turbina, los cepillos y la bomba de agua reciclada.\note [recyclingwaterpump]
Pulsar la tecla de parada~\textSymb{joy:stop} para desactivar el sistema.
\stopSymVpad

\footnotetext[recyclingwaterpump]{También se conecta la bomba de agua limpia cuando el interruptor~\textBigSymb{temoin_busebalais} está en \aW{automático} (ver \in [sec:ctrl:central] en la \atpage [sec:ctrl:central]).}
\startSymVpad
\externalfigure[joy:ftbrs:act]
\SymVpad
\textDescrHead{Cepillo delantero activado} Activar/desactivar cepillo delantero.
%% NOTE @Andrew: Singular
\stopSymVpad

\startSymVpad
\externalfigure[joy:ftbrs:right]
\SymVpad
\textDescrHead{Cepillo delantero izquierdo} Sentido de giro para trabajar con el cepillo delantero en el lado izquierdo (sentido de giro: el de las agujas del reloj).
\stopSymVpad

\startSymVpad
\externalfigure[joy:ftbrs:left]
\SymVpad
\textDescrHead{Cepillo delantero derecho} Sentido de giro para trabajar con el cepillo delantero en el lado derecho (sentido de giro: en el sentido contrario al de las agujas del reloj).
\stopSymVpad

\stopcolumns

\stopsection

\stopchapter

\stopcomponent













