
\startcomponent c_80_maintenance_s2_095-es

\startchapter [title={Mantenimiento y puesta a punto},
reference={chap:maintenance}]

\setups[pagestyle:marginless]


\startsection [title={Indicaciones generales}]


\subsection{Protección medioambiental}

\starttextbackground [FC]
\setupparagraphs [PictPar][1][width=2.45em,inner=\hfill]

\startPictPar
\Penvironment
\PictPar
\Boschung\ pone en práctica la protección medioambiental\index{protección medioambiental}. Trabajamos en las causas e integramos todos los efectos que el proceso de producción y el producto tienen sobre el medio ambiente en las decisiones de la empresa. Nuestros objetivos son un uso económico de los recursos y una manipulación cuidadosa de los medios de subsistencia naturales, cuya conservación beneficia tanto al hombre como a la naturaleza. Respetando ciertas reglas al utilizar el vehículo podrá realizar su aportación a la protección medioambiental. Una de estas normas es la manipulación comedida y según lo prescrito de las sustancias y los materiales necesarios para el mantenimiento del vehículo (\eG\ el desecho de productos químicos y residuos tóxicos).

El consumo de combustible y el desgaste de un motor dependen de las condiciones de servicio. Por este motivo le pedimos que preste atención a algunos puntos:

\startitemize
\item No precaliente el motor en punto muerto.
\item Apague el motor durante los periodos de espera condicionados por el servicio.
\item Controle regularmente el consumo de combustible.
\item {\em Deje que un taller técnico competente realice los trabajos de mantenimiento conforme al plan de mantenimiento.}
\stopitemize
\stopSymList
\stoptextbackground

\page [yes]


\subsection{Disposiciones de seguridad}

\startSymList
\PHgeneric
\SymList
Para\index{mantenimiento+disposiciones de seguridad} evitar daños en el vehículo y en los equipos incorporados, así como accidentes durante las tareas de mantenimiento, es imprescindible cumplir las siguientes disposiciones de seguridad. Tenga también en cuenta las disposiciones de seguridad generales (\about[safety:risques], \at{a partir de la página}[safety:risques]).
\stopSymList

\starttextbackground [FC]
\startPictPar
\PMgeneric
\PictPar
\textDescrHead{Prevención de accidentes}
Controle\index{prevención de accidentes} el estado del vehículo después de cada mantenimiento o reparación. Antes de circular por vías públicas asegúrese especialmente de que todos los componentes relativos a la seguridad, así como los dispositivos de señalización, funcionan correctamente.
\stopPictPar
\stoptextbackground
\blank [big]

\start
\setupparagraphs [SymList][1][width=6em,inner=\hfill]
\startSymList\PHcrushing\PHfalling\SymList
\textDescrHead{Estabilizar el vehículo}
Antes de realizar cualquier trabajo de mantenimiento deberá asegurarse el vehículo para evitar movimientos involuntarios del mismo: Ponga la palanca de marchas en \aW{neutral}, active el freno de estacionamiento y asegure el vehículo con cuñas para ruedas.
\stopSymList
\stop

\starttextbackground[CB]
\startPictPar\PHpoison\PictPar
\textDescrHead{Arrancar el motor}
Si \index{peligro+intoxicación} tiene que arrancar el motor en un lugar con mala ventilación, deje que este marche únicamente el tiempo necesario\index{peligro+gases de escape} para evitar intoxicaciones por monóxido de carbono.
\stopPictPar
\startitemize
\item Arranque el motor únicamente cuando la batería esté conectada adecuadamente.
\item No desborne nunca la batería con el motor en marcha.
\item No arranque el motor con ayuda de un cargador rápido.
Cuando\index{batería+cargador} sea necesario cargar la batería con un cargador rápido, deberá separarse antes la batería del vehículo. Tenga en cuenta las disposiciones de servicio del cargador rápido.
\stopitemize
\stoptextbackground

\page [bigpreference]

\subsubsection{Protección de los componentes electrónicos}

\startitemize
\item Antes\index{soldadura eléctrica} de iniciar los trabajos de soldadura, separe el cable de la batería y conecte juntos el cable positivo y el de tierra.
\item Conecte\index{sistema electrónico} y separe las unidades de control electrónicas únicamente cuando la instalación eléctrica ya no se encuentre bajo tensión.
\item Una\index{unidad de control} polaridad incorrecta en el suministro de corriente (\eG\ debida a baterías conectadas incorrectamente) puede destruir los componentes electrónicos y los aparatos.
\item Con\index{temperatura ambiente+extrema} temperaturas ambiente por encima de los 80 °C (\eG\ en una cámara secadora) deben retirarse los componentes/aparatos electrónicos.
\stopitemize


\subsubsection{Diagnóstico y mediciones}

\startitemize
\item Para los trabajos de medición y de diagnóstico emplee solo cables de prueba {\em adecuados} (\eG\ los cables originales del aparato).
\item Los teléfonos móviles\index{teléfonos móviles} y los aparatos que emitan señales pueden perjudicar las funciones del vehículo, del instrumento de diagnóstico y, con ello, la seguridad operativa.
\stopitemize


%%%%%%%%%%%%%%%%%%%%%%%%%%%%%%%%%%%%%%%%%%%%%%%%%%%%%%%%%%%%%%%%%%%%%%%%%%%%%%%%%%%%%%%%%

\subsubsection{Cualificación del personal}

\starttextbackground[CB]
\startPictPar
\PHgeneric
\PictPar
\textDescrHead{Riesgo de accidentes}
Si\index{cualificación+personal de mantenimiento} se realizan de forma inadecuada tareas de mantenimiento en el vehículo, estas pueden afectar negativamente al funcionamiento y a la seguridad del vehículo. Esto supone un riesgo elevado de accidentes y lesiones.

Para\index{cualificación+taller} los trabajos de mantenimiento y reparación póngase en contacto con un taller técnico cualificado que disponga de los conocimientos y las herramientas necesarios.

En caso de dudas, póngase en contacto con el Servicio de Atención al Cliente de \Boschung.
\stopPictPar
\stoptextbackground

% \page [yes]

El \ProductId únicamente puede ser operado, mantenido o reparado por personal cualificado y formado por el Servicio de Atención al Cliente de \Boschung.

El Servicio de Atención al Cliente de \Boschung otorgará las competencias para la operación, la puesta a punto y la reparación.

%\adaptlayout [height=+5mm]


\subsubsection{Modificaciones y transformaciones}

\starttextbackground[CB]
\startPictPar
\PHgeneric
\PictPar
\textDescrHead{Riesgo de accidentes}
Cualquier\index{modificaciones en el vehículo} modificación en el vehículo que realice por cuenta propia podría afectar negativamente al funcionamiento y a la seguridad operativa del \ProductId y, con ello, suponer un riesgo de accidentes y de lesiones que deberá evitarse.
\stopPictPar

\startPictPar
\PMwarranty
\PictPar
\Boschung\ \index{garantía+condiciones} no ofrece garantía alguna por los daños ocasionados por intervenciones o modificaciones por cuenta propia en el \ProductId o en un equipo incorporado.
\stopPictPar
\stoptextbackground

\stopsection


\startsection [title={Sustancias y lubricantes}, reference={sec:liquids}]


\subsection{Correcta manipulación}

\starttextbackground[CB]
\startPictPar
\PHpoison
\PictPar
\textDescrHead{Peligro de lesiones y de intoxicación}
El\index{combustible} contacto con la piel\index{lubricantes} o\index{peligro+intoxicación} la ingestión de sustancias industriales y lubricantes pueden\index{combustible+seguridad} ser causa de lesiones o intoxicaciones considerables. Al manipular, almacenar y desechar estas sustancias respete siempre las disposiciones legales.
\stopPictPar
\stoptextbackground

\starttextbackground [FC]
\startPictPar
\PMproteyes\par
\PMprothands
\PictPar
Al manipular sustancias industriales y lubricantes lleve siempre ropa protectora adecuada y protección respiratoria. Evite la inhalación de vapores.
Evite cualquier contacto con la piel, los ojos o la ropa. Aclare inmediatamente con agua y jabón las zonas de la piel que hayan entrado en contacto con sustancias industriales. Si las sustancias industriales entran en contacto con los ojos, aclárelos con abundante agua limpia y acuda a un oftalmólogo. ¡Después de ingerir sustancias industriales deberá acudirse inmediatamente a un médico!
\stopPictPar
\stoptextbackground

\startSymList
\PPchildren
\SymList
Las sustancias industriales deben mantenerse fuera del alcance de los niños.
\stopSymList

\startSymList
\PPfire
\SymList
\textDescrHead{Peligro de incendio}
Debido\index{peligro+fuego} a la alta inflamabilidad de las sustancias industriales, el riesgo de incendio aumenta durante la manipulación de las mismas. Queda terminantemente prohibido fumar y encender fuego\index{prohibido fumar} durante la manipulación de sustancias industriales.
\stopSymList


%% TODO; en
\starttextbackground [FC]
\startPictPar
\PMgeneric
\PictPar
Únicamente pueden emplearse lubricantes adecuados para los componentes empleados en el \ProductId. Por esta razón, utilice únicamente productos autorizados y homologados por \Boschung. Encontrará los mismos en la lista de sustancias industriales \atpage[sec:liqquantities]. No son necesarios los aditivos\index{aditivos} para los lubricantes. Si va a añadir aditivos, esto puede anular los derechos por garantía\index{garantía+condiciones}.
Para más información, póngase en contacto con el Servicio de Atención al Cliente de \Boschung.
\stopPictPar
\stoptextbackground

\starttextbackground [FC]
\startPictPar
\Penvironment
\PictPar
\textDescrHead{Protección medioambiental}
Al desechar\index{lubricantes+eliminación de desechos} sustancias industriales y\crlf lubricantes\index{protección medioambiental} u objetos contaminados con ellos (\eG\ filtros, paños) preste atención\index{sustancias industriales+eliminación de desechos} al cumplimiento de las disposiciones medioambientales.
\stopPictPar
\stoptextbackground

\page [yes]

\setups [pagestyle:normal]


\subsection[sec:liqquantities]{Especificaciones y niveles de llenado}

Todas\index{sustancias industriales+cantidad de llenado}\index{lubricante+cantidad de llenado}\index{cantidades de llenado+sustancias industriales y lubricantes}\index{especificaciones+sustancias industriales y lubricantes} las cantidades de llenado en la siguiente tabla son valores de referencia. Después de cada cambio de sustancia industrial/lubricante deberá controlarse el nivel de llenado real y, en caso necesario, aumentarse o reducirse la cantidad de llenado.
% \blank[big]

\placetable[margin][tab:glyco]{Anticongelante (\index{anticongelante}motor)}
{\noteF\startframedcontent[FrTabulate]
%\starttabulate[|Bp(80pt)|r|r|]
\starttabulate[|Bp|r|r|]
\NC Protección anticongelante hasta {[}°C{]}\NC \bf \textminus 25 \NC \bf \textminus 40 \NC\NR
\NC Agua destilada [Vol.-\%] \NC 60 \NC 40 \NC\NR
\NC Anticongelante \break [Vol.-\%] \NC 40 \NC {\em máx.} 60 \NC\NR
\stoptabulate\stopframedcontent\endgraf
Atención: ¡Con una proporción en el volumen de más del 60\hairspace\percent\ de anticongelante la protección anticongelante {\em se reduce} y la refrigeración empeora!}

\placefig[margin][fig:hydrgauge]{\select{caption}{Indicador de nivel del líquido hidráulico (lado izquierdo del vehículo)}{Indicador de nivel del líquido hidráulico}}
{\externalfigure[main:hy:level_temp]
\noteF El nivel de llenado del depósito hidráulico puede verse en la mirilla y deberá revisarse {\em a diario}.}

%\placetable[here,split][tbl:liquids]{Spécifications et volumes de remplissage des consommables}
%{\readfile{tbl_jb-fr_liquids}{}{\Warn}}


\vskip -8pt
\start
\define [1] \TableSmallSymb {\externalfigure[#1][height=4ex]}
\define\UC\emptY
\pagereference[page:table:liquids]

\setupTABLE [frame=off,style={\ssx\setupinterlinespace[line=.86\lH]},background=color,
option=stretch,
split=repeat]
\setupTABLE [r] [each] [topframe=on,
framecolor=TableWhite,
% rulethickness=.8pt
]

\setupTABLE [c] [odd] [backgroundcolor=TableMiddle]
\setupTABLE [c] [even] [backgroundcolor=TableLight]
\setupTABLE [c] [1][width=30mm]
\setupTABLE [c] [2][width=22mm]
\setupTABLE [c] [4][width=26mm]
\setupTABLE [c] [last] [width=10mm]
\setupTABLE [r] [first] [topframe=off,style={\bfx\setupinterlinespace[line=.95\lH]},
% backgroundcolor=TableDark
]
\setupTABLE [r] [2][framecolor=black]

\bTABLE

\bTABLEhead
\bTR
\bTC Grupo \eTC
\bTC Categoría \eTC
\bTC Clasificación \eTC
\bTC Producto\note[Produkt] \eTC
\bTC Cdad \eTC
\eTR
\eTABLEhead

\bTABLEbody
\bTR \bTD Motor diésel \eTD
\bTD Aceite del motor \eTD
\bTD \liqC{SAE 5W-30}; \liqC{VW 507.00}\eTD
\bTD Total Quartz INEO Long Life \eTD
\bTD 4,3\,l\eTD
\eTR
\bTR \bTD Circuito hidráulico \eTD
\bTD Aceite ATF \eTD
\bTD \liqC{dexron iii} \eTD
\bTD Total Equiviz ZS 46 (depósito aprox. 40\,l) \eTD
\bTD aprox. 50\,l\eTD
\eTR
\bTR \bTD Circuito hidráulico (opción \aW{Bio})\eTD
\bTD Aceite ATF \eTD
\bTD \liqC{dexron iii} \eTD
\bTD Total Biohydran TMP SE 46 (depósito aprox. 45\,l) \eTD
\bTD aprox. 60\,l\eTD
\eTR
\bTR \bTD Válvulas magnéticas: Núcleos de bobina \eTD
\bTD Lubricante\eTD
\bTD Grasa de cobre \eTD
\bTD \emptY\eTD
\bTD s. n.\note[Bedarf] \eTD
\eTR
\bTR \bTD Varios: cerraduras, mecánica de la puerta, pedal del freno \eTD
\bTD Lubricante\eTD
\bTD Spray Universal\eTD
\bTD \emptY\eTD
\bTD s. n.\note[Bedarf] \eTD
\eTR
\bTR \bTD Engrase centralizado \eTD
\bTD Grasa Universal\eTD
\bTD {\liqC{nlgi}} 000 o 00 (grasa de jabón li)\eTD
\bTD Total Multis EP 00\eTD
\bTD s. n.\note[Bedarf] \eTD
\eTR
\bTR \bTD Sistema de refrigeración \eTD
\bTD Anticongelante/anticorrosivo\eTD
\bTD {\liqC{tl vw 774 F/G}}; máx. 60\hairspace\% vol.\eTD
\bTD G12+/G12++ (rosa/violeta)\eTD
\bTD aprox. 14\,l \eTD
\eTR
\bTR \bTD Bomba de agua de alta presión \eTD
\bTD Aceite del motor \eTD
\bTD \liqC{sae 10w-40}; \liqC{api cf – acea e6}\eTD
\bTD Total Rubia TIR 8900 \eTD
\bTD 0,29\,l\eTD
\eTR
\bTR \bTD Aire acondicionado \eTD
\bTD Refrigerante\eTD
\bTD +Aceite POE de 20\,ml\eTD
\bTD R 134a\eTD
\bTD 700\,g\eTD
\eTR
\bTR \bTD Sistema limpiaparabrisas \eTD
\bTD [nc=2] Agua y concentrado para limpiaparabrisas, \aW{S} verano, \aW{W} invierno; tener en cuenta la proporción \eTD
\bTD Minoristas \eTD
\bTD s. n.\note[Bedarf] \eTD
\eTR
\eTABLEbody

\eTABLE

\stop
% \setups[pagestyle:normal]
\footnotetext[Bedarf]{{\it s. n.} según sea necesario y según indicado en el manual correspondiente}
\footnotetext[Produkt]{Productos empleados por \Boschung\. Pueden emplearse también otros productos que cumplan las especificaciones.}

\stopsection

\page [yes]

\setups [pagestyle:marginless]


\startsection [title={Mantenimiento del motor diésel},
reference={sec:workshop:vw},
]


\subsection [sSec:vw:diagTool]{Sistema de diagnóstico de a bordo}

La\startregister[index][reg:main:vw]{mantenimiento+motor diésel} unidad de control del motor (J623) está equipada con una memoria de errores.
Si aparecen anomalías en los sensores o piezas supervisados, estas se almacenan en la memoria de errores junto con el tipo de error.

Al evaluar la información la\index{motor diésel+diagnóstico} unidad de control del motor distingue entre las diferentes clases de error y las almacena hasta que se borre el contenido de la memoria de errores.

Los errores que solo aparecen {\em esporádicamente} se muestran con el sufijo \aW{SP}. La causa de los errores esporádicos puede ser \eG\ un contacto flojo o una interrupción breve de la corriente. Si un error esporádico deja de aparecer después de arrancar el motor 50 veces más, este será borrado de la memoria de errores.

Si se han encontrado errores que afectan al comportamiento de la marcha del motor, en la pantalla del Vpad se enciende el símbolo de control \aW{Diagnóstico de motor} \textSymb{vpadWarningEngine1}.

Los errores almacenados pueden leerse con el sistema de diagnóstico, medición e información del vehículo \aW{VAS 5051/B}.

Después de haber eliminado el error o los errores deberá borrarse la memoria de errores.


\subsubsection[sSec:vw:diagTool:connect]{Puesta en marcha del sistema de diagnóstico}

\starttextbackground [FC]
\startPictPar
\PMgeneric
\PictPar
En el manual de instrucciones del sistema encontrará información detallada sobre el sistema de diagnóstico del vehículo VAS 5051/B.

También puede emplear otros sistemas de diagnóstico compatibles, \eG\ \aW{DiagRA}.
\stopPictPar
\stoptextbackground

\page [yes]


\subsubsubsubject{Requisitos previos}

\startitemize
\item Los fusibles deben estar en buen estado.
\item La tensión de la batería debe ser superior a 11,5 V.
\item Deberán desconectarse todos los consumidores eléctricos.
\item La conexión a tierra debe estar en buen estado.
\stopitemize


\subsubsubsubject{Modo de proceder}

\startSteps
\item Enchufe el conector del cable de diagnóstico VAS 5051B/1 en la conexión de diagnóstico.
\item Dependiendo de la función, conectar la ignición o arrancar el motor.
\stopSteps

\subsubsubsubject{Seleccionar el modo de servicio}

\startSteps [continue]
\item En la pantalla pulse sobre el botón \aW{Autodiagnóstico del vehículo}.
\stopSteps


\subsubsubsubject{Seleccionar sistema del vehículo}

\startSteps [continue]
\item En la pantalla pulse sobre el botón \aW{01-Electrónica del motor}.
\stopSteps

En la pantalla aparece ahora la identificación de las unidades de control y la codificación de la unidad de control del motor.

Si las codificaciones no coinciden, deberá revisarse la codificación de las unidades de control.


\subsubsubsubject{Seleccionar la función de diagnóstico}

En la pantalla verá todas las funciones de diagnóstico que pueden realizarse.

\startSteps [continue]
\item Pulse el botón para la función deseada.
\stopSteps



\subsection [sSec:vw:faultMemory]{Memoria de errores}


\subsubsection{Leer memoria de errores}

\subsubsubject{Rutina de trabajo}

\startSteps
\item Precaliente el motor en punto muerto.
\item Conecte el VAS 5051/B (ver el \in{apartado}[sSec:vw:diagTool:connect]) y seleccione la unidad de control del motor.
\item Seleccione la función de diagnóstico \aW{004-Contenido de la memoria de errores}.
\item Seleccione la función de diagnóstico \aW{004.01-Leer memoria de errores}.
\stopSteps

{\sla Solo si el motor no arranca:}

\startitemize [2]
\item Conecte la ignición.
\item Si no hay ningún error en la unidad de control del motor, en la pantalla aparece \aW{0 Error detectado}.
\item Si hay errores en la unidad de control del motor, aparecerán listados en la pantalla.
\item Concluya la función de diagnóstico.
\item Desconecte la ignición.
\item Si se aplica, elimine los errores mostrados en la tabla de errores (ver la documentación de servicio) y borre a continuación la memoria de errores.
\stopitemize

\starttextbackground [FC]
\startPictPar
\PMrtfm
\PictPar
Si no se puede borrar un error, póngase en contacto con el Servicio de Atención al Cliente de \boschung.
\stopPictPar
\stoptextbackground


\subsubsubject{Errores estáticos}

Si en la memoria de datos hay uno o varios errores estáticos, póngase en contacto con el Servicio de Atención al Cliente de Boschung para eliminarlos con ayuda de la \aW{Búsqueda de errores guiada}.


\subsubsubject{Errores esporádicos}

Si en la memoria de errores solo hay errores esporádicos o avisos y no se pueden determinar funciones incorrectas del sistema electrónico del vehículo, podrá borrarse la memoria de errores:

\startSteps [continue]
\item Vuelva a pulsar la tecla \aW{Continuar} \inframed[strut=local]{>} para ir al plan de comprobación.
\item Para concluir la búsqueda de errores pulse la tecla \aW{Salto} y después \aW{Finalizar}.
\stopSteps

Se volverán a consultar todas las memorias de errores.

En una ventana se confirma que se han borrado todos los errores esporádicos. El protocolo de diagnóstico se envía automáticamente (online).

Con ello se ha concluido el test del sistema del vehículo.


\subsubsection[sSec:vw:faultMemory:errase]{Borrar la memoria de errores}

\subsubsubject{Rutina de trabajo}

{\sla Requisitos previos:}

\startitemize [2]
\item Se deben solucionar todos los errores y eliminar las causas de los mismos.
\stopitemize

\page [yes]


{\sla Modo de proceder:}

\starttextbackground [FC]
\startPictPar
\PMrtfm
\PictPar
Después de eliminar los errores se debe volver a consultar la memoria de errores y borrarse a continuación:
\stopPictPar
\stoptextbackground

\startSteps
\item Precaliente el motor en punto muerto.
\item Conecte el VAS 5051/B (ver el \in{apartado}[sSec:vw:diagTool:connect]) y seleccione la unidad de control del motor.
\item Seleccione la función de diagnóstico \aW{004-Leer memoria de errores}.
\item Seleccione la función de diagnóstico \aW{004.10-Borrar memoria de errores}.
\stopSteps

\starttextbackground [FC]
\startPictPar
\PMrtfm
\PictPar
Si no se puede borrar la memoria de errores, es que todavía hay un error que debe eliminarse.
\stopPictPar
\stoptextbackground

\startSteps [continue]
\item Concluya la función de diagnóstico.
\item Desconecte la ignición.
\stopSteps


\subsection [sSec:vw:lub] {Lubricar el motor diésel}

\subsubsection [ssSec:vw:oilLevel] {Comprobar el nivel de aceite del motor}

\starttextbackground [FC]
\startPictPar
\PMrtfm
\PictPar
El\index{aceite del motor+nivel} nivel de aceite no puede superar bajo ninguna circunstancia la marca de \aW{máximo}. De lo contrario existe\index{nivel de llenado+aceite del motor} peligro de daños en el catalizador.
\stopPictPar
\stoptextbackground

\startSteps
\item Apagar el motor y esperar al menos 3 minutos para que el aceite pueda volver al cárter de aceite.
\item Extraer la varilla de medición y limpiar. Volver a introducir la varilla hasta el tope.
\item Volver a extraer la varilla y determinar el nivel del aceite:

\startfigtext[right][fig:vw:gauge]{Leer el nivel del aceite}
{\externalfigure[VW_Oil_Gauge][width=50mm]}
\startitemize [A]
\item Nivel de llenado máximo; no debe añadirse aceite.
\item Nivel de llenado suficiente; se {\em puede} añadir aceite hasta la marca \aW{A}.
\item Nivel de llenado insuficiente; se {\em debe} añadir aceite hasta que el nivel de llenado esté en el área \aW{B}.
\stopitemize
{\em Cuando el nivel de llenado sobrepasa la marca \aW{A} existe el riesgo de que se produzcan daños en el catalizador.}
\item Si el nivel de llenado está por debajo de la marca \aW{C} añada aceite de motor hasta la marca \aW{A}.
\stopfigtext
\stopSteps


\subsubsection [ssSec:vw:oilDraining] {Cambiar el aceite del motor}

\starttextbackground [FC]
\startPictPar
\PMrtfm
\PictPar
El filtro de aceite del motor de la S2 está montado en vertical. Esto significa que hay que cambiar el filtro {\em antes} del cambio de aceite. Extrayendo el elemento filtrante se abre una válvula y el aceite en la carcasa del filtro fluye automáticamente en el cárter del cigüeñal.
\stopPictPar
\stoptextbackground

\startSteps
\item Coloque un\index{motor diésel+cambio de aceite} recipiente colector adecuado debajo del motor.
\item Desenroscar el tornillo de purga de aceite\index{aceite del motor+cambiar} y dejar salir el aceite.
\stopSteps

\starttextbackground [FC]
\startPictPar
\PMrtfm
\PictPar
Preste atención a que el recipiente colector puede contener todo el volumen de aceite usado.
La especificación de aceite requerida y la cantidad de llenado la encontrará en el \in{apartado}[sec:liqquantities].

El tornillo de purga de aceite está provisto con un anillo obturador imperdible. Por tanto, el tornillo de purga de aceite deberá reemplazarse siempre.
\stopPictPar
\stoptextbackground

\startSteps [continue]
\item Enrosque un tornillo de purga de aceite nuevo con anillo obturador (\TorqueR 30 Nm).
\item Añadir un aceite de motor con la especificación adecuada (ver el \in{apartado}[sec:liqquantities]).
\stopSteps


\subsubsection [ssSec:vw:oilFilter] {Reemplazar el filtro de aceite del motor}

\starttextbackground [FC]
\startPictPar
\PMrtfm
\PictPar
\startitemize [1]
\item Tenga en cuenta\index{motor diésel+filtro de aceite} las disposiciones sobre eliminación de desechos y reciclaje.
\item Cambie\index{filtro de aceite+motor diésel} el filtro {\em antes} de cambiar el aceite (ver el \in{apartado}[ssSec:vw:oilDraining]).
\item Antes del montaje engrase ligeramente la junta del filtro nuevo.
\stopitemize
\stopPictPar
\stoptextbackground

\startfigtext[right][fig:vw:oilFilter]{Filtro de aceite}
{\externalfigure[VW_OilFilter_03][width=50mm]}
\startSteps
\item Desenroscar la tapa \Lone\ de la carcasa del filtro con una llave plana adecuada.
\item Limpie las superficies de obturación de la tapa y de la carcasa del filtro.
\item Reemplace el elemento filtrante \Lthree.
\item Reemplace las juntas tóricas \Ltwo\ y \Lfour.
\item Volver a enroscar la tapa en la carcasa del filtro (\TorqueR 25 Nm).
\stopSteps



%\subsubsubject{Données techniques}
%
%
%\hangDescr{Couple de serrage du couvercle:} \TorqueR 25 Nm.
%
%\hangDescr{Huile moteur prescrite:} Selon tableau \atpage[sec:liqquantities].
%% NOTE: Redundant [tf]

\stopfigtext



\subsubsection [ssSec:vw:oilreplenish] {Añadir aceite de motor}

\starttextbackground [FC]
\startPictPar
\PMrtfm
\PictPar
\startitemize [1]
\item Limpie\index{aceite del motor} con un paño la boquilla de llenado {\em antes} de retirar la tapa.
\item Añada\index{motor diésel+rellenar aceite} solo aceite con la especificación prescrita.
\item Añada poco a poco cantidades pequeñas.
\item Para evitar que se llene demasiado, después de echar aceite espere un momento para que este pueda fluir en el cárter de aceite del motor hasta la marca de la varilla de medición (ver el \in{apartado}[ssSec:vw:oilLevel]).
\stopitemize
\stopPictPar
\stoptextbackground

\startfigtext[right][fig:vw:oilFilter]{Rellenar aceite}
{\externalfigure[s2_bouchonRemplissage][width=50mm]}
\startSteps
\item Extraiga la varilla de medición de aceite unos 10 cm para que el aire pueda salir al rellenar el aceite.
\item Abra la boquilla para llenado.
\item Rellene aceite teniendo en cuenta las especificaciones anteriores.
\item Cierre cuidadosamente la boquilla para llenado.
\item Arranque el motor.
\item Realice un control del nivel de llenado. (ver el \in{apartado}[ssSec:vw:oilLevel].)
\stopSteps

\stopfigtext


\subsection [sSec:vw:fuel] {Sistema de suministro de combustible}

\subsubsection [ssSec:vw:fuelFilter] {Reemplazar el filtro de combustible}

\starttextbackground [FC]
\startPictPar
\PMrtfm
\PictPar
\startitemize [1]
\item Tenga en cuenta\index{motor diésel+filtro de combustible} las disposiciones legales sobre eliminación de desechos y reciclaje de residuos tóxicos.
\item No retire los conductos de combustible de la pieza superior del filtro.
\item No ejerza fuerza de extracción sobre los puntos de fijación de los conductos de combustible, ya que podrían producirse daños en la pieza superior del filtro.
\stopitemize
\stopPictPar
\stoptextbackground

\startfigtext[right][fig:vw:oilFilter]{Filtro de combustible}
{\externalfigure[s2_fuelFilter_location][width=50mm]}

{\sla Preparación:}

La\index{filtro de combustible} carcasa del filtro de combustible está sujeta delante del motor en el lado derecho del chasis.
Retire los dos tornillos de fijación con una llave de vaso de 10 mm y una llave de estrella de 10 mm.

\stopfigtext


\page [yes]

\setups [pagestyle:normal]

{\sla Modo de proceder:}

\startLongsteps
\item Retire todos los tornillos de la pieza superior del filtro. Extraiga la pieza superior del filtro.
\stopLongsteps

\starttextbackground [FC]
\startPictPar
\PMrtfm
\PictPar
Saque la pieza superior. En caso necesario coloque para ello en la ranura de montaje un destornillador con cabeza en ángulo (\in{\LAa, fig.}[fig:fuelfilter:detach]) y extraiga la pieza superior mediante palanca.
\stopPictPar
\stoptextbackground

\placefig [margin] [fig:fuelfilter:detach]{Extraer el filtro de combustible}
{\externalfigure[fuelfilter:detach]}

\placefig [margin] [fig:fuelfilter:explosion]{Filtro de combustible}
{\externalfigure[fuelfilter:explosion]}

\startLongsteps [continue]
\item Extraiga el elemento filtrante de la pieza inferior del filtro.
\item Retire la junta (\in{\Ltwo, fig.}[fig:fuelfilter:explosion]) de la pieza superior del filtro.
\item Limpie a fondo la pieza superior e inferior del filtro.
\item Coloque un elemento filtrante nuevo en la pieza inferior del filtro.
\item Aplique un poco de combustible en una junta nueva (\in{\Ltwo, fig.}[fig:fuelfilter:explosion]) y colóquela en la pieza superior.
\item Ajuste la pieza superior coincidiendo con la pieza inferior del filtro y presione uniformemente de tal forma que la pieza superior repose por igual con toda la superficie.
\item Vuelva a atornillar la pieza inferior y la superior {\em a mano} con todos los tornillos. Apriete entonces todos los tornillos en cruz con el par de apriete prescrito (\TorqueR 5 Nm).
\stopLongsteps

% \subsubsubject{Données techniques}
%
% \hangDescr{Couple de serrage des vis de fixation du couvercle:} \TorqueR 5 Nm.
%% NOTE: redundant [tf]

\startLongsteps [continue]
\item Encienda la ignición para purgar el aire del sistema, arranque el motor y déjelo marchar de 1 a 2 minutos en ralentí.
\item Borre la memoria de errores como se describe en \atpage[sSec:vw:faultMemory:errase].
\stopLongsteps


\subsection [sSec:vw:cooling] {Sistema de refrigeración}

\starttextbackground [FC]
\startPictPar
\PMrtfm
\PictPar
\startitemize [1]
\item Emplear únicamente\index{motor diésel+refrigeración} refrigerante con la especificación prescrita (ver la tabla \atpage[sec:liqquantities]).
\item Para\index{refrigerante} garantizar la protección anticorrosiva y anticongelante, el refrigerante únicamente podrá diluirse con agua destilada y conforme a la siguiente tabla.
\item No rellene nunca el circuito de refrigerante con agua ya que esto podría afectar negativamente a la protección anticorrosiva y anticongelante.
\stopitemize
\stopPictPar
\stoptextbackground


\subsubsection [sSec:vw:coolingLevel] {Nivel de refrigerante}

\placefig [margin] [fig:coolant:level] {Nivel de refrigerante}
{\externalfigure[coolant:level]}


\placefig [margin] [fig:refractometer] {Refractómetro VW T 10007}
{\externalfigure[coolant:refractometer]}

\placefig [margin] [fig:antifreeze] {Control del grosor de anticongelante}
{\externalfigure[coolant:antifreeze]}


\startSteps
\item Eleve el depósito de material barrido.
\item Determine\index{nivel de llenado+refrigerante} el nivel de llenado del refrigerante en el recipiente de dilatación: Deberá estar por encima de la marca de \aW{min}.
\stopSteps

\start
\define [1] \TableSmallSymb {\externalfigure[#1][height=4ex]}
\define\UC\emptY
\pagereference[page:table:liquids]


\setupTABLE [frame=off,style={\ssx\setupinterlinespace[line=.86\lH]},background=color,
option=stretch,
split=repeat]
\setupTABLE [r] [each] [topframe=on,
framecolor=TableWhite,
% rulethickness=.8pt
]

\setupTABLE [c] [odd] [backgroundcolor=TableMiddle]
\setupTABLE [c] [even] [backgroundcolor=TableLight]
\setupTABLE [r] [first] [topframe=off,style={\bfx\setupinterlinespace[line=.95\lH]},
% backgroundcolor=TableDark
]
\setupTABLE [r] [2][framecolor=black]

\bTABLE

\bTABLEhead
\bTR
\bTC Anticongelante hasta … \eTC
\bTC Proporción G12\hairspace ++\eTC
\bTC Vol. de anticongelante \eTC
\bTC Vol. de agua destilada \eTC
\eTR
\eTABLEhead

\bTABLEbody
\bTR \bTD \textminus 25 °C \eTD
\bTD 40\hairspace\% \eTD
\bTD 3,8 l \eTD
\bTD 4,2 l \eTD
\eTR
\bTR \bTD \textminus 35 °C \eTD
\bTD 50\hairspace\% \eTD
\bTD 4,0 l \eTD
\bTD 4,0 l \eTD
\eTR
\bTR \bTD \textminus 40 °C \eTD
\bTD 60\hairspace\% \eTD
\bTD 4,2 l \eTD
\bTD 3,8 l \eTD
\eTR
\eTABLEbody

\eTABLE
\stop

\adaptlayout [height=+20pt]
\subsubsection [sSec:vw:coolingFreeze] {Nivel de refrigerante}

Compruebe\index{grosor de anticongelante} el grosor de anticongelante con ayuda de un refractómetro adecuado (ver la \in{fig.}[fig:refractometer]: VW T 10007).
Preste atención a la escala 1: G12\hairspace ++ (ver la \in{fig.}[fig:antifreeze]).

\page [yes]


\subsection [sSec:vw:airFilter] {Suministro de aire}

Se puede acceder al filtro de aire a través de la puerta de mantenimiento trasera a la derecha del vehículo (ver la \in{fig.}[fig:airFilter]).

\placefig [margin] [fig:airFilter] {Filtro de aire del motor}
{\externalfigure[vw:air:filter]
\noteF
\startLeg
\item Lengüeta de seguridad
\item Pieza inferior de la carcasa
\item Apertura de purga de aire
\item Sensor de presión
\stopLeg}


\subsubsubject{Condiciones de uso}

Una barredora se emplea frecuentemente en entornos con una formación de polvo elevada. Por este motivo es necesario comprobar y limpiar el filtro de aire una vez por semana. Ver también \about[table:scheduleweekly], \atpage[table:scheduleweekly]. Cuando sea necesario, deberá reemplazarse el filtro de aire.


\subsubsubject{Autodiagnóstico}

La tubería de admisión dispone de un sensor de presión (\Lfour, \in{fig.}[fig:airFilter]) mediante la cual pueden calcularse las pérdidas de carga\footnote{Flujo de aire reducido debido a una permeabilidad al aire reducida del filtro}.
Cuando el filtro de aire está obstruido, en la pantalla del Vpad se enciende el símbolo de control \textSymb{vpadWarningFilter} y se registra el mensaje de error\VpadEr{851}.


\subsubsubject{Poner a punto/reemplazar}

\startSteps
\item Tire de la lengüeta de seguridad \Lone hacia abajo (\in{fig.}[fig:airFilter]).
\item Gire la pieza inferior de la carcasa \Ltwo en el sentido contrario al de las agujas del reloj y extráigala.
\item Extraiga el elemento filtrante y revíselo. Reemplácelo en caso necesario.
\item Limpie el interior del filtro y vuelva a montar el filtro de aire en sentido inverso.
\stopSteps

\page [yes]


\subsection [sSec:vw:belt] {Correa trapezoidal}

La\index{motor diésel+correa trapezoidal} correa trapezoidal transfiere el movimiento del disco de inercia del cigüeñal al alternador y al compresor del aire acondicionado (equipamiento opcional).
Un\index{correa trapezoidal} elemento tensor en el último segmento (entre el alternador y el cigüeñal) mantiene la correa bajo tensión.


\subsubsection [sSec:belt:change] {Reemplazar la correa trapezoidal}

\placefig [margin] [fig:belt:tool] {Mandril VW T 10060 A}
{\externalfigure[vw:belt:tool]}

\placefig [margin] [fig:belt:overview] {Elemento tensor}
{\externalfigure[vw:belt:overview]}

\placefig [margin] [fig:belt:tens] {Punto de inserción del mandril}
{\externalfigure[vw:belt:tens]}


\subsubsubject{Con compresor de aire acondicionado}


{\sla Herramientas especiales necesarias:}

Mandril de inserción \aW{VW T 10060 A} para sujetar el elemento tensor.

\startSteps
\item Marque el sentido de marcha de la correa trapezoidal.
\item Con una llave de estrella con ángulo vire el brazo del elemento tensor en el sentido de las agujas del reloj (\in {fig.}[fig:belt:overview]).
\item Ponga los orificios (ver las flechas, \in {fig.}[fig:belt:tens]) en cubierto y bloquee el elemento tensor con el mandril de inserción.
\item Extraiga la correa trapezoidal.
\stopSteps

El montaje de la correa trapezoidal se realiza en sentido inverso.

\starttextbackground [FC]
\startPictPar
\PMrtfm
\PictPar
\startitemize [1]
\item Preste atención al sentido de marcha de la correa trapezoidal.
\item Preste atención al correcto asiento de la correa en las poleas.
\item Arranque el motor y controle la marcha de la correa.
\stopitemize
\stopPictPar
\stoptextbackground


\subsubsubject{Sin compresor de aire acondicionado}

{\sla Material necesario:}

Kit de reparaciones compuesto de un manual de reparaciones, una correa trapezoidal y herramientas especiales.\footnote{ver el catálogo de piezas de recambio, \aW{piezas de mantenimiento}.}

\startSteps
\item Separe la correa trapezoidal.
\item Siga los siguientes pasos del manual de reparaciones.
\stopSteps

\starttextbackground [FC]
\startPictPar
\PMrtfm
\PictPar
\startitemize [1]
\item Preste atención al correcto asiento de la correa en las poleas.
\item Arranque el motor y controle la marcha de la correa.
\stopitemize
\stopPictPar
\stoptextbackground


\subsubsection [sSec:belt:tens] {Reemplazar el elemento tensor}

{\sla Solo para el modelo con compresor de aire acondicionado}

\blank [medium]

\placefig [margin] [fig:belt:tens:change] {Reemplazar el elemento tensor}
{\externalfigure[vw:belt:tens:change]
\noteF
\startLeg
\item Elemento tensor
\item Tornillo de seguridad
\stopLeg

{\bf Par de apriete}

Tornillo de seguridad:

\TorqueR 20 Nm\:+ ½ vuelta (180°).}

\startSteps
\item Desmonte la correa trapezoidal como se describe (ver \atpage[sSec:belt:change]).
\item Desmonte las piezas periféricas (según el equipamiento).
\item Desenrosque los tornillos de seguridad (\in{\Ltwo, fig.}[fig:belt:tens:change]).
\stopSteps

El montaje del elemento tensor se realiza en sentido inverso.

\starttextbackground [FC]
\startPictPar
\PMrtfm
\PictPar
\startitemize [1]
\item Después del montaje es imprescindible que emplee un tornillo de seguridad nuevo.
\item Par de apriete: Ver la \in{fig.}[fig:belt:tens:change].
\stopitemize
\stopPictPar
\stoptextbackground

\stopregister[index][reg:main:vw]

\stopsection

\page[yes]


\setups[pagestyle:marginless]


\startsection[title={Equipo hidráulico},
reference={sec:hydraulic}]

\starttextbackground [FC]
% \startfiguretext[left,none]{}
% {\externalfigure[toni_melangeur][width=30mm]}

\startSymPar
\externalfigure[toni_melangeur][width=4em]
\SymPar
\textDescrHead{Reciclar sustancias industriales}
Las sustancias industriales y el lubricante no deben desecharse en la naturaleza ni quemarse.

Los lubricantes usados no deben llegar a la red de aguas ni a la naturaleza por lo que no podrán ser desechados con la basura doméstica.

Los lubricantes usados no pueden mezclarse con otros líquidos ya que se podrían producir sustancias tóxicas o sustancias muy difíciles de desechar.
\stopSymPar
\stoptextbackground
\blank [big]

% \starthangaround{\PMgeneric}
% \textDescrHead{Qualification du personnel}
% Toute intervention sur l’installation hydraulique de votre véhicule ne peut être réalisée que par une personne dument qualifiée, ou par un service reconnu par \boschung.
% \stophangaround
% \blank[big]

\startSymList
\PHgeneric
\SymList
\textDescrHead{Limpieza} El equipo hidráulico es muy sensible a las impurezas en el aceite. Por esta razón es importante trabajar en un entorno completamente limpio.
\stopSymList

\startSymList
\PHhot
\SymList
\textDescrHead{Peligro de salpicaduras}
Antes de trabajar en el equipo hidráulico \sdeux\ deberá extraerse la presión residual del circuito hidráulico correspondiente. Las salpicaduras de aceite caliente pueden causar quemaduras.
\stopSymList

\startSymList
\PHhand
\SymList
\textDescrHead{Peligro de aplastamiento}
Antes de realizar trabajos en el equipo hidráulico de la \sdeux\ es imprescindible bajar o asegurar mecánicamente mediante puntales de seguridad el depósito de material barrido.
\stopSymList

\startSymList
\PImano
\SymList
\textDescrHead{Medición de presión}
Para medir la presión hidráulica coloque un manómetro en una de las conexiones \aW{Minimess} del circuito. Preste atención a que el manómetro tenga un rango de medición adecuado.
\stopSymList

\page [yes]

\setups[pagestyle:normal]

\subsection{Intervalos de mantenimiento}

\start

\setupTABLE [frame=off,
style={\ssx\setupinterlinespace[line=.93\lH]},
background=color,
option=stretch,
split=repeat]
\setupTABLE [r] [each] [
topframe=on,
framecolor=white,
backgroundcolor=TableLight,
% rulethickness=.8pt,
]

% \setupTABLE [c] [odd] [backgroundcolor=TableMiddle]
% \setupTABLE [c] [even] [backgroundcolor=TableLight]
\setupTABLE [c] [1][ % width=30mm,
style={\bfx\setupinterlinespace[line=.93\lH]},
]
\setupTABLE [r] [first] [topframe=off,
style={\bfx\setupinterlinespace[line=.93\lH]},
backgroundcolor=TableMiddle,
]
% \setupTABLE [r] [2][style={\ssBfx\setupinterlinespace[line=.93\lH]}]


\bTABLE

\bTABLEhead
\bTR\bTD Trabajo de mantenimiento \eTD\bTD Intervalo \eTD\eTR
\eTABLEhead

\bTABLEbody
\bTR\bTD Comprobar fugas \eTD\bTD a diario \eTD\eTR
\bTR\bTD Controlar el nivel de aceite hidráulico \eTD\bTD a diario \eTD\eTR
\bTR\bTD Comprobar el estado de los conductos/las mangueras hidráulicas; reemplazar en caso necesario \eTD\bTD 600 h/12 meses \eTD\eTR
\bTR\bTD Reemplazar el filtro de retorno del aceite hidráulico y el filtro de aspiración \eTD\bTD 600 h/12 meses \eTD\eTR
\bTR\bTD Engrasar los núcleos de las bobinas de las válvulas magnéticas con grasa de cobre \eTD\bTD 600 h/12 meses \eTD\eTR
\bTR\bTD Cambiar el aceite hidráulico \eTD\bTD 1200 h/24 meses \eTD\eTR
\eTABLEbody
\eTABLE
\stop


\subsection[niveau_hydrau]{Nivel de llenado}

\placefig[margin][fig:hydraulic:level]{Nivel de llenado del líquido hidráulico}
{\externalfigure[hydraulic:level]
\noteF
\startLeg
\item Nivel de llenado óptimo
\stopLeg}

Una mirilla transparente\index{nivel de llenado+líquido hidráulico}\index{mantenimiento+equipo hidráulico} permite comprobar el nivel de aceite hidráulico.
Cuando el nivel de aceite hidráulico se reduzca, deberá determinarse la causa antes de volver a llenarlo. Cumpla los intervalos de cambio prescritos (tabla arriba) y las especificaciones para el líquido hidráulico (tabla \at{página }[sec:liqquantities]).


\subsubsection{Rellenar líquido hidráulico}

Rellene líquido hidráulico hasta que el nivel de llenado llegue al centro de la mirilla.
Arranque el motor y rellene algo de aceite en caso necesario hasta alcanzar el nivel de llenado requerido.


\subsection{Cambiar el líquido hidráulico}

La cantidad de llenado y las especificaciones requeridas del líquido hidráulico las encontrará en la tabla de la \at{página}[sec:liqquantities].

\startSteps
\item Abra la apertura de llenado del depósito hidráulico.
\item Vacíe el depósito con ayuda de un succionador de aceite o quite el tornillo de purgado.

El tornillo de purgado está abajo en el depósito hidráulico, delante de la rueda trasera izquierda (\in{fig.}[fig:hydraulic:fluidDrain]).
\item Rellene líquido hidráulico hasta que el nivel de llenado llegue al centro de la mirilla.
Arranque el motor y rellene algo de aceite en caso necesario hasta alcanzar el nivel de llenado requerido.
\stopSteps

\placefig[margin][fig:hydraulic:fluidDrain]{Tornillo de purgado}
{\externalfigure[hydraulic:fluidDrain]}


\placefig[margin][fig:hydraulic:returnFilter]{Filtro del sistema hidráulico}
{\externalfigure[hydraulic:returnFilter]}

\subsection[filtres:nettoyage]{Filtro de retorno y de aspiración}

\startSteps
\item Eleve el depósito de material barrido y coloque puntales de seguridad.
\item Quite la tapa del filtro en el depósito hidráulico (\in{fig.}[fig:hydraulic:returnFilter]).
\item Reemplace\index{filtro de aceite+sistema hidráulico} el elemento filtrante por uno nuevo.
\item Aplique un poco de líquido hidráulico sobre una junta tórica nueva y colóquela.
\item Vuelva a enroscar la tapa con las dos manos (\TorqueR aprox. 20 Nm).
\stopSteps

\page [yes]


\subsection{Lubricar las válvulas magnéticas}

\placefig[margin][graissage_bobine]{Lubricar las válvulas magnéticas}
{\externalfigure[graissage_bobine][M]
\noteF
\startLeg
\item Bobina de la válvula magnética
\item Núcleo de la bobina
\stopLeg}

La humedad y los restos de sal que lleguen al núcleo de las bobinas electromecánicas pueden corroer los núcleos. Los núcleos de las bobinas deberán engrasarse con grasa de cobre una vez al año. La grasa deberá ser resistente a la corrosión, al agua y a temperaturas de hasta 50 °C:
\startSteps
\item Desmonte la bobina de la válvula magnética (\in{\Lone, fig.}[graissage_bobine]).
\item Lubrique el núcleo (\in{\Ltwo, fig.}[graissage_bobine]) con la grasa especial prescrita y vuelva a montar la bobina.
\stopSteps


\subsection{Cambiar las mangueras}

La goma de protección\index{mangueras+intervalos de cambio} y el tejido de refuerzo de las mangueras están expuestos a un envejecimiento natural. Por este motivo, es imprescindible cambiar las mangueras del equipo hidráulico en los intervalos prescritos, incluso si no hay daños {\em visibles}.

Preste atención a que las mangueras se abridan correctamente al vehículo para evitar un desgaste prematuro por fricción. Deberán estar a una distancia suficiente respecto a otros componentes, con el fin de evitar daños por fricción y vibración.

\stopsection

\page [yes]

\setups [pagestyle:bigmargin]


\startsection[title={Sistema de frenos},
reference={sec:brake}]

\placefig[margin][fig:brake:rear]{Freno de tambor}
{\startcombination [1*2]
{\externalfigure[brake:wheelHub]}{\slx Cubo de rueda trasera}
{\externalfigure[brake:drum]}{\slx Mecanismo y juego de frenos}
\stopcombination}

Los tambores de freno \Lfour\ deberán desmontarse en cada mantenimiento, el mecanismo de freno deberá limpiarse \Lseven\ y el juego de frenos \Lfive, \Lsix\ deberá someterse a un control visual (\in{fig.}[fig:brake:rear]).


\subsubject {Desmontar}

\startSteps
\item Coloque el vehículo sobre una plataforma elevadora adecuada y eleve las ruedas.
\item Desmonte las ruedas.
\stopSteps


{\sla Desmontar los frenos de las ruedas delanteras}

\startSteps [continue]
\item Desmonte el tambor de frenos \Lfour.
\stopSteps

{\sla Desmontar los frenos de las ruedas traseras}

\startSteps [continue]
\item Quite la cubierta \Lone\ del cubo.
\item Retire el tornillo \Ltwo\ y extraiga la pieza intermedia.
\item Desenrosque la tuerca del cubo \Lthree\ con una llave de vaso.
\item Quite el cubo con el tambor de freno.
\stopSteps


\subsubject {Volver a montar}

Vuelva a montar el tambor de freno en sentido inverso. Apriete las tuercas de los cubos de las ruedas traseras \Lthree\ con el par de apriete prescrito de 190 Nm.

\stopsection

\page [yes]

\setups [pagestyle:normal]


\startsection[title={Control de los neumáticos},
reference={sec:pneumatiques}]

Los neumáticos\index{neumáticos+mantenimiento} deberán estar siempre en perfecto estado para poder realizar sus dos funciones principales: buena adherencia y un frenado perfecto. Un desgaste ilícito demasiado elevado y una presión de llenado incorrecta, especialmente una presión demasiado baja, son factores importantes que favorecen los accidentes.


\subsection{Puntos relativos a la seguridad}

\subsubsection{Control de desgaste}

El desgaste de los neumáticos deberá controlarse según los indicadores de desgaste que hay en un canal del perfil (\in{fig.}[pneususure]).
Las irregularidades en el neumático y sus causas se pueden determinar mediante un control visual:

\placefig[margin][pneususure]{Control de desgaste}
{\Framed{\externalfigure[pneusUsure][M]}}

\placefig[margin][pneusdomages]{Neumáticos dañados}
{\Framed{\externalfigure[pneusDomages][M]}}

\startitemize
\item Desgaste en los laterales de la superficie de rodadura: Presión de llenado demasiado baja.
\item Desgaste intenso en el centro: Presión de llenado demasiado alta.
\item Desgaste asimétrico en los laterales del neumático: Eje delantero (huella, forma del eje) ajustado incorrectamente.
\item Grietas en la superficie de rodadura: Neumáticos demasiado viejos; con el tiempo la goma de los neumáticos endurece y se agrieta (\in{fig.}[pneusdomages]).
\stopitemize

\starttextbackground[CB]
\startPictPar
\PHgeneric
\PictPar
\textDescrHead{Riesgos por neumáticos desgastados}
Un neumático desgastado ya no cumple su función, especialmente en lo relativo a la desviación del agua y el lodo; la distancia de frenado aumenta y la conducción empeora. Un neumático desgastado resbala con más facilidad, especialmente en suelos húmedos. Aumenta el peligro de que el neumático pierda adherencia.
\stopPictPar
\stoptextbackground


\subsubsection{Presión de llenado del neumático}

La presión de llenado del neumático prescrita está anotada en la placa de las ruedas, delante en la consola del lado del conductor (ver \atpage [sec:plateWheel]).

Incluso\index{neumáticos+presión de llenado} cuando las ruedas están en buen estado, con el tiempo pierden más o menos aire (cuanto más se utilice el vehículo, mayor será la pérdida de presión). Por este motivo, la presión de los neumáticos deberá comprobarse una vez al mes con los neumáticos en frío. Si comprueba la presión con los neumáticos calientes, deberá añadir 0,3 bar a la presión prescrita.

\start
\setupcombinations[M]
\placefig[margin][pneuspression]{Presión de llenado del neumático}
{\Framed{\externalfigure[pneusPression][M]}
\noteF
\startLeg
\item Presión correcta
\item Presión demasiado elevada
\item Presión demasiado baja
\stopLeg
La presión de neumáticos prescrita está indicada en la placa de características de las ruedas, dentro de la cabina en el lado del conductor.}
\stop

\starttextbackground[CB]
\startPictPar
\PHgeneric
\PictPar
\textDescrHead{Riesgos por una presión de neumáticos demasiado baja}
Un neumático podrían rasgarse si la presión es demasiado baja. La presión ejercida sobre el neumático es mayor cuando no está inflado lo suficiente o cuando el vehículo está sobrecargado. Esto hace que la goma se caliente y en una curva podrían soltarse algunas partes del neumático.
\stopPictPar
\stoptextbackground

\stopsection

\page [yes]

\setups[pagestyle:marginless]


\startsection[title={Chasis},
reference={main:chassis}]

\subsection{Fijación relevante para la seguridad de los componentes}

En cada mantenimiento deberá comprobarse el correcto asiento de los tornillos de fijación que son relevantes para la seguridad de determinados componentes, incluido el par de apriete prescrito. Esto se aplica especialmente al sistema de dirección articulada y a los ejes.

\blank [big]

\startfigtext [left] [fig:frontAxle:fixing] {Eje delantero}
{\externalfigure [frontAxle:fixing]}
{\sla Fijaciones del eje delantero}
\startLeg
\item Fijación de la hoja de ballesta: \TorqueR 150 Nm
\item Fijación de las unidades de tracción: \TorqueR 78 Nm
\stopLeg

{\sla Fijaciones del eje trasero}
\startLeg
\item Fijación de la hoja de ballesta: \TorqueR 150 Nm
\stopLeg

\stopfigtext

\start

\setupTABLE [frame=off,style={\ssx\setupinterlinespace[line=.93\lH]},background=color,
option=stretch,
split=repeat]

\setupTABLE [r] [each] [topframe=on,
framecolor=white,
% rulethickness=.8pt
]

\setupTABLE [c] [odd] [backgroundcolor=TableMiddle]
\setupTABLE [c] [even] [backgroundcolor=TableLight]
\setupTABLE [c] [1][style={\bfx\setupinterlinespace[line=.93\lH]}]
\setupTABLE [r] [first] [topframe=off,style={\bfx\setupinterlinespace[line=.93\lH]},
]
% \setupTABLE [r] [2][style={\bfx\setupinterlinespace[line=.93\lH]}]


\bTABLE

\bTABLEhead
\bTR [backgroundcolor=TableDark] \bTD [nc=3] Pares de apriete \eTD\eTR
% \bTR\bTD Position \eTD\bTD Type de vis \eTD\bTD Couple \eTD\eTR
\eTABLEhead

\bTABLEbody
\bTR\bTD Motores de accionamiento izquierdo/derecho \eTD\bTD M12\:×\:35 8.8 \eTD\bTD 78 Nm \eTD\eTR
%% NOTE @Andrew: das sind Hydraulikmotoren
\bTR\bTD Bomba de trabajo \eTD\bTD M16\:×\:40 100 \eTD\bTD 330 Nm \eTD\eTR
\bTR\bTD Bomba de accionamiento \eTD\bTD M12\:×\:40 100 \eTD\bTD 130 Nm \eTD\eTR
\bTR\bTD Hojas de ballesta delanteras/traseras \eTD\bTD M16\:×\:90/160 8.8 \eTD\bTD 150 Nm \eTD\eTR
% \bTR\bTD Fixation du système oscillant \eTD\bTD M12\:×\:40 8.8 \eTD\bTD 78 Nm \eTD\eTR
\bTR\bTD Fijación del depósito de material barrido \eTD\bTD M10\:×\:30 Verbus Ripp 100 \eTD\bTD 80 Nm \eTD\eTR
\bTR\bTD Tuercas de rueda \eTD\bTD M14\:×\:1,5 \eTD\bTD 180 Nm \eTD\eTR
\bTR\bTD Fijación del cepillo delantero \eTD\bTD M16\:×\:40 100 \eTD\bTD 180 Nm \eTD\eTR
\eTABLEbody
\eTABLE
\stop


\stopsection

\page [yes]


\startsection[title={Engrase centralizado},
reference={main:graissageCentral}]


\subsection{Descripción del módulo de control}

La \sdeux\ puede equiparse con\index{engrase centralizado} un engrase centralizado (opción). El engrase centralizado suministra lubricante a cada punto de lubricación del vehículo en periodos regulares.

\startfigtext [left] [vogel_affichage] {Módulo de pantalla}
{\externalfigure[vogel_base2][W50]}
\blank
\startLeg
\item Pantalla de 7 dígitos: Valores y estado de servicio
\item \LED: Sistema en pausa (modo standby)
\item \LED: Bomba en servicio
\item \LED: Control del sistema mediante interruptor de ciclos
\item \LED: Control del sistema mediante interruptor de presión
\item \LED: Mensaje de error
\item Teclas de movimiento por imagen:
\startLeg [R]
\item Activar pantalla
\item Mostrar valores
\item Modificar valores
\stopLeg
\item Tecla para cambiar el modo de servicio; confirmación de los valores
\item Activación de un ciclo de engrase intermedio
\stopLeg
\stopfigtext

El engrase centralizado comprende la bomba de lubricante, el depósito de lubricante transparente en el lado izquierdo del chasis y el módulo de control en el sistema eléctrico central.
% \blank
\page [yes]


\subsubsubject{Indicación y teclas del módulo de control}

\start

\setupTABLE [frame=off,style={\ssx\setupinterlinespace[line=.93\lH]},background=color,
option=stretch,
split=repeat]

\setupTABLE [r] [each] [topframe=on,
framecolor=white,
% rulethickness=.8pt
]

\setupTABLE [c] [odd] [backgroundcolor=TableMiddle]
\setupTABLE [c] [even] [backgroundcolor=TableLight]
\setupTABLE [c] [1][width=9mm,style={\bfx\setupinterlinespace[line=.93\lH]}]
\setupTABLE [r] [first] [topframe=off,style={\bfx\setupinterlinespace[line=.93\lH]},
]
% \setupTABLE [r] [2][style={\bfx\setupinterlinespace[line=.93\lH]}]


\bTABLE
\bTABLEhead
% \bTR [backgroundcolor=TableDark]
% \bTD [nc=4]Indicación y teclas del módulo de control \eTD\eTR
\bTR\bTD Pos. \eTD
\bTD \LED \eTD\bTD Modo de indicación \eTD
\bTD Modo de programación \eTD\eTR
\eTABLEhead

\bTABLEbody
\bTR\bTD 2 \eTD
\bTD Estado de servicio {\em Pausa}\hskip.5em\null \eTD
\bTD El equipo está en modo standby\hskip.5em\null \eTD %
\bTD Pueden modificarse los tiempos de pausa \eTD\eTR
\bTR\bTD 3 \eTD
\bTD Modo de servicio {\em Contacto} \eTD
\bTD La bomba trabaja \eTD
\bTD Puede modificarse el tiempo de trabajo \eTD\eTR
\bTR\bTD 4 \eTD
\bTD Control del sistema {\em CS} \eTD
\bTD Con el interruptor de ciclos externo \eTD
\bTD Puede desactivarse o modificarse el modo de control \eTD\eTR
\bTR\bTD 5 \eTD
\bTD Control del sistema {\em PS} \eTD
\bTD Con el interruptor de presión externo \eTD
\bTD Puede desactivarse o modificarse el modo de control \eTD\eTR
\bTR\bTD 6 \eTD
\bTD Anomalía {\em Fallo} \eTD
\bTD [nc=2] Se ha producido una anomalía en el funcionamiento. La causa puede mostrarse en forma de código de error una vez pulsada la tecla \textSymb{vogel_DK}. Se interrumpen las funciones. \eTD\eTR
\bTR\bTD 7 \eTD
\bTD Teclas de flecha \textSymb{vogelTop} \textSymb{vogelBottom} \eTD
\bTD [nc=2] \items[symbol=R]{Activar la pantalla, Consultar los parámetros (modo de pantalla), Ajustar el valor mostrado (I) (modo de programación)}
\eTD\eTR
\bTR\bTD 8 \eTD
\bTD Tecla \textSymb{vogelSet} \eTD
\bTD [nc=2] Cambiar entre el modo de programación y el modo de indicación o confirmar los valores introducidos. \eTD\eTR
\bTR\bTD 9 \eTD
\bTD Tecla \textSymb{vogel_DK} \eTD
\bTD [nc=2] Cuando el aparato está en el estado de {\em pausa} al accionar esta tecla se activa el ciclo de engrase intermedio. Se confirman y borran los mensajes de error. \eTD\eTR
\eTABLEbody
\eTABLE
\stop
\vfill

\startfigtext [left] [vogel_touches]{Módulo de pantalla}
{\externalfigure[vogel_base][width=50mm]}
\textDescrHead{Modo de pantalla} Pulsar brevemente una de las teclas de flecha \textSymb{vogelTop} \textSymb{vogelBottom} para activar la pantalla de 7 dígitos \textSymb{led_huit}. Volviendo a pulsar la tecla \textSymb{vogelTop} pueden mostrarse los diferentes parámetros seguidos de su valor. El modo de {\em indicación} puede reconocerse por los \LED\char"2060s iluminado continuamente (\in{2 a 6, fig.}[vogel_affichage]).
\blank [medium]
\textDescrHead{Modo de programación} Para modificar los valores, pulse durante al menos 2 segundos la tecla \textSymb{vogelSet}
\stopfigtext

\page [yes]


\subsection{Submenús en el modo {\em Indicación}}

\vskip -9pt

\adaptlayout [height=+5mm]

\startcolumns[balance=no]\stdfontsemicn

\startSymVogel
\externalfigure[vogel_tpa][width=26mm]
\SymVogel
\textDescrHead{Tiempo de pausa [h]} Pulse la tecla \textSymb{vogelTop} para mostrar los valores programados.
\stopSymVogel

\startSymVogel
\externalfigure[vogel_068][width=26mm]
\SymVogel
\textDescrHead{Tiempo de pausa restante [h]} Tiempo que queda hasta el siguiente ciclo de lubricación.
\stopSymVogel

\startSymVogel
\externalfigure[vogel_090][width=26mm]
\SymVogel
\textDescrHead{Tiempo de pausa total [h]} Tiempo de pausa total entre dos ciclos.
\stopSymVogel

\startSymVogel
\externalfigure[vogel_tco][width=26mm]
\SymVogel
\textDescrHead{Tiempo de lubricación [min]} Pulse \textSymb{vogelTop} para mostrar los valores programados.
\stopSymVogel

\startSymVogel
\externalfigure[vogel_tirets][width=26mm]
\SymVogel
\textDescrHead{Aparato en standby} Indicación no posible ya que el aparato está en standby (pausa).
\stopSymVogel

\startSymVogel
\externalfigure[vogel_026][width=26mm]
\SymVogel
\textDescrHead{Tiempo de lubricación [min]} Duración de un proceso de lubricación.
\stopSymVogel

\startSymVogel
\externalfigure[vogel_cop][width=26mm]
\SymVogel
\textDescrHead{Control del sistema} Pulse \textSymb{vogelTop} para mostrar los valores programados.
\stopSymVogel

\startSymVogel
\externalfigure[vogel_off][width=26mm]
\SymVogel
\textDescrHead{Modo de control} \hfill PS: Interruptor de presión;\crlf
CS: Interruptor de ciclos; OFF: desactivado.
\stopSymVogel

\startSymVogel
\externalfigure[vogel_0h][width=26mm]
\SymVogel
\textDescrHead{Horas de servicio} Pulse \textSymb{vogelTop} para mostrar el valor en dos pasos.
\stopSymVogel

\startSymVogel
\externalfigure[vogel_005][width=26mm]
\SymVogel
\textDescrHead{Parte 1: 005} El tiempo de servicio se muestra en dos partes; ir a la parte 2 con la tecla \textSymb{vogelTop}.
\stopSymVogel

\startSymVogel
\externalfigure[vogel_338][width=26mm]
\SymVogel
\textDescrHead{Parte 2: 33,8} La segunda parte del número es 33,8; en total resultad en 533,8 h.
\stopSymVogel

\startSymVogel
\externalfigure[vogel_fh][width=26mm]
\SymVogel
\textDescrHead{Tiempo de errores} Pulse \textSymb{vogelTop} para mostrar el valor en dos pasos.
\stopSymVogel

\startSymVogel
\externalfigure[vogel_000][width=26mm]
\SymVogel
\textDescrHead{Parte 1: 000} El tiempo de error se muestra en dos partes;\crlf
ir a la parte 2 con \textSymb{vogelTop}.
\stopSymVogel

\startSymVogel
\externalfigure[vogel_338][width=26mm]
\SymVogel
\textDescrHead{Parte 2: 33,8} La segunda parte del número es 33,8; resulta en un tiempo de error total de 33,8 h.
\stopSymVogel

\stopcolumns

\page [yes]


\setups [pagestyle:marginless]


\startsection[title={Plan de lubricación para una lubricación manual},
reference={sec:grasing:plan}]

\starttextbackground [FC]
\startPictPar
\PMgeneric
\PictPar
Los puntos de lubricación indicados en el plan de lubricación (\in{fig.}[fig:greasing:plan]) deben lubricarse regularmente. Es imprescindible realizar una lubricación regular para garantizar una {\em reducción de la fricción} duradera y evitar la humedad y otras sustancias corrosivas.
\stopPictPar
\stoptextbackground

\blank [big]

\start

\setupcombinations [width=\textwidth]

\placefig[here][fig:greasing:plan]{Plan de lubricación del vehículo}
{\startcombination [3*1]
{\externalfigure[frame:steering:greasing]}{\ssx Dirección articulada y mecanismo de oscilación}
{\externalfigure[frame:axles:greasing]}{\ssx Ejes}
{\externalfigure[frame:sucMouth:greasing]}{\ssx Boca de aspiración}
\stopcombination}

\stop

\vfill

\startLeg [columns,three]
\item Cilindro elevador de la dirección articulada\crlf {\sl 2 racores de lubricación por cilindro}
\item Cojinete de la dirección articulada\crlf {\sl 2 racores de lubricación en el lado izquierdo}
\columnbreak
\item Cojinete del mecanismo oscilante\crlf {\sl 1 racor de lubricación delante del depósito}
\item Muelles de ballesta\crlf {\sl 2 racores de lubricación por hoja de ballesta}
\columnbreak
\item Boca de aspiración\crlf {\sl 1 racor de lubricación por rueda}
\item Boca de aspiración\crlf {\sl 1 racor de lubricación en el brazo de tracción}
\stopLeg



\page [yes]


\setups [pagestyle:bigmargin]


\subsubject{Lubricación del depósito de material barrido}

El depósito de material barrido dispone de 6 puntos de lubricación (2\:×\:3) que deben lubricarse semanalmente.

\blank [big]


\placefig[here][fig:greasing:container]{Mecanismo de elevación del depósito}
{\externalfigure[container:mechanisme]}


\placelegende [margin,none]{}
{{\sla Leyenda:}

\startLeg
\item Cojinete izquierdo del depósito
\item Cojinete derecho del depósito
\item Cilindro hidráulico izquierdo (arriba)
\item Cilindro hidráulico izquierdo (abajo)

{\em Como el cilindro derecho}
\item Cilindro hidráulico derecho (arriba)
\item [greasing:point;hide]Cilindro hidráulico derecho (abajo)
\stopLeg}

\stopsection

\page [yes]



\startsection[title={Equipo eléctrico},
reference={sec:main:electric}]

\subsection{Sistema eléctrico central en el chasis}

\startbuffer [fuses:preventive]
\starttextbackground [CB]
\startPictPar
\PHvoltage
\PictPar
\textDescrHead{Disposiciones de seguridad}
Tenga en cuenta las disposiciones de seguridad en\index{fusibles+chasis} este\index{relés+chasis} manual: Reemplace los fusibles solo con fusibles con el número de amperios prescritos; quítese las joyas metálicas antes de trabajar en el equipo eléctrico\index{equipo eléctrico} (anillos, pulseras, etc.).
\stopPictPar
\stoptextbackground
\stopbuffer


\subsubsubject{Fusibles MIDI}

\starttabulate[|l|r|p|]
\HL
\NC\md F 1 \NC 5 A \NC Luz de freno, \aW{+\:15} OBD \NC\NR
\NC\md F 2 \NC 5 A \NC \aW{+\:15} Unidad de control del motor \NC\NR
\NC\md F 3 \NC 7,5 A \NC \aW{+\:30} Unidad de control del motor y OBD \NC\NR
\NC\md F 4 \NC 20 A \NC Bomba de combustible \NC\NR
\NC\md F 5 \NC 20 A \NC \aW{D\:+} Alternador, \aW{+\:15} relé K 1 \NC\NR
\NC\md F 6 \NC 5 A \NC Unidad de control del motor \NC\NR
\NC\md F 7 \NC 10 A\NC Tratamiento de gases del motor \NC\NR
\NC\md F 8 \NC 20 A \NC Sistema electrónico del motor (unidad de control) \NC\NR
\NC\md F 9 \NC 15 A \NC Tratamiento de gases del motor, suministro, precalentamiento \NC\NR
\NC\md F 10\NC 30 A \NC Unidad de control del motor \NC\NR
\NC\md F 11\NC 5 A \NC Focos de trabajo detrás \NC\NR
%% NOTE @Andrew: Singular
\HL
\stoptabulate

\placefig [margin] [fig:electric:power:rear] {Sistema eléctrico central en el chasis}
{\externalfigure [electric:power:rear]
\noteF
\startKleg
\sym{K 1} Unidad de control del motor electrónica
\sym{K 2} Bomba de combustible
\sym{K 3} Liberación del motor de arranque
\sym{K 4} Luces de freno
\sym{K 5} {[}reserva{]}
\sym{K 6} Foco de trabajo trasero
%% NOTE @Andrew: Singular
\sym{K 7} Sistema de precalentamiento
\stopKleg
}


\subsubsubject{Fusibles MAXI}

% \startcolumns [n=2]
\starttabulate[|l|r|p|]
\HL
\NC\md F 15 \NC 50 A \NC Suministro principal del sistema eléctrico central \NC\NR
\HL
\stoptabulate

\page [yes]

\setups[pagestyle:marginless]


\subsection{Sistema eléctrico central en la cabina}

\startcolumns[rule=on]

\placefig [bottom] [fig:fuse:cab] {Fusibles y relés en la cabina}
{\externalfigure [electric:power:front]}

%\vfill

\subsubsubject{Relés}

\index{fusibles+cabina}\index{relés+cabina}

\starttabulate[|lB|p|]
\NC K 2\NC Compresor de aire acondicionado\NC\NR
\NC K 3\NC Compresor de aire acondicionado\NC\NR
\NC K 4\NC Bomba de agua eléctrica\NC\NR
\NC K 5\NC Luz giratoria de aviso\NC\NR
\NC K 10 \NC Transductor intermitente\NC\NR
\NC K 11 \NC Luces de cruce\NC\NR
\NC K 12 \NC Luces de carretera\NC\NR
\NC K 13 \NC Focos de trabajo\NC\NR
\NC K 14 \NC Conexión de intervalos de limpiaparabrisas\NC\NR
\stoptabulate

\vskip -24pt

\placefig [bottom] [fig:fuse:access] {Tapa de acceso al sistema eléctrico central}
{\externalfigure [electric:power:cabin]}

\stopcolumns

\page [yes]


\subsubsubject{Fusibles MINI}

\startcolumns[rule=on]
% \setuptabulate[frame=on]
%\placetable[here][tab:fuses:cab]{Fusibles en la cabina}
%{\noteF
\starttabulate[|lB|r|p|]
\NC F 1 \NC 3 A \NC Luz de posición izquierda \NC\NR
\NC F 2 \NC 3 A \NC Luz de posición derecha \NC\NR
\NC F 3 \NC 7,5 A \NC Luz de cruce izquierda \NC\NR
\NC F 4 \NC 7,5 A \NC Luz de cruce derecha \NC\NR
\NC F 5 \NC 7,5 A \NC Luz de carretera izquierda {[}reserva{]} \NC\NR
\NC F 6 \NC 7,5 A \NC Luz de carretera derecha {[}reserva{]} \NC\NR
\NC F 7 \NC 10 A \NC Focos de trabajo arriba \NC\NR
%% NOTE @Andrew: Plural
\NC F 8 \NC 10 A \NC Focos de trabajo abajo (reserva) \NC\NR
%% NOTE @Andrew: Plural
\NC F 9 \NC — \NC {[}libre{]} \NC\NR
\NC F 10 \NC 10 A \NC Limpiaparabrisas \NC\NR
\NC F 11 \NC 5 A \NC Interruptor de iluminación y del dispositivo de luces de aviso \NC\NR
\NC F 12 \NC 5 A \NC {[}reserva{]} \NC\NR
\NC F 13 \NC 10 A \NC Calefacción de espejos exteriores \NC\NR
\NC F 14 \NC 7,5 A \NC \aW{+\:15} Radio y cámara \NC\NR
\NC F 15 \NC 10 A \NC \aW{+\:30} Dispositivo de luces de aviso \NC\NR
\NC F 16 \NC 5 A \NC Iluminación de columna de dirección \NC\NR
\NC F 17 \NC 7,5 A \NC \aW{+\:30} Radio e iluminación interior \NC\NR
\NC F 18 \NC — \NC {[}libre{]} \NC\NR
\NC F 19 \NC 20 A \NC \aW{+\:30} RC 12 delante \NC\NR
\NC F 20 \NC 20 A \NC \aW{+\:30} RC 12 detrás \NC\NR
\NC F 21 \NC 15 A \NC Enchufe de 12 V \NC\NR
\NC F 22 \NC 5 A \NC Llave de contacto, consola multifuncional, Vpad \NC\NR
\NC F 23 \NC 5 A \NC Parada de emergencia, consola central, RC 12 delante \NC\NR
\NC F 24 \NC 5 A \NC Parada de emergencia, consola central, RC 12 detrás \NC\NR
\NC F 25 \NC 2 A \NC \aW{+\:15} RC 12 delante \NC\NR
\NC F 26 \NC 2 A \NC \aW{+\:15} RC 12 detrás \NC\NR
\NC F 27 \NC 15 A \NC Ventiladores de calefacción \NC\NR
\NC F 28 \NC 10 A \NC Compresor de aire acondicionado, engrase centralizado \NC\NR
\NC F 29 \NC 15 A \NC Condensador de aire acondicionado \NC\NR
\NC F 30 \NC 5 A \NC Termostato del aire acondicionado \NC\NR
\NC F 31 \NC 5 A \NC \aW{+\:15} Consola multifuncional/Vpad \NC\NR
\NC F 32 \NC 15 A \NC Bomba de agua eléctrica, luz giratoria de aviso \NC\NR
\NC F 33 \NC — \NC {[}libre{]} \NC\NR
\NC F 34 \NC — \NC {[}libre{]} \NC\NR
\NC F 35 \NC — \NC {[}libre{]} \NC\NR
\NC F 36 \NC — \NC {[}libre{]} \NC\NR
\stoptabulate
\stopcolumns

\page [yes]

\setups [pagestyle:bigmargin]


\subsection[sec:lighting]{Dispositivo de iluminación y señalización}


\placefig [here] [fig:lighting] {Dispositivo de iluminación y señalización del vehículo}
{\externalfigure [vhc:electric:lighting]}

\placelegende [margin,none]{}{%
\vskip 30pt
{\sla Leyenda:}
\startLongleg
\item Luces de posición\hfill 12\,V–5\,W
\item Luces de cruce\hfill H7 12\,V–55\,W
\item Intermitente\hfill naranja 12\,V–21\,W
\item Focos de trabajo\hfill G886 12\,V–55\,W
\item Intermitente\hfill 12\,V–21\,W
\item Posición/freno\hfill 12\,V–5/21\,W
\item Focos de marcha atrás\hfill 12\,V–21\,W
\item {[}libre{]}
\item Iluminación de la matrícula\hfill 12\,V–5\,W
\item Faro\hfill H1 12\,V–55\,W
\stopLongleg}

\subsubsubject{Ajustar los focos}

\placefig [margin] [fig:lighting:adjustment] {Foco de luz a 5 m}
{\externalfigure [vhc:lighting:adjustment]
\startitemize
\sym{H\low{1}} Altura del foco de luz: 100\,cm
\sym{H\low{2}} Corrección a 2\hairspace\%: 10\,cm
\stopitemize}

{\md Requisitos previos:} Depósito de agua fresca/reciclada lleno, conductor al volante.

La alineación de los focos viene ajustada de fábrica. La altura y la inclinación del foco de luz pueden ajustarse virando el soporte de plástico.

Si durante una revisión se determina que debe modificarse el ajuste, afloje el tornillo de seguridad y corrija la inclinación conforme a las disposiciones legales (ver la \in{fig.}[fig:lighting:adjustment]). Vuelva a apretar el tornillo de seguridad.

\page [yes]
\setups [pagestyle:marginless]


\subsection[sec:battcheck]{Batería}

\subsubsection{Disposiciones de seguridad}

\startSymList
\PPfire
\SymList
\textDescrHead{Peligro de explosión}
Al\index{batería+indicaciones de seguridad}\index{peligro+explosión} cargar baterías se forma gas fulminante\index{gas fulminante} explosivo. ¡Cargue las baterías solo en espacios ventilados! ¡Evite la formación de chispas! No trabaje con fuego, llamas ni fume en las inmediaciones de la batería.
\stopSymList

\startSymList
\PHvoltage
\SymList
\textDescrHead{Riesgo de cortocircuitos}
Cuando\index{batería+mantenimiento} el borne positivo de la batería conectada entra en contacto con partes del vehículo, existe\index{riesgo+fuego}\index{peligro+explosión} riesgo de cortocircuitos. Esto puede hacer que la mezcla explosiva de gas que sale de la batería explote, y usted y otras personas podrían resultar gravemente lesionadas.

\startitemize
\item No coloque objetos de metal ni herramientas sobre la batería.
\item Al desbornar la batería, retirar siempre primero el borne negativo y después del positivo.
\item Al conectar la batería, poner primero el borne positivo y luego el negativo.
\item Con el motor en marcha, no soltar o desconectar los bornes de conexión de la batería.
\stopitemize
\stopSymList


\startSymList
\PHcorrosive
\SymList
\textDescrHead{Peligro de abrasión}
Utilice\index{peligro+abrasión} gafas protectoras y guantes protectores resistentes a ácidos. El líquido de la batería contiene un 27\hairspace\% de ácido sulfúrico (H\low{2}SO\low{4}) y puede producir abrasión. Neutralice\index{batería+peligro}\index{batería+líquido} el líquido de la batería que entre en contacto con la piel con una solución de sosa bicarbonatada y aclare con abundante agua limpia. Si se produjera el contacto con los ojos del líquido de la batería, aclare estos con abundante agua limpia y acuda inmediatamente a un médico.
\stopSymList

\startSymList
\startcombination[1*2]
{\PHcorrosive}{}
{\PHfire}{}
\stopcombination
\SymList
\textDescrHead{Almacenamiento de baterías}
Almacenar\index{batería+almacenar} las baterías siempre en vertical. De lo contrario, el líquido de la batería podría salir y producir abrasión o, al reaccionar con otras sustancias, fuego. \par\null\par\null
\stopSymList

\testpage [16]

\starttextbackground [FC]
\setupparagraphs [PictPar][1][width=2.4em,inner=\hfill]

\startPictPar
\PMproteyes
\PictPar
\textDescrHead{Gafas protectoras}
Al\index{peligro+lesión en los ojos} mezclar agua con ácidos, el líquido puede salpicar los ojos. Si el ácido salpica los ojos, aclarar inmediatamente con abundante agua limpia y ¡acudir de inmediato a un médico!
\stopPictPar
\blank [small]

\startPictPar
\PMrtfm
\PictPar
\textDescrHead{Documentación}
Al manipular baterías, es imprescindible cumplir las indicaciones de seguridad, las medidas de protección y las formas de proceder contenidas en este manual de servicio.
\stopPictPar
\blank [small]

\startPictPar
\PStrash
\PictPar
\textDescrHead{Protección medioambiental}
Las baterías\index{protección medioambiental} contienen sustancias tóxicas. No deseche nunca las baterías usadas en la basura doméstica. Deseche las baterías de forma respetuosa con el medio ambiente. Entréguelas en un taller técnico o en un punto de recogida de baterías usadas.

Transportar y almacenar siempre en vertical las baterías llenas. Durante el transporte, deberán asegurarse las baterías para evitar que vuelquen. Puede salir líquido de la batería por las aperturas de ventilación de los tapones de cierre y llegar al medio ambiente.
\stopPictPar
\stoptextbackground

\page [yes]

\setups[pagestyle:normal]


\subsubsection{Consejos prácticos}

Para obtener la máxima vida útil, las baterías deberán estar siempre suficientemente cargadas.

Una\index{batería+vida útil} carga de conservación de las baterías durante periodos de parada prolongados no solo aumenta la vida útil de las baterías sino que garantiza además una preparación continua a la hora del arranque.

\placefig[margin][fig:batterycompartment]{\select{caption}{Compartimento para batería (tapa de mantenimiento)}{Compartimento para batería}}
{\externalfigure[batt:compartment]}


\subsubsection{Puesta a punto}

La batería de \sdeux\ es una batería de plomo {\em que no necesita mantenimiento}. Aparte del mantenimiento del estado cargado y de la limpieza, la batería no requiere ninguna otra medida de puesta a punto.

\startitemize
\item Preste atención a que los polos de la batería estén siempre limpios y secos. Engrase ligeramente los polos con grasa repelente de ácidos.
\item Recargar las baterías que \index{batería+cargar} tengan una tensión en reposo de\index{batería+tensión en reposo} menos de 12,4 V.
\stopitemize

\placefig[margin][fig:bclean]{Limpiar los polos}
{\externalfigure[batt:clean]
\noteF
\index{batería+limpiar}\index{limpieza+baterías} Use agua caliente para retirar el polvo blanco formado por la corrosión. Si un polo está oxidado, desconecte el cable de la batería y limpie el polo con un cepillo de púas de metal. Aplique una capa fina de grasa a los polos.}


\subsubsection[sec:battery:switch]{Uso del interruptor para desconexión de la batería}

{\sl ¡No se recomienda accionar regularmente el interruptor para desconexión de la batería (por ejemplo a diario)!}

\startSteps
\item Desconecte\index{interruptor para desconexión de la batería} la ignición y espere aproximadamente 1 minuto.
\item Abra el compartimento de la batería (\inF[fig:batterycompartment]).
\item Pulse el botón rojo del interruptor para desconexión de la batería para interrumpir el circuito eléctrico.
\item Para volver a cerrar el circuito eléctrico, gire el interruptor para desconexión de la batería ¼ de vuelta en el sentido de las agujas del reloj.
\stopSteps

% \starttextbackground [FCnb]
% \startPictPar
% \PMgeneric
% \PictPar
% Der Batterietrennschalter ist dafür vorgesehen, die Batterie für bestimmte Wartungs- und Reparaturarbeiten vorübergehend vom Stromkreis zu trennen. Es ist nicht empfehlenswert, den Batterietrennschalter regelmäßig (\eG\ täglich) zu betätigen: Bestimmte elektronische Komponenten sollten ständig unter Spannung stehen, ansonsten kann es zu Fehlermeldungen im Fehlerspeicher kommen.
% \stopPictPar
% \stoptextbackground

\stopsection

\page [yes]


\setups[pagestyle:marginless]

\section[sec:cleaning]{Limpiar el vehículo}

Antes\startregister[index][vhc:lavage]{mantenimiento+limpieza} de la limpieza en sí, retire de la carrocería el lodo y la suciedad gruesas con agua. No limpie solo las superficies laterales sino también la caja de las ruedas y la parte baja del vehículo.

Especialmente en invierno deberá limpiarse a fondo el vehículo para eliminar los restos de\index{corrosión+prevención} sal descongelante, que son muy corrosivos.

\starttextbackground [FC]
\startPictPar
\PHgeneric
\PictPar
\textDescrHead{Evitar daños por agua}
No limpie nunca el vehículo con {\em cañones de agua} (\eG\ de los bomberos) ni {\em limpiadores en frío con base de hidrocarburos.} Cuando trabaje con un limpiador al vapor de alta presión, tenga las disposiciones relativas indicadas más abajo.
\stopPictPar
\blank[small]

\startPictPar
\pTwo[monde]
\PictPar
\textDescrHead{Protección medioambiental}
La limpieza del vehículo puede ser causa de graves daños medioambientales.
Limpie el vehículo únicamente en un lugar\index{protección medioambiental} equipado con un separador de aceite. Tenga en cuenta las disposiciones medioambientales vigentes.
\stopPictPar
\blank[small]

\startPictPar
\PMwarranty
\PictPar
\textDescrHead{¡Limpiar debidamente!}
Los daños ocasionados por el incumplimiento de las disposiciones de limpieza no pueden ser reclamados a la empresa \BosFull{Boschung} ni existe derecho alguno por responsabilidad o garantía.
\stopPictPar
\stoptextbackground


\subsection{Limpieza de alta presión}

Para limpiar\index{limpieza+alta presión} el vehículo a alta presión es adecuado un limpiador de alta presión de uso comercial.

Al realizar la limpieza a alta presión, deben tenerse en cuenta los siguientes puntos:

\startitemize
\item Presión de trabajo máxima 50\,bar
\item Boquilla plana con un ángulo de rociado de 25°
\item Distancia de pulverizado mínima 80\,cm
\item Temperatura máxima del agua 40\, °C
\item Tenga en cuenta el apartado \about[reiMi], \atpage[reiMi].
\stopitemize

El incumplimiento de estas\index{pintura+daños} disposiciones puede derivar en daños en la pintura y la protección anticorrosión\index{daños+pintura}.

Tenga en cuenta también el manual de instrucciones y las disposiciones de seguridad del limpiador de alta presión.

\starttextbackground[FC]
\startPictPar\PPspray\PictPar
Al realizar la limpieza de alta presión puede penetrar agua en algunos puntos en los que podría causar daños. Por este motivo, no dirija nunca el chorro de agua a partes sensibles ni a aperos:
\stopPictPar

\startitemize
\item Sensores, uniones eléctricas y conexiones
\item Motores de arranque, alternador, sistema de inyección
\item Válvulas magnéticas
\item Aperturas de ventilación
\item Componentes mecánicos que no han enfriado
\item Adhesivos de aviso, advertencia y peligros
\item Unidades de control electrónicas
\stopitemize

\textDescrHead{Limpieza del motor}
Es imprescindible evitar la entrada de agua en aperturas de succión, de ventilación y aireación. Cuando se utilicen limpiadores de alta presión, no dirigir el chorro directamente a los componentes ni a cables eléctricos. ¡No dirigir el chorro al sistema de inyección! Aplicar anticorrosivo al motor después de limpiarlo. No aplicar anticorrosivo sobre el sistema de la correa.
\stoptextbackground

\starttextbackground [FC]
\setupparagraphs [PictPar][1][width=6em,inner=\hfill]
\startPictPar
\framed[frame=off,offset=none]{\PMproteyes\PMprotears}
\PictPar
\textDescrHead{Agua restante}
Durante la limpieza se acumula agua en determinados puntos del vehículo (\eG\ en las hendiduras del bloque de motor o del engranaje). Retirarla con aire comprimido. Tenga que al manipular aire comprimido debe llevarse el equipo de protección personal correspondiente y el equipo debe cumplir las disposiciones de seguridad correspondientes (tobera múltiple).
\stopPictPar
\stoptextbackground


\subsubsection[reiMi]{Limpiadores adecuados}

Utilice\index{limpiadores} únicamente limpiadores con las siguientes propiedades:

\startitemize
\item No abrasivos
\item Valor PH de 6–7
\item Sin disolventes
\stopitemize

Para eliminar marchas resistentes, emplee con prudencia white spirit o alcohol sobre superficies pintadas pequeñas, nunca otros disolventes. Elimine los restos de disolvente de la pintura. ¡La limpieza de piezas de plástico con bencina puede producir grietas o decoloración!

Limpie con agua las superficies con\index{limpieza+adhesivos} adhesivos de aviso o de indicación y utilizando una esponja suave.

Evite la entrada de agua en componentes eléctricos: No utilice cepillos de coche para limpiar la caja de los intermitentes y de las luces, sino un paño o esponja suaves.

\starttextbackground [CB]
\startPictPar
\GHSgeneric\par
\GHSfire
\PictPar
\textDescrHead{Peligro por sustancias químicas}
Los limpiadores puede suponer un riesgo para la salud y para la seguridad (sustancias fácilmente inflamables). Tenga en cuenta las disposiciones de seguridad vigentes para el limpiador utilizado; tenga en cuenta la hoja de riesgos y de datos de los productos empleados.
\stopPictPar
\stoptextbackground

\stopregister[index][vhc:lavage]


\page [yes]


\setups [pagestyle:bigmargin]

\startsection [title={Ajuste de la boca de aspiración},
reference={sec:main:suctionMouth}]


La distancia\index{boca de aspiración+ajustar} óptima entre la superficie de la carretera y el labio de goma de la boca de aspiración es de 10\,mm.
Para controlar o ajustar la distancia, utilice los tres calibres de ajuste que encontrará en la caja de herramientas (cabina, lado del conductor).


\placefig [margin] [fig:suctionMouth] {Ajustar la boca de aspiración}
{\externalfigure [suctionMouth:adjust]}

\placeNote[][service_picto]{}{%
\noteF
\starttextrule{\PHasphyxie\enskip Peligro de intoxicación y asfixia \enskip}
{\md Aviso:} Durante los trabajos de ajuste el motor del vehículo deberá estar en marcha para poder mantener la boca de aspiración en la posición de flotación. Para evitar riesgos de intoxicación o asfixia, es imprescindible utilizar un equipo de extracción de gases de escape, o bien realizar los trabajos únicamente en un lugar muy bien ventilado.
\stoptextrule}

\startSteps
\item Coloque el vehículo en un lugar bien ventilado sobre una superficie horizontal y lisa.
\item Active\index{aspiración} el modo de \aW{trabajo} (botón exterior en la palanca de marchas).

Deje marchar el motor en punto muerto. (Pulse la tecla~\textSymb{joy_key_engine_decrease} en la consola multifuncional para reducir el número de revoluciones del motor.)
\item Ponga el freno de estacionamiento y asegure las ruedas traseras con cuñas.
\item Pulse la tecla~\textSymb{joy_key_suction} para bajar la boca de aspiración.
\item Coloque los tres calibres de ajuste~\LAa\ debajo del labio de goma de la boca de aspiración como se muestra en la figura.
\item [sucMouth:adjust]Afloje los tornillos de ajuste~\Lone\ y de fijación~\Ltwo\ de cada una de las ruedas; las cuatro ruedas bajan hasta el suelo.
\item Bloquee los tornillos~\Lone\ y~\Ltwo retire entonces los tres calibres de ajuste.
\item Suba/baje la boca de aspiración y compruebe el ajuste con los calibres de ajuste. Si el ajuste sigue sin ser del todo correcto, repita el procedimiento de ajuste a partir del punto~\in[sucMouth:adjust].

\stopSteps


\stopsection

\stopchapter
\stopcomponent



