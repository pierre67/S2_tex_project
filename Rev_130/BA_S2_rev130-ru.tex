\defineparagraphs[SymVpad][n=2,distance=4mm,rule=off,before={\page[preference]},after={\nobreak\hrule\blank [2*medium]}]
\setupparagraphs [SymVpad][1][width=4em,inner=\hfill]

% \startcolumns

\subsection{Контрольные индикаторы на экране Vpad} % nouveau

%% ajout

\startSymVpad
\externalfigure[vpadWarningService][height=1.7\lH]
\SymVpad
\textDescrHead{Обратиться в мастерскую}(желтый) Наступил срок регулярного техобслуживания машины (см. \about [sec:schedule] \atpage [sec:schedule]), или была обнаружена неисправность двигателя (требуется специализированная мастерская).

+\:Сообщения об ошибках \# 650 – \# 653, или \# 703
\stopSymVpad


\startSymVpad
\externalfigure[vpadTDPF][height=1.7\lH]
\SymVpad
\textDescrHead{Пылевой фильтр}(желтый) При первой возможности начнется регенерация пылевого фильтра.

{\md Указание:} {\lt Пока горит этот индикатор, по возможности {\em не} останавливайте двигатель!}
\stopSymVpad



%%%%%%%%%%%%%%%%%%%%%%%%%%%%


\bTR\bTD \externalfigure [v:symbole:power] \eTD\bTD Выключение экрана \eTD\bTD Удерживать около 5 секунд для выключения экрана Vpad. \eTD\eTR
\bTR\bTD \framed[frame=off]{\externalfigure [v:symbole:frontBrush]\externalfigure [v:symbole:frontBrush:black]}
\eTD\bTD Третья щетка\index{3-я щетка} (опция) \eTD\bTD Разблокирование третьей щетки.
Теперь третью щетку можно включить в соответствии с описанием на странице \at[sec:using:frontBrush]. \eTD\eTR


%%%%%%%%%%%%%%%%%%%%%%%%%%%% corriger

\startsection [title={Меню Vpad}, reference={vpad:menu}]



\subsection{Другие пиктограммы на экране Vpad}


\subsubsubject{Запас чистой и регенерированной воды}


\subsubsubject{Система всасывания} % nouveau

{\em Эта пиктограмма отображается только, когда щетки деактивированы.}

\startSymVpad
\externalfigure[sym:vpad:sucker]
\SymVpad
\textDescrHead{Патрубок всасывания} Система всасывания\index{Патрубок всасывания} активирован:
патрубок всасывания опущен, турбина активирована.
\stopSymVpad


\subsubsubject{Боковая щетка} % nouveau

{\em Эта пиктограмма отображается только, когда третья щетка деактивирована.}

\startSymVpad
\externalfigure[sym:vpad:sideBrush:83]
\SymVpad
\textDescrHead{Боковые щетки} Щетки\index{Уборка}\index{Боковые щетки} активированы. Скорость вращения (в \% максимальной скорости вращения [V\low{max}]) отображается под пиктограммой, текущая компенсация нагрузки с соответствующей щетки – над пиктограммой (\type{ } = плавающее положение, 14 = максимальная компенсация).

{\md Компенсация нагрузки:} {\lt Чем ниже компенсация нагрузки, тем выше давление прижима щеток к поверхности земли.}
\stopSymVpad


\startSymVpad
\externalfigure[sym:vpad:sideBrush:float:60]
\SymVpad
\textDescrHead{Плавающее положение}(нижняя кромка зеленого цвета)
Чтобы выключить компенсацию нагрузки, переместите джойстик вперед на примерно 2 с; щетка будет опираться на поверхность всем своим весом. Скорость вращения щеток составляет 60\hairspace\% V\low{max} (пример).
\stopSymVpad

\startSymVpad
\externalfigure[sym:vpad:sideBrush]
\SymVpad
\textDescrHead{Боковые щетки} Щетки активированы. Они неподвижны и подняты.
\stopSymVpad


\subsubsubject{Третья щетка (опция)} % nouveau

\startSymVpad
\externalfigure[sym:vpad:frontBrush]
\SymVpad
\textDescrHead{Третья щетка} Третья щетка\index{3-я щетка} активирована. Скорость вращения (в \% максимальной скорости вращения [V\low{max}]) отображается под пиктограммой.
\stopSymVpad


\startSymVpad
\externalfigure[sym:vpad:frontBrush:left]
\SymVpad
\textDescrHead{Плавающее положение}(нижняя кромка зеленого цвета)
Чтобы выключить компенсацию нагрузки, переместите джойстик вперед на примерно 2 с; щетка будет опираться на поверхность всем своим весом. Скорость вращения щеток составляет 70\hairspace\% V\low{max} (пример).

{\md Направление вращения:} {\lt В верхней части отображается направление вращения (черная стрелка на желтом фоне).}
\stopSymVpad

\stopsection




\setups [pagestyle:marginless]


\startsection [title={Работа с третьей щеткой (опция)},
reference={sec:using:frontBrush},
]

\startSteps
\item Введите\index{Уборка} машину в эксплуатацию в соответствии с описанием в \in{разделе}[sec:using:start] \atpage[sec:using:start].
\item Активируйте\index{3-я щетка} \aW{рабочий} режим (кнопка на внешней стороне рычага переключения передач).
\stopSteps

% \getbuffer [work:config]

\startSteps [continue]
\item Убедитесь, что третья щетка активирована на экране Vpad (см. \textSymb{vpadFrontBrush} \textSymb{vpadFrontBrushK}, \atpage[vpad:menu]).
\item Чтобы задействовать гидравлику третьей щетки, нажмите кнопку~\textSymb{joy_key_frontbrush_act}.
\item Чтобы активировать вращение третьей щетки в нужном направлении, нажмите кнопку~\textSymb{joy_key_frontbrush_left} или~\textSymb{joy_key_frontbrush_right}.

\item Настройте скорость вращения кнопками~\textSymb{joy_key_frontbrush_increase} и~\textSymb{joy_key_frontbrush_decrease} на многофункциональной консоли.

\item Джойстиками установите щетку в нужное положение так, как показано на иллюстрациях.

\stopSteps

{\md Указание:} {\lt Чтобы положение боковых щеток можно было изменить, нужно деактивировать гидравлику третьей щетки кнопкой~\textSymb{joy_key_frontbrush_act}.}
\vfill

\start
\setupcombinations [width=\textwidth]

\placefig[here][fig:brush:position]{Позиционирование третьей щетки}
{\startcombination [2*1]
{\externalfigure [work:frontBrush:move]}{Вверх/вниз; влево/вправо}
{\externalfigure [work:frontBrush:incline]}{Наклон в поперечном/продольном направлении}
\stopcombination}
\stop



\stopsection


%%% Änderungen Benedikt Sturny

% S. 68
\item Поставьте машину на стояночный тормоз и переведите рычаг переключения передач в \aW{нейтральное} положение. (Это необходимо для разблокирования переключателя опрокидывания бункера.)


% S. 93

\subsection[niveau_hydrau]{Уровень наполнения}

Прозрачное смотровое стекло\index{Уровень наполнения+Гидравлическая жидкость}\index{Техническое обслуживание+Гидравлическая система} позволяет визуально контролировать уровень гидравлического масла.
Если уровень гидравлического масла понизился, необходимо определить причину, прежде чем снова доливать масло до нужного уровня. Проводите замену в указанные сроки (таблица вверху) и придерживайтесь спецификаций на гидравлическую жидкость (таблица \at{страница}[sec:liqquantities]).


\subsubsection{Долив гидравлической жидкости}

Залейте гидравлическую жидкость, чтобы среднее смотровое окошко было полностью закрыто.
Запустите двигатель и долейте жидкость, если это необходимо, до нужного уровня.
