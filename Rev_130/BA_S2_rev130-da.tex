\defineparagraphs[SymVpad][n=2,distance=4mm,rule=off,before={\page[preference]},after={\nobreak\hrule\blank [2*medium]}]
\setupparagraphs [SymVpad][1][width=4em,inner=\hfill]

% \startcolumns

\subsection{Kontrolvisninger på Vpad-skærmen} % nouveau

%% ajout

\startSymVpad
\externalfigure[vpadWarningService][height=1.7\lH]
\SymVpad
\textDescrHead{Køretøjet skal på værksted}(gul) Køretøjet skal til regelmæssigt serviceeftersyn (se \about [sec:schedule] \atpage [sec:schedule]) eller der er registreret en motorfejl (specialværksted nødvendigt).

+\:Fejlmeldinger \# 650 til \# 653, eller \# 703
\stopSymVpad


\startSymVpad
\externalfigure[vpadTDPF][height=1.7\lH]
\SymVpad
\textDescrHead{Partikelfilter}(gul) Regenerering af partikelfilteret startes, så snart driftstilstanden tillader det.

{\md Bemærk:} {\lt Sluk {\em ikke} motoren, hvis muligt, så længe denne visning lyser!}
\stopSymVpad



%%%%%%%%%%%%%%%%%%%%%%%%%%%%


\bTR\bTD \externalfigure [v:symbole:power] \eTD\bTD Sluk skærmen \eTD\bTD Hold den inde ca. 5 s for at slukke Vpad-skærmen. \eTD\eTR
\bTR\bTD \framed[frame=off]{\externalfigure [v:symbole:frontBrush]\externalfigure [v:symbole:frontBrush:black]}
\eTD\bTD Tredje kost\index{3. kost} (ekstraudstyr) \eTD\bTD Aktiver tredje kost.
Den tredje kost kan nu aktiveres, som beskrevet på side \at[sec:using:frontBrush]. \eTD\eTR


%%%%%%%%%%%%%%%%%%%%%%%%%%%% corriger

\startsection [title={Menuer i Vpad}, reference={vpad:menu}]



\subsection{Andre symboler på Vpad-skærmen}


\subsubsubject{Ferskvands- og genbrugsvandbeholdning}


\subsubsubject{Sugesystem} % nouveau

{\em Dette symbol vises kun, hvis kostene er deaktiveret.}

\startSymVpad
\externalfigure[sym:vpad:sucker]
\SymVpad
\textDescrHead{Sugemund} Sugesystem\index{Sugemund} aktiveret:
Sugemunden er sænket og turbinen er aktiveret.
\stopSymVpad


\subsubsubject{Sidekoste} % nouveau

{\em Dette symbol vises kun, hvis tredje kost ikke er aktiveret.}

\startSymVpad
\externalfigure[sym:vpad:sideBrush:83]
\SymVpad
\textDescrHead{Sidekoste} Koste\index{Fejning}\index{Sidekoste} aktiveret. Rotationshastigheden (i \% af maks. rotationshastighed [V\low{max}]) vises under symbolet, den aktuelle aflastning af den pågældende kost vises oven over symbolet (\type{ } = Flydeposition, 14 = Maksimal aflastning).

{\md Aflastning:} {\lt Jo lavere aflastning, desto mere trykker kostene mod jorden.}
\stopSymVpad


\startSymVpad
\externalfigure[sym:vpad:sideBrush:float:60]
\SymVpad
\textDescrHead{Flydeposition}(grøn kant nederst)
For at deaktivere aflastningen skal joysticket trykkes fremad i ca. 2 s; kosten ligger nu på jorden med hele sin egenvægt. Kostenes rotationshastighed er på 60\hairspace\% af V\low{max} (eksempel).
\stopSymVpad

\startSymVpad
\externalfigure[sym:vpad:sideBrush]
\SymVpad
\textDescrHead{Sidekoste} Kostene er aktiveret. De står stille og er løftet.
\stopSymVpad


\subsubsubject{Tredje kost (ekstraudstyr)} % nouveau

\startSymVpad
\externalfigure[sym:vpad:frontBrush]
\SymVpad
\textDescrHead{Tredje kost} Den tredje kost\index{3. kost} er aktiveret. Rotationshastigheden (i \% af maks. rotationshastighed [V\low{max}]) vises under symbolet.
\stopSymVpad


\startSymVpad
\externalfigure[sym:vpad:frontBrush:left]
\SymVpad
\textDescrHead{Flydeposition}(grøn kant nederst)
For at deaktivere aflastningen skal joysticket trykkes fremad i ca. 2 s; kosten ligger nu på jorden med hele sin egenvægt. Kostenes rotationshastighed er på 70\hairspace\% af V\low{max} (eksempel).

{\md Rotationsretning:} {\lt I øverste kant vises rotationsretningen (sort pil på gul baggrund).}
\stopSymVpad

\stopsection




\setups [pagestyle:marginless]


\startsection [title={Arbejde med tredje kost (ekstraudstyr)},
reference={sec:using:frontBrush},
]

\startSteps
\item Sæt\index{Fejning} køretøjet i drift, som beskrevet i \in{afsnit}[sec:using:start] \atpage[sec:using:start].
\item Aktiver\index{3. kost} \aW{Arbejds}modus (knap uden på gearvælgeren).
\stopSteps

% \getbuffer [work:config]

\startSteps [continue]
\item Kontroller, at den tredje kost er aktiveret på Vpad-skærmen (se \textSymb{vpadFrontBrush} \textSymb{vpadFrontBrushK}, \atpage[vpad:menu]).
\item Tryk på tasten~\textSymb{joy_key_frontbrush_act} for at aktivere tredje kosts hydraulik.
\item Tryk på tasten~\textSymb{joy_key_frontbrush_left} eller~\textSymb{joy_key_frontbrush_right} for at lade tredje kost rotere i den ønskede retning.

\item Indstil rotationshastigheden med tasterne~\textSymb{joy_key_frontbrush_increase} og~\textSymb{joy_key_frontbrush_decrease} på multifunktionskonsollen.

\item Positioner kosten med joysticket, som vist i illustrationerne nedenfor.

\stopSteps

{\md Bemærk:} {\lt For at kunne positionere sidekostene, skal tredje kosts hydraulik deaktiveres med tasten~\textSymb{joy_key_frontbrush_act}.}
\vfill

\start
\setupcombinations [width=\textwidth]

\placefig[here][fig:brush:position]{Positionering af tredje kost}
{\startcombination [2*1]
{\externalfigure [work:frontBrush:move]}{Opad/nedad; mod venstre/højre}
{\externalfigure [work:frontBrush:incline]}{Hældning på tværs/langs}
\stopcombination}
\stop



\stopsection


%%% Änderungen Benedikt Sturny

% S. 68
\item Aktiver håndbremsen og sæt gearvælgeren på \aW{Neutral}. (Nødvendigt til aktivering af kontakten til at tippe beholderen).


% S. 93

\subsection[niveau_hydrau]{Påfyldningsniveau}

Et gennemsigtigt skueglas\index{Påfyldningsniveau+Hydraulikvæske}\index{Vedligeholdelse+Hydrauliksystem} giver mulighed for at kontrollere hydraulikoliestanden.
Hvis hydraulikoliestanden er faldet, skal årsagen findes, før der fyldes olie på igen. Overhold de foreskrevne udskiftningsintervaller (tabellen ovenfor) og specifikationer for hydraulikvæske (tabellen \at{side}[sec:liqquantities]).


\subsubsection{Påfyldning af hydraulikvæske}

Fyld hydraulikvæske på, indtil det mellemste skueglas er helt dækket.
Start motoren og fyld evt. mere olie på, indtil det påkrævede påfyldningsniveau.
